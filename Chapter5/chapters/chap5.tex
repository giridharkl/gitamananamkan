\slcol{ಅರ್ಜುನ ಉವಾಚ ।\\
\Index{ಸಂನ್ಯಾಸಂ ಕರ್ಮಣಾಂ} ಕೃಷ್ಣ ಪುನರ್ಯೋಗಂ ಚ ಶಂಸಸಿ ।\\
ಯಚ್ಛ್ರೇಯ ಏತಯೋರೇಕಂ ತನ್ಮೇ ಬ್ರೂಹಿ ಸುನಿಶ್ಚಿತಮ್ ॥ ೧ ॥}
\cquote{ಅರ್ಜುನನ್ನು ಹೇಳಿದನು,\\
 ಕೃಷ್ಣ, ಕರ್ಮಗಳ ತ್ಯಾಗವನ್ನು ಹೇಳುತ್ತೀ, ಫಲದಾಸೆ ಇಲ್ಲದೆ ಮಾಡೆಂದು ಹೇಳುತ್ತಿ. ಇವೆರಡರಲ್ಲಿ ಯಾವುದು ಮೇಲೋ ಆ ಒಂದನ್ನು ನನಗೆ ಗೊತ್ತು ಮಾಡಿ ಹೇಳು.\\}
\slcol{ಶ್ರೀಭಗವಾನುವಾಚ ।\\
\Index{ಸಂನ್ಯಾಸಃ ಕರ್ಮಯೋಗಶ್ಚ} ನಿಃಶ್ರೇಯಸಕರಾವುಭೌ ।\\
ತಯೋಸ್ತು ಕರ್ಮಸಂನ್ಯಾಸಾತ್ಕರ್ಮಯೋಗೋ ವಿಶಿಷ್ಯತೇ ॥ ೨ ॥}
\cquote{ಭಗವಂತನು ಹೇಳಿದನು,\\
 ಸನ್ಯಾಸ ಹಾಗೂ ಕರ್ಮಯೋಗ ಎರಡು ಶ್ರೇಯಸ್ಸನ್ನು ಉಂಟುಮಾಡುವಂತಹವೇ. ಅವುಗಳಲ್ಲಿ ಕರ್ಮ ಸನ್ಯಾಸಕ್ಕಿಂತ ಕರ್ಮ ಯೋಗವು ಶ್ರೇಯಸ್ಕರ.\\}
\slcol{\Index{ಜ್ಞೇಯಃ ಸ ನಿತ್ಯಸಂನ್ಯಾಸೀ} ಯೋ ನ ದ್ವೇಷ್ಟಿ ನ ಕಾಂಕ್ಷತಿ ।\\
ನಿರ್ದ್ವಂದ್ವೋ ಹಿ ಮಹಾಬಾಹೋ ಸುಖಂ ಬಂಧಾತ್ಪ್ರಮುಚ್ಯತೇ ॥ ೩ ॥}
\cquote{ಮಹಾಬಾಹೋ! ಒಂದು ಬೇಕೆನ್ನದೆ, ಒಂದು ಬೇಡವೆನ್ನದೆ ಇರುವವನೇ ನಿತ್ಯ ಸನ್ಯಾಸಿ. ದ್ವಂದ್ವ ರಹಿತನಾದ ಅವನು ಬಂಧನದಿಂದ ಸುಖವಾಗಿ ತಪ್ಪಿಸಿಕೊಳ್ಳುತ್ತಾನೆ.\\}


\newpage
\begin{mananam}{\mananamfont \large ಮನನ ಶ್ಲೋಕ - ೨}
\mananamtext ನಾನು ಅಗತ್ಯಕ್ಕಿಂತ ಹೆಚ್ಚಾದ ಚಟುವಟಿಕೆಗಳನ್ನು ತೆಗೆದುಕೊಳ್ಳುತ್ತಿದ್ದೇನೆಯೇ? ಯಾವುದು ಉಪಯೋಗವಿಲ್ಲವೋ ಅಥವಾ ಅಗತ್ಯವಿಲ್ಲವೋ ಅಂತಹ ಚಟುವಟಿಕೆಗಳನ್ನು ಹಿಂತೆಗೆದುಕೊಳ್ಳಬಲ್ಲೆನೇ? ಇನ್ನೊಂದು ಕಡೆಯಲ್ಲಿ ನಾನು, ಒತ್ತಡ ಅಥವಾ ಅಡಚಣೆಗಳು ಬಂದಾಗ, ನನ್ನ ಅಗತ್ಯವಾದ ಕೆಲಸಗಳನ್ನು ಅಥವಾ ಜವಾಬ್ದಾರಿಗಳನ್ನು ದೂರ ಮಾಡಲು ಹವಣಿಸುತ್ತಿದ್ದೇನೆಯೇ? ನಾನು ಮಾಡಲೇ  ಬೇಕಾದ ಕೆಲಸ ಮತ್ತು ಮಾಡಬಾರದ ಕೆಲಸಗಳ ಬಗ್ಗೆ ಸಮತೋಲನ ಮಾಡಲು ಹೇಗೆ ಕಲಿಯಲಿ? ನನ್ನೊಳಗಿನ ನಿರ್ಲಿಪ್ತತೆಯನ್ನು ಉಳಿಸಿಕೊಂಡು ನಾನು ಹೇಗೆ ಕಾರ್ಯಗಳನ್ನು ಸಂಪೂರ್ಣವಾಗಿ ನಿಶ್ಚಯಿಸಲಿ?
\end{mananam}
\WritingHand\enspace\textbf{ಆತ್ಮ ವಿಮರ್ಶೆ}\\
\begin{inspiration}{\mananamfont \large ಸ್ಪೂರ್ತಿ}
\mananamtext ತ್ಯಾಗ ಮತ್ತು ಲಾಭದ ನಿರೀಕ್ಷೆ ಇಲ್ಲದೆ ಮಾಡುವ ಕೆಲಸ ಇವೆರಡು ಕರ್ತವ್ಯಗಳ ಕಡೆಗೆ ಇರುವ ಅಧ್ಯಾತ್ಮಿಕ ಮನೋವೃತ್ತಿಯಲ್ಲಿ, ಯಾರಿಗೆ ಜವಾಬ್ದಾರಿ ಮತ್ತು ಕರ್ತವ್ಯಗಳನ್ನು ಬಿಡಲಾರರೋ ಅವರಿಗೆ ಎರಡನೆಯದು ಹೆಚ್ಚಿನ ಪ್ರಯೋಜನಕಾರಿಯಾಗಿದೆ. ಅದಾಗಿಯೂ ಒಬ್ಬರ ಅಧ್ಯಾತ್ಮಿಕ ದಾರಿಯಲ್ಲಿ ಯಾವ ಚಟುವಟಿಕೆಯ ಕಡೆಗೆ ಯಾವಾಗ ಗಮನಹರಿಸಬೇಕು, ಉಪಯೋಗವಿಲ್ಲದ ಯಾವ ಚಟುವಟಿಕೆಯಿಂದ ಹೊರಬರಬೇಕು ಎಂಬುದನ್ನು ಬುದ್ಧಿವಂತಿಕೆಯಿಂದ ತಿಳಿದಿರಬೇಕು. ಎಲ್ಲರೂ ಯಾವ ಚಟುವಟಿಕೆಗೆ ಕಾರ್ಯ ಪ್ರವೃತ್ತರಾಗಬೇಕು ಮತ್ತು ಅದರಿಂದ ಯಾವಾಗ ಹೊರಬರಬೇಕು ಎಂಬ ಬುದ್ಧಿವಂತಿಕೆಯನ್ನು ಬೆಳೆಸಿಕೊಳ್ಳಬೇಕು.
\end{inspiration}
\newpage

\begin{mananam}{\mananamfont \large ಮನನ ಶ್ಲೋಕ - ೩}
\mananamtext ನನಗೆ ನನ್ನ ಅಭಿರುಚಿಗಳು ಮತ್ತು ಇಷ್ಟವಿಲ್ಲದವುಗಳ ಬಗ್ಗೆ ತಿಳಿದಿದೆಯೇ? ಮತ್ತು ಅವು ನನ್ನ ಜೀವನದ ನಿರ್ಧಾರಗಳಲ್ಲಿ ಹೇಗೆ ನನ್ನ ಆಯ್ಕೆಗಳನ್ನು ನಿಬಂಧಿಸುತ್ತವೆ? ನಾನು ಅರಿವಿಲ್ಲದ ಅಭಿರುಚಿಗಳಿಗೆ ಆಶ್ರಯ ಕೊಟ್ಟಗ್ಯೂ ಮತ್ತು ಬಲವಾದ ಇಚ್ಛೆಗಳಿಗೆ ಆಶ್ರಯ ಕೊಟ್ಟಾಗಲೂ ಅಥವಾ ಅದರ ಮೇಲೆ ಅತಿಯಾದ ಮೋಹವಿದ್ದಾಗಲೂ ಫಲಕಾರಿಯಲ್ಲದ ಕೃತ್ಯಗಳನ್ನು ಮುಂದುವರೆಸದಿರಲು ಮತ್ತು ಕರ್ತವ್ಯ ತತ್ಪರವಾದ ಕೃತ್ಯಗಳನ್ನು ಮುಂದುವರಿಸಲು ಸಾಮರ್ಥನಾಗಿದ್ದೇನೆಯೇ? ನಾನು ಮೋಹಕ್ಕೆ ಒಳಗಾಗದೆ ಉತ್ಸಾಹಭರಿತವಾಗಿ ಕೆಲಸಮಾಡುವುದನ್ನು ಹೇಗೆ ಕಲಿಯಲಿ?
\end{mananam}
\WritingHand\enspace\textbf{ಆತ್ಮ ವಿಮರ್ಶೆ}\\
\begin{inspiration}{\mananamfont \large ಸ್ಪೂರ್ತಿ}
\ssmall \mananamtext ವಿಷಯಕ್ಕೆ ಸಂಬಂಧಿಸಿದಂತೆ ಒಬ್ಬ ಸನ್ಯಾಸಿಯಾಗಲಿ ಅಥವಾ ಸಂಸಾರಿಯಾಗಲಿ ಮನದೊಳಗಿನ ವೈರಾಗ್ಯವು ಜೀವನದ ಜಾಣತನಕ್ಕೆ ಅಗತ್ಯವಾದದ್ದು. ಈ ತರಹ ಇರುವವರು ಮನಸ್ಸಿನಲ್ಲಿ ವಿರಕ್ತಿ ಜ್ಞಾನವನ್ನು ಸಂಪೂರ್ಣವಾಗಿ ಅಭ್ಯಾಸ ಮಾಡಿದ ಸನಾತನ ಸನ್ಯಾಸಿ ಆಗಿರುತ್ತಾರೆ. ಅವನು ಅಥವಾ ಅವಳು ಸಾಧನೆಗೆ ಸಹಕಾರಿಯಾಗುವಂತಹ ಕೆಲವು ಜೀವನದ ಶೈಲಿ ಅಥವಾ ಅಭ್ಯಾಸಗಳಲ್ಲಿ ಇಚ್ಛೆ ಉಳ್ಳವರಾಗಿರುತ್ತಾರೆ. ಆದರೆ ಅವರು ಆಂತರ್ಯದಲ್ಲಿ ಯಾವುದರ ಅಥವಾ ಯಾರ ಬಗ್ಗೆ ಯಾಗಲಿ ಇಷ್ಟಪಡುವುದು ಅಥವಾ ಇಷ್ಟಪಡದೇ ಇರುವುದಕ್ಕೆ ಆಶ್ರಯಿಸಿರುತ್ತಾರೆ. ಅಂತಹ ಸಂದರ್ಭ ಬಂದರೆ ಅವರು ತೆಗೆದುಕೊಂಡಿರುವ ಏನನ್ನು ಬೇಕಾದರೂ ತ್ವರಿತವಾಗಿ ತ್ಯಜಿಸಿ, ಜೀವನದಲ್ಲಿ ಮುಂದೆ ಸಾಗುತ್ತಾರೆ.
\end{inspiration}
\newpage

\slcol{\Index{ಸಾಂಖ್ಯಯೋಗೌ ಪೃಥಗ್ಬಾಲಾಃ} ಪ್ರವದಂತಿ ನ ಪಂಡಿತಾಃ ।\\
ಏಕಮಪ್ಯಾಸ್ಥಿತಃ ಸಮ್ಯಗುಭಯೋರ್ವಿಂದತೇ ಫಲಮ್ ॥ ೪ ॥}
\cquote{ಜ್ಞಾನಮಾರ್ಗ ಕರ್ಮ ಮಾರ್ಗಗಳು ಬೇರೆ ಬೇರೆ ಎಂಬುದು ತಿಳಿಯದವರ ಮಾತೇ ಹೊರತು ತಿಳಿದವರ ಮಾತಲ್ಲ. ಯಾವ ಒಂದನ್ನು ಚೆನ್ನಾಗಿ ನಡೆಸಿದರು ಅವನು ಎರಡರ ಫಲವನ್ನು ಪಡೆಯುತ್ತಾನೆ.\\}
\slcol{\Index{ಯತ್ಸಾಂಖ್ಯೈಃ ಪ್ರಾಪ್ಯತೇ} ಸ್ಥಾನಂ ತದ್ಯೋಗೈರಪಿ ಗಮ್ಯತೇ ।\\
ಏಕಂ ಸಾಂಖ್ಯಂ ಚ ಯೋಗಂ ಚ ಯಃ ಪಶ್ಯತಿ ಸ ಪಶ್ಯತಿ ॥ ೫ ॥}
\cquote{ಯಾವ ಯೋಗಿಗಳಿಗೆ ಲಭಿಸುವ ಫಲವು ಕರ್ಮಯೋಗಿಗಳಿಗೂ ಲಭಿಸುತ್ತದೆ. ಆ ಎರಡನ್ನು ಒಂದೇ ಎಂದು ತಿಳಿದವನು ಯಥಾರ್ಥ ಜ್ಞಾನಿ.\\}
\slcol{\Index{ಸಂನ್ಯಾಸಸ್ತು ಮಹಾಬಾಹೋ} ದುಃಖಮಾಪ್ತುಮಯೋಗತಃ ।\\
ಯೋಗಯುಕ್ತೋ ಮುನಿರ್ಬ್ರಹ್ಮ ನಚಿರೇಣಾಧಿಗಚ್ಛತಿ ॥ ೬ ॥}
\cquote{ಅರ್ಜುನ ಕರ್ಮ ಯೋಗದ ನೆರವಿಲ್ಲದೆ ರಾಗ ದ್ವೇಷಗಳನ್ನು ಮೀರಿ ನಿಲ್ಲುವುದು ಕಷ್ಟ. ಫಲದಾಸೆ ಇಲ್ಲದೆ ಕರ್ಮ ಮಾಡುವ ದ್ವಂದ್ವತೀತನು ತಡವಿಲ್ಲದೆ ಆತ್ಮ ಜ್ಞಾನವನ್ನು ಹೊಂದುತ್ತಾನೆ.\\}
\slcol{\Index{ಯೋಗಯುಕ್ತೋ ವಿಶುದ್ಧಾತ್ಮಾ} ವಿಜಿತಾತ್ಮಾ ಜಿತೇಂದ್ರಿಯಃ ।\\
ಸರ್ವಭೂತಾತ್ಮಭೂತಾತ್ಮಾ ಕುರ್ವನ್ನಪಿ ನ ಲಿಪ್ಯತೇ ॥ ೭ ॥}
\cquote{ಕರ್ಮ ಯೋಗದಲ್ಲಿ ನಿರತನಾಗಿ ಶುದ್ಧವಾದ ಮನಸ್ಸಿನಿಂದ ದೇಹವನ್ನು ಹಿಡಿತದಲ್ಲಿ ಇಟ್ಟುಕೊಂಡು ಇಂದ್ರಿಯಗಳನ್ನು ಬಿಗಿ ಹಿಡಿದು ಎಲ್ಲೆಡೆಯೂ ಅಂತರ್ಯಾಮಿಯಾಗಿರುವ ಭಗವಂತನನ್ನೇ ಸದಾ ನೆನೆಸುವವನು ಕರ್ಮ ಮಾಡಿದರೂ ಅದರ ಕಟ್ಟಿಗೊಳಗಾಗುವುದಿಲ್ಲ.\\}
\slcol{\Index{ನೈವ ಕಿಂಚಿತ್ಕರೋಮೀತಿ} ಯುಕ್ತೋ ಮನ್ಯೇತ ತತ್ತ್ವವಿತ್ ।\\
ಪಶ್ಯನ್‍ ಶೃಣ್ವನ್‍ ಸ್ಪೃಶನ್‍ ಜಿಘ್ರನ್ನಶ್ನನ್‍ ಗಚ್ಛನ್‍ ಸ್ವಪನ್‍ ಶ್ವಸನ್ ॥ ೮ ॥}
\cquote{\textenglish{missing}\\}

\newpage
\begin{mananam}{\mananamfont \large ಮನನ ಶ್ಲೋಕ - ೬}
\mananamtext ನಾನು ಹೃದಯಪೂರ್ವಕವಾಗಿ ಅಧ್ಯಾತ್ಮಿಕ ಜೀವನವನ್ನು ಅನುಸರಿಸಲು ಬಯಸುತ್ತೇನೆಯೇ? ನಾನು ಅಧ್ಯಾತ್ಮದ ಅನುಸರಣೆಗಾಗಿ ಎಲ್ಲಾ ಚಟುವಟಿಕೆಗಳನ್ನು ಅಥವಾ ಹೆಚ್ಚಿನ ಬೇರೆ ಚಟುವಟಿಕೆಗಳನ್ನು ಬಿಡಲು ಸಿದ್ದನಿದ್ದೇನೆಯೇ? ಪ್ರಾಪಂಚಿಕ ಚಟುವಟಿಕೆಗಳನ್ನು ತ್ಯಜಿಸಲು ಅಥವಾ ನಿವೃತ್ತಿಯನ್ನು ಅವಧಿಗೆ ಮೊದಲೇ ತೆಗೆದುಕೊಳ್ಳಲು ಅಥವಾ ಸರಿಯಾದ ಸಮಯ ಯಾವಾಗ ಬರುತ್ತದೆ ಆಗ ಪಡೆಯಲು ನನಗೆ ಯೋಗ್ಯತೆ ಇದೆಯೇ? ನನಗೆ ಅಧ್ಯಾತ್ಮದ ಶಿಸ್ತು ಇದೆಯೇ ಅಥವಾ ಸಾಧನೆಗೆ ಕಟ್ಟು ಬೀಳಬಲ್ಲೆನೇ? ನಾನು ಹೇಗೆ ಸಮಯದ ಸದುಪಯೋಗ ಪಡಿಸಿಕೊಳ್ಳಲಿ ಮತ್ತು ನಾನು ಅಧ್ಯಾತ್ಮದ ಹಾದಿಯಲ್ಲಿ ಹೇಗೆ ಮುಂದೆ ಸಾಗಲಿ?
\end{mananam}
\WritingHand\enspace\textbf{ಆತ್ಮ ವಿಮರ್ಶೆ}\\
\begin{inspiration}{\mananamfont \large ಸ್ಪೂರ್ತಿ}
\mananamtext ಹೃದಯಪೂರ್ವಕವಾಗಿ ಅಧ್ಯಾತ್ಮಿಕ ದಾರಿಯಲ್ಲಿ ಸಾಗಲು ಸಾಧನೆಗೆ ಕಟ್ಟು ಬೀಳಲೇಬೇಕು. ಶಿಸ್ತಿನ ಜೀವನವಿಲ್ಲದ ಹೊರತು ಪ್ರಪಂಚಿಕ ಜೀವನವನ್ನು ಭಾಗಶಃವಾಗಲೀ ಅಥವಾ ಪೂರ್ತಿಯಾಗಲೀ ತ್ಯಜಿಸುವುದು ಖಂಡಿತವಾಗಿಯೂ ಜ್ಞಾನವನ್ನು ತೃಪ್ತಿಪಡಿಸುವುದು ಅಥವಾ ಸೋಮಾರಿತನ ವಾಗಿರುತ್ತದೆ. (ಇದನ್ನು ಬೇರೆ ರೀತಿಯಲ್ಲಿ ತಾಮಸ ಜೀವನ ಎನ್ನುತ್ತಾರೆ) ಸಂಪ್ರದಾಯ ಬದ್ಧವಾದ ಆಧ್ಯಾತ್ಮಿಕ ದಾರಿ ಎಂದರೆ ಸೇವೆ, ಆರಾಧನೆ, ಧ್ಯಾನ ಮತ್ತು ಅಧ್ಯಯನಗಳಿಂದ ಕೂಡಿದ್ದಾಗಿದೆ.ಮತ್ತು ಸಂಪೂರ್ಣವಾಗಿ ಆಧ್ಯಾತ್ಮಿಕ ಜೀವನ ಬಯಸುವವರಿಗೆ, ಅವರ ಜೀವನ ಯಾವ ಹಂತದಲ್ಲೇ ಇರಲಿ, ಅಧ್ಯಯನವೇ ಉತ್ತಮವಾದ ದಾರಿಯಾಗಿದೆ.
\end{inspiration}
\newpage

\slcol{\Index{ಪ್ರಲಪನ್‍ ವಿಸೃಜನ್ ಗೃಹ್ಣನ್ನು}ನ್ಮಿಷನ್ನಿಮಿಷನ್ನಪಿ ।\\
ಇಂದ್ರಿಯಾಣೀಂದ್ರಿಯಾರ್ಥೇಷು ವರ್ತಂತ ಇತಿ ಧಾರಯನ್ ॥ ೯ ॥}
\cquote{ಆತ್ಮ ಸ್ವರೂಪವನ್ನು ತಿಳಿದು ಮನಸ್ಸನ್ನು ಹಿಡಿತದಲ್ಲಿ ಇಟ್ಟುಕೊಂಡಿರುವವನು ನೋಡಿದರೂ ಕೇಳಿದರೂ ಮುಟ್ಟಿದರರೂ ಮೂಸಿದರೂ ತಿಂದರೂ ಹೋದರೂ ನಿದ್ರೆ ಮಾಡಿದರರೂ ಉಸಿರುಬಿಟ್ಟರೂ ಮಾತನಾಡಿದರೂ ಮಲಮೂತ್ರಗಳನ್ನು ಬಿಟ್ಟರೂ ತೆಗೆದುಕೊಂಡರೂ ಕಣ್ಣು ಬಿಟ್ಟರೂ ಕಣ್ಣು ಮುಚ್ಚಿದರೂ ಇಂದ್ರಿಯಗಳು ವಿಷಯಗಳೊಡನೆ ವ್ಯವಹರಿಸುತ್ತಿರುವವೆಂದು ಮನಸ್ಸಿನಲ್ಲಿ ಎಣಿಸುತ್ತಾ ತಾನು ಏನನ್ನೂ ಮಾಡುವವನಲ್ಲವೆಂದು ತಿಳಿಯುವನು.\\}
\slcol{\Index{ಬ್ರಹ್ಮಣ್ಯಾಧಾಯ ಕರ್ಮಾಣಿ} ಸಂಗಂ ತ್ಯಕ್ತ್ವಾ ಕರೋತಿ ಯಃ ।\\
ಲಿಪ್ಯತೇ ನ ಸ ಪಾಪೇನ ಪದ್ಮಪತ್ರಮಿವಾಂಭಸಾ ॥ ೧೦ ॥}
\cquote{ಈಶ್ವರಾರ್ಪಣ ಬುದ್ಧಿಯಿಂದ ಫಲಾಸಕ್ತಿಯನ್ನು ಬಿಟ್ಟು ಕರ್ಮ ಮಾಡುವವನನ್ನು ಕಮಲದ ಎಲೆಯನ್ನು ನೀರು  ಮುಟ್ಟದಂತೆ, ಪಾಪವು ಅಂಟುವುದಿಲ್ಲ.\\}
\slcol{\Index{ಕಾಯೇನ ಮನಸಾ ಬುದ್ಧ್ಯಾ} ಕೇವಲೈರಿಂದ್ರಿಯೈರಪಿ ।\\
ಯೋಗಿನಃ ಕರ್ಮ ಕುರ್ವಂತಿ ಸಂಗಂ ತ್ಯಕ್ತ್ವಾತ್ಮಶುದ್ಧಯೇ ॥ ೧೧ ॥}
\cquote{ಕರ್ಮಯೋಗಿಗಳು ಆಸಕ್ತಿಯನ್ನು ಬಿಟ್ಟು ಶರೀರದಿಂದ ಮನಸ್ಸಿನಿಂದ ಬುದ್ಧಿಯಿಂದ ಆಯಾ ಇಂದ್ರಿಯಗಳಿಂದ ಆತ್ಮ ಶುದ್ಧಿಗಾಗಿ ಕರ್ಮಗಳನ್ನು ಮಾಡುತ್ತಾರೆ.\\}
\slcol{\Index{ಯುಕ್ತಃ ಕರ್ಮಫಲಂ ತ್ಯಕ್ತ್ವಾ} ಶಾಂತಿಮಾಪ್ನೋತಿ ನೈಷ್ಠಿಕೀಮ್ ।\\
ಅಯುಕ್ತಃ ಕಾಮಕಾರೇಣ ಫಲೇ ಸಕ್ತೋ ನಿಬಧ್ಯತೇ ॥ ೧೨ ॥}
\cquote{ಕರ್ಮದ ಫಲವನ್ನು ತೊರೆದು ಈಶ್ವರನ ಪ್ರೀತಿಗಾಗಿ ಮಾಡುವವನು ಶಾಶ್ವತವಾದ ಶಾಂತಿಯನ್ನು ಪಡೆಯುತ್ತಾನೆ. ಬಯಕೆಗಳಿಗೆ ಬಲಿಯಾಗಿ ಫಲದಾಸೆಯಿಂದ ಕರ್ಮ ಮಾಡುವವನು ಬಾಳಿನ ಬಂಧನಕ್ಕೂಳಗಾಗುವನು.\\}
\slcol{\Index{ಸರ್ವಕರ್ಮಾಣಿ ಮನಸಾ} ಸಂನ್ಯಸ್ಯಾಸ್ತೇ ಸುಖಂ ವಶೀ ।\\
ನವದ್ವಾರೇ ಪುರೇ ದೇಹೀ ನೈವ ಕುರ್ವನ್ನ ಕಾರಯನ್ ॥ ೧೩ ॥}
\cquote{ಜ್ಞಾನಿಯು ದೇಹೇಂದ್ರಿಯಗಳನ್ನು ಹಿಡಿತದಲ್ಲಿಟ್ಟುಕೊಂಡು ತಾನು ಮಾಡುವವನಲ್ಲವೆಂಬ ಅನುಸಂಧಾನದಿಂದ 9 ಬಾಗಿಲು ಉಳ್ಳ ದೇಹವೆಂಬ ಪಟ್ಟಣದಲ್ಲಿ ಹಾಯಾಗಿರುವನು. ಅವನು ಮಾಡಿದರೂ ಮಾಡಿದಂತಲ್ಲ. ಮಾಡಿಸಿದರೂ ಮಾಡಿಸಿದ್ದಂತಲ್ಲ. \\}

\newpage

\begin{mananam}{\mananamfont \large ಮನನ ಶ್ಲೋಕ - ೯}
\mananamtext ಈ ಪ್ರಪಂಚದಲ್ಲಿ ನನ್ನ ಕೃತ್ಯಗಳಿಗೆ ಗುರುತಿಸಲ್ಪಡದೆ ಹೇಗೆ ನಾನು ಕಾರ್ಯನಿರ್ವಹಿಸುವುದನ್ನು ಕಲಿಯಲಿ? ನನ್ನ ಜ್ಞಾನೇಂದ್ರಿಯಗಳ ಕಾರ್ಯನಿರ್ವಹಣೆ ಮತ್ತು ನನ್ನ ಅದರ ಬಗೆಗಿನ ಅಗಲಿಕೆಯ ಮಧ್ಯೆ ಅಂತರವನ್ನು ಸೃಷ್ಟಿಸಬಲ್ಲೆನೇ? ಅಥವಾ ‘ನಾನು ಇದ್ದೇನೆ’ ಎಂಬ ಪ್ರಜ್ಞೆ ಇದೆ ಎಂದು ನಾನು ಗಮನಿಸುತ್ತೇನೆ ಮತ್ತು ನಂತರ ಹುಟ್ಟಿಕೊಳ್ಳುವ ‘ನಾನೇ ಮಾಡುವವನು’ ಎಂಬ ಪ್ರಜ್ಞೆಯನ್ನು ಗಮನಿಸುತ್ತೇನೆಯೇ? ನಾನು ನೋಡುವುದರ ರೀತಿಯ ಬಗ್ಗೆ ನಾನು ಧನಾತ್ಮಕ ಅಥವಾ ಋಣಾತ್ಮಕವಾದ ಅಂತಹ ಮಾನಸಿಕ ಪ್ರತಿಕ್ರಿಯೆ ಕೊಡುವಂತಹ ದಿವ್ಯ ಜ್ಞಾನಿ ಆಗುವ ತನಕ ನಾನು ನೋಡುವ ವಸ್ತು ಒಂದು ತಟಸ್ಥವಾದದ್ದು  ಎಂದು ಭಾವಿಸುತ್ತೇನೆ. ಇವೆರಡೂ ಹೇಗೆ ಎಂಬುದನ್ನು ಮನದಲ್ಲಿ ಇರಿಸಿ ನನ್ನ ನಿತ್ಯ ಜೀವನದಲ್ಲಿ ಅನುಸರಿಸಬಲ್ಲೆನೇ? (ಕೇವಲ ನೋಡುವುದರ ಎದುರು ಪೂರ್ವಾಗ್ರಹಿತವಾಗಿ ನೋಡುವುದು) ಇದು ಮನಸ್ಸಿನ ಮೇಲೆ ಬೇರೆ ಬೇರೆ ಪ್ರಭಾವ ಬೀರುತ್ತದೆಯೇ?
\end{mananam}
\WritingHand\enspace\textbf{ಆತ್ಮ ವಿಮರ್ಶೆ}\\
\begin{inspiration}{\mananamfont \large ಸ್ಪೂರ್ತಿ}
\mananamtext ನಮ್ಮ ದೈನಂದಿನ ಚಟುವಟಿಕೆಗಳೊಂದಿಗೆ ಆತ್ಮ ಜ್ಞಾನದ ಮೇಲೆ ಹಿಡಿತವನ್ನು ಸಾಧಿಸುವ ಕಲೆಯನ್ನು ಕಲಿಯುದರಿಂದ ನಮಗೆಲ್ಲರಿಗೂ ಲಾಭದಾಯಕವಾಗುತ್ತದೆ. ಅನಾವಶ್ಯಕವಾದ ಉದ್ರೇಕ ಅಥವಾ ಒತ್ತಡಕ್ಕೆ ಒಳಗಾಗದೆ ನಮ್ಮ ಕೆಲಸವನ್ನು ಮಾಡುವುದು ಮತ್ತು ನಮ್ಮ ಎಲ್ಲಾ ಅನುಭವಗಳನ್ನು ಎದುರಿಸುವುದೇ ಇದರ ಧ್ಯೇಯವಾಗಿದೆ. ನಮ್ಮೊಳಗೆ ಸದಾ ಕಾಲಕ್ಕೂ ನಾವು ಎಷ್ಟರಮಟ್ಟಿಗೆ ವೈಯಕ್ತಿಕವಾಗಿ ಗುರುತಿಸಲ್ಪಟ್ಟಿದ್ದೇವೇ ಮತ್ತು ನಮ್ಮ ಚಟುವಟಿಕೆಗಳನ್ನು ಎಷ್ಟು ವಿನಯೋಗಿಸಿದ್ದೇವೆ ಅನ್ನುವ ಆಯ್ಕೆ ಇರುತ್ತದೆ ಮತ್ತು ನಮ್ಮ ಕೆಲಸಗಳಿಂದ ನಾವು ಗುರುತಿಸಲ್ಪಟ್ಟಿದ್ದರೂ ಕೂಡ ನಾವು ತ್ವರಿತವಾಗಿ ಹೇಗೆ ನಮ್ಮ ಸ್ವಾಭಾವಿಕ ಶಾಂತಿಯ ಅಸ್ತಿತ್ವಕ್ಕೆ ವಾಪಸ್ಸು ಬರುತ್ತೇವೆ.
\end{inspiration}
\newpage


\slcol{\Index{ನ ಕರ್ತೃತ್ವಂ ನ ಕರ್ಮಾಣಿ} ಲೋಕಸ್ಯ ಸೃಜತಿ ಪ್ರಭುಃ ।\\
ನ ಕರ್ಮಫಲಸಂಯೋಗಂ ಸ್ವಭಾವಸ್ತು ಪ್ರವರ್ತತೇ ॥ ೧೪ ॥}
\cquote{ಜಗತ್ತಿನಲ್ಲಿ ಮಾಡುವಿಕೆ, ಮಾಡಿದ ಕ್ರಿಯೆ ಮತ್ತು ಕ್ರಿಯೆಗೆ ತಕ್ಕ ಪ್ರತಿಫಲ ಇವು ಯಾವುದನ್ನು ಆತ್ಮನು ನಿರ್ಮಿಸಲಾರ. ಎಲ್ಲವೂ ಸ್ವಭಾವಕ್ಕೆ ತಕ್ಕಂತೆ ಭಗವಂತನ ಇಚ್ಛೆಯಂತೆ ನಡೆದಿದೆ.\\}
\slcol{\Index{ನಾದತ್ತೇ ಕಸ್ಯಚಿತ್ಪಾಪಂ} ನ ಚೈವ ಸುಕೃತಂ ವಿಭುಃ ।\\
ಅಜ್ಞಾನೇನಾವೃತಂ ಜ್ಞಾನಂ ತೇನ ಮುಹ್ಯಂತಿ ಜಂತವಃ ॥ ೧೫ ॥}
\cquote{ಎಲ್ಲೆಲ್ಲಿಯೂ ತುಂಬಿರುವ ಭಗವಂತ ಯಾರ ಪಾಪ ಪುಣ್ಯಗಳಿಗೂ ತಾನು ಬಾಗಿಯಲ್ಲ. ಅರಿವನ್ನು ಅಜ್ಞಾನದ ತೆರೆ ಮುಚ್ಚಿದೆ. ಅದರಿಂದ ಜನ ಭ್ರಮಿಸುತ್ತಾರೆ, ಅಷ್ಟೇ.\\}
\slcol{\Index{ಜ್ಞಾನೇನ ತು ತದಜ್ಞಾನಂ} ಯೇಷಾಂ ನಾಶಿತಮಾತ್ಮನಃ ।\\
ತೇಷಾಮಾದಿತ್ಯವ{ಜ್ಜಾ\;\,}\Uchar"0CCD\Uchar"0C9Eನಂ ಪ್ರಕಾಶಯತಿ ತತ್ಪರಮ್ ॥ ೧೬ ॥}
\cquote{ಆತ್ಮದ ಅರಿವಿನ ಬೆಳಕಿನಿಂದ ಅಜ್ಞಾನದ ಕತ್ತಲನ್ನು ಕಳೆದುಕೊಂಡವರಿಗೆ ಅವರ ಆ ಅರಿವೇ ಸೂರ್ಯನಂತೆ ಆ ಪರತತ್ವವನ್ನು ಬೆಳಗಿಸುತ್ತದೆ.\\}
\slcol{\Index{ತದ್ಬುದ್ಧಯಸ್ತದಾತ್ಮಾನಸ್ತ}ನ್ನಿಷ್ಠಾಸ್ತತ್ಪರಾಯಣಾಃ ।\\
ಗಚ್ಛಂತ್ಯಪುನರಾವೃತ್ತಿಂ ಜ್ಞಾನನಿರ್ಧೂತಕಲ್ಮಷಾಃ ॥ ೧೭ ॥}
\cquote{ಭಗವಂತನಲ್ಲಿ ಬುದ್ಧಿ ನೆಲೆಗೊಳ್ಳಬೇಕು, ಬಾಳು ಅವನಿಗೆ ಅರ್ಪಿತವಾಗಬೇಕು.ಅವನೆ ಜೀವನದ ಗುರಿ ನೆಲೆಯಾಗಬೇಕು. ಇಂಥ ಅರವಿನಿಂದ ಪಾಪದ ಕೊಳೆಯನ್ನು ತೊಳೆದುಕೊಂಡವರು ಮತ್ತೆ ಮರಳದ ಶಾಶ್ವತ ಪದವನ್ನು ಪಡೆಯುವರು.\\}
\slcol{\Index{ವಿದ್ಯಾವಿನಯಸಂಪನ್ನೇ} ಬ್ರಾಹ್ಮಣೇ ಗವಿ ಹಸ್ತಿನಿ ।\\
ಶುನಿ ಚೈವ ಶ್ವಪಾಕೇ ಚ ಪಂಡಿತಾಃ ಸಮದರ್ಶಿನಃ ॥ ೧೮ ॥}
\cquote{ವಿದ್ಯೆ ವಿನಯಗಳ ನೆಲೆಯಾದ ಬ್ರಹ್ಮ ಜ್ಞಾನಿ, ಹಸು, ಆನೆ, ನಾಯಿ, ನಾಯಮಾಂಸ ತಿಂದು ಬದುಕುವ ಅನಾಗರಿಕ ಈ ಎಲ್ಲರಲ್ಲಿಯೂ ಒಬ್ಬನೇ ಭಗವಂತ ನೆಲೆಸಿದ್ದಾನೆ.ಅವನಿಗೆ ಕೀಳು ಮೇಲೆಂಬುದಿಲ್ಲ ಎಂದು ಜ್ಞಾನಿಗಳು ತಿಳಿಯುತ್ತಾರೆ.\\}
\slcol{\Index{ಇಹೈವ ತೈರ್ಜಿತಃ ಸರ್ಗೋ} ಯೇಷಾಂ ಸಾಮ್ಯೇ ಸ್ಥಿತಂ ಮನಃ ।\\
ನಿರ್ದೋಷಂ ಹಿ ಸಮಂ ಬ್ರಹ್ಮ ತಸ್ಮಾದ್ಬ್ರಹ್ಮಣಿ ತೇ ಸ್ಥಿತಾಃ ॥ ೧೯ ॥}
\cquote{ನಿರ್ದೋಷವಾದ ಬ್ರಹ್ಮವು ಸಮವಾಗಿದೆ. ಆ ಬ್ರಹ್ಮದಲ್ಲಿಯೇ ಇರುತ್ತಾರೆ. ಸಾಮ್ಯ ಬುದ್ಧಿಯುಳ್ಳ ದೊಡ್ಡವರು ಬದುಕಿರುವಾಗಲೇ ಮುಕ್ತರಾಗುತ್ತಾರೆ.\\}

\newpage
\begin{mananam}{\mananamfont \large ಮನನ ಶ್ಲೋಕ - ೧೧}
\mananamtext ಕೃತ್ಯಗಳಲ್ಲಿ ಅಹಂನ ಸ್ವಭಾವವೇನು?ಅದು ನನ್ನ ಜೀವನದಲ್ಲಿ ಹೇಗೆ ಪ್ರಕಟಿಸಲ್ಪಡುತ್ತದೆ. ಈ ಅಹಂನ ಶುದ್ಧೀಕರಣವನ್ನು ಹೇಗೆ ಪಡೆಯಬಹುದು? ಅಹಂ ಅನ್ನು ಉತ್ತೇಜಿಸುವ ಎಲ್ಲ ಕೃತ್ಯಗಳನ್ನು ತ್ಯಾಗ ಮಾಡುವ ಅವಶ್ಯಕತೆ ಇದೆಯೇ? ಅಥವಾ ಸಮತೋಲನದ ಯಾವುದಾದರೂ ಮಾರ್ಗವಿದೆಯೇ? ಯೋಗಿಗಳು ಯಾವುದಕ್ಕೂ ಅಂಟಿಕೊಳ್ಳದೆ ಕರ್ತವ್ಯ ನಿರ್ವಹಿಸಲು ಸಮರ್ಥರಾಗಿರುತ್ತಾರೆ ಎನ್ನುವುದರ ಅರ್ಥವೇನು? ಅವರು ಅಹಂನಿಂದ ಸಂಪೂರ್ಣ ಸ್ವಾತಂತ್ರರಾಗುತ್ತಾರೆಯೇ? ಅಥವಾ ಅಲ್ಲಿ ಒಂದು ತರಹದ ಕೆಲಸ ಕಾರ್ಯದ ಅಹಂ ಇರುತ್ತದೆಯೇ? ಹಾಗಿದ್ದರೆ ಅದರ ಸ್ವಭಾವವೇನು?
\end{mananam}
\WritingHand\enspace\textbf{ಆತ್ಮ ವಿಮರ್ಶೆ}\\
\begin{inspiration}{\mananamfont \large ಸ್ಪೂರ್ತಿ}
\mananamtext ಎಲ್ಲಾ ಸನಾತನ ಸಂಪ್ರದಾಯಗಳು ಮನಸ್ಸನ್ನು ಪರಿಶುದ್ಧಕರಿಸುವ ಸಲುವಾಗಿ ಸ್ವಾರ್ಥ ರಹಿತ ಕೃತ್ಯಗಳಿಗೆ ಒತ್ತು ಕೊಟ್ಟಿವೆ. ಒಬ್ಬನು ಯಾವುದೇ ರೀತಿಯ ಸಾಧನೆಯನ್ನು ಸ್ವೀಕರಿಸಿದ್ದರು ಸೇವೆ ಮತ್ತು ಕರ್ಮ ಯೋಗಗಳು ಒಬ್ಬ ಸಾಧಕನಿಗೆ ಅಮೂಲ್ಯವಾದವುಗಳು. ಬೇರೆ ಬೇರೆ ಹಂತಗಳಲ್ಲಿ  ಕಾರ್ಯ  ನಿರ್ವಹಿಸುವ ಅಹಂ ಅನ್ನು ತೊಡೆದು ಹಾಕಲು ಇದರ ಹೊರೆತು ಬೇರೆ ಮಾರ್ಗವು ಕಠಿಣವಾಗಿರುತ್ತದೆ ಮತ್ತು ಸಾಧಕನ ಜೀವನದಲ್ಲಿ ಬೇರೆ ಬೇರೆ ರೀತಿಯಲ್ಲಿ ಪ್ರದರ್ಶಿಸಲ್ಪಡುತ್ತದೆ.
\end{inspiration}

\newpage
\begin{mananam}{\mananamfont \large ಮನನ ಶ್ಲೋಕ - ೧೩}
\mananamtext ದೇಹದ ಬಾಗಿಲುಗಳಾದ ಕಣ್ಣುಗಳು ಮತ್ತು ಕಿವಿಗಳು  ಭಾಹ್ಯಜಗತ್ತಿನೊಂದಿಗೆ ಪರಸ್ಪರ ಕೆಲಸ ಮಾಡಲು ಮಾಧ್ಯಮಗಳಾಗಿವೆ ಎನ್ನುವುದರ ತಿಳುವಳಿಕೆ  ನನಗಿದೆಯೇ? ಈ ಬಾಗಿಲುಗಳಿಂದ ಏನು ಒಳಗೆ ಬರುತ್ತದೆ ಏನು ಹೊರಗೆ ಹೋಗುತ್ತದೆ ಎನ್ನುವುದು ನನ್ನ ಲೌಕಿಕ ಮತ್ತು ದೈಹಿಕ ಪ್ರತಿಕ್ರಿಯೆಗಳಿಂದ ಮನಸ್ಸಿನಲ್ಲಿ ಯಾವ ರೀತಿಯ ಪ್ರಭಾವ ಬೀರುತ್ತವೆ ಎನ್ನುವುದರ ತಿಳುವಳಿಕೆ ನನಗಿದೆಯೇ? ಆಯ್ಕೆಯ ಸಮಯದಲ್ಲಿ ನನ್ನ ಗಮನಕ್ಕೆ ಬರುವ ಮತ್ತು ಈ ಬಾಗಿಲುಗಳಿಂದ ಹೊರಹೋಗುವ ಮತ್ತು ಗಮನಕ್ಕೆ ಬರುವ ವಸ್ತುಗಳ ಚೈತನ್ಯದ ಬಗ್ಗೆ ಸಂಪೂರ್ಣ ಗುರುತಿಸುವಿಕೆಯ ಬಗ್ಗೆ ತಿಳುವಳಿಕೆ ನನಗಿದೆಯೇ?
\end{mananam}
\WritingHand\enspace\textbf{ಆತ್ಮ ವಿಮರ್ಶೆ}\\
\begin{inspiration}{\mananamfont \large ಸ್ಪೂರ್ತಿ}
\ssmall \mananamtext ನಮ್ಮ ಒಳಗಿರುವ ಅಸ್ತಿತ್ವವು ಉನ್ನತ ಮಟ್ಟದ್ದಾಗಿದೆ. ಹೊರ ಪ್ರಪಂಚದೊಳಗಿನ ನಮ್ಮ ಪರಸ್ಪರ ಮಾತುಗಳಿಂದ ನಮ್ಮ ಪ್ರಾಮಾಣಿಕ ವ್ಯಕ್ತಿತ್ವವನ್ನು ಕಳೆದುಕೊಳ್ಳಲು ಕಾರಣವಾಗಿರುತ್ತದೆ. ಮತ್ತೊಬ್ಬರೊಂದಿಗಿನ ಮಾತುಕತೆಯಲ್ಲಿ ಅನೇಕ ಸಮಸ್ಯೆಗಳು ಉದ್ಭವಿಸುವಂತೆ ಮಾಡುತ್ತದೆ. ನಾವು ಆತ್ಮಪ್ರಜ್ಞೆಗೆ ಅಂಟಿಕೊಳ್ಳಲು ಕಲಿಯಬೇಕು. ಅಥವಾ ನಮ್ಮ ಎಲ್ಲ ಚಟುವಟಿಕೆಗಳನ್ನು ನಡೆಸುವಾಗ ಪ್ರಜ್ಞಾಪೂರ್ವಕವಾಗಿ   ಸಾಕ್ಷಿಯಾಗಿರಬೇಕು. ಹೂವನ್ನು ನೋಡುವುದು, ಪಕ್ಷಿಯು ಆಕಾಶದಲ್ಲಿ ಹಾರಾಡುವುದು, ದೂರದಿಂದ ಕೇಳಿ ಬರುತ್ತಿರುವ ಶಬ್ದಗಳನ್ನು ಆಲಿಸುವುದು, ನಿಮ್ಮ ಮನೆಯಲ್ಲಿ ಅಥವಾ ಮನೆಯ ಮಾಳಿಗೆಯಲ್ಲಿ ನಡೆದಾಡುವುದು, ನಿಮ್ಮದೇ ಉಸಿರಾಟವನ್ನು ಗಮನಿಸುವುದು ಮುಂತಾದ ಸಣ್ಣ ಸಣ್ಣ ಪ್ರಯೋಗಗಳನ್ನು ಮೊದಲಿಗೆ ಮಾಡಬೇಕು. ಈ ಎಲ್ಲ ಚಟುವಟಿಕೆಗಳನ್ನು ನಮ್ಮ ಒಳಾಂತರಾಳ ಅಥವಾ ನಮ್ಮ ಉತ್ಕೃಷ್ಟ ಸ್ವಭಾವದ ಕಡೆಗೆ ಭಾಗಶಃ ಪ್ರಜ್ಞೆಯನ್ನು ಇರಿಸಿಕೊಂಡಿರಬೇಕು.
\end{inspiration}
\newpage

\begin{mananam}{\mananamfont \large ಮನನ ಶ್ಲೋಕ - ೧೬ ೧೭}
\mananamtext ನಮ್ಮ ಸನಾತನ ಗ್ರಂಥಗಳು ನಾವು ಅಜ್ಞಾನಿಗಳೆಂದು ಏಕೆ ಹೇಳುತ್ತವೆ? ನನ್ನ ಶೈಕ್ಷಣಿಕ ವಿದ್ಯಾರ್ಹತೆ ಮತ್ತು ನಾಗರಿಕ ಪಾಲನೆಯ ಹೊರತಾಗಿಯೂ ಜ್ಞಾನದ ಯಾವ ಕ್ಷೇತ್ರಗಳು ನನ್ನಿಂದ ತಪ್ಪಿಸಿಕೊಳ್ಳುತ್ತಿವೆ? ನನಗೆ ಪ್ರಾಪಂಚಿಕ ಜ್ಞಾನ ಮತ್ತು ಅಧ್ಯಾತ್ಮಿಕ ತಿಳುವಳಿಕೆಗಳ ವ್ಯತ್ಯಾಸವನ್ನು ತಿಳಿಯುವ ವಿವೇಚನೆಯ ಸಾಮರ್ಥ್ಯವಿದೆಯೇ? ಯಾವ ಶ್ರೇಷ್ಠವಾದ  ಜ್ಞಾನವು  ಸತ್ಯದ ಕಡೆಗೆ ನನ್ನನ್ನು ಕರೆದೊಯ್ಯಬಹುದೇ? ನನ್ನ ಯಾವ ಅಭ್ಯಾಸಗಳು ಅಂತಿಮ ಸತ್ಯದೆಡೆಗೆ ನನ್ನ ಮನಸ್ಸನ್ನು ಸಮ ಸ್ಥಿತಿಗೆ ತರಬಹುದು? ಈ ಅಭ್ಯಾಸಗಳು ಯಾವ ರೀತಿ ನನ್ನ ನಿತ್ಯ ಜೀವನದಲ್ಲಿ ಬದಲಾವಣೆ ತರಬಹುದು?
\end{mananam}
\WritingHand\enspace\textbf{ಆತ್ಮ ವಿಮರ್ಶೆ}\\
\begin{inspiration}{\mananamfont \large ಸ್ಪೂರ್ತಿ}
\mananamtext ಉಚ್ಛ್ರಾಯ   ಸ್ಥಿತಿಯಲ್ಲಿರುವ ಎಲ್ಲಾ ಆಧ್ಯಾತ್ಮಿಕ ದಾರಿಯು ಅಜ್ಞಾನವನ್ನು ಉಚ್ಚಾಟಿಸುವುದೇ   ಆಗಿದೆ. ಹೇಗೆ ಕತ್ತಲೆಯು ಬೆಳಕಿನಿಂದ ಹೋಗಲಾಡಿಸಲ್ಪಡುವುದೊ   ಹಾಗೆಯೇ ಜ್ಞಾನವು ಅಜ್ಞಾನವನ್ನು ಹೋಗಲಾಡಿಸುವುದು. ಅಜ್ಞಾನದ ಒಂದು ಮಹತ್ತರವಾದ ಪರಿಣಾಮವೆಂದರೆ ಅದು ನಮ್ಮ ನಿಜವಾದ ಅಸ್ತಿತ್ವವನ್ನು ಮುಚ್ಚಿಡುತ್ತದೆ. ಅದರ ಪರಿಣಾಮವಾಗಿ ನಮ್ಮ ಎಲ್ಲ ಕ್ರಿಯೆಗಳು ಕತ್ತಲೆಯಿಂದ ಹುಟ್ಟುತ್ತದೆ. ನಮ್ಮ ಅತ್ಯುನ್ನತವಾದ ಬುದ್ಧಿವಂತಿಕೆಯು ನಮ್ಮ ಸತತ ಪ್ರಯತ್ನಗಳ ಫಲವಾಗಿ ಲಭಿಸಿದ ಪ್ರಜ್ಞೆಯು ನಮ್ಮ ಎಲ್ಲಾ ನಿಬಂಧನೆಗಳನ್ನು ಒಡೆದು ಹಾಕುತ್ತವೆ.
\end{inspiration}
\newpage

\begin{mananam}{\mananamfont \large ಮನನ ಶ್ಲೋಕ - ೧೮}
\mananamtext ನಾನು ಯಾವ ಮಾಪನದಿಂದ ಸಮಾಜದ ಜನರನ್ನು ಒರೆಗೆ ಹಚ್ಚಲಿ – ಅವರ ಸ್ಥಾನಮಾನವೇ, ಅವರ ಶ್ರೀಮಂತಿಕೆಯಿಂದಲೇ, ಅವರ ವಿದ್ಯಾರ್ಹತೆಯಿಂದಲೇ, ಅವರ ಸೌಂದರ್ಯದಿಂದಲೇ ಅಥವಾ ಅವರ ಸಾಧನೆಗಳಿಂದಲೇ? ಯಾರನ್ನು ನಾನು ಹೆಚ್ಚು ಗೌರವಿಸಲಿ ಅಥವಾ ಯಾರಿಗೆ ನಾನು ಕಡಿಮೆ ಗೌರವ ತೋರಿಸಲಿ? ನಾನು ಕೆಲವು ಜನರನ್ನು ಕಡಿಮೆ ಆದರದಿಂದ ನೋಡುತ್ತೇನೆಯೇ? ಅವಶ್ಯಕತೆ ಇರುವವರ ಕಡೆಗೆ ಗಮನಹರಿಸಿ ಮತ್ತು ಅವರ ಕಡೆ ಲಕ್ಷ ಕೊಡುವುದರಿಂದ ನನಗೆ ಹೇಗೆ ಪ್ರಯೋಜನಕಾರಿ ಆಗುತ್ತದೆ. ಯಾರ ಮಹತ್ವ ಮತ್ತು ಅನುಭವದಿಂದ ನನ್ನ ಗ್ರಾಹಕ ಶಕ್ತಿಯನ್ನು ಮತ್ತು ಅಧ್ಯಾತ್ಮಿಕ ಬೆಳವಣಿಗೆಯನ್ನು ಉನ್ನತಿಗೆ ಏರಿಸಬಲ್ಲ ವಿಶೇಷವಾದ ಮಾದರಿಯಾಗಿ ನಾನು ಯಾರನ್ನು ನೋಡಲಿ?
\end{mananam}
\WritingHand\enspace\textbf{ಆತ್ಮ ವಿಮರ್ಶೆ}\\
\begin{inspiration}{\mananamfont \large ಸ್ಪೂರ್ತಿ}
\tiny \mananamtext ಭಯಭೀತಿ ಉಂಟುಮಾಡುವಂತಹ ಶ್ರೀಮಂತನಿರಲಿ, ಪ್ರಭಾವಿ ವ್ಯಕ್ತಿಯಾಗಿರಲಿ, ಅಥವಾ ಸಮಾಜದಲ್ಲಿ ಕೆಳಮಟ್ಟದ ಸ್ಥಾನ ಮಾನವಿರಲಿ, ಒಬ್ಬ ಉತ್ಕರ್ಷಿತ ಮಾನವನು ಎಲ್ಲರಲ್ಲಿಯೂ ಸಮಭಾವ ಹೊಂದಿರುತ್ತಾನೆ. \\
ಸಮಭಾವವನ್ನು ಉಪೇಕ್ಷೆ ಎಂದು ತಪ್ಪು ಭಾವಿಸಬಾರದು. ಬದಲಿಗೆ ನಮ್ಮ ಗಮನದ ಅವಶ್ಯಕತೆ ಯಾರಿಗಿದೆ ಮತ್ತು ಯಾರು ಭಯ ಭಕ್ತಿಗೆ ಯೋಗ್ಯರು ಎಂದು ಗುರುತಿಸಲು ಕಲಿಯಬೇಕು. ಅಭ್ಯಾಸದಂತೆ ಮೌಲ್ಯಯುತ ಪದ್ದತಿಯನ್ನು ರೂಪಿಸುವಲ್ಲಿ ನಮ್ಮ ಗಮನವನ್ನು ಶ್ರೀಮಂತರನ್ನು, ಪ್ರಭಾವಿ ವ್ಯಕ್ತಿಗಳನ್ನು ಗೌರವಿಸುವ ಬದಲಿಗೆ ಆಧ್ಯಾತ್ಮಿಕವಾಗಿ ಬುದ್ಧಿವಂತರಾಗಿರುವವರನ್ನು ಗೌರವಿಸುವುದರ ಕಡೆ ಪ್ರಾಶಸ್ತ್ಯ ಕೊಡಬೇಕು. ಈ ತರಹದ ಮೌಲ್ಯಯುತ ಪದ್ದತಿಯು ಆಧ್ಯಾತ್ಮಿಕವಾಗಿ ಮುಂದುವರಿದ ಸಮಾಜದ ಗುರುತಾಗಿದೆ.
\end{inspiration}
\newpage

\slcol{\Index{ನ ಪ್ರಹೃಷ್ಯೇತ್ಪ್ರಿಯಂ ಪ್ರಾಪ್ಯ} ನೋದ್ವಿಜೇತ್ಪ್ರಾಪ್ಯ ಚಾಪ್ರಿಯಮ್ ।\\
ಸ್ಥಿರಬುದ್ಧಿರಸಂಮೂಢೋ ಬ್ರಹ್ಮವಿದ್ಬ್ರಹ್ಮಣಿ ಸ್ಥಿತಃ ॥ ೨೦ ॥}
\cquote{ಸ್ಥಿರಬುದ್ದಿಯುಳ್ಳವನೂ ಮೋಹರಹಿತನು ಆದ ಬ್ರಹ್ಮ ಜ್ಞಾನಿಯು ಬ್ರಹ್ಮದಲ್ಲಿಯೇ ಸ್ಥಿರವಾಗಿರುವುದರಿಂದ ಅವನಿಗೆ ಪ್ರಿಯ ವಸ್ತುವಿನಲ್ಲಿ ಇಚ್ಚೆ ಇಲ್ಲ, ಅಪ್ರಿಯವಾದ ವಸ್ತುವಿನಲ್ಲಿ ದ್ವೇಷವು ಇರುವುದಿಲ್ಲ.\\}
\slcol{\Index{ಬಾಹ್ಯಸ್ಪರ್ಶೇಷ್ವಸಕ್ತಾತ್ಮಾ} ವಿಂದತ್ಯಾತ್ಮನಿ ಯತ್ಸುಖಮ್ ।\\
ಸ ಬ್ರಹ್ಮಯೋಗಯುಕ್ತಾತ್ಮಾ ಸುಖಮಕ್ಷಯಮಶ್ನುತೇ ॥ ೨೧ ॥}
\cquote{ಹೊರಗಣ ವಿಷಯಗಳಿಗೆ ಮನಸೋಲದವನು ತನ್ನೊಳಗೆಯೆ ಯಾವ ಸುಖವನ್ನು ಪಡೆಯುವನೂ ಅದೇ ಸುಖವನ್ನು ಬ್ರಹ್ಮತತ್ವದಲ್ಲಿ ಮನಸ್ಸನ್ನು ನೆಲೆಗೊಳಿಸಿದವನು ಅನಂತವಾಗಿ ಪಡೆಯುತ್ತಾನೆ.\\}
\slcol{\Index{ಯೇ ಹಿ ಸಂಸ್ಪರ್ಶಜಾ ಭೋಗಾ} ದುಃಖಯೋನಯ ಏವ ತೇ ।\\
ಆದ್ಯಂತವಂತಃ ಕೌಂತೇಯ ನ ತೇಷು ರಮತೇ ಬುಧಃ ॥ ೨೨ ॥}
\cquote{ವಿಷಯಗಳ ಸಂಬಂಧದಿಂದಾಗುವ ಸುಖಗಳೆಲ್ಲ ದುಃಖಕ್ಕೆ ಮೂಲವಾದುವುಗಳೇ. ಅರ್ಜುನ,ಅವು ಮೊದಲೂ ಕೊನೆಯೂ ಉಳ್ಳವುಗಳು. ವಿವೇಕಿಯು ಅವುಗಳಲ್ಲಿ ಪ್ರೀತಿಯನ್ನಿಡುವುದಿಲ್ಲ.\\}
\slcol{\Index{ಶಕ್ನೋತೀಹೈವ ಯಃ ಸೋಢುಂ} ಪ್ರಾಕ್ಶರೀರವಿಮೋಕ್ಷಣಾತ್ ।\\
ಕಾಮಕ್ರೋಧೋದ್ಭವಂ ವೇಗಂ ಸ ಯುಕ್ತಃ ಸ ಸುಖೀ ನರಃ ॥ ೨೩ ॥}
\cquote{\textenglish{missing missing missing missing missing missing missing missing missing missing missing missing missing}\\}
\slcol{\Index{ಯೋಽಂತಃಸುಖೋಽಂತರಾರಾಮಸ್ತ}ಥಾಂತರ್ಜ್ಯೋತಿರೇವ ಯಃ ।\\
ಸ ಯೋಗೀ  ಬ್ರಹ್ಮನಿರ್ವಾಣಂ ಬ್ರಹ್ಮಭೂತೋಧಿಗಚ್ಛತಿ ॥ ೨೪ ॥}
\cquote{ಆತ್ಮಸುಖದಲ್ಲಿ ಲೀನನಾಗಿ ಭಗವಂತನ ದರ್ಶನದಿಂದ ಆನಂದಗೊಳ್ಳುತ್ತ. ಒಳಗೆ ಆ ಬೆಳಕನ್ನೇ ತುಂಬಿಕೊಂಡಿರುವನೋ ಬ್ರಹ್ಮನಲ್ಲಿಯೇ ನೆಲೆಗೊಂಡ ಆ ಯೋಗಿಯು ಆನಂದರೂಪಿಯಾದ ಬ್ರಹ್ಮನನ್ನೇ ಪಡೆಯುತ್ತಾನೆ.\\}
\newpage


\begin{mananam}{\mananamfont \large ಮನನ ಶ್ಲೋಕ - ೨೦}
\footnotesize \mananamtext ಅಹ್ಲಾದಕರವಾದ ಅನುಭವವಾದಾಗ ನಾನು ಹೇಗೆ  ಪ್ರತಿಕ್ರಿಯಿಸುತ್ತೇನೆ.  ಅಹಿತಕರ ಸಂದರ್ಭಗಳನ್ನು  ನಾನು ಹೇಗೆ ನಿಭಾಯಿಸುತ್ತೇನೆ? ನನಗೆ, ಆನಂದಮಯವಾದದ್ದನ್ನು  ಮೇಲಿನದು ಮತ್ತು ನೋವನ್ನು ದೂರವಿರಿಸುವುದು ನನ್ನ ಮಾನಸಿಕ ಆರೋಗ್ಯಕ್ಕೆ ಪ್ರಯೋಜನಕಾರಿ ಎಂದು ಮನದಟ್ಟು ಮಾಡಿಕೊಂಡಿದ್ದೇನೆಯೇ? ಹಾಗಿದ್ದರೆ, ಹೇಗೆ?\\
ನನಗೆ, ಜೀವನದ ಎಲ್ಲಾ ಮಜಲುಗಳಲ್ಲಿ, ಶೀಘ್ರವಾಗಿ ಲಭಿಸುವ ಸಂತೋಷ ಮತ್ತು ದೀರ್ಘಕಾಲದ ಒಳಿತುಗಳ ಆಯ್ಕೆಯ ವಿಚಾರದಲ್ಲಿ ತಿಳುವಳಿಕೆ ಇದೆಯೇ? ನಾನು ಮಹತ್ವಪೂರ್ಣ ಪ್ರಯತ್ನ ಹಾಕಿದರೂ ಅದರ ಫಲಿತಾಂಶವನ್ನು ಅಥವಾ ಅದರ ಪರಿಣಾಮವನ್ನು ಪ್ರಶಾಂತ ಭಾವದಿಂದ ಸ್ವೀಕರಿಸುವುದರ ಬೆಲೆ ಏನು?
\end{mananam}
\WritingHand\enspace\textbf{ಆತ್ಮ ವಿಮರ್ಶೆ}\\
\begin{inspiration}{\mananamfont \large ಸ್ಪೂರ್ತಿ}
\footnotesize \mananamtext ಮನುಷ್ಯನಲ್ಲಿ ಕಂಡುಬರುವ ಸ್ವಭಾವ ಸಿದ್ದ ಮಾನಸಿಕ ಆವೇಗ ಏನು ಇದೆಯೋ ಅದನ್ನು, ಋಷಿಗಳು ಮತ್ತು ಯೋಗಿಗಳು ಅಪಾಯಕಾರಿ ಯಾದದ್ದು ಎಂದು  ನಿರ್ಣಯಿಸಿದ್ದಾರೆ.  ಬಹುಮಟ್ಟಿಗೆ ಎಲ್ಲರೂ ಹರ್ಷದಾಯಕವಾಗಿರುವುದರ ಕಡೆಗೆ ಮತ್ತು ಅಪ್ರಿಯವಾದದ್ದನ್ನು ದೂರವಿಡುವುದರತ್ತ ನುಗ್ಗುತ್ತಾರೆ. ಅದಾಗಿಯೂ,  ಈ ಪ್ರಿಯ ಅಪ್ರಿಯ  ವಿಚಾರಗಳು ಆಂತರ್ಯದ ಅಸ್ಥಿರತೆಗೆ ನಿರ್ದಿಷ್ಟವಾಗಿ ದಾರಿಯಾಗಿಸುತ್ತವೆ. ಮಾನಸಿಕ ತಮ್ಮ ಸಮಚಿತ್ತತೆಯನ್ನು  ಸಾಧಿಸಲು, ಪ್ರಿಯವಾದ ಮತ್ತು ಅಪ್ರಿಯವಾದ ಅನುಭವಗಳನ್ನು ಸ್ವೀಕರಿಸುವುದು ತುಂಬಾ ಲಾಭದಾಯಕ ಅಭ್ಯಾಸವಾಗಿದೆ.
\end{inspiration}
\newpage

\begin{mananam}{\mananamfont \large ಮನನ ಶ್ಲೋಕ - ೨೧ ೨೨}
\footnotesize \mananamtext ನನಗೆ ಇಂದ್ರಿಯ ಜ್ಞಾನ  ಮತ್ತು ವಸ್ತುಗಳ ಜ್ಞಾನಗಳ ನಡುವೆ ನಡೆಯುವ ಪರಸ್ಪರ ವಿನಿಮಯಗಳ ಬಗ್ಗೆ ಪ್ರಜ್ಞಾಪೂರ್ವಕವಾಗಿ  ತಿಳಿದಿರುವೆನೆ? ಯಾವಾಗ ನನ್ನ ಕಣ್ಣುಗಳು ಆಕರ್ಷಕವಾಗಿದ್ದನ್ನು ನೋಡಿದಾಗ ಅದನ್ನು ಪಡೆಯಬೇಕು ಅಥವಾ ಸ್ವಾಧೀನ ಪಡಿಸಿಕೊಳ್ಳಬೇಕೆಂಬ ಹಂಬಲವನ್ನು ನನ್ನ ಮನಸ್ಸು ಪ್ರಚೋದಿಸಿ, ಪ್ರತಿಕ್ರಿಯಿಸುವುದರ ಬಗ್ಗೆ ನನಗೆ ತಿಳಿದಿದೆಯೇ? ಬಾಹ್ಯ ವಸ್ತುಗಳಾದ ಆಹಾರ, ಮನೋರಂಜನೆ ಅಥವಾ ಸಾಮಾಜಿಕ ವರ್ತನೆಗಳು ನಿಜವಾಗಿಯೂ ಮುಗಿಯದ ಆತ್ಮ ಸಂತೋಷವನ್ನು ತರಬಹುದೇ? ಅಥವಾ ಅವುಗಳು ಕೇವಲ ಅಲ್ಪಕಾಲದ ತೃಪ್ತಿ ಕೊಡುವ ಸಾಮರ್ಥ್ಯ ಹೊಂದಿರುವವೇ? ನನಗೆ ಬಾಹ್ಯ ಮೂಲಗಳಿಂದ ಮಾತ್ರ ಅಶಾಶ್ವತ ಸಂತೋಷವನ್ನು ಕಾಣಲು ಸಾಧ್ಯವೆಂಬುದು ತಿಳಿದಿದೆಯೇ?
\end{mananam}
\WritingHand\enspace\textbf{ಆತ್ಮ ವಿಮರ್ಶೆ}\\
\begin{inspiration}{\mananamfont \large ಸ್ಪೂರ್ತಿ}
\ssmall \mananamtext ನಾವು ಆಗಾಗ ಬಾಹ್ಯ ವಸ್ತುಗಳಿಂದ ಆಶಾಶ್ವತವಾದ ಸಂತೋಷವನ್ನು ಪಡೆದುಕೊಳ್ಳಲು ಪ್ರಯತ್ನಿಸುತ್ತೇವೆ. ಪರಿಣಾಮವಾಗಿ ಈ ಸಂತೋಷವು ಕೂಡ ಆ ವಸ್ತುಗಳು ಅದೃಶ್ಯವಾದಂತೆ ಮಾಯವಾಗುತ್ತವೆ ಮತ್ತು ದುಃಖದೆಡೆಗೆ ಕರೆದೊಯ್ಯುತ್ತವೆ. ನಮ್ಮ ಒಳಗಿನ ಯೋಚನೆಗಳು ಮತ್ತು ಉದ್ವೇಗಗಳು ಬಾಹ್ಯ  ಪ್ರಪಂಚದೊಂದಿಗೆ ಬಲವಾಗಿ ಬಂಧಿಸಸಲ್ಪಟ್ಟಿವೆ.ಸಂತೋಷ ಮತ್ತು ದುಃಖಗಳನ್ನು ನಮ್ಮ ಒಳ ಮನಸ್ಸಿನಿಂದ ಅನುಭವಿಸುತ್ತೇವೆ. ಇದನ್ನು ನೋಡಿ ನಮಗೆ ಅಂತರ್ ಪ್ರಪಂಚದಿಂದ ಬಾಹ್ಯ ಪ್ರಪಂಚವನ್ನು ಬೇರ್ಪಡಿಸುವ ಶಕ್ತಿಯನ್ನು ಸ್ವಷ್ಟವಾಗಿ ಕೊಡುತ್ತದೆ.  ಬಾಹ್ಯ ಸನ್ನಿವೇಶಗಳ ಹೊರತಾಗಿಯೂ ಬೇಕೆಂದಾಗ ಸಂತೋಷವಾಗಿ ಇರಲು ಕಲಿಯುವುದು ಆಧ್ಯಾತ್ಮಿಕ ದಾರಿಯಲ್ಲಿ ಅತ್ಯವಶ್ಯಕವಾದ ಜಾಣ್ಮೆಯಾಗಿದೆ.
\end{inspiration}
\newpage

\slcol{\Index{ಲಭಂತೇ ಬ್ರಹ್ಮನಿರ್ವಾಣ}ಮೃಷಯಃ ಕ್ಷೀಣಕಲ್ಮಷಾಃ ।\\
ಛಿನ್ನದ್ವೈಧಾ ಯತಾತ್ಮಾನಃ ಸರ್ವಭೂತಹಿತೇ ರತಾಃ ॥ ೨೫ ॥}
\cquote{ಪಾಪರಹಿತರು, ಸಂಶಯ ಶೂನ್ಯರು, ಜಿತೇಂದ್ರಿಯರು, ಸರ್ವಭೂತಗಳ ಹಿತದಲ್ಲಿ ನಿರತರಾದ ಋಷಿಗಳು ಪರಮಮುಕ್ತಿಯನ್ನು ಪ್ರಾಪ್ತಿ ಮಾಡಿಕೊಳ್ಳುತ್ತಾರೆ.\\}
\slcol{\Index{ಕಾಮಕ್ರೋಧವಿಯುಕ್ತಾನಾಂ} ಯತೀನಾಂ ಯತಚೇತಸಾಮ್ ।\\
ಅಭಿತೋ ಬ್ರಹ್ಮನಿರ್ವಾಣಂ ವರ್ತತೇ ವಿದಿತಾತ್ಮನಾಮ್ ॥ ೨೬ ॥}
\cquote{ಬಯಕೆ ಸಿಟ್ಟುಗಳನ್ನು ತೊರೆದು ಮನಸ್ಸನ್ನು ಬಿಗಿಹಿಡಿದು ಆತ್ಮವನ್ನರಿತ ಸನ್ಯಾಸಿಗಳಿಗೆ ಎಲ್ಲೆಡೆಯೂ ಆನಂದ ರೂಪವಾದ ಬ್ರಹ್ಮವೇ ತುಂಬಿದೆ.\\}
\slcol{\Index{ಸ್ಪರ್ಶಾನ್ಕೃತ್ವಾ ಬಹಿರ್ಬಾಹ್ಯಾಂ}ಶ್ಚಕ್ಷುಶ್ಚೈವಾಂತರೇ ಭ್ರುವೋಃ ।\\
ಪ್ರಾಣಾಪಾನೌ ಸಮೌ ಕೃತ್ವಾ ನಾಸಾಭ್ಯಂತರಚಾರಿಣೌ ॥ ೨೭ ॥}
\cquote{\textenglish{missing}}
\slcol{\Index{ಯತೇಂದ್ರಿಯಮನೋಬುದ್ಧಿ}ರ್ಮುನಿರ್ಮೋಕ್ಷಪರಾಯಣಃ ।\\
ವಿಗತೇಚ್ಛಾಭಯಕ್ರೋಧೋ ಯಃ ಸದಾ ಮುಕ್ತ ಏವ ಸಃ ॥ ೨೮ ॥}
\cquote{ಹೊರಗಣ ವಿಷಯಗಳನ್ನು ಬಹಿಷ್ಕರಿಸಿ ಕಣ್ಣನ್ನು ಹುಬ್ಬುಗಳ ನಡುವೆ ನೆಲೆಗೊಳಿಸಿ ಮೂಗಿನೊಳಗೆ ಓಡಾಡುವ ಉಸಿರನ್ನು ಕುಂಭಕದಲ್ಲಿ ಬಿಗಿಹಿಡಿದು, ಇಂದ್ರಿಯ, ಮನಸ್ಸು, ಬುದ್ಧಿ ಇವುಗಳನ್ನು ಹಿಡಿತದಲ್ಲಿಟ್ಟುಕೊಂಡು, ಬಯಕೆ, ಭಯ, ಕೋಪಗಳನ್ನು ತೊರೆದು ಭವದ ಬಿಡುಗಡೆಯನ್ನೇ ಬಯಸುವ ಮುನಿ ಯಾವಾಗಲೂ ಮುಕ್ತನೇ.\\}
\slcol{\Index{ಭೋಕ್ತಾರಂ ಯಙ್ಞತಪಸಾಂ} ಸರ್ವಲೋಕಮಹೇಶ್ವರಮ್ ।\\
ಸುಹೃದಂ ಸರ್ವಭೂತಾನಾಂ ಜ್ಞಾತ್ವಾ ಮಾಂ ಶಾಂತಿಮೃಚ್ಛತಿ ॥ ೨೯ ॥}
\cquote{ನನ್ನನ್ನು ಯಜ್ಞದ ಮತ್ತು ತಪಸ್ಸಿನ ಫಲವನ್ನು ಉಣ್ಣುವವನೆಂದೂ ಎಲ್ಲಾ ಲೋಕಗಳಿಗೂ ಒಡೆಯನೆಂದು ಎಲ್ಲ ಪ್ರಾಣಿಗಳಿಗೂ ಗೆಳೆಯನೆಂದು ತಿಳಿದವನಿಗೆ ಮುಕ್ತಿ ಕೈಯಗಂಟು.\\}

\begin{center}
ಓಂ ತತ್ಸದಿತಿ ಶ್ರೀಮದ್ಭಗವದ್ಗೀತಾಸೂಪನಿಷತ್ಸು\\
ಬ್ರಹ್ಮವಿದ್ಯಾಯಾಂ ಯೋಗಶಾಸ್ತ್ರೇ ಶ್ರೀಕೃಷ್ಣಾರ್ಜುನಸಂವಾದೇ\\
ಕರ್ಮಸಂನ್ಯಾಸಯೋಗೋ ನಾಮ ಪಂಚಮೋಽಧ್ಯಾಯಃ 
\end{center}

\newpage
\begin{mananam}{\mananamfont \large ಮನನ ಶ್ಲೋಕ - ೨೬}
\footnotesize \mananamtext ನನ್ನ ಜೀವನದಲ್ಲಿ ಆಕಾಂಕ್ಷೆಗಳು ಮತ್ತು ಅದಕ್ಕೆ ಸಂಬಂಧಿಸಿದ ಉದ್ವೇಗಗಳು ಯಾವ ಪಾತ್ರ ವಹಿಸುತ್ತವೆ? ನನ್ನ ದೀರ್ಘಕಾಲದ ಒಳತಿಗೆ ನನ್ನ ಆಕಾಂಕ್ಷೆಗಳು ಆರೋಗ್ಯಕರ ಮತ್ತು ರಚನಾತ್ಮಕವಾಗಿವೆಯೇ? ನನ್ನ ಜೀವನದಲ್ಲಿನ ಪ್ರೇರಣೆಗಳು ಸ್ವಾರ್ಥದಲ್ಲಿ ಕೇಂದ್ರೀಕೃತವಾಗಿರುವ ಆಕಾಂಕ್ಷೆಗಳಿಂದ ಮಾತ್ರ ದೂಡಲ್ಪಟ್ಟಿವೆಯೇ? ಬೇರೆಯವರ ಹಿತದ ಇಂಗಿತದ ತಳಪಾಯದ ಅಡಿಯಲ್ಲಿ ನನ್ನ ಪ್ರೇರಣೆಗಳು ಪಾತ್ರ ವಹಿಸುವುದನ್ನು ಕಾಣಬಹುದೇ? ಬಲವಾದ ಮೋಹದಿಂದ ನನ್ನ ಆಕಾಂಕ್ಷೆಗಳಿಗೆ ಆದ ಅಡಚಣೆಗಳ ಗ್ರಹಿಕೆಯಿಂದ ಉಂಟಾದ ಕೋಪದ ಸಂದರ್ಭಗಳನ್ನು ಸ್ಮರಿಸಿಕೊಳ್ಳಬಲ್ಲೆನೇ? ಮನೋವಿಕಾರವಿಲ್ಲದ ಮತ್ತು ಹಂಬಲಿಕೆ ಇಲ್ಲದ ಜ್ಞಾನಿಯ ಸ್ಥಾನದ ಸಿದ್ದಿಯು ನನಗೆ ಏನು ಅರ್ಥ ಕೊಡುತ್ತದೆ?
\end{mananam}
\WritingHand\enspace\textbf{ಆತ್ಮ ವಿಮರ್ಶೆ}\\
\begin{inspiration}{\mananamfont \large ಸ್ಪೂರ್ತಿ}
\ssmall \mananamtext ಸಾಮಾನ್ಯ ಮನುಷ್ಯನ ದೃಷ್ಟಿಯಲ್ಲಿ ತಮ್ಮ ಮನಸ್ಸಿನಲ್ಲಿ ಹುಟ್ಟಿದ ತಮಗೆ ಬೇಕಾದ ಎಲ್ಲಾ ತರಹದ ಅವಶ್ಯಕತೆಗಳು, ಆಕಾಂಕ್ಷೆಗಳನ್ನು ಪಡೆಯುವುದು ಸ್ವಾತಂತ್ರದ ಕಲ್ಪನೆಯಾಗಿದೆ. ಆದರೆ ವ್ಯತ್ಯಾಸವೆಂದರೆ ಯೋಗಿಗಳು ಸ್ವಾತಂತ್ರ್ಯವನ್ನು ಇಂದ್ರಿಯ ತೃಪ್ತಿಗಾಗಲಿ ಅಥವಾ ಸ್ವಾರ್ಥದ ಚಲಾವಣೆಗಾಗಿ ಬರುವ ಯಾವ ಅನಪೇಕ್ಷಿತ ಆಕಾಂಕ್ಷೆಗಳನ್ನು ಬರಗೊಡದೇ ಬಹು ಎಚ್ಚರವಾದ ಮಾನಸಿಕ ಜಾಗೃತಿಯಿಂದ ಪಡೆಯುತ್ತಾರೆ. ಒಂದು ವೇಳೆ ಆಕಾಂಕ್ಷೆಗಳು ಬಿಡದೆ ಹುಟ್ಟಿಕೊಂಡರೆ ಬೇರೆಯವರ ಒಳತಿಗಾಗಿ ಕೂಡಲೇ ಅವನ್ನು ತಡೆಗಟ್ಟುತ್ತಾರೆ ಮತ್ತು ಮಾರ್ಗ ತೋರುತ್ತಾರೆ. ಆಕಾಂಕ್ಷೆಗಳು ಏನಿವೆ ಅವು, ನಮ್ಮ ದೈಹಿಕ ವ್ಯಕ್ತಿತ್ವದಿಂದ ನಮ್ಮನ್ನು ಕೆಳಮುಖವಾಗಿ ಸೆಳೆದುಕೊಳ್ಳುತ್ತವೆ. ನಿಜವಾದ ಸ್ವಾತಂತ್ರ್ಯವೆಂದರೆ, ಆತ್ಮದ ಸಹಜ ಸ್ಥಿತಿಯ  ಶ್ರೇಷ್ಠವಾದ  ಆತ್ಮಾನಂದಕ್ಕೆ   ಬದ್ಧನಾಗಿರುವುದು.
\end{inspiration}