%Term definitions
\mananamtext{
\begin{description}
   \item[ಧರ್ಮ] ಧರ್ಮವೆಂಬುವುದು ಸಾಮಾನ್ಯತಃ, ಒಬ್ಬ ವ್ಯಕ್ತಿಯ ಜೀವನದಲ್ಲಿ, ಸಮಾಜದಲ್ಲಿ ಹಾಗೂ ಪ್ರಕೃತಿಯ ನಿಯಮಕ್ಕೆ ಹೊಂದಿಕೊಂಡು ಹೋಗುವಂತಹ, ಅವನ ನೈತಿಕ ಬಾಧ್ಯತೆ ಕರ್ತವ್ಯಗಳ ಮಹತ್ವವನ್ನು ಸೂಚಿಸುತ್ತದೆ. ಗೀತೆ ಮತ್ತು ವೇದಗಳ ದೃಷ್ಟಿಕೋನದಿಂದ ನೋಡಿದರೆ, ಕೇವಲ ಲೌಕಿಕ ಲಾಭಗಳಿಗೆ ಮಾತ್ರವಲ್ಲದೇ, ಆಧ್ಯಾತ್ಮಿಕ ಜೀವನ ಮತ್ತು ಮೋಕ್ಷದ ಅನ್ವೇಷಣೆಗಾಗಿ ಧರ್ಮದ ಪರಿಪಾಲನೆ ಮಾಡುವುದೇ ಆಗಿದೆ. ಅಹಂ ಮತ್ತು ಅದರ ಇಷ್ಟಾನಿಷ್ಟಗಳನ್ನು ಮೆಟ್ಟಿ, ಈ ಸೃಷ್ಟಿಯ ದೈವೀಕ ಯೋಜನೆಗೆ ಅನುಗುಣವಾಗಿ ಜೀವಿಸುವುದೇ ಧರ್ಮದ ಕರೆಯಾಗಿದೆ.
   \item[ಭಗವಂತ] ಭಗವಾನ್ ಎಂದರೆ, ಆರು ದೈವೀಕ ಗುಣಗಳಾದ  ಪ್ರಭುತ್ವ, ಅಧಿಕಾರತ್ವ, ಖ್ಯಾತಿ, ಶಕ್ತಿ, ಜ್ಞಾನ ಮತ್ತು ತಟಸ್ಥತೆ ಹೊಂದಿರುವ ಒಂದು  ಪದವಿ. ಗೀತೆಯಲ್ಲಿ ಭಗವಾನ್ ಎಂದರೆ,  ‘ಪರಮಗುರು’;  ಸಾಕಾರ ರೂಪದಲ್ಲಿರುವ ಆ ಪರಮಾತ್ಮನೇ; ಗ್ರಹಣಾಕಾಂಕ್ಷಿಯಾದ ಸಾಧಕನಿಗೆ ಕರುಣೆಯಿಂದ ಮುಕ್ತಿಮಾರ್ಗಕ್ಕೆ ಬೇಕಾದ ವಿವೇಕವನ್ನು ದಯಪಾಲಿಸುವವನು ; ಶ್ರದ್ಧಾಳು ಭಕ್ತರಿಗೆ ಭಗವಂತನೆಂದರೆ ಪರಮ ಆಶ್ರಯದಾತನು ಹಾಗೂ, ಆತ್ಮಸಾಕ್ಷಾತ್ಕಾರಕ್ಕಾಗಿ ಮಾರ್ಗವನ್ನು ತೋರಿಸುವ ದೀವಿಗೆ.
   \item[ಸ್ಥಿತಪ್ರಜ್ಞ] ಯಾವ ವ್ಯಕ್ತಿಗೆ ಸ್ಥಿರವಾದ ವಿವೇಕ ಮತ್ತು ಬುದ್ಧಿಶಕ್ತಿ ಇರುವುದೋ, ಯಾರು ಸದಾ ಸಮಚಿತ್ತತೆಯಿಂದ ಇರುವನೋ, ಯಾರ ಇಂದ್ರಿಯಗಳು ಅವನ ಹಿಡಿತದಲ್ಲಿರುವುವೋ, ಯಾರು ಆಸೆ ಮತ್ತು ಭಯದಿಂದ ಮುಕ್ತನೋ ಹಾಗೂ, ಜೀವನದಲ್ಲಿ ಸಂತೋಷ ಬಂದಾಗ ತೀರಾ ಹಿಗ್ಗದೇ, ದುಃಖ ಬಂದಾಗ ಹತಾಶನಾಗದೇ ಇರುವನೋ,ಅಂಥವನು ‘ಸ್ಥಿತಪ್ರಜ್ಞ’ ; ಅಲ್ಲದೇ, ಯಾವನು ತನ್ನ ಆತ್ಮದಲ್ಲಿಯೇ ಸಂತೃಪ್ತನೋ, ಹಾಗೂ ಸಾಮಾನ್ಯತಃ, ಆತ್ಮಜ್ಞಾನವಿಲ್ಲದ ಒಬ್ಬನಿಗೆ ಕಾಡುವ ಅಪೂರ್ಣತೆಯ ಭಾವನೆ ಯಾವನಿಗೆ ಇಲ್ಲವೋ, ಅವನೇ ‘ಸ್ಥಿತಪ್ರಜ್ಞ’ ಎಂದೆನಿಸಿಕೊಳ್ಳುತ್ತಾನೆ.
\end{description}
}