\documentclass[12pt,twoside,openright,a5paper]{book}
\usepackage[margin=0.75in]{geometry}
\usepackage{layout}
\usepackage{pgffor}
\usepackage{tocloft}
\usepackage[table, dvipsnames]{xcolor}
\usepackage{fontspec}
\usepackage{fancybox}
\usepackage[skins]{tcolorbox}
\usepackage{fancyhdr}
\usepackage{setspace}
\usepackage[utf8]{inputenc}
\usepackage{emptypage}
\usepackage[vskip=0pt,rightmargin=0cm]{quoting}
\usepackage{sectsty}
\usepackage{polyglossia}
\usepackage{changepage}%
\usepackage{imakeidx}
\usepackage{setspace}
\usepackage{longtable,array}
\usepackage{tikz} % Package for drawing
\usepackage{tikzpagenodes}
\usepackage[toc,acronym]{glossaries}
\usepackage{fontawesome5}
\usepackage{marvosym}
\usepackage[totoc, font=footnotesize]{idxlayout}
\usepackage{eso-pic} % background image in titlepage
\usepackage{anyfontsize} % any font size




\newfontfamily\engfont[Script=Kannada]{Noto Serif Kannada}
\usepackage[hpos=24mm,fontsize=32pt, hanchor=l,anchor=lc,angle=90,color={[gray]{0.5}}, text={\engfont{DRAFT \ COPY}}]{draftwatermark}

\newskip\linepagesep \linepagesep 5pt\relax
\renewcommand\footrulewidth{0.5pt}
\def\vfootline{%
    \begingroup
        \color{blue}\rule[-990pt]{20pt}{1000pt}
    \endgroup}



\setmainfont[Script=Kannada, Renderer=HarfBuzz]{Noto Serif Kannada}
\setmainlanguage[numerals=kannada]{kannada}
\setotherlanguages{english}
\newfontfamily\kannadafont[Script=Kannada]{Noto Serif Kannada}
\newfontfamily\kannadafontsf[Script=Kannada]{Noto Serif Kannada}
\newfontfamily\kanBold[Script=Kannada]{Noto Serif Kannada Bold}
\newfontfamily\kanfont[Script=Kannada]{Noto Serif Kannada}
\newfontfamily\mananamfont[Script=Kannada]{NudiUni08k}
\newfontfamily\mananamtext[Script=Kannada]{AdishilaVedic}
\fancyhf{}
  \fancyfoot[RO]{\vfootline\hskip\linepagesep\thepage}
  \fancyfoot[LE]{\thepage\hskip\linepagesep\vfootline}
  \fancyhead[RO]{\small\kanfont ದಿನಾಂಕ ..../..../.....}
  \fancyhead[LO]{\small\kanfont ಗೀತಾ ಮನನಂ}
  \fancyhead[LE]{\small\kanfont ಗೀತಾ ಮನನಂ}
  \fancyhead[RE]{\small\kanfont ದಿನಾಂಕ ..../..../.....}
  \renewcommand\headrulewidth{1pt}
  \fancypagestyle{plain}{%
    \fancyhf{}
    \fancyfoot[RO]{\vfootline\hskip\linepagesep\thepage}
    \fancyfoot[LE]{\thepage\hskip\linepagesep\vfootline}
    \renewcommand\headrulewidth{0pt}
  }
\quotingsetup{font={itshape,footnotesize}}
\title{\Huge \kanfont \textbf{ಗೀತಾ ಮನನಂ}\\
{\normalsize Gita Mananam\\}
{\small(ದೈನಂದಿನ ಸ್ಪೂರ್ತಿ ಹಾಗೂ ಆತ್ಮಾವಲೋಕನಕ್ಕಾಗಿ)}}
\author{\large \kanfont ಸ್ವಾಮಿ ನಿರ್ಗುಣಾನಂದಗಿರಿ\\
{\normalsize Swamy Nirgunanandagiri}\\
\vspace{15mm}
{\normalsize Bhageeratha Publications}\\
{Rishikesh, India}
}


\addto\captionskannada{\renewcommand{\contentsname}{\color{blue}{ವಿಷಯ ಸೂಚಿ}}} 

% Reduce space between TOC
\setlength\cftparskip{-2pt}
\setlength\cftbeforechapskip{2pt}
\setcounter{secnumdepth}{-1}
\chapterfont{\color{blue}}  % sets colour of chapters
\setlength\parindent{0pt} % no-indent for entire file
\date{} % clear date
\makeindex
\indexsetup{othercode=\small}
%%Term definitions
\mananamtext{
\begin{description}
   \item[ಧರ್ಮ] ಧರ್ಮವೆಂಬುವುದು ಸಾಮಾನ್ಯತಃ, ಒಬ್ಬ ವ್ಯಕ್ತಿಯ ಜೀವನದಲ್ಲಿ, ಸಮಾಜದಲ್ಲಿ ಹಾಗೂ ಪ್ರಕೃತಿಯ ನಿಯಮಕ್ಕೆ ಹೊಂದಿಕೊಂಡು ಹೋಗುವಂತಹ, ಅವನ ನೈತಿಕ ಬಾಧ್ಯತೆ ಕರ್ತವ್ಯಗಳ ಮಹತ್ವವನ್ನು ಸೂಚಿಸುತ್ತದೆ. ಗೀತೆ ಮತ್ತು ವೇದಗಳ ದೃಷ್ಟಿಕೋನದಿಂದ ನೋಡಿದರೆ, ಕೇವಲ ಲೌಕಿಕ ಲಾಭಗಳಿಗೆ ಮಾತ್ರವಲ್ಲದೇ, ಆಧ್ಯಾತ್ಮಿಕ ಜೀವನ ಮತ್ತು ಮೋಕ್ಷದ ಅನ್ವೇಷಣೆಗಾಗಿ ಧರ್ಮದ ಪರಿಪಾಲನೆ ಮಾಡುವುದೇ ಆಗಿದೆ. ಅಹಂ ಮತ್ತು ಅದರ ಇಷ್ಟಾನಿಷ್ಟಗಳನ್ನು ಮೆಟ್ಟಿ, ಈ ಸೃಷ್ಟಿಯ ದೈವೀಕ ಯೋಜನೆಗೆ ಅನುಗುಣವಾಗಿ ಜೀವಿಸುವುದೇ ಧರ್ಮದ ಕರೆಯಾಗಿದೆ.
   \item[ಭಗವಂತ] ಯಾವ ವ್ಯಕ್ತಿಗೆ ಸ್ಥಿರವಾದ ವಿವೇಕ ಮತ್ತು ಬುದ್ಧಿಶಕ್ತಿ ಇರುವುದೋ, ಯಾರು ಸದಾ ಸಮಚಿತ್ತತೆಯಿಂದ ಇರುವನೋ, ಯಾರ ಇಂದ್ರಿಯಗಳು ಅವನ ಹಿಡಿತದಲ್ಲಿರುವುವೋ, ಯಾರು ಆಸೆ ಮತ್ತು ಭಯದಿಂದ ಮುಕ್ತನೋ ಹಾಗೂ, ಜೀವನದಲ್ಲಿ ಸಂತೋಷ ಬಂದಾಗ ತೀರಾ ಹಿಗ್ಗದೇ, ದುಃಖ ಬಂದಾಗ ಹತಾಶನಾಗದೇ ಇರುವನೋ,ಅಂಥವನು ‘ಸ್ಥಿತಪ್ರಜ್ಞ’ ; ಅಲ್ಲದೇ, ಯಾವನು ತನ್ನ ಆತ್ಮದಲ್ಲಿಯೇ ಸಂತೃಪ್ತನೋ, ಹಾಗೂ ಸಾಮಾನ್ಯತಃ, ಆತ್ಮಜ್ಞಾನವಿಲ್ಲದ ಒಬ್ಬನಿಗೆ ಕಾಡುವ ಅಪೂರ್ಣತೆಯ ಭಾವನೆ ಯಾವನಿಗೆ ಇಲ್ಲವೋ, ಅವನೇ ‘ಸ್ಥಿತಪ್ರಜ್ಞ’ ಎಂದೆನಿಸಿಕೊಳ್ಳುತ್ತಾನೆ.
   \item[ಸ್ಥಿತಪ್ರಜ್ಞ] ಯಾವ ವ್ಯಕ್ತಿಗೆ ಸ್ಥಿರವಾದ ವಿವೇಕ ಮತ್ತು ಬುದ್ಧಿಶಕ್ತಿ ಇರುವುದೋ, ಯಾರು ಸದಾ ಸಮಚಿತ್ತತೆಯಿಂದ ಇರುವನೋ, ಯಾರ ಇಂದ್ರಿಯಗಳು ಅವನ ಹಿಡಿತದಲ್ಲಿರುವುವೋ, ಯಾರು ಆಸೆ ಮತ್ತು ಭಯದಿಂದ ಮುಕ್ತನೋ ಹಾಗೂ, ಜೀವನದಲ್ಲಿ ಸಂತೋಷ ಬಂದಾಗ ತೀರಾ ಹಿಗ್ಗದೇ, ದುಃಖ ಬಂದಾಗ ಹತಾಶನಾಗದೇ ಇರುವನೋ,ಅಂಥವನು ‘ಸ್ಥಿತಪ್ರಜ್ಞ’ ; ಅಲ್ಲದೇ, ಯಾವನು ತನ್ನ ಆತ್ಮದಲ್ಲಿಯೇ ಸಂತೃಪ್ತನೋ, ಹಾಗೂ ಸಾಮಾನ್ಯತಃ, ಆತ್ಮಜ್ಞಾನವಿಲ್ಲದ ಒಬ್ಬನಿಗೆ ಕಾಡುವ ಅಪೂರ್ಣತೆಯ ಭಾವನೆ ಯಾವನಿಗೆ ಇಲ್ಲವೋ, ಅವನೇ ‘ಸ್ಥಿತಪ್ರಜ್ಞ’ ಎಂದೆನಿಸಿಕೊಳ್ಳುತ್ತಾನೆ.
\end{description}
}
%\makeglossaries


\newcommand\Linepage[1][0.40in]{% Change to suit
  \vbox to \dimexpr\textheight-\pagetotal-#1\relax {% Let TeX do the work...
    \leaders\hbox to \linewidth{\rule{0pt}{#1}\hrulefill}\vfil
  }%
}

\newtcolorbox{inspiration}[2][]{%
  floatplacement=b,float, 
  enhanced,colback=white,colframe=purple,coltitle=black,
  sharp corners,boxrule=1pt, width=\textwidth,
  fonttitle=\sffamily\scshape\color{teal},
  rightrule=3mm,
  fontupper=\itshape,
  fontlower=\tiny,
  attach boxed title to top left={yshift=-0.5\baselineskip-0.4pt,xshift=2mm},
  boxed title style={tile,size=minimal,left=0.5mm,right=0.5mm,
    colback=white,before upper=\strut},
  title=#2,#1
}
\newtcolorbox{mananam}[2][fontlower=\small]{%
  enhanced,colback=white,colframe=teal,coltitle=black,
  sharp corners,boxrule=1pt, width=\textwidth,
  fonttitle=\sffamily\scshape\color{purple},
  leftrule=3mm,
  fontupper=\itshape,
  fontlower=\tiny,
  attach boxed title to top left={yshift=-0.5\baselineskip-0.4pt,xshift=4mm},
  boxed title style={tile,size=minimal,left=0.5mm,right=0.5mm,
    colback=white,before upper=\strut},
  title=#2,#1
}

\newcommand\Index[1]{#1\index{#1}}

\newcommand\wave{%
  \begin{tikzpicture}[overlay,remember picture]
    \fill[red,line width=.5mm,draw=red] (current page.north west) circle (3mm);
    \def\max{20}
    \foreach \i in {4,...,\max} {
      \draw[red,line width=.5mm,draw opacity={(\max-\i)/(\max-4)}]
      (current page.north west) ++(\i mm,0)
      arc[start angle=0,end angle=-90,radius=\i mm];
    }
  \end{tikzpicture}%
}

\definecolor{pastelblue}{rgb}{0.68, 0.78, 0.81}

% commands
\definecolor{aurometalsaurus}{rgb}{0.43, 0.5, 0.5}
\newcommand{\slcol}[1]{{\color{MidnightBlue}{#1}}}
\newcommand{\cquote}[1]{\begin{quoting}{\color{black}{#1}}\end{quoting}}
%% Custom macro to convert Arabic numerals to Kannada numerals
\makeatletter
\newcommand{\converttoKannada}[1]{%
  \ifcase#1 ೦\or ೧\or ೨\or ೩\or ೪\or ೫\or ೬\or ೭\or ೮\or ೯\else\expandafter\converttoKannadaLoop\fi
}
\newcommand{\converttoKannadaLoop}[1]{%
  \expandafter\converttoKannada\expandafter{\the\numexpr#1/10}\converttoKannada{\the\numexpr#1\mod10}%
}
\makeatother

% Redefine page numbering globally to use Kannada numerals
\renewcommand{\thepage}{\converttoKannada{\arabic{page}}}

% Redefine index page numbers (the page numbers shown in the index itself)
\renewcommand{\index}{\@ifnextchar[{\@indexwithpage}{\@indexwithoutpage}}

\def\@indexwithpage[#1]#2{\indexentry{#2}{\converttoKannada{#1}}}
\def\@indexwithoutpage#1{\indexentry{#1}{\converttoKannada{\thepage}}}
\renewcommand{\cftchapfont}{\normalfont}
\renewcommand{\cftchappagefont}{\normalfont}
%column decoration
\setlength{\columnseprule}{1pt}
\def\columnseprulecolor{\color{blue}}

\begin{document}

\pagenumbering{roman}
\begin{titlepage}
	%\pagecolor{pastelblue}
	\AddToShipoutPictureBG*{%
    \AtPageLowerLeft{%
        \includegraphics[width=\paperwidth,height=\paperheight]{./images/frontcover.png}%
    }%
}
    \begin{center}
        \vspace*{0.5cm}
            
        {\Huge
        %\textbf{\color{white}\fontsize{50}{60}\selectfont ಗೀತಾ ಮನನಂ}
		}
        %\textbf{\\ \small \color{white}ದೈನಂದಿನ ಸ್ಪೂರ್ತಿ ಹಾಗೂ ಆತ್ಮಾವಲೋಕನಕ್ಕಾಗಿ}    
        \vspace{1.0cm}
            
        
		
            
        \vfill
            
        
            
        \vspace{0.1cm}
        {\color{white}    
		%\textbf{{\Large \mananamfont ಸ್ವಾಮಿ ನಿರ್ಗುಣಾನಂದ ಗಿರಿ}}\\
		%{\normalsize Swami Nirgunananda Giri\\Rishikesh, India}
		}
    \end{center}
\end{titlepage}
\nopagecolor% Use this to restore the color pages to white
\begin{titlepage}
	%\pagecolor{pastelblue}
	%\AddToShipoutPictureBG*{%
    %\AtPageLowerLeft{%
    %    \includegraphics[width=\paperwidth,height=\paperheight]{./images/krishna4.jpg}%
    %}%
%}
    \begin{center}
        \vspace*{0.5cm}
            
        {\Huge
        \textbf{\color{blue}\fontsize{50}{60}\selectfont ಗೀತಾ ಮನನಂ}}
        \textbf{\\ \small \color{black}ದೈನಂದಿನ ಸ್ಪೂರ್ತಿ ಹಾಗೂ ಆತ್ಮಾವಲೋಕನಕ್ಕಾಗಿ}    
        \vspace{3.0cm}\\
        \textbf{{\Large \color{blue}\mananamfont ಸ್ವಾಮಿ ನಿರ್ಗುಣಾನಂದ ಗಿರಿ}}\\    
        
		
            
        \vfill
            
        
            
        \vspace{0.1cm}
        {\color{black}    
		
		{{\large \color{blue}ಹೃಷೀಕೇಶ, ಉತ್ತರಾ ಖಂಡ}\\\normalsize ಭಾರತ}
        }
    \end{center}
\end{titlepage}
\nopagecolor% Use this to restore the color pages to white
%\maketitle

\thispagestyle{empty}
Copyright \textcopyright\ Swamy Nirgunanandagiri\\
\\
All rights reserved\\
\\
Edition - First, 2024\\
\vfill
Without written permission of the author it is forbidden to reproduce or adapt in any form or by any means any part of this  publication.\\
\\
Rishikesh, India\\
\newpage
\thispagestyle{empty}
\frontmatter

\doublespacing
\tableofcontents
\singlespacing
%\clearpage
%\newgeometry{margin=0pt} % Apply margin only for this page
%\thispagestyle{empty}
%\begin{center}
%\includegraphics[width=0.9\textwidth, height=\paperheight, keepaspectratio]{./images/ganapa.jpg}
%\end{center}
%\restoregeometry % Restore original geometry settings
%\newpage

\clearpage
\newgeometry{margin=0pt} % Apply margin only for this page
\thispagestyle{empty}
%\begin{figure}
%\centering
%\includegraphics[width=0.9\textwidth, height=\paperheight, keepaspectratio]{./images/ganapa.jpg}
%\includegraphics[width=\paperwidth, height=\paperheight]{./images/ganapa.jpg}
%\end{figure}
%\restoregeometry % Restore original geometry settings
%\newpage

\begin{figure}[h!]
    \centering
    \begin{overpic}[width=\paperwidth, height=\paperheight]{./images/001.jpg}
        \put(13,85){\color{white}\kanfont ಗಜಾನನಂ ಭೂತಗಣಾದಿ ಸೇವಿತಂ ಕಪಿತ್ಥ ಜಂಬೂಫಲಸಾರ ಭಕ್ಷಿತಮ್। }\put(10,82){\color{white}\kanfont ಉಮಾಸುತಂ ಶೋಕ ವಿನಾಶಕಾರಣಂ ನಮಾಮಿ ವಿಘ್ನೇಶ್ವರ ಪಾದಪಂಕಜಮ್॥ }
    \end{overpic}
    \caption{This is the standard figure caption below the image.}
    \label{fig:example}
\end{figure}

\restoregeometry

\thispagestyle{empty}
\thispagestyle{empty}
\pagestyle{fancy}


\chapter{\kanfont ಮುನ್ನುಡಿ}
\begin{center}
ದಿಶಂತು ಶಂ ಮೇ ಗುರುಪಾದಪಾಂಸವಃ ॥\\
\end{center}
\footnotesize \mananamtext{ಶ್ರಿ\!\char"0CD5ಮದ್ ಭಗವದ್ಗೀತೆಯು ಎಲ್ಲಾ ಧರ್ಮಗ್ರಂಥಗಳಲ್ಲಿ ಒಂದು ಅನನ್ಯ ಮತ್ತು ಸಾಟಿಯಿಲ್ಲದ ರತ್ನವಾಗಿದೆ.  ಇದು ಪರಿತ್ಯಾಗ ಮಾಡುವವರಿಗೆ ಮಾತ್ರವಲ್ಲದೆ ಲೌಕಿಕ ಜವಾಬ್ದಾರಿಗಳನ್ನು ಹೊತ್ತಿರುವವರಿಗೂ ಮಾರ್ಗದರ್ಶಿ ಬೆಳಕಾಗಿ ಕಾರ್ಯನಿರ್ವಹಿಸುತ್ತದೆ, ಆಧ್ಯಾತ್ಮಿಕದೊಂದಿಗೆ ಪ್ರಾಪಂಚಿಕತೆಯನ್ನು ಸಮತೋಲನಗೊಳಿಸಲು ಪ್ರಯತ್ನಿಸುತ್ತದೆ; ಇದು, ಅಜ್ಞಾನದಿಂದ ಅವರು (ಸಾಮಾನ್ಯ ಜನ) ತಮ್ಮನ್ನು ತಾವು,   ದೇಹ ಮತ್ತು ಮನಸ್ಸಿನ ವ್ಯವಹಾರಗಳ  ಜೊತೆ ಗುರುತಿಸಿಕೊಂಡಾಗ, ಅಂತಹ ವ್ಯವಹಾರಗಳ ಬಗ್ಗೆ ನಿಷ್ಪಕ್ಷಪಾತವಾಗಿರುವಂತೆ ಪ್ರತಿಪಾದಿಸುತ್ತದೆ.\\
ಜೀವನದಲ್ಲಿ ಒಬ್ಬರ ಕರ್ತವ್ಯಗಳನ್ನು ಸಾಧಿಸಲು, ಬಾಂಧವ್ಯದ ಅಥವಾ, ಮೋಹದ ಭಾವನೆ ಇರಬೇಕು ಎಂಬುದು ಒಂದು ತಪ್ಪು ಕಲ್ಪನೆ.  ಭಗವಾನ್ ಕೃಷ್ಣ, ಎಲ್ಲರಿಗಿಂತಲೂ  ದೊಡ್ಡ ಸಂಸಾರಿ ಹಾಗೂ, ಪರಿಪೂರ್ಣವಾದ ಮೋಹರಹಿತನಾದ ಅಸಂಸಾರಿ; ದಿವ್ಯವಾದ ಆನಂದದಲ್ಲಿ ನೆಲೆಗೊಂಡು, ಈ ಮೂರ್ತ, ಭೌತಿಕ ಜಗತ್ತಿನಲ್ಲಿ ಹೇಗೆ ಕಾರ್ಯನಿರ್ವಹಿಸಬೇಕು ಎಂಬುದನ್ನು ಅವನು ತನ್ನ ಕಾರ್ಯಗಳು, ಭಾವ ಮತ್ತು ಅವನು ಉಚ್ಛರಿಸುವ ಪ್ರತಿಯೊಂದೂ ಪರಮಪದದ  ಮೂಲಕ ಪ್ರದರ್ಶಿಸುತ್ತಾನೆ. ಆರಂಭದಲ್ಲಿ ‘ಸಂಘರ್ಷ ಮತ್ತು ಸವಾಲುಗಳಿಂದ ತುಂಬಿರುವ ಮಾರ್ಗ’ ಎಂದು ಕಂಡುಬoದರೂ ಸಹ, ಈ ಸ್ಥಿತಿಯನ್ನು ಸಾಧಿಸಲು (ದಿವ್ಯವಾದ ಆನಂದದಲ್ಲಿ ನೆಲೆಗೊಂಡು, ಈ ಮೂರ್ತ, ಭೌತಿಕ ಜಗತ್ತಿನಲ್ಲಿ  ಕಾರ್ಯನಿರ್ವಹಿಸುವುದು) ಸಮರ್ಥ ಶಿಕ್ಷಕರಿಂದ ಸರಿಯಾದ ಮಾರ್ಗದರ್ಶನದ ಅಗತ್ಯವಿದೆ. \\
ಈ ‘ನಿಪುಣ ಮಾರ್ಗದರ್ಶಿ ಕೈಪಿಡಿ’ಯಲ್ಲಿ, ಲೇಖಕರ ಆಳವಾದ ಒಳನೋಟದಿಂದ ಅಧ್ಯಾಯ 2 ರ 52-53 ಪದ್ಯಗಳಲ್ಲಿ ಸೂಚಿಸಿದಂತೆ: “ಶಿಕ್ಷಕರ ಮತ್ತು ಧರ್ಮಗ್ರಂಥಗಳ ಉದ್ದೇಶವು ನಮ್ಮನ್ನು ಭ್ರಮೆಯಿಂದ ಜಗ್ಗಿಸಿ, ಮುಕ್ತರನ್ನಾಗಿ ಮಾಡುವುದೇ ಆಗಿದೆ. ನಾವು ನಮ್ಮ ಲೌಕಿಕ ಚಿಂತನೆಯ ಮಾದರಿಗಳನ್ನು ಬಿಟ್ಟು, ಒಂದು ಉನ್ನತ ಸತ್ಯದಲ್ಲಿ (ಪಾರಮಾರ್ಥಿಕದಲ್ಲಿ) ಆಶ್ರಯ ಪಡೆಯಲು ಪ್ರಾರಂಭಿಸುತ್ತಿದ್ದಂತೆಯೇ, ಆಧ್ಯಾತ್ಮಿಕ ಪ್ರಯಾಣದ ಒಂದು ಭಾಗವಾದ ಗೊಂದಲಗಳು ಮತ್ತು ಸವಾಲುಗಳು ಏಳುತ್ತವೆ; ಆದರೆ, ನಾವು ಹೀಗೆ ಈ ಹಾದಿಯಲ್ಲಿ ಪ್ರಗತಿ ಹೊಂದುತ್ತಿದ್ದಂತೆ, ಸ್ಪಷ್ಟತೆ ಪಡೆಯಲು ಪ್ರಾರಂಭಿಸುತ್ತೇವೆ”.\\
ಈ ಪುಸ್ತಕವು ಲೇಖಕರ ಕ್ರಾಂತಿಕಾರಿ ಚಿಂತನೆಗಳ ಮೂಲಕ ಓದುಗರನ್ನು ದೇಹದಿಂದ, ಮನಸ್ಸಿಗೆ ಮತ್ತು ಮನಸ್ಸಿನಿಂದ ಪ್ರಜ್ಞೆಗೆ (ಚೈತನ್ಯಕ್ಕೆ), ಒಬ್ಬರ ಅಸ್ತಿತ್ವದ ಪದರಗಳನ್ನು ಭೇದಿಸುವಂತೆ ಮಾಡುತ್ತದೆ. \\
ಪ್ರತಿ ವಿಭಾಗದಲ್ಲಿ, ‘ಮನನಂ’ ಶೀರ್ಷಿಕೆಯಡಿಯಲ್ಲಿರುವ ಆತ್ಮಾವಲೋಕನದ ಪ್ರಶ್ನೆಗಳು, ಸ್ವಯಂ ಸಮಾಧಾನ ಮತ್ತು ಆತ್ಮವಂಚನೆಯಲ್ಲಿ (ಆತ್ಮ ಪ್ರವಂಚನ) ತೊಡಗಿರುವವರಿಗೆ ಯಾವುದೇ ವಿರಾಮ ನೀಡುವುದಿಲ್ಲ ಮತ್ತು ಪ್ರಾಮಾಣಿಕವಾದ ಸ್ವಯಂ ಮೌಲ್ಯಮಾಪನ ಮಾಡಲು ಅವರಿಗೆ ಸವಾಲು ಒಡ್ಡುತ್ತವೆ .  ಮತ್ತೊಂದೆಡೆ, ‘ಸ್ಫೂರ್ತಿ’ಯ ಅಡಿಯಲ್ಲಿರುವ ಪದಗಳು, ಪ್ರಯಾಸಕರ ಮತ್ತು ಗೊಂದಲಮಯ ಆಧ್ಯಾತ್ಮಿಕ ಮಾರ್ಗವನ್ನು ತುಲನಾತ್ಮಕವಾಗಿ ಸುಲಭಗೊಳಿಸಿ, ಸಕಾರಾತ್ಮಕತೆಯ ಒಂದು, ಅಕ್ಷಯವಾದ ಮೂಲವಾಗಿ ಕಾರ್ಯನಿರ್ವಹಿಸುತ್ತವೆ.\\
ಈ ಕೈಪಿಡಿಯು, ಆಕಾಂಕ್ಷಿಗಳಿಗೆ ಜೀವನದ ವಿವಿಧ ಸಮಸ್ಯೆಗಳಿಗೆ ಪರಿಹಾರವನ್ನು ಕಂಡುಕೊಳ್ಳಲು ಸಾಕಷ್ಟು ಸಮಾಧಾನಗಳನ್ನು  ನೀಡುತ್ತದೆ, ಆದರೆ ಬಾಹ್ಯವಾಗಿ ಅಲ್ಲ, ಆಂತರಿಕವಾಗಿ.\\
ಈ ಪುಸ್ತಕದ ವಿಷಯವು, ‘ಮಾನವ ಮನೋವಿಜ್ಞಾನ’ದ ಬಗ್ಗೆ ಲೇಖಕರ ಆಳವಾದ ತಿಳುವಳಿಕೆಯನ್ನು ತೋರಿಸುತ್ತದೆ, ಮೊದಲು ಸಕಾರಾತ್ಮಕ ಮಾನಸಿಕ ಸ್ಥಿತಿಯ ಕಡೆಗೆ ಮತ್ತು ನಂತರ ಅದನ್ನು ಮೀರಿ, ಆಂತರಿಕ ಶಾಶ್ವತವಾದ ಆತ್ಮದೆಡೆಗೆ, ಸರಳವಾಗಿ ಮುಂದುವರಿಯುತ್ತದೆ. \\
ಈ ಪುಸ್ತಕದ ಓದುಗರು ಈ ಪ್ರಯಾಣವನ್ನು ಪ್ರಾರಂಭಿಸಿದಾಗ, ಅವರು ಖಂಡಿತವಾಗಿಯೂ ಮಾನವ ಮನಸ್ಸಿನ ಕ್ಷೇತ್ರವನ್ನು ಮೀರುತ್ತಾರೆ ಮತ್ತು ಈ ಸ್ವರ್ಗೀಯ ಗೀತೆಯಾದ  ‘ಭಗವದ್ಗೀತೆ’ ಗಾಯಕನ ಕೃಪೆಯಿಂದ ‘ಅಧಿಷ್ಠಾನ ಚೈತನ್ಯಮ್’ (ಚೇತನಾತ್ಮಕದ ಅಂತಿಮ ಮೂಲತತ್ವ) ಅನ್ನು ತಲುಪುತ್ತಾರೆ. ಸಾಧಕರ ಅನುಕೂಲಕ್ಕಾಗಿ ತಮ್ಮ ಅಮೂಲ್ಯವಾದ ಆಲೋಚನೆಗಳನ್ನು ಲೇಖಿಸಿದ, ಸ್ವಾಮಿ ನಿರ್ಗುಣಾನಂದ ಗಿರಿ, ಇವರ ನಿಸ್ವಾರ್ಥ ಪ್ರಯತ್ನವನ್ನು,  ಸನಾತನ ಗುರುವಾದ, ಆ ಭಗವಂತ ಶ್ರೀಕೃಷ್ಣನು ಆಶೀರ್ವದಿಸಲಿ.\\\\
\begin{center}
ಮಂಗಳಂ ಸರ್ವಂ
\end{center}

{\kanBold ಸ್ವಾಮಿ ಸ್ವಾನಂದ ತೀರ್ಥ} \\
ಆಚಾರ್ಯ, ಕೈಲಾಸ್ ಆಶ್ರಮ\\
ಋಷಿಕೇಶ – ಉತ್ತರಖಂಡ\\
}
%\thispagestyle{empty}
\begin{onehalfspace}
\chapter{\kanfont ಪ್ರಸ್ತಾವನೆ}
\textit{\indent ನಾವೆಲ್ಲರೂ ಜೀವನದ ಹೋರಾಟಗಳನ್ನು ಎದುರಿಸಬೇಕು. ಕುರುಕ್ಷೇತ್ರ ಯುದ್ಧದಲ್ಲಿ ಶ್ರೀ ಕೃಷ್ಣ ಪರಮಾತ್ಮ ತನ್ನ ವೇದನೆಯುಕ್ತ ಶಿಷ್ಯ ಅರ್ಜುನನಿಗೆ ಅಧ್ಯಾತ್ಮಿಕ ಹಾದಿಯಲ್ಲಿ ಆಚರಣೆಗೆ ತರುವಂತ ಬಹಳ ಪವಿತ್ರವಾದ ಬೋಧನೆಗಳನ್ನು ಕೊಟ್ಟಿದ್ದಾನೆ. ಈ ಶ್ರೇಷ್ಠವಾದ ಉಪನಿಷತ್ತುಗಳ ಸತ್ವಗಳನೊಳಗೊಂಡ  ಬೋಧನೆಗಳನ್ನು ಪವಿತ್ರವಾದ ಭಗವದ್ಗೀತೆಯನ್ನು ಸಂತ ವೇದವ್ಯಾಸರು ನಮ್ಮ ಕೈಗೆ ನೀಡಿದ್ದಾರೆ.\\
\\
 ಅರ್ಜುನನು ಇದ್ದ ಪರಿಸ್ಥಿತಿಗೂ ನಾವು ಇರುವ ಪರಿಸ್ಥಿತಿ ಮತ್ತು ಸಂಘರ್ಷಗಳಿಗೂ ವ್ಯತ್ಯಾಸಗಳಿರಬಹುದು. ಸಾರ್ವರ್ತಿಕ ಉಪದೇಶಗಳು ಸತ್ಯಾನ್ವೇಷಣೆ ಮಾಡಲು ಬಯಸುವ ಎಲ್ಲರಿಗೂ ಆತ್ಮೋನತಿ  ಮತ್ತು ಅಧ್ಯಾತ್ಮಿಕ ಪ್ರಗತಿ ಸಾಧಿಸಲು ಬೇಕಾಗುವ ಮಾದರಿಯಾಗಿದೆ.\\
\\
 ಭಗವದ್ಗೀತೆಯ ಉಪದೇಶಗಳು ಕೇವಲ ಆಧ್ಯಾತ್ಮಿಕ ಅನ್ವೇಷಣೆ ಮಾಡುವವರಿಗೆ ಸಮರ್ಪಿತವಾದದ್ದು ಮಾತ್ರವಲ್ಲದೆ ಜೀವನದಲ್ಲಿ ಬೇಕಾಗುವ ಅತ್ಯಮೂಲ್ಯವಾದ ಕೈಪಿಡಿಯಾಗಿದೆ. ಯಾರು ಕೆಲಸದ ಸಮತೋಲನ ಮತ್ತು ಕೌಟುಂಬಿಕ ಜವಾಬ್ದಾರಿಗಳನ್ನು ಮಾನಸಿಕ ನೆಮ್ಮದಿ ಮತ್ತು ಒತ್ತಡ ರಹಿತವಾಗಿ ಮಾಡಲು ಬಯಸುತ್ತಾರೋ ಅವರಿಗೆ ಈ ಬೋಧನೆಗಳು ತುಂಬಾ ಮಹತ್ವದ್ದಾಗಿ ಕಾಣುತ್ತದೆ.\\
\\
 ಅನೇಕ ಗುರುಗಳು ಮತ್ತು ವಿದ್ವಾಂಸರು ಆಗಲೇ ಮಾಡಿರುವಂತೆ ಈ ದಿನಚರಿ ಪುಸ್ತಕ ಮತ್ತು ನಿಯತಕಾಲಿಕವು, ಗೀತೆಯ ಬೋಧನೆಗಳನ್ನು ತಿಳಿಸುವ ಪ್ರಯತ್ನ ಅಥವಾ ವ್ಯಾಖ್ಯಾನ ಕೊಡುವುದಾಗಿಲ್ಲ. ಈ ಗೀತಾ ಮನನವು ಬೋಧನೆಗಳ ಚಿಂತನೆ ಮಾಡುವುದು ಮತ್ತು ಅದನ್ನು ನಮ್ಮ ಸ್ವಂತದ್ದನ್ನಾಗಿ ಮಾಡಿಕೊಳ್ಳುವುದಾಗಿದೆ. ದೇವ ನಾಗರಿಯಲ್ಲಿರುವ `ಮನನ` ಎಂಬ ಪದವು ಆಗಲೇ ಕೇಳಿದ್ದನ್ನು ಅಥವಾ ಓದಿದ್ದನ್ನು ಚಿಂತನೆ ಮಾಡುವ ಕಾರ್ಯವಿಧಾನವನ್ನು ಅನ್ವಯಿಸುವುದಾಗಿದೆ.\\
\\
 ಈ ದಿನಚರಿ ಪುಸ್ತಕವನ್ನು ನೀವು ನಿಮ್ಮ ಮನಸ್ಸಿನ ಇಂಗಿತವನ್ನು ಸ್ವಾತಂತ್ರ್ಯವಾಗಿ ವ್ಯಕ್ತಪಡಿಸಲು ದಾರಿ ಮಾಡಿಕೊಳ್ಳುವುದಕ್ಕೆ ಮತ್ತು ನಿಮ್ಮ ಜೀವನದಲ್ಲಿ ಅಳವಡಿಸಿಕೊಳ್ಳಲು ಉದ್ದೇಶದಿಂದ ರೂಪಿಸಲಾಗಿದೆ. ಗೀತೆಯಲ್ಲಿರುವ ಶ್ಲೋಕಗಳಲ್ಲಿನ ಪ್ರಶ್ನೆಗಳು ಈ ಬೋಧನೆಗಳ ಸನ್ನಿವೇಶಕ್ಕೆ ಸಂಬಂಧಿಸಿದಂತೆ, ನಿಮ್ಮ ವೈಯಕ್ತಿಕ ಅರ್ಥಗಳನ್ನು ಹುಡುಕಲು ಮತ್ತು ಅದರಿಂದ ಜೀವನದ ಸಂದರ್ಭದೊಳಗೆ ನಿಶ್ಚಲವಾದ ಸ್ಪಷ್ಟನೆ ಹುಡುಕಲು ರೂಪಿಸಲಾಗಿದೆ.ಶ್ರೀ ಕೃಷ್ಣ ಪರಮಾತ್ಮನು ಹೇಗೆ ಅರ್ಜುನನಿಗೆ ನ್ಯಾಯವಾದ ಯುದ್ಧವನ್ನು ಮಾಡಲು ಪ್ರೇರೇಪಿಸಿದಂತೆ ನಿಮ್ಮ ಜೀವನದ ಕರ್ತವ್ಯಗಳನ್ನು ಈ ಗೀತೆ ಎಂಬ ಕೆಲಸದಿಂದ,ಆ ಭಗವಂತ ನಿಮಗೂ ಪ್ರೇರೇಪಿಸಲಿ ಎಂದು ನಂಬುತ್ತೇನೆ. ನಿಮ್ಮ ಕರ್ತವ್ಯಗಳನ್ನು ಕುಶಲತೆಯಿಂದ ಯಶಸ್ವಿಯಾಗಿ ನಿರ್ವಹಿಸಲು ನಿಮ್ಮ ಅಂತರಂಗದ ಶಾಂತಿಯನ್ನು ಉಪಯೋಗಿಸದೆ ವೈಯಕ್ತಿಕ ಪ್ರಗತಿ ಮತ್ತು ದೈವತ್ವಕ್ಕೆ ನಂಬಿಕೆಯಿಂದ ಇರುವ ನಿರಂತರ ಉದ್ದೇಶದಿಂದ ಕೂಡಿರುವುದೇ ಈ ದಿವ್ಯವಾದ ಗೀತೆ.}

\end{onehalfspace}
\mainmatter
%ಅಥ ಪ್ರಥಮೋऽಧ್ಯಾಯಃ ।\\
ಮೊಟ್ಟ ಮೊದಲನೆಯ ಶ್ಲೋಕವೇ ನಮಗೆ ಚಿಂತನೆ, ಮನನ ಪ್ರಾರಂಭಿಸಲು ಬೇಕಾಗುವ ಸೂಕ್ಷ್ಮವಾದ ಸಂದೇಶವನ್ನು ಕೊಡುತ್ತದೆ.\\
\slcol{ಧೃತರಾಷ್ಟ್ರ ಉವಾಚ ।\\
ಧರ್ಮಕ್ಷೇತ್ರೇ ಕುರುಕ್ಷೇತ್ರೇ ಸಮವೇತಾ ಯುಯುತ್ಸವಃ ।\\
ಮಾಮಕಾಃ ಪಾಂಡವಾಶ್ಚೈವ ಕಿಮಕುರ್ವತ ಸಂಜಯ ॥ 1 ॥}
\cquote{ಧೃತರಾಷ್ಟ್ರನು ಹೇಳಿದನು,\\
ಸಂಜಯನೇ, ಯುದ್ಧದ ಬಯಕೆಯಿಂದ ಧರ್ಮಭೂಮಿಯಾದ ಕುರುಕ್ಷೇತ್ರದಲ್ಲಿ ಕಲೆತ ನನ್ನ ಮಕ್ಕಳೂ ಪಾಂಡವರೂ ಏನು ಮಾಡಿದರು?\\}
\slcol{ಸಂಜಯ ಉವಾಚ ।\\
ದೃಷ್ಟ್ವಾ ತು ಪಾಂಡವಾನೀಕಂ ವ್ಯೂಢಂ ದುರ್ಯೋಧನಸ್ತದಾ ।\\
ಆಚಾರ್ಯಮುಪಸಂಗಮ್ಯ ರಾಜಾ ವಚನಮಬ್ರವೀತ್ ॥ 2 ॥}
\cquote{ಸಂಜಯನು ಹೇಳಿದನು,\\
ಪಾಂಡವರ ದಂಡು ಸಜ್ಜಾಗಿ ನಿಂತಿದ್ದುದನ್ನು ನೋಡಿದ ಅರಸನಾದ ದುರ್ಯೋಧನನು ಗುರುಗಳಾದ ದ್ರೋಣರ ಬಳಿಗೆ ಬಂದು ಹೀಗೆ ಹೇಳಿದನು. \\}
\begin{inspiration}{\kanfont ಸ್ಪೂರ್ತಿ}
ನಿನಗೆ ನೀನು ಸತ್ಯವಾಗಿರು ಮತ್ತು ನೀನು ಉನ್ನತಿಯತ್ತ ಬದಲಾಗುವೆ. ಜೀವನದಲ್ಲಿ ಜಾಣನಿಗೆ ಅವಶ್ಯಕವಾದುದು ಪಕ್ಷಪಾತ ರಹಿತ ಅವಲೋಕನ. ನಮ್ಮನ್ನು ನಾವು ಬದಲಾಯಿಸಿಕೊಳ್ಳಲು ಕೇವಲ ಬಯಕೆ ಇದ್ದರೆ ಮಾತ್ರ ಸಾಲದು. ಜ್ಞಾನಿಗಳ ಮಹತ್ವದ, ಉನ್ನತವಾದ ಬೋಧನೆಗಳಿಂದ ನಮ್ಮ ಯೋಚನೆಗಳು, ಮಾತುಗಳು ಮತ್ತು ಕೃತಿಗಳನ್ನು ತಹಬಂದಿಗೆ ತಂದು, ಪ್ರತಿದಿನವೂ ನಮ್ಮನ್ನು ನಾವು ಆತ್ಮ ವಿಮರ್ಶೆ ಮಾಡಿಕೊಳ್ಳಲೇಬೇಕು.
\end{inspiration}
\newpage
\begin{mananam}{\kanfont ಮನನ}
ನನ್ನ ಜೀವನದ ದೈನಂದಿನ ನಿತ್ಯಕರ್ಮದಲ್ಲಿ ಯಾವಾಗ ನನ್ನ ದೇಹವು, ಆಸೆ, ಕೋಪ, ಭಯ, ಮತ್ಸರ ಇತ್ಯಾದಿಗಳಲ್ಲಿ ಒಲವು ತೋರುವುದನ್ನು ಗುರುತಿಸಿತು, ಅವುಗಳನ್ನು ಸ್ವಾತಂತ್ರ್ಯವನ್ನು ಆಳವಾಗಿ ಪ್ರೇರೇಪಿಸುವ ನನ್ನನ್ನು ಪ್ರತಿಭಟಿಸುವಂತೆ ಮಾಡುವ ಮತ್ತು ಸನಾತನ ಗ್ರಂಥ ಮತ್ತು ಬೋಧಕರಿಂದ ಪಡೆದ ಜ್ಞಾನವನ್ನು ಯಾವ ಬಲವನ್ನು ಅನುಸರಿಸಿದೆ? ನನ್ನ ಹಂಬಲ ಮತ್ತು ಸಂಕಲ್ಪಗಳನ್ನು ತಳ್ಳಿಹಾಕುವ ನನ್ನ ದುರಭ್ಯಾಸಗಳು ಮತ್ತು ಅಪಾಯಕಾರಿ ನಡವಳಿಕೆಗಳಿಂದಾಗಿ ನನ್ನ ನಿತ್ಯ ಜೀವನದಲ್ಲಿ ಏನೇನು ಕಷ್ಟ ಪಡಬೇಕಾಯಿತು?
\end{mananam}
\Linepage
\newpage

\slcol{ಪಶ್ಯೈತಾಂ ಪಾಂಡುಪುತ್ರಾಣಾಮಾಚಾರ್ಯ ಮಹತೀಂ ಚಮೂಮ್ ।\\
ವ್ಯೂಢಾಂ ದ್ರುಪದಪುತ್ರೇಣ ತವ ಶಿಷ್ಯೇಣ ಧೀಮತಾ ॥ 3 ॥}
\cquote{ಗುರುಗಳೇ, ದೃಪದರಾಜನ ಮಗ ನಿಮ್ಮ ಶಿಷ್ಯ, ಬುದ್ಧಿಶಾಲಿಯಾದ ದೃಷ್ಟದ್ಯುಮ್ನ ಪಾಂಡವರ ಈ ದೊಡ್ಡ ದಂಡನ್ನು ಸಜ್ಜುಗೊಳಿಸಿರುವುದನ್ನು ನೋಡಿರಿ.\\}
\slcol{ಅತ್ರ ಶೂರಾ ಮಹೇಷ್ವಾಸಾ ಭೀಮಾರ್ಜುನಸಮಾ ಯುಧಿ ।\\
ಯುಯುಧಾನೋ ವಿರಾಟಶ್ಚ ದ್ರುಪದಶ್ಚ ಮಹಾರಥಃ ॥ 4 ॥}
\cquote{ಈ ದಂಡಿನಲ್ಲಿ ಹೋರಾಟದಲ್ಲಿ ಭೀಮಾರ್ಜುನರಿಗೆ ಸರಿ ಜೋಡಿಯಾದ ಶೂರರಾಗಿ ದೊಡ್ಡ ದೊಡ್ಡ ಬಿಲ್ಲುಗಳನ್ನು ಹಿಡಿದುಕೊಂಡು ಕಾದುವುದರಲ್ಲಿ ಕುಶಲರಾದ ಸಾತ್ಯಕಿ ವಿರಾಟರಿದ್ದಾರೆ. ಸಹಸ್ರ ಜನರೊಡನೆ ಏಕಾಂಗಿಯಾಗಿ ಹೋರಾಡಬಲ್ಲ ದ್ರುಪದನಿದ್ದಾನೆ.\\}
\slcol{ಧೃಷ್ಟಕೇತುಶ್ಚೇಕಿತಾನಃ ಕಾಶಿರಾಜಶ್ಚ ವೀರ್ಯವಾನ್ ।\\
ಪುರುಜಿತ್ಕುಂತಿಭೋಜಶ್ಚ ಶೈಬ್ಯಶ್ಚ ನರಪುಂಗವಃ ॥ 5 ॥}
\cquote{ದೃಷ್ಟಕೇತು, ಚೀಕಿತಾನ, ವೀರನಾದ ಕಾಶಿರಾಜ, ಮತ್ತು ಮನುಷ್ಯರಲ್ಲಿ ಶ್ರೇಷ್ಠನಾದ ಶೈಭ್ಯ ಇವರೆಲ್ಲ ಇದ್ದಾರೆ. \\} 
\slcol{ಯುಧಾಮನ್ಯುಶ್ಚ ವಿಕ್ರಾಂತ ಉತ್ತಮೌಜಾಶ್ಚ ವೀರ್ಯವಾನ್ ।\\
ಸೌಭದ್ರೋ ದ್ರೌಪದೇಯಾಶ್ಚ ಸರ್ವ ಏವ ಮಹಾರಥಾಃ ॥ 6 ॥}
\cquote{ಬಲಶಾಲಿಯಾದ ಯುಧಾಮನ್ಯು, ವೀರನಾದ ಉತ್ತಮೌಜ, ಸುಭದ್ರೆಯ ಮಗ ಅಭಿಮನ್ಯು ಮತ್ತು ದ್ರೌಪದಿಯ ಮಕ್ಕಳು ಇದ್ದಾರೆ. ಎಲ್ಲರೂ ಒಬ್ಬೊಬ್ಬರು ಹತ್ತು ಸಹಸ್ರ ಜನರೊಡನೆ ಹೋರಾಡಬಲ್ಲ ಮಹಾರುತರು. \\}
\slcol{ಅಸ್ಮಾಕಂ ತು ವಿಶಿಷ್ಟಾ ಯೇ ತಾನ್ನಿಬೋಧ ದ್ವಿಜೋತ್ತಮ ।\\
ನಾಯಕಾ ಮಮ ಸೈನ್ಯಸ್ಯ ಸಂಙ್ಞಾರ್ಥಂ ತಾನ್ಬ್ರವೀಮಿ ತೇ ॥ 7 ॥}
\cquote{ಬ್ರಾಹ್ಮಣ ಶ್ರೇಷ್ಠರೇ, ನಮ್ಮ ಕಡೆಯಲ್ಲಿರುವ ವೀರರನ್ನು ನೆನಪಿಗೆ ತಂದುಕೊಳ್ಳಿ. ತಮಗೆ ನೆನಪಾಗಲೆಂದು ಅವರ ಹೆಸರುಗಳನ್ನು ಹೇಳುತ್ತೇನೆ.\\} 
\slcol{ಭವಾನ್ಭೀಷ್ಮಶ್ಚ ಕರ್ಣಶ್ಚ ಕೃಪಶ್ಚ ಸಮಿತಿಂಜಯಃ ।\\
ಅಶ್ವತ್ಥಾಮಾ ವಿಕರ್ಣಶ್ಚ ಸೌಮದತ್ತಿಸ್ತಥೈವ ಚ ॥ 8 ॥}
\cquote{ತಾವು ಭೀಷ್ಮ ಕರ್ಣ ಜಯಶೀಲನಾದ ಕೃಪಾ, ಅಶ್ವತ್ಥಾಮ, ವಿಕರ್ಣ ಸೋಮದತ್ತನ ಮಗನಾದ ಭೂರಿಶ್ರವ ಮತ್ತು ಜಯದ್ರಥ. \\}
\slcol{ಅನ್ಯೇ ಚ ಬಹವಃ ಶೂರಾ ಮದರ್ಥೇ ತ್ಯಕ್ತಜೀವಿತಾಃ ।\\
ನಾನಾಶಸ್ತ್ರಪ್ರಹರಣಾಃ ಸರ್ವೇ ಯುದ್ಧವಿಶಾರದಾಃ ॥ 9 ॥}
\cquote{ಇನ್ನೂ ಅನೇಕ ಶೂರರು ನನಗಾಗಿ ಜೀವ ತೆರಲು ಸಿದ್ದರಾಗಿ ಇದ್ದಾರೆ. ಎಲ್ಲರೂ ಎಲ್ಲ ಬಗಯ ಆಯುಧಗಳನ್ನು ಉಪಯೋಗಿಸಬಲ್ಲವರು ಮತ್ತು ಯುದ್ಧದಲ್ಲಿ ಗಟ್ಟಿಗರು.\\}
\slcol{ಅಪರ್ಯಾಪ್ತಂ ತದಸ್ಮಾಕಂ ಬಲಂ ಭೀಷ್ಮಾಭಿರಕ್ಷಿತಮ್ ।\\
ಪರ್ಯಾಪ್ತಂ ತ್ವಿದಮೇತೇಷಾಂ ಬಲಂ ಭೀಮಾಭಿರಕ್ಷಿತಮ್ ॥ 10 ॥}
\cquote{ಭೀಷ್ಮರ ರಕ್ಷಣೆಗೆ ಒಳಪಟ್ಟಿರುವ ನಮ್ಮ ದೊಡ್ಡ ಆ ದಂಡು ಸಾಲದೇನೋ ಎನಿಸುತ್ತದೆ. ಭೀಮನ ರಕ್ಷಣೆಗೆ ಒಳಪಟ್ಟಿರುವ ಪಾಂಡವರ ಈ ಸೇನೆ ಸಾಕಷ್ಟು ಸಮರ್ಥವಾಗಿದೆ.\\}
\slcol{ಅಯನೇಷು ಚ ಸರ್ವೇಷು ಯಥಾಭಾಗಮವಸ್ಥಿತಾಃ ।\\
ಭೀಷ್ಮಮೇವಾಭಿರಕ್ಷಂತು ಭವಂತಃ ಸರ್ವ ಏವ ಹಿ ॥ 11 ॥}
\cquote{ನೀವೆಲ್ಲರೂ ದಂಡಿನ ಬೇರೆ ಬೇರೆ ಮಾರ್ಗಗಳಲ್ಲಿ ನಿಮ್ಮ ನಿಮ್ಮ ಪಾಲಿಗೆ ಬಂದ ಕಡೆ ಇದ್ದುಕೊಂಡು ಭೀಷ್ಮನನ್ನು ರಕ್ಷಿಸಿರಿ.\\}
\slcol{ತಸ್ಯ ಸಂಜನಯನ್ಹರ್ಷಂ ಕುರುವೃದ್ಧಃ ಪಿತಾಮಹಃ ।\\
ಸಿಂಹನಾದಂ ವಿನದ್ಯೋಚ್ಚೈಃ ಶಂಖಂ ದಧ್ಮೌ ಪ್ರತಾಪವಾನ್ ॥ 12 ॥}
\cquote{ಹೀಗೆಂದು ಹೇಳಿದ ದುರ್ಯೋಧನನಿಗೆ ಹರ್ಷ ಉಂಟಾಗುವಂತೆ ಆಗ ಕುರುವಂಶದ ಹಿರಿಯ ಕೌರವರ ಅಜ್ಜ, ಪರಾಕ್ರಮಶಾಲಿ ಭೀಷ್ಮನು ಗಟ್ಟಿಯಾಗಿ ಸಿಂಹನಾದ ಮಾಡಿ ಶಂಖವನ್ನು ಊದಿದನು.\\}
\slcol{ತತಃ ಶಂಖಾಶ್ಚ ಭೇರ್ಯಶ್ಚ ಪಣವಾನಕಗೋಮುಖಾಃ ।\\
ಸಹಸೈವಾಭ್ಯಹನ್ಯಂತ ಸ ಶಬ್ದಸ್ತುಮುಲೋऽಭವತ್ ॥ 13 ॥}
\cquote{ಆಮೇಲೆ ಒಮ್ಮೆಲೆ ಶಂಖಗಳು, ಭೇರಿಗಳು, ಮೃದಂಗಗಳು, ನಗಾಡಿಗಳು, ರಣ ಸಿಂಹಗಳು ಒಳಗಿದವು. ಆ ಗದ್ದಲವು ಎಲ್ಲೆಲ್ಲಿಯೂ ತುಂಬಿತು.\\}
\slcol{ತತಃ ಶ್ವೇತೈರ್ಹಯೈರ್ಯುಕ್ತೇ ಮಹತಿ ಸ್ಯಂದನೇ ಸ್ಥಿತೌ ।\\
ಮಾಧವಃ ಪಾಂಡವಶ್ಚೈವ ದಿವ್ಯೌ ಶಂಖೌ ಪ್ರದಘ್ಮತುಃ ॥ 14 ॥}
\cquote{ಆಮೇಲೆ ಬಿಳಿ ಕುದುರೆಯನ್ನು ಹೂಡಿದ ದೊಡ್ಡ ತೇರಿನ ಮೇಲೆ ಕುಳಿತಿದ್ದ ಕೃಷ್ಣನೂ ಅರ್ಜುನನೂ ಹೆಸರುವಾಸಿಯಾದ ದಿವ್ಯವಾದ ತಮ್ಮ ಶಂಖಗಳನ್ನು ಊದಿದರು.\\}
\slcol{ಪಾಂಚಜನ್ಯಂ ಹೃಷೀಕೇಶೋ ದೇವದತ್ತಂ ಧನಂಜಯಃ ।\\
ಪೌಂಡ್ರಂ ದಧ್ಮೌ ಮಹಾಶಂಖಂ ಭೀಮಕರ್ಮಾ ವೃಕೋದರಃ ॥ 15 ॥}
\cquote{ಕೃಷ್ಣನು ಪಾಂಚಜನ್ಯವನ್ನೂ ಅರ್ಜುನನ್ನು ದೇವದತ್ತವನ್ನೂ, ಶತ್ರುಗಳನ್ನು ಎದೆಗೂಡಿಸುವ ಭೀಮನು ಪೌಂಡ್ರವೆಂಬ ದೊಡ್ಡ ಶಂಖವನ್ನು ಓದಿದನು.\\}
\slcol{ಅನಂತವಿಜಯಂ ರಾಜಾ ಕುಂತೀಪುತ್ರೋ ಯುಧಿಷ್ಠಿರಃ ।\\
ನಕುಲಃ ಸಹದೇವಶ್ಚ ಸುಘೋಷಮಣಿಪುಷ್ಪಕೌ ॥ 16 ॥}
\cquote{ಕುಂತಿಯ ಹಿರಿಯ ಮಗ, ಅರಸನಾದ ಧರ್ಮರಾಯನು ಅನಂತ ವಿಜಯವನ್ನೂ ನಕುಲನೂ ಸುಘೋಷವನ್ನೂ ಸಹದೇವನು ಮಣಿಪುಷ್ಪಕವನ್ನೂ ಊದಿದರು. \\}
\slcol{ಕಾಶ್ಯಶ್ಚ ಪರಮೇಷ್ವಾಸಃ ಶಿಖಂಡೀ ಚ ಮಹಾರಥಃ ।\\
ಧೃಷ್ಟದ್ಯುಮ್ನೋ ವಿರಾಟಶ್ಚ ಸಾತ್ಯಕಿಶ್ಚಾಪರಾಜಿತಃ ॥ 17 ॥\\
ದ್ರುಪದೋ ದ್ರೌಪದೇಯಾಶ್ಚ ಸರ್ವಶಃ ಪೃಥಿವೀಪತೇ ।\\
ಸೌಭದ್ರಶ್ಚ ಮಹಾಬಾಹುಃ ಶಂಖಾಂದಧ್ಮುಃ ಪೃಥಕ್ಪೃಥಕ್ ॥ 18 ॥}
\cquote{ಓ ಧೃತರಾಷ್ಟ್ರ ಕೇಳು, ಹಿರಿಯ ಬಿಲ್ಲೋಜ ಕಾಶಿರಾಜ, ಮಹಾರಥನಾದ ಶಿಖಂಡಿ, ಧೃಷ್ಟದ್ಯುಮ್ನ,  ವಿರಾಟ, ಸೋಲರಿಯದ ಸಾತ್ಯಕಿ, ದ್ರುಪದ, ದ್ರೌಪದಿಯ ಮಕ್ಕಳು, ಮಹಾಬಾಹುವಾದ ಅಭಿಮನ್ಯು ಹೀಗೆ ಎಲ್ಲರೂ ತಮ್ಮ ತಮ್ಮ ಶಂಖಗಳನ್ನು ಊದಿದರು.\\}
\slcol{ಸ ಘೋಷೋ ಧಾರ್ತರಾಷ್ಟ್ರಾಣಾಂ ಹೃದಯಾನಿ ವ್ಯದಾರಯತ್ ।\\
ನಭಶ್ಚ ಪೃಥಿವೀಂ ಚೈವ ತುಮುಲೋ ವ್ಯನುನಾದಯನ್ ॥ 19 ॥}
\cquote{ಆ ಗದ್ದಲವು ಭೂಮಿಯಲ್ಲಿಯೂ ಆಕಾಶದಲ್ಲಿಯೂ ತುಂಬಿ ಪ್ರತಿಧ್ವನಿಯನ್ನು ಹಬ್ಬಿಸಿ ಕೌರವರ ಎದೆ ಬಿರಿಯುವಂತೆ ಮಾಡಿತು.\\}
\slcol{ಅಥ ವ್ಯವಸ್ಥಿತಾಂದೃಷ್ಟ್ವಾ ಧಾರ್ತರಾಷ್ಟ್ರಾನ್ಕಪಿಧ್ವಜಃ ।\\
ಪ್ರವೃತ್ತೇ ಶಸ್ತ್ರಸಂಪಾತೇ ಧನುರುದ್ಯಮ್ಯ ಪಾಂಡವಃ ॥ 20 ॥\\
ಹೃಷೀಕೇಶಂ ತದಾ ವಾಕ್ಯಮಿದಮಾಹ ಮಹೀಪತೇ ।}
\cquote{ಓ ಧೃತರಾಷ್ಟ್ರ, ಸಜ್ಜಾಗಿ ಎದುರಿಗೆ ನಿಂತಿರುವ ಕೌರವರನ್ನು ನೋಡಿ ಕಪಿಧ್ವಜನಾದ ಅರ್ಜುನನು ಹೊಡೆದಾಟಕ್ಕೆ ಮೊದಲು ಮಾಡಬೇಕಾದ ಆ ಸಮಯದಲ್ಲಿ ಗಾಂಡೀವವನ್ನು ಕೈಗೆ ತೆಗೆದುಕೊಂಡು ಕೃಷ್ಣನನ್ನು ಕುರಿತು ಈ ಮಾತನ್ನು ಹೇಳಿದನು.\\}
\slcol{ಅರ್ಜುನ ಉವಾಚ ।\\
ಸೇನಯೋರುಭಯೋರ್ಮಧ್ಯೇ ರಥಂ ಸ್ಥಾಪಯ ಮೇऽಚ್ಯುತ ॥ 21 ॥}
\cquote{ಅರ್ಜುನನ್ನು ಹೇಳಿದನು, ಕೃಷ್ಣ, ಎರಡು ದಂಡುಗಳ ನಡುವೆ ನನ್ನ ರಥವನ್ನು ನಿಲ್ಲಿಸು.\\}
\slcol{ಯಾವದೇತಾನ್ನಿರೀಕ್ಷೇऽಹಂ ಯೋದ್ಧುಕಾಮಾನವಸ್ಥಿತಾನ್ ।\\
ಕೈರ್ಮಯಾ ಸಹ ಯೋದ್ಧವ್ಯಮಸ್ಮಿನ್ರಣಸಮುದ್ಯಮೇ ॥ 22 ॥}
\cquote{ಕಾದಬೇಕೆಂದು ನಿಂತಿರುವವರನ್ನು, ಈ ಯುದ್ಧದಲ್ಲಿ ನಾನು ಯಾರೊಡನೆ ಕಾದಬೇಕಾಗಿದೆ ಎಂಬುದನ್ನು ಒಮ್ಮೆ ನೋಡುತ್ತೇನೆ.\\}
\slcol{ಯೋತ್ಸ್ಯಮಾನಾನವೇಕ್ಷೇऽಹಂ ಯ ಏತೇऽತ್ರ ಸಮಾಗತಾಃ ।\\
ಧಾರ್ತರಾಷ್ಟ್ರಸ್ಯ ದುರ್ಬುದ್ಧೇರ್ಯುದ್ಧೇ ಪ್ರಿಯಚಿಕೀರ್ಷವಃ ॥ 23 ॥}
\cquote{ದುರ್ಬುದ್ಧಿಯ ದುರ್ಯೋಧನನಿಗೆ ಈ ಯುದ್ಧದಲ್ಲಿ ನೆರವಾಗಬೇಕೆಂದು ಕಾದುವುದಕ್ಕಾಗಿ ಯಾರು ಯಾರು ಇಲ್ಲಿಗೆ ಬಂದಿರುತ್ತಾರೆ ಎಂಬುದನ್ನು ನಾನೊಮ್ಮೆ ನೋಡುತ್ತೇನೆ.\\}
\slcol{ಸಂಜಯ ಉವಾಚ ।\\
ಏವಮುಕ್ತೋ ಹೃಷೀಕೇಶೋ ಗುಡಾಕೇಶೇನ ಭಾರತ ।\\
ಸೇನಯೋರುಭಯೋರ್ಮಧ್ಯೇ ಸ್ಥಾಪಯಿತ್ವಾ ರಥೋತ್ತಮಮ್ ॥ 24 ॥\\
ಭೀಷ್ಮದ್ರೋಣಪ್ರಮುಖತಃ ಸರ್ವೇಷಾಂ ಚ ಮಹೀಕ್ಷಿತಾಮ್ ।\\
ಉವಾಚ ಪಾರ್ಥ ಪಶ್ಯೈತಾನ್ಸಮವೇತಾನ್ಕುರೂನಿತಿ ॥ 25 ॥}
\cquote{ಸಂಜಯನು ಹೇಳಿದನು,\\
ಧೃತರಾಷ್ಟ್ರನೇ, ಅರ್ಜುನನು ಹೀಗೆ ಹೇಳಿದಾಗ ಕೃಷ್ಣನು ಭೀಷ್ಮ ದ್ರೋಣರ ಮತ್ತು ಎಲ್ಲಾ ಅರಸರ ಎದುರಿಗೆ ಎರಡು ದಂಡುಗಳ ನಡುವೆ ರಥವನ್ನು ನಿಲ್ಲಿಸಿ ‘ಅರ್ಜುನನೇ ಇಲ್ಲಿ ನೆರೆದಿರುವರನ್ನು ನೋಡು’ ಎಂದು ಹೇಳಿದನು.\\}
\slcol{ತತ್ರಾಪಶ್ಯತ್ಸ್ಥಿತಾನ್ಪಾರ್ಥಃ ಪಿತೂನಥ ಪಿತಾಮಹಾನ್ ।\\
ಆಚಾರ್ಯಾನ್ಮಾತುಲಾನ್ಭ್ರಾತೂನ್ಪುತ್ರಾನ್ಪೌತ್ರಾನ್ಸಖೀಂಸ್ತಥಾ ॥ 26 ॥}
\cquote{ಅರ್ಜುನು ಅಲ್ಲಿ ನಿಂತಿರುವ ಪಿತೃತುಲ್ಯರು, ಅಜ್ಜಂದಿರು, ಗುರುಗಳು, ಸೋದರ ಮಾವಂದಿರು, ಅಣ್ಣತಮ್ಮಂದಿರು, ಮಕ್ಕಳು, ಮೊಮ್ಮಕ್ಕಳು, ಜೊತೆಗಾರರು, ಮಾವಂದಿರು, ಸ್ನೇಹಿತರು- ಹೀಗೆ ಎಲ್ಲ ಬಗೆಯ ಬಂಧುಗಳನ್ನು ಎರಡು ಕಡೆಯ ದಂಡಿನಲ್ಲಿ ಕಂಡನು.\\}
\slcol{ಶ್ವಶುರಾನ್ಸುಹೃದಶ್ಚೈವ ಸೇನಯೋರುಭಯೋರಪಿ ।\\
ತಾನ್ಸಮೀಕ್ಷ್ಯ ಸ ಕೌಂತೇಯಃ ಸರ್ವಾನ್ಬಂಧೂನವಸ್ಥಿತಾನ್ ॥ 27 ॥}
\cquote{ಹೀಗೆ ಅಲ್ಲಿ ನೆರೆದಿರುವ ಬಂಧುಗಳನ್ನೆಲ್ಲ ನೋಡಿ ಅರ್ಜುನನು ತುಂಬಾ ಕನಿಕರಗೊಂಡು ವಿಷಾದದಿಂದ ಈ ಮಾತನ್ನು ಹೇಳಿದನು.\\}
\slcol{ಕೃಪಯಾ ಪರಯಾವಿಷ್ಟೋ ವಿಷೀದನ್ನಿದಮಬ್ರವೀತ್ ।\\
ಅರ್ಜುನ ಉವಾಚ ।\\
ದೃಷ್ಟ್ವೇಮಂ ಸ್ವಜನಂ ಕೃಷ್ಣ ಯುಯುತ್ಸುಂ ಸಮುಪಸ್ಥಿತಮ್ ॥ 28 ॥\\
ಸೀದಂತಿ ಮಮ ಗಾತ್ರಾಣಿ ಮುಖಂ ಚ ಪರಿಶುಷ್ಯತಿ ।\\
ವೇಪಥುಶ್ಚ ಶರೀರೇ ಮೇ ರೋಮಹರ್ಷಶ್ಚ ಜಾಯತೇ ॥ 29 ॥}
\cquote{ಅರ್ಜುನನು ಹೇಳಿದನು,\\
ಕೃಷ್ಣ, ಕಾದುವುದಕೆಂದು ನೆರೆದಿರುವ ಈ ನನ್ನವರನ್ನು ನೋಡಿ ನನ್ನ ಅವಯವಗಳು ಸೊರುಗುತ್ತಿವೆ. ಬಾಯಿ ಒಣಗುತ್ತಿದೆ. ನನ್ನ ಮೈಯಲ್ಲಿ ನಡುಕ ಮೂಡಿ ರೋಮ ನಿಗುರಿ ನಿಂತಿದೆ.\\}
\begin{inspiration}{\kanfont ಸ್ಪೂರ್ತಿ}
ನಿಮ್ಮ ಯೋಚನೆಗಳ ಬಗ್ಗೆ ಎಚ್ಚರ ವಹಿಸಬೇಕು.ನಿಮ್ಮ ಮಾನಸಿಕ ಸ್ಥಿತಿ ನಿಮ್ಮ ದೇಹದ ಮೇಲೆ ಪರಿಣಾಮ ಬೀರುತ್ತದೆ. ಪ್ರತಿನಿತ್ಯದ ಒತ್ತಡದಿಂದ ಮನಸ್ಸನ್ನು ಸ್ವಾತಂತ್ರ್ಯಗೊಳಿಸಲು ಕೆಲವು ಸರಳ ಯೋಗದ ಮತ್ತು ಉಸಿರಾಟದ ಪ್ರಕ್ರಿಯೆಗಳು ಸಹಕಾರಿಯಾಗುತ್ತವೆ.\\
\end{inspiration}
\newpage
\begin{mananam}{\kanfont ಮನನ - \textenglish{28,29,30}}
ನನ್ನ ಜೀವನದಲ್ಲಿ ಎದುರಿಸಿದ ಭಯಂಕರವಾದ ಉದ್ವೇಗಗಳನ್ನು ಎದುರಿಸಬೇಕಾದ ಸಂದರ್ಭದಲ್ಲಿ ಪರ್ಯಾಲೋಚಿಸುತ್ತೇವೆ. ಮತ್ತು ಹೊರಗಿನ ಸನ್ನಿವೇಶಗಳಿಂದಾಗಿ ನನ್ನೊಳಗೆ ಮಿತಿಮೀರಿದವು ಇರುವಂತಾಯಿತು.ಜೀವನದ ಅಂತಹ ಸಂದರ್ಭಗಳಲ್ಲಿ ನನ್ನ ಮಾನಸಿಕ ಭಯಗಳಿಂದಾಗಿ ನನ್ನ ದೈಹಿಕ ಸ್ಥಿತಿ ಕುಂಟಿತ ವಾಯಿತೆಂಬುದನ್ನು ನಾನು ಅರಿತಿದ್ದೇನೆಯೇ? ನಾನು ನನ್ನ ಜೀವನದಲ್ಲಿನ ಉದ್ವೇಗ ಮತ್ತು ಭಯವನ್ನು ಹೇಗೆ ಎದುರಿಸಲಿ?\\
\end{mananam}
\Linepage
\newpage
\slcol{ಗಾಂಡೀವಂ ಸ್ರಂಸತೇ ಹಸ್ತಾತ್ತ್ವಕ್ಚೈವ ಪರಿದಹ್ಯತೇ ।\\
ನ ಚ ಶಕ್ನೋಮ್ಯವಸ್ಥಾತುಂ ಭ್ರಮತೀವ ಚ ಮೇ ಮನಃ ॥ 30 ॥}
\cquote{ಕೈಯಿಂದ ಗಾಂಡೀವ ಧನುಸ್ಸು ಕುಸಿಯುತ್ತಿದೆ. ಚರ್ಮವು ಸುಡುತ್ತಿದೆ. ನನಗೆ ನಿಲ್ಲುವುದಕ್ಕೂ ಆಗುವುದಿಲ್ಲ. ನನ್ನ ಮನಸ್ಸು ತಳಮಳಗೊಂಡಿದೆ.\\}
\slcol{ನಿಮಿತ್ತಾನಿ ಚ ಪಶ್ಯಾಮಿ ವಿಪರೀತಾನಿ ಕೇಶವ ।\\
ನ ಚ ಶ್ರೇಯೋऽನುಪಶ್ಯಾಮಿ ಹತ್ವಾ ಸ್ವಜನಮಾಹವೇ ॥ 31 ॥}
\cquote{ಕೃಷ್ಣ, ಕೆಟ್ಟ ಅಪಶಕುನಗಳನ್ನು ಕಾಣುತ್ತಿದ್ದೇನೆ. ಯುದ್ಧದಲ್ಲಿ ನನ್ನವರನ್ನು ಕೊಂದರೆ ಒಳ್ಳೆಯದಾದೀತೆಂದು ನನಗೆ ಅನ್ನಿಸುವುದಿಲ್ಲ.\\}
\slcol{ನ ಕಾಂಕ್ಷೇ ವಿಜಯಂ ಕೃಷ್ಣ ನ ಚ ರಾಜ್ಯಂ ಸುಖಾನಿ ಚ ।\\
ಕಿಂ ನೋ ರಾಜ್ಯೇನ ಗೋವಿಂದ ಕಿಂ ಭೋಗೈರ್ಜೀವಿತೇನ ವಾ ॥ 32 ॥}
\cquote{ಕೃಷ್ಣ, ನನಗೆ ಗೆಲ್ಲುವ ಬಯಕೆ ಇಲ್ಲ. ನನಗೆ ರಾಜ್ಯವು ಬೇಡ, ಸುಖಗಳೂ ಬೇಡ. ಗೋವಿಂದ, ಇಂಥ ರಾಜ್ಯದಿಂದಾಗಲಿ ಭೋಗದಿಂದಾಗಲಿ ಬದುಕಿನಿಂದಲೆ ಆಗಲಿ ಏನು ಪ್ರಯೋಜನ?\\}
\slcol{ಯೇಷಾಮರ್ಥೇ ಕಾಂಕ್ಷಿತಂ ನೋ ರಾಜ್ಯಂ ಭೋಗಾಃ ಸುಖಾನಿ ಚ ।\\
ತ ಇಮೇऽವಸ್ಥಿತಾ ಯುದ್ಧೇ ಪ್ರಾಣಾಂಸ್ತ್ಯಕ್ತ್ವಾ ಧನಾನಿ ಚ ॥ 33 ॥}
\cquote{ಯಾರಿಗಾಗಿ ನಾವು ರಾಜ್ಯವನ್ನೂ ಭೋಗಗಳನ್ನೂ ಸುಖಗಳನ್ನೂ ಬಯಸಿದೆವೋ, ಆ ಜನರೆಲ್ಲ ಜೀವದಾಸೆಯನ್ನೂ ಸಿರಿಯನ್ನೂ ತೊರೆದು ಇಲ್ಲಿ ಕಾದುವುದಕ್ಕೆ ನಿಂತಿದ್ದಾರೆ.\\}
\slcol{ಆಚಾರ್ಯಾಃ ಪಿತರಃ ಪುತ್ರಾಸ್ತಥೈವ ಚ ಪಿತಾಮಹಾಃ ।\\
ಮಾತುಲಾಃ ಶ್ವಶುರಾಃ ಪೌತ್ರಾಃ ಶ್ಯಾಲಾಃ ಸಂಬಂಧಿನಸ್ತಥಾ ॥ 34 ॥}
\cquote{ಗುರುಗಳು, ಪಿತೃತುಲ್ಯಯರು, ಮಕ್ಕಳು, ಅಜ್ಜಂದಿರು, ಸೋದರ ಮಾವಂದಿರು, ಮಾವಂದಿರು, ಮೊಮ್ಮಕ್ಕಳು, ಭಾವ ಮೈದುನರು, ಅದರಂತೆ ಬೇರೆ ಬೇರೆ ಸಂಬಂಧವುಳ್ಳವರು ಇಲ್ಲಿ ಎದುರು ನಿಂತಿದ್ದಾರೆ.\\}
\slcol{ಏತಾನ್ನ ಹಂತುಮಿಚ್ಛಾಮಿ ಘ್ನತೋऽಪಿ ಮಧುಸೂದನ ।\\
ಅಪಿ ತ್ರೈಲೋಕ್ಯರಾಜ್ಯಸ್ಯ ಹೇತೋಃ ಕಿಂ ನು ಮಹೀಕೃತೇ ॥ 35 ॥}
\cquote{ಕೃಷ್ಣ, ಅವರಿಂದ ನಾನು ಸತ್ತರೂ ಸರಿ. ಮೂರು ಲೋಕಗಳೇ ದೊರೆಯುವುದೆಂದರೂ ಇವರನ್ನು ಸಾಯಿಸಲಾರೆ. ಇನ್ನು ಈ ನೆಲಕ್ಕಾಗಿ ಹೊಡೆದೇನೆ?\\}
\slcol{ನಿಹತ್ಯ ಧಾರ್ತರಾಷ್ಟ್ರಾನ್ನಃ ಕಾ ಪ್ರೀತಿಃ ಸ್ಯಾಜ್ಜನಾರ್ದನ ।\\
ಪಾಪಮೇವಾಶ್ರಯೇದಸ್ಮಾನ್ಹತ್ವೈತಾನಾತತಾಯಿನಃ ॥ 36 ॥}
\cquote{ಕೃಷ್ಣ, ಕೌರವರನ್ನು ಕೊಂದು ನಮಗೇನು ತೃಪ್ತಿ? ಈ ಕೇಡಿಗಳನ್ನು ಕೊಲ್ಲುವುದರಿಂದ ನಮಗೆ ಪಾಪವೇ ಗಂಟುಬಿದ್ದೀತು.\\}
\slcol{ತಸ್ಮಾನ್ನಾರ್ಹಾ ವಯಂ ಹಂತುಂ ಧಾರ್ತರಾಷ್ಟ್ರಾನ್ಸ್ವಬಾಂಧವಾನ್ ।\\
ಸ್ವಜನಂ ಹಿ ಕಥಂ ಹತ್ವಾ ಸುಖಿನಃ ಸ್ಯಾಮ ಮಾಧವ ॥ 37 ॥}
\cquote{ಆದ್ದರಿಂದ ನಮ್ಮವರಾದ ಕೌರವರನ್ನು ನಾವು ಕೊಲ್ಲಬಾರದು, ಮಾಧವ ನಮ್ಮವರನ್ನೇ ಕೊಂದು ನಾವು ಹೇಗೆ ಸುಖಿಗಳಾಗಿರುವೆವು?\\}
\slcol{ಯದ್ಯಪ್ಯೇತೇ ನ ಪಶ್ಯಂತಿ ಲೋಭೋಪಹತಚೇತಸಃ ।\\
ಕುಲಕ್ಷಯಕೃತಂ ದೋಷಂ ಮಿತ್ರದ್ರೋಹೇ ಚ ಪಾತಕಮ್ ॥ 38 ॥}
\cquote{ಆಸೆಗೆ ಬಲಿಯಾಗಿ ಬುದ್ಧಿ ಕಳಕೊಂಡ ಈ ಜನ ಕುಲನಾಶದ ಕೆಟ್ಟ ಪರಿಣಾಮವನ್ನೂ ಗೆಳೆಯರಿಗೆ ಮೋಸ ಮಾಡಿದ ಪಾಪವನ್ನೂ ಅರ್ಥಮಾಡಿಕೊಳ್ಳುತ್ತಿಲ್ಲ, ನಿಜ.\\}
\slcol{ಕಥಂ ನ ಙ್ಞೇಯಮಸ್ಮಾಭಿಃ ಪಾಪಾದಸ್ಮಾನ್ನಿವರ್ತಿತುಮ್ ।\\
ಕುಲಕ್ಷಯಕೃತಂ ದೋಷಂ ಪ್ರಪಶ್ಯದ್ಭಿರ್ಜನಾರ್ದನ ॥ 39 ॥}
\cquote{ಆದರೆ ಓ ಜನಾರ್ಧನ, ಕುಲನಾಶದ ದುರಂತವನ್ನು ತಿಳಿದ ನಮಗೆ ಈ ಪಾಪದಿಂದ ಹಿಮ್ಮೆಟ್ಟಬೇಕೆಂದು ತಿಳಿಯದಿರುವುದು ಹೇಗೆ? \\}
\slcol{ಕುಲಕ್ಷಯೇ ಪ್ರಣಶ್ಯಂತಿ ಕುಲಧರ್ಮಾಃ ಸನಾತನಾಃ ।\\
ಧರ್ಮೇ ನಷ್ಟೇ ಕುಲಂ ಕೃತ್ಸ್ನಮಧರ್ಮೋऽಭಿಭವತ್ಯುತ ॥ 40 ॥}
\cquote{ಕುಲ ನಾಶವಾದರೆ ಬಹು ಕಾಲದಿಂದ ನಡೆದು ಬಂದ ಕುಲ ಧರ್ಮಗಳೆಲ್ಲ ಹೋಗಿ ಬಿಡುವು. ಕುಲಧರ್ಮ ಹಾಳಾದರೆ ಕುಲವನ್ನೆಲ್ಲ ಅಧರ್ಮವು ಆಕ್ರಮಿಸಿ ಬಿಡುವು.\\}
\slcol{ಅಧರ್ಮಾಭಿಭವಾತ್ಕೃಷ್ಣ ಪ್ರದುಷ್ಯಂತಿ ಕುಲಸ್ತ್ರಿಯಃ ।\\
ಸ್ತ್ರೀಷು ದುಷ್ಟಾಸು ವಾರ್ಷ್ಣೇಯ ಜಾಯತೇ ವರ್ಣಸಂಕರಃ ॥ 41 ॥}
\cquote{ಕೃಷ್ಣ, ಅಧರ್ಮದ ಆಕ್ರಮಣದಿಂದ ಕುಲೀನ ಹೆಂಗಸರು ಕೆಡುವರು. ಹೆಂಗಸರು ಕೆಟ್ಟರೆ ಸಮಾಜ ಬಣ್ಣಗೆಡುತ್ತದೆ. \\}
\begin{inspiration}{\kanfont ಸ್ಪೂರ್ತಿ}
ಜೀವನದ ಸ್ಪರ್ಧೆಗಳಿಗೆ ಎದ್ದು ನಿಲ್ಲಬೇಕು. ನಮ್ಮದೇ ಸ್ವಂತ ಜೀವನಕ್ಕಾಗಿ ಜವಾಬ್ದಾರಿಗಳನ್ನು ತೆಗೆದುಕೊಳ್ಳಬೇಕು. ನಿಷ್ಕಾರುಣ್ಯವಾಗಿ, ಎಲ್ಲಾ ಋಣಾತ್ಮಕ ಸಹವಾಸಗಳಿಂದ ಮತ್ತು ಪರಿಸರಗಳಿಂದ ದೂರವಾಗಿರಬೇಕು. ಇನ್ನೊಬ್ಬರ ಕೈಯಿಂದ ನಿಮ್ಮ ಮಾನಸಿಕ ನೆಮ್ಮದಿಯನ್ನು ಕಳೆದುಕೊಳ್ಳುವಂತಹದರ ಬಗ್ಗೆ ರಾಜಿ ಮಾಡಿಕೊಳ್ಳಬಾರದು. ಭೂತಕಾಲವನ್ನು ಹೋಗಲು ಬಿಡಬೇಕು ಮತ್ತು ವರ್ತಮಾನದಲ್ಲಿ ಉತ್ತಮವಾದದ್ದನ್ನು ಮಾಡಬೇಕು. ಉತ್ತಮವಾದ ಭವಿಷ್ಯ ನಿಮ್ಮ ಹಿಡಿತದಲ್ಲಿರುವುದು. 
\end{inspiration}
\newpage
\begin{mananam}{\kanfont ಮನನ}
ಯಾವ ಸಮಯದಲ್ಲಾದರೂ ಜವಾಬ್ದಾರಿಯ ಕೊರತೆಯಿಂದಾಗಿ ನಾನು ನನ್ನ ಕ್ರಿಯೆ ಮತ್ತು ನಿಷ್ಕ್ರಿಯೆಗಳನ್ನು ಸಮರ್ಥಿಸಿಕೊಳ್ಳುತ್ತೇನೆಯೇ? ಪೊಳ್ಳು ಅರ್ಥದ ಅನುಕಂಪದಿಂದ ನನ್ನನ್ನು ಅಧ್ಯಾತ್ಮದಿಂದ ಕೆಳಗೆ ತಳ್ಳುವವರು ಮತ್ತು ಋಣಾತ್ಮಕವಾಗಿ ಪ್ರಭಾವ ಬೀರುವವರಿಂದ ಸಂಬಂಧ ಕಡಿದುಕೊಳ್ಳುವ ಭಯ ನನಗಿದೆಯೇ? ನನ್ನ ಆಧ್ಯಾತ್ಮಿಕ ಜೀವನಕ್ಕೆ ಉಪಯೋಗವಿಲ್ಲದ ಜನರಿಗೆ ಮತ್ತು ಆಹ್ವಾನಕ್ಕೆ `ಇಲ್ಲ` ಅಥವಾ ಬೇಡ ಎಂದು ಹೇಳಲಾರದಷ್ಟು ದುರ್ಬಲನೆ ನಾನು?\\
\end{mananam}
\Linepage
\newpage


\slcol{ಸಂಕರೋ ನರಕಾಯೈವ ಕುಲಘ್ನಾನಾಂ ಕುಲಸ್ಯ ಚ ।\\
ಪತಂತಿ ಪಿತರೋ ಹ್ಯೇಷಾಂ ಲುಪ್ತಪಿಂಡೋದಕಕ್ರಿಯಾಃ ॥ 42 ॥}
\cquote{ಇಂಥ ಬೆರಕೆ ಸಮಾಜ ಕುಲವನ್ನು ಕುಲಕಂಠಕರನ್ನೂ ಜನತೆಯನ್ನು ನರಕಕ್ಕೆ ತಳ್ಳುತ್ತದೆ. ಅದರಿಂದ ಇಂಥವರಿಂದ ಹಿರಿಯರು ಪಿಂಡಪ್ರದಾನ, ಜಲತರ್ಪಣ ಇಲ್ಲದವರಾಗಿ ಕೆಳಕ್ಕೆ ಬೀಳುವರು.\\}
\slcol{ದೋಷೈರೇತೈಃ ಕುಲಘ್ನಾನಾಂ ವರ್ಣಸಂಕರಕಾರಕೈಃ ।\\
ಉತ್ಸಾದ್ಯಂತೇ ಜಾತಿಧರ್ಮಾಃ ಕುಲಧರ್ಮಾಶ್ಚ ಶಾಶ್ವತಾಃ ॥ 43 ॥}
\cquote{ಸಮಾಜದ ವ್ಯವಸ್ಥೆಯನ್ನು ಕೆಡಿಸುವ ಇಂತ ಈ ಕುಲನಾಶಕರ ದೋಷಗಳಿಂದಾಗಿ ನಿರಂತವಾಗಿ ನಡೆದು ಬಂದ ಜಾತಿಧರ್ಮಗಳೂ ಕುಲ ಧರ್ಮಗಳೂ ನಿರ್ಮೂಲವಾಗುತ್ತವೆ.\\}
\slcol{ಉತ್ಸನ್ನಕುಲಧರ್ಮಾಣಾಂ ಮನುಷ್ಯಾಣಾಂ ಜನಾರ್ದನ ।\\
ನರಕೇऽನಿಯತಂ ವಾಸೋ ಭವತೀತ್ಯನುಶುಶ್ರುಮ ॥ 44 ॥}
\cquote{ಜನಾರ್ದನ, ಕುಲಕರ್ಮಗಳನ್ನೆಲ್ಲ ಹಾಳು ಮಾಡಿಕೊಂಡ ಮನುಷ್ಯರು ಯಾವಾಗಲೂ ನರಕದಲ್ಲಿರಬೇಕಾಗುವುದೆಂದು ಕೇಳಿದ್ದುಂಟು.\\}
\slcol{ಅಹೋ ಬತ ಮಹತ್ಪಾಪಂ ಕರ್ತುಂ ವ್ಯವಸಿತಾ ವಯಮ್ ।\\
ಯದ್ರಾಜ್ಯಸುಖಲೋಭೇನ ಹಂತುಂ ಸ್ವಜನಮುದ್ಯತಾಃ ॥ 45 ॥}
\cquote{ರಾಜ್ಯದಿಂದ ಲಭಿಸುವ ಸುಖದ ಮೋಹದಿಂದ ನಮ್ಮವರನ್ನೇ ಕೊಲ್ಲ ಹೊರಟಿರುವ ನಾವು ಆಹಾ! ಎಂಥ ದೊಡ್ಡ ಪಾಪವನ್ನು ಮಾಡುವುದಕ್ಕೆ ಹೊರಟಿರುವೆವು.\\}
\slcol{ಯದಿ ಮಾಮಪ್ರತೀಕಾರಮಶಸ್ತ್ರಂ ಶಸ್ತ್ರಪಾಣಯಃ ।\\
ಧಾರ್ತರಾಷ್ಟ್ರಾ ರಣೇ ಹನ್ಯುಸ್ತನ್ಮೇ ಕ್ಷೇಮತರಂ ಭವೇತ್ ॥ 46 ॥}
\cquote{ಒಂದು ವೇಳೆ ಹೋರಾಡಬಯಸದೆ ನಿರಾಯುಧನಾಗಿ ನಿಂತ ನನ್ನನ್ನು ಆಯುಧ ಪಾಣಿಗಳಾದ ಕೌರವರು ಯುದ್ಧದಲ್ಲಿ ಕೊಂದರೆ ಅದು ನನಗೆ ಹೆಚ್ಚಿನ ಒಳ್ಳೆಯದೇ ಆದೀತು.\\}
\slcol{ಸಂಜಯ ಉವಾಚ ।\\
ಏವಮುಕ್ತ್ವಾರ್ಜುನಃ ಸಂಖ್ಯೇ ರಥೋಪಸ್ಥ ಉಪಾವಿಶತ್ ।\\
ವಿಸೃಜ್ಯ ಸಶರಂ ಚಾಪಂ ಶೋಕಸಂವಿಗ್ನಮಾನಸಃ ॥ 47 ॥ }
\cquote{ಸಂಜಯನು ಹೇಳಿದನು,\\
ದುಃಖದಿಂದ ತಳಮಳಗೊಂಡ ಅರ್ಜುನನು ಹೀಗೆ ಹೇಳಿ, ಬಿಲ್ಲು ಬಾಣಗಳನ್ನು ಕೆಳಕ್ಕೆ ಚೆಲ್ಲಿ ರಣರಂಗದಲ್ಲಿ ರಥದಲ್ಲಿ ಕುಳಿತುಬಿಟ್ಟನು.\\}
\begin{center}
{\tiny\color{brown}
ಓಂ ತತ್ಸದಿತಿ ಶ್ರೀಮದ್ಭಗವದ್ಗೀತಾಸೂಪನಿಷತ್ಸು \\
ಬ್ರಹ್ಮವಿದ್ಯಾಯಾಂ ಯೋಗಶಾಸ್ತ್ರೇ ಶ್ರೀಕೃಷ್ಣಾರ್ಜುನಸಂವಾದೇ\\
ಅರ್ಜುನವಿಷಾದಯೋಗೋ ನಾಮ ಪ್ರಥಮೋऽಧ್ಯಾಯಃ ॥1॥\\}
\end{center}
\chapter{\kanfont ೩ ಕರ್ಮ ಯೋಗ}
\pagenumbering{arabic}
%\renewcommand* \thepage {\localnumeral*{page}} %Page numbers in Kannada
%ಅಥ ಪ್ರಥಮೋऽಧ್ಯಾಯಃ ।\\
ಮೊಟ್ಟ ಮೊದಲನೆಯ ಶ್ಲೋಕವೇ ನಮಗೆ ಚಿಂತನೆ, ಮನನ ಪ್ರಾರಂಭಿಸಲು ಬೇಕಾಗುವ ಸೂಕ್ಷ್ಮವಾದ ಸಂದೇಶವನ್ನು ಕೊಡುತ್ತದೆ.\\
\slcol{ಧೃತರಾಷ್ಟ್ರ ಉವಾಚ ।\\
ಧರ್ಮಕ್ಷೇತ್ರೇ ಕುರುಕ್ಷೇತ್ರೇ ಸಮವೇತಾ ಯುಯುತ್ಸವಃ ।\\
ಮಾಮಕಾಃ ಪಾಂಡವಾಶ್ಚೈವ ಕಿಮಕುರ್ವತ ಸಂಜಯ ॥ 1 ॥}
\cquote{ಧೃತರಾಷ್ಟ್ರನು ಹೇಳಿದನು,\\
ಸಂಜಯನೇ, ಯುದ್ಧದ ಬಯಕೆಯಿಂದ ಧರ್ಮಭೂಮಿಯಾದ ಕುರುಕ್ಷೇತ್ರದಲ್ಲಿ ಕಲೆತ ನನ್ನ ಮಕ್ಕಳೂ ಪಾಂಡವರೂ ಏನು ಮಾಡಿದರು?\\}
\slcol{ಸಂಜಯ ಉವಾಚ ।\\
ದೃಷ್ಟ್ವಾ ತು ಪಾಂಡವಾನೀಕಂ ವ್ಯೂಢಂ ದುರ್ಯೋಧನಸ್ತದಾ ।\\
ಆಚಾರ್ಯಮುಪಸಂಗಮ್ಯ ರಾಜಾ ವಚನಮಬ್ರವೀತ್ ॥ 2 ॥}
\cquote{ಸಂಜಯನು ಹೇಳಿದನು,\\
ಪಾಂಡವರ ದಂಡು ಸಜ್ಜಾಗಿ ನಿಂತಿದ್ದುದನ್ನು ನೋಡಿದ ಅರಸನಾದ ದುರ್ಯೋಧನನು ಗುರುಗಳಾದ ದ್ರೋಣರ ಬಳಿಗೆ ಬಂದು ಹೀಗೆ ಹೇಳಿದನು. \\}
\begin{inspiration}{\kanfont ಸ್ಪೂರ್ತಿ}
ನಿನಗೆ ನೀನು ಸತ್ಯವಾಗಿರು ಮತ್ತು ನೀನು ಉನ್ನತಿಯತ್ತ ಬದಲಾಗುವೆ. ಜೀವನದಲ್ಲಿ ಜಾಣನಿಗೆ ಅವಶ್ಯಕವಾದುದು ಪಕ್ಷಪಾತ ರಹಿತ ಅವಲೋಕನ. ನಮ್ಮನ್ನು ನಾವು ಬದಲಾಯಿಸಿಕೊಳ್ಳಲು ಕೇವಲ ಬಯಕೆ ಇದ್ದರೆ ಮಾತ್ರ ಸಾಲದು. ಜ್ಞಾನಿಗಳ ಮಹತ್ವದ, ಉನ್ನತವಾದ ಬೋಧನೆಗಳಿಂದ ನಮ್ಮ ಯೋಚನೆಗಳು, ಮಾತುಗಳು ಮತ್ತು ಕೃತಿಗಳನ್ನು ತಹಬಂದಿಗೆ ತಂದು, ಪ್ರತಿದಿನವೂ ನಮ್ಮನ್ನು ನಾವು ಆತ್ಮ ವಿಮರ್ಶೆ ಮಾಡಿಕೊಳ್ಳಲೇಬೇಕು.
\end{inspiration}
\newpage
\begin{mananam}{\kanfont ಮನನ}
ನನ್ನ ಜೀವನದ ದೈನಂದಿನ ನಿತ್ಯಕರ್ಮದಲ್ಲಿ ಯಾವಾಗ ನನ್ನ ದೇಹವು, ಆಸೆ, ಕೋಪ, ಭಯ, ಮತ್ಸರ ಇತ್ಯಾದಿಗಳಲ್ಲಿ ಒಲವು ತೋರುವುದನ್ನು ಗುರುತಿಸಿತು, ಅವುಗಳನ್ನು ಸ್ವಾತಂತ್ರ್ಯವನ್ನು ಆಳವಾಗಿ ಪ್ರೇರೇಪಿಸುವ ನನ್ನನ್ನು ಪ್ರತಿಭಟಿಸುವಂತೆ ಮಾಡುವ ಮತ್ತು ಸನಾತನ ಗ್ರಂಥ ಮತ್ತು ಬೋಧಕರಿಂದ ಪಡೆದ ಜ್ಞಾನವನ್ನು ಯಾವ ಬಲವನ್ನು ಅನುಸರಿಸಿದೆ? ನನ್ನ ಹಂಬಲ ಮತ್ತು ಸಂಕಲ್ಪಗಳನ್ನು ತಳ್ಳಿಹಾಕುವ ನನ್ನ ದುರಭ್ಯಾಸಗಳು ಮತ್ತು ಅಪಾಯಕಾರಿ ನಡವಳಿಕೆಗಳಿಂದಾಗಿ ನನ್ನ ನಿತ್ಯ ಜೀವನದಲ್ಲಿ ಏನೇನು ಕಷ್ಟ ಪಡಬೇಕಾಯಿತು?
\end{mananam}
\Linepage
\newpage

\slcol{ಪಶ್ಯೈತಾಂ ಪಾಂಡುಪುತ್ರಾಣಾಮಾಚಾರ್ಯ ಮಹತೀಂ ಚಮೂಮ್ ।\\
ವ್ಯೂಢಾಂ ದ್ರುಪದಪುತ್ರೇಣ ತವ ಶಿಷ್ಯೇಣ ಧೀಮತಾ ॥ 3 ॥}
\cquote{ಗುರುಗಳೇ, ದೃಪದರಾಜನ ಮಗ ನಿಮ್ಮ ಶಿಷ್ಯ, ಬುದ್ಧಿಶಾಲಿಯಾದ ದೃಷ್ಟದ್ಯುಮ್ನ ಪಾಂಡವರ ಈ ದೊಡ್ಡ ದಂಡನ್ನು ಸಜ್ಜುಗೊಳಿಸಿರುವುದನ್ನು ನೋಡಿರಿ.\\}
\slcol{ಅತ್ರ ಶೂರಾ ಮಹೇಷ್ವಾಸಾ ಭೀಮಾರ್ಜುನಸಮಾ ಯುಧಿ ।\\
ಯುಯುಧಾನೋ ವಿರಾಟಶ್ಚ ದ್ರುಪದಶ್ಚ ಮಹಾರಥಃ ॥ 4 ॥}
\cquote{ಈ ದಂಡಿನಲ್ಲಿ ಹೋರಾಟದಲ್ಲಿ ಭೀಮಾರ್ಜುನರಿಗೆ ಸರಿ ಜೋಡಿಯಾದ ಶೂರರಾಗಿ ದೊಡ್ಡ ದೊಡ್ಡ ಬಿಲ್ಲುಗಳನ್ನು ಹಿಡಿದುಕೊಂಡು ಕಾದುವುದರಲ್ಲಿ ಕುಶಲರಾದ ಸಾತ್ಯಕಿ ವಿರಾಟರಿದ್ದಾರೆ. ಸಹಸ್ರ ಜನರೊಡನೆ ಏಕಾಂಗಿಯಾಗಿ ಹೋರಾಡಬಲ್ಲ ದ್ರುಪದನಿದ್ದಾನೆ.\\}
\slcol{ಧೃಷ್ಟಕೇತುಶ್ಚೇಕಿತಾನಃ ಕಾಶಿರಾಜಶ್ಚ ವೀರ್ಯವಾನ್ ।\\
ಪುರುಜಿತ್ಕುಂತಿಭೋಜಶ್ಚ ಶೈಬ್ಯಶ್ಚ ನರಪುಂಗವಃ ॥ 5 ॥}
\cquote{ದೃಷ್ಟಕೇತು, ಚೀಕಿತಾನ, ವೀರನಾದ ಕಾಶಿರಾಜ, ಮತ್ತು ಮನುಷ್ಯರಲ್ಲಿ ಶ್ರೇಷ್ಠನಾದ ಶೈಭ್ಯ ಇವರೆಲ್ಲ ಇದ್ದಾರೆ. \\} 
\slcol{ಯುಧಾಮನ್ಯುಶ್ಚ ವಿಕ್ರಾಂತ ಉತ್ತಮೌಜಾಶ್ಚ ವೀರ್ಯವಾನ್ ।\\
ಸೌಭದ್ರೋ ದ್ರೌಪದೇಯಾಶ್ಚ ಸರ್ವ ಏವ ಮಹಾರಥಾಃ ॥ 6 ॥}
\cquote{ಬಲಶಾಲಿಯಾದ ಯುಧಾಮನ್ಯು, ವೀರನಾದ ಉತ್ತಮೌಜ, ಸುಭದ್ರೆಯ ಮಗ ಅಭಿಮನ್ಯು ಮತ್ತು ದ್ರೌಪದಿಯ ಮಕ್ಕಳು ಇದ್ದಾರೆ. ಎಲ್ಲರೂ ಒಬ್ಬೊಬ್ಬರು ಹತ್ತು ಸಹಸ್ರ ಜನರೊಡನೆ ಹೋರಾಡಬಲ್ಲ ಮಹಾರುತರು. \\}
\slcol{ಅಸ್ಮಾಕಂ ತು ವಿಶಿಷ್ಟಾ ಯೇ ತಾನ್ನಿಬೋಧ ದ್ವಿಜೋತ್ತಮ ।\\
ನಾಯಕಾ ಮಮ ಸೈನ್ಯಸ್ಯ ಸಂಙ್ಞಾರ್ಥಂ ತಾನ್ಬ್ರವೀಮಿ ತೇ ॥ 7 ॥}
\cquote{ಬ್ರಾಹ್ಮಣ ಶ್ರೇಷ್ಠರೇ, ನಮ್ಮ ಕಡೆಯಲ್ಲಿರುವ ವೀರರನ್ನು ನೆನಪಿಗೆ ತಂದುಕೊಳ್ಳಿ. ತಮಗೆ ನೆನಪಾಗಲೆಂದು ಅವರ ಹೆಸರುಗಳನ್ನು ಹೇಳುತ್ತೇನೆ.\\} 
\slcol{ಭವಾನ್ಭೀಷ್ಮಶ್ಚ ಕರ್ಣಶ್ಚ ಕೃಪಶ್ಚ ಸಮಿತಿಂಜಯಃ ।\\
ಅಶ್ವತ್ಥಾಮಾ ವಿಕರ್ಣಶ್ಚ ಸೌಮದತ್ತಿಸ್ತಥೈವ ಚ ॥ 8 ॥}
\cquote{ತಾವು ಭೀಷ್ಮ ಕರ್ಣ ಜಯಶೀಲನಾದ ಕೃಪಾ, ಅಶ್ವತ್ಥಾಮ, ವಿಕರ್ಣ ಸೋಮದತ್ತನ ಮಗನಾದ ಭೂರಿಶ್ರವ ಮತ್ತು ಜಯದ್ರಥ. \\}
\slcol{ಅನ್ಯೇ ಚ ಬಹವಃ ಶೂರಾ ಮದರ್ಥೇ ತ್ಯಕ್ತಜೀವಿತಾಃ ।\\
ನಾನಾಶಸ್ತ್ರಪ್ರಹರಣಾಃ ಸರ್ವೇ ಯುದ್ಧವಿಶಾರದಾಃ ॥ 9 ॥}
\cquote{ಇನ್ನೂ ಅನೇಕ ಶೂರರು ನನಗಾಗಿ ಜೀವ ತೆರಲು ಸಿದ್ದರಾಗಿ ಇದ್ದಾರೆ. ಎಲ್ಲರೂ ಎಲ್ಲ ಬಗಯ ಆಯುಧಗಳನ್ನು ಉಪಯೋಗಿಸಬಲ್ಲವರು ಮತ್ತು ಯುದ್ಧದಲ್ಲಿ ಗಟ್ಟಿಗರು.\\}
\slcol{ಅಪರ್ಯಾಪ್ತಂ ತದಸ್ಮಾಕಂ ಬಲಂ ಭೀಷ್ಮಾಭಿರಕ್ಷಿತಮ್ ।\\
ಪರ್ಯಾಪ್ತಂ ತ್ವಿದಮೇತೇಷಾಂ ಬಲಂ ಭೀಮಾಭಿರಕ್ಷಿತಮ್ ॥ 10 ॥}
\cquote{ಭೀಷ್ಮರ ರಕ್ಷಣೆಗೆ ಒಳಪಟ್ಟಿರುವ ನಮ್ಮ ದೊಡ್ಡ ಆ ದಂಡು ಸಾಲದೇನೋ ಎನಿಸುತ್ತದೆ. ಭೀಮನ ರಕ್ಷಣೆಗೆ ಒಳಪಟ್ಟಿರುವ ಪಾಂಡವರ ಈ ಸೇನೆ ಸಾಕಷ್ಟು ಸಮರ್ಥವಾಗಿದೆ.\\}
\slcol{ಅಯನೇಷು ಚ ಸರ್ವೇಷು ಯಥಾಭಾಗಮವಸ್ಥಿತಾಃ ।\\
ಭೀಷ್ಮಮೇವಾಭಿರಕ್ಷಂತು ಭವಂತಃ ಸರ್ವ ಏವ ಹಿ ॥ 11 ॥}
\cquote{ನೀವೆಲ್ಲರೂ ದಂಡಿನ ಬೇರೆ ಬೇರೆ ಮಾರ್ಗಗಳಲ್ಲಿ ನಿಮ್ಮ ನಿಮ್ಮ ಪಾಲಿಗೆ ಬಂದ ಕಡೆ ಇದ್ದುಕೊಂಡು ಭೀಷ್ಮನನ್ನು ರಕ್ಷಿಸಿರಿ.\\}
\slcol{ತಸ್ಯ ಸಂಜನಯನ್ಹರ್ಷಂ ಕುರುವೃದ್ಧಃ ಪಿತಾಮಹಃ ।\\
ಸಿಂಹನಾದಂ ವಿನದ್ಯೋಚ್ಚೈಃ ಶಂಖಂ ದಧ್ಮೌ ಪ್ರತಾಪವಾನ್ ॥ 12 ॥}
\cquote{ಹೀಗೆಂದು ಹೇಳಿದ ದುರ್ಯೋಧನನಿಗೆ ಹರ್ಷ ಉಂಟಾಗುವಂತೆ ಆಗ ಕುರುವಂಶದ ಹಿರಿಯ ಕೌರವರ ಅಜ್ಜ, ಪರಾಕ್ರಮಶಾಲಿ ಭೀಷ್ಮನು ಗಟ್ಟಿಯಾಗಿ ಸಿಂಹನಾದ ಮಾಡಿ ಶಂಖವನ್ನು ಊದಿದನು.\\}
\slcol{ತತಃ ಶಂಖಾಶ್ಚ ಭೇರ್ಯಶ್ಚ ಪಣವಾನಕಗೋಮುಖಾಃ ।\\
ಸಹಸೈವಾಭ್ಯಹನ್ಯಂತ ಸ ಶಬ್ದಸ್ತುಮುಲೋऽಭವತ್ ॥ 13 ॥}
\cquote{ಆಮೇಲೆ ಒಮ್ಮೆಲೆ ಶಂಖಗಳು, ಭೇರಿಗಳು, ಮೃದಂಗಗಳು, ನಗಾಡಿಗಳು, ರಣ ಸಿಂಹಗಳು ಒಳಗಿದವು. ಆ ಗದ್ದಲವು ಎಲ್ಲೆಲ್ಲಿಯೂ ತುಂಬಿತು.\\}
\slcol{ತತಃ ಶ್ವೇತೈರ್ಹಯೈರ್ಯುಕ್ತೇ ಮಹತಿ ಸ್ಯಂದನೇ ಸ್ಥಿತೌ ।\\
ಮಾಧವಃ ಪಾಂಡವಶ್ಚೈವ ದಿವ್ಯೌ ಶಂಖೌ ಪ್ರದಘ್ಮತುಃ ॥ 14 ॥}
\cquote{ಆಮೇಲೆ ಬಿಳಿ ಕುದುರೆಯನ್ನು ಹೂಡಿದ ದೊಡ್ಡ ತೇರಿನ ಮೇಲೆ ಕುಳಿತಿದ್ದ ಕೃಷ್ಣನೂ ಅರ್ಜುನನೂ ಹೆಸರುವಾಸಿಯಾದ ದಿವ್ಯವಾದ ತಮ್ಮ ಶಂಖಗಳನ್ನು ಊದಿದರು.\\}
\slcol{ಪಾಂಚಜನ್ಯಂ ಹೃಷೀಕೇಶೋ ದೇವದತ್ತಂ ಧನಂಜಯಃ ।\\
ಪೌಂಡ್ರಂ ದಧ್ಮೌ ಮಹಾಶಂಖಂ ಭೀಮಕರ್ಮಾ ವೃಕೋದರಃ ॥ 15 ॥}
\cquote{ಕೃಷ್ಣನು ಪಾಂಚಜನ್ಯವನ್ನೂ ಅರ್ಜುನನ್ನು ದೇವದತ್ತವನ್ನೂ, ಶತ್ರುಗಳನ್ನು ಎದೆಗೂಡಿಸುವ ಭೀಮನು ಪೌಂಡ್ರವೆಂಬ ದೊಡ್ಡ ಶಂಖವನ್ನು ಓದಿದನು.\\}
\slcol{ಅನಂತವಿಜಯಂ ರಾಜಾ ಕುಂತೀಪುತ್ರೋ ಯುಧಿಷ್ಠಿರಃ ।\\
ನಕುಲಃ ಸಹದೇವಶ್ಚ ಸುಘೋಷಮಣಿಪುಷ್ಪಕೌ ॥ 16 ॥}
\cquote{ಕುಂತಿಯ ಹಿರಿಯ ಮಗ, ಅರಸನಾದ ಧರ್ಮರಾಯನು ಅನಂತ ವಿಜಯವನ್ನೂ ನಕುಲನೂ ಸುಘೋಷವನ್ನೂ ಸಹದೇವನು ಮಣಿಪುಷ್ಪಕವನ್ನೂ ಊದಿದರು. \\}
\slcol{ಕಾಶ್ಯಶ್ಚ ಪರಮೇಷ್ವಾಸಃ ಶಿಖಂಡೀ ಚ ಮಹಾರಥಃ ।\\
ಧೃಷ್ಟದ್ಯುಮ್ನೋ ವಿರಾಟಶ್ಚ ಸಾತ್ಯಕಿಶ್ಚಾಪರಾಜಿತಃ ॥ 17 ॥\\
ದ್ರುಪದೋ ದ್ರೌಪದೇಯಾಶ್ಚ ಸರ್ವಶಃ ಪೃಥಿವೀಪತೇ ।\\
ಸೌಭದ್ರಶ್ಚ ಮಹಾಬಾಹುಃ ಶಂಖಾಂದಧ್ಮುಃ ಪೃಥಕ್ಪೃಥಕ್ ॥ 18 ॥}
\cquote{ಓ ಧೃತರಾಷ್ಟ್ರ ಕೇಳು, ಹಿರಿಯ ಬಿಲ್ಲೋಜ ಕಾಶಿರಾಜ, ಮಹಾರಥನಾದ ಶಿಖಂಡಿ, ಧೃಷ್ಟದ್ಯುಮ್ನ,  ವಿರಾಟ, ಸೋಲರಿಯದ ಸಾತ್ಯಕಿ, ದ್ರುಪದ, ದ್ರೌಪದಿಯ ಮಕ್ಕಳು, ಮಹಾಬಾಹುವಾದ ಅಭಿಮನ್ಯು ಹೀಗೆ ಎಲ್ಲರೂ ತಮ್ಮ ತಮ್ಮ ಶಂಖಗಳನ್ನು ಊದಿದರು.\\}
\slcol{ಸ ಘೋಷೋ ಧಾರ್ತರಾಷ್ಟ್ರಾಣಾಂ ಹೃದಯಾನಿ ವ್ಯದಾರಯತ್ ।\\
ನಭಶ್ಚ ಪೃಥಿವೀಂ ಚೈವ ತುಮುಲೋ ವ್ಯನುನಾದಯನ್ ॥ 19 ॥}
\cquote{ಆ ಗದ್ದಲವು ಭೂಮಿಯಲ್ಲಿಯೂ ಆಕಾಶದಲ್ಲಿಯೂ ತುಂಬಿ ಪ್ರತಿಧ್ವನಿಯನ್ನು ಹಬ್ಬಿಸಿ ಕೌರವರ ಎದೆ ಬಿರಿಯುವಂತೆ ಮಾಡಿತು.\\}
\slcol{ಅಥ ವ್ಯವಸ್ಥಿತಾಂದೃಷ್ಟ್ವಾ ಧಾರ್ತರಾಷ್ಟ್ರಾನ್ಕಪಿಧ್ವಜಃ ।\\
ಪ್ರವೃತ್ತೇ ಶಸ್ತ್ರಸಂಪಾತೇ ಧನುರುದ್ಯಮ್ಯ ಪಾಂಡವಃ ॥ 20 ॥\\
ಹೃಷೀಕೇಶಂ ತದಾ ವಾಕ್ಯಮಿದಮಾಹ ಮಹೀಪತೇ ।}
\cquote{ಓ ಧೃತರಾಷ್ಟ್ರ, ಸಜ್ಜಾಗಿ ಎದುರಿಗೆ ನಿಂತಿರುವ ಕೌರವರನ್ನು ನೋಡಿ ಕಪಿಧ್ವಜನಾದ ಅರ್ಜುನನು ಹೊಡೆದಾಟಕ್ಕೆ ಮೊದಲು ಮಾಡಬೇಕಾದ ಆ ಸಮಯದಲ್ಲಿ ಗಾಂಡೀವವನ್ನು ಕೈಗೆ ತೆಗೆದುಕೊಂಡು ಕೃಷ್ಣನನ್ನು ಕುರಿತು ಈ ಮಾತನ್ನು ಹೇಳಿದನು.\\}
\slcol{ಅರ್ಜುನ ಉವಾಚ ।\\
ಸೇನಯೋರುಭಯೋರ್ಮಧ್ಯೇ ರಥಂ ಸ್ಥಾಪಯ ಮೇऽಚ್ಯುತ ॥ 21 ॥}
\cquote{ಅರ್ಜುನನ್ನು ಹೇಳಿದನು, ಕೃಷ್ಣ, ಎರಡು ದಂಡುಗಳ ನಡುವೆ ನನ್ನ ರಥವನ್ನು ನಿಲ್ಲಿಸು.\\}
\slcol{ಯಾವದೇತಾನ್ನಿರೀಕ್ಷೇऽಹಂ ಯೋದ್ಧುಕಾಮಾನವಸ್ಥಿತಾನ್ ।\\
ಕೈರ್ಮಯಾ ಸಹ ಯೋದ್ಧವ್ಯಮಸ್ಮಿನ್ರಣಸಮುದ್ಯಮೇ ॥ 22 ॥}
\cquote{ಕಾದಬೇಕೆಂದು ನಿಂತಿರುವವರನ್ನು, ಈ ಯುದ್ಧದಲ್ಲಿ ನಾನು ಯಾರೊಡನೆ ಕಾದಬೇಕಾಗಿದೆ ಎಂಬುದನ್ನು ಒಮ್ಮೆ ನೋಡುತ್ತೇನೆ.\\}
\slcol{ಯೋತ್ಸ್ಯಮಾನಾನವೇಕ್ಷೇऽಹಂ ಯ ಏತೇऽತ್ರ ಸಮಾಗತಾಃ ।\\
ಧಾರ್ತರಾಷ್ಟ್ರಸ್ಯ ದುರ್ಬುದ್ಧೇರ್ಯುದ್ಧೇ ಪ್ರಿಯಚಿಕೀರ್ಷವಃ ॥ 23 ॥}
\cquote{ದುರ್ಬುದ್ಧಿಯ ದುರ್ಯೋಧನನಿಗೆ ಈ ಯುದ್ಧದಲ್ಲಿ ನೆರವಾಗಬೇಕೆಂದು ಕಾದುವುದಕ್ಕಾಗಿ ಯಾರು ಯಾರು ಇಲ್ಲಿಗೆ ಬಂದಿರುತ್ತಾರೆ ಎಂಬುದನ್ನು ನಾನೊಮ್ಮೆ ನೋಡುತ್ತೇನೆ.\\}
\slcol{ಸಂಜಯ ಉವಾಚ ।\\
ಏವಮುಕ್ತೋ ಹೃಷೀಕೇಶೋ ಗುಡಾಕೇಶೇನ ಭಾರತ ।\\
ಸೇನಯೋರುಭಯೋರ್ಮಧ್ಯೇ ಸ್ಥಾಪಯಿತ್ವಾ ರಥೋತ್ತಮಮ್ ॥ 24 ॥\\
ಭೀಷ್ಮದ್ರೋಣಪ್ರಮುಖತಃ ಸರ್ವೇಷಾಂ ಚ ಮಹೀಕ್ಷಿತಾಮ್ ।\\
ಉವಾಚ ಪಾರ್ಥ ಪಶ್ಯೈತಾನ್ಸಮವೇತಾನ್ಕುರೂನಿತಿ ॥ 25 ॥}
\cquote{ಸಂಜಯನು ಹೇಳಿದನು,\\
ಧೃತರಾಷ್ಟ್ರನೇ, ಅರ್ಜುನನು ಹೀಗೆ ಹೇಳಿದಾಗ ಕೃಷ್ಣನು ಭೀಷ್ಮ ದ್ರೋಣರ ಮತ್ತು ಎಲ್ಲಾ ಅರಸರ ಎದುರಿಗೆ ಎರಡು ದಂಡುಗಳ ನಡುವೆ ರಥವನ್ನು ನಿಲ್ಲಿಸಿ ‘ಅರ್ಜುನನೇ ಇಲ್ಲಿ ನೆರೆದಿರುವರನ್ನು ನೋಡು’ ಎಂದು ಹೇಳಿದನು.\\}
\slcol{ತತ್ರಾಪಶ್ಯತ್ಸ್ಥಿತಾನ್ಪಾರ್ಥಃ ಪಿತೂನಥ ಪಿತಾಮಹಾನ್ ।\\
ಆಚಾರ್ಯಾನ್ಮಾತುಲಾನ್ಭ್ರಾತೂನ್ಪುತ್ರಾನ್ಪೌತ್ರಾನ್ಸಖೀಂಸ್ತಥಾ ॥ 26 ॥}
\cquote{ಅರ್ಜುನು ಅಲ್ಲಿ ನಿಂತಿರುವ ಪಿತೃತುಲ್ಯರು, ಅಜ್ಜಂದಿರು, ಗುರುಗಳು, ಸೋದರ ಮಾವಂದಿರು, ಅಣ್ಣತಮ್ಮಂದಿರು, ಮಕ್ಕಳು, ಮೊಮ್ಮಕ್ಕಳು, ಜೊತೆಗಾರರು, ಮಾವಂದಿರು, ಸ್ನೇಹಿತರು- ಹೀಗೆ ಎಲ್ಲ ಬಗೆಯ ಬಂಧುಗಳನ್ನು ಎರಡು ಕಡೆಯ ದಂಡಿನಲ್ಲಿ ಕಂಡನು.\\}
\slcol{ಶ್ವಶುರಾನ್ಸುಹೃದಶ್ಚೈವ ಸೇನಯೋರುಭಯೋರಪಿ ।\\
ತಾನ್ಸಮೀಕ್ಷ್ಯ ಸ ಕೌಂತೇಯಃ ಸರ್ವಾನ್ಬಂಧೂನವಸ್ಥಿತಾನ್ ॥ 27 ॥}
\cquote{ಹೀಗೆ ಅಲ್ಲಿ ನೆರೆದಿರುವ ಬಂಧುಗಳನ್ನೆಲ್ಲ ನೋಡಿ ಅರ್ಜುನನು ತುಂಬಾ ಕನಿಕರಗೊಂಡು ವಿಷಾದದಿಂದ ಈ ಮಾತನ್ನು ಹೇಳಿದನು.\\}
\slcol{ಕೃಪಯಾ ಪರಯಾವಿಷ್ಟೋ ವಿಷೀದನ್ನಿದಮಬ್ರವೀತ್ ।\\
ಅರ್ಜುನ ಉವಾಚ ।\\
ದೃಷ್ಟ್ವೇಮಂ ಸ್ವಜನಂ ಕೃಷ್ಣ ಯುಯುತ್ಸುಂ ಸಮುಪಸ್ಥಿತಮ್ ॥ 28 ॥\\
ಸೀದಂತಿ ಮಮ ಗಾತ್ರಾಣಿ ಮುಖಂ ಚ ಪರಿಶುಷ್ಯತಿ ।\\
ವೇಪಥುಶ್ಚ ಶರೀರೇ ಮೇ ರೋಮಹರ್ಷಶ್ಚ ಜಾಯತೇ ॥ 29 ॥}
\cquote{ಅರ್ಜುನನು ಹೇಳಿದನು,\\
ಕೃಷ್ಣ, ಕಾದುವುದಕೆಂದು ನೆರೆದಿರುವ ಈ ನನ್ನವರನ್ನು ನೋಡಿ ನನ್ನ ಅವಯವಗಳು ಸೊರುಗುತ್ತಿವೆ. ಬಾಯಿ ಒಣಗುತ್ತಿದೆ. ನನ್ನ ಮೈಯಲ್ಲಿ ನಡುಕ ಮೂಡಿ ರೋಮ ನಿಗುರಿ ನಿಂತಿದೆ.\\}
\begin{inspiration}{\kanfont ಸ್ಪೂರ್ತಿ}
ನಿಮ್ಮ ಯೋಚನೆಗಳ ಬಗ್ಗೆ ಎಚ್ಚರ ವಹಿಸಬೇಕು.ನಿಮ್ಮ ಮಾನಸಿಕ ಸ್ಥಿತಿ ನಿಮ್ಮ ದೇಹದ ಮೇಲೆ ಪರಿಣಾಮ ಬೀರುತ್ತದೆ. ಪ್ರತಿನಿತ್ಯದ ಒತ್ತಡದಿಂದ ಮನಸ್ಸನ್ನು ಸ್ವಾತಂತ್ರ್ಯಗೊಳಿಸಲು ಕೆಲವು ಸರಳ ಯೋಗದ ಮತ್ತು ಉಸಿರಾಟದ ಪ್ರಕ್ರಿಯೆಗಳು ಸಹಕಾರಿಯಾಗುತ್ತವೆ.\\
\end{inspiration}
\newpage
\begin{mananam}{\kanfont ಮನನ - \textenglish{28,29,30}}
ನನ್ನ ಜೀವನದಲ್ಲಿ ಎದುರಿಸಿದ ಭಯಂಕರವಾದ ಉದ್ವೇಗಗಳನ್ನು ಎದುರಿಸಬೇಕಾದ ಸಂದರ್ಭದಲ್ಲಿ ಪರ್ಯಾಲೋಚಿಸುತ್ತೇವೆ. ಮತ್ತು ಹೊರಗಿನ ಸನ್ನಿವೇಶಗಳಿಂದಾಗಿ ನನ್ನೊಳಗೆ ಮಿತಿಮೀರಿದವು ಇರುವಂತಾಯಿತು.ಜೀವನದ ಅಂತಹ ಸಂದರ್ಭಗಳಲ್ಲಿ ನನ್ನ ಮಾನಸಿಕ ಭಯಗಳಿಂದಾಗಿ ನನ್ನ ದೈಹಿಕ ಸ್ಥಿತಿ ಕುಂಟಿತ ವಾಯಿತೆಂಬುದನ್ನು ನಾನು ಅರಿತಿದ್ದೇನೆಯೇ? ನಾನು ನನ್ನ ಜೀವನದಲ್ಲಿನ ಉದ್ವೇಗ ಮತ್ತು ಭಯವನ್ನು ಹೇಗೆ ಎದುರಿಸಲಿ?\\
\end{mananam}
\Linepage
\newpage
\slcol{ಗಾಂಡೀವಂ ಸ್ರಂಸತೇ ಹಸ್ತಾತ್ತ್ವಕ್ಚೈವ ಪರಿದಹ್ಯತೇ ।\\
ನ ಚ ಶಕ್ನೋಮ್ಯವಸ್ಥಾತುಂ ಭ್ರಮತೀವ ಚ ಮೇ ಮನಃ ॥ 30 ॥}
\cquote{ಕೈಯಿಂದ ಗಾಂಡೀವ ಧನುಸ್ಸು ಕುಸಿಯುತ್ತಿದೆ. ಚರ್ಮವು ಸುಡುತ್ತಿದೆ. ನನಗೆ ನಿಲ್ಲುವುದಕ್ಕೂ ಆಗುವುದಿಲ್ಲ. ನನ್ನ ಮನಸ್ಸು ತಳಮಳಗೊಂಡಿದೆ.\\}
\slcol{ನಿಮಿತ್ತಾನಿ ಚ ಪಶ್ಯಾಮಿ ವಿಪರೀತಾನಿ ಕೇಶವ ।\\
ನ ಚ ಶ್ರೇಯೋऽನುಪಶ್ಯಾಮಿ ಹತ್ವಾ ಸ್ವಜನಮಾಹವೇ ॥ 31 ॥}
\cquote{ಕೃಷ್ಣ, ಕೆಟ್ಟ ಅಪಶಕುನಗಳನ್ನು ಕಾಣುತ್ತಿದ್ದೇನೆ. ಯುದ್ಧದಲ್ಲಿ ನನ್ನವರನ್ನು ಕೊಂದರೆ ಒಳ್ಳೆಯದಾದೀತೆಂದು ನನಗೆ ಅನ್ನಿಸುವುದಿಲ್ಲ.\\}
\slcol{ನ ಕಾಂಕ್ಷೇ ವಿಜಯಂ ಕೃಷ್ಣ ನ ಚ ರಾಜ್ಯಂ ಸುಖಾನಿ ಚ ।\\
ಕಿಂ ನೋ ರಾಜ್ಯೇನ ಗೋವಿಂದ ಕಿಂ ಭೋಗೈರ್ಜೀವಿತೇನ ವಾ ॥ 32 ॥}
\cquote{ಕೃಷ್ಣ, ನನಗೆ ಗೆಲ್ಲುವ ಬಯಕೆ ಇಲ್ಲ. ನನಗೆ ರಾಜ್ಯವು ಬೇಡ, ಸುಖಗಳೂ ಬೇಡ. ಗೋವಿಂದ, ಇಂಥ ರಾಜ್ಯದಿಂದಾಗಲಿ ಭೋಗದಿಂದಾಗಲಿ ಬದುಕಿನಿಂದಲೆ ಆಗಲಿ ಏನು ಪ್ರಯೋಜನ?\\}
\slcol{ಯೇಷಾಮರ್ಥೇ ಕಾಂಕ್ಷಿತಂ ನೋ ರಾಜ್ಯಂ ಭೋಗಾಃ ಸುಖಾನಿ ಚ ।\\
ತ ಇಮೇऽವಸ್ಥಿತಾ ಯುದ್ಧೇ ಪ್ರಾಣಾಂಸ್ತ್ಯಕ್ತ್ವಾ ಧನಾನಿ ಚ ॥ 33 ॥}
\cquote{ಯಾರಿಗಾಗಿ ನಾವು ರಾಜ್ಯವನ್ನೂ ಭೋಗಗಳನ್ನೂ ಸುಖಗಳನ್ನೂ ಬಯಸಿದೆವೋ, ಆ ಜನರೆಲ್ಲ ಜೀವದಾಸೆಯನ್ನೂ ಸಿರಿಯನ್ನೂ ತೊರೆದು ಇಲ್ಲಿ ಕಾದುವುದಕ್ಕೆ ನಿಂತಿದ್ದಾರೆ.\\}
\slcol{ಆಚಾರ್ಯಾಃ ಪಿತರಃ ಪುತ್ರಾಸ್ತಥೈವ ಚ ಪಿತಾಮಹಾಃ ।\\
ಮಾತುಲಾಃ ಶ್ವಶುರಾಃ ಪೌತ್ರಾಃ ಶ್ಯಾಲಾಃ ಸಂಬಂಧಿನಸ್ತಥಾ ॥ 34 ॥}
\cquote{ಗುರುಗಳು, ಪಿತೃತುಲ್ಯಯರು, ಮಕ್ಕಳು, ಅಜ್ಜಂದಿರು, ಸೋದರ ಮಾವಂದಿರು, ಮಾವಂದಿರು, ಮೊಮ್ಮಕ್ಕಳು, ಭಾವ ಮೈದುನರು, ಅದರಂತೆ ಬೇರೆ ಬೇರೆ ಸಂಬಂಧವುಳ್ಳವರು ಇಲ್ಲಿ ಎದುರು ನಿಂತಿದ್ದಾರೆ.\\}
\slcol{ಏತಾನ್ನ ಹಂತುಮಿಚ್ಛಾಮಿ ಘ್ನತೋऽಪಿ ಮಧುಸೂದನ ।\\
ಅಪಿ ತ್ರೈಲೋಕ್ಯರಾಜ್ಯಸ್ಯ ಹೇತೋಃ ಕಿಂ ನು ಮಹೀಕೃತೇ ॥ 35 ॥}
\cquote{ಕೃಷ್ಣ, ಅವರಿಂದ ನಾನು ಸತ್ತರೂ ಸರಿ. ಮೂರು ಲೋಕಗಳೇ ದೊರೆಯುವುದೆಂದರೂ ಇವರನ್ನು ಸಾಯಿಸಲಾರೆ. ಇನ್ನು ಈ ನೆಲಕ್ಕಾಗಿ ಹೊಡೆದೇನೆ?\\}
\slcol{ನಿಹತ್ಯ ಧಾರ್ತರಾಷ್ಟ್ರಾನ್ನಃ ಕಾ ಪ್ರೀತಿಃ ಸ್ಯಾಜ್ಜನಾರ್ದನ ।\\
ಪಾಪಮೇವಾಶ್ರಯೇದಸ್ಮಾನ್ಹತ್ವೈತಾನಾತತಾಯಿನಃ ॥ 36 ॥}
\cquote{ಕೃಷ್ಣ, ಕೌರವರನ್ನು ಕೊಂದು ನಮಗೇನು ತೃಪ್ತಿ? ಈ ಕೇಡಿಗಳನ್ನು ಕೊಲ್ಲುವುದರಿಂದ ನಮಗೆ ಪಾಪವೇ ಗಂಟುಬಿದ್ದೀತು.\\}
\slcol{ತಸ್ಮಾನ್ನಾರ್ಹಾ ವಯಂ ಹಂತುಂ ಧಾರ್ತರಾಷ್ಟ್ರಾನ್ಸ್ವಬಾಂಧವಾನ್ ।\\
ಸ್ವಜನಂ ಹಿ ಕಥಂ ಹತ್ವಾ ಸುಖಿನಃ ಸ್ಯಾಮ ಮಾಧವ ॥ 37 ॥}
\cquote{ಆದ್ದರಿಂದ ನಮ್ಮವರಾದ ಕೌರವರನ್ನು ನಾವು ಕೊಲ್ಲಬಾರದು, ಮಾಧವ ನಮ್ಮವರನ್ನೇ ಕೊಂದು ನಾವು ಹೇಗೆ ಸುಖಿಗಳಾಗಿರುವೆವು?\\}
\slcol{ಯದ್ಯಪ್ಯೇತೇ ನ ಪಶ್ಯಂತಿ ಲೋಭೋಪಹತಚೇತಸಃ ।\\
ಕುಲಕ್ಷಯಕೃತಂ ದೋಷಂ ಮಿತ್ರದ್ರೋಹೇ ಚ ಪಾತಕಮ್ ॥ 38 ॥}
\cquote{ಆಸೆಗೆ ಬಲಿಯಾಗಿ ಬುದ್ಧಿ ಕಳಕೊಂಡ ಈ ಜನ ಕುಲನಾಶದ ಕೆಟ್ಟ ಪರಿಣಾಮವನ್ನೂ ಗೆಳೆಯರಿಗೆ ಮೋಸ ಮಾಡಿದ ಪಾಪವನ್ನೂ ಅರ್ಥಮಾಡಿಕೊಳ್ಳುತ್ತಿಲ್ಲ, ನಿಜ.\\}
\slcol{ಕಥಂ ನ ಙ್ಞೇಯಮಸ್ಮಾಭಿಃ ಪಾಪಾದಸ್ಮಾನ್ನಿವರ್ತಿತುಮ್ ।\\
ಕುಲಕ್ಷಯಕೃತಂ ದೋಷಂ ಪ್ರಪಶ್ಯದ್ಭಿರ್ಜನಾರ್ದನ ॥ 39 ॥}
\cquote{ಆದರೆ ಓ ಜನಾರ್ಧನ, ಕುಲನಾಶದ ದುರಂತವನ್ನು ತಿಳಿದ ನಮಗೆ ಈ ಪಾಪದಿಂದ ಹಿಮ್ಮೆಟ್ಟಬೇಕೆಂದು ತಿಳಿಯದಿರುವುದು ಹೇಗೆ? \\}
\slcol{ಕುಲಕ್ಷಯೇ ಪ್ರಣಶ್ಯಂತಿ ಕುಲಧರ್ಮಾಃ ಸನಾತನಾಃ ।\\
ಧರ್ಮೇ ನಷ್ಟೇ ಕುಲಂ ಕೃತ್ಸ್ನಮಧರ್ಮೋऽಭಿಭವತ್ಯುತ ॥ 40 ॥}
\cquote{ಕುಲ ನಾಶವಾದರೆ ಬಹು ಕಾಲದಿಂದ ನಡೆದು ಬಂದ ಕುಲ ಧರ್ಮಗಳೆಲ್ಲ ಹೋಗಿ ಬಿಡುವು. ಕುಲಧರ್ಮ ಹಾಳಾದರೆ ಕುಲವನ್ನೆಲ್ಲ ಅಧರ್ಮವು ಆಕ್ರಮಿಸಿ ಬಿಡುವು.\\}
\slcol{ಅಧರ್ಮಾಭಿಭವಾತ್ಕೃಷ್ಣ ಪ್ರದುಷ್ಯಂತಿ ಕುಲಸ್ತ್ರಿಯಃ ।\\
ಸ್ತ್ರೀಷು ದುಷ್ಟಾಸು ವಾರ್ಷ್ಣೇಯ ಜಾಯತೇ ವರ್ಣಸಂಕರಃ ॥ 41 ॥}
\cquote{ಕೃಷ್ಣ, ಅಧರ್ಮದ ಆಕ್ರಮಣದಿಂದ ಕುಲೀನ ಹೆಂಗಸರು ಕೆಡುವರು. ಹೆಂಗಸರು ಕೆಟ್ಟರೆ ಸಮಾಜ ಬಣ್ಣಗೆಡುತ್ತದೆ. \\}
\begin{inspiration}{\kanfont ಸ್ಪೂರ್ತಿ}
ಜೀವನದ ಸ್ಪರ್ಧೆಗಳಿಗೆ ಎದ್ದು ನಿಲ್ಲಬೇಕು. ನಮ್ಮದೇ ಸ್ವಂತ ಜೀವನಕ್ಕಾಗಿ ಜವಾಬ್ದಾರಿಗಳನ್ನು ತೆಗೆದುಕೊಳ್ಳಬೇಕು. ನಿಷ್ಕಾರುಣ್ಯವಾಗಿ, ಎಲ್ಲಾ ಋಣಾತ್ಮಕ ಸಹವಾಸಗಳಿಂದ ಮತ್ತು ಪರಿಸರಗಳಿಂದ ದೂರವಾಗಿರಬೇಕು. ಇನ್ನೊಬ್ಬರ ಕೈಯಿಂದ ನಿಮ್ಮ ಮಾನಸಿಕ ನೆಮ್ಮದಿಯನ್ನು ಕಳೆದುಕೊಳ್ಳುವಂತಹದರ ಬಗ್ಗೆ ರಾಜಿ ಮಾಡಿಕೊಳ್ಳಬಾರದು. ಭೂತಕಾಲವನ್ನು ಹೋಗಲು ಬಿಡಬೇಕು ಮತ್ತು ವರ್ತಮಾನದಲ್ಲಿ ಉತ್ತಮವಾದದ್ದನ್ನು ಮಾಡಬೇಕು. ಉತ್ತಮವಾದ ಭವಿಷ್ಯ ನಿಮ್ಮ ಹಿಡಿತದಲ್ಲಿರುವುದು. 
\end{inspiration}
\newpage
\begin{mananam}{\kanfont ಮನನ}
ಯಾವ ಸಮಯದಲ್ಲಾದರೂ ಜವಾಬ್ದಾರಿಯ ಕೊರತೆಯಿಂದಾಗಿ ನಾನು ನನ್ನ ಕ್ರಿಯೆ ಮತ್ತು ನಿಷ್ಕ್ರಿಯೆಗಳನ್ನು ಸಮರ್ಥಿಸಿಕೊಳ್ಳುತ್ತೇನೆಯೇ? ಪೊಳ್ಳು ಅರ್ಥದ ಅನುಕಂಪದಿಂದ ನನ್ನನ್ನು ಅಧ್ಯಾತ್ಮದಿಂದ ಕೆಳಗೆ ತಳ್ಳುವವರು ಮತ್ತು ಋಣಾತ್ಮಕವಾಗಿ ಪ್ರಭಾವ ಬೀರುವವರಿಂದ ಸಂಬಂಧ ಕಡಿದುಕೊಳ್ಳುವ ಭಯ ನನಗಿದೆಯೇ? ನನ್ನ ಆಧ್ಯಾತ್ಮಿಕ ಜೀವನಕ್ಕೆ ಉಪಯೋಗವಿಲ್ಲದ ಜನರಿಗೆ ಮತ್ತು ಆಹ್ವಾನಕ್ಕೆ `ಇಲ್ಲ` ಅಥವಾ ಬೇಡ ಎಂದು ಹೇಳಲಾರದಷ್ಟು ದುರ್ಬಲನೆ ನಾನು?\\
\end{mananam}
\Linepage
\newpage


\slcol{ಸಂಕರೋ ನರಕಾಯೈವ ಕುಲಘ್ನಾನಾಂ ಕುಲಸ್ಯ ಚ ।\\
ಪತಂತಿ ಪಿತರೋ ಹ್ಯೇಷಾಂ ಲುಪ್ತಪಿಂಡೋದಕಕ್ರಿಯಾಃ ॥ 42 ॥}
\cquote{ಇಂಥ ಬೆರಕೆ ಸಮಾಜ ಕುಲವನ್ನು ಕುಲಕಂಠಕರನ್ನೂ ಜನತೆಯನ್ನು ನರಕಕ್ಕೆ ತಳ್ಳುತ್ತದೆ. ಅದರಿಂದ ಇಂಥವರಿಂದ ಹಿರಿಯರು ಪಿಂಡಪ್ರದಾನ, ಜಲತರ್ಪಣ ಇಲ್ಲದವರಾಗಿ ಕೆಳಕ್ಕೆ ಬೀಳುವರು.\\}
\slcol{ದೋಷೈರೇತೈಃ ಕುಲಘ್ನಾನಾಂ ವರ್ಣಸಂಕರಕಾರಕೈಃ ।\\
ಉತ್ಸಾದ್ಯಂತೇ ಜಾತಿಧರ್ಮಾಃ ಕುಲಧರ್ಮಾಶ್ಚ ಶಾಶ್ವತಾಃ ॥ 43 ॥}
\cquote{ಸಮಾಜದ ವ್ಯವಸ್ಥೆಯನ್ನು ಕೆಡಿಸುವ ಇಂತ ಈ ಕುಲನಾಶಕರ ದೋಷಗಳಿಂದಾಗಿ ನಿರಂತವಾಗಿ ನಡೆದು ಬಂದ ಜಾತಿಧರ್ಮಗಳೂ ಕುಲ ಧರ್ಮಗಳೂ ನಿರ್ಮೂಲವಾಗುತ್ತವೆ.\\}
\slcol{ಉತ್ಸನ್ನಕುಲಧರ್ಮಾಣಾಂ ಮನುಷ್ಯಾಣಾಂ ಜನಾರ್ದನ ।\\
ನರಕೇऽನಿಯತಂ ವಾಸೋ ಭವತೀತ್ಯನುಶುಶ್ರುಮ ॥ 44 ॥}
\cquote{ಜನಾರ್ದನ, ಕುಲಕರ್ಮಗಳನ್ನೆಲ್ಲ ಹಾಳು ಮಾಡಿಕೊಂಡ ಮನುಷ್ಯರು ಯಾವಾಗಲೂ ನರಕದಲ್ಲಿರಬೇಕಾಗುವುದೆಂದು ಕೇಳಿದ್ದುಂಟು.\\}
\slcol{ಅಹೋ ಬತ ಮಹತ್ಪಾಪಂ ಕರ್ತುಂ ವ್ಯವಸಿತಾ ವಯಮ್ ।\\
ಯದ್ರಾಜ್ಯಸುಖಲೋಭೇನ ಹಂತುಂ ಸ್ವಜನಮುದ್ಯತಾಃ ॥ 45 ॥}
\cquote{ರಾಜ್ಯದಿಂದ ಲಭಿಸುವ ಸುಖದ ಮೋಹದಿಂದ ನಮ್ಮವರನ್ನೇ ಕೊಲ್ಲ ಹೊರಟಿರುವ ನಾವು ಆಹಾ! ಎಂಥ ದೊಡ್ಡ ಪಾಪವನ್ನು ಮಾಡುವುದಕ್ಕೆ ಹೊರಟಿರುವೆವು.\\}
\slcol{ಯದಿ ಮಾಮಪ್ರತೀಕಾರಮಶಸ್ತ್ರಂ ಶಸ್ತ್ರಪಾಣಯಃ ।\\
ಧಾರ್ತರಾಷ್ಟ್ರಾ ರಣೇ ಹನ್ಯುಸ್ತನ್ಮೇ ಕ್ಷೇಮತರಂ ಭವೇತ್ ॥ 46 ॥}
\cquote{ಒಂದು ವೇಳೆ ಹೋರಾಡಬಯಸದೆ ನಿರಾಯುಧನಾಗಿ ನಿಂತ ನನ್ನನ್ನು ಆಯುಧ ಪಾಣಿಗಳಾದ ಕೌರವರು ಯುದ್ಧದಲ್ಲಿ ಕೊಂದರೆ ಅದು ನನಗೆ ಹೆಚ್ಚಿನ ಒಳ್ಳೆಯದೇ ಆದೀತು.\\}
\slcol{ಸಂಜಯ ಉವಾಚ ।\\
ಏವಮುಕ್ತ್ವಾರ್ಜುನಃ ಸಂಖ್ಯೇ ರಥೋಪಸ್ಥ ಉಪಾವಿಶತ್ ।\\
ವಿಸೃಜ್ಯ ಸಶರಂ ಚಾಪಂ ಶೋಕಸಂವಿಗ್ನಮಾನಸಃ ॥ 47 ॥ }
\cquote{ಸಂಜಯನು ಹೇಳಿದನು,\\
ದುಃಖದಿಂದ ತಳಮಳಗೊಂಡ ಅರ್ಜುನನು ಹೀಗೆ ಹೇಳಿ, ಬಿಲ್ಲು ಬಾಣಗಳನ್ನು ಕೆಳಕ್ಕೆ ಚೆಲ್ಲಿ ರಣರಂಗದಲ್ಲಿ ರಥದಲ್ಲಿ ಕುಳಿತುಬಿಟ್ಟನು.\\}
\begin{center}
{\tiny\color{brown}
ಓಂ ತತ್ಸದಿತಿ ಶ್ರೀಮದ್ಭಗವದ್ಗೀತಾಸೂಪನಿಷತ್ಸು \\
ಬ್ರಹ್ಮವಿದ್ಯಾಯಾಂ ಯೋಗಶಾಸ್ತ್ರೇ ಶ್ರೀಕೃಷ್ಣಾರ್ಜುನಸಂವಾದೇ\\
ಅರ್ಜುನವಿಷಾದಯೋಗೋ ನಾಮ ಪ್ರಥಮೋऽಧ್ಯಾಯಃ ॥1॥\\}
\end{center}
\makeatletter\@openrightfalse
%\chapter{\kanfont ೨ ಸಾಂಖ್ಯ ಯೋಗ}
\slcol{ಅರ್ಜುನ ಉವಾಚ ।\\
\Index{ಜ್ಯಾಯಸೀ ಚೇತ್ಕರ್ಮಣಸ್ತೇ} ಮತಾ ಬುದ್ಧಿರ್ಜನಾರ್ದನ ।\\
ತತ್ಕಿಂ ಕರ್ಮಣಿ ಘೋರೇ ಮಾಂ ನಿಯೋಜಯಸಿ ಕೇಶವ ॥ ೧ ॥}
\cquote{ಅರ್ಜುನನು ಹೇಳಿದನು,
ಜನಾರ್ದನಾ!  ಕರ್ಮಕ್ಕಿಂತ ಜ್ಞಾನವು ಶ್ರೇಷ್ಠವೆಂದು ನಿನ್ನ ಅಭಿಪ್ರಾಯವಾದರೆ, ಹೇ ಕೇಶವಾ!  ಭಯಂಕರವಾದ ಈ ಯುದ್ಧ ಕರ್ಮಕ್ಕೆ ನನ್ನನ್ನು ಏತಕ್ಕೆ ಪ್ರೆರೇಪಿಸುವೆ?\\}
\slcol{\Index{ವ್ಯಾಮಿಶ್ರೇಣೇವ ವಾಕ್ಯೇನ} ಬುದ್ಧಿಂ ಮೋಹಯಸೀವ ಮೇ । \\
ತದೇಕಂ ವದ ನಿಶ್ಚಿತ್ಯ ಯೇನ ಶ್ರೇಯೋಹ ಮಾಪ್ನುಯಮ್ ॥ ೨ ॥}
\cquote{ನಿನ್ನ ಸಂಧಿಗ್ಧ ವಚನಗಳಿಂದ ನನ್ನ ಬುದ್ಧಿಯು ಮೋಹಗೊಂಡಿದೆ. ನನಗೆ ಶ್ರೇಯಸ್ಸನ್ನುಂಟುಮಾಡುವ ಮಾರ್ಗ ಒಂದೇ ಒಂದನ್ನು ಖಂಡಿತವಾಗಿ ತಿಳಿಸು.}
\slcol{ಶ್ರೀ ಭಗವಾನ್ ಉವಾಚ\\
\Index{ಲೋಕೇಽಸ್ಮಿನ್  ದ್ವಿವಿಧಾ ನಿಷ್ಠಾ} ಪುರಾ ಪ್ರೋಕ್ತಾ ಮಯಾನಘ ।\\
ಜ್ಞಾನಯೋಗೇನ ಸಾಂಖ್ಯಾನಾಂ ಕರ್ಮಯೋಗೇನ ಯೋಗಿನಾಮ್ ॥ ೩ ॥}
\cquote{ಭಗವಂತನು ಹೇಳಿದನು,\\
ಅರ್ಜುನ, ಈ ಲೋಕದಲ್ಲಿ ನಾನು ಹಿಂದೆಯೇ ಎರಡು ಸ್ಥಿತಿಗಳನ್ನು ಹೇಳಿರುವೆನು. ಜ್ಞಾನಿಗಳಿಗೆ ಜ್ಞಾನ ಯೋಗ, ಸಾಧಕರಿಗೆ ಕರ್ಮ ಯೋಗ.\\}


\newpage
\begin{mananam}{\mananamfont ಮನನ ಶ್ಲೋಕ - ೧}
\small \mananamtext ಜೀವನದಲ್ಲಿ ನಾನು ಚಟುವಟಿಕೆಯಿಂದ ಇರಬೇಕೋ ಬೇಡವೋ ಎನ್ನುವ ಅನುಮಾನದಲ್ಲಿದ್ದೇನೆಯೇ? ಪ್ರಾಪಂಚಿಕ ವಿಷಯ ಮತ್ತು ವ್ಯವಹಾರಗಳಲ್ಲಿ  ನಾನು ಅರೆ ಮನಸ್ಸಿನಿಂದ ತೊಡಗಿಸಿಕೊಂಡಿದ್ದೇನೆಯೇ? ನಾನು ಸ್ವತಃಕ್ಕಾಗಿ ಏಕಾಂತತೆಯನ್ನು ಬಯಸುತ್ತೇನೆಯೇ? ಅಥವಾ ಜೀವನದ ಎಲ್ಲಾ ಚಟುವಟಿಕೆಗಳಿಂದ ದೂರವಾಗಿ, ಆಳವಾಗಿ ಆಧ್ಯಾತ್ಮಿಕತೆಯಲ್ಲಿ ತೊಡಗಿಸಿಕೊಳ್ಳಬೇಕೆಂಬ ಭಾವನೆ ಕಾಡುತ್ತದೆಯೇ? ಹಾಗಿದ್ದಲ್ಲಿ, ಜೀವನದ ಕಷ್ಟ ಮತ್ತು ಜವಾಬ್ದಾರಿಗಳನ್ನು ಎದುರಿಸಲಾರದೇ, ಅದರಿಂದ ಪಲಾಯನ ಮಾಡುವುದಕ್ಕಾಗಿ ಈ ತರಹದ ಇಚ್ಛೆ ಉತ್ಪನ್ನವಾಗಿದೆಯೇ? ಈ ನಿಷ್ಕ್ರಿಯತೆಯ ಪ್ರಲೋಭನೆಯು, ದೈಹಿಕ ಮತ್ತು ಮಾನಸಿಕ ನಿರುತ್ಸಾಹದಿಂದಾಗಿ ಉತ್ಪನ್ನವಾಗಿದೆಯೇ?
\end{mananam}
\WritingHand\enspace\textbf{ಆತ್ಮ ವಿಮರ್ಶೆ}\\
\begin{inspiration}{\mananamfont ಸ್ಫೂರ್ತಿ}
\small \mananamtext ‘ನಿಜವಾದ ಜ್ಞಾನ’ ನಮಗೆ ಸರಿಯಾದ ದಾರಿಯಲ್ಲಿ ನಡೆಯಲು ಶಕ್ತಿ ತುಂಬುತ್ತದೆ. ‘ಆತ್ಮವಿದ್ಯೆ’ಯು ನಮಗೆ, ಅಹಂನನ್ನು ಮೀರಿ ನಿಲ್ಲಲು ಸಹಾಯ ಮಾಡುತ್ತದೆ; ಆಗ ಕ್ರಿಯೆಯು (ಭೌತಿಕ, ಮಾನಸಿಕ ಹಾಗೂ ಆಧ್ಯಾತ್ಮಿಕ ಕಾರ್ಯ) ಯಾವ ಅಡಚಣೆಯೂ ಇಲ್ಲದೆ ಪ್ರವಹಿಸುತ್ತದೆ. ಹೀಗೆ ಉಂಟಾದ ಕ್ರಿಯೆಯು ತುಂಬಾ ಉತ್ಸಾಹದಾಯಕವಾಗಿದ್ದು, ನವ ಚೈತನ್ಯವನ್ನು ತುಂಬುವುದಲ್ಲದೇ, ನಮ್ಮ ಜೀವನದ ಯಾವ ಕ್ಷೇತ್ರದಲ್ಲಾದರೂ ಯಶಸ್ಸಿನ ಹಾದಿಯಲ್ಲಿ ಮುನ್ನಡೆಸುತ್ತದೆ.  
\end{inspiration}
\newpage

\newpage
\begin{mananam}{\mananamfont ಮನನ ಶ್ಲೋಕ - ೩}
\small \mananamtext ನಾನು ನನ್ನ ಪ್ರಧಾನ ವ್ಯಕ್ತಿತ್ವವನ್ನು ಆಧ್ಯಾತ್ಮಿಕ ಅಧಿಪತ್ಯದೊಳಗೆ ಹೇಗೆ ವರ್ಗೀಕರಿಸಬಲ್ಲೆ? ನನ್ನ ಸ್ವಭಾವವು ಯಾವುದನ್ನು ಅಧಿಕವಾಗಿ ಹೊಂದಿಕೊಂಡಿದೆ? ನಾನು ‘ಬಹಿರ್ಮುಖಿ’ ಹಾಗೂ ಸದಾ ಕಾರ್ಯ ತತ್ಪರನೇ? ನನ್ನಲ್ಲಿ ಪರರ ಸೇವೆ ಮಾಡುವ  ಇಚ್ಛೆ ಇದೆಯೇ?  ನನ್ನ ಅಂತಾರಾಳದ ಅಭಿವೃದ್ಧಿಗಾಗಿ, “ಹೊರಗಿನ ಪ್ರಪಂಚದಲ್ಲಿ ಬದಲಾವಣೆಗಳನ್ನು ತರಬೇಕು” ಎಂಬುದು ನನಗೆ ಅಗತ್ಯವೆನಿಸುತ್ತದೆಯೇ? ಅಥವಾ ನಾನು ಅಂತರ್ಮುಖಿಯಾಗಿ ಹಾಗೂ ಚಿಂತನಾಶೀಲನಾಗಿರುವೆನೇ? ನನ್ನ ಅಂತಾರಾಳದಲ್ಲಿ ಮುಳುಗಿ, ಆತ್ಮ ಹಾಗೂ ದೈವೀಕ ಅನ್ವೇಷಣೆಯಲ್ಲಿ ಆಳವಾದ ಅಧ್ಯಯನದಲ್ಲಿ ಪ್ರವೃತ್ತನಾಗಲು, ನಾನು ಸಿದ್ಧನಿದ್ದೇನೆಯೇ? ಇವೆರಡರ ಮಧ್ಯೆ (ಆತ್ಮ ಹಾಗೂ ದೈವೀಕ ಅನ್ವೇಷಣೆ) ಸಮತೋಲನ ಕಾಪಾಡಿದಾಗ ಅದು ಏನನ್ನು ಪ್ರತಿಪಾದಿಸುತ್ತದೆ? ಅಥವಾ, ಅದನ್ನು ಏನೆಂದು ತಿಳಿಯಬಹುದು?
\end{mananam}
\WritingHand\enspace\textbf{ಆತ್ಮ ವಿಮರ್ಶೆ}\\
\begin{inspiration}{\mananamfont ಸ್ಫೂರ್ತಿ}
\small \mananamtext  ಆಧ್ಯಾತ್ಮಕ ದಾರಿಯಲ್ಲಿ ವಿಶಾಲವಾದ ವರ್ಗೀಕರಣವು, ನಿರ್ದಿಷ್ಟವಾದ ನಮ್ಮದೇ ಜೀವನ ಪಥಗಳನ್ನು ಕೆತ್ತುವಲ್ಲಿ ಸಹಾಯ ಮಾಡುವುದಕ್ಕಾಗಿ ಇದೆ. ನಮ್ಮ ಪ್ರತಿ ಒಬ್ಬರೊಳಗೂ ವಿವಿಧ ರೀತಿಯ ವ್ಯಕ್ತಿತ್ವಗಳಿದ್ದು, ಪ್ರಗತಿ ಹೊಂದಲು, ಈ ವ್ಯಕ್ತಿತ್ವಗಳಲ್ಲಿ ಅಗತ್ಯವಾದ ಸಮಚಿತ್ತತೆ ಕಾಪಾಡಿಕೊಳ್ಳಬೇಕಾಗುತ್ತದೆ. ಒಬ್ಬರಿಗೆ ಬೇಕಾಗಿರುಬಹುದಾದ ಸಮಚಿತ್ತತೆಯು, ಮತ್ತೊಬ್ಬರಿಗೆ ಆಗಿಬರದಿರಬಹುದು.
\end{inspiration}
\newpage

\slcol{\Index{ನ ಕರ್ಮಣಾಮನಾರಂಭಾ}ನ್ನೈಷ್ಕರ್ಮ್ಯಂ ಪುರುಷೋಽಶ್ನುತೇ ।\\
ನ ಚ ಸಂನ್ಯಸನಾದೇವ ಸಿದ್ಧಿಂ ಸಮಧಿಗಚ್ಛತಿ ॥ ೪ ॥}
\cquote{ಕೆಲಸ ಮಾಡದೆ ಇರುವುದರಿಂದ ಕರ್ಮ ಸಂಬಂಧವನ್ನು ತೊಡೆದುಹಾಕಲಾಗುವುದಿಲ್ಲ. ಕರ್ಮಗಳನ್ನು ಬಿಟ್ಟ ಮಾತ್ರದಿಂದಲೇ ಸಿದ್ಧಿಯನ್ನು ಪಡೆಯುವುದು ಸಾಧ್ಯವಿಲ್ಲ.}
\slcol{\Index{ನ ಹಿ ಕಶ್ಚಿತ್ಕ್ಷಣಮಪಿ} ಜಾತು ತಿಷ್ಠತ್ಯಕರ್ಮಕೃತ್ ।\\
ಕಾರ್ಯತೇ ಹ್ಯವಶಃ ಕರ್ಮ ಸರ್ವಃ ಪ್ರಕೃತಿಜೈರ್ಗುಣೈಃ ॥ ೫ ॥} 
\cquote{ಕ್ಷಣಮಾತ್ರವೂ ಕರ್ಮವನ್ನು ಮಾಡದೆ ಯಾರೂ ಇರಲಾರರು. ಎಲ್ಲರೂ ತಮ್ಮ ಹುಟ್ಟು ಗುಣಗಳಿಂದ ತಮಗರಿವಿಲ್ಲದಂತೆ ಕರ್ಮ ಮಾಡುತ್ತಲೇ ಇರುತ್ತಾರೆ.}
\slcol{\Index{ಕರ್ಮೇಂದ್ರಿಯಾಣಿ ಸಂಯಮ್ಯ} ಯ ಆಸ್ತೇ ಮನಸಾ ಸ್ಮರನ್ ।\\
ಇಂದ್ರಿಯಾರ್ಥಾನ್ವಿಮೂಢಾತ್ಮಾ ಮಿಥ್ಯಾಚಾರಃ ಸ ಉಚ್ಯತೇ ॥ ೬ ॥}
\cquote{ಯಾವನು ಕರ್ಮೇoದ್ರಿಯಗಳನ್ನು ಬಿಗಿಹಿಡಿದು ಇಂದ್ರಿಯಗಳು ಬಯಸುವ ವಿಷಯಗಳನ್ನು ಮನಸ್ಸಿನಲ್ಲಿ ಹಂಬಲಿಸುತ್ತಿರುವನೋ ಆ ಮೂಢನು ಸುಳ್ಳು ನಟನೆಯ ಡಂಭಾಚಾರಿಯೆನಿಸುವನು.}
\slcol{\Index{ಯಸ್ತ್ವಿಂದ್ರಿಯಾಣಿ ಮನಸಾ} ನಿಯಮ್ಯಾರಭತೇಽರ್ಜುನ ।\\
ಕರ್ಮೇಂದ್ರಿಯೈಃ ಕರ್ಮಯೋಗಮಸಕ್ತಃ ಸ ವಿಶಿಷ್ಯತೇ ॥ ೭ ॥}
\cquote{ಅರ್ಜುನ, ಯಾವನು ಇಂದ್ರಿಯಗಳನ್ನು ಮನಸ್ಸಿನಿಂದ ಬಿಗಿಹಿಡಿದು ಫಲದಾಸಕ್ತಿ ಇಲ್ಲದೆ ಕರ್ಮೇಂದ್ರಿಯಗಳಿಂದ ಕೆಲಸದಲ್ಲಿ ತೊಡಗಿರುವನೋ ಅವನು ಉತ್ತಮನು.}
\slcol{\Index{ನಿಯತಂ ಕುರು ಕರ್ಮ ತ್ವಂ} ಕರ್ಮ ಜ್ಯಾಯೋ ಹ್ಯಕರ್ಮಣಃ ।\\
ಶರೀರಯಾತ್ರಾಪಿ ಚ ತೇ ನ ಪ್ರಸಿದ್ಧ್ಯೇದಕರ್ಮಣಃ ॥ ೮ ॥}
\cquote{ನೀನು ಮಾಡತಕ್ಕದ್ದೆಂದು ಗೊತ್ತಾಗಿರುವ ಕೆಲಸವನ್ನು ಮಾಡು. ಏನೂ ಮಾಡದೇ ಇರುವುದಕ್ಕಿಂತ, ಮಾಡುವುದು ಮೇಲು. ನೀನು ಯಾವ ಕರ್ಮವನ್ನೂ ಮಾಡದೇ ಇದ್ದರೆ, ಬದುಕುವುದೇ  ಆಗಲಾರದು.}

\newpage
\begin{mananam}{\mananamfont {ಮನನ ಶ್ಲೋಕ - ೫, ೬}}
\small \mananamtext ಹದಿಹರೆಯದವರಾಗಿ, ವಿದ್ಯಾಭ್ಯಾಸ ಬಿಟ್ಟವರಾಗಿದ್ದರೆ; ಯುವಕನಾಗಿ, ಉದ್ಯೋಗವನ್ನು ಹುಡುಕದಿದ್ದರೆ ಅಥವಾ, ಸಕ್ರಿಯವಾಗಿ ಯಾವುದೇ ಉದ್ಯೋಗದಲ್ಲೂ ತೊಡಗಿಸಿಕೊಳ್ಳದಿದ್ದರೆ, ಪ್ರೌಢ ವಯಸ್ಕನಾಗಿ, ಸಂಸಾರದ ಜವಾಬ್ದಾರಿಗಳನ್ನು ಪೂರ್ಣಗೊಳಿಸುವವನಾಗಿಲ್ಲದಿದ್ದರೆ ಅಥವಾ ಸೃಜನಾತ್ಮಕ ಕ್ರಿಯೆಯನ್ನು ತಪ್ಪಿಸಿಕೊಳ್ಳುವವನಾಗಿದ್ದಾರೆ, ಎಚ್ಚರ! ನೀವು, ಪ್ರಕೃತಿಯ ಸಹಜ ಹರಿವಿಗೆ ವಿರುದ್ಧವಾಗಿ (ಅಂದರೆ,ಪ್ರಗತಿಗೆ ವಿರುದ್ಧವಾಗಿ) ಕಾರ್ಯದಲ್ಲಿ ತೊಡಗಿದ್ದೀರಿ ಎಂಬ ಅರಿವಿರಲಿ! ಹಾಗಿದ್ದಲ್ಲಿ,  ನಿಮ್ಮ ಶಕ್ತಿಯನ್ನು ಎಲ್ಲಿ ಮತ್ತು ಹೇಗೆ ಹರಿಯಬಿಟ್ಟಿದ್ದೀರಿ? ನಿಮ್ಮ ಜೀವನದಲ್ಲಿ ಕ್ರಿಯೆಯನ್ನು ಮಾಡದಿರುವ  ನಿಮ್ಮ ಇಂದಿನ ಆಯ್ಕೆಗಳು,  ಮಾನಸಿಕ ಅಥವಾ ಭಾವನಾತ್ಮಕ ಕಳವಳಕ್ಕೆ ಕಾರಣವಾಗುತ್ತಿವೆಯೇ? ನೀವು ಮಾನಸಿಕ ಭ್ರಮೆಯಲ್ಲಿ ಕಳೆದು ಹೋಗಿದ್ದೀರಿಯೇ? ನಿಮ್ಮ ಜೀವನದಲ್ಲಿ ಸಕಾರಾತ್ಮಕ ಬದಲಾವಣೆ ತರಲು, ನೀವು ಏನಾದರೂ ಸೃಜನಾತ್ಮಕವಾದ ಕಾರ್ಯವನ್ನು ಮಾಡುವ ವಿಚಾರ ಮಾಡಿದ್ದೀರಿಯೆ?
\end{mananam}
\WritingHand\enspace\textbf{ಆತ್ಮ ವಿಮರ್ಶೆ}\\
\begin{inspiration}{\mananamfont ಸ್ಫೂರ್ತಿ}
\small \mananamtext ಒಬ್ಬರು ನಿಶ್ಕ್ರಿಯವಾಗಿದ್ದಲ್ಲಿ ಅಥವಾ ಸೋಮಾರಿಯಾಗಿದ್ದಲ್ಲಿ, ಅವರು ತಮ್ಮ ಇಂದ್ರಿಯಗಳನ್ನು ಹತೋಟಿಯಲ್ಲಿಟ್ಟಿದ್ದಾರೆಂದು ಅರ್ಥವಲ್ಲ. ತರಬೇತಿ ಪಡೆಯದ ಮನಸ್ಸು ನಿಜವಾಗಿಯೂ ಇಂದ್ರಿಯಗಳನ್ನು ತಡೆಯಲು ಸಾಧ್ಯವಿಲ್ಲ. ಒಬ್ಬರು ತಮ್ಮ ಮನಸ್ಸು ಮತ್ತು ಇಂದ್ರಿಯಗಳನ್ನು ಹತೋಟಿಯಲ್ಲಿಡಲು ಕಲಿತಾಗ ಅವರ ಕ್ರಿಯೆಗಳು ಹತೋಟಿಗೆ ಬರಲು ಪ್ರಾರಂಭಿಸುತ್ತವೆ. ಇಂತಹ ಸಂದರ್ಭಗಳಲ್ಲಿ, ನಿಶ್ಕ್ರಿಯವಾಗಿರುವುದಕ್ಕಿಂತ, ಫಲದಾಯಕವಾದ, ಕಾರ್ಯ ಕಲಾಪಗಳನ್ನು ಮಾಡುವುದಕ್ಕೆ ಇಂದ್ರಿಯಗಳನ್ನು ತರಬೇತಿಗೆ ಒಳಪಡಿಸುವುದು ಸೂಕ್ತ; ಇದರಿಂದ ಒಬ್ಬರನ್ನು, ಮಾನಸಿಕವಾಗಿಯೂ, ಶಾರೀರಿಕವಾಗಿಯೂ ಋಣಾತ್ಮಕತೆಯಿಂದ, ಧನಾತ್ಮಕತೆಗೆಡೆಗೆ ಮೆಲ್ಲನೆ ಎಳೆಯಬಹುದು.
\end{inspiration}
\newpage

\begin{mananam}{\mananamfont {ಮನನ ಶ್ಲೋಕ - ೭, ೮}}
\small \mananamtext ನನ್ನ ಉದ್ಯೋಗದಲ್ಲಿ ಹಾಗೂ ದೈನಂದಿನ ಕಾರ್ಯಗಳಲ್ಲಿ, ನಾನು ಕರ್ಮ ಯೋಗವನ್ನು ಹೇಗೆ ಅಭ್ಯಸಿಸಲಿ? (ಫಲಿತಾಂಶದ ಮೇಲಿನ ಮೋಹವನ್ನು ಬಿಟ್ಟ ನಮ್ಮ ಕ್ರಿಯೆಗಳು) ನಾನು ಪ್ರತಿಫಲದ ಪ್ರೇರಣೆ ಇಲ್ಲದೆ ಕೆಲಸ ಮಾಡಬಲ್ಲೆನೆ? ಅಥವಾ ಯಾವದೇ ಒಂದು ಕೆಲಸದ ಬಗ್ಗೆ ಒಂದು ನಿರ್ದಿಷ್ಟ ಗುರಿ ಇಲ್ಲದಿದ್ದಲ್ಲಿ, ನನಗೆ ಚೈತನ್ಯ ಮತ್ತು ಉತ್ಸಾಹದ ಕೊರತೆ ಇದೆ ಎಂಬ ಭಾವನೆ ಬರುತ್ತದೆಯೇ? ನನಗೆ, ಆಯಾ ಕ್ಷಣಗಳ ಬೇಡಿಕೆಗೆ ತಕ್ಕಂತೆ, ಕಾರ್ಯನಿರ್ವಹಿಸುವ ತಿಳುವಳಿಕೆ ಇದೆಯೇ? ಹಾಗೂ, ಪ್ರತಿಯೊಂದೂ ಕಷ್ಟಕರವಾದ ಕಾರ್ಯವನ್ನೂ, ಸಣ್ಣ ಸಣ್ಣ ವಿಭಾಗಗಳಾಗಿ ವಿಂಗಡಿಸಿ, ಸುಲಭವಾಗುವಂತೆ ಕೆಲಸ ನಿರ್ವಹಿಸಲು ತಿಳಿಯುವುದೇ? ಎಷ್ಟೇ ಸಣ್ಣ ಕೆಲಸವಾಗಲಿ ಅಥವಾ ನಾನು ನಿಕೃಷ್ಟವೆಂದು ತಿಳಿದ ಕಾರ್ಯವಿರಲಿ, ಅದರಲ್ಲಿ ನಾನು ಶತ ಪ್ರತಿಶತ ಮನಸ್ಸನ್ನು ತೊಡಗಿಸಿ ಕೆಲಸ ಮಾಡಬಲ್ಲೆನೇ?
\end{mananam}
\WritingHand\enspace\textbf{ಆತ್ಮ ವಿಮರ್ಶೆ}\\
\begin{inspiration}{\mananamfont ಸ್ಫೂರ್ತಿ}
\small \mananamtext  ಕರ್ಮಯೋಗದ ಒಳ ತಿರುಳು, ಸಾಧನೆಯ ಬಗ್ಗೆ  ಕೇಂದ್ರೀಕೃತವಾಗಿರುವುದೇ ಹೊರತು, ಅದರ ಪ್ರತಿಫಲ ಅಥವಾ ಗುರಿಯ ಕಡೆಗೆ ಇರುವುದಲ್ಲ. ಒಬ್ಬರು, ಕಾರ್ಯವಿಧಾನವನ್ನು ಸರಿಯಾಗಿ ಅನುಸರಿಸಿದಲ್ಲಿ, ಅದಕ್ಕೆ ತಕ್ಕ ಪ್ರತಿಫಲ ಕೊನೆಯಲ್ಲಿ  ತನ್ನಷ್ಟಕ್ಕೆ ತಾನೇ ಸಿಗುವುದು. ಯಾವುದೇ ನಿರ್ದಿಷ್ಟ ಆಸೆಯ ಪ್ರೆರೇಪಣೆ ಇಲ್ಲದೆಲೇ, “ಕೆಲಸವನ್ನು, ಕೆಲಸದ ಸಲುವಾಗಿ”ಯಷ್ಟೇ ಮಾಡುವುದು, ತುಂಬಾ ಆರೋಗ್ಯಕರ ಮತ್ತು ಉತ್ತಮ ಮಟ್ಟದ ಕಾರ್ಯವಾಗಿರುತ್ತದೆ. ಇದರಿಂದ, ಮನಸ್ಸಿನ ಸಮತೋಲನ ಹಾಗೂ ಶಾಂತಿ ವೃದ್ಧಿಸುತ್ತದೆ. ಕ್ರಿಯಾತ್ಮಕ ಚಟುವಟಿಕೆಯು, ದುರ್ಬಲಗೊಳಿಸುವ ಸೋಮಾರಿತನಕ್ಕಿಂತ ಉತ್ತಮ ಎಂಬುದನ್ನು ಮನಗಾಣಬೇಕು. 
\end{inspiration}
\newpage

\slcol{\Index{ಯಜ್ಞಾರ್ಥಾತ್ಕರ್ಮಣೋಽನ್ಯತ್ರ} ಲೋಕೋಽಯಂ ಕರ್ಮಬಂಧನಃ ।\\
ತದರ್ಥಂ ಕರ್ಮ ಕೌಂತೇಯ ಮುಕ್ತಸಂಗಃ ಸಮಾಚರ ॥ ೯ ॥}
\cquote{ಈಶ್ವರನಿಗೆ ಪ್ರೀತಿಯಾಗಲೆಂದಲ್ಲದೆ ಸ್ವಾರ್ಥಕ್ಕಾಗಿ ಮಾಡುವ ಕರ್ಮಗಳಿಂದ ಈ ಲೋಕದ ಕಟ್ಟಿಗೆ ಒಳಪಡಬೇಕಾಗುವುದು. ಅರ್ಜುನ, ಫಲವನ್ನು ಬಯಸದೆ ಈಶ್ವರನಿಗೆ ಪ್ರೀತಿಯಾಗಲೆಂದು ಕರ್ಮವನ್ನು ನಡೆಸು.}
\slcol{\Index{ಸಹಯಜ್ಞಾಃ ಪ್ರಜಾಃ} ಸೃಷ್ಟ್ವಾ ಪುರೋವಾಚ ಪ್ರಜಾಪತಿಃ ।\\
ಅನೇನ ಪ್ರಸವಿಷ್ಯಧ್ವಮೇಷ ವೋಽಸ್ತ್ವಿಷ್ಟಕಾಮಧುಕ್ ॥ ೧೦ ॥} 
\cquote{ಮೊದಲು ಪ್ರಜಾಪತಿ ಬ್ರಹ್ಮನು ಯಜ್ಞಗಳೊಂದಿಗೆ ಪ್ರಜೆಗಳನ್ನು ಸೃಷ್ಟಿಸಿ “ಈ ಯಜ್ಞದಿಂದ ನೀವು ಸಮೃದ್ಧಿಯನ್ನು ಪಡೆಯಿರಿ, ಇದು ನಿಮ್ಮ ಇಷ್ಟಾರ್ಥಗಳನ್ನು ದೊರಕಿಸುವ ಕಾಮಧೇನುವು” ಎಂದು ಹೇಳಿದನು.}
\slcol{\Index{ದೇವಾನ್ಭಾವಯತಾನೇನ ತೇ} ದೇವಾ ಭಾವಯಂತು ವಃ ।\\
ಪರಸ್ಪರಂ ಭಾವಯಂತಃ ಶ್ರೇಯಃ ಪರಮವಾಪ್ಸ್ಯಥ ॥ ೧೧ ॥}
\cquote{ಇದರಿಂದ ದೇವತೆಗಳನ್ನು ಸಂತೋಷಗೊಳಿಸಿರಿ. ಆ ದೇವತೆಗಳು ನಿಮ್ಮನ್ನು ಸಂತೋಷಗೊಳಿಸಲಿ. ಒಬ್ಬರನ್ನೊಬ್ಬರು ಸಂತೋಷಗೊಳಿಸುವರಾಗಿ ಹೆಚ್ಚಿನ ಯಶಸ್ಸನ್ನು ಪಡೆಯಿರಿ.}
\slcol{\Index{ಇಷ್ಟಾನ್ಭೋಗಾನ್ಹಿ ವೋ} ದೇವಾ ದಾಸ್ಯಂತೇ ಯಜ್ಞಭಾವಿತಾಃ ।\\
ತೈರ್ದತ್ತಾನಪ್ರದಾಯೈಭ್ಯೋ ಯೋ ಭುಂಕ್ತೇ ಸ್ತೇನ ಏವ ಸಃ ॥ ೧೨ ॥}
\cquote{ಯಜ್ಞಗಳಿಂದ ಸಂತೋಷಗೊಂಡ ದೇವತೆಗಳು ನಿಮಗೆ ಬಯಸಿದ್ದನ್ನೆಲ್ಲಾ ಕೊಡುತ್ತಾರೆ. ಅವರು ಕೊಟ್ಟಿದ್ದನ್ನು ಅವರಿಗೆ ನಿವೇದಿಸದೆ ಯಾವನು ಉಣ್ಣುತ್ತಾನೋ ಅವನು ಕಳ್ಳನೇ ಸರಿ.}
\slcol{\Index{ಯಜ್ಞಶಿಷ್ಟಾಶಿನಃ ಸಂತೋ} ಮುಚ್ಯಂತೇ ಸರ್ವಕಿಲ್ಬಿಷೈಃ ।\\
ಭುಂಜತೇ ತೇ ತ್ವಘಂ ಪಾಪಾ ಯೇ ಪಚಂತ್ಯಾತ್ಮಕಾರಣಾತ್ ॥ ೧೩ ॥}
\cquote{ಯಜ್ಞಗಳನ್ನು ಮಾಡಿ ಉಳಿದದ್ದನ್ನು ಉಣ್ಣುವವರು ಎಲ್ಲ ಪಾಪಗಳಿಂದಲೂ ಬಿಡುಗಡೆಯನ್ನು ಹೊಂದುತ್ತಾರೆ. ಯಾರು ತಮ್ಮ ಹೊಟ್ಟೆಗಾಗಿಯೇ ಬೇಯಿಸಿಕೊಳ್ಳುವರೋ, ಆ ಪಾಪಿಗಳು ಪಾಪವನ್ನು ಉಣ್ಣುತ್ತಾರೆ.}

\newpage
\begin{mananam}{\mananamfont ಮನನ ಶ್ಲೋಕ - ೯}
\small \mananamtext ತ್ಯಾಗದ್ಯೋತಕವಾದ ಯಜ್ಞದಂತೆ, ನನ್ನ ಎಲ್ಲಾ ಕೆಲಸಗಳಿಗೂ  ಪ್ರಾಧಾನ್ಯತೆ ಕೊಟ್ಟು ಹಾಗೂ ಒಂದು ಸಮರ್ಪಣಾ ಮನೋಭಾವದಿಂದ ಕೆಲಸ ಮಾಡಲು ನಾನು ಕಲಿಯಬಲ್ಲೆನೇ? ಅಥವಾ ಕೆಲಸ ಪ್ರಾರಂಭಿಸುವ ಮೊದಲೇ, ಕೆಲಸಗಳನ್ನು ಇಷ್ಟ, ಅನಿಷ್ಟವೆಂದು ವಿಂಗಡಿಸುತ್ತಾ ಇರುತ್ತೇನೆಯೇ? ಫಲಿತಾಂಶ ಮತ್ತು ಪ್ರತಿಫಲಗಳ ಬಗೆಗಿನ ವಿಚಾರಗಳ ಪ್ರಚೋದನೆಯಿಂದಾಗಿ, ನನ್ನನ್ನು ನಾನು ಕಾರ್ಯಗಳಲ್ಲಿ ತೊಡಗಿಸಿಕೊಳ್ಳುತ್ತೇನೆಯೇ? ನಾನು, ನನ್ನನ್ನು ಕೆಲಸದಲ್ಲಿ ತೊಡಗಿಸಿಕೊಂಡಾಗ, ನಾನು, ಆ ಕ್ಷಣದ ಕೆಲಸದಲ್ಲಿ ಮನವನ್ನು ಕೇಂದ್ರೀಕರಿಸುತ್ತೇನೆಯೇ ಅಥವಾ, ಅದರ ಮುಂದಿನ ಫಲಿತಾಂಶ ಅಥವಾ ಪ್ರತಿಫಲದ  ಬಗ್ಗೆಯೇ? ನಾನು, ನನ್ನ ಕೆಲಸವನ್ನು ಪೂರ್ತಿಗೊಳಿಸಿದ ಮೇಲೆ, ನನ್ನ ಕೆಲಸದ ಫಲಿತಾಂಶದಿಂದ ಸಂತೋಷ ಅಥವಾ ದುಃಖವನ್ನು ಅನುಭವಿಸುತ್ತೇನೆಯೇ?
\end{mananam}
\WritingHand\enspace\textbf{ಆತ್ಮ ವಿಮರ್ಶೆ}\\
\begin{inspiration}{\mananamfont ಸ್ಫೂರ್ತಿ}
\small \mananamtext ಪ್ರಾಚೀನ ಕಾಲದ ವೇದಗಳಲ್ಲಿ ತಿಳಿಸಿರುವ ಯಜ್ಞವು, ಪುರೋಹಿತರು, ಆರಾಧಕರು ಚಿರಸ್ಮರಣೀಯಗೊಳಿಸಿದಂತೆ, ತಾನು ಅರ್ಪಿಸುವ (ಯಜ್ಞಕ್ಕೆ) ಪ್ರಾಪಂಚಿಕ ವಸ್ತುಗಳ ಆಹುತಿಗಳು, ಮೌಲ್ಯಯುತವಾದ ಲಾಭ ಮತ್ತು ಉತ್ತಮವಾದ ಕರ್ಮ ಹಾಗೂ ದಿವ್ಯವಾದ (ಸ್ವರ್ಗೀಯ) ಫಲವನ್ನು ಪಡೆಯುವುದಕ್ಕೋಸ್ಕರವೇ ಇರುವುದಾಗಿದೆ. ಈ ತರಹದ ಯಜ್ಞವು, ಒಂದನ್ನು ಪಡೆಯಲು ಇನ್ನೊಂದನ್ನು ಕೊಡುವ, ಕೇವಲ ವ್ಯಾಪಾರದಂತಾಗುತ್ತದೆ;  ‘ಕರ್ಮ ಯೋಗ’ವು ಇಂಥಹ ಮಾನಸಿಕ ಸ್ಥಿತಿಯನ್ನು ರೂಪಾoತರಗೊಳಿಸಿ ‘ಕೇವಲ ನೀಡುವ ಕ್ರಿಯೆ’ಯ ಮೌಲ್ಯವನ್ನು ತಿಳಿಸಿಕೊಡುತ್ತದೆ. ಹಾಗೂ ಸಮರ್ಪಣಾ ಭಾವದಿಂದ ಮಾಡಿದ ಕ್ರಿಯೆಗಳಿಗೆ ಯಾವ ಕರ್ಮವೂ ಅಂಟುವುದಿಲ್ಲ.
\end{inspiration}
\newpage


\slcol{\Index{ಅನ್ನಾದ್ಭವಂತಿ ಭೂತಾನಿ} ಪರ್ಜನ್ಯಾದನ್ನಸಂಭವಃ ।\\
ಯಜ್ಞಾದ್ಭವತಿ ಪರ್ಜನ್ಯೋ ಯಜ್ಞಃ ಕರ್ಮಸಮುದ್ಭವಃ ॥ ೧೪ ॥} 
\cquote{ಅನ್ನದಿಂದ ಜೀವಿಗಳು ಹುಟ್ಟುತ್ತವೆ. ಮಳೆಯಿಂದ ಅನ್ನ ಹುಟ್ಟುತ್ತದೆ. ಯಜ್ಞದಿಂದ ಮಳೆ ಉಂಟಾಗುತ್ತದೆ. ಯಜ್ಞವು ಕರ್ಮದಿಂದ ನಡೆಯುವುದು.}
\slcol{\Index{ಕರ್ಮ ಬ್ರಹ್ಮೋದ್ಭವಂ} ವಿದ್ಧಿ ಬ್ರಹ್ಮಾಕ್ಷರಸಮುದ್ಭವಮ್ ।\\
ತಸ್ಮಾತ್ಸರ್ವಗತಂ ಬ್ರಹ್ಮ ನಿತ್ಯಂ ಯಜ್ಞೇ ಪ್ರತಿಷ್ಠಿತಮ್ ॥ ೧೫ ॥}
\cquote{ಎಲ್ಲ ಕರ್ಮಗಳ ಮೂಲ, ಭಗವಂತ. ವೇದಾಕ್ಷರಗಳಿಂದ ಭಗವಂತನ ಅಭಿವ್ಯಕ್ತಿ. ವೇದಾಕ್ಷರಗಳನ್ನು ಜೀವಿಗಳು ಉಚ್ಚರಿಸುತ್ತವೆ. ಆದ್ದರಿಂದ ಎಲ್ಲೆಡೆಯೂ ತುಂಬಿರುವ ಭಗವಂತನು ಸರ್ವದಾ ಯಜ್ಞದಲ್ಲಿ ಸದಾ ಪ್ರತಿಷ್ಠಿತನಾಗಿದ್ದಾನೆ.}
\slcol{\Index{ಏವಂ ಪ್ರವರ್ತಿತಂ ಚಕ್ರಂ} ನಾನುವರ್ತಯತೀಹ ಯಃ ।\\
ಅಘಾಯುರಿಂದ್ರಿಯಾರಾಮೋ ಮೋಘಂ ಪಾರ್ಥ ಸ ಜೀವತಿ ॥ ೧೬ ॥}
\cquote{ಅರ್ಜುನ, ಹೀಗೆ ಪ್ರವೃತ್ತವಾದ ಈ ಜೀವನ ಚಕ್ರವನ್ನು ಯಾವನು ಮುಂದುವರಿಸುವುದಿಲ್ಲವೋ ಅವನು ಪಾಪದ ಬಾಳಿನವನೂ ಇಂದ್ರಿಯಗಳೊಡನೆ ವಿನೋದಿಸುವವನೂ ಆಗುವುದರಿಂದ ಅವನ ಬದುಕು ವ್ಯರ್ಥ.}
\slcol{\Index{ಯಸ್ತ್ವಾತ್ಮರತಿರೇವ} ಸ್ಯಾದಾತ್ಮತೃಪ್ತಶ್ಚ ಮಾನವಃ ।\\
ಆತ್ಮನ್ಯೇವ ಚ ಸಂತುಷ್ಟಸ್ತಸ್ಯ ಕಾರ್ಯಂ ನ ವಿದ್ಯತೇ ॥ ೧೭ ॥}
\cquote{ಯಾವ ಮನುಷ್ಯನು ಪರಮಾತ್ಮನಲ್ಲಿಯೇ ಪ್ರೇಮವುಳ್ಳವನಾಗಿ ಪರಮಾತ್ಮನಿಂದ ತೃಪ್ತನಾಗಿ ಆತನಲ್ಲಿಯೇ ಸಂತೋಷಗೊಳ್ಳುತ್ತಿರುವನೋ ಅವನು ಮಾಡಬೇಕಾದದ್ದೇನೂ ಇಲ್ಲ.}
\slcol{\Index{ನೈವ ತಸ್ಯ ಕೃತೇನಾರ್ಥೋ} ನಾಕೃತೇನೇಹ ಕಶ್ಚನ ।\\
ನ ಚಾಸ್ಯ ಸರ್ವಭೂತೇಷು ಕಶ್ಚಿದರ್ಥವ್ಯಪಾಶ್ರಯಃ ॥ ೧೮ ॥}
\cquote{ಅವನಿಗೆ ಮಾಡಿದ್ದರಿಂದಲೂ ಪ್ರಯೋಜನವಿಲ್ಲ, ಬಿಟ್ಟಿದ್ದರಿಂದ ಈ ಲೋಕದಲ್ಲಿ ಯಾವ ಹಾನಿಯೂ ಇಲ್ಲ.ಅವನಿಗೆ ಪ್ರಪಂಚದ ಯಾವ ಜೀವಿಯಿಂದಲೂ ಯಾವ ಪ್ರಯೋಜನದ ಅಪೇಕ್ಷೆಯೂ ಇಲ್ಲ.}

\newpage
\begin{mananam}{\mananamfont ಮನನ ಶ್ಲೋಕ - ೧೫}
\small \mananamtext ಈ ದೇಹಕ್ಕೆ ಆಧಾರ ಈ ಜಗತ್ತು, ಹಾಗಿದ್ದಲ್ಲಿ, ನಾವು ಪಡೆದದ್ದೆಲ್ಲವನ್ನೂ ಈ ಜಗತ್ತಿಗೆ ಮರಳಿ ಸಲ್ಲಿಸುವುದು ಸೂಕ್ತವಲ್ಲವೇ?  ನಾನು ತ್ಯಾಗದ ಮನೋಸ್ಥಿತಿಯನ್ನು ಹೇಗೆ ಪೋಷಿಸಲಿ ಮತ್ತು ಧರ್ಮಸಮ್ಮತವಾದ ಕಾರ್ಯಗಳಲ್ಲಿ ಹೇಗೆ ಮುಳುಗಲಿ? ನಾನು ಮಾಡುವ ಸಮಸ್ತ ಕೆಲಸ ಕಾರ್ಯಗಳನ್ನೂ,  ಎಲ್ಲವನ್ನೂ ಪೂರಯಿಸುವ ಮತ್ತು ನನ್ನ ಸಂಪೂರ್ಣ ಅಸ್ತಿತ್ವಕ್ಕೆ ಪೋಷಣೆ ನೀಡುವ, ಆ ಭಗವಂತನಿಗೆ, ಆ ಮಾತೆ ಪ್ರಕೃತಿಗೆ ಎಲ್ಲವನ್ನೂ ಅರ್ಪಿಸುವ ಮನೋಭಾವನೆಯನ್ನು ನನ್ನದಾಗಿಸಿಕೊಳ್ಳಬಲ್ಲೆನೇ?
\end{mananam}
\WritingHand\enspace\textbf{ಆತ್ಮ ವಿಮರ್ಶೆ}\\
\begin{inspiration}{\mananamfont ಸ್ಫೂರ್ತಿ}
\small \mananamtext ಯಜ್ಞ ಅಥವಾ ತ್ಯಾಗವೆಂಬುದರ ನಿಜವಾದ ಅರ್ಥ, ಏನನ್ನೂ ಮರಳಿ ನಿರೀಕ್ಷಿಸದೆ ಎಲ್ಲವನ್ನೂ ಸಮರ್ಪಿಸುವುದಾಗಿದೆ; ಈ ಭಾವನೆಯು ನಮ್ಮ ಸಹಜ ಗುಣವಾದ ಶ್ರದ್ಧೆಯನ್ನು (ಭಕ್ತಿ, ವಿಶ್ವಾಸ, ನಿಷ್ಠೆ) ಜಾಗೃತಗೊಳಿಸಿದಾಗ  ಮಾತ್ರ ಸಾಧ್ಯವಾಗುವುದು. ಸಂಪೂರ್ಣ ವಿಶ್ವದ ಎಲ್ಲ ಆಗು ಹೋಗುಗಳೂ ಈ ಶ್ರದ್ಧೆಯಿಂದ ನಡೆಯುತ್ತಿರುತ್ತದೆ ಮತ್ತು ಇದುವೇ ಜೀವನದ ಮೂಲತತ್ವವಾಗಿದೆ.ಯಾರು, ಏನನ್ನೂ ಬಿಗಿಯಾಗಿ ಹಿಡಿದಿಟ್ಟುಕೊಳ್ಳದೇ ಎಲ್ಲವನ್ನೂ ತ್ಯಾಗ ಮಾಡುತ್ತಾರೋ ಅವರನ್ನು, ಪ್ರಕೃತಿಯು ಪೋಷಿಸುತ್ತದೆ.
\end{inspiration}
\newpage

\begin{mananam}{\mananamfont ಮನನ ಶ್ಲೋಕ - ೧೬}
\small \mananamtext
ಪ್ರಾಪಂಚಿಕ ವಿಷಯಗಳ ಮೇಲಿನ ಅತಿಯಾದ ಆಸಕ್ತಿ ಹಾಗೂ, ಸಮಾಜದಲ್ಲಿನ ತೀವ್ರವಾದ ಸ್ಪರ್ಧೆಯಿಂದಾಗಿ ನಾನು, ಈ ಜೀವನ ಚಕ್ರವನ್ನು ನೋಡಿ ಜಿಗುಪ್ಸೆ ಪಟ್ಟುಕೊಳ್ಳುತ್ತಿದ್ದೇನೆಯೇ? ಇದು ಸೋತವನು ‘ದ್ರಾಕ್ಷಿ ಹುಳಿ’ ಎನ್ನುವ ಮನೋಭಾವವೇ ಅಥವಾ ಸೋಮಾರಿತನವೇ? ಹಾಗಿದ್ದರೆ, ನಾನು ಸಮಾಜದ ಪದ್ಧತಿಗಳಲ್ಲಿ ಭಾಗವಹಿಸುವುದಿಲ್ಲವಾದರೆ, ಅದರ ಲಾಭದಲ್ಲಿ ಮಾತ್ರ ನಾನು ಭಾಗಿದಾರನೇ?  ನಾನು ಸಮಾಜಕ್ಕೆ ಧನಾತ್ಮಕವಾದ ಕೊಡುಗೆಯನ್ನು ನೀಡುತ್ತಿದ್ದೇನೆಯೆ? ನನಗೆ ದೈನಂದಿನ ಚಟುವಟಿಕೆಗಳಲ್ಲಿ ಯಾವುದೇ ನೈತಿಕ ಶಿಸ್ತನ್ನು ಪರಿಪಾಲಿಸುವ ಮನಸ್ಸಿಲ್ಲವೇ? ನಾನು ಯಾವದೇ ಸ್ವಾರ್ಥಕ್ಕಾಗಿಯಲ್ಲದೇ, ತೃಪ್ತಿ ಪಡೆಯುವ ಅವಶ್ಯಕತೆ ಇಲ್ಲದೆಯೇ,  ಬರೇ ಕಾರ್ಯಕ್ಕೋಸ್ಕರವಾಗಿಯೇ, ತತ್ಕ್ಷಣ ಕಾರ್ಯ ಪ್ರವೃತ್ತನಾಗಲು ಸಮರ್ಥನಾಗಿದ್ದೇನೆಯೇ?
\end{mananam}
\WritingHand\enspace\textbf{ಆತ್ಮ ವಿಮರ್ಶೆ}\\
\begin{inspiration}{\mananamfont ಸ್ಫೂರ್ತಿ}
\small \mananamtext ಆಧುನಿಕ ಸಮಾಜದ ಭೌತಿಕತೆಯ ಮೇಲಿನ ಮತ್ತು ಅತಿಯಾದ ಕ್ರಿಯೆಗಳ ಮೇಲಿನ ಗಮನ ‘ರಾಜಸ’ ಸ್ವಭಾವ ಉಳ್ಳದ್ದಾಗಿದೆ . ರಾಜಸವನ್ನು ಸರಿದೂಗಿಸಲು ‘ತಮಸ್’ ಅನ್ನು ವರ್ಜಿಸಿ, ‘ಸತ್ವ’ವನ್ನು ಆಲಂಗಿಸಬೇಕು. ‘ತಮಸ್ ’ನೊಂದಿಗೆ ಹೋಲಿಸಿದಾಗ, ‘ರಜಸ್’ ಉತ್ತಮ ಆದರೆ, ‘ಸತ್ವವು’ ಎಲ್ಲಕ್ಕಿಂತಲೂ ಶ್ರೇಷ್ಠ ಎಂದು ಗೀತೆಯು ಸೂಚಿಸುತ್ತದೆ. ಕರ್ಮಯೋಗದ ಪಥದಲ್ಲಿ, ‘ರಾಜಸ’ ಗುಣವು ‘ಸತ್ವ’ ಗುಣದೊಂದಿಗೆ ಬೆರೆತು, ಸತ್ವ ಗುಣದಿಂದ ಮಾರ್ಗದರ್ಶನ ಪಡೆಯುತ್ತಿದ್ದರೆ ಆದರ್ಶಪ್ರಾಯವಾಗಿರುತ್ತದೆ. ಒಬ್ಬನು, ಉತ್ತಮೋತ್ತಮ ಕರ್ತವ್ಯಗಳನ್ನು ನಿರ್ವಹಿಸುತ್ತಿದ್ದಲ್ಲಿ, ಅಂತಹವನು, ಸುಲಭವಾಗಿ ಹಾಗೂ ಸ್ಥಿರವಾಗಿ, ವಿಕಾಸದೆಡೆಗೆ ಮೇಲೇರುತ್ತಾ ಹೋಗುತ್ತಾನೆ.
\end{inspiration}
\newpage

\begin{mananam}{\mananamfont {ಮನನ ಶ್ಲೋಕ - ೧೭, ೧೮}}
\small \mananamtext ವ್ಯವಹಾರಗಳು ಮತ್ತು ನಿರೀಕ್ಷೆಗಳು, ಸಹಜವಾಗಿಯೂ   ಅನಿವಾರ್ಯವಾಗಿಯೂ ಇರಬಹುದಾದ - ತುಲನಾತ್ಮಕವಾದ ಈ ಪ್ರಪಂಚದಲ್ಲಿ ನಾನು ಕಾರ್ಯ ನಿರ್ವಹಿಸಬಲ್ಲೆನೇ? ಹಾಗಿದ್ದರೆ, ಇದು ಆಂತರಿಕ ದಾಸ್ಯಕ್ಕೆ ಕಾರಣವಾಗಬಹುದೇ? ವಿಶೇಷವಾಗಿ, ಮಾನಸಿಕವಾಗಿಯೂ ಭಾವನಾತ್ಮಕವಾಗಿಯೂ, ಬೇರೆ ವ್ಯಕ್ತಿಗಳ ಮೇಲಿನ ಅವಲಂಬನೆಗಳಿಂದ ನಾನು ಸ್ವತಂತ್ರನಾಗಿದ್ದೇನೆಯೇ? ನಾನು, ಇತರರಿಂದ ಯಾವ ರೀತಿಯ ವೈಯಕ್ತಿಕ ನಿರೀಕ್ಷೆಗಳನ್ನು ಮನದಲ್ಲಿ ಹೊಂದಿದ್ದೇನೆ? 
\end{mananam}
\WritingHand\enspace\textbf{ಆತ್ಮ ವಿಮರ್ಶೆ}\\
\begin{inspiration}{\mananamfont ಸ್ಫೂರ್ತಿ}
\small \mananamtext ಆತ್ಮಸಾಕ್ಷಾತ್ಕಾರ ಪಡೆದ ಋಷಿಯು, ಯಾರಿಂದಲೂ, ಯಾವುದರಿಂದಲೂ, ಏನನ್ನೂ ನಿರೀಕ್ಷಿಸದೆ ತನ್ನೊಳಗೆ ತಾನು  ಸಂತೃಪ್ತಿ ಹೊಂದಿರುತ್ತಾನೆ. ಇಂತಹ ಒಬ್ಬ ಸ್ವತಂತ್ರ ಋಷಿಯು ಸಮಾಜದ ಎಲ್ಲಾ ತರಹದ ಋಣಗಳಿಂದ ಮುಕ್ತನಾಗಿರುತ್ತಾನೆ.  ಯಾವುದೇ ಸ್ವಾರ್ಥದಿಂದ ಪ್ರೇರಿತನಾಗದೇ,  ಉನ್ನತ ಸಾಧನೆ ಮಾಡಿದ ಯೋಗಿಯು, ಇತರರಿಗೆ (ಸಾಮಾನ್ಯ ಜನರಿಗೆ ) ಒಂದು ಉತ್ತಮ ಮಾದರಿಯಾಗಿರಬೇಕೆಂಬ ಏಕೈಕ ಉದ್ದೇಶಕ್ಕಾಗಿ, ಕರ್ತವ್ಯದ ಕ್ರಿಯೆಗಳಲ್ಲಿ ತೊಡಗಬಹುದು. ಆದರೆ, ಸಮಾಜದ ಎಲ್ಲರ ಕ್ಷೇಮಾಭಿವೃದ್ಧಿಗಾಗಿ, ಆ ಜ್ಞಾನಿಯ (ಯೋಗಿ) ಜೀವನಕ್ಕೆ ಬೇಕಾಗುವ ಅಗತ್ಯತೆಗಳನ್ನು ಪೂರೈಸಲು, ಈ ಸಮಾಜವು ಸದಾ ಬದ್ಧವಾಗಿರಬೇಕಾಗುತ್ತದೆ.
\end{inspiration}
\newpage

\slcol{\Index{ತಸ್ಮಾದಸಕ್ತಃ ಸತತಂ} ಕಾರ್ಯಂ ಕರ್ಮ ಸಮಾಚರ ।\\
ಅಸಕ್ತೋ ಹ್ಯಾಚರನ್ಕರ್ಮ ಪರಮಾಪ್ನೋತಿ ಪೂರುಷಃ ॥ ೧೯ ॥ }
\cquote{ಆದ್ದರಿಂದ ಯಾವಾಗಲೂ ನಿನ್ನ ಕೆಲಸವನ್ನು ಫಲದಾಸೆ ಇಲ್ಲದೆ ಮಾಡು. ಫಲದಾಸೆ ಇಲ್ಲದೆ ಕರ್ಮವನ್ನು ಮಾಡುವವನು ಪರಮಾತ್ಮನನ್ನು ಪಡೆಯುತ್ತಾನೆ.}
\slcol{\Index{ಕರ್ಮಣೈವ ಹಿ} ಸಂಸಿದ್ಧಿಮಾಸ್ಥಿತಾ ಜನಕಾದಯಃ ।\\
ಲೋಕಸಂಗ್ರಹಮೇವಾಪಿ ಸಂಪಶ್ಯನ್ಕರ್ತುಮರ್ಹಸಿ ॥ ೨೦ ॥}
\cquote{ಜನಕನೇ ಮೊದಲಾದವರು ಕರ್ಮದಿಂದಲೇ ಜ್ಞಾನಸಿದ್ಧಿಯನ್ನು ಹೊಂದಿದರು. ಜನರಿಗೆ ದಾರಿಯನ್ನು ತೋರಿಸಬೇಕೆಂಬುದನ್ನಾದರೂ ಮನಸ್ಸಿಗೆ ತಂದು ನೀನು ಕರ್ಮವನ್ನು ಮಾಡತಕ್ಕದ್ದು.}
\slcol{\Index{ಯದ್ಯದಾಚರತಿ ಶ್ರೇಷ್ಠ}ಸ್ತತ್ತದೇವೇತರೋ ಜನಃ ।\\
ಸ ಯತ್ಪ್ರಮಾಣಂ ಕುರುತೇ ಲೋಕಸ್ತದನುವರ್ತತೇ ॥ ೨೧ ॥}
\cquote{ದೊಡ್ಡವನೆನಿಸಿಕೊಂಡವನು ಏನೇನು ಮಾಡುತ್ತಾನೋ ಉಳಿದವರೂ ಅದನ್ನೇ ಮಾಡುತ್ತಾರೆ. ಅವನು ಯಾವುದನ್ನು ಸರಿ ಎಂದು ತಿಳಿದು ಮಾಡುತ್ತಾನೋ ಜನರು ಅದನ್ನು ಹಿಂಬಾಲಿಸುತ್ತಾರೆ.}
\slcol{\Index{ನ ಮೇ ಪಾರ್ಥಾಸ್ತಿ ಕರ್ತವ್ಯಂ} ತ್ರಿಷು ಲೋಕೇಷು ಕಿಂಚನ ।\\
ನಾನವಾಪ್ತಮವಾಪ್ತವ್ಯಂ ವರ್ತ ಏವ ಚ ಕರ್ಮಣಿ ॥ ೨೨ ॥}
\cquote{ಅರ್ಜುನ,ನಾನು ಮಾಡಬೇಕಾದದ್ದೆಂಬುದು ಮೂರು ಲೋಕದಲ್ಲಿಯೂ ಏನೇನೂ ಇಲ್ಲ. ನಾನು ಕರ್ಮದಿಂದ ಪಡೆಯಬೇಕಾದ್ದು ಏನೂ ಇಲ್ಲ. ಆದರೂ ನಾನು ಕರ್ಮದಲ್ಲಿ ತೊಡಗಿಕೊಂಡೇ ಇದ್ದೇನೆ.}
\slcol{\Index{ಯದಿ ಹ್ಯಹಂ ನ} ವರ್ತೇಯಂ ಜಾತು ಕರ್ಮಣ್ಯತಂದ್ರಿತಃ ।\\
ಮಮ ವರ್ತ್ಮಾನುವರ್ತಂತೇ ಮನುಷ್ಯಾಃ ಪಾರ್ಥ ಸರ್ವಶಃ ॥ ೨೩ ॥}
\cquote{ನಾನು ಯಾವಾಗಲೂ ಕ್ರಿಯೆಯಲ್ಲಿ ನಿರತನಾಗಿರದಿದ್ದರೆ, ವಿಶ್ರಾಂತಿಯಿಲ್ಲದೆ, ಪುರುಷರು ಎಲ್ಲಾ ರೀತಿಯಲ್ಲಿ ನನ್ನ ಮಾರ್ಗವನ್ನು ಅನುಸರಿಸುತ್ತಾರೆ, ಓ ಅರ್ಜುನ!}


\newpage
\begin{mananam}{\mananamfont ಮನನ ಶ್ಲೋಕ - ೧೯}
\small \mananamtext ನನ್ನ ಜೀವನ ನಿರ್ವಹಣೆಗಾಗಿ ಮಾಡುವ ವೃತ್ತಿ ಮತ್ತು ಇತರ ದೈನಂದಿನ ಕ್ರಿಯೆಗಳು, ‘ಇಂತಹದು ಇಷ್ಟವಾದದ್ದು ಅಥವಾ ಇಷ್ಟವಿಲ್ಲದ್ದು’ ಎಂಬ ಭಾವನೆಯಿಂದ ಹುಟ್ಟುತ್ತವೆಯೇ? ಕೆಲವೊಮ್ಮೆ, ಕೆಲವು ಕೆಲಸಗಳ ಬಗೆಗಿನ ನನ್ನ ಒಲವು, ಸ್ವಾರ್ಥಯುತವಾದ ಆಸೆಗಳು ಮತ್ತು ಅದರಿಂದಾಗುವ ಫಲಿತಾಂಶಗಳು, ನನ್ನ ಮೇಲೆ  ಹೇಗೆ ಒತ್ತಡ ಮತ್ತು ವ್ಯಾಕುಲತೆಗಳಿಗೆ ಕಾರಣವಾಗುವುತ್ತದೆ ಎಂಬುದನ್ನು ನೋಡುತ್ತೇನೆಯೇ? ಇದು ಅತೀ ಚಡಪಡಿಕೆ, ನಿದ್ರಾಹೀನ ಸ್ಥಿತಿ ಅಥವಾ ಇತರ ಮನೋದೈಹಿಕ ಸ್ಥಿತಿಗೆ ಎಳೆದೊಯ್ಯುತ್ತಿದೆಯೇ? ಸಾತ್ವಿಕತೆಯಿಂದ ದೂರ ಸರಿಯುತ್ತಿದ್ದೇನೆಯೇ? ನಾನು, ಯಾವುದೇ ಒಂದು ನಮೂನೆಯ ಕೆಲಸ ಅಥವಾ ವೃತ್ತಿಯ ರೀತಿ ಮತ್ತು ಫಲಿತಾಂಶಗಳಿಗೆ ಮೋಹಗೊಳ್ಳದೆ, ಕೆಲಸದಲ್ಲಿ ನನ್ನನ್ನು ತೊಡಗಿಸಿಕೊಳ್ಳುವ ಮೂಲತತ್ವವನ್ನು ಅನ್ವಯಿಸುವುದು ಹೇಗೆ? 
\end{mananam}
\WritingHand\enspace\textbf{ಆತ್ಮ ವಿಮರ್ಶೆ}\\
\begin{inspiration}{\mananamfont ಸ್ಫೂರ್ತಿ}
\small \mananamtext ಯಾವುದೇ ಒಂದು ನಿರ್ದಿಷ್ಟ ಕೆಲಸದಲ್ಲಿ ಅಥವಾ ಅಧ್ಯಯನದಲ್ಲಿ, ಪ್ರತಿಯೊಬ್ಬ ಯಶಸ್ವೀ ವ್ಯಕ್ತಿ, ತನ್ನಮನಸ್ಸನ್ನು ಸಂಪೂರ್ಣವಾಗಿ ಏಕಾಗ್ರಗೊಳಿಸಿದಾಗ, ತಾತ್ಕಾಲಿಕವಾಗಿಯಾದರೂ, ಧನ್ಯತಾ ಭಾವದ ಸ್ಥಿತಿಯನ್ನು ಅನುಭವಿಸುತ್ತಾನೆ; ಈ ಮಗ್ನತೆಯಿಂದಾಗಿ ಅವನಿಗೆ, ಇನ್ನ್ಯಾರೂ, ಬೇರೆ ಯಾವ ವಿಷಯವೂ ಕಾಣುವುದಿಲ್ಲ; ಅಂದರೆ, ಅಂತಹ ಕ್ಷಣದಲ್ಲಿ ಕರ್ತೃ (ಪ್ರಮಾತ), ಕೃತ್ಯ (ಪ್ರಮೇಯ) ಹಾಗೂ ಕೃತ್ಯಕ್ಕೆ ಕಾರಣವಾದ ವಸ್ತು (ಪ್ರಮಾಣ) ಎಲ್ಲಾ ಸಮ್ಮಿಳಿತವಾಗುತ್ತದೆ; ಇಂಥಹ ಧನ್ಯತಾ ಭಾವದ ಸ್ಥಿತಿಯೇ, ತಾತ್ಕಾಲಿಕ ನಿರ್ವಾಣದ ಅನುಭವ, ಪರಮಾನಂದ ಹಾಗೂ ಜೀವನದ ಶಾಂತಿ! 
\end{inspiration}
\newpage

\begin{mananam}{\mananamfont {ಮನನ ಶ್ಲೋಕ - ೨೦, ೨೧}}
\small \mananamtext ನನಗೆ ಮಾದರಿಯಾದ ವ್ಯಕ್ತಿಗಳು ಯಾರು? ಅವರು ನನಗೆ, ಶಾಂತ ಮತ್ತು ಸ್ವತಂತ್ರವಾಗಿ ಕ್ರಿಯೆ ಮಾಡುವಂತೆ  ಪ್ರೇರೇಪಿಸುತ್ತಾರೆಯೇ? ಯಾವ ಮೌಲ್ಯಗಳು ನನ್ನನ್ನು ಕರ್ತವ್ಯದಲ್ಲಿ ತೊಡಗಿಸಿಕೊಳ್ಳುವಂತೆ ಪ್ರೇರೇಪಿಸುತ್ತವೆ? ಅವುಗಳಲ್ಲಿ, ಇತರರ ಕಲ್ಯಾಣಕ್ಕಾಗಿ ಕೆಲಸ ಮಾಡುವುದೂ ಒಳಗೊಂಡಿದೆಯೇ? ಸಾಮಾನ್ಯ ಜನರಿಗಾಗಿ ಒಳಿತನ್ನು ಬಯಸುವ ದೃಷ್ಟಿಯಿಂದ, ನನ್ನನ್ನು ನಾನು ಸಬಲೀಕರಣಗೊಳಿಸಿಕೊಳ್ಳಬಲ್ಲೆನೇ?
\end{mananam}
\WritingHand\enspace\textbf{ಆತ್ಮ ವಿಮರ್ಶೆ}\\
\begin{inspiration}{\mananamfont ಸ್ಫೂರ್ತಿ}
\small \mananamtext ಯಾರು, ಜನರ ಒಳಿತಿಗಾಗಿ ತಮ್ಮ ಸರ್ವಸ್ವವನ್ನೂ ತ್ಯಾಗಮಾಡಿರುವರೋ, ಅಂತಹವರ ಮರಣದ ನಂತರವೂ, ಅವರ ಚೈತನ್ಯವು ಬಹಳ ಕಾಲ ಜನರ ಮನದಲ್ಲಿ ಅಚ್ಚಳಿಯದೆ ಉಳಿಯುತ್ತದೆ. ಸ್ವಾರ್ಥಯುತ ಅಭಿಲಾಷೆಗಳು ಒಬ್ಬನಲ್ಲಿ, ತುಂಬಾ ಒತ್ತಡವನ್ನು ಸೃಷ್ಟಿಸುತ್ತದೆ ಆದರೆ, ಜನರ ಅಭಿವೃದ್ಧಿಗಾಗಿ ಒಬ್ಬ ವ್ಯಕ್ತಿಯು ತನ್ನ ಜೀವನ ಮುಡಿಪಾಗಿಟ್ಟಲ್ಲಿ, ಆ ವ್ಯಕ್ತಿಯ ಹಾಗೂ ಸಮಾಜದ ಉದ್ಧಾರ ನಿಶ್ಚಿತ!
\end{inspiration}
\newpage

\begin{mananam}{\mananamfont ಮನನ ಶ್ಲೋಕ - ೨೨}
\small \mananamtext  ಈ ಸೃಷ್ಟಿಯನ್ನು ನಡೆಸುವವರು ಯಾರು? ಅವನ ಅಥವಾ ಅವಳ ಉದ್ದೇಶ ಏನಿರಬಹುದು? ನಿಸ್ವಾರ್ಥವಾಗಿ ಕರ್ತವ್ಯ ಮಾಡುವ, ಈ ಬ್ರಹ್ಮಾಂಡದಲ್ಲಿರುವ ಪ್ರಭಾವೀ ಶಕ್ತಿಗಳಾದ ಸೂರ್ಯ, ಚಂದ್ರ, ವಾಯು, ಸಮುದ್ರ ಇತ್ಯಾದಿಗಳಿಂದ, ನಾನು ಪಾಠ ಕಲಿಯಬಲ್ಲೆನೇ? ಇದನ್ನೆಲ್ಲ ನಡೆಸಿಕೊಂಡು ಹೋಗುತ್ತಿರುವುದು ಯಾವುದು? ಅವುಗಳ ಮೂಲ ಯಾವುದು ಮತ್ತು ಅವುಗಳ ಪೋಷಣೆ ಯಾವುದರಿಂದ ಆಗುತ್ತಿದೆ? 
\end{mananam}
\WritingHand\enspace\textbf{ಆತ್ಮ ವಿಮರ್ಶೆ}\\
\begin{inspiration}{\mananamfont ಸ್ಫೂರ್ತಿ}
\small \mananamtext  ಈ ಜಗತ್ತಿಗೆ ಚಾಲನೆ ಕೊಟ್ಟು ನಡೆಸುತ್ತಿರುವುದು ಪ್ರೀತಿ ಮತ್ತು ಕರುಣೆ ಎಂಬ ಜೋಡಿ ಯಂತ್ರಗಳಾದ ದೊಡ್ಡ ಶಕ್ತಿಗಳು. ನಿಜವಾದ ಪ್ರೀತಿ ಎಂದರೆ, ಎಲ್ಲರನ್ನೂ, ಎಲ್ಲವನ್ನೂ ತನ್ನದೇ ಆದ ಭಾಗವೆಂದು ನೋಡುವುದು; ಹೀಗೆ ಎಲ್ಲವೂ, ಎಲ್ಲರೂ ನಮ್ಮ ಪ್ರೀತಿ ಪಾತ್ರರೇ ಆಗಿದ್ದಾಗ, ಅವರ ಒಳಿತಿಗಾಗಿ ಕಾರ್ಯನಿರ್ವಹಿಸುವುದು ಯಾವತ್ತೂ ‘ಹೊರೆ’ ಎಂಬ ಭಾವನೆ ಬರಲಾರದು; ಹೃದಯದಾಳದಿಂದ ಹುಟ್ಟಿದ ಪೀತಿಯಿಂದ ಮಾಡಿದ ಕಾರ್ಯಗಳಿಂದ (ನಮ್ಮ ಅಥವಾ ಇತರರ ಪ್ರೀಥ್ಯರ್ಥಕ್ಕಾಗಿರಬಹುದು) ನಮ್ಮ ಜೀವನವು ಸಂಪೂರ್ಣ ಸಮರಸದಿಂದ ಕೂಡಿರುತ್ತದಲ್ಲದೇ, ಮನಸ್ಸು ಶುದ್ಧವಾಗಿ ರೂಪಾoತರವಾಗುತ್ತದೆ; ಇದು ಒಂದು ರೀತಿಯ ಪ್ರಕೃತಿಯ ಆರಾಧನೆಯೇ ಆಗಿದೆ!
\end{inspiration}
\newpage


\slcol{\Index{ಉತ್ಸೀದೇಯುರಿಮೇ ಲೋಕಾ} ನ ಕುರ್ಯಾಂ ಕರ್ಮ ಚೇದಹಮ್ ।\\
ಸಂಕರಸ್ಯ ಚ ಕರ್ತಾ ಸ್ಯಾಮುಪಹನ್ಯಾಮಿಮಾಃ ಪ್ರಜಾಃ ॥ ೨೪ ॥}
\cquote{ನಾನು ಕರ್ಮವನ್ನು ಮಾಡದೇ ಹೋದರೆ ಈ ಲೋಕಗಳು ಹಾಳಾದಾವು, ನಾನು ವರ್ಣಸಂಕರಕ್ಕೂ ಕಾರಣನಾದೇನು, ಈ ಜನರ ಪತನಕ್ಕೆ ಕಾರಣನಾದೇನು.}
\slcol{\Index{ಸಕ್ತಾಃ ಕರ್ಮಣ್ಯವಿದ್ವಾಂಸೋ} ಯಥಾ ಕುರ್ವಂತಿ ಭಾರತ ।\\
ಕುರ್ಯಾದ್ವಿದ್ವಾಂಸ್ತಥಾಸಕ್ತಶ್ಚಿಕೀರ್ಷುರ್ಲೋಕಸಂಗ್ರಹಮ್ ॥ ೨೫ ॥} 
\cquote{ಓ ಅರ್ಜುನ! ಅಜ್ಞಾನಿಗಳು ಕ್ರಿಯೆಯಲ್ಲಿ ಮೋಹದಿಂದ ಕೆಲಸ ಮಾಡುವಂತೆ, ಜ್ಞಾನಿಗಳು ಕೇವಲ ಲೋಕಕಲ್ಯಾಣಕ್ಕಾಗಿ ಕರ್ತವ್ಯವನ್ನು ಮಾಡಬೇಕು.}
\slcol{\Index{ನ ಬುದ್ಧಿಭೇದಂ ಜನಯೇದ}ಜ್ಞಾನಾಂ ಕರ್ಮಸಂಗಿನಾಮ್ ।\\
ಜೋಷಯೇತ್ಸರ್ವಕರ್ಮಾಣಿ ವಿದ್ವಾನ್ಯುಕ್ತಃ ಸಮಾಚರನ್ ॥ ೨೬ ॥}
\cquote{ಕರ್ಮದಲ್ಲಿ ಅಭಿರುಚಿಯಿರುವ ಪಾಮರಜನರ ಬುದ್ಧಿಗೆ ಪಂಡಿತನು ಭೇದವನ್ನು ಉಂಟುಮಾಡಕೂಡದು. ತಾನು ಯೋಗಾಯುಕ್ತನಾಗಿ ಸರಿಯಾಗಿ ಕರ್ಮ ಮಾಡುತ್ತಾ, ಅವರನ್ನು ಕರ್ಮಕ್ಕೆ ಪ್ರೋತ್ಸಾಹಿಸಬೇಕು.}
\slcol{\Index{ಪ್ರಕೃತೇಃ ಕ್ರಿಯಮಾಣಾನಿ} ಗುಣೈಃ ಕರ್ಮಾಣಿ ಸರ್ವಶಃ ।\\
ಅಹಂಕಾರವಿಮೂಢಾತ್ಮಾ ಕರ್ತಾಹಮಿತಿ ಮನ್ಯತೇ ॥ ೨೭ ॥}
\cquote{ಅಹಂಕಾರದಿಂದ ತಲೆಕೆಡಿಸಿಕೊಂಡವನು ಮಾಯೆಯ ಅಧೀನವಾಗಿ ಇಂದ್ರಿಯಗಳಿಂದಾಗುವ ಕರ್ಮಗಳನ್ನು ತಾನೇ ಮಾಡುವುದೆಂದು ತಿಳಿಯುತ್ತಾನೆ.}
\slcol{\Index{ಮಹಾಬಾಹೋ ಗುಣ}ಕರ್ಮವಿಭಾಗಯೋಃ ।\\
ಗುಣಾ ಗುಣೇಷು ವರ್ತಂತ ಇತಿ ಮತ್ವಾ ನ ಸಜ್ಜತೇ ॥ ೨೮ ॥}
\cquote{ಅರ್ಜುನ, ಗುಣಗಳ ಮತ್ತು ಕರ್ಮಗಳ ವಿಂಗಡದ ನಿಜವನ್ನರಿತವನಾದರೋ ಇಂದ್ರಿಯ ಮತ್ತು ವಿಷಯಗಳ ಸಂಬಂಧದ ತಿರುಳನ್ನು ತಿಳಿದು ನಾನು ಮಾಡುವವನೆಂದು ಅಭಿಮಾನಗೊಳ್ಳುವುದಿಲ್ಲ.}

\newpage
\begin{mananam}{\mananamfont ಮನನ ಶ್ಲೋಕ - ೨೫}
\small \mananamtext ನನ್ನ ಜೀವನದಲ್ಲಿ ಜ್ಞಾನದ ಪಾತ್ರವೇನು? ಪ್ರಬುದ್ಧತೆಯು, ನಾನು ಜೀವನ ನೋಡುವ ರೀತಿಯನ್ನು ಹೇಗೆ ಬದಲಾಯಿಸಿತು? (ಅಂದರೆ, ಪ್ರಾಪಂಚಿಕ ಪ್ರಚೋದನೆಗಳಾದ ಹೆಸರು, ಕೀರ್ತಿ, ಹಣ, ಸಂವೃದ್ಧಿ ಹಾಗೂ ಇಂದ್ರಿಯ ಸುಖಗಳಿಂದ, ಪಾರಮಾರ್ಥಿಕದೆಡೆಗೆ ನಡೆಯುವ ನನ್ನ ಮನೋಭಾವನೆ). ನನಗೆ ಉತ್ತಮ ಶ್ರೇಣಿಯಲ್ಲಿ, ವಿವೇಕದಿಂದ ಕಾರ್ಯನಿರ್ವಹಿಸಲು ಮಾರ್ಗದರ್ಶನ ಮಾಡುವ ನಂಬಿಕಸ್ಥ ಮೂಲ ಯಾರು ಅಥವಾ ಯಾವುದಾಗಿರಬಹುದು? ನನ್ನ ಪ್ರಾಪಂಚಿಕ ಬುದ್ಧಿಯನ್ನು, ಋಷಿಗಳ ಮತ್ತು ಬುದ್ಧಿವಂತರ ವಿವೇಕವಾಗಿ ನಾನು ಹೇಗೆ ಪರಿವರ್ತಿಸಿಕೊಳ್ಳಲಿ? 
\end{mananam}
\WritingHand\enspace\textbf{ಆತ್ಮ ವಿಮರ್ಶೆ}\\
\begin{inspiration}{\mananamfont ಸ್ಫೂರ್ತಿ}
\small \mananamtext ಒಬ್ಬರು, ತಮಗೋಸ್ಕರವೇ ಜೀವಿಸುವುದು ಮತ್ತು ಉಳಿವಿಗಾಗಿ ಹೋರಾಡುವುದು ಸಹಜ ಪ್ರವೃತ್ತಿ. ಆದರೆ “ಬುದ್ಧಿಶಕ್ತಿ” ಎಂಬ ವಿಶೇಷತೆಯನ್ನು, ಮಾನವನಿಗೆ ಆ ಭಗವಂತನು ದಯಪಾಲಿಸಿದ್ದಾನೆ. ನಮ್ಮ ವಿವೇಕವನ್ನು ಉಪಯೋಗಿಸಿಕೊಂಡು, ಬರೇ ನಮಗಾಗಿ ಮಾತ್ರವಲ್ಲದೇ, ಒಂದು ದಿವ್ಯಾತಾ ಭಾವನೆಯಿಂದ, ಒಂದು ಶಾಂತ ಹಾಗೂ ನಿರಾಸಕ್ತತಾ ಭಾವದಿಂದ, ಮಾನವೀಯತೆಗಾಗಿ ಹಾಗೂ ಇಡೀ ಮಾನವ ಕಲ್ಯಾಣಕ್ಕಾಗಿ ಕಾರ್ಯನಿರತರಾಗಬೇಕು. 
\end{inspiration}
\newpage

\begin{mananam}{\mananamfont ಮನನ ಶ್ಲೋಕ - ೨೬}
\small \mananamtext ಯಾರು ಆಧ್ಯಾತ್ಮಿಕ ದಾರಿಯಲ್ಲಿ ಪ್ರಗತಿ ಕಾಣುತ್ತಾರೋ ಅಥವಾ ಉನ್ನತವಾದ ಆಶ್ರಮಗಳಿಗೆ (ಅಂದರೆ, ಮೊದಲನೆಯದಾಗಿ ಬ್ರಹ್ಮಚರ್ಯ, ನಂತರ ಗೃಹಸ್ಥ, ವಾನಪ್ರಸ್ಥ, ಸoನ್ಯಾಸ ಹೀಗೆ) ಹೋಗುತ್ತಾರೋ ಅವರು ಹೊಸ ಆಕಾಂಕ್ಷಿಗಳಿಗೆ ಕೊಟ್ಟ ಉಪದೇಶ ಮತ್ತು ಅಭ್ಯಾಸಗಳನ್ನು ಕೀಳಾಗಿ ಕಾಣದೆ ಇರುವ ಜವಾಬ್ದಾರಿಯನ್ನು ಹೊಂದಿರುತ್ತಾರೆ. ಬೇರೆ ಬೇರೆ ಆಕಾಂಕ್ಷಿಗಳಿಗೆ, ಬೇರೆಬೇರೆ ಹಂತದಲ್ಲಿ ಆಧ್ಯಾತ್ಮಿಕ ಬೋಧನೆಗಳನ್ನು ನೀಡಲಾಗುತ್ತದೆ (ಅಂದರೆ, ಅವರವರ ಆಧ್ಯಾತ್ಮಿಕ ಪ್ರಗತಿಯ ಮಟ್ಟಕ್ಕೆ ತಕ್ಕಂತೆ). ಆಧ್ಯಾತ್ಮದ ತುತ್ತ ತುದಿಯಲ್ಲಿದ್ದವರಿಗೆ ಮಾತ್ರ, ಎಲ್ಲಾ ತರಹದ ಸಹಾಯ ಮತ್ತು ಆಶ್ರಯವನ್ನು ಬಿಟ್ಟು, ತಮ್ಮಲ್ಲೇ ತಾವು (ತಮ್ಮ ಆತ್ಮದಲ್ಲಿಯೇ), ಆಶ್ರಯ ಪಡೆಯುವ ಸಮರ್ಥತೆ ಇರುತ್ತದೆ.
\end{mananam}
\WritingHand\enspace\textbf{ಆತ್ಮ ವಿಮರ್ಶೆ}\\
\begin{inspiration}{\mananamfont ಸ್ಫೂರ್ತಿ}
\small \mananamtext ಗುರಿಯ ಮೇಲೆ ಮಾತ್ರ ಗಮನ ಕೇಂದ್ರೀಕರಿಸಿದರೆ, ಒತ್ತಡಕ್ಕೆ ಒಳಗಾಗುವುದು ಖಂಡಿತ; ಆದರೆ, ‘ಕಾರ್ಯ ವಿಧಾನ’ದ ಬಗ್ಗೆ ನಮ್ಮ ಗಮನ ಕೇಂದ್ರೀಕರಿಸಿದಾಗ, ನಮ್ಮ ಮುಂದಿರುವ ಯಾವುದೇ ಸವಾಲುಗಳನ್ನೂ ನಿರ್ವಹಿಸಲು ನಮಗೆ ಸಾಧ್ಯವಾಗುತ್ತದೆ. ಯಾವುದೇ ಅಹಂ ಇಲ್ಲದೇ ( ಅಂದರೆ, ನನ್ನಿಂದಾಗಿಯೇ ಕಾರ್ಯ ನಡೆಯುತ್ತಿದೆ ಎಂಬ ಹಮ್ಮು ) ಹಾಗೂ, ಇದು ಇಷ್ಟ ಅಥವಾ, ಇದು ಅನಿಷ್ಟ ಎಂದುಕೊಳ್ಳದೇ  ಎಲ್ಲಾ ಕಾರ್ಯಗಳನ್ನೂ  ಮಾಡುವುದು – ಎಲ್ಲಾ ಗುಣಗಳನ್ನೂ ಮೀರಿ, ಒಬ್ಬನು ದೈವೀ ಸ್ವರೂಪವಾಗುವುದೇ ಆಗಿದೆ.
\end{inspiration}
\newpage

\begin{mananam}{\mananamfont ಮನನ ಶ್ಲೋಕ - ೨೭}
\small \mananamtext ಈ ದೇಹ ಕೆಲಸ ಮಾಡಲು ಬೇಕಾಗುವ ಕೆಲವು ಒಳ ಅಂಗಗಳ ಕ್ರಿಯೆಗಳಾದ ಉಸಿರಾಟ ಮತ್ತು ಪಚನಕ್ರಿಯೆಯನ್ನು ನಾನು ಹತೋಟಿಯಲ್ಲಿಡುತ್ತಿದ್ದೇನೆಯೇ? ಪ್ರಕೃತಿಯ ಬೆಂಬಲದಿಂದಾಗಿ ನಾನು ಮಾತನಾಡಲು ಸಶಕ್ತನಾಗುವಂತೆ ಮಾಡುವ ಶಾರೀರಿಕ ಪ್ರಕ್ರಿಯೆ ಇಲ್ಲವೇ? ಭೌತಿಕ ವಸ್ತುಗಳ ಪೋಷಣೆ ಇಲ್ಲದೆ ಮತ್ತು ಪ್ರಕೃತಿಯ ಉತ್ತೇಜನವಿಲ್ಲದೆ ಯೋಚಿಸಲು ಮತ್ತು ತರ್ಕಿಸಲು ನನ್ನ ಮನಸ್ಸಿಗೆ ಶಕ್ತಿ ಇದೆಯೇ? ನಾನು ನನ್ನದೇ ಎಂದು  ಹಕ್ಕಿನಿಂದ ಹೇಳಿಕೊಳ್ಳುವ, ಏನೆಲ್ಲಾ ಕೌಶಲ್ಯಗಳ ಮತ್ತು ಸಾಮರ್ಥ್ಯಗಳ ಅಭಿವೃದ್ಧಿಗೆ ಕೂಡ, ಯಾರೆಲ್ಲಾ ಅಥವಾ ಏನೆಲ್ಲಾ ಕಾರಣವಾಗಿದೆ ಹಾಗೂ, ಶ್ರಮದಾನದ ಕೊಡುಗೆ ಇದೆ ಎಂದು ತಿಳಿಯಬಲ್ಲೆನೇ? ಇದನ್ನೆಲ್ಲಾ ಪರ್ಯಾಲೋಚಿಸಿದರೆ, ನಾನು, ನನ್ನ ಜೀವನದಲ್ಲಿ ಹಕ್ಕಿನಿಂದ “ಇದು ನನ್ನದು” ಎಂದು ಹೇಳುವ ಕೆಲವು ಸಾಧನೆಗಳು ಮತ್ತು ಸಿದ್ಧಿಗಳನ್ನು ಮಾಡುವವನು ನಾನೆಯೇ? ಎಲ್ಲವೂ ನನ್ನಿಂದಲೇ ಆಗಿದೆಯೇ?
\end{mananam}
\WritingHand\enspace\textbf{ಆತ್ಮ ವಿಮರ್ಶೆ}\\
\begin{inspiration}{\mananamfont ಸ್ಫೂರ್ತಿ}
\small \mananamtext ‘ನಾನು ಅಥವಾ ನನ್ನದು’ ಎಂಬ ನಮ್ಮ ಅನಿಸಿಕೆಯು, ಯಾವುದನ್ನು ಒಳಗೊಂಡಿದೆ ಎಂಬುದನ್ನು ಪ್ರಾಮಾಣಿಕವಾಗಿ ಹಾಗೂ ಆಳವಾಗಿ ಅನ್ವೇಷಿಸಿದಾಗ, ನಮಗೆ ನಮ್ಮದೇ ಆದಂತಹ ಯಾವುದೇ ಒಂದು ‘ನಿರ್ದಿಷ್ಟ  ಅಸ್ತಿತ್ವ’ ಕಂಡುಬರುವುದಿಲ್ಲ. ‘ನಾನು’ ಎಂಬುದು ನನ್ನದೇ ಆದ ‘ಒಂದು ನಿರ್ದಿಷ್ಟ ವ್ಯಕ್ತಿತ್ವದಿಂದ ಒಡಗೂಡಿರುವ  ದೇಹ’ ಎಂದು ಲಘುವಾಗಿ ಪರಿಗಣಿಸುವ, ತಪ್ಪಾದ ಭಾವನೆ ಎಂದು ಕುಸಿಯುತ್ತದೋ, ಅಂದು ನಮಗೆ, ನಮ್ಮ ‘ಅಸ್ತಿತ್ವ ರಹಿತ’ ಸ್ವಭಾವದ ಜ್ಞಾನೋದಯವಾಗುತ್ತದೆ; ಅದೇ ವಾಸ್ತವಿಕತೆಯ ಅರಿವು ಹಾಗೂ ಅತ್ಯುನ್ನತ ಸ್ವಾತಂತ್ರ್ಯ; ಅದಮ್ಯ ‘ಮುಕ್ತಿ’.
\end{inspiration}
\newpage

\slcol{\Index{ಪ್ರಕೃತೇರ್ಗುಣಸಂಮೂಢಾಃ} ಸಜ್ಜಂತೇ ಗುಣಕರ್ಮಸು ।\\
ತಾನಕೃತ್ಸ್ನವಿದೋ ಮಂದಾನ್ಕೃತ್ಸ್ನವಿನ್ನ ವಿಚಾಲಯೇತ್ ॥ ೨೯ ॥}
\cquote{ಇಂದ್ರಿಯಗಳ ಮಾಯೆಗೆ ಒಳಗಾದವರು ವಿಷಯ ಮೋಹದಲ್ಲಿ ಮುಳುಗಿಬಿಡುತ್ತಾರೆ. ತತ್ವದ ತಿರುಳು ತಿಳಿದಿಲ್ಲದ ಆ ದಡ್ಡರನ್ನು ಚೆನ್ನಾಗಿ ತಿಳಿದವರು ವ್ಯಾಕುಲಗೊಳಿಸಬಾರದು.}
\slcol{\Index{ಮಯಿ ಸರ್ವಾಣಿ ಕರ್ಮಾಣಿ} ಸಂನ್ಯಸ್ಯಾಧ್ಯಾತ್ಮಚೇತಸಾ ।\\
ನಿರಾಶೀರ್ನಿರ್ಮಮೋ ಭೂತ್ವಾ ಯುಧ್ಯಸ್ವ ವಿಗತಜ್ವರಃ ॥ ೩೦ ॥}
\cquote{ಎಲ್ಲರೊಳಗೂ ನಾನು ಇರುವೆನೆಂದರಿತು ಎಲ್ಲ ಕರ್ಮಗಳನ್ನೂ ನನಗೊಪ್ಪಿಸಿ, ಫಲದ ಬಯಕೆಯನ್ನೂ, ನನ್ನದೆಂಬ ಅಭಿಮಾನವನ್ನೂ ತೊರೆದು ನಿಶ್ಚಿಂತನಾಗಿ ಯುದ್ಧ ಮಾಡು.}
\slcol{\Index{ಯೇ ಮೇ ಮತಮಿದಂ} ನಿತ್ಯಮನುತಿಷ್ಠಂತಿ ಮಾನವಾಃ ।\\
ಶ್ರದ್ಧಾವಂತೋಽನಸೂಯಂತೋ ಮುಚ್ಯಂತೇ ತೇಽಪಿ ಕರ್ಮಭಿಃ ॥ ೩೧ ॥}
\cquote{ಈ ನನ್ನ ಅಭಿಪ್ರಾಯವನ್ನು ಯಾರು ಅಸೂಯೆ ತಾಳದೆ ಯಾವಾಗಲೂ ನನ್ನ ಮೇಲಿನ ವಿಶ್ವಾಸದಿಂದ ಆಚರಿಸುತ್ತಾರೋ ಅವರು ಕೂಡ ಕರ್ಮ ಬಂಧನದಿಂದ ಬಿಡುಗಡೆಯನ್ನು ಹೊಂದುತ್ತಾರೆ.}
\slcol{\Index{ಯೇ ತ್ವೇತದಭ್ಯಸೂಯಂತೋ} ನಾನುತಿಷ್ಠಂತಿ ಮೇ ಮತಮ್ ।\\
ಸರ್ವಜ್ಞಾನವಿಮೂಢಾಂಸ್ತಾನ್ವಿದ್ಧಿ ನಷ್ಟಾನಚೇತಸಃ ॥ ೩೨ ॥}
\cquote{ಅಸೂಯೆಯಿಂದ ನನ್ನ ಅಭಿಪ್ರಾಯವನ್ನು ಆಚರಣೆಗೆ ತರದೆ ತಿರಸ್ಕರಿಸುವವರು ಜ್ಞಾನದ ಮಾರ್ಗವನ್ನೇ ಅರಿಯದ ಅವಿವೇಕಿಗಳು.ಅವರು ತಮ್ಮ ನಾಶವನ್ನು ತಾವೇ ಮಾಡಿಕೊಳ್ಳುವರೆಂದು ತಿಳಿ.}
\slcol{\Index{ಸದೃಶಂ ಚೇಷ್ಟತೇ ಸ್ವಸ್ಯಾಃ} ಪ್ರಕೃತೇರ್ಜ್ಞಾನವಾನಪಿ ।\\
ಪ್ರಕೃತಿಂ ಯಾಂತಿ ಭೂತಾನಿ ನಿಗ್ರಹಃ ಕಿಂ ಕರಿಷ್ಯತಿ ॥ ೩೩ ॥}
\cquote{ಬಲ್ಲವನೂ ಕೂಡ ತನ್ನ ಸ್ವಭಾವಕ್ಕೆ ಸರಿಯಾಗಿ ನಡೆಯುವನು. ಎಲ್ಲಾ ಪ್ರಾಣಿಗಳೂ ಹುಟ್ಟುಗುಣವನ್ನು ಹಿಂಬಾಲಿಸುತ್ತವೆ. ನಿಗ್ರಹದಿಂದ ಏನೂ ನಡೆಯದು.}


\newpage
\begin{mananam}{\mananamfont ಮನನ ಶ್ಲೋಕ - ೩೦}
\small \mananamtext ನಾನು ಯಾವಾಗ, ಯಾವುದೇ ಕಾರ್ಯವನ್ನು, ಉತ್ಕೃಷ್ಟವಾದ ಪ್ರಯತ್ನದಿಂದ ಮಾಡುತ್ತೇನೆಯೋ ಆಗ, ನಾನು ಉದ್ವೇಗಕ್ಕೆ ಒಳಗಾಗುತ್ತೇನೆಯೇ? ಒತ್ತಡಕ್ಕೆ ಒಳಗಾಗಿ, ಬೇಗನೇ ದಣಿಯುತ್ತೇನೆಯೇ? ನನಗೆ ಇಷ್ಟವಾದ ಅಥವಾ ಇಷ್ಟವಿಲ್ಲದ ಕಾರ್ಯಗಳನ್ನು ಮಾಡುವಾಗ ನನ್ನ ಮನಸ್ಸಿನ ಭಾವನೆಗಳನ್ನು ಗಮನಿಸಿದ್ದೇನೆಯೇ? ನಾನು ಸಹೋದ್ಯೋಗಿಗಳೊಡನೆ ಕುಟುಂಬದ ಸದಸ್ಯರೊಡನೆ ಘರ್ಷಣೆ ಮತ್ತು ಒತ್ತಡಕ್ಕೆ ಹಾಗೂ ಮಾನಸಿಕ ಹಿಂಸೆಗೆ ಒಳಗಾಗುತ್ತಿದ್ದೇನೆಯೇ? ಎಲ್ಲವನ್ನೂ ‘ಬಿಟ್ಟು ಬಿಡು ’ ಅಥವಾ, ‘ಹೋದರೆ ಹೋಗಲಿ ಬಿಡು’ ಎಂಬುವುದರ ಅರ್ಥವೇನು? ನಾನು, ನನ್ನ ಪ್ರಯತ್ನದ ತೀವ್ರತೆಯನ್ನು  ರಾಜಿಮಾಡಿಕೊಳ್ಳುವ ಬದಲು, ಫಲಿತಾಂಶದ ಮೇಲಿನ ಮೋಹವನ್ನು ತ್ಯಜಿಸಲು ಸಾಧ್ಯವಿದೆಯೇ? 
\end{mananam}
\WritingHand\enspace\textbf{ಆತ್ಮ ವಿಮರ್ಶೆ}\\
\begin{inspiration}{\mananamfont ಸ್ಫೂರ್ತಿ}
\small \mananamtext ಯಾರಲ್ಲಿ ‘ಶ್ರದ್ದೆ’ ಇರುತ್ತದೆಯೋ ಅವರಿಗೆ ಉದ್ವೇಗ ರಹಿತವಾದ, ತೀಕ್ಷ್ಣ  ಮಟ್ಟದ ಪ್ರಯತ್ನದ ಗುಟ್ಟು ಗೊತ್ತಿರುತ್ತದೆ. ಯಾವಾಗ ನಾವು ನಮ್ಮ ಜೀವನವನ್ನು ಮಾನಸಿಕವಾಗಿ, ಒಂದು ಉನ್ನತ ದೈವೀಕ ಶಕ್ತಿಯಲ್ಲಿ ನಂಬಿಕೆ ಇಟ್ಟು, ಆ ಶಕ್ತಿಗೆ ಒಪ್ಪಿಸುತ್ತೇವೆಯೋ, ಆಗ ನಮ್ಮಿಂದ ದೊಡ್ಡ ಹೊರೆಯೊಂದು ದೂರವಾಗುತ್ತದೆ; ಆಗ, ನಮ್ಮ ಕಾರ್ಯಗಳನ್ನು ಅತ್ತ್ಯುತ್ತಮವಾಗಿ ನಿರ್ವಹಿಸುತ್ತಾ, ‘ನಮಗೆ ಯಾವುದಕ್ಕೆ ಅರ್ಹತೆ ಇದೆಯೋ ಅದನ್ನು ಖಂಡಿತಾ ಯಾರೂ ನಿರಾಕರಿಸಲು ಸಾಧ್ಯವಿಲ್ಲ, ಅದನ್ನು ಪಡೆದೇ ಪಡೆಯುತ್ತೇವೆ’ ಎಂದು ತಿಳಿಯುತ್ತಾ ಮುಂದುವರೆಯಬಹುದು.
\end{inspiration}
\newpage

\begin{mananam}{\mananamfont {ಮನನ ಶ್ಲೋಕ - ೩೧, ೩೨}}
\small \mananamtext ನಾನು ಯಾವ ವರ್ಗಕ್ಕೆ ಸೇರುತ್ತೇನೆ -  ಈ ಏಳಿಗೆ ತರುವಂತಹ ಉಪದೇಶಗಳಿಗೆ, ಮನಸ್ಸು ಮತ್ತು ಶ್ರದ್ಧೆ ಇರುವವರ ಗುಂಪಿಗೋ ಅಥವಾ, ಇದರ ಪರಿವರ್ತನೆಯ ಪರಿಣಾಮದಲ್ಲಿ ನಂಬಿಕೆ ಇಲ್ಲದವರ ಗುಂಪಿಗೋ ಅಥವಾ ಇದನ್ನು ತಿರಸ್ಕರಿಸುವವರ ಗುಂಪಿಗೋ? ನಾನು ಸ್ವಲ್ಪವಾದರೂ ಈ ಬೋಧನೆಗಳನ್ನು ಅಧ್ಯಯನ ಮಾಡಿ ಅರ್ಥಮಾಡಿಕೊಳ್ಳಲು ಪ್ರಯತ್ನಿಸುತ್ತಿದ್ದೇನೆಯೇ? ಜೀವನದಲ್ಲಿ ಅಳವಡಿಸಿಕೊಳ್ಳಲು ಪ್ರಯತ್ನಿಸುತ್ತಿದ್ದೇನೆಯೇ? ಅಥವಾ ಇದರ ವಿರುದ್ಧವಾಗಿ ಪಕ್ಷಪಾತ ಮಾಡುತ್ತಿದ್ದೇನೆಯೇ?
\end{mananam}
\WritingHand\enspace\textbf{ಆತ್ಮ ವಿಮರ್ಶೆ}\\
\begin{inspiration}{\mananamfont ಸ್ಫೂರ್ತಿ}
\small \mananamtext ಈ ರೀತಿಯ ಬೋಧನೆಗಳಲ್ಲಿ ಅಂತರ್ಗತ ನಂಬಿಕೆ ಹೊಂದಿ, ಇದಕ್ಕೆ ತಕ್ಕಂತೆ, ನಮ್ಮ ಜೀವನವನ್ನು ಸರಿಯಾದ ಕ್ರಮದಲ್ಲಿ ನಡೆಸಿದಲ್ಲಿ, ಅದೊಂದು ದೈವಾನುಗ್ರಹವೇ ಆಗುತ್ತದೆ; ಜೀವನದಲ್ಲಿ ಹೆಚ್ಚಿನ  ಸಂಕಷ್ಟಕ್ಕೆ ಈಡಾಗದೇ, ಜೀವನದ ದುಃಖಗಳನ್ನು ಧೈರ್ಯವಾಗಿ ಎದುರಿಸುವ ಮನೋಬಲ ಬರುತ್ತದೆ.  ಅಂತರ್ಗತವಾದ ನಂಬಿಕೆ ಬಂದಿರದಿದ್ದಲ್ಲಿ, ಈ ಸನಾತನ ಗ್ರಂಥಗಳು, ನಮ್ಮ ಜೀವನದಲ್ಲಿಯೇ ಈ ಬೋಧನೆಗಳನ್ನು ಪ್ರಯೋಗಮಾಡಿ, ಇದರ ಮಾನ್ಯತೆಯನ್ನು ಪರೀಕ್ಷಿಸುವ ದಾರಿಯನ್ನೂ ಮಾಡಿಕೊಡುತ್ತವೆ; ಇದರಲ್ಲಿ ಒಂದನ್ನಾದರೂ (ಅಂತರ್ಗತ ನಂಬಿಕೆ ಅಥವಾ ಗ್ರಂಥಗಳ ಮಾನ್ಯತೆ ಪರೀಕ್ಷಣೆ) ಅಳವಡಿಸಿಕೊಳ್ಳದಿದ್ದಲ್ಲಿ, ಆಮೇಲೆ ನಾವೇ ಯಾತನೆಗೊಳಗಾಗಿ, ಜೀವನದ ಪಾಠಗಳನ್ನು ಕಷ್ಟಕರ ರೀತಿಯಲ್ಲಿ ಕಲಿಯಬೇಕಾಗುತ್ತದೆ. 
\end{inspiration}
\newpage


\begin{mananam}{\mananamfont ಮನನ ಶ್ಲೋಕ - ೩೩}
\small \mananamtext ನನಗೆ ಯಾವ ಅಭ್ಯಾಸಗಳನ್ನು ಬಿಡಲು ತುಂಬಾ ಕಷ್ಟಕರವಾಗಿ ಕಾಣುವುದು? ನನಗೆ ಈ ಅಭ್ಯಾಸಗಳ ಋಣಾತ್ಮಕವಾದ ಮುಖದ ಬಗ್ಗೆ ತಿಳಿದಿದೆಯೇ? ಈ ಅಭ್ಯಾಸಗಳ ಮೇಲೆ ಒಂದು ಭಾವನಾತ್ಮಕವಾದ ಅವಲಂಬನೆ ಇದೆಯೇ? ಈ ಅಭ್ಯಾಸಗಳನ್ನು ಸತತವಾಗಿ ಮುರಿಯಲು ಪ್ರಯತ್ನಮಾಡುವ ಧೃಢಸಂಕಲ್ಪ ನನಗಿದೆಯೇ? ಪ್ರಲೋಭೆನೆಗಳ ಆಕ್ರಮಣಗಳನ್ನು ನಿರ್ವಹಿಸುವಲ್ಲಿ ನಾನು ಎಷ್ಟರಮಟ್ಟಿಗೆ ಸಾವಧಾನತೆಯನ್ನು ಕಾಪಾಡಿಕೊಳ್ಳಬಹುದು? ಈ ಅಭ್ಯಾಸಗಳಿಂದ ಮುಕ್ತನಾಗಲು,  ಉತ್ತಮ ಹಂತದಲ್ಲಿ ಕಾರ್ಯನಿರ್ವಹಿಸುವ ಯೋಗದ ಜೀವನಕ್ರಮ ಮತ್ತು ಅಂತಹ ಆಧ್ಯಾತ್ಮಿಕ ಅಭ್ಯಾಸಗಳನ್ನು ನಾನು ಹೇಗೆ ಅಳವಡಿಸಿಕೊಳ್ಳಲಿ?
\end{mananam}
\WritingHand\enspace\textbf{ಆತ್ಮ ವಿಮರ್ಶೆ}\\
\begin{inspiration}{\mananamfont ಸ್ಫೂರ್ತಿ}
\small \mananamtext ಪ್ರಲೋಭನೆಗಳ ಮುಂದೆ ಎಲ್ಲಾ ಬುದ್ಧಿವಂತಿಕೆ ಮತ್ತು ಸಕಾರಾತ್ಮಕ ಉದ್ದೇಶಗಳೂ ವಿಫಲವಾಗುತ್ತವೆ. ವ್ಯಸನದ ಶಕ್ತಿ ಎಷ್ಟೆಂದರೆ, ಅಪಾಯಕಾರಿ ಪರಿಣಾಮಗಳ ಬಗ್ಗೆ ಗೊತ್ತಿದ್ದರೂ ಕೂಡ, ವಸ್ತುಗಳ ಆಕರ್ಷಣೀಯ ಉಪಸ್ಥಿತಿಯಲ್ಲಿ, ಒಬ್ಬ ವ್ಯಸನಿಯು ತನ್ನನ್ನು ತಾನು ನಿಗ್ರಹಿಸಿಕೊಳ್ಳಲು ಸಾಧ್ಯವಾಗುವುದಿಲ್ಲ. ನಿಯಮಿತವಾದ ಆಧ್ಯಾತ್ಮದ ಅಭ್ಯಾಸಗಳಿಂದ, ದೃಢವಾದ ಮನಸ್ಸಿನಿಂದ ಹಾಗೂ ಎಡೆಬಿಡದ ಸಾವಧಾನತೆಯಿಂದ ಮಾತ್ರ, ಪ್ರಲೋಭನೆಗಳಿಂದ ಶಾಶ್ವತವಾಗಿ ಮುಕ್ತವಾಗಲು ಸಾಧ್ಯ.
\end{inspiration}
\newpage

\slcol{\Index{ಇಂದ್ರಿಯಸ್ಯೇಂದ್ರಿಯಸ್ಯಾರ್ಥೇ} ರಾಗದ್ವೇಷೌ ವ್ಯವಸ್ಥಿತೌ ।\\
ತಯೋರ್ನ ವಶಮಾಗಚ್ಛೇತ್ತೌ ಹ್ಯಸ್ಯ ಪರಿಪಂಥಿನೌ ॥ ೩೪ ॥}
\cquote{ಪ್ರತಿಯೊಂದು ಇಂದ್ರಿಯದ ವಿಷಯಗಳಲ್ಲೂ ರಾಗ ದ್ವೇಷಗಳು ನೆಲೆಸಿವೆ. ಅವುಗಳಿಗೆ ಅಡಿಯಾಳಾಗಬಾರದು. ಈ ರಾಗ ದ್ವೇಷಗಳೆ ಸಾಧಕನಿಗೆ ಶತ್ರುಗಳು.}
\slcol{\Index{ಶ್ರೇಯಾನ್ಸ್ವಧರ್ಮೋ ವಿಗುಣಃ} ಪರಧರ್ಮಾತ್ಸ್ವನುಷ್ಠಿತಾತ್ ।\\
ಸ್ವಧರ್ಮೇ ನಿಧನಂ ಶ್ರೇಯಃ ಪರಧರ್ಮೋ ಭಯಾವಹಃ ॥ ೩೫ ॥} 
\cquote{ತನಗೆ ಅಸಹಜವಾದ ಧರ್ಮವನ್ನು ಚೆನ್ನಾಗಿ ನಡೆಸುವುದಕ್ಕಿಂತ ಕಿಂಚಿದೂನವಾದರೂ, ಸಹಜ ಧರ್ಮವನ್ನು ಆಚರಿಸುವುದೇ ಮೇಲು. ತನ್ನ ಧರ್ಮದಲ್ಲಿ ಸಾಯುವುದಾದರೂ ಮೇಲು. ಪರಧರ್ಮವು ಆಪತ್ತಿಗೆ ಆಹ್ವಾನ.}
\slcol{ಅರ್ಜುನ ಉವಾಚ ।\\
\Index{ಅಥ ಕೇನ ಪ್ರಯುಕ್ತೋಽಯಂ} ಪಾಪಂ ಚರತಿ ಪೂರುಷಃ ।\\
ಅನಿಚ್ಛನ್ನಪಿ ವಾರ್ಷ್ಣೇಯ ಬಲಾದಿವ ನಿಯೋಜಿತಃ ॥ ೩೬ ॥} 
\cquote{ಅರ್ಜುನ ಹೇಳಿದನು,\\
ಕೃಷ್ಣ,ಹಾಗಾದರೆ ಈ ಮನುಷ್ಯನು ತನಗೆ ಬೇಡವಾದರೂ ಬಲವಂತಕ್ಕೆ ಬಲಿಯಾದವನಂತೆ ಯಾರಿಂದ ಪ್ರೇರಿತನಾಗಿ ಪಾಪವನ್ನು ಮಾಡುತ್ತಾನೆ?}
\slcol{ಶ್ರೀಭಗವಾನುವಾಚ  ।\\
\Index{ಕಾಮ ಏಷ ಕ್ರೋಧ ಏಷ} ರಜೋಗುಣಸಮುದ್ಭವಃ ।\\
ಮಹಾಶನೋ ಮಹಾಪಾಪ್ಮಾ ವಿದ್ಧ್ಯೇನಮಿಹ ವೈರಿಣಮ್ ॥ ೩೭ ॥} 
\cquote{ಶ್ರೀ ಭಗವಂತನು ಹೇಳಿದನು,\\
ರಜೋಗುಣದಿಂದ ಹುಟ್ಟಿದ, ಸಿಟ್ಟಿಗೂ ತವರಾದ ಈ ಬಯಕೆ ಇದಕ್ಕೆಲ್ಲ ಕಾರಣ.ಎಷ್ಟು ತಿಳಿಸಿದರೂ ಇನ್ನಷ್ಟು ಬೇಕೆನ್ನುವ ಮಹಾ ಪಾಪಿ. ಈ ಬಯಕೆಯನ್ನು ಬಾಳಿನಲ್ಲಿ ದೊಡ್ಡ ಶತ್ರು ಎಂದು ತಿಳಿ.}

\newpage
\begin{mananam}{\mananamfont ಮನನ ಶ್ಲೋಕ - ೩೫}
\small \mananamtext ಈ ಜೀವನದಲ್ಲಿ ನನ್ನ ಯಾವ ಜಾಣ್ಮೆಗಳು ಮತ್ತು ಸಾಮರ್ಥ್ಯಗಳು ಯಾವುವು? ನನ್ನ ಕೆಲಸ ಕಾರ್ಯಗಳಲ್ಲಿ ಮತ್ತು ಚಟುವಟಿಕೆಗಳಲ್ಲಿ ಅವುಗಳನ್ನು ಯಶಸ್ವಿಯಾಗಿ ಉಪಯೋಗಿಸಲು ಸಾಧ್ಯವೇ? ನಾನು ನನ್ನ ವಿಶೇಷಗಳು ಮತ್ತು ಸಾಮರ್ಥ್ಯಗಳನ್ನು, ಕೆಲಸದ ವ್ಯಾಪ್ತಿಯೊಳಗೆ ಹೇಗೆ ಉಪಯೋಗಿಸಬಹುದು? ನಾನು ಮತ್ತೊಂದು ಕ್ಷೇತ್ರದಲ್ಲಿ ಬೇರೆಯೇ  ವೃತ್ತಿಯಲ್ಲಿ ಹೆಚ್ಚು ಯಶಸ್ವಿಯಾಗಬಲ್ಲೆನೆಂಬ ಭಾವನೆ ಇದೆಯೇ? ಅಂತಹ ಬೇರೆಯೇ ವೃತ್ತಿಯ ಕ್ಷೇತ್ರಗಳು ನನ್ನ ಕೈಗೆ ಎಟುಕುವಂತಿವೆಯೇ?
\end{mananam}
\WritingHand\enspace\textbf{ಆತ್ಮ ವಿಮರ್ಶೆ}\\
\begin{inspiration}{\mananamfont ಸ್ಫೂರ್ತಿ}
\small \mananamtext ಈ ಜೀವನದಲ್ಲಿ ನಮ್ಮ ಕರ್ತವ್ಯವನ್ನು ನಿರ್ವಹಿಸುವುದು, ಮಾನಸಿಕ ನಿಯಂತ್ರಣದ ಮೇಲೆ ಗೆಲುವು ಸಾಧಿಸುವ ಬಹಳ ಉತ್ತಮವಾದ ಮಾರ್ಗ. ತನ್ನ ಕರ್ತವ್ಯ ಅಥವಾ ಪಾತ್ರವನ್ನು ಇತರರಿಗೆ ಹೋಲಿಸಿಕೊಳ್ಳುವುದು ವಿವೇಕವಲ್ಲ. ಬೇರೆಯವರ ಪಾತ್ರಗಳಿಗೆ ಹಂಬಲಿಸುವುದಕ್ಕಿಂತ, ನಮ್ಮ ಜವಾಬ್ದಾರಿಗಳನ್ನು ನಡೆಸಿಕೊಂಡು ಹೋಗುವುದರಿಂದ, ಈ ಅವ್ಯಕ್ತ ಪ್ರವೃತ್ತಿಗಳನ್ನು ನೆಮ್ಮದಿಯಾಗಿ, ಮತ್ತೆ ತಲೆ ಎತ್ತದಂತೆ ಸುಟ್ಟುಬಿಡಬಹುದು.
\end{inspiration}
\newpage

\begin{mananam}{\mananamfont ಮನನ ಶ್ಲೋಕ - ೩೬}
\small \mananamtext ನನ್ನ ಒಳ್ಳೆಯ ಉದ್ದೇಶಗಳ ವಿರುದ್ಧವಾಗಿ ನಡೆದುಕೊಳ್ಳುವಂತೆ ಕೆಲವು ಶಕ್ತಿಗಳು ನನ್ನನ್ನು ಬಲವಂತ ಪಡಿಸುತ್ತವೆ ಎಂಬುದನ್ನು ನಾನು ಅರಿತಿದ್ದೇನೆಯೇ? ನನಗೆ ಸುಪ್ತಾವಸ್ಥೆಯ ಸ್ಥಿತಿ ಮತ್ತು ಪ್ರಕಟವಾಗುವ ಸ್ಥಿತಿ ಇವೆರಡರಲ್ಲಿಯೂ ಇರುವ ದುಷ್ಟ ಪ್ರವೃತ್ತಿಯ ಬಗ್ಗೆ ಅರ್ಥವಾಗುತ್ತದೆಯೇ?  ಮನಸ್ಸು ಮತ್ತು ಇಂದ್ರಿಯಗಳ ನಿಗ್ರಹದ ಮೂಲಕ, ಇದರಲ್ಲಿರುವ (ಎರಡೂ ಸ್ಥಿತಿಗಳಲ್ಲಿರುವ) ಅಂಶಗಳ ಬಗ್ಗೆ ನಾನು ಅರಿತುಕೊಂಡಿದ್ದೇನೆಯೇ?  ಇವೆರಡರಲ್ಲಿ – ಮನಸ್ಸಿನ ನಿಗ್ರಹ ಮತ್ತು ಇಂದ್ರಿಯ ನಿಗ್ರಹದ ನಡುವೆ-  ನಿರ್ದಿಷ್ಟ ಪರಿಸ್ಥಿತಿಗಳಲ್ಲಿ ಯಾವುದು ಮೇಲುಗೈ ಸಾಧಿಸುತ್ತದೆ? ‘ಸ್ವ-ನಿಯಂತ್ರಣವೆಂದರೆ’ ನಾನು ಏನೆಂದು ತಿಳಿದಿದ್ದೇನೆ? ನನಗೆ ಇಂದ್ರಿಯಗಳ ಹಂತದಲ್ಲಿ ಪ್ರಲೋಭನೆಗಳ ಶಕ್ತಿಯನ್ನು ಹತೋಟಿಯಲ್ಲಿಡುವ ಸಾಮರ್ಥ್ಯವಿದೆಯೇ? ನನಗೆ ಯೋಚನೆಗಳ ಹಂತದಲ್ಲಿ ಅದನ್ನು ಹತೋಟಿಯಲ್ಲಿಡುವ ಸಾಮರ್ಥ್ಯವಿದೆಯೇ? 
\end{mananam}
\WritingHand\enspace\textbf{ಆತ್ಮ ವಿಮರ್ಶೆ}\\
\begin{inspiration}{\mananamfont ಸ್ಫೂರ್ತಿ}
\small \mananamtext ಜನ್ಮ, ಜನ್ಮಾoತರಗಳಿಂದ ಒಟ್ಟುಗೂಡಿದ, ವಿವಿಧ ರೀತಿಯ ಹಾನಿಕಾರಕ ಪ್ರವೃತ್ತಿಗಳು ಜೀವಮಾನದ ಉದ್ದಕ್ಕೂ, ನಮ್ಮೊಳಗೆ ಯಾವಾಗಲೂ ಸುಪ್ತ ರೀತಿಯಲ್ಲಿ ಸುಳಿಯುತ್ತಿರುತ್ತವೆ ಹಾಗೂ ಘರ್ಷಣೆ ಉಂಟು ಮಾಡುತ್ತಿರುತ್ತವೆ. ಅತಿಯಾದ ಆತ್ಮವಿಶ್ವಾಸದಿಂದ, ‘ಇಂದ್ರಿಯ ಪ್ರಚೋದನೆಗಳ ಮೇಲೆ ನನ್ನ ಪ್ರಭುತ್ವ ಇದೆ’ ಎoದು ತಿಳಿದು, ಯಾವಾಗಲೂ ಸಣ್ಣ, ಸೂಕ್ಷ್ಮವಾದ ಮಟ್ಟದಲ್ಲಿಯೂ ಪ್ರಲೋಭನೆಗಳಿಗೆ ತುತ್ತಾಗುವುದು ಎಂದಿಗೂ ವಿವೇಕವಲ್ಲ.
\end{inspiration}
\newpage

\begin{mananam}{\mananamfont ಮನನ ಶ್ಲೋಕ - ೩೭}
\small \mananamtext ನನ್ನ ಇಚ್ಛೆ ಮತ್ತು ಹಂಬಲಗಳ ಲಕ್ಷಣಗಳೇನು? ಹಾಗೂ ಅವುಗಳ ಮೂಲವನ್ನು ಗುರುತಿಸಬಲ್ಲೆನೇ? ನನ್ನ ಇಚ್ಛೆಗಳಿಗೆ ಅಡಚಣೆಯಾದಾಗ ಏನಾಗುತ್ತದೆ? ನನಗೆ ಕೋಪದ ಭಾವನೆ ಬರುತ್ತದೆಯೇ? ನನಗೆ ಭಯ ಅಥವಾ ವ್ಯಾಕುಲತೆ ಆಗುತ್ತದೆಯೇ? ನನ್ನ ಇಚ್ಛೆಗಳು ನನಗೆ, ಒಂದು ಸೀಮಿತ ಮತ್ತು ಮೋಹದ ಭಾವನೆ ಕೊಡುತ್ತವೆಯೇ? ಅವುಗಳು ಕೇವಲ ನನ್ನ ವೈಯಕ್ತಿಕ ಸಂತೋಷಕ್ಕೆ ಮಾತ್ರ ನಿರ್ದಿಷ್ಟವಾಗಿದೆಯೆ ಅಥವಾ, ಬೇರೆಯವರ ಒಳಿತನ್ನೂ ಒಳಗೊಂಡಿದೆಯೇ? ಇದರಲ್ಲಿ ಆರೋಗ್ಯಕರ ಇಚ್ಛೆಗಳು ಮತ್ತು ಅನಾರೋಗ್ಯಕರ ಇಚ್ಛೆಗಳಂತಹವು ಇರಲು ಸಾಧ್ಯವೇ? 
\end{mananam}
\WritingHand\enspace\textbf{ಆತ್ಮ ವಿಮರ್ಶೆ}\\
\begin{inspiration}{\mananamfont ಸ್ಫೂರ್ತಿ}
\small \mananamtext ನಮ್ಮ ನಿಜವಾದ ಗುಣ ‘ಅನಂತತೆ’. ನಾವು ಅದನ್ನು ಮರೆತು,ನಮ್ಮನ್ನು ಸಂಪೂರ್ಣವಾಗಿಸಲು, ಪ್ರಾಪಂಚಿಕ ವಸ್ತುಗಳ ಹಿಂದೆ ಹೋಗುತ್ತೇವೆ. ಯಾವಾಗ ನಾವು ಒಂದು ಪರಿಮಿತಿಯ ಭಾವನೆಗೆ ಒಳಗಾಗುತ್ತೇವೆಯೋ ಆವಾಗ, ಸ್ವಾರ್ಥಯುತ ಇಚ್ಛೆಗಳು ಆವಿರ್ಭಸುತ್ತವೆ; ಇವು ನಮ್ಮನ್ನು ಇತರರಿಗಿಂತ ಹೆಚ್ಚು ದೊಡ್ಡವರು ಹಾಗೂ ಉತ್ತಮರನ್ನಾಗಿ (ಸಮಾಜದ ಕಣ್ಣಿನಲ್ಲಿ) ಮಾಡಬಹುದು ಎಂದು ಆಶಿಸುತ್ತೇವೆ. ಆಸೆ ಮತ್ತು ಕೋಪ, ಇವೆರಡೂ ನಮ್ಮನ್ನು ಈ ಪ್ರಾಪಂಚಿಕ ಮಾಯದಲ್ಲಿ ಬಂಧಿಸುತ್ತವೆ.
\end{inspiration}
\newpage

\slcol{\Index{ಧೂಮೇನಾವ್ರಿಯತೇ} ವಹ್ನಿರ್ಯಥಾದರ್ಶೋ ಮಲೇನ ಚ ।\\
ಯಥೋಲ್ಬೇನಾವೃತೋ ಗರ್ಭಸ್ತಥಾ ತೇನೇದಮಾವೃತಮ್ ॥ ೩೮ ॥}
\cquote{ ಬೆಂಕಿಗೆ ಹೊಗೆಯ ಮುಸುಕು. ಕನ್ನಡಿಗೆ ಕೊಳೆಯ ಮುಸುಕು. ಗರ್ಭಕ್ಕೆ ಕೋಶದ ಮುಸುಕು. ಹಾಗೆ ಜಗತ್ತಿಗೆಲ್ಲ ಕಾಮದ ಮುಸುಕು.}
\slcol{\Index{ಆವೃತಂ ಜ್ಞಾನಮೇತೇನ} ಜ್ಞಾನಿನೋ ನಿತ್ಯವೈರಿಣಾ ।\\
ಕಾಮರೂಪೇಣ ಕೌಂತೇಯ ದುಷ್ಪೂರೇಣಾನಲೇನ ಚ ॥ ೩೯ ॥} 
\cquote{ಅರ್ಜುನ,ತಿಳಿದವರ ನಿತ್ಯಶತ್ರುವಾದ, ತೃಪ್ತಿಯನ್ನೇ ಅರಿಯದ ಈ ಕಾಮದ ಮುಸುಕಿನಿಂದ ತಿಳಿವು ಮರೆಯಾಗಿದೆ.}
\slcol{\Index{ಇಂದ್ರಿಯಾಣಿ ಮನೋ} ಬುದ್ಧಿರಸ್ಯಾಧಿಷ್ಠಾನಮುಚ್ಯತೇ ।\\
ಏತೈರ್ವಿಮೋಹಯತ್ಯೇಷ ಜ್ಞಾನಮಾವೃತ್ಯ ದೇಹಿನಮ್ ॥ ೪೦ ॥}
\cquote{ಇಂದ್ರಿಯಗಳು, ಮನಸ್ಸು, ಬುದ್ದಿ ಇವೇ ಕಾಮದ ವಾಸಸ್ಥಾನ.ಇದು ಇವುಗಳ ಮೂಲಕ ತಿಳಿವನ್ನು ಮರೆಮಾಡಿ ಮನುಷ್ಯನನ್ನು ಮಂಕು ಗೊಳಿಸುತ್ತದೆ.}
\slcol{\Index{ತಸ್ಮಾತ್ತ್ವಮಿಂದ್ರಿಯಾಣ್ಯಾದೌ} ನಿಯಮ್ಯ ಭರತರ್ಷಭ ।\\
ಪಾಪ್ಮಾನಂ ಪ್ರಜಹಿ ಹ್ಯೇನಂ ಜ್ಞಾನವಿಜ್ಞಾನನಾಶನಮ್ ॥ ೪೧ ॥} 
\cquote{ಅರ್ಜುನ, ಆದುದರಿಂದ ನೀನು ಮೊದಲು ಇಂದ್ರಿಯಗಳನ್ನು ಬಿಗಿಹಿಡಿದು ಜ್ಞಾನವನ್ನೂ ಅನುಭವವನ್ನೂ ಹಾಳು ಮಾಡುವ ಈ ಪಾಪಿಯನ್ನು ಗೆಲ್ಲು.}


\newpage
\begin{mananam}{\mananamfont {ಮನನ ಶ್ಲೋಕ - ೩೯, ೪೦}}
\small \mananamtext ನನ್ನ ಆಳವಾದ ತಿಳುವಳಿಕೆಯನ್ನು ಮತ್ತು ಉದ್ದೇಶವನ್ನು ಏನು ಮುಸುಕಿದೆ? ನನಗೆ, ನನ್ನ ಅಭಿವೃದ್ಧಿಯ ಮಹತ್ವಾಕಾಂಕ್ಷೆಯು ಇದ್ದಾಗ್ಯೂ ಕೂಡ, ನಾನು ಏಕೆ ಪ್ರಲೋಭನೆ ಮತ್ತು ಸೋಮಾರಿತನಕ್ಕೆ ಈಡಾಗುತ್ತಿದ್ದೇನೆ? ಸದಾ, ನನ್ನ ಉನ್ನತ ಗುರಿಯನ್ನು ಸಾಧಿಸುವತ್ತ ಹರಿಸಬೇಕಾದ ನನ್ನ ನೆನಪನ್ನು ಏಕೆ ಕಳೆದುಕೊಳ್ಳುತ್ತೇನೆ?\\
ಇಂದ್ರಿಯ ಸುಖಗಳನ್ನು ಬೆನ್ನಟ್ಟುವುದರಿಂದ ಸಿಗುವ ಸಂತೋಷವು ಶಾಶ್ವತವೇ? ಇದರಿಂದ ಇಚ್ಛೆ ಪೂರೈಕೆಯಾದಮೇಲೆ, ನನ್ನ ದೇಹ ಮತ್ತು ಮನಸ್ಸಿನ ಮೇಲೆ ಯಾವ ರೀತಿಯ ಋಣಾತ್ಮಕ ಪರಿಣಾಮವಾಗುತ್ತದೆ? 
\end{mananam}
\WritingHand\enspace\textbf{ಆತ್ಮ ವಿಮರ್ಶೆ}\\
\begin{inspiration}{\mananamfont ಸ್ಫೂರ್ತಿ}
\small \mananamtext ಮಾತ್ರ ಬುದ್ಧಿಶಕ್ತಿಯ ಹಂತದಲ್ಲಿ, ವಿವೇಕದಿಂದ ಪರಿವರ್ತನೆಯನ್ನು ತರಲಾಗುವುದಿಲ್ಲ; ಅದನ್ನು ನಮ್ಮ ಹೃದಯದ ಮೂಲಕ ಶೋಧಿಸಿದ ನಂತರ ಅದು, ನಮ್ಮ ಜ್ಞಾನೇಂದ್ರಿಯಗಳ ಹಾಗೂ ಕರ್ಮೇoದ್ರಿಯಗಳ ಮೂಲಕ  ವ್ಯಕ್ತವಾಗಬೇಕು. ನಾವು ಮಾಡುವ ಕಾರ್ಯಗಳಲ್ಲಿ ಶುದ್ಧತೆ ಇದ್ದರೆ ಅದು, ನಮ್ಮ ಹೃದಯದ ತಪ್ಪು ಪ್ರವೃತ್ತಿಗಳ ಮೂಲವನ್ನು  ಶುದ್ಧೀಕರಿಸುತ್ತದೆ. 
\end{inspiration}
\newpage

\newpage
\begin{mananam}{\mananamfont ಮನನ ಶ್ಲೋಕ - ೪೧}
\small \mananamtext ನನ್ನ ಜೀವನದ ಯಾವ ಕ್ಷೇತ್ರದಲ್ಲಿ ಸ್ವ–ನಿಯಂತ್ರಣದ ಅಗತ್ಯತೆ ಇದೆ? ನೋಡುವುದು, ಕೇಳುವುದು, ಆಸ್ವಾದಿಸುವುದು ಮುಂತಾದ, ನನ್ನ ಜ್ಞಾನೇಂದ್ರಿಯಗಳನ್ನು ಹತೋಟಿಯಲ್ಲಿಡುವ ಸಾಮರ್ಥ್ಯ ನನಗಿದೆಯೇ? ನನಗೆ,  ಮಾತನಾಡುವುದು, ನಡೆಯುವುದು, ಲೈಂಗಿಕತೆ ಮುಂತಾದ ಕರ್ಮೇoದ್ರಿಯಗಳನ್ನು  ಹತೋಟಿಯಲ್ಲಿಡುವ ಸಾಮರ್ಥ್ಯ ಇದೆಯೇ? ಈ ನಿರ್ಬಂಧನೆಗಳನ್ನು ದಿನನಿತ್ಯದಲ್ಲಿ ಅನುಷ್ಠಾನಗೊಳಿಸಲು ನಾನು ಯಾವ ಮೊದಲ ಹೆಜ್ಜೆಯನ್ನು ತೆಗೆದುಕೊಳ್ಳಬೇಕು? ಈ ‘ಸ್ವ-ನಿಯಂತ್ರಣವು’ ನನ್ನ ಜೀವನದಲ್ಲಿ ಕಡುಕಷ್ಟ ಬಂದಾಗಲಷ್ಟೇ ನಿಜವಾದ ಪರೀಕ್ಷೆಯ ಒರೆಗಲ್ಲಿಗೆ ಒಳಪಡುತ್ತದೆ ಎಂಬುದನ್ನು ಗ್ರಹಿಸಿದ್ದೇನೆಯೇ? 
\end{mananam}
\WritingHand\enspace\textbf{ಆತ್ಮ ವಿಮರ್ಶೆ}\\
\begin{inspiration}{\mananamfont ಸ್ಫೂರ್ತಿ}
\small \mananamtext ಯೋಗವು ಅಶಿಸ್ತಿನ ಇಂದ್ರಿಯಗಳನ್ನು ನಿಗ್ರಹಿಸುವ ಕಲೆಯಾಗಿದೆ. ಯಾವಾಗ ಇಂದ್ರಿಯಗಳನ್ನು ನಿಯಂತ್ರಿಸಲಾಗುತ್ತದೆಯೋ ಆವಾಗ, ವಿವೇಕವನ್ನು ಅಜ್ಞಾನವು ಆವರಿಸುವುದಿಲ್ಲ; ಸರಿಯಾದ ಕಾರ್ಯವನ್ನು ಮಾಡಲು, ನಮಗೆ ಮಾರ್ಗದರ್ಶನ ನೀಡುವ ಅಗತ್ಯವಿರುವ ಸಮಯದಲ್ಲಿ ಅದು (ವಿವೇಕ), ನಮಗೆ ಲಭ್ಯವಾಗುತ್ತದೆ.
\end{inspiration}
\newpage

\slcol{\Index{ಇಂದ್ರಿಯಾಣಿ ಪರಾಣ್ಯಾಹು}ರಿಂದ್ರಿಯೇಭ್ಯಃ ಪರಂ ಮನಃ ।\\
ಮನಸಸ್ತು ಪರಾ ಬುದ್ಧಿರ್ಯೋ ಬುದ್ಧೇಃ ಪರತಸ್ತು ಸಃ ॥ ೪೨ ॥}
\cquote{ಸ್ತೂಲ ದೇಹಕ್ಕಿಂತ ಇಂದ್ರಿಯಗಳು ಹೆಚ್ಚಿನವು ಎನ್ನುವರು. ಇಂದ್ರಿಯಗಳಿಗಿಂತಲೂ ಮನಸ್ಸು ಹೆಚ್ಚಿನದು.ಮನಸ್ಸಿಗಿಂತಲೂ ಬುದ್ಧಿ ಹೆಚ್ಚಿನದು. ಬುದ್ಧಿಗೂ ನಿಲುಕದೆ ಅದರಾಚೆ ಇರುವವನೆ  ಭಗವಂತ.}
\slcol{\Index{ಏವಂ ಬುದ್ಧೇಃ ಪರಂ} ಬುದ್ಧ್ವಾ ಸಂಸ್ತಭ್ಯಾತ್ಮಾನಮಾತ್ಮನಾ ।\\
ಜಹಿ ಶತ್ರುಂ ಮಹಾಬಾಹೋ ಕಾಮರೂಪಂ ದುರಾಸದಮ್ ॥ ೪೩ ॥}
\cquote{ಅರ್ಜುನ, ಹೀಗೆ ಬುದ್ಧಿಗೂ ನಿಲುಕದ ಹಿರಿಯ ತತ್ವವನ್ನು ತಿಳಿದು ಪ್ರಯತ್ನದಿಂದ ಮನಸ್ಸನ್ನು ನಿಯಂತ್ರಿಸಿ ಕಾಮವೆಂಬ ಕೆಟ್ಟ ಶತ್ರುವನ್ನು ಹೋಗಲಾಡಿಸು.}

\newpage
\begin{mananam}{\mananamfont {ಮನನ ಶ್ಲೋಕ - ೪೨, ೪೩}}
\small \mananamtext ‘ಸೂಕ್ಷ್ಮವು ಶ್ರೇಷ್ಠವಾದುದು ಮತ್ತು ಅದು ಸ್ಥೂಲವಾದುದನ್ನು ನಿಯಂತ್ರಿಸುತ್ತದೆ’ ಎಂಬುದು ನನಗೆ ಏನು  ಅರ್ಥ ಕೊಡುತ್ತದೆ? ನನ್ನ ಇಂದ್ರಿಯಗಳು ಹೇಗೆ ನನ್ನ ದೇಹವನ್ನು ನಿರ್ದೇಶಿಸುತ್ತವೆ, ಎಂಬ ಬಗ್ಗೆ ನನ್ನ ಗಮನವಿದೆಯೇ? ಜಾಗ್ರತಾವಸ್ಥೆಯಲ್ಲಿ ನನ್ನ ಮನಸ್ಸು ಇಂದ್ರಿಯ ಕ್ರಿಯೆಯನ್ನು ನಿರ್ದೇಶಿಸಿ, ಅದೇ ಮನಸ್ಸು ಸ್ವತಃ ವಿಶ್ರಾಂತಿಯಲ್ಲಿದ್ದಾಗ ಇಂದ್ರಿಯಗಳನ್ನೂ ನಿಷ್ಕ್ರಿಯಗೊಳಿಸುವುದಿಲ್ಲವೇ?
ದೇಹ, ಇಂದ್ರಿಯಗಳು, ಮನಸ್ಸು ಮತ್ತು ಬುದ್ಧಿ ಇವುಗಳಿಗಿಂತ ಶ್ರೇಷ್ಠವಾದ ಅಂತರಾತ್ಮನ, ಅಂತರಂಗ ಸ್ವಭಾವ ಮತ್ತು ಅತ್ಯುನ್ನತ ಸ್ವಭಾವವೆಂದರೆ ನಾನು ಏನೆಂದು ಅರಿತಿದ್ದೇನೆ? ಇದನ್ನು ತಿಳಿಯುವುದರಿಂದ, ನನ್ನ ಜೀವನದ ದೃಷ್ಟಿಕೋನದಲ್ಲಿ ಮತ್ತು  ಪ್ರಾಪಂಚಿಕ ಕಾರ್ಯ ನಿರ್ವಹಿಸುವಲ್ಲಿ,  ಅದು ಬದಲಾವಣೆಗೆ ಹೇಗೆ ಕಾರಣವಾಗುತ್ತದೆ?
\end{mananam}
\WritingHand\enspace\textbf{ಆತ್ಮ ವಿಮರ್ಶೆ}\\
\begin{inspiration}{\mananamfont ಸ್ಫೂರ್ತಿ}
\small \mananamtext ನಮ್ಮ ಚೈತನ್ಯವೇ ನಮ್ಮೊಳಗಿನ ಅತ್ಯಂತ ನೇರ ಮತ್ತು ಶ್ರೇಷ್ಠ ಅಸ್ತಿತ್ವವಾಗಿದೆ.\\
ಚೈತನ್ಯವು ವಿಷಯಾತೀತ, ಶುದ್ಧ ಮತ್ತು ನಿಜವಾದ ಆತ್ಮಸ್ವರೂಪ; ಆತ್ಮನ  (ದೃಷ್ಟ) ದೃಷ್ಟಿಯಿಂದ ನೋಡಿದಾಗ ಬುದ್ಧಿ, ಮನಸ್ಸು, ಇಂದ್ರಿಯಗಳು ಮತ್ತು ದೇಹ ಇವುಗಳೆಲ್ಲ (ದೃಶ್ಯ) ವಿಷಯಗಳು,ವಸ್ತುಗಳು.\\
ಕಾಮನೆಗಳ ಪೂರಣೆಗಾಗಿ, ವಿಷಯಾತೀತನಾದ ಆತ್ಮನು ತನ್ನನ್ನು ತಾನು ಸೂಕ್ಷ್ಮ ಹಾಗೂ ಸ್ಥೂಲ ರೂಪದಲ್ಲಿರುವ  ವಸ್ತುಗಳಲ್ಲಿ, ತಪ್ಪಾಗಿ ವಸ್ತುನಿಷ್ಠವಾಗಿ ಗುರುತಿಸುಕೊಳ್ಳುವುದೋ ಎಂದು ತೋರುತ್ತದೆ. ಈ ಕಾಮನೆಯ ಅರಸುವಿಕೆಯು ಸತ್ಯಾತ್ಮನನ್ನು ತನ್ನ ನಿಜ ಸ್ವಭಾವದಿಂದ ದೂರ ಸೆಳೆಯುತ್ತದೆ. 
\end{inspiration}
\newpage

\chapEndSloka{ಕರ್ಮಯೋಗ}

\chapter{\kanfont ಪದ ಕೋಶ}
%Term definitions
\mananamtext{
\begin{description}
   \item[ಧರ್ಮ] ಧರ್ಮವೆಂಬುವುದು ಸಾಮಾನ್ಯತಃ, ಒಬ್ಬ ವ್ಯಕ್ತಿಯ ಜೀವನದಲ್ಲಿ, ಸಮಾಜದಲ್ಲಿ ಹಾಗೂ ಪ್ರಕೃತಿಯ ನಿಯಮಕ್ಕೆ ಹೊಂದಿಕೊಂಡು ಹೋಗುವಂತಹ, ಅವನ ನೈತಿಕ ಬಾಧ್ಯತೆ ಕರ್ತವ್ಯಗಳ ಮಹತ್ವವನ್ನು ಸೂಚಿಸುತ್ತದೆ. ಗೀತೆ ಮತ್ತು ವೇದಗಳ ದೃಷ್ಟಿಕೋನದಿಂದ ನೋಡಿದರೆ, ಕೇವಲ ಲೌಕಿಕ ಲಾಭಗಳಿಗೆ ಮಾತ್ರವಲ್ಲದೇ, ಆಧ್ಯಾತ್ಮಿಕ ಜೀವನ ಮತ್ತು ಮೋಕ್ಷದ ಅನ್ವೇಷಣೆಗಾಗಿ ಧರ್ಮದ ಪರಿಪಾಲನೆ ಮಾಡುವುದೇ ಆಗಿದೆ. ಅಹಂ ಮತ್ತು ಅದರ ಇಷ್ಟಾನಿಷ್ಟಗಳನ್ನು ಮೆಟ್ಟಿ, ಈ ಸೃಷ್ಟಿಯ ದೈವೀಕ ಯೋಜನೆಗೆ ಅನುಗುಣವಾಗಿ ಜೀವಿಸುವುದೇ ಧರ್ಮದ ಕರೆಯಾಗಿದೆ.
   \item[ಭಗವಂತ] ಯಾವ ವ್ಯಕ್ತಿಗೆ ಸ್ಥಿರವಾದ ವಿವೇಕ ಮತ್ತು ಬುದ್ಧಿಶಕ್ತಿ ಇರುವುದೋ, ಯಾರು ಸದಾ ಸಮಚಿತ್ತತೆಯಿಂದ ಇರುವನೋ, ಯಾರ ಇಂದ್ರಿಯಗಳು ಅವನ ಹಿಡಿತದಲ್ಲಿರುವುವೋ, ಯಾರು ಆಸೆ ಮತ್ತು ಭಯದಿಂದ ಮುಕ್ತನೋ ಹಾಗೂ, ಜೀವನದಲ್ಲಿ ಸಂತೋಷ ಬಂದಾಗ ತೀರಾ ಹಿಗ್ಗದೇ, ದುಃಖ ಬಂದಾಗ ಹತಾಶನಾಗದೇ ಇರುವನೋ,ಅಂಥವನು ‘ಸ್ಥಿತಪ್ರಜ್ಞ’ ; ಅಲ್ಲದೇ, ಯಾವನು ತನ್ನ ಆತ್ಮದಲ್ಲಿಯೇ ಸಂತೃಪ್ತನೋ, ಹಾಗೂ ಸಾಮಾನ್ಯತಃ, ಆತ್ಮಜ್ಞಾನವಿಲ್ಲದ ಒಬ್ಬನಿಗೆ ಕಾಡುವ ಅಪೂರ್ಣತೆಯ ಭಾವನೆ ಯಾವನಿಗೆ ಇಲ್ಲವೋ, ಅವನೇ ‘ಸ್ಥಿತಪ್ರಜ್ಞ’ ಎಂದೆನಿಸಿಕೊಳ್ಳುತ್ತಾನೆ.
   \item[ಸ್ಥಿತಪ್ರಜ್ಞ] ಯಾವ ವ್ಯಕ್ತಿಗೆ ಸ್ಥಿರವಾದ ವಿವೇಕ ಮತ್ತು ಬುದ್ಧಿಶಕ್ತಿ ಇರುವುದೋ, ಯಾರು ಸದಾ ಸಮಚಿತ್ತತೆಯಿಂದ ಇರುವನೋ, ಯಾರ ಇಂದ್ರಿಯಗಳು ಅವನ ಹಿಡಿತದಲ್ಲಿರುವುವೋ, ಯಾರು ಆಸೆ ಮತ್ತು ಭಯದಿಂದ ಮುಕ್ತನೋ ಹಾಗೂ, ಜೀವನದಲ್ಲಿ ಸಂತೋಷ ಬಂದಾಗ ತೀರಾ ಹಿಗ್ಗದೇ, ದುಃಖ ಬಂದಾಗ ಹತಾಶನಾಗದೇ ಇರುವನೋ,ಅಂಥವನು ‘ಸ್ಥಿತಪ್ರಜ್ಞ’ ; ಅಲ್ಲದೇ, ಯಾವನು ತನ್ನ ಆತ್ಮದಲ್ಲಿಯೇ ಸಂತೃಪ್ತನೋ, ಹಾಗೂ ಸಾಮಾನ್ಯತಃ, ಆತ್ಮಜ್ಞಾನವಿಲ್ಲದ ಒಬ್ಬನಿಗೆ ಕಾಡುವ ಅಪೂರ್ಣತೆಯ ಭಾವನೆ ಯಾವನಿಗೆ ಇಲ್ಲವೋ, ಅವನೇ ‘ಸ್ಥಿತಪ್ರಜ್ಞ’ ಎಂದೆನಿಸಿಕೊಳ್ಳುತ್ತಾನೆ.
\end{description}
}
\printindex
\newpage% move to the next empty page
\ifodd\thepage% if that page is odd,
    \thispagestyle{empty}% supress page num and headers for page to be left empty
       \ \clearpage% add a protected space to force tex to move to the next page
	   %\thispagestyle{empty}
	   \newpage
	   \AddToShipoutPictureBG*{%
    \AtPageLowerLeft{%
        \includegraphics[width=\paperwidth,height=\paperheight]{./images/page02.jpg}%
    }%
	}
   \else% do nothing
   \AddToShipoutPictureBG*{%
    \AtPageLowerLeft{%
        \includegraphics[width=\paperwidth,height=\paperheight]{./images/page02.jpg}%
    }%
	}
\fi
\begin{titlepage}
	%\pagecolor{pastelblue}
	\AddToShipoutPictureBG*{%
    \AtPageLowerLeft{%
        \includegraphics[width=\paperwidth,height=\paperheight]{./images/backcover.png}%
    }%
}
    \begin{center}
        \vspace*{0.5cm}
            
        {\Huge
        %\textbf{\color{white}\fontsize{50}{60}\selectfont ಗೀತಾ ಮನನಂ}
		}
        %\textbf{\\ \small \color{white}ದೈನಂದಿನ ಸ್ಪೂರ್ತಿ ಹಾಗೂ ಆತ್ಮಾವಲೋಕನಕ್ಕಾಗಿ}    
        \vspace{1.0cm}
            
        
		
            
        \vfill
            
        
            
        \vspace{0.1cm}
        {\color{white}    
		%\textbf{{\Large \mananamfont ಸ್ವಾಮಿ ನಿರ್ಗುಣಾನಂದ ಗಿರಿ}}\\
		%{\normalsize Swami Nirgunananda Giri\\Rishikesh, India}
        }
    \end{center}
\end{titlepage}
\nopagecolor% Use this to restore the color pages to white
\end{document}
