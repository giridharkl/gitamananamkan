\slcol{ಅರ್ಜುನ ಉವಾಚ ।\\
\Index{ಜ್ಯಾಯಸೀ ಚೇತ್ಕರ್ಮಣಸ್ತೇ} ಮತಾ ಬುದ್ಧಿರ್ಜನಾರ್ದನ ।\\
ತತ್ಕಿಂ ಕರ್ಮಣಿ ಘೋರೇ ಮಾಂ ನಿಯೋಜಯಸಿ ಕೇಶವ ॥ ೧ ॥}
\cquote{ಅರ್ಜುನನು ಹೇಳಿದನು,
ಜನಾರ್ದನಾ!  ಕರ್ಮಕ್ಕಿಂತ ಜ್ಞಾನವು ಶ್ರೇಷ್ಠವೆಂದು ನಿನ್ನ ಅಭಿಪ್ರಾಯವಾದರೆ, ಹೇ ಕೇಶವಾ!  ಭಯಂಕರವಾದ ಈ ಯುದ್ಧ ಕರ್ಮಕ್ಕೆ ನನ್ನನ್ನು ಏತಕ್ಕೆ ಪ್ರೆರೇಪಿಸುವೆ ?\\}
\slcol{\Index{ಕೋವ್ಯಾಮಿಶ್ರೇಣಿವ ವಾಕ್ಯನಾ} ಬುದ್ಧಿಂ ಮೋಹಯಸೀವ ಮೇ । \\
ತದೇಕಂ ವದ ನಿಶ್ಚಿತ್ಯ ಯೇನ ಶ್ರೇಯೋಹ ಮಾಪ್ನುಯಮ್ ॥ ೨ ॥}
\cquote{ನಿನ್ನ ಸಂಧಿಗ್ಧ ವಚನಗಳಿಂದ ನನ್ನ ಬುದ್ಧಿಯು ಮೋಹಗೊಂಡಿದೆ. ನನಗೆ ಶ್ರೇಯಸ್ಸನ್ನುಂಟುಮಾಡುವ ಮಾರ್ಗ ಒಂದೇ ಒಂದನ್ನು ಖಂಡಿತವಾಗಿ ತಿಳಿಸು.\\}
\slcol{ಶ್ರೀ ಭಗವಾನ್ ಉವಾಚ\\
\Index{ಲೋಕೇऽಸ್ಮಿನ್  ದ್ವಿವಿಧಾ ನಿಷ್ಠಾ} ಪುರಾ ಪ್ರೋಕ್ತಾ ಮಯಾನಘ ।\\
ಜ್ಞಾನಯೋಗೇನ ಸಾಂಖ್ಯಾನಾಂ ಕರ್ಮಯೋಗೇನ ಯೋಗಿನಾಮ್ ॥ ೩ ॥}
\cquote{ಭಗವಂತನು ಹೇಳಿದನು,\\
ಅರ್ಜುನ, ಈ ಲೋಕದಲ್ಲಿ ನಾನು ಹಿಂದೆಯೇ ಎರಡು ಸ್ಥಿತಿಗಳನ್ನು ಹೇಳಿರುವೆನು. ಜ್ಞಾನಿಗಳಿಗೆ ಜ್ಞಾನ ಯೋಗ, ಸಾಧಕರಿಗೆ ಕರ್ಮ ಯೋಗ.\\}
\slcol{\Index{ನ ಕರ್ಮಣಾಮನಾರಂಭಾ}ನ್ನೈಷ್ಕರ್ಮ್ಯಂ ಪುರುಷೋऽಶ್ನುತೇ ।\\
ನ ಚ ಸಂನ್ಯಸನಾದೇವ ಸಿದ್ಧಿಂ ಸಮಧಿಗಚ್ಛತಿ ॥ ೪ ॥}
\cquote{ಕೆಲಸ ಮಾಡದೆ ಇರುವುದರಿಂದ ಕರ್ಮ ಸಂಬಂಧವನ್ನು ತೊಡೆದುಹಾಕಲಾಗುವುದಿಲ್ಲ. ಕರ್ಮಗಳನ್ನು ಬಿಟ್ಟ ಮಾತ್ರದಿಂದಲೇ ಸಿದ್ಧಿಯನ್ನು ಪಡೆಯುವುದು ಸಾಧ್ಯವಿಲ್ಲ.\\}

\newpage
\begin{mananam}{\mananamfont ಮನನ ಶ್ಲೋಕ - ೧}
\mananamtext ಜೀವನದಲ್ಲಿ ನಾನು ಚಟುವಟಿಕೆಯಿಂದ ಇರಬೇಕೋ ಬೇಡವೋ ಎನ್ನುವ ಅನುಮಾನದಲ್ಲಿದ್ದೇನೆಯೇ? ಪ್ರಾಪಂಚಿಕ ವಿಷಯ ಮತ್ತು ವ್ಯವಹಾರಗಳಲ್ಲಿ  ನಾನು ಅರೆ ಮನಸ್ಸಿನಿಂದ ತೊಡಗಿಸಿಕೊಂಡಿದ್ದೇನೆಯೇ? ನಾನು ಸ್ವತಃಕ್ಕಾಗಿ ಏಕಾಂತತೆಯನ್ನು ಬಯಸುತ್ತೇನೆಯೇ? ಅಥವಾ ಜೀವನದ ಎಲ್ಲಾ ಚಟುವಟಿಕೆಗಳಿಂದ ದೂರವಾಗಿ, ಆಳವಾಗಿ ಆಧ್ಯಾತ್ಮಿಕತೆಯಲ್ಲಿ ತೊಡಗಿಸಿಕೊಳ್ಳಬೇಕೆಂಬ ಭಾವನೆ ಕಾಡುತ್ತದೆಯೇ? ಹಾಗಿದ್ದಲ್ಲಿ, ಜೀವನದ ಕಷ್ಟ ಮತ್ತು ಜವಾಬ್ದಾರಿಗಳನ್ನು ಎದುರಿಸಲಾರದೇ, ಅದರಿಂದ ಪಲಾಯನ ಮಾಡುವುದಕ್ಕಾಗಿ ಈ ತರಹದ ಇಚ್ಛೆ ಉತ್ಪನ್ನವಾಗಿದೆಯೇ? ಈ ನಿಷ್ಕ್ರಿಯತೆಯ ಪ್ರಲೋಭನೆಯು, ದೈಹಿಕ ಮತ್ತು ಮಾನಸಿಕ ನಿರುತ್ಸಾಹದಿಂದಾಗಿ ಉತ್ಪನ್ನವಾಗಿದೆಯೇ?
\end{mananam}
\WritingHand\enspace\textbf{ಆತ್ಮ ವಿಮರ್ಶೆ}\\
\begin{inspiration}{\mananamfont ಸ್ಪೂರ್ತಿ}
\mananamtext ‘ನಿಜವಾದ ಜ್ಞಾನ’ ನಮಗೆ ಸರಿಯಾದ ದಾರಿಯಲ್ಲಿ ನಡೆಯಲು ಶಕ್ತಿ ತುಂಬುತ್ತದೆ. ‘ಆತ್ಮವಿದ್ಯೆ’ಯು ನಮಗೆ, ಅಹಂನನ್ನು ಮೀರಿ ನಿಲ್ಲಲು ಸಹಾಯ ಮಾಡುತ್ತದೆ; ಆಗ ಕ್ರಿಯೆಯು (ಭೌತಿಕ, ಮಾನಸಿಕ ಹಾಗೂ ಆಧ್ಯಾತ್ಮಿಕ ಕಾರ್ಯ) ಯಾವ ಅಡಚಣೆಯೂ ಇಲ್ಲದೆ ಪ್ರವಹಿಸುತ್ತದೆ. ಹೀಗೆ ಉಂಟಾದ ಕ್ರಿಯೆಯು ತುಂಬಾ ಉತ್ಸಾಹದಾಯಕವಾಗಿದ್ದು, ನವ ಚೈತನ್ಯವನ್ನು ತುಂಬುವುದಲ್ಲದೇ, ನಮ್ಮ ಜೀವನದ ಯಾವ ಕ್ಷೇತ್ರದಲ್ಲಾದರೂ ಯಶಸ್ಸಿನ ಹಾದಿಯಲ್ಲಿ ಮುನ್ನಡೆಸುತ್ತದೆ.  
\end{inspiration}
\newpage

\newpage
\begin{mananam}{\mananamfont ಮನನ ಶ್ಲೋಕ - ೩}
\mananamtext ನಾನು ನನ್ನ ಪ್ರಧಾನ ವ್ಯಕ್ತಿತ್ವವನ್ನು ಆಧ್ಯಾತ್ಮಿಕ ಅಧಿಪತ್ಯದೊಳಗೆ ಹೇಗೆ ವರ್ಗೀಕರಿಸಬಲ್ಲೆ? ನನ್ನ ಸ್ವಭಾವವು ಯಾವುದನ್ನು ಅಧಿಕವಾಗಿ ಹೊಂದಿಕೊಂಡಿದೆ? ನಾನು ‘ಬಹಿರ್ಮುಖಿ’ ಹಾಗೂ ಸದಾ ಕಾರ್ಯ ತತ್ಪರನೇ? ನನ್ನಲ್ಲಿ ಪರರ ಸೇವೆ ಮಾಡುವ  ಇಚ್ಛೆ ಇದೆಯೇ?  ನನ್ನ ಅಂತಾರಾಳದ ಅಭಿವೃದ್ಧಿಗಾಗಿ, “ಹೊರಗಿನ ಪ್ರಪಂಚದಲ್ಲಿ ಬದಲಾವಣೆಗಳನ್ನು ತರಬೇಕು” ಎಂಬುದು ನನಗೆ ಅಗತ್ಯವೆನಿಸುತ್ತದೆಯೇ? ಅಥವಾ ನಾನು ಅಂತರ್ಮುಖಿಯಾಗಿ ಹಾಗೂ ಚಿಂತನಾಶೀಲನಾಗಿರುವೆನೇ? ನನ್ನ ಅಂತಾರಾಳದಲ್ಲಿ ಮುಳುಗಿ, ಅಹ0 ಹಾಗೂ ದೈವೀಕ ಅನ್ವೇಷಣೆಯಲ್ಲಿ ಆಳವಾದ ಅಧ್ಯಯನದಲ್ಲಿ ಪ್ರವೃತ್ತನಾಗಲು, ನಾನು ಸಿದ್ಧನಿದ್ದೇನೆಯೇ? ಇವೆರಡರ ಮಧ್ಯೆ(ಅಹಂ ಹಾಗೂ ದೈವೀಕ ಅನ್ವೇಷಣೆ) ಸಮತೋಲನ ಕಾಪಾಡಿದಾಗ ಅದು ಏನನ್ನು ಪ್ರತಿಪಾದಿಸುತ್ತದೆ? ಅಥವಾ, ಅದನ್ನು ಏನೆಂದು ತಿಳಿಯಬಹುದು?
\end{mananam}
\WritingHand\enspace\textbf{ಆತ್ಮ ವಿಮರ್ಶೆ}\\
\begin{inspiration}{\mananamfont ಸ್ಪೂರ್ತಿ}
\mananamtext  ಅಧ್ಯಾತ್ಮಕ ದಾರಿಯಲ್ಲಿ ವಿಶಾಲವಾದ ವರ್ಗೀಕರಣವು, ನಿರ್ಧಿಷ್ಟವಾದ ನಮ್ಮದೇ ಜೀವನ ಪಥಗಳನ್ನು ಕೆತ್ತುವಲ್ಲಿ ಸಹಾಯ ಮಾಡುವುದಕ್ಕಾಗಿ ಇದೆ. ನಮ್ಮ ಪ್ರತಿ ಒಬ್ಬರೊಳಗೂ ವಿವಿಧ ರೀತಿಯ ವ್ಯಕ್ತಿತ್ವಗಳಿದ್ದು, ಪ್ರಗತಿ ಹೊಂದಲು, ಈ ವ್ಯಕ್ತಿತ್ವಗಳಲ್ಲಿ ಅಗತ್ಯವಾದ ಸಮಚಿತ್ತತೆ ಕಾಪಾಡಿಕೊಳ್ಳಬೇಕಾಗುತ್ತದೆ. ಒಬ್ಬರಿಗೆ ಬೇಕಾಗಿರುಬಹುದಾದ ಸಮಚಿತ್ತತೆಯು, ಮತ್ತೊಬ್ಬರಿಗೆ ಆಗಿಬರದಿರಬಹುದು.
\end{inspiration}
\newpage


\slcol{\Index{ನ ಹಿ ಕಶ್ಚಿತ್ಕ್ಷಣಮಪಿ} ಜಾತು ತಿಷ್ಠತ್ಯಕರ್ಮಕೃತ್ ।\\
ಕಾರ್ಯತೇ ಹ್ಯವಶಃ ಕರ್ಮ ಸರ್ವಃ ಪ್ರಕೃತಿಜೈರ್ಗುಣೈಃ ॥ ೫ ॥} 
\cquote{ಕ್ಷಣಮಾತ್ರವೂ ಕರ್ಮವನ್ನು ಮಾಡದೆ ಯಾರೂ ಇರಲಾರರು. ಎಲ್ಲರೂ ತಮ್ಮ ಹುಟ್ಟು ಗುಣಗಳಿಂದ ತಮಗರಿವಿಲ್ಲದಂತೆ ಕರ್ಮ ಮಾಡುತ್ತಲೇ ಇರುತ್ತಾರೆ.\\}
\slcol{\Index{ಕರ್ಮೇಂದ್ರಿಯಾಣಿ ಸಂಯಮ್ಯ} ಯ ಆಸ್ತೇ ಮನಸಾ ಸ್ಮರನ್ ।\\
ಇಂದ್ರಿಯಾರ್ಥಾನ್ವಿಮೂಢಾತ್ಮಾ ಮಿಥ್ಯಾಚಾರಃ ಸ ಉಚ್ಯತೇ ॥ ೬ ॥}
\cquote{ಯಾವನು ಕರ್ಮೇoದ್ರಿಯಗಳನ್ನು ಬಿಗಿಹಿಡಿದು ಇಂದ್ರಿಯಗಳು ಬಯಸುವ ವಿಷಯಗಳನ್ನು ಮನಸ್ಸಿನಲ್ಲಿ ಹಂಬಲಿಸುತ್ತಿರುವನೋ ಆ ಮೂಢನು ಸುಳ್ಳು ನಟನೆಯ ಡಂಭಾಚಾರಿಯೆನಿಸುವನು.\\}
\slcol{\Index{ಯಸ್ತ್ವಿಂದ್ರಿಯಾಣಿ ಮನಸಾ} ನಿಯಮ್ಯಾರಭತೇऽರ್ಜುನ ।\\
ಕರ್ಮೇಂದ್ರಿಯೈಃ ಕರ್ಮಯೋಗಮಸಕ್ತಃ ಸ ವಿಶಿಷ್ಯತೇ ॥ ೭ ॥}
\cquote{ಅರ್ಜುನ, ಯಾವನು ಇಂದ್ರಿಯಗಳನ್ನು ಮನಸ್ಸಿನಿಂದ ಬಿಗಿಹಿಡಿದು ಫಲದಾಸಕ್ತಿ ಇಲ್ಲದೆ ಕರ್ಮೇಂದ್ರಿಯಗಳಿಂದ ಕೆಲಸದಲ್ಲಿ ತೊಡಗಿರುವನೋ ಅವನು ಉತ್ತಮನು.\\}
\slcol{\Index{ನಿಯತಂ ಕುರು ಕರ್ಮ ತ್ವಂ} ಕರ್ಮ ಜ್ಯಾಯೋ ಹ್ಯಕರ್ಮಣಃ ।\\
ಶರೀರಯಾತ್ರಾಪಿ ಚ ತೇ ನ ಪ್ರಸಿದ್ಧ್ಯೇದಕರ್ಮಣಃ ॥ ೮ ॥}
\cquote{ನೀನು ಮಾಡತಕ್ಕದ್ದೆಂದು ಗೊತ್ತಾಗಿರುವ ಕೆಲಸವನ್ನು ಮಾಡು. ಏನೂ ಮಾಡದೇ ಇರುವುದಕ್ಕಿಂತ, ಮಾಡುವುದು ಮೇಲು. ನೀನು ಯಾವ ಕರ್ಮವನ್ನೂ ಮಾಡದೇ ಇದ್ದರೆ, ಬದುಕುವುದೇ  ಆಗಲಾರದು.\\}
\slcol{\Index{ಯಜ್ಞಾರ್ಥಾತ್ಕರ್ಮಣೋऽನ್ಯತ್ರ} ಲೋಕೋऽಯಂ ಕರ್ಮಬಂಧನಃ ।\\
ತದರ್ಥಂ ಕರ್ಮ ಕೌಂತೇಯ ಮುಕ್ತಸಂಗಃ ಸಮಾಚರ ॥ ೯ ॥}
\cquote{ಈಶ್ವರನಿಗೆ ಪ್ರೀತಿಯಾಗಲೆಂದಲ್ಲದೆ ಸ್ವಾರ್ಥಕ್ಕಾಗಿ ಮಾಡುವ ಕರ್ಮಗಳಿಂದ ಈ ಲೋಕದ ಕಟ್ಟಿಗೋಳಪಡುವದು. ಅರ್ಜುನ, ಫಲವನ್ನು ಬಯಸದೆ ಈಶ್ವರನಿಗೆ ಪ್ರೀತಿಯಾಗಲೆಂದು ಕರ್ಮವನ್ನು ನಡೆಸು.\\}
\slcol{\Index{ಸಹಯಜ್ಞಾಃ ಪ್ರಜಾಃ} ಸೃಷ್ಟ್ವಾ ಪುರೋವಾಚ ಪ್ರಜಾಪತಿಃ ।\\
ಅನೇನ ಪ್ರಸವಿಷ್ಯಧ್ವಮೇಷ ವೋऽಸ್ತ್ವಿಷ್ಟಕಾಮಧುಕ್ ॥ ೧೦ ॥} 
\cquote{ಮೊದಲು ಪ್ರಜಾಪತಿ ಬ್ರಹ್ಮನು ಯಜ್ಞಗಳೊಂದಿಗೆ ಪ್ರಜೆಗಳನ್ನು ಸೃಷ್ಟಿಸಿ “ಈ ಯಜ್ಞದಿಂದ ನೀವು ಸಮೃದ್ಧಿಯನ್ನು ಪಡೆಯಿರಿ, ಇದು ನಿಮ್ಮ ಇಷ್ಟಾರ್ಥಗಳನ್ನು ದೊರಕಿಸುವ ಕಾಮಧೇನುವು” ಎಂದು ಹೇಳಿದನು.\\}
\slcol{\Index{ದೇವಾನ್ಭಾವಯತಾನೇನ ತೇ} ದೇವಾ ಭಾವಯಂತು ವಃ ।\\
ಪರಸ್ಪರಂ ಭಾವಯಂತಃ ಶ್ರೇಯಃ ಪರಮವಾಪ್ಸ್ಯಥ ॥ ೧೧ ॥}
\cquote{ಇದರಿಂದ ದೇವತೆಗಳನ್ನು ಸಂತೋಷಗೊಳಿಸಿರಿ. ಆ ದೇವತೆಗಳು ನಿಮ್ಮನ್ನು ಸಂತೋಷಗೊಳಿಸಲಿ. ಒಬ್ಬರನ್ನೊಬ್ಬರು ಸಂತೋಷಗೊಳಿಸುವರಾಗಿ ಹೆಚ್ಚಿನ ಯಶಸ್ಸನ್ನು ಪಡೆಯಿರಿ.\\}

\newpage
\begin{mananam}{\mananamfont ಮನನ ಶ್ಲೋಕ - ೫ ೬}
\mananamtext ಯುವಕರಾಗಿ ನೀವು, ವಿದ್ಯಾಭ್ಯಾಸ ಬಿಟ್ಟವರಾಗಿದ್ದರೆ; ಉದ್ಯೋಗವನ್ನು ಹುಡುಕದಿದ್ದರೆ ಅಥವಾ ಉದ್ಯೋಗದಲ್ಲೂ ತೊಡಗಿಸಿಕೊಳ್ಳದಿದ್ದರೆ, ವಯಸ್ಕನಾಗಿ, ಸಂಸಾರದ ಜವಾಬ್ದಾರಿಗಳನ್ನು ಹೊರದಿದ್ದರೆ, ಎಚ್ಚರ! ನೀವು, ಪ್ರಕೃತಿಯ ಅಸಹಜ ಕಾರ್ಯದಲ್ಲಿ ತೊಡಗಿದ್ದೀರಿ ಎಂಬ ಅರಿವಿರಲಿ! ಹಾಗಿದ್ದಲ್ಲಿ, ನಿಮ್ಮ ಶಕ್ತಿಯನ್ನು ಎಲ್ಲಿ ಮತ್ತು ಹೇಗೆ ಹರಿಯಬಿಟ್ಟಿದ್ದೀರಿ? ನಿಮ್ಮ ಮಾನಸಿಕ ಯಾತನೆ ಮತ್ತು ಭಾವನೆಗಳ ಉದ್ವೇಗಕ್ಕೆ ಕಾರಣವಾಗಿದೆಯೇ? ನೀವು ಮಾನಸಿಕ ಭ್ರಮೆಯಲ್ಲಿ ಕಳೆದು ಹೋಗಿದ್ದೀರಿಯೇ? ನಿಮ್ಮ ಜೀವನದಲ್ಲಿ ಧನಾತ್ಮಕ ಬದಲಾವಣೆ ತರಲು, ನೀವು ಏನಾದರೂ ಸೃಜನಾತ್ಮಕವಾದ ಕಾರ್ಯವನ್ನು ಮಾಡುವ ವಿಚಾರ ಮಾಡಿದ್ದೀರಿಯೆ?
\end{mananam}
\WritingHand\enspace\textbf{ಆತ್ಮ ವಿಮರ್ಶೆ}\\
\begin{inspiration}{\mananamfont ಸ್ಪೂರ್ತಿ}
\mananamtext ಒಬ್ಬರು ನಿಶ್ಕ್ರಿಯವಾಗಿದ್ದಲ್ಲಿ ಅಥವಾ ಸೋಮಾರಿಯಾಗಿದ್ದಲ್ಲಿ, ಅವರು ತಮ್ಮ ಇಂದ್ರಿಯಗಳನ್ನು ಹತೋಟಿಯಲ್ಲಿಟ್ಟಿದ್ದಾರೆಂದು ಅರ್ಥವಲ್ಲ. ತರಬೇತಿ ಪಡೆಯದ ಮನಸ್ಸು ನಿಜವಾಗಿಯೂ ಇಂದ್ರಿಯಗಳನ್ನು ತಡೆಯಲು ಸಾಧ್ಯವಿಲ್ಲ. ಒಬ್ಬರು ತಮ್ಮ ಮನಸ್ಸು ಮತ್ತು ಇಂದ್ರಿಯಗಳನ್ನು ಹತೋಟಿಯಲ್ಲಿಡಲು ಕಲಿತಾಗ ಅವರ ಕ್ರಿಯೆಗಳು ಹತೋಟಿಗೆ ಬರಲು ಪ್ರಾರಂಭಿಸುತ್ತವೆ. ಇಂತಹ ಸಂದರ್ಭಗಳಲ್ಲಿ, ನಿಶ್ಕ್ರಿಯವಾಗಿರುವುದಕ್ಕಿಂತ, ಫಲದಾಯಕವಾದ, ಕಾರ್ಯ ಕಲಾಪಗಳನ್ನು ಮಾಡುವುದಕ್ಕೆ ಇಂದ್ರಿಯಗಳನ್ನು ತರಬೇತಿಗೆ ಒಳಪಡಿಸುವುದು ಸೂಕ್ತ; ಇದರಿಂದ ಒಬ್ಬರನ್ನು, ಮಾನಸಿಕವಾಗಿಯೂ, ಶಾರೀರಿಕವಾಗಿಯೂ ಋಣಾತ್ಮಕತೆಯಿಂದ, ಧನಾತ್ಮಕತೆಗೆಡೆಗೆ ಮೆಲ್ಲನೆ ಎಳೆಯಬಹುದು.
\end{inspiration}
\newpage

\begin{mananam}{\mananamfont ಮನನ ಶ್ಲೋಕ - ೭ ೮}
\mananamtext ನನ್ನ ಉದ್ಯೋಗದಲ್ಲಿ ಹಾಗೂ ದೈನಂದಿನ ಕಾರ್ಯಗಳಲ್ಲಿ, ನಾನು ಕರ್ಮ ಯೋಗವನ್ನು ಹೇಗೆ ಅಭ್ಯಸಿಸಲಿ? (ಫಲಿತಾಂಶದ ಮೇಲಿನ ಮೋಹವನ್ನು ಬಿಟ್ಟ ನಮ್ಮ ಕ್ರಿಯೆಗಳು) ನಾನು ಪ್ರತಿಫಲದ ಪ್ರೇರಣೆ ಇಲ್ಲದೆ ಕೆಲಸ ಮಾಡಬಲ್ಲೆನೆ? ಅಥವಾ ಯಾವದೇ ಒಂದು ಕೆಲಸದ ಬಗ್ಗೆ ಒಂದು ನಿರ್ಧಿಷ್ಟ ಗುರಿ ಇಲ್ಲದಿದ್ದಲ್ಲಿ, ನನಗೆ ಚೈತನ್ಯ ಮತ್ತು ಉತ್ಸಾಹದ ಕೊರತೆ ಇದೆ ಎಂಬ ಭಾವನೆ ಬರುತ್ತದೆಯೇ? ನನಗೆ, ಆಯಾ ಕ್ಷಣಗಳ ಬೇಡಿಕೆಗೆ ತಕ್ಕಂತೆ, ಕಾರ್ಯನಿರ್ವಹಿಸುವ ತಿಳುವಳಿಕೆ ಇದೆಯೇ? ಹಾಗೂ, ಪ್ರತಿಯೊಂದೂ ಕಷ್ಟಕರವಾದ ಕಾರ್ಯವನ್ನೂ, ಸಣ್ಣ ಸಣ್ಣ ವಿಭಾಗಗಳಾಗಿ ವಿಂಗಡಿಸಿ, ಸುಲಭವಾಗುವಂತೆ ಕೆಲಸ ನಿರ್ವಹಿಸಲು ತಿಳಿಯುವುದೇ? ಎಷ್ಟೇ ಸಣ್ಣ ಕೆಲಸವಾಗಲಿ ಅಥವಾ ನಾನು ನಿಕೃಷ್ಟವೆಂದು ತಿಳಿದ ಕಾರ್ಯವಿರಲಿ, ಅದರಲ್ಲಿ ನಾನು ಶತ ಪ್ರತಿಶತ ಮನಸ್ಸನ್ನು ತೊಡಗಿಸಿ ಕೆಲಸ ಮಾಡಬಲ್ಲೆನೇ?
\end{mananam}
\WritingHand\enspace\textbf{ಆತ್ಮ ವಿಮರ್ಶೆ}\\
\begin{inspiration}{\mananamfont ಸ್ಪೂರ್ತಿ}
\mananamtext  ಕರ್ಮ ಯೋಗದ ಒಳ ತಿರುಳು, ಸಾಧನೆಯ ಬಗ್ಗೆ  ಕೇಂದ್ರೀಕೃತವಾಗಿರುವುದೇ ಹೊರತು, ಅದರ ಪ್ರತಿಫಲ ಅಥವಾ ಗುರಿಯ ಕಡೆಗೆ ಇರುವುದಿಲ್ಲ. ಒಬ್ಬರು, ಕಾರ್ಯವಿಧಾನವನ್ನು ಸರಿಯಾಗಿ ಅನುಸರಿಸಿದಲ್ಲಿ, ಅದಕ್ಕೆ ತಕ್ಕ ಪ್ರತಿಫಲ ಕೊನೆಯಲ್ಲಿ  ತನ್ನಷ್ಟಕ್ಕೆ ತಾನೇ ಸಿಗುವುದು. ಯಾವುದೇ ನಿರ್ಧಿಷ್ಟ ಆಸೆಯ ಪ್ರೆರೇಪಣೆ ಇಲ್ಲದೆಲೇ, “ಕೆಲಸವನ್ನು, ಕೆಲಸದ ಸಲುವಾಗಿ”ಯಷ್ಟೇ ಮಾಡುವುದು, ತುಂಬಾ ಆರೋಗ್ಯಕರ ಮತ್ತು ಉತ್ತಮ ಮಟ್ಟದ ಕಾರ್ಯವಾಗಿರುತ್ತದೆ. ಇದರಿಂದ, ಮನಸ್ಸಿನ ಸಮತೋಲನ ಹಾಗೂ ಶಾಂತಿ ವೃದ್ಧಿಸುತ್ತದೆ. ಕ್ರಿಯಾತ್ಮಕ ಚಟುವಟಿಕೆಯು, ದುರ್ಬಲಗೊಳಿಸುವ ಸೋಮಾರಿತನಕ್ಕಿಂತ ಉತ್ತಮ ಎಂಬುದನ್ನು ಮನಗಾಣಬೇಕು. 
\end{inspiration}
\newpage

\begin{mananam}{\mananamfont ಮನನ ಶ್ಲೋಕ - ೯}
\mananamtext ತ್ಯಾಗದ್ಯೋತಕವಾದ ಯಜ್ಞದಂತೆ, ನನ್ನ ಎಲ್ಲಾ ಕೆಲಸಗಳಿಗೂ  ಪ್ರಾಧಾನ್ಯತೆ ಕೊಟ್ಟು ಹಾಗೂ ಒಂದು ಸಮರ್ಪಣಾ ಮನೋಭಾವದಿಂದ ಕೆಲಸ ಮಾಡಲು ನಾನು ಕಲಿಯಬಲ್ಲೆನೇ? ಅಥವಾ ಕೆಲಸ ಪ್ರಾರಂಭಿಸುವ ಮೊದಲೇ, ಕೆಲಸಗಳನ್ನು ಇಷ್ಟ, ಅನಿಷ್ಟವೆಂದು ವಿಂಗಡಿಸುತ್ತಾ ಇರುತ್ತೇನೆಯೇ? ಫಲಿತಾಂಶ ಮತ್ತು ಪ್ರತಿಫಲಗಳ ಬಗೆಗಿನ ವಿಚಾರಗಳ ಪ್ರಚೋದನೆಯಿಂದಾಗಿ, ನನ್ನನ್ನು ನಾನು ಕಾರ್ಯಗಳಲ್ಲಿ ತೊಡಗಿಸಿಕೊಳ್ಳುತ್ತೇನೆಯೇ? ನಾನು, ನನ್ನನ್ನು ಕೆಲಸದಲ್ಲಿ ತೊಡಗಿಸಿಕೊಂಡಾಗ, ನಾನು, ಆ ಕ್ಷಣದ ಕೆಲಸದಲ್ಲಿ ಮನವನ್ನು ಕೇಂದ್ರೀಕರಿಸುತ್ತೇನೆಯೇ ಅಥವಾ, ಅದರ ಮುಂದಿನ ಫಲಿತಾಂಶ ಅಥವಾ ಪ್ರತಿಫಲದ  ಬಗ್ಗೆಯೇ? ನಾನು, ನನ್ನ ಕೆಲಸವನ್ನು ಪೂರ್ತಿಗೊಳಿಸಿದ ಮೇಲೆ, ನನ್ನ ಕೆಲಸದ ಫಲಿತಾಂಶದಿಂದ ಸಂತೋಷ ಅಥವಾ ದುಃಖವನ್ನು ಅನುಭವಿಸುತ್ತೇನೆಯೇ?
\end{mananam}
\WritingHand\enspace\textbf{ಆತ್ಮ ವಿಮರ್ಶೆ}\\
\begin{inspiration}{\mananamfont ಸ್ಪೂರ್ತಿ}
\mananamtext ಪ್ರಾಚೀನ ಕಾಲದ ವೇದಗಳಲ್ಲಿ ತಿಳಿಸಿರುವ ಯಜ್ಞವು, ಪುರೋಹಿತರು, ಆರಾಧಕರು ಚಿರಸ್ಮರಣೀಯಗೊಳಿಸಿದಂತೆ, ತಾನು ಅರ್ಪಿಸುವ (ಯಜ್ಞಕ್ಕೆ) ಪ್ರಾಪಂಚಿಕ ವಸ್ತುಗಳ ಆಹುತಿಗಳು, ಮೌಲ್ಯಯುತವಾದ ಲಾಭ ಮತ್ತು ಉತ್ತಮವಾದ ಕರ್ಮ ಹಾಗೂ ದಿವ್ಯವಾದ (ಸ್ವರ್ಗೀಯ) ಫಲವನ್ನು ಪಡೆಯುವುದಕ್ಕೋಸ್ಕರವೇ ಇರುವುದಾಗಿದೆ. ಈ ತರಹದ ಯಜ್ಞವು, ಒಂದನ್ನು ಪಡೆಯಲು ಇನ್ನೊಂದನ್ನು ಕೊಡುವ, ಕೇವಲ ವ್ಯಾಪಾರದಂತಾಗುತ್ತದೆ;  ‘ಕರ್ಮ ಯೋಗ’ವು ಇಂಥಹ ಮಾನಸಿಕ ಸ್ಥಿತಿಯನ್ನು ರೂಪಾoತರಗೊಳಿಸಿ ‘ಕೇವಲ ನೀಡುವ ಕ್ರಿಯೆ’ಯ ಮೌಲ್ಯವನ್ನು ತಿಳಿಸಿಕೊಡುತ್ತದೆ. ಹಾಗೂ ಸಮರ್ಪಣಾ ಭಾವದಿಂದ ಮಾಡಿದ ಕ್ರಿಯೆಗಳಿಗೆ ಯಾವ ಕರ್ಮವೂ ಅಂಟುವುದಿಲ್ಲ.
\end{inspiration}
\newpage


\slcol{\Index{ಇಷ್ಟಾನ್ಭೋಗಾನ್ಹಿ ವೋ} ದೇವಾ ದಾಸ್ಯಂತೇ ಯಜ್ಞಭಾವಿತಾಃ ।\\
ತೈರ್ದತ್ತಾನಪ್ರದಾಯೈಭ್ಯೋ ಯೋ ಭುಂಕ್ತೇ ಸ್ತೇನ ಏವ ಸಃ ॥ ೧೧ ॥}
\cquote{ಯಜ್ಞಗಳಿಂದ ಸಂತೋಷಗೊಂಡ ದೇವತೆಗಳು ನಿಮಗೆ ಬಯಸಿದ್ದನ್ನೆಲ್ಲಾ ಕೊಡುತ್ತಾರೆ. ಅವರು ಕೊಟ್ಟಿದ್ದನ್ನು ಅವರಿಗೆ ನಿವೇದಿಸದೆ ಯಾವನು ಉಣ್ಣುತ್ತಾನೋ ಅವನು ಕಳ್ಳನೇ ಸರಿ.\\}
\slcol{\Index{ಯಜ್ಞಶಿಷ್ಟಾಶಿನಃ ಸಂತೋ} ಮುಚ್ಯಂತೇ ಸರ್ವಕಿಲ್ಬಿಷೈಃ ।\\
ಭುಂಜತೇ ತೇ ತ್ವಘಂ ಪಾಪಾ ಯೇ ಪಚಂತ್ಯಾತ್ಮಕಾರಣಾತ್ ॥ ೧೩ ॥}
\cquote{ಯಜ್ಞಗಳನ್ನು ಮಾಡಿ ಉಳಿದದ್ದನ್ನು ಉಣ್ಣುವವರು ಎಲ್ಲ ಪಾಪಗಳಿಂದಲೂ ಬಿಡುಗಡೆಯನ್ನು ಹೊಂದುತ್ತಾರೆ. ಯಾರು ತಮ್ಮ ಹೊಟ್ಟೆಗಾಗಿ ಬೇಯಿಸಿಕೊಳ್ಳುವರೋ, ಆ ಪಾಪಿಗಳು ಪಾಪವನ್ನು ಉಣ್ಣುತ್ತಾರೆ.\\}
\slcol{\Index{ಅನ್ನಾದ್ಭವಂತಿ ಭೂತಾನಿ} ಪರ್ಜನ್ಯಾದನ್ನಸಂಭವಃ ।\\
ಯಜ್ಞಾದ್ಭವತಿ ಪರ್ಜನ್ಯೋ ಯಜ್ಞಃ ಕರ್ಮಸಮುದ್ಭವಃ ॥ ೧೪ ॥} 
\cquote{ಅನ್ನದಿಂದ ಜೀವಿಗಳು ಹುಟ್ಟುತ್ತವೆ. ಮಳೆಯಿಂದ ಅನ್ನ ಹುಟ್ಟುತ್ತದೆ. ಯಜ್ಞದಿಂದ ಮಳೆ ಉಂಟಾಗುತ್ತದೆ. ಯಜ್ಞವು ಕರ್ಮದಿಂದ ನಡೆಯುವುದು.\\}
\slcol{\Index{ಕರ್ಮ ಬ್ರಹ್ಮೋದ್ಭವಂ} ವಿದ್ಧಿ ಬ್ರಹ್ಮಾಕ್ಷರಸಮುದ್ಭವಮ್ ।\\
ತಸ್ಮಾತ್ಸರ್ವಗತಂ ಬ್ರಹ್ಮ ನಿತ್ಯಂ ಯಜ್ಞೇ ಪ್ರತಿಷ್ಠಿತಮ್ ॥ ೧೫ ॥}
\cquote{ಎಲ್ಲ ಕರ್ಮಗಳ ಮೂಲ, ಭಗವಂತ. ವೇದಾಕ್ಷರಗಳಿಂದ ಭಗವಂತನ ಅಭಿವ್ಯಕ್ತಿ. ವೇದಾಕ್ಷರಗಳನ್ನು ಜೀವಿಗಳು ಉಚ್ಚರಿಸುತ್ತವೆ. ಆದ್ದರಿಂದ ಎಲ್ಲೆಡೆಯೂ ತುಂಬಿರುವ ಭಗವಂತನು ಸರ್ವದಾ ಯಜ್ಞದಲ್ಲಿ ಸದಾ ಪ್ರತಿಷ್ಠಿತನಾಗಿದ್ದಾನೆ.\\}
\slcol{\Index{ಏವಂ ಪ್ರವರ್ತಿತಂ ಚಕ್ರಂ} ನಾನುವರ್ತಯತೀಹ ಯಃ ।\\
ಅಘಾಯುರಿಂದ್ರಿಯಾರಾಮೋ ಮೋಘಂ ಪಾರ್ಥ ಸ ಜೀವತಿ ॥ ೧೬ ॥}
\cquote{ಅರ್ಜುನ,ಹೀಗೆ ಪ್ರವೃತ್ತವಾದ ಈ ಜೀವನ ಚಕ್ರವನ್ನು ಯಾವನು ಮುಂದುವರಿಸುವುದಿಲ್ಲವೋ ಅವನು ಪಾಪದ ಬಾಳಿನವನೂ ಇಂದ್ರಿಯಗಳೊಡನೆ ವಿನೋದಿಸುವವನೂ ಆಗುವುದರಿಂದ ಅವನ ಬದುಕು ವ್ಯರ್ಥ.\\}
\slcol{\Index{ಯಸ್ತ್ವಾತ್ಮರತಿರೇವ} ಸ್ಯಾದಾತ್ಮತೃಪ್ತಶ್ಚ ಮಾನವಃ ।\\
ಆತ್ಮನ್ಯೇವ ಚ ಸಂತುಷ್ಟಸ್ತಸ್ಯ ಕಾರ್ಯಂ ನ ವಿದ್ಯತೇ ॥ ೧೭ ॥}
\cquote{ಯಾವ ಮನುಷ್ಯನು ಪರಮಾತ್ಮನಲ್ಲಿಯೇ ಪ್ರೇಮವುಳ್ಳವನಾಗಿ ಪರಮಾತ್ಮನಿಂದ ತೃಪ್ತನಾಗಿ ಆತನಲ್ಲಿಯೇ ಸಂತೋಷಗೊಳ್ಳುತ್ತಿರುವನೋ ಅವನು ಮಾಡಬೇಕಾದದ್ದೇನೂ ಇಲ್ಲ.\\}

\newpage
\begin{mananam}{\mananamfont ಮನನ ಶ್ಲೋಕ - ೧೫}
\mananamtext ಈ ದೇಹಕ್ಕೆ ಆಧಾರ ಈ ಜಗತ್ತು, ಹಾಗಿದ್ದಲ್ಲಿ, ನಾವು ಪಡೆದದ್ದೆಲ್ಲವನ್ನೂ ಈ ಜಗತ್ತಿಗೆ ಮರಳಿ ಸಲ್ಲಿಸುವುದು ಸೂಕ್ತವಲ್ಲವೇ?  ನಾನು ತ್ಯಾಗದ ಮನೋಸ್ಥಿತಿಯನ್ನು ಹೇಗೆ ಪೋಷಿಸಲಿ ಮತ್ತು ಧರ್ಮಸಮ್ಮತವಾದ ಕಾರ್ಯಗಳಲ್ಲಿ ಹೇಗೆ ಮುಳುಗಲಿ? ನಾನು ಮಾಡುವ ಸಮಸ್ತ ಕೆಲಸ ಕಾರ್ಯಗಳನ್ನೂ,  ಎಲ್ಲವನ್ನೂ ಪೂರಯಿಸುವ ಮತ್ತು ನನ್ನ ಸಂಪೂರ್ಣ ಅಸ್ತಿತ್ವಕ್ಕೆ ಪೋಷಣೆ ನೀಡುವ, ಆ ಭಗವಂತನಿಗೆ, ಆ ಮಾತೆ ಪ್ರಕೃತಿಗೆ ಎಲ್ಲವನ್ನೂ ಅರ್ಪಿಸುವ ಮನೋಭಾವನೆಯನ್ನು ನನ್ನದಾಗಿಸಿಕೊಳ್ಳಬಲ್ಲೆನೇ?
\end{mananam}
\WritingHand\enspace\textbf{ಆತ್ಮ ವಿಮರ್ಶೆ}\\
\begin{inspiration}{\mananamfont ಸ್ಪೂರ್ತಿ}
\mananamtext ಯಜ್ಞ ಅಥವಾ ತ್ಯಾಗವೆಂಬುದರ ನಿಜವಾದ ಅರ್ಥ, ಏನನ್ನೂ ಮರಳಿ ನಿರೀಕ್ಷಿಸದೆ ಎಲ್ಲವನ್ನೂ ಸಮರ್ಪಿಸುವುದಾಗಿದೆ; ಈ ಭಾವನೆಯು ನಮ್ಮ ಸಹಜ ಗುಣವಾದ ಶ್ರದ್ಧೆಯನ್ನು [ನಂಬಿಕೆ] ಜಾಗೃತಗೊಳಿಸಿದಾಗ  ಮಾತ್ರ ಸಾಧ್ಯವಾಗುವುದು. ಸಂಪೂರ್ಣ ವಿಶ್ವದ ಎಲ್ಲ ಆಗು ಹೋಗುಗಳೂ ಈ ಶ್ರದ್ಧೆಯಿಂದ (ನಂಬಿಕೆಯಿಂದ) ನಡೆಯುತ್ತಿರುತ್ತದೆ ಮತ್ತು ಇದುವೇ ಜೀವನದ ಮೂಲತತ್ವವಾಗಿದೆ.ಯಾರು, ಏನನ್ನೂ ಬಿಗಿಯಾಗಿ ಹಿಡಿದಿಟ್ಟುಕೊಳ್ಳದೇ ಎಲ್ಲವನ್ನೂ ತ್ಯಾಗ ಮಾಡುತ್ತಾರೋ ಅವರನ್ನು, ಪ್ರಕೃತಿಯು ಪೋಷಿಸುತ್ತದೆ.
\end{inspiration}
\newpage

\begin{mananam}{\mananamfont ಮನನ ಶ್ಲೋಕ - ೧೬}
\mananamtext
ಪ್ರಾಪಂಚಿಕ ವಿಷಯಗಳ ಮೇಲಿನ ಅತಿಯಾದ ಆಸಕ್ತಿ ಹಾಗೂ, ಸಮಾಜದಲ್ಲಿನ ತೀವ್ರವಾದ ಸ್ಪರ್ಧೆಯಿಂದಾಗಿ ನಾನು, ಈ ಜೀವನ ಚಕ್ರವನ್ನು ನೋಡಿ ಜಿಗುಪ್ಸೆ ಪಟ್ಟುಕೊಳ್ಳುತ್ತಿದ್ದೇನೆಯೇ? ಇದು ಸೋತವನು ‘ದ್ರಾಕ್ಷಿ ಹುಳಿ’ ಎನ್ನುವ ಮನೋಭಾವವೇ ಅಥವಾ ಸೋಮಾರಿತನವೇ? ಹಾಗಿದ್ದರೆ, ನಾನು ಸಮಾಜದ ಪದ್ಧತಿಗಳಲ್ಲಿ ಭಾಗವಹಿಸುವುದಿಲ್ಲವಾದರೆ, ಅದರ ಲಾಭದಲ್ಲಿ ಮಾತ್ರ ನಾನು ಭಾಗಿದಾರನೇ?  ನಾನು ಸಮಾಜಕ್ಕೆ ಧನಾತ್ಮಕವಾದ ಕೊಡುಗೆಯನ್ನು ನೀಡುತ್ತಿದ್ದೇನೆಯೆ? ನನಗೆ ದೈನಂದಿನ ಚಟುವಟಿಕೆಗಳಲ್ಲಿ ಯಾವುದೇ ನೈತಿಕ ಶಿಸ್ತನ್ನು ಪರಿಪಾಲಿಸುವ ಮನಸ್ಸಿಲ್ಲವೇ? ನಾನು ಯಾವದೇ ಸ್ವಾರ್ಥಕ್ಕಾಗಿ, ತೃಪ್ತಿ ಪಡೆಯುವ ಅವಶ್ಯಕತೆ ಇಲ್ಲದೆಯೇ,  ಬರೇ ಕಾರ್ಯಕ್ಕೋಸ್ಕರವಾಗಿಯೇ, ತತ್ತಕ್ಷಣ ಕಾರ್ಯ ಪ್ರವೃತ್ತನಾಗಲು ಸಮರ್ಥನಾಗಿದ್ದೇನೆಯೇ?
\end{mananam}
\WritingHand\enspace\textbf{ಆತ್ಮ ವಿಮರ್ಶೆ}\\
\begin{inspiration}{\mananamfont ಸ್ಪೂರ್ತಿ}
\mananamtext ಆಧುನಿಕ ಸಮಾಜದ ಬೌದ್ಧಿಕವಾದ ಮತ್ತು ಅತಿಯಾದ ಚಟುವಟಿಕೆಗಳು ‘ರಾಜಸ’ ಸ್ವಭಾವ ಉಳ್ಳದ್ದಾಗಿವೆ. ರಾಜಸ ಸ್ವಭಾವವನ್ನು ಸರಿದೂಗಿಸಲು ‘ತಮಸ್’ ಅನ್ನು ವರ್ಜಿಸಿ, ‘ಸತ್ವ’ವನ್ನು ಆಲಂಗಿಸಬೇಕು. ‘ರಾಜಸ’ ಸ್ವಭಾವವು, ‘ತಾಮಸ’ ಸ್ವಭಾವಕ್ಕಿಂತ ಉತ್ತಮ ಎಂದು ಗೀತೆಯು ತೋರಿಸಿಕೊಡುತ್ತದೆ. ಹಾಗೆಯೇ, ‘ರಾಜಸ’ ಗುಣವು ‘ಸತ್ವ’ ಗುಣದೊಂದಿಗೆ ಬೆರೆತಿದ್ದರೆ ಆದರ್ಶಪ್ರಾಯವಾಗಿರುತ್ತದೆ. ಒಬ್ಬನು, ಉತ್ತಮೋತ್ತಮ ಕರ್ತವ್ಯಗಳನ್ನು ನಿರ್ವಹಿಸುತ್ತಿದ್ದಲ್ಲಿ, ಅಂತಹವನು, ಸುಲಭವಾಗಿ ಹಾಗೂ ಸ್ಥಿರವಾಗಿ, ವಿಕಾಸದೆಡೆಗೆ ಮೇಲೇರುತ್ತಾ ಹೋಗುತ್ತಾನೆ.
\end{inspiration}
\newpage

\slcol{\Index{ನೈವ ತಸ್ಯ ಕೃತೇನಾರ್ಥೋ} ನಾಕೃತೇನೇಹ ಕಶ್ಚನ ।\\
ನ ಚಾಸ್ಯ ಸರ್ವಭೂತೇಷು ಕಶ್ಚಿದರ್ಥವ್ಯಪಾಶ್ರಯಃ ॥ ೧೮ ॥}
\cquote{ಅವನಿಗೆ ಮಾಡಿದ್ದರಿಂದಲೂ ಪ್ರಯೋಜನವಿಲ್ಲ, ಬಿಟ್ಟಿದ್ದರಿಂದ ಈ ಲೋಕದಲ್ಲಿ ಯಾವ ಹಾನಿಯೂ ಇಲ್ಲ.ಅವನಿಗೆ ಪ್ರಪಂಚದ ಯಾವ ಜೀವಿಯಿಂದಲೂ ಯಾವ ಪ್ರಯೋಜನದ ಅಪೇಕ್ಷೆಯೂ ಇಲ್ಲ.\\}
\slcol{\Index{ತಸ್ಮಾದಸಕ್ತಃ ಸತತಂ} ಕಾರ್ಯಂ ಕರ್ಮ ಸಮಾಚರ ।\\
ಅಸಕ್ತೋ ಹ್ಯಾಚರನ್ಕರ್ಮ ಪರಮಾಪ್ನೋತಿ ಪೂರುಷಃ ॥ ೧೯ ॥ }
\cquote{ಆದ್ದರಿಂದ ಯಾವಾಗಲೂ ನಿನ್ನ ಕೆಲಸವನ್ನು ಫಲದಾಸೆ ಇಲ್ಲದೆ ಮಾಡು. ಫಲದಾಸೆ ಇಲ್ಲದೆ ಕರ್ಮವನ್ನು ಮಾಡುವವನು ಪರಮಾತ್ಮನನ್ನು ಪಡೆಯುತ್ತಾನೆ.\\}
\slcol{\Index{ಕರ್ಮಣೈವ ಹಿ} ಸಂಸಿದ್ಧಿಮಾಸ್ಥಿತಾ ಜನಕಾದಯಃ ।\\
ಲೋಕಸಂಗ್ರಹಮೇವಾಪಿ ಸಂಪಶ್ಯನ್ಕರ್ತುಮರ್ಹಸಿ ॥ ೨೦ ॥}
\cquote{ಜನಕನೇ ಮೊದಲಾದವರು ಕರ್ಮದಿಂದಲೇ ಜ್ಞಾನ ಸಿದ್ದಿಯನ್ನು ಹೊಂದಿದರು. ಜನರಿಗೆ ದಾರಿಯನ್ನು ತೋರಿಸಬೇಕೆಂಬುದನ್ನಾದರೂ ಮನಸ್ಸಿಗೆ ತಂದು ನೀನು ಕರ್ಮವನ್ನು ಮಾಡತಕ್ಕದ್ದು.\\}
\slcol{\Index{ಯದ್ಯದಾಚರತಿ ಶ್ರೇಷ್ಠ}ಸ್ತತ್ತದೇವೇತರೋ ಜನಃ ।\\
ಸ ಯತ್ಪ್ರಮಾಣಂ ಕುರುತೇ ಲೋಕಸ್ತದನುವರ್ತತೇ ॥ ೨೧ ॥}
\cquote{ದೊಡ್ಡವನೆನಿಸಿಕೊಂಡವನು ಏನೇನು ಮಾಡುತ್ತಾನೋ ಉಳಿದವರು ಅದನ್ನೇ ಮಾಡುತ್ತಾರೆ. ಅವನು ಯಾವುದನ್ನು ಸರಿ ಎಂದು ತಿಳಿದು ಮಾಡುತ್ತಾನೋ ಜನರು ಅದನ್ನು ಹಿಂಬಾಲಿಸುತ್ತಾರೆ.\\}
\slcol{\Index{ನ ಮೇ ಪಾರ್ಥಾಸ್ತಿ ಕರ್ತವ್ಯಂ} ತ್ರಿಷು ಲೋಕೇಷು ಕಿಂಚನ ।\\
ನಾನವಾಪ್ತಮವಾಪ್ತವ್ಯಂ ವರ್ತ ಏವ ಚ ಕರ್ಮಣಿ ॥ ೨೨ ॥}
\cquote{ಅರ್ಜುನ,ನಾನು ಮಾಡಬೇಕಾದದ್ದೆಂಬುದು ಮೂರು ಲೋಕದಲ್ಲಿಯೂ ಏನೇನೂ ಇಲ್ಲ. ನಾನು ಕರ್ಮದಿಂದ ಪಡೆಯಬೇಕಾದ್ದು ಏನೂ ಇಲ್ಲ. ಆದರೂ ನಾನು ಕರ್ಮದಲ್ಲಿ ತೊಡಗಿಕೊಂಡೇ ಇದ್ದೇನೆ.\\}
\slcol{\Index{ಯದಿ ಹ್ಯಹಂ ನ} ವರ್ತೇಯಂ ಜಾತು ಕರ್ಮಣ್ಯತಂದ್ರಿತಃ ।\\
ಮಮ ವರ್ತ್ಮಾನುವರ್ತಂತೇ ಮನುಷ್ಯಾಃ ಪಾರ್ಥ ಸರ್ವಶಃ ॥ ೨೩ ॥}
\cquote{ನಾನು ಯಾವಾಗಲೂ ಕ್ರಿಯೆಯಲ್ಲಿ ನಿರತನಾಗಿರದಿದ್ದರೆ, ವಿಶ್ರಾಂತಿಯಿಲ್ಲದೆ, ಪುರುಷರು ಎಲ್ಲಾ ರೀತಿಯಲ್ಲಿ ನನ್ನ ಮಾರ್ಗವನ್ನು ಅನುಸರಿಸುತ್ತಾರೆ, ಓ ಅರ್ಜುನ!\\}

\newpage
\begin{mananam}{\mananamfont ಮನನ ಶ್ಲೋಕ - ೧೭ ೧೮}
\mananamtext ನಾನು ಈ ಪ್ರಪಂಚದ ಸಹಜ ಮತ್ತು ಅನಿರ್ವಾರ್ಯವಾಗಿರಬಹುದಾದ ಕೆಲಸ ಕಾರ್ಯಗಳಿಗೆ ಮತ್ತು ನಿರೀಕ್ಷೆಗಳಿಗೆ ತಕ್ಕಂತೆ ಕೆಲಸ ಮಾಡಲು ಸಮರ್ಥನೆ? ಹಾಗಿದ್ದರೆ, ಇದು ಆಂತರಿಕ ದಾಸ್ಯಕ್ಕೆ ಕಾರಣವಾಗಬಹುದೇ? ನಾನು ವಿಶೇಷವಾಗಿ, ಮಾನಸಿಕವಾಗಿ ಮತ್ತು ಭಾವನಾತ್ಮಕವಾಗಿ, ಬೇರೆ ವ್ಯಕ್ತಿಗಳ ಮೇಲಿನ ಅವಲಂಬನೆಗಳಿಂದ ಸ್ವತಂತ್ರ್ಯನಾಗಿದ್ದೇನೆಯೇ? ನಾನು, ಬೇರೆಯವರಿಂದ ಯಾವ ರೀತಿಯ ವೈಯಕ್ತಿಕ ನಿರೀಕ್ಷೆಗಳಿಗೆ ನನ್ನಲ್ಲಿ ಎಡೆಕೊಡುತ್ತಿದ್ದೇನೆ?
\end{mananam}
\WritingHand\enspace\textbf{ಆತ್ಮ ವಿಮರ್ಶೆ}\\
\begin{inspiration}{\mananamfont ಸ್ಪೂರ್ತಿ}
\mananamtext ಒಬ್ಬ ಜ್ಞಾನಿಯು (ಯೋಗಿ), ಯಾರಿಂದಲೂ, ಯಾವುದರಿಂದಲೂ, ಏನನ್ನೂ ನಿರೀಕ್ಷಿಸದೆ ತನ್ನೊಳಗೆ ತಾನು ಸಂತೃಪ್ತಿ ಹೊಂದಿರುತ್ತಾನೆ. ಇಂತಹ ಒಬ್ಬ ಸ್ವತಂತ್ರ ಜ್ಞಾನಿಯು (ಯೋಗಿ), ಸಮಾಜದ ಎಲ್ಲಾ ತರಹದ ಋಣಗಳಿಂದ ಮುಕ್ತನಾಗಿರುತ್ತಾನೆ. ಯಾವುದೇ ಸ್ವಾರ್ಥದಿಂದ ಪ್ರೇರಿತನಾಗದೇ, ಒಬ್ಬ ಉನ್ನತ ಸಾಧನೆ ಮಾಡಿದ ಯೋಗಿಯು, ಇತರರಿಗೆ (ಸಾಮಾನ್ಯ ಜನರಿಗೆ) ಒಂದು ಉತ್ತಮ ಮಾದರಿಯಾಗಿರಬೇಕೆಂಬ ಏಕೈಕ ಉದ್ದೇಶಕ್ಕಾಗಿ, ಕರ್ತವ್ಯದ ಕ್ರಿಯೆಗಳಲ್ಲಿ ತೊಡಗಬಹುದು. ಆದರೆ, ಸಮಾಜದ ಎಲ್ಲರ ಕ್ಷೇಮಾಭಿವೃದ್ಧಿಗಾಗಿ, ಆ ಜ್ಞಾನಿಯ (ಯೋಗಿಯ) ಜೀವನಕ್ಕೆ ಬೇಕಾಗುವ ಅಗತ್ಯತೆಗಳನ್ನು ಪೂರೈಸಲು, ಈ ಸಮಾಜವು ಸದಾ ಬದ್ಧವಾಗಿರಬೇಕಾಗುತ್ತದೆ.
\end{inspiration}
\newpage

\begin{mananam}{\mananamfont ಮನನ ಶ್ಲೋಕ - ೧೯}
\mananamtext ನನ್ನ ಜೀವನ ನಿರ್ವಹಣೆಗಾಗಿ ಮಾಡುವ ವೃತ್ತಿ ಮತ್ತು ಇತರ ದೈನಂದಿನ ಕ್ರಿಯೆಗಳು, ‘ಇಂತಹದು ಇಷ್ಟವಾದದ್ದು ಅಥವಾ ಇಷ್ಟವಿಲ್ಲದ್ದು’ ಎಂಬ ಭಾವನೆಯಿಂದ ಹುಟ್ಟುತ್ತವೆಯೇ? ಕೆಲವೊಮ್ಮೆ, ಕೆಲವು ಕೆಲಸಗಳ ಬಗೆಗಿನ ನನ್ನ ಒಲವು, ಸ್ವಾರ್ಥಯುತವಾದ ಆಸೆಗಳು ಮತ್ತು ಅದರಿಂದಾಗುವ ಫಲಿತಾಂಶಗಳು, ನನ್ನ ಮೇಲೆ  ಹೇಗೆ ಒತ್ತಡ ಮತ್ತು ವ್ಯಾಕುಲತೆಗಳಿಗೆ ಕಾರಣವಾಗುವುತ್ತದೆ ಎಂಬುದನ್ನು ನೋಡುತ್ತೇನೆಯೇ? ಇದು ಅತೀ ಚಡಪಡಿಕೆ, ನಿದ್ರಾಹೀನ ಸ್ಥಿತಿ ಅಥವಾ ಇತರ ಮನೋದೈಹಿಕ ಸ್ಥಿತಿಗೆ ಎಳೆದೊಯ್ಯುತ್ತಿದೆಯೇ? ಸಾತ್ವಿಕತೆಯಿಂದ ದೂರ ಸರಿಯುತ್ತಿದ್ದೇನೆಯೇ? ನಾನು, ಯಾವುದೇ ಒಂದು ನಮೂನೆಯ ಕೆಲಸ ಅಥವಾ ವೃತ್ತಿಯ ರೀತಿ ಮತ್ತು ಫಲಿತಾಂಶಗಳಿಗೆ ಮೋಹಗೊಳ್ಳದೆ, ಕೆಲಸದಲ್ಲಿ ನನ್ನನ್ನು ತೊಡಗಿಸಿಕೊಳ್ಳುವ ಮೂಲತತ್ವವನ್ನು ಅನ್ವಯಿಸುವುದು ಹೇಗೆ? 
\end{mananam}
\WritingHand\enspace\textbf{ಆತ್ಮ ವಿಮರ್ಶೆ}\\
\begin{inspiration}{\mananamfont ಸ್ಪೂರ್ತಿ}
\mananamtext ಯಾವುದೇ ಒಂದು ನಿರ್ಧಿಷ್ಟ ಕೆಲಸದಲ್ಲಿ ಅಥವಾ ಅಧ್ಯಯನದಲ್ಲಿ, ಪ್ರತಿಯೊಬ್ಬ ಯಶಸ್ವೀ ವ್ಯಕ್ತಿ, ತನ್ನಮನಸ್ಸನ್ನು ಸಂಪೂರ್ಣವಾಗಿ ಏಕಾಗ್ರಗೊಳಿಸಿದಾಗ, ತಾತ್ಕಾಲಿಕವಾಗಿಯಾದರೂ, ಧನ್ಯತಾ ಭಾವದ ಸ್ಥಿತಿಯನ್ನು ಅನುಭವಿಸುತ್ತಾನೆ; ಈ ಮಗ್ನತೆಯಿಂದಾಗಿ ಅವನಿಗೆ, ಇನ್ನ್ಯಾರೂ, ಬೇರೆ ಯಾವ ವಿಷಯವೂ ಕಾಣುವುದಿಲ್ಲ; ಅಂದರೆ, ಅಂತಹ ಕ್ಷಣದಲ್ಲಿ ಕರ್ತೃ (ಪ್ರಮಾತ), ಕೃತ್ಯ (ಪ್ರಮೇಯ) ಹಾಗೂ ಕೃತ್ಯಕ್ಕೆ ಕಾರಣವಾದ ವಸ್ತು (ಪ್ರಮಾಣ) ಎಲ್ಲಾ ಸಮ್ಮಿಳಿತವಾಗುತ್ತದೆ; ಇಂಥಹ ಧನ್ಯತಾ ಭಾವದ ಸ್ಥಿತಿಯೇ, ತಾತ್ಕಾಲಿಕ ನಿರ್ವಾಣದ ಅನುಭವ, ಪರಮಾನಂದ ಹಾಗೂ ಜೀವನದ ಶಾಂತಿ! 
\end{inspiration}
\newpage

\begin{mananam}{\mananamfont ಮನನ ಶ್ಲೋಕ - ೨೦ ೨೧}
\mananamtext ನನಗೆ ಮಾದರಿಯಾದ ವ್ಯಕ್ತಿಗಳು ಯಾರು? ಅವರು ನನಗೆ, ಶಾಂತ ಮತ್ತು ಸ್ವಾತಂತ್ರ್ಯವಾಗಿರುವಂತೆ ಪ್ರೇರೇಪಿಸುತ್ತಾರೆಯೇ? ಯಾವ ಮೌಲ್ಯಗಳು ನನ್ನನ್ನು ಕರ್ತವ್ಯದಲ್ಲಿ ತೊಡಗಿಸಿಕೊಳ್ಳುವಂತೆ ಪ್ರೇರೇಪಿಸುತ್ತವೆ? ಇದರಲ್ಲಿ,  ಇನ್ನೊಬ್ಬರ ಏಳಿಗೆಗೋಸ್ಕರ ಕೆಲಸ ಮಾಡುವಂಥದ್ದು ಒಳಗೊಂಡಿದೆಯೇ? ಜನರಿಗಾಗಿ ಒಳಿತನ್ನು ಬಯಸುವ ದೃಷ್ಟಿಯಿಂದ, ನನ್ನನ್ನು ನಾನು ಸಬಲೀಕರಣಗೊಳಿಸಿಕೊಳ್ಳಬಲ್ಲೆನೇ? 
\end{mananam}
\WritingHand\enspace\textbf{ಆತ್ಮ ವಿಮರ್ಶೆ}\\
\begin{inspiration}{\mananamfont ಸ್ಪೂರ್ತಿ}
\mananamtext ಯಾರು, ಜನರ ಒಳಿತಿಗಾಗಿ ತಮ್ಮ ಸರ್ವಸ್ವವನ್ನೂ ತ್ಯಾಗಮಾಡಿರುವರೋ, ಅಂತಹವರ ಮರಣದ ನಂತರವೂ, ಅವರ ಚೈತನ್ಯವು ಬಹಳ ಕಾಲ ಜನರ ಮನದಲ್ಲಿ ಅಚ್ಚಳಿಯದೆ ಉಳಿಯುತ್ತದೆ. ಸ್ವಾರ್ಥಯುತ ಅಭಿಲಾಷೆಗಳು ಒಬ್ಬನಲ್ಲಿ, ತುಂಬಾ ಒತ್ತಡವನ್ನು ಸೃಷ್ಟಿಸುತ್ತದೆ ಆದರೆ, ಜನರ ಅಭಿವೃದ್ಧಿಗಾಗಿ ಒಬ್ಬ ವ್ಯಕ್ತಿಯು ತನ್ನ ಜೀವನ ಮುಡಿಪಾಗಿಟ್ಟಲ್ಲಿ, ಆ ವ್ಯಕ್ತಿಯ ಹಾಗೂ ಸಮಾಜದ ಉದ್ಧಾರ ನಿಶ್ಚಿತ!
\end{inspiration}
\newpage

\begin{mananam}{\mananamfont ಮನನ ಶ್ಲೋಕ - ೨೨}
\mananamtext  ಈ ಸೃಷ್ಟಿಯನ್ನು ನಡೆಸುವವರು ಯಾರು? ಅವನ ಅಥವಾ ಅವಳ ಉದ್ದೇಶ ಏನಿರಬಹುದು? ನಿಸ್ವಾರ್ಥವಾಗಿ ಕರ್ತವ್ಯ ಮಾಡುವ, ಪ್ರಕೃತಿಯ ಪ್ರಭಾವೀ ಶಕ್ತಿಗಳಾದ ಸೂರ್ಯ, ಚಂದ್ರ, ವಾಯು, ಸಮುದ್ರ ಇತ್ಯಾದಿಗಳಿಂದ, ನಾನು ಪಾಠ ಕಲಿಯಬಲ್ಲೆನೇ? ಇದನ್ನೆಲ್ಲ ನಡೆಸಿಕೊಂಡು ಹೋಗುತ್ತಿರುವುದು ಯಾವುದು? ಅವುಗಳ ಮೂಲ ಯಾವುದು ಮತ್ತು ಅವುಗಳ ಪೋಷಣೆ ಯಾವುದರಿಂದ ಆಗುತ್ತಿದೆ? 
\end{mananam}
\WritingHand\enspace\textbf{ಆತ್ಮ ವಿಮರ್ಶೆ}\\
\begin{inspiration}{\mananamfont ಸ್ಪೂರ್ತಿ}
\mananamtext  ಈ ಜಗತ್ತಿಗೆ ಚಾಲನೆ ಕೊಟ್ಟು ನಡೆಸುತ್ತಿರುವುದು ಪ್ರೀತಿ ಮತ್ತು ಕರುಣೆ ಎಂಬ ದೊಡ್ಡ ಶಕ್ತಿಗಳು. ನಿಜವಾದ ಪ್ರೀತಿ ಎಂದರೆ, ಎಲ್ಲರನ್ನೂ, ಎಲ್ಲವನ್ನೂ ತನ್ನದೇ ಆದ ಭಾಗವೆಂದು ನೋಡುವುದು; ಹೀಗೆ ಎಲ್ಲವೂ, ಎಲ್ಲರೂ ನಮ್ಮ ಪ್ರೀತಿ ಪಾತ್ರರೇ ಆಗಿದ್ದಾಗ, ಅವರ ಒಳಿತಿಗಾಗಿ ಕಾರ್ಯನಿರ್ವಹಿಸುವುದು ಯಾವತ್ತೂ ‘ಹೊರೆ’ ಎಂಬ ಭಾವನೆ ಬರಲಾರದು; ಹೃದಯದಾಳದಿಂದ ಹುಟ್ಟಿದ ಪೀತಿಯಿಂದ ಮಾಡಿದ ಕಾರ್ಯಗಳಿಂದ (ನಮ್ಮ ಅಥವಾ ಇತರರ ಪ್ರೀಥ್ಯರ್ಥಕ್ಕಾಗಿರಬಹುದು) ನಮ್ಮ ಜೀವನವು ಸಂಪೂರ್ಣ ಸಮರಸದಿಂದ ಕೂಡಿರುತ್ತದಲ್ಲದೇ, ಮನಸ್ಸು ಶುದ್ಧವಾಗಿ ರೂಪಾoತರವಾಗುತ್ತದೆ; ಇದು ಒಂದು ರೀತಿಯ ಪ್ರಕೃತಿಯ ಆರಾಧನೆಯೇ ಆಗಿದೆ!
\end{inspiration}
\newpage


\slcol{\Index{ಉತ್ಸೀದೇಯುರಿಮೇ ಲೋಕಾ} ನ ಕುರ್ಯಾಂ ಕರ್ಮ ಚೇದಹಮ್ ।\\
ಸಂಕರಸ್ಯ ಚ ಕರ್ತಾ ಸ್ಯಾಮುಪಹನ್ಯಾಮಿಮಾಃ ಪ್ರಜಾಃ ॥ ೨೪ ॥}
\cquote{ನಾನು ಕರ್ಮವನ್ನು ಮಾಡದೇ ಹೋದರೆ ಈ ಲೋಕಗಳು ಹಾಳಾದಾವು, ನಾನು ವರ್ಣಸಂಕರಕ್ಕೂ ಕಾರಣನಾದೇನು, ಈ ಜನರ ಪತನಕ್ಕೆ ಕಾರಣನಾದೇನು.\\}
\slcol{\Index{ಸಕ್ತಾಃ ಕರ್ಮಣ್ಯವಿದ್ವಾಂಸೋ} ಯಥಾ ಕುರ್ವಂತಿ ಭಾರತ ।\\
ಕುರ್ಯಾದ್ವಿದ್ವಾಂಸ್ತಥಾಸಕ್ತಶ್ಚಿಕೀರ್ಷುರ್ಲೋಕಸಂಗ್ರಹಮ್ ॥ ೨೫ ॥} 
\cquote{ಓ ಅರ್ಜುನ! ಅಜ್ಞಾನಿಗಳು ಕ್ರಿಯೆಯಲ್ಲಿ ಮೋಹದಿಂದ ಕೆಲಸ ಮಾಡುವಂತೆ, ಜ್ಞಾನಿಗಳು ಕೇವಲ ಲೋಕಕಲ್ಯಾಣಕ್ಕಾಗಿ ಕರ್ತವ್ಯವನ್ನು ಮಾಡಬೇಕು.\\}
\slcol{\Index{ನ ಬುದ್ಧಿಭೇದಂ ಜನಯೇದ}ಜ್ಞಾನಾಂ ಕರ್ಮಸಂಗಿನಾಮ್ ।\\
ಜೋಷಯೇತ್ಸರ್ವಕರ್ಮಾಣಿ ವಿದ್ವಾನ್ಯುಕ್ತಃ ಸಮಾಚರನ್ ॥ ೨೬ ॥}
\cquote{ಕರ್ಮದಲ್ಲಿ ಅಭಿರುಚಿಯಿರುವ ಪಾಮರಜನರ ಬುದ್ಧಿಗೆ ಪಂಡಿತನು ಭೇದವನ್ನು ಉಂಟುಮಾಡಕೂಡದು. ತಾನು ಯೋಗಾಯುಕ್ತನಾಗಿ ಸರಿಯಾಗಿ ಕರ್ಮ ಮಾಡುತ್ತಾ, ಅವರನ್ನು ಕರ್ಮಕ್ಕೆ ಪ್ರೋತ್ಸಾಹಿಸಬೇಕು.\\}
\slcol{\Index{ಪ್ರಕೃತೇಃ ಕ್ರಿಯಮಾಣಾನಿ} ಗುಣೈಃ ಕರ್ಮಾಣಿ ಸರ್ವಶಃ ।\\
ಅಹಂಕಾರವಿಮೂಢಾತ್ಮಾ ಕರ್ತಾಹಮಿತಿ ಮನ್ಯತೇ ॥ ೨೭ ॥}
\cquote{ಅಹಂಕಾರದಿಂದ ತಲೆಕೆಡಿಸಿಕೊಂಡವನು ಮಾಯೆಯ ಅಧೀನವಾಗಿ ಇಂದ್ರಿಯಗಳಿಂದಾಗುವ ಕರ್ಮಗಳನ್ನು ತಾನೇ ಮಾಡುವುದೆಂದು ತಿಳಿಯುತ್ತಾನೆ.\\}
\slcol{\Index{ಮಹಾಬಾಹೋ ಗುಣ}ಕರ್ಮವಿಭಾಗಯೋಃ ।\\
ಗುಣಾ ಗುಣೇಷು ವರ್ತಂತ ಇತಿ ಮತ್ವಾ ನ ಸಜ್ಜತೇ ॥ ೨೮ ॥}
\cquote{ಅರ್ಜುನ, ಗುಣಗಳ ಮತ್ತು ಕರ್ಮಗಳ ವಿಂಗಡದ ನಿಜವನ್ನರಿತವನಾದರೋ ಇಂದ್ರಿಯ ಮತ್ತು ವಿಷಯಗಳ ಸಂಬಂಧದ ತಿರುಳನ್ನು ತಿಳಿದು ನಾನು ಮಾಡುವವನೆಂದು ಅಭಿಮಾನಕೊಳ್ಳುವುದಿಲ್ಲ.\\}
\slcol{\Index{ಪ್ರಕೃತೇರ್ಗುಣಸಂಮೂಢಾಃ} ಸಜ್ಜಂತೇ ಗುಣಕರ್ಮಸು ।\\
ತಾನಕೃತ್ಸ್ನವಿದೋ ಮಂದಾನ್ಕೃತ್ಸ್ನವಿನ್ನ ವಿಚಾಲಯೇತ್ ॥ ೨೯ ॥}
\cquote{ಇಂದ್ರಿಯಗಳ ಮಾಯೆಗೆ ಒಳಗಾದವರು ವಿಷಯ ಮೋಹದಲ್ಲಿ ಮುಳುಗಿಬಿಡುತ್ತಾರೆ. ತತ್ವದ ತಿರುಳು ತಿಳಿದಿಲ್ಲದ ಆ ದಡ್ಡರನ್ನು ಚೆನ್ನಾಗಿ ತಿಳಿದವರು ಕದಲಗೊಡಬಾರದು.\\}

\newpage
\begin{mananam}{\mananamfont ಮನನ ಶ್ಲೋಕ - ೨೫}
\mananamtext ನನ್ನ ಜೀವನದಲ್ಲಿ ಜ್ಞಾನದ ಪಾತ್ರವೇನು? ಪ್ರಬುದ್ಧತೆಯು, ನಾನು ಜೀವನ ನೋಡುವ ರೀತಿಯನ್ನು ಹೇಗೆ ಬದಲಾಯಿಸಿತು? (ಅಂದರೆ, ಪ್ರಾಪಂಚಿಕ ಪ್ರಚೋದನೆಗಳಾದ ಹೆಸರು, ಕೀರ್ತಿ, ಹಣ, ಸಂವೃದ್ಧಿ ಹಾಗೂ ಇಂದ್ರಿಯ ಸುಖಗಳಿಂದ, ಪಾರಮಾರ್ಥಿಕದೆಡೆಗೆ ನಡೆಯುವ ನನ್ನ ಮನೋಭಾವನೆ). ನನಗೆ ಉತ್ತಮ ಶ್ರೇಣಿಯಲ್ಲಿ, ವಿವೇಕದಿಂದ ಕಾರ್ಯನಿರ್ವಹಿಸಲು ಮಾರ್ಗದರ್ಶನ ಮಾಡುವ ನಂಬಿಕಸ್ಥ ಮೂಲ ಯಾರು ಅಥವಾ ಯಾವುದಾಗಿರಬಹುದು? ನನ್ನ ಪ್ರಾಪಂಚಿಕ ಬುದ್ಧಿಯನ್ನು, ಋಷಿಗಳ ಮತ್ತು ಬುದ್ಧಿವಂತರ ಜ್ಞಾನಕ್ಕೆ ನಾನು ಹೇಗೆ ರೂಪಾಂತರಗೊಳಿಸಲಿ? 
\end{mananam}
\WritingHand\enspace\textbf{ಆತ್ಮ ವಿಮರ್ಶೆ}\\
\begin{inspiration}{\mananamfont ಸ್ಪೂರ್ತಿ}
\mananamtext ಒಬ್ಬರು, ತಮಗೋಸ್ಕರವೇ ಜೀವಿಸುವುದು ಮತ್ತು ಉಳಿವಿಗಾಗಿ ಹೋರಾಡುವುದು ಸಹಜ ಪ್ರವೃತ್ತಿ. ಆದರೆ “ಬುದ್ಧಿಶಕ್ತಿ” ಎಂಬ ವಿಶೇಷತೆಯನ್ನು, ಮಾನವನಿಗೆ ಆ ಭಗವಂತನು ದಯಪಾಲಿಸಿದ್ದಾನೆ. ನಮ್ಮ ವಿವೇಕವನ್ನು ಉಪಯೋಗಿಸಿಕೊಂಡು, ಬರೇ ನಮಗಾಗಿ ಮಾತ್ರವಲ್ಲದೇ, ಒಂದು ದಿವ್ಯಾತಾ ಭಾವನೆಯಿಂದ, ಒಂದು ಶಾಂತ ಹಾಗೂ ನಿರಾಸಕ್ತತಾ ಭಾವದಿಂದ, ಮಾನವೀಯತೆಗಾಗಿ ಹಾಗೂ ಇಡೀ ಮಾನವ ಕಲ್ಯಾಣಕ್ಕಾಗಿ ಕಾರ್ಯನಿರತರಾಗಬೇಕು. 
\end{inspiration}
\newpage

\begin{mananam}{\mananamfont ಮನನ ಶ್ಲೋಕ - ೨೬}
\mananamtext ಯಾರು ಆಧ್ಯಾತ್ಮಿಕ ದಾರಿಯಲ್ಲಿ ಪ್ರಗತಿ ಕಾಣುತ್ತಾರೋ ಅಥವಾ ಉನ್ನತವಾದ ಆಶ್ರಮಗಳಿಗೆ (ಅಂದರೆ, ಮೊದಲನೆಯದಾಗಿ ಬ್ರಹ್ಮಚರ್ಯ, ನಂತರ ಗೃಹಸ್ಥ, ವಾನಪ್ರಸ್ಥ, ಸoನ್ಯಾಸ ಹೀಗೆ) ಹೋಗುತ್ತಾರೋ ಅವರು ಹೊಸ ಆಕಾಂಕ್ಷಿಗಳಿಗೆ ಕೊಟ್ಟ ಉಪದೇಶ ಮತ್ತು ಅಭ್ಯಾಸಗಳನ್ನು ಕೀಳಾಗಿ ಕಾಣದೆ ಇರುವ ಜವಾಬ್ದಾರಿಯನ್ನು ಹೊಂದಿರುತ್ತಾರೆ. ಬೇರೆ ಬೇರೆ ಆಕಾಂಕ್ಷಿಗಳಿಗೆ, ಬೇರೆಬೇರೆ ಹಂತದಲ್ಲಿ ಅಧ್ಯಾತ್ಮಿಕ ಬೋಧನೆಗಳನ್ನು ನೀಡಲಾಗುತ್ತದೆ (ಅಂದರೆ, ಅವರವರ ಆಧ್ಯಾತ್ಮಿಕ ಪ್ರಗತಿಯ ಮಟ್ಟಕ್ಕೆ ತಕ್ಕಂತೆ). ಅಧ್ಯಾತ್ಮದ ತುತ್ತ ತುದಿಯಲ್ಲಿದ್ದವರಿಗೆ ಮಾತ್ರ, ಎಲ್ಲಾ ತರಹದ ಸಹಾಯ ಮತ್ತು ಆಶ್ರಯವನ್ನು ಬಿಟ್ಟು, ತಮ್ಮಲ್ಲೇ ತಾವು (ತಮ್ಮ ಆತ್ಮದಲ್ಲಿಯೇ), ಆಶ್ರಯ ಪಡೆಯುವ ಸಮರ್ಥತೆ ಇರುತ್ತದೆ.
\end{mananam}
\WritingHand\enspace\textbf{ಆತ್ಮ ವಿಮರ್ಶೆ}\\
\begin{inspiration}{\mananamfont ಸ್ಪೂರ್ತಿ}
\mananamtext ಗುರಿಯ ಮೇಲೆ ಮಾತ್ರ ಗಮನ ಕೇಂದ್ರೀಕರಿಸಿದರೆ, ಒತ್ತಡಕ್ಕೆ ಒಳಗಾಗುವುದು ಖಂಡಿತ; ಆದರೆ, ‘ಕಾರ್ಯ ವಿಧಾನ’ದ ಬಗ್ಗೆ ನಮ್ಮ ಗಮನ ಕೇಂದ್ರೀಕರಿಸಿದಾಗ, ನಮ್ಮ ಮುಂದಿರುವ ಯಾವುದೇ ಸವಾಲುಗಳನ್ನೂ ನಿರ್ವಹಿಸಲು ನಮಗೆ ಸಾಧ್ಯವಾಗುತ್ತದೆ. ಯಾವುದೇ ಅಹಂ ಇಲ್ಲದೇ (ಅಂದರೆ, ನನ್ನಿಂದಾಗಿಯೇ ಕಾರ್ಯ ನಡೆಯುತ್ತಿದೆ ಎಂಬ ಹಮ್ಮು) ಹಾಗೂ, ಇದು ಇಷ್ಟ ಅಥವಾ, ಇದು ಅನಿಷ್ಟ ಎಂದುಕೊಳ್ಳದೇ ಮಾಡಿದ ಎಲ್ಲಾ ಕಾರ್ಯಗಳೂ, ಎಲ್ಲಾ ಗುಣಗಳನ್ನೂ ಮೀರಿ, ದೈವ ಸ್ವರೂಪವನ್ನೇ ಪಡೆಯುತ್ತವೆ.
\end{inspiration}
\newpage

\begin{mananam}{\mananamfont ಮನನ ಶ್ಲೋಕ - ೨೭}
\mananamtext ಈ ದೇಹ ಕೆಲಸ ಮಾಡಲು ಬೇಕಾಗುವ ಕೆಲವು ಒಳ ಅಂಗಗಳ ಕ್ರಿಯೆಗಳಾದ ಉಸಿರಾಟ ಮತ್ತು ಪಚನಕ್ರಿಯೆಯನ್ನು ನಾನು ಹತೋಟಿಯಲ್ಲಿಡುತ್ತಿದ್ದೇನೆಯೇ? ಪ್ರಕೃತಿಯ ಬೆಂಬಲದಿಂದಾಗಿ ನಾನು ಮಾತನಾಡಲು ಸಶಕ್ತನಾಗುವಂತೆ ಮಾಡುವ ಶಾರೀರಿಕ ಪ್ರಕ್ರಿಯೆ ಇಲ್ಲವೇ? ಭೌತಿಕ ವಸ್ತುಗಳ ಪೋಷಣೆ ಇಲ್ಲದೆ ಮತ್ತು ಪ್ರಕೃತಿಯ ಉತ್ತೇಜನವಿಲ್ಲದೆ ಯೋಚಿಸಲು ಮತ್ತು ತರ್ಕಿಸಲು ನನ್ನ ಮನಸ್ಸಿಗೆ ಶಕ್ತಿ ಇದೆಯೇ? ನಾನು ನನ್ನದೇ ಎಂದು  ಹಕ್ಕಿನಿಂದ ಹೇಳಿಕೊಳ್ಳುವ, ಏನೆಲ್ಲಾ ಕೌಶಲ್ಯಗಳ ಮತ್ತು ಸಾಮರ್ಥ್ಯಗಳ ಅಭಿವೃದ್ಧಿಗೆ ಕೂಡ, ಯಾರೆಲ್ಲಾ ಅಥವಾ ಏನೆಲ್ಲಾ ಕಾರಣವಾಗಿದೆ ಹಾಗೂ, ಶ್ರಮದಾನದ ಕೊಡುಗೆ ಇದೆ ಎಂದು ತಿಳಿಯಬಲ್ಲೆನೇ? ಇದನ್ನೆಲ್ಲಾ ಪರ್ಯಾಲೋಚಿಸಿದರೆ, ನಾನು, ನನ್ನ ಜೀವನದಲ್ಲಿ ಹಕ್ಕಿನಿಂದ “ಇದು ನನ್ನದು” ಎಂದು ಹೇಳುವ ಕೆಲವು ಸಾಧನೆಗಳು ಮತ್ತು ಸಿದ್ದಿಗಳನ್ನು ಮಾಡುವವನು ನಾನೆಯೇ? ಎಲ್ಲವೂ ನನ್ನಿಂದಲೇ ಆಗಿದೆಯೇ?
\end{mananam}
\WritingHand\enspace\textbf{ಆತ್ಮ ವಿಮರ್ಶೆ}\\
\begin{inspiration}{\mananamfont ಸ್ಪೂರ್ತಿ}
\mananamtext ‘ನಾನು ಅಥವಾ ನನ್ನದು’ ಎಂಬ ನಮ್ಮ ಅನಿಸಿಕೆಯು, ಯಾವುದನ್ನು ಒಳಗೊಂಡಿದೆ ಎಂಬುದನ್ನು ಪ್ರಾಮಾಣಿಕವಾಗಿ ಹಾಗೂ ಆಳವಾಗಿ ಅನ್ವೇಷಿಸಿದಾಗ, ನಮಗೆ ನಮ್ಮದೇ ಆದಂತಹ ಯಾವುದೇ ಒಂದು ‘ನಿರ್ಧಿಷ್ಟ  ಅಸ್ತಿತ್ವ’ ಕಂಡುಬರುವುದಿಲ್ಲ. ‘ನಾನು’ ಎಂಬುದು ನನ್ನದೇ ಆದ ‘ಒಂದು ನಿರ್ಧಿಷ್ಟ ವ್ಯಕ್ತಿತ್ವದಿಂದ ಒಡಗೂಡಿರುವ  ದೇಹ’ ಎಂದು ಲಘುವಾಗಿ ಪರಿಗಣಿಸುವ, ತಪ್ಪಾದ ಭಾವನೆ ಎಂದು ಕುಸಿಯುತ್ತದೋ, ಅಂದು ನಮಗೆ, ನಮ್ಮ ‘ಅಸ್ತಿತ್ವ ರಹಿತ’ ಸ್ವಭಾವದ ಜ್ಞಾನೋದಯವಾಗುತ್ತದೆ; ಅದೇ ವಾಸ್ತವಿಕತೆಯ ಅರಿವು ಹಾಗೂ ಅತ್ಯುನ್ನತ ಸ್ವಾತಂತ್ರ್ಯ; ಅದಮ್ಯ ‘ಮುಕ್ತಿ’.
\end{inspiration}
\newpage

\slcol{\Index{ಮಯಿ ಸರ್ವಾಣಿ ಕರ್ಮಾಣಿ} ಸಂನ್ಯಸ್ಯಾಧ್ಯಾತ್ಮಚೇತಸಾ ।\\
ನಿರಾಶೀರ್ನಿರ್ಮಮೋ ಭೂತ್ವಾ ಯುಧ್ಯಸ್ವ ವಿಗತಜ್ವರಃ ॥ ೩೦ ॥}
\cquote{ಎಲ್ಲರೊಳಗೂ ನಾನು ಇರುವೆನೆಂದರಿತು ಎಲ್ಲ ಕರ್ಮಗಳನ್ನೂ ನನಗೊಪ್ಪಿಸಿ, ಫಲದ ಬಯಕೆಯನ್ನೂ, ನನ್ನದೆಂಬ ಅಭಿಮಾನವನ್ನೂ ತೊರೆದು ನಿಶ್ಚಿಂತನಾಗಿ ಯುದ್ಧ ಮಾಡು.\\}
\slcol{\Index{ಯೇ ಮೇ ಮತಮಿದಂ} ನಿತ್ಯಮನುತಿಷ್ಠಂತಿ ಮಾನವಾಃ ।\\
ಶ್ರದ್ಧಾವಂತೋऽನಸೂಯಂತೋ ಮುಚ್ಯಂತೇ ತೇऽಪಿ ಕರ್ಮಭಿಃ ॥ ೩೧ ॥}
\cquote{ಈ ನನ್ನ ಅಭಿಪ್ರಾಯವನ್ನು ಯಾರು ಅಸೂಯೆ ತಾಳದೆ ಯಾವಾಗಲೂ ನನ್ನ ಮೇಲಿನ ವಿಶ್ವಾಸದಿಂದ ಆಚರಿಸುತ್ತಾರೋ ಅವರು ಕೂಡ ಕರ್ಮ ಬಂಧನದಿಂದ ಬಿಡುಗಡೆಯನ್ನು ಹೊಂದುತ್ತಾರೆ.\\}
\slcol{\Index{ಯೇ ತ್ವೇತದಭ್ಯಸೂಯಂತೋ} ನಾನುತಿಷ್ಠಂತಿ ಮೇ ಮತಮ್ ।\\
ಸರ್ವಜ್ಞಾನವಿಮೂಢಾಂಸ್ತಾನ್ವಿದ್ಧಿ ನಷ್ಟಾನಚೇತಸಃ ॥ ೩೨ ॥}
\cquote{ಅಸೂಯೆಯಿಂದ ನನ್ನ ಅಭಿಪ್ರಾಯವನ್ನು ಆಚರಣೆಗೆ ತರದೆ ತಿರಸ್ಕರಿಸುವವರು ಜ್ಞಾನದ ಮಾರ್ಗವನ್ನೇ ಅರಿಯದ ಅವಿವೇಕಿಗಳು.ಅವರು ತಮ್ಮ ನಾಶವನ್ನು ತಾವೇ ಮಾಡಿಕೊಳ್ಳುವರೆಂದು ತಿಳಿ.\\}
\slcol{\Index{ಸದೃಶಂ ಚೇಷ್ಟತೇ ಸ್ವಸ್ಯಾಃ} ಪ್ರಕೃತೇರ್ಜ್ಞಾನವಾನಪಿ ।\\
ಪ್ರಕೃತಿಂ ಯಾಂತಿ ಭೂತಾನಿ ನಿಗ್ರಹಃ ಕಿಂ ಕರಿಷ್ಯತಿ ॥ ೩೩ ॥}
\cquote{ಬಲ್ಲವನೂ ಕೂಡ ತನ್ನ ಸ್ವಭಾವಕ್ಕೆ ಸರಿಯಾಗಿ ನಡೆಯುವನು. ಎಲ್ಲಾ ಪ್ರಾಣಿಗಳೂ ಹುಟ್ಟುಗುಣವನ್ನು ಹಿಂಬಾಲಿಸುತ್ತವೆ. ನಿಗ್ರಹದಿಂದ ಏನೂ ನಡೆಯದು.\\}
\slcol{\Index{ಇಂದ್ರಿಯಸ್ಯೇಂದ್ರಿಯಸ್ಯಾರ್ಥೇ} ರಾಗದ್ವೇಷೌ ವ್ಯವಸ್ಥಿತೌ ।\\
ತಯೋರ್ನ ವಶಮಾಗಚ್ಛೇತ್ತೌ ಹ್ಯಸ್ಯ ಪರಿಪಂಥಿನೌ ॥ ೩೪ ॥}
\cquote{ಪ್ರತಿಯೊಂದು ಇಂದ್ರಿಯದ ವಿಷಯಗಳಲ್ಲೂ ರಾಗ ದ್ವೇಷಗಳು ನೆಲೆಸಿವೆ. ಅವುಗಳಿಗೆ ಅಡಿಯಾಳಾಗಬಾರದು. ಈ ರಾಗ ದ್ವೇಷಗಳೆ ಸಾಧಕನಿಗೆ ಶತ್ರುಗಳು.\\}
\slcol{\Index{ಶ್ರೇಯಾ{{ನ್ಸ \:}\Uchar"0CCD\Uchar"0CB5}ಧರ್ಮೋ ವಿಗುಣಃ} ಪರಧರ್ಮಾ{{ತ್ಸ \:}\Uchar"0CCD\Uchar"0CB5}ನುಷ್ಠಿತಾತ್ ।\\
ಸ್ವಧರ್ಮೇ ನಿಧನಂ ಶ್ರೇಯಃ ಪರಧರ್ಮೋ ಭಯಾವಹಃ ॥ ೩೫ ॥} 
\cquote{ತನಗೆ ಅಸಹಜವಾದ ಧರ್ಮವನ್ನು ಚೆನ್ನಾಗಿ ನಡೆಸುವುದಕ್ಕಿಂತ ಕಿಂಚಿದೂನವಾದರೂ, ಸಹಜ ಧರ್ಮವನ್ನೆ ಆಚರಿಸುವುದೇ ಮೇಲು. ತನ್ನ ಧರ್ಮದಲ್ಲಿ ಸಾಯುವುದಾದರೂ ಮೇಲು. ಪರಧರ್ಮವು ಆಪತ್ತಿಗೆ ಆಹ್ವಾನ.}

\newpage
\begin{mananam}{\mananamfont ಮನನ ಶ್ಲೋಕ - ೩೦}
\mananamtext ನಾನು ಯಾವಾಗ, ಯಾವುದೇ ಕಾರ್ಯವನ್ನು, ಉತ್ಕೃಷ್ಟವಾದ ಪ್ರಯತ್ನದಿಂದ ಮಾಡುತ್ತೇನೆಯೋ ಆಗ, ನಾನು ಉದ್ವೇಗಕ್ಕೆ ಒಳಗಾಗುತ್ತೇನೆಯೇ? ಒತ್ತಡಕ್ಕೆ ಒಳಗಾಗಿ, ಬೇಗನೇ ದಣಿಯುತ್ತೇನೆಯೇ? ನನಗೆ ಇಷ್ಟವಾದ ಅಥವಾ ಇಷ್ಟವಿಲ್ಲದ ಕಾರ್ಯಗಳನ್ನು ಮಾಡುವಾಗ ನನ್ನ ಮನಸ್ಸಿನ ಭಾವನೆಗಳನ್ನು ಗಮನಿಸಿದ್ದೇನೆಯೇ? ನಾನು ಸಹೋದ್ಯೋಗಿಗಳೊಡನೆ ಕುಟುಂಬದ ಸದಸ್ಯರೊಡನೆ ಘರ್ಷಣೆ ಮತ್ತು ಒತ್ತಡಕ್ಕೆ ಹಾಗೂ ಮಾನಸಿಕ ಹಿಂಸೆಗೆ ಒಳಗಾಗುತ್ತಿದ್ದೇನೆಯೇ? ಎಲ್ಲವನ್ನೂ ‘ಬಿಟ್ಟು ಬಿಡು ’ ಅಥವಾ, ‘ಹೋದರೆ ಹೋಗಲಿ ಬಿಡು’ ಎಂಬುವುದರ ಅರ್ಥವೇನು? ನಾನು, ನನ್ನ ಪ್ರಯತ್ನದ ತೀವ್ರತೆಯನ್ನು  ರಾಜಿಮಾಡಿಕೊಳ್ಳುವ ಬದಲು, ಫಲಿತಾಂಶದ ಮೇಲಿನ ಮೋಹವನ್ನು ತ್ಯಜಿಸಲು ಸಾಧ್ಯವಿದೆಯೇ? 
\end{mananam}
\WritingHand\enspace\textbf{ಆತ್ಮ ವಿಮರ್ಶೆ}\\
\begin{inspiration}{\mananamfont ಸ್ಪೂರ್ತಿ}
\mananamtext ಯಾರಲ್ಲಿ ‘ಶ್ರದ್ದೆ’ ಇರುತ್ತದೆಯೋ ಅವರಿಗೆ ಉದ್ವೇಗ ರಹಿತವಾದ, ತೀಕ್ಷ್ಣ  ಮಟ್ಟದ ಪ್ರಯತ್ನದ ಗುಟ್ಟು ಗೊತ್ತಿರುತ್ತದೆ. ಯಾವಾಗ ನಾವು ನಮ್ಮ ಜೀವನವನ್ನು ಮಾನಸಿಕವಾಗಿ, ಒಂದು ಉನ್ನತ ದೈವೀಕ ಶಕ್ತಿಯಲ್ಲಿ ನಂಬಿಕೆ ಇಟ್ಟು, ಆ ಶಕ್ತಿಗೆ ಒಪ್ಪಿಸುತ್ತೇವೆಯೋ, ಆಗ ನಮ್ಮಿಂದ ದೊಡ್ಡ ಹೊರೆಯೊಂದು ದೂರವಾಗುತ್ತದೆ; ಆಗ, ನಮ್ಮ ಕಾರ್ಯಗಳನ್ನು ಅತ್ತ್ಯುತ್ತಮವಾಗಿ ನಿರ್ವಹಿಸುತ್ತಾ, ‘ನಮಗೆ ಯಾವುದಕ್ಕೆ ಅರ್ಹತೆ ಇದೆಯೋ ಅದನ್ನು ಖಂಡಿತಾ ಯಾರೂ ನಿರಾಕರಿಸಲು ಸಾಧ್ಯವಿಲ್ಲ, ಅದನ್ನು ಪಡೆದೇ ಪಡೆಯುತ್ತೇವೆ’ ಎಂದು ತಿಳಿಯುತ್ತಾ ಮುಂದುವರೆಯಬಹುದು.
\end{inspiration}
\newpage

\begin{mananam}{\mananamfont ಮನನ ಶ್ಲೋಕ - ೩೧ ೩೨}
\mananamtext ನಾನು ಯಾವ ವರ್ಗಕ್ಕೆ ಸೇರುತ್ತೇನೆ -  ಈ ಏಳಿಗೆ ತರುವಂತಹ ಉಪದೇಶಗಳಿಗೆ, ಮನಸ್ಸು ಮತ್ತು ಶ್ರದ್ಧೆ ಇರುವವರ ಗುಂಪಿಗೋ ಅಥವಾ, ಇದರ ಪರಿವರ್ತನೆಯ ಪರಿಣಾಮದಲ್ಲಿ ನಂಬಿಕೆ ಇಲ್ಲದವರ ಗುಂಪಿಗೋ ಅಥವಾ ಇದನ್ನು ತಿರಸ್ಕರಿಸುವವರ ಗುಂಪಿಗೋ? ನಾನು ಸ್ವಲ್ಪವಾದರೂ ಈ ಬೋಧನೆಗಳನ್ನು ಅಧ್ಯಯನ ಮಾಡಿ ಅರ್ಥಮಾಡಿಕೊಳ್ಳಲು ಪ್ರಯತ್ನಿಸುತ್ತಿದ್ದೇನೆಯೇ? ಜೀವನದಲ್ಲಿ ಅಳವಡಿಸಿಕೊಳ್ಳಲು ಪ್ರಯತ್ನಿಸುತ್ತಿದ್ದೇನೆಯೇ? ಅಥವಾ ಇದರ ವಿರುದ್ಧವಾಗಿ ಪಕ್ಷಪಾತ ಮಾಡುತ್ತಿದ್ದೇನೆಯೇ?
\end{mananam}
\WritingHand\enspace\textbf{ಆತ್ಮ ವಿಮರ್ಶೆ}\\
\begin{inspiration}{\mananamfont ಸ್ಪೂರ್ತಿ}
\small \mananamtext ಈ ರೀತಿಯ ಬೋಧನೆಗಳಲ್ಲಿ ಅಂತರ್ಗತ ನಂಬಿಕೆ ಹೊಂದಿ, ಇದಕ್ಕೆ ತಕ್ಕಂತೆ, ನಮ್ಮ ಜೀವನವನ್ನು ಸರಿಯಾದ ಕ್ರಮದಲ್ಲಿ ನಡೆಸಿದಲ್ಲಿ, ಅದೊಂದು ದೈವಾನುಗ್ರಹವೇ ಆಗುತ್ತದೆ; ಜೀವನದಲ್ಲಿ ಹೆಚ್ಚಿನ  ಸಂಕಷ್ಟಕ್ಕೆ ಈಡಾಗದೇ, ಜೀವನದ ದುಃಖಗಳನ್ನು ಧೈರ್ಯವಾಗಿ ಎದುರಿಸುವ ಮನೋಬಲ ಬರುತ್ತದೆ.  ಅಂತರ್ಗತವಾದ ನಂಬಿಕೆ ಬಂದಿರದಿದ್ದಲ್ಲಿ, ಈ ಸನಾತನ ಗ್ರಂಥಗಳು, ನಮ್ಮ ಜೀವನದಲ್ಲಿಯೇ ಈ ಬೋಧನೆಗಳನ್ನು ಪ್ರಯೋಗಮಾಡಿ, ಇದರ ಮಾನ್ಯತೆಯನ್ನು ಪರೀಕ್ಷಿಸುವ ದಾರಿಯನ್ನೂ ಮಾಡಿಕೊಡುತ್ತವೆ; ಇದರಲ್ಲಿ ಒಂದನ್ನಾದರೂ (ಅಂತರ್ಗತ ನಂಬಿಕೆ ಅಥವಾ ಧರ್ಮಗ್ರಂಥಗಳು) ಅಳವಡಿಸಿಕೊಳ್ಳದಿದ್ದಲ್ಲಿ, ಆಮೇಲೆ ನಾವೇ ಯಾತನೆಗೊಳಗಾಗಿ, ಜೀವನದ ಪಾಠಗಳನ್ನು ಕಷ್ಟಕರ ರೀತಿಯಲ್ಲಿ ಕಲಿಯಬೇಕಾಗುತ್ತದೆ. 
\end{inspiration}
\newpage

\begin{mananam}{\mananamfont ಮನನ ಶ್ಲೋಕ - ೩೩}
\mananamtext ನನಗೆ ಯಾವ ಅಭ್ಯಾಸಗಳನ್ನು ಬಿಡಲು ತುಂಬಾ ಕಷ್ಟಕರವಾಗಿ ಕಾಣುವುದು? ನನಗೆ ಈ ಅಭ್ಯಾಸಗಳ ಋಣಾತ್ಮಕವಾದ ಮುಖದ ಬಗ್ಗೆ ತಿಳಿದಿದೆಯೇ? ಈ ಅಭ್ಯಾಸಗಳ ಮೇಲೆ ಒಂದು ಭಾವನಾತ್ಮಕವಾದ ಅವಲಂಬನೆ ಇದೆಯೇ? ಈ ಅಭ್ಯಾಸಗಳಿಗೆ ವಿರಾಮ ಕೊಡಲು ಸತತ ಪ್ರಯತ್ನ ಮಾಡುವ ನಿರ್ಧಾರ ಕೈಗೊಳ್ಳುತ್ತಿದ್ದೇನೆಯೇ? ಪ್ರಲೋಭೆನೆಗಳ ಆಕ್ರಮಣಗಳನ್ನು ನಿರ್ವಹಿಸುವಲ್ಲಿ ನಾನು ಎಷ್ಟರಮಟ್ಟಿಗೆ ಸಾವಧಾನತೆಯನ್ನು ಕಾಪಾಡಿಕೊಳ್ಳಬಹುದು? ಈ ಅಭ್ಯಾಸಗಳನ್ನು ಬಿಡಿಸಿಕೊಳ್ಳಲು ಬೇರೆ ಬೇರೆ ಹಂತಗಳಲ್ಲಿ ಕೆಲಸ ಮಾಡುವ, ಯೋಗದ ಜೀವನಕ್ರಮ ಮತ್ತು ಅಧ್ಯಾತ್ಮಿಕ ಅಭ್ಯಾಸಗಳನ್ನು ನಾನು ಹೇಗೆ ಅಳವಡಿಸಿಕೊಳ್ಳಲಿ?
\end{mananam}
\WritingHand\enspace\textbf{ಆತ್ಮ ವಿಮರ್ಶೆ}\\
\begin{inspiration}{\mananamfont ಸ್ಪೂರ್ತಿ}
\mananamtext ಪ್ರಲೋಭನೆಗಳ ಮುಂದೆ ಎಲ್ಲಾ ಬುದ್ಧಿವಂತಿಕೆ ಮತ್ತು ಸಕರಾತ್ಮಕ ಉದ್ದೇಶಗಳೂ ವಿಫಲವಾಗುತ್ತವೆ. ವ್ಯಸನದ ಶಕ್ತಿ ಎಷ್ಟೆಂದರೆ, ಅಪಾಯಕಾರಿ ಪರಿಣಾಮಗಳ ಬಗ್ಗೆ ಗೊತ್ತಿದ್ದರೂ ಕೂಡ, ವಸ್ತುಗಳ ಆಕರ್ಷಣೀಯ ಉಪಸ್ಥಿತಿಯಲ್ಲಿ, ಒಬ್ಬ ವ್ಯಸನಿಯು ತನ್ನನ್ನು ತಾನು ನಿಗ್ರಹಿಸಿಕೊಳ್ಳಲು ಸಾಧ್ಯವಾಗುವುದಿಲ್ಲ. ನಿಯಮಿತವಾದ ಆಧ್ಯಾತ್ಮದ ಅಭ್ಯಾಸಗಳಿಂದ, ದೃಢವಾದ ಮನಸ್ಸಿನಿಂದ ಹಾಗೂ ಎಡೆಬಿಡದ ಸಾವಧಾನತೆಯಿಂದ ಮಾತ್ರ, ಪ್ರಲೋಭನೆಗಳಿಂದ ಶಾಶ್ವತವಾಗಿ ಮುಕ್ತವಾಗಲು ಸಾಧ್ಯ.
\end{inspiration}
\newpage

\begin{mananam}{\mananamfont ಮನನ ಶ್ಲೋಕ - ೩೫}
\mananamtext ಈ ಜೀವನದಲ್ಲಿ ನನ್ನ ಜಾಣ್ಮೆಗಳು ಮತ್ತು ಸಾಮರ್ಥ್ಯಗಳು ಯಾವುವು? ನನ್ನ ಕೆಲಸ ಕಾರ್ಯಗಳಲ್ಲಿ ಮತ್ತು ಚಟುವಟಿಕೆಗಳಲ್ಲಿ ಅವುಗಳನ್ನು ಯಶಸ್ವಿಯಾಗಿ ಉಪಯೋಗಿಸಲು ಸಾಧ್ಯವೇ? ನಾನು ನನ್ನ ವಿಶೇಷಗಳು ಮತ್ತು ಸಾಮರ್ಥ್ಯಗಳನ್ನು, ಕೆಲಸದ ವ್ಯಾಪ್ತಿಯೊಳಗೆ ಹೇಗೆ ಉಪಯೋಗಿಸಬಹುದು? ನಾನು ಬೇರೆಯವರ ಪಾತ್ರವನ್ನು ನಿರ್ವಹಿಸಲು ಹಂಬಲಿಸುತ್ತೇನೆಯೇ? ನಾನು ಬೇರೆಯೇ ತರಹದ ಕೆಲಸದಲ್ಲಿ ಹೆಚ್ಚು ಯಶಸ್ವಿಯಾಗಬಲ್ಲೆನೆಂಬ ಭಾವನೆ ಇದೆಯೇ? ಆ ಕೆಲಸದ ವ್ಯಾಪ್ತಿಯು ನನ್ನ ಕೈಗೆ ಎಟುಕುವಂತಿದೆಯೇ? 
\end{mananam}
\WritingHand\enspace\textbf{ಆತ್ಮ ವಿಮರ್ಶೆ}\\
\begin{inspiration}{\mananamfont ಸ್ಪೂರ್ತಿ}
\mananamtext ಈ ಜೀವನದಲ್ಲಿ ನಮ್ಮ ಕರ್ತವ್ಯವನ್ನು ನಿರ್ವಹಿಸುವುದು, ಮಾನಸಿಕ ನಿಯಂತ್ರಣದ ಮೇಲೆ ಗೆಲುವು ಸಾಧಿಸುವ ಬಹಳ ಉತ್ತಮವಾದ ಮಾರ್ಗ. ತನ್ನ ಕರ್ತವ್ಯ ಅಥವಾ ಪಾತ್ರವನ್ನು ಇತರರಿಗೆ ಹೋಲಿಸಿಕೊಳ್ಳುವುದು ವಿವೇಕವಲ್ಲ. ಬೇರೆಯವರ ಪಾತ್ರಗಳಿಗೆ ಹಂಬಲಿಸುವುದಕ್ಕಿಂತ, ನಮ್ಮ ಜವಾಬ್ದಾರಿಗಳನ್ನು ನಡೆಸಿಕೊಂಡು ಹೋಗುವುದರಿಂದ, ಈ ಅವ್ಯಕ್ತ ಪ್ರವೃತ್ತಿಗಳನ್ನು ನೆಮ್ಮದಿಯಾಗಿ, ಮತ್ತೆ ತಲೆ ಎತ್ತದಂತೆ ಸುಟ್ಟುಬಿಡಬಹುದು.
\end{inspiration}
\newpage

\slcol{ಅರ್ಜುನ ಉವಾಚ ।\\
\Index{ಅಥ ಕೇನ ಪ್ರಯುಕ್ತೋऽಯಂ} ಪಾಪಂ ಚರತಿ ಪೂರುಷಃ ।\\
ಅನಿಚ್ಛನ್ನಪಿ ವಾರ್ಷ್ಣೇಯ ಬಲಾದಿವ ನಿಯೋಜಿತಃ ॥ ೩೬ ॥} 
\cquote{ಅರ್ಜುನ ಹೇಳಿದನು,\\
ಕೃಷ್ಣ,ಹಾಗಾದರೆ ಈ ಮನುಷ್ಯನು ತನಗೆ ಬೇಡವಾದರೂ ಬಲವಂತಕ್ಕೆ ಬಲಿಯಾದವನಂತೆ ಯಾರಿಂದ ಪ್ರೇರಿತನಾಗಿ ಪಾಪವನ್ನು ಮಾಡುತ್ತಾನೆ?\\}
\slcol{ಶ್ರೀಭಗವಾನುವಾಚ  ।\\
\Index{ಕಾಮ ಏಷ ಕ್ರೋಧ ಏಷ} ರಜೋಗುಣಸಮುದ್ಭವಃ ।\\
ಮಹಾಶನೋ ಮಹಾಪಾಪ್ಮಾ ವಿ{ದ್ಧೆ\:}\Uchar"0CCD\Uchar"0CAF\Uchar"0CD5ನಮಿಹ ವೈರಿಣಮ್ ॥ ೩೭ ॥} 
\cquote{ಶ್ರೀ ಭಗವಂತನು ಹೇಳಿದನು,\\
ರಾಜೋಗುಣದಿಂದ ಹುಟ್ಟಿದ, ಸಿಟ್ಟಿಗೂ ತವರಾದ ಈ ಬಯಕೆ ಇದಕ್ಕೆಲ್ಲ ಕಾರಣ.ಎಷ್ಟು ತಿಳಿಸಿದರು ಇನ್ನಷ್ಟು ಬೇಕೆನ್ನುವ ಮಹಾ ಪಾಪಿ. ಈ ಬಯಕೆಯನ್ನು ಬಾಳಿನಲ್ಲಿ ದೊಡ್ಡ ಶತ್ರು ಎಂದು ತಿಳಿ.\\}
\slcol{\Index{ಧೂಮೇನಾವ್ರಿಯತೇ} ವಹ್ನಿರ್ಯಥಾದರ್ಶೋ ಮಲೇನ ಚ ।\\
ಯಥೋಲ್ಬೇನಾವೃತೋ ಗರ್ಭಸ್ತಥಾ ತೇನೇದಮಾವೃತಮ್ ॥ ೩೮ ॥}
\cquote{ ಬೆಂಕಿಗೆ ಹೊಗೆಯ ಮುಸುಕು. ಕನ್ನಡಿಗೆ ಕೊಳೆಯ ಮುಸುಕು. ಗರ್ಭಕ್ಕೆ ಕೋಶದ ಮುಸುಕು. ಹಾಗೆ ಜಗತ್ತಿಗೆಲ್ಲ ಕಾಮದ ಮುಸುಕು.\\}
\slcol{\Index{ಆವೃತಂ ಜ್ಞಾನಮೇತೇನ} ಜ್ಞಾನಿನೋ ನಿತ್ಯವೈರಿಣಾ ।\\
ಕಾಮರೂಪೇಣ ಕೌಂತೇಯ ದುಷ್ಪೂರೇಣಾನಲೇನ ಚ ॥ ೩೯ ॥} 
\cquote{ಅರ್ಜುನ,ತಿಳಿದವರ ನಿತ್ಯಶತ್ರುವಾದ, ತೃಪ್ತಿಯನ್ನೇ ಅರಿಯದ ಈ ಕಾಮದ ಮುಸುಕಿನಿಂದ ತಿಳಿವು ಮರೆಯಾಗಿದೆ.\\}
\slcol{\Index{ಇಂದ್ರಿಯಾಣಿ ಮನೋ} ಬುದ್ಧಿರಸ್ಯಾಧಿಷ್ಠಾನಮುಚ್ಯತೇ ।\\
ಏತೈರ್ವಿಮೋಹಯತ್ಯೇಷ ಜ್ಞಾನಮಾವೃತ್ಯ ದೇಹಿನಮ್ ॥ ೪೦ ॥}
\cquote{ಇಂದ್ರಿಯಗಳು,ಮನಸ್ಸು,ಬುದ್ದಿ ಇವೇ ಕಾಮದ ವಾಸಸ್ಥಾನ.ಇದು ಇವುಗಳ ಮೂಲಕ ತಿಳಿವನ್ನು ಮರೆಮಾಡಿ ಮನುಷ್ಯನನ್ನು ಮಂಕು ಗೊಳಿಸುತ್ತದೆ.\\}

\newpage
\begin{mananam}{\mananamfont ಮನನ ಶ್ಲೋಕ - ೩೬}
\mananamtext ನನ್ನ ಒಳ್ಳೆಯ ಉದ್ದೇಶಗಳ ವಿರುದ್ಧವಾಗಿ ನಡೆದುಕೊಳ್ಳುವಂತೆ ಕೆಲವು ಶಕ್ತಿಗಳು ನನ್ನನ್ನು ಬಲವಂತ ಪಡಿಸುತ್ತವೆ ಎಂಬುದನ್ನು ನಾನು ಅರಿತಿದ್ದೇನೆಯೇ? ನನಗೆ ಪ್ರವೃತ್ತಿಯ ಸುಪ್ತಾವಸ್ಥೆಯ ಸ್ಥಿತಿ ಮತ್ತು ಪ್ರಕಟವಾಗುವ ಸ್ಥಿತಿಗಳ ಬಗ್ಗೆ ತಿಳಿದಿದೆಯೇ? ಇದರ ಅಂಗಗಳಾದ, ಮಾನಸಿಕ ನಿರ್ಬಂಧ ಮತ್ತು ಇಂದ್ರಿಯಗಳ ನಿರ್ಬಂಧದ ಬಗ್ಗೆ ನನಗೆ ಅರಿವಿದೆಯೇ? ಈ ಎರಡರಲ್ಲಿ, ಯಾವ ಅವಸ್ಥೆಯಲ್ಲಿ ಯಾವುದರ ಪ್ರಾಬಲ್ಯ ಜಾಸ್ತಿ? ‘ಸ್ವ-ನಿಯಂತ್ರಣವೆಂದರೆ’ ನಾನು ಏನೆಂದು ತಿಳಿದಿದ್ದೇನೆ? ನನಗೆ ಇಂದ್ರಿಯಗಳ ಹಂತದಲ್ಲಿ ಪ್ರಲೋಭನೆಗಳ ಶಕ್ತಿಯನ್ನು ಹತೋಟಿಯಲ್ಲಿಡುವ ಸಾಮರ್ಥ್ಯವಿದೆಯೇ? ನನಗೆ ಯೋಚನೆಗಳ ಹಂತದಲ್ಲಿ ಅವುಗಳನ್ನು ಹತೋಟಿಯಲ್ಲಿಡುವ ಸಾಮರ್ಥ್ಯವಿದೆಯೇ? 
\end{mananam}
\WritingHand\enspace\textbf{ಆತ್ಮ ವಿಮರ್ಶೆ}\\
\begin{inspiration}{\mananamfont ಸ್ಪೂರ್ತಿ}
\mananamtext ಜನ್ಮ, ಜನ್ಮಾoತರಗಳಿಂದ ಒಟ್ಟುಗೂಡಿದ, ವಿವಿಧ ರೀತಿಯ ಹಾನಿಕಾರಕ ಪ್ರವೃತ್ತಿಗಳು ಜೀವಮಾನದ ಉದ್ದಕ್ಕೂ, ನಮ್ಮೊಳಗೆ ಯಾವಾಗಲೂ ಸುಪ್ತ ರೀತಿಯಲ್ಲಿ ಸುಳಿಯುತ್ತಿರುತ್ತವೆ ಹಾಗೂ ಘರ್ಷಣೆ ಉಂಟು ಮಾಡುತ್ತಿರುತ್ತವೆ. ಅತಿಯಾದ ಆತ್ಮವಿಶ್ವಾಸದಿಂದ, ‘ಇಂದ್ರಿಯ ಪ್ರಚೋದನೆಗಳ ಮೇಲೆ ನನ್ನ ಪ್ರಭುತ್ವ ಇದೆ’ ಎoದು ತಿಳಿದು, ಯಾವಾಗಲೂ ಸಣ್ಣ, ಸಣ್ಣ ಮಟ್ಟದಲ್ಲಿಯೂ ಪ್ರಲೋಭನೆಗಳಿಗೆ ತುತ್ತಾಗುವುದು ವಿವೇಕವಲ್ಲ. 
\end{inspiration}
\newpage

\begin{mananam}{\mananamfont ಮನನ ಶ್ಲೋಕ - ೩೭}
\mananamtext ನನ್ನ ಇಚ್ಛೆ ಮತ್ತು ಹಂಬಲಗಳ ಲಕ್ಷಣಗಳೇನು? ಹಾಗೂ ಅವುಗಳ ಮೂಲವನ್ನು ಗುರುತಿಸಬಲ್ಲೆನೇ? ನನ್ನ ಇಚ್ಛೆಗಳಿಗೆ ಅಡಚಣೆಯಾದಾಗ ಏನಾಗುತ್ತದೆ? ನನಗೆ ಕೋಪದ ಭಾವನೆ ಬರುತ್ತದೆಯೇ? ನನಗೆ ಭಯ ಅಥವಾ ವ್ಯಾಕುಲತೆ ಆಗುತ್ತದೆಯೇ? ನನ್ನ ಇಚ್ಛೆಗಳು ನನಗೆ, ಒಂದು ಸೀಮಿತ ಮತ್ತು ಮೋಹದ ಭಾವನೆ ಕೊಡುತ್ತವೆಯೇ? ಅವುಗಳು ಕೇವಲ ನನ್ನ ವೈಯಕ್ತಿಕ ಸಂತೋಷಕ್ಕೆ ಮಾತ್ರ ನಿರ್ಧಿಷ್ಟವಾಗಿದೆಯೆ ಅಥವಾ, ಬೇರೆಯವರ ಒಳಿತನ್ನೂ ಒಳಗೊಂಡಿದೆಯೇ? ಇದರಲ್ಲಿ ಆರೋಗ್ಯಕರ ಇಚ್ಛೆಗಳು ಮತ್ತು ಅನಾರೋಗ್ಯಕರ ಇಚ್ಛೆಗಳಂತಹವು ಇರಲು ಸಾಧ್ಯವೇ? 
\end{mananam}
\WritingHand\enspace\textbf{ಆತ್ಮ ವಿಮರ್ಶೆ}\\
\begin{inspiration}{\mananamfont ಸ್ಪೂರ್ತಿ}
\mananamtext ನಮ್ಮ ನಿಜವಾದ ಗುಣ ‘ಅನಂತತೆ’. ನಾವು ಅದನ್ನು ಮರೆತು,ನಮ್ಮನ್ನು ಸಂಪೂರ್ಣವಾಗಿಸಲು, ಪ್ರಾಪಂಚಿಕ ವಸ್ತುಗಳ ಹಿಂದೆ ಹೋಗುತ್ತೇವೆ. ಯಾವಾಗ ನಾವು ಒಂದು ಪರಿಮಿತಿಯ ಭಾವನೆಗೆ ಒಳಗಾಗುತ್ತೇವೆಯೋ ಆವಾಗ, ಸ್ವಾರ್ಥಯುತ ಇಚ್ಛೆಗಳು ಆವಿರ್ಭಸುತ್ತವೆ; ಇವು ನಮ್ಮನ್ನು ಇತರರಿಗಿಂತ ಹೆಚ್ಚು ದೊಡ್ಡವರು ಹಾಗೂ ಉತ್ತಮರನ್ನಾಗಿ (ಸಮಾಜದ ಕಣ್ಣಿನಲ್ಲಿ) ಮಾಡಬಹುದು ಎಂದು ಆಶಿಸುತ್ತೇವೆ. ಆಸೆ ಮತ್ತು ಕೋಪ, ಇವೆರಡೂ ನಮ್ಮನ್ನು ಈ ಪ್ರಾಪಂಚಿಕ ಮಾಯದಲ್ಲಿ ಬಂಧಿಸುತ್ತವೆ.
\end{inspiration}
\newpage

\begin{mananam}{\mananamfont ಮನನ ಶ್ಲೋಕ - ೩೯ ೪೦}
\mananamtext ನನ್ನ ಆಳವಾದ ತಿಳುವಳಿಕೆಯನ್ನು ಮತ್ತು ಉದ್ದೇಶವನ್ನು ಏನು ಮುಸುಕಿದೆ? ನನಗೆ, ನನ್ನ ಅಭಿವೃದ್ಧಿಯ ಮಹತ್ವಾಕಾಂಕ್ಷೆಯು ಇದ್ದಾಗ್ಯೂ ಕೂಡ, ನಾನು ಏಕೆ ಪ್ರಲೋಭನೆ ಮತ್ತು ಸೋಮಾರಿತನಕ್ಕೆ ಈಡಾಗುತ್ತಿದ್ದೇನೆ? ಸದಾ, ನನ್ನ ಉನ್ನತ ಗುರಿಯನ್ನು ಸಾಧಿಸುವತ್ತ ಹರಿಸಬೇಕಾದ ನನ್ನ ನೆನಪನ್ನು ಏಕೆ ಕಳೆದುಕೊಳ್ಳುತ್ತೇನೆ?\\
ಇಂದ್ರಿಯ ಸುಖಗಳನ್ನು ಬೆನ್ನಟ್ಟುವುದರಿಂದ ಸಿಗುವ ಸಂತೋಷವು ಶಾಶ್ವತವೇ? ಇದರಿಂದ ಇಚ್ಛೆ ಪೂರೈಕೆಯಾದಮೇಲೆ, ನನ್ನ ದೇಹ ಮತ್ತು ಮನಸ್ಸಿನ ಮೇಲೆ ಯಾವ ರೀತಿಯ ಋಣಾತ್ಮಕ ಪರಿಣಾಮವಾಗುತ್ತದೆ? 
\end{mananam}
\WritingHand\enspace\textbf{ಆತ್ಮ ವಿಮರ್ಶೆ}\\
\begin{inspiration}{\mananamfont ಸ್ಪೂರ್ತಿ}
\mananamtext ಮಾತ್ರ ಬುದ್ಧಿಶಕ್ತಿಯ ಹಂತದಲ್ಲಿ, ವಿವೇಕದಿಂದ ಪರಿವರ್ತನೆಯನ್ನು ತರಲಾಗುವುದಿಲ್ಲ; ಅದನ್ನು ನಮ್ಮ ಹೃದಯದ ಮೂಲಕ ಶೋಧಿಸಿದ ನಂತರ ಅದು, ನಮ್ಮ ಜ್ಞಾನೇಂದ್ರಿಯಗಳ ಹಾಗೂ ಕರ್ಮೇoದ್ರಿಯಗಳ ಮೂಲಕ  ವ್ಯಕ್ತವಾಗಬೇಕು. ನಾವು ಮಾಡುವ ಕಾರ್ಯಗಳಲ್ಲಿ ಶುದ್ಧತೆ ಇದ್ದರೆ ಅದು, ನಮ್ಮ ಹೃದಯದ ತಪ್ಪು ಪ್ರವೃತ್ತಿಗಳನ್ನು ಶುದ್ಧೀಕರಿಸುತ್ತದೆ.
\end{inspiration}
\newpage

\slcol{\Index{ತಸ್ಮಾತ್ತ್ವಮಿಂದ್ರಿಯಾಣ್ಯಾದೌ} ನಿಯಮ್ಯ ಭರತರ್ಷಭ ।\\
ಪಾಪ್ಮಾನಂ ಪ್ರಜಹಿ ಹ್ಯೇನಂ ಜ್ಞಾನವಿಜ್ಞಾನನಾಶನಮ್ ॥ ೪೧ ॥} 
\cquote{ಅರ್ಜುನ, ಆದುದರಿಂದ ನೀನು ಮೊದಲು ಇಂದ್ರಿಯಗಳನ್ನು ಬಿಗಿಹಿಡಿದು ಜ್ಞಾನವನ್ನೂ ಅನುಭವವನ್ನೂ ಹಾಳು ಮಾಡುವ ಈ ಪಾಪಿಯನ್ನು ಗೆಲ್ಲು.\\}
\slcol{\Index{ಇಂದ್ರಿಯಾಣಿ ಪರಾಣ್ಯಾಹು}ರಿಂದ್ರಿಯೇಭ್ಯಃ ಪರಂ ಮನಃ ।\\
ಮನಸಸ್ತು ಪರಾ ಬುದ್ಧಿರ್ಯೋ ಬುದ್ಧೇಃ ಪರತಸ್ತು ಸಃ ॥ ೪೨ ॥}
\cquote{ಸ್ತೂಲ ದೇಹಕ್ಕಿಂತ ಇಂದ್ರಿಯಗಳು ಹೆಚ್ಚಿನವು ಎನ್ನುವರು. ಇಂದ್ರಿಯಗಳಿಗಿಂತಲೂ ಮನಸ್ಸು ಹೆಚ್ಚಿನದು.ಮನಸ್ಸಿಗಿಂತಲೂ ಬುದ್ಧಿ ಹೆಚ್ಚಿನದು. ಬುದ್ಧಿಗೂ ನಿಲುಕದೆ ಅದರಾಚೆ ಇರುವವನೆ  ಭಗವಂತ.\\}
\slcol{\Index{ಏವಂ ಬುದ್ಧೇಃ ಪರಂ} ಬುದ್ಧ್ವಾ ಸಂಸ್ತಭ್ಯಾತ್ಮಾನಮಾತ್ಮನಾ ।\\
ಜಹಿ ಶತ್ರುಂ ಮಹಾಬಾಹೋ ಕಾಮರೂಪಂ ದುರಾಸದಮ್ ॥ ೪೩ ॥}
\cquote{ಅರ್ಜುನ, ಹೀಗೆ ಬುದ್ದಿಗೂ ನಿಲುಕದ ಹಿರಿಯ ತತ್ವವನ್ನು ತಿಳಿದು ಪ್ರಯತ್ನಿದಿಂದ ಮನಸ್ಸನ್ನು ನಿಯಂತ್ರಿಸಿ ಕಾಮವೆಂಬ ಕೆಟ್ಟ ಶತ್ರುವನ್ನು ಹೋಗಲಾಡಿಸು.\\}
\begin{center}
ಓಂ ತತ್ಸದಿತಿ ಶ್ರೀಮದ್ಭಗವದ್ಗೀತಾಸೂಪನಿಷತ್ಸು \\ಬ್ರಹ್ಮವಿದ್ಯಾಯಾಂ ಯೋಗಶಾಸ್ತ್ರೇ ಶ್ರೀಕೃಷ್ಣಾರ್ಜುನಸಂವಾದೇ\\
ಕರ್ಮಯೋಗೋ ನಾಮ ತೃತೀಯೋऽಧ್ಯಾಯಃ ॥ 3 ॥
\end{center}

\newpage
\begin{mananam}{\mananamfont ಮನನ ಶ್ಲೋಕ - ೪೧}
\mananamtext ನನ್ನ ಜೀವನದ ಯಾವ ಕ್ಷೇತ್ರದಲ್ಲಿ ಸ್ವ–ನಿಯಂತ್ರಣದ ಅಗತ್ಯತೆ ಇದೆ? ನೋಡುವುದು, ಕೇಳುವುದು, ಆಸ್ವಾದಿಸುವುದು ಮುಂತಾದ, ನನ್ನ ಜ್ಞಾನೇಂದ್ರಿಯಗಳನ್ನು ಹತೋಟಿಯಲ್ಲಿಡುವ ಸಾಮರ್ಥ್ಯ ನನಗಿದೆಯೇ? ನನಗೆ,  ಮಾತನಾಡುವುದು, ನಡೆಯುವುದು, ಲೈಂಗಿಕತೆ ಮುಂತಾದ ಕರ್ಮೇoದ್ರಿಯಗಳನ್ನು  ಹತೋಟಿಯಲ್ಲಿಡುವ ಸಾಮರ್ಥ್ಯ ಇದೆಯೇ? ಈ ನಿರ್ಬಂಧನೆಗಳನ್ನು ದಿನನಿತ್ಯದಲ್ಲಿ ಅನುಷ್ಠಾನಗೊಳಿಸಲು ನಾನು ಯಾವ ಮೊದಲ ಹೆಜ್ಜೆಯನ್ನು ತೆಗೆದುಕೊಳ್ಳಬೇಕು? ಈ ‘ಸ್ವ-ನಿಯಂತ್ರಣವು’ ನನ್ನ ಜೀವನದಲ್ಲಿ ಕಡುಕಷ್ಟ ಬಂದಾಗಲಷ್ಟೇ ನಿಜವಾದ ಪರೀಕ್ಷೆಯ ಒರೆಗಲ್ಲಿಗೆ ಒಳಪಡುತ್ತದೆ ಎಂಬುದನ್ನು ಗ್ರಹಿಸಿದ್ದೇನೆಯೇ? 
\end{mananam}
\WritingHand\enspace\textbf{ಆತ್ಮ ವಿಮರ್ಶೆ}\\
\begin{inspiration}{\mananamfont ಸ್ಪೂರ್ತಿ}
\mananamtext ಯೋಗವು ಅಶಿಸ್ತಿನ ಇಂದ್ರಿಯಗಳನ್ನು ನಿಗ್ರಹಿಸುವ ಕಲೆಯಾಗಿದೆ. ಯಾವಾಗ ಇಂದ್ರಿಯಗಳನ್ನು ನಿಯಂತ್ರಿಸಲಾಗುತ್ತದೆಯೋ ಆವಾಗ, ವಿವೇಕವನ್ನು ಅಜ್ಞಾನವು ಆವರಿಸುವುದಿಲ್ಲ; ಸರಿಯಾದ ಕಾರ್ಯವನ್ನು ಮಾಡಲು, ನಮಗೆ ಮಾರ್ಗದರ್ಶನ ನೀಡುವ ಅಗತ್ಯವಿರುವ ಸಮಯದಲ್ಲಿ ಅದು (ವಿವೇಕ), ನಮಗೆ ಲಭ್ಯವಾಗುತ್ತದೆ.
\end{inspiration}

\newpage
\begin{mananam}{\mananamfont ಮನನ ಶ್ಲೋಕ - ೪೨ ೪೩}
\mananamtext ‘ಸೂಕ್ಷ್ಮವು ಶ್ರೇಷ್ಠವಾದುದು ಮತ್ತು ಅದು ಸ್ಥೂಲವಾದುದನ್ನು ನಿಯಂತ್ರಿಸುತ್ತದೆ’ ಎಂದು ನಾನು ನೋಡುವುದರ  ಅರ್ಥವೇನು?
ನನ್ನ ಇಂದ್ರಿಯಗಳು ಹೇಗೆ ನನ್ನ ದೇಹವನ್ನು ನಿರ್ದೇಶಿಸುತ್ತವೆ, ಎಂಬ ಬಗ್ಗೆ ನನ್ನ ಗಮನವಿದೆಯೇ?\\
ಮನಸ್ಸು ಮತ್ತು ಇಂದ್ರಿಯಗಳ ನಡುವಿನ ಸಂಬಂಧಕ್ಕೆ, ವಿಶೇಷವಾಗಿ ನಿದ್ರೆಯ ಸ್ಥಿತಿಯಲ್ಲಿ, ಮನಸ್ಸು ಸಕ್ರಿಯವಾಗಿಲ್ಲದಿದ್ದಾಗ ಮತ್ತು ಇಂದ್ರಿಯಗಳ ಮೇಲೆ ಅದು ಬೀರುವ ಪರಿಣಾಮಕ್ಕೆ  ನಾನು ಸಂಬಂಧಿಸಬಲ್ಲೆನೇ? ಅದರ ಅರ್ಥ ಏನೆಂದು ಗ್ರಹಿಸಬಲ್ಲೆನೇ?\\
‘ನನ್ನ ನಿಜವಾದ ಸ್ವಭಾವವನ್ನು ನಾನು ತಿಳಿದುಕೊಳ್ಳುವುದು’ ಎಂದರೆ ನನಗೆ ಏನು ಅರ್ಥ ಕೊಡುತ್ತದೆ? ನನ್ನ ಜೀವನದ ದೃಷ್ಟಿಕೋನದಲ್ಲಿ ಮತ್ತು ಪ್ರಾಪಂಚಿಕ ಕಾರ್ಯ ನಿರ್ವಹಿಸುವಲ್ಲಿ, ಅದು ಬದಲಾವಣೆಗೆ ಹೇಗೆ ಕಾರಣವಾಗುತ್ತದೆ?
\end{mananam}
\WritingHand\enspace\textbf{ಆತ್ಮ ವಿಮರ್ಶೆ}\\
\begin{inspiration}{\mananamfont ಸ್ಪೂರ್ತಿ}
\tiny \mananamtext ನಮ್ಮ ಚೈತನ್ಯ ಪ್ರಜ್ಞೆಯು ನಮ್ಮಲ್ಲಿರುವ ಅತೀ ನೇರವಾದ ಮತ್ತು ಅತ್ಯುತ್ತಮವಾದ ಅಸ್ತಿತ್ವವಾಗಿದೆ; ಇದು ‘ದೃಷ್ಟ’ ವಾದರೆ (ಸಾಕ್ಷ್ಯಾಭಾವ ಅಥವಾ ಆತ್ಮ), ಆ ಮತ್ತೆಲ್ಲವೂ (ಬುದ್ಧಿಶಕ್ತಿ, ಮನಸ್ಸು, ಜ್ಞಾನ ಮತ್ತು ದೇಹ) ಇದಕ್ಕೆ ‘ದೃಶ್ಯ’ ವಾಗಿವೆ (ಗಮನಕ್ಕೊಡಲ್ಪಡುವ ಸೂಕ್ಷ್ಮ ಹಾಗೂ ಸ್ಥೂಲ  ವಸ್ತುಗಳಾಗಿವೆ).\\
ಆಸೆಗಳ ಬೆಂಬೆತ್ತುವುದು, ಆತ್ಮವು ತನ್ನನ್ನು ತಾನು, ವಸ್ತುನಿಷ್ಠವಾಗಿ (ಸೂಕ್ಷ್ಮ ಹಾಗೂ ಸ್ಥೂಲ ರೂಪದಲ್ಲಿರುವ  ವಸ್ತುಗಳಲ್ಲಿ) ಗುರುತಿಸುಕೊಳ್ಳುವುದಾಗಿದೆ; ಐoದ್ರಿಕ ಇಚ್ಛೆಗಳ ಪ್ರವೃತ್ತಿಗಳು ನಮ್ಮ ನೈಜ ಸ್ವಭಾವದಿಂದ ನಮ್ಮನ್ನು ದೂರ ಮಾಡುತ್ತದೆ.\\
ನಮ್ಮ ಅಸ್ತಿತ್ವದ ತತ್ವವಾದ, ಎಲ್ಲಾ ಸಂತೋಷಗಳಿಗೂ ಮೂಲವಾದ ಇದನ್ನು (ಸಾಕ್ಷ್ಯಾಭಾವ ಅಥವಾ ಆತ್ಮ), ಸರಿಯಾಗಿ ಅರ್ಥ ಮಾಡಿಕೊಂಡಾಗ, ಇದು, ಇಚ್ಛೆಗಳ ಹಿಡಿತವನ್ನು ಸಡಿಲಗೊಳಿಸುತ್ತದೆ.
\end{inspiration}
\newpage


