\mananamtext{\indent ನಾವೆಲ್ಲರೂ ಜೀವನದ ಹೋರಾಟಗಳನ್ನು ಎದುರಿಸಲೇಬೇಕು. ಕುರುಕ್ಷೇತ್ರ ಯುದ್ಧದಲ್ಲಿ ಶ್ರೀ ಕೃಷ್ಣ ಪರಮಾತ್ಮನು, ತನ್ನ ವೇದನಾಯುಕ್ತ ಶಿಷ್ಯ ಅರ್ಜುನನಿಗೆ, ಪ್ರಾಪಂಚಿಕತೆಯಲ್ಲಿಯೂ ಅಧ್ಯಾತ್ಮಿಕತೆಯನ್ನು ಆಚರಣೆಗೆ ತರುವಂತಹ, ಸಂಕ್ಷಿಪ್ತ ಹಾಗೂ ಪ್ರಾಯೋಗಿಕವಾದ,  ಅತೀ ಪವಿತ್ರವಾದ ಬೋಧನೆಗಳನ್ನು ಕೊಟ್ಟಿದ್ದಾನೆ. ಈ ಶ್ರೇಷ್ಠವಾದ ಉಪನಿಷತ್ತುಗಳ ಸತ್ವಗಳನ್ನೊಳಗೊಂಡ  ಬೋಧನೆಗಳನ್ನು ಪವಿತ್ರವಾದ, ‘ಭಗವದ್ಗೀತ, ಒಂದು ಪವಿತ್ರ ಗಾನ’ ದ ಸ್ವರೂಪದಲ್ಲಿ, ಋಷಿ ವೇದವ್ಯಾಸರು ನಮಗೆ ನೀಡಿರುವ ಅಸೀಮವಾದ ಕೊಡುಗೆ. \\
 ಅರ್ಜುನನು ಇದ್ದ ಪರಿಸ್ಥಿತಿಗೂ, ನಾವು ಇರುವ ಪರಿಸ್ಥಿತಿ ಮತ್ತು ಸಂಘರ್ಷಗಳಿಗೂ ವ್ಯತ್ಯಾಸಗಳಿರಬಹುದು. ಆದರೆ, ಗೀತೆಯ ಸಾರ್ವತ್ರಿಕ ಉಪದೇಶಗಳು ಸತ್ಯಾನ್ವೇಷಣೆ ಮಾಡಲು ಬಯಸುವ ಪ್ರತಿಯೊಬ್ಬನಿಗೂ ಆತ್ಮೋನ್ನತಿ  ಮತ್ತು ಅಧ್ಯಾತ್ಮಿಕ ಪ್ರಗತಿ ಸಾಧಿಸಲು ಬೇಕಾಗುವ ಮಾದರಿಯಾಗಿದೆ.\\
 ಭಗವದ್ಗೀತೆಯ ಉಪದೇಶಗಳು ಕೇವಲ ಆಧ್ಯಾತ್ಮಿಕ ಅನ್ವೇಷಣೆ ಮಾಡುವವರಿಗೆ ಸಮರ್ಪಿತವಾದದ್ದು ಮಾತ್ರವೇ ಅಲ್ಲ, ಜೀವನಕ್ಕೆ ಬೇಕಾಗುವ ಅತ್ಯಮೂಲ್ಯವಾದ ಕೈಪಿಡಿಯೂ ಆಗಿದೆ. ಯಾರು, ಕೆಲಸದಲ್ಲಿ ಮತ್ತು ಕೌಟುಂಬಿಕ ಜವಾಬ್ದಾರಿಗಳಲ್ಲಿ ಒತ್ತಡ ರಹಿತವಾಗಿ, ಸಮತೋಲನ ಮತ್ತು ಮಾನಸಿಕ ನೆಮ್ಮದಿ ಕಾಪಾಡಿಕೊಳ್ಳಲು  ಬಯಸುತ್ತಾರೋ ಅವರಿಗೆ ಈ ಬೋಧನೆಗಳು ಬಹಳ ಮಹತ್ವದ್ದಾಗಿರುತ್ತವೆ. \\
 ಅನೇಕ ಗುರುಗಳು ಮತ್ತು ವಿದ್ವಾಂಸರು ಈಗಾಗಲೇ ಮಾಡಿರುವಂತೆ ಈ ದಿನಚರಿ ಪುಸ್ತಕ ಮತ್ತು ನಿಯತಕಾಲಿಕವು, ಗೀತೆಯ ಬೋಧನೆಗಳನ್ನು ತಿಳಿಸುವ ಪ್ರಯತ್ನ ಅಥವಾ ವ್ಯಾಖ್ಯಾನ ಕೊಡುವುದಾಗಿಲ್ಲ. ಈ ಗೀತಾ ಮನನವು, ಬೋಧನೆಗಳ ಚಿಂತನೆ ಮಾಡುವುದು ಮತ್ತು ಅದನ್ನು ನಮ್ಮ ಸ್ವಂತದ್ದನ್ನಾಗಿ ಅಂದರೆ, ಜೀವನದಲ್ಲಿ ಅಳವಡಿಸಿಕೊಳ್ಳಲು ಸುಲಭವಾಗುವಂತೆ ಮಾಡಿಕೊಳ್ಳುವುದೇ ಆಗಿದೆ . ದೇವ ನಾಗರಿಯಲ್ಲಿರುವ `ಮನನ` ಎಂಬ ಪದವು ಆಗಲೇ ಕೇಳಿದ್ದನ್ನು ಅಥವಾ ಓದಿದ್ದನ್ನು ಚಿಂತನೆ ಮಾಡುವ ಕಾರ್ಯವಿಧಾನವನ್ನು ಅನ್ವಯಿಸುವುದಾಗಿದೆ.\\
 ಈ ದಿನಚರಿ ಪುಸ್ತಕವನ್ನು ನೀವು, ನಿಮ್ಮ ಮನಸ್ಸಿನ ಇಂಗಿತವನ್ನು ಸ್ವತಂತ್ರವಾಗಿ ವ್ಯಕ್ತಪಡಿಸಲು  ಮತ್ತು ನಿಮ್ಮ ಜೀವನದಲ್ಲಿ ಅಳವಡಿಸಿಕೊಳ್ಳಲು ಅವಕಾಶ ಮಾಡಿಕೊಡುವ ಸಲುವಾಗಿ  ರೂಪಿಸಲಾಗಿದೆ. ಗೀತೆಯಲ್ಲಿರುವ ಶ್ಲೋಕಗಳ ಆಧಾರದ ಮೇಲೆ ರಚಿಸಲಾಗಿರುವ ಈ ಪ್ರಶ್ನೆಗಳು, ಆಯಾ ಬೋಧನೆಗಳ ಸನ್ನಿವೇಶಕ್ಕೆ ತಕ್ಕಂತೆ, ನಿಮ್ಮ ವೈಯಕ್ತಿಕ ಅರ್ಥಗಳನ್ನು ಹುಡುಕಲು ಮತ್ತು ಅದರಿಂದ ಜೀವನದ ಸಂದರ್ಭದೊಳಗೆ ಅಪಾರ ಸ್ಪಷ್ಟನೆ ದೊರಕಿಸಲು ಸಹಾಯಕವಾಗುವಂತೆ ರೂಪಿಸಲಾಗಿದೆ.\\
 ಶ್ರಿ\!\char"0CD5ಕೃಷ್ಣ ಪರಮಾತ್ಮನು  ಅರ್ಜುನನಿಗೆ ಧಾರ್ಮಿಕ ಯುದ್ಧವನ್ನು ಮಾಡಲು ಪ್ರೇರೇಪಿಸಿದಂತೆ, ನಿಮ್ಮ ಜೀವನದ ದಿನನಿತ್ಯದ ಕರ್ತವ್ಯಗಳನ್ನು ಈ “ಗೀತಾ ಮನನಮ್ “ ಮೂಲಕ  ಸಮರ್ಪಕವಾಗಿ ನಿರ್ವಹಿಸಲು,  ಆ ಭಗವಂತ ನಿಮ್ಮನ್ನೂ ಪ್ರೇರೇಪಿಸುತ್ತಾನೆ ಎಂದು ನಂಬುತ್ತೇನೆ. ನಿಮ್ಮ ಅಂತರಂಗದ ಶಾಂತಿ, ವೈಯಕ್ತಿಕ ಪ್ರಗತಿಯನ್ನು ನಿರ್ಲಕ್ಷಿಸದೇ, ನಿಮ್ಮ ಕರ್ತವ್ಯಗಳನ್ನು ಕುಶಲತೆಯಿಂದ ಯಶಸ್ವಿಯಾಗಿ ನಿರ್ವಹಿಸುತ್ತಾ  ಮತ್ತು ನಿಶ್ಚಲವಾಗಿ ದೈವತ್ವದಲ್ಲಿ ಮನಸನ್ನಿಡುವುದೇ, ಈ ದಿವ್ಯವಾದ ಗೀತೆಯ ನಿರಂತರ ಉದ್ದೇಶ.\\
ನಾನು ಈ ಪುಸ್ತಕದಲ್ಲಿ ಬಳಸಿರುವ ಚಿತ್ರಕಲೆ ಮತ್ತು ರೇಖಾಚಿತ್ರಗಳಿಗಾಗಿ ಶ್ರಿ\!\char"0CD5ಯುತ ಕೆ.ಎಂ.ಶೇಷಗಿರಿ ಅವರಿಗೆ ಧನ್ಯವಾದಗಳನ್ನು ಸಲ್ಲಿಸುತ್ತೇನೆ. ನನ್ನ ಗೀತಾ ತರಗತಿಯಲ್ಲಿ ಭಾಗವಹಿಸಿದ್ದ ಅನೇಕ ವಿದ್ಯಾರ್ಥಿಗಳು ಶ್ಲೋಕಗಳ ಭಾಷಾಂತರ, ತಿದ್ದುವಿಕೆ, ಸಂಪಾದನೆ, ವಿನ್ಯಾಸ ಮತ್ತು ಮುದ್ರಣ ಪ್ರಕ್ರಿಯೆಯನ್ನು ಗಮನಿಸುವಲ್ಲಿ ತೊಡಗಿಸಿಕೊಂಡಿದ್ದಾರೆ. ಈ ಗ್ರಂಥವನ್ನು ಓದುಗರ ಹಿತಾರ್ಥಕ್ಕಾಗಿ ಸಮರ್ಪಣೆಯಿಂದ ಮಾಡಿದ ಅವರ ತ್ಯಾಗಮಯ ಸೇವೆಗೆ ಭಗವಂತನ ಕೃಪೆ ಹಾಗು ನನ್ನ ಆಶೀರ್ವಾದಗಳು. \\\\
}
{
\kanBold{ಸ್ವಾಮಿ ನಿರ್ಗುಣಾನಂದ ಗಿರಿ}
}
