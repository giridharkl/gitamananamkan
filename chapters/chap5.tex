\slcol{ಅರ್ಜುನ ಉವಾಚ ।\\
\Index{ಸಂನ್ಯಾಸಂ ಕರ್ಮಣಾಂ} ಕೃಷ್ಣ ಪುನರ್ಯೋಗಂ ಚ ಶಂಸಸಿ ।\\
ಯಚ್ಛ್ರೇಯ ಏತಯೋರೇಕಂ ತನ್ಮೇ ಬ್ರೂಹಿ ಸುನಿಶ್ಚಿತಮ್ ॥ ೧ ॥}
\cquote{ಅರ್ಜುನನು ಹೇಳಿದನು, ಕೃಷ್ಣ, ಕರ್ಮಗಳ ತ್ಯಾಗವನ್ನು ಹೇಳುತ್ತೀ, ಫಲದಾಸೆ ಇಲ್ಲದೆ ಮಾಡೆಂದೂ ಹೇಳುತ್ತಿ. ಇವೆರಡರಲ್ಲಿ ಯಾವುದು ಮೇಲೋ ಆ ಒಂದನ್ನು ನನಗೆ ಗೊತ್ತು ಮಾಡಿ ಹೇಳು.}
\slcol{ಶ್ರೀಭಗವಾನುವಾಚ ।\\
\Index{ಸಂನ್ಯಾಸಃ ಕರ್ಮಯೋಗಶ್ಚ} ನಿಃಶ್ರೇಯಸಕರಾವುಭೌ ।\\
ತಯೋಸ್ತು ಕರ್ಮಸಂನ್ಯಾಸಾತ್ಕರ್ಮಯೋಗೋ ವಿಶಿಷ್ಯತೇ ॥ ೨ ॥}
\cquote{ಭಗವಂತನು ಹೇಳಿದನು, ಸಂನ್ಯಾಸ ಹಾಗೂ ಕರ್ಮಯೋಗ ಎರಡೂ ಶ್ರೇಯಸ್ಸನ್ನು ಉಂಟುಮಾಡುವಂತಹವೇ. ಅವುಗಳಲ್ಲಿ ಕರ್ಮಸಂನ್ಯಾಸಕ್ಕಿಂತ ಕರ್ಮಯೋಗವು ಶ್ರೇಯಸ್ಕರ.}
\slcol{\Index{ಜ್ಞೇಯಃ ಸ ನಿತ್ಯಸಂನ್ಯಾಸೀ} ಯೋ ನ ದ್ವೇಷ್ಟಿ ನ ಕಾಂಕ್ಷತಿ ।\\
ನಿರ್ದ್ವಂದ್ವೋ ಹಿ ಮಹಾಬಾಹೋ ಸುಖಂ ಬಂಧಾತ್ಪ್ರಮುಚ್ಯತೇ ॥ ೩ ॥}
\cquote{ಮಹಾಬಾಹೋ! ಒಂದು ಬೇಕೆನ್ನದೆ, ಒಂದು ಬೇಡವೆನ್ನದೆ ಇರುವವನೇ ನಿತ್ಯಸಂನ್ಯಾಸಿ. ದ್ವಂದ್ವ ರಹಿತನಾದ ಅವನು ಬಂಧನದಿಂದ ಸುಖವಾಗಿ ತಪ್ಪಿಸಿಕೊಳ್ಳುತ್ತಾನೆ.}


\newpage
\begin{mananam}{\mananamfont ಮನನ ಶ್ಲೋಕ - ೨}
\small \mananamtext ನಾನು ಅಗತ್ಯಕ್ಕಿಂತ ಹೆಚ್ಚಾದ ಚಟುವಟಿಕೆಗಳಲ್ಲಿ ಮುಳುಗಿದ್ದೇನೆಯೇ?  ಉಪಯೋಗವಿಲ್ಲದ ಅಥವಾ ಅಗತ್ಯವಿಲ್ಲದ  ಚಟುವಟಿಕೆಗಳಿಂದ ಹಿಂದಕ್ಕೆ ಸರಿಯಬಲ್ಲೆನೇ? ಮತ್ತೊಂದೆಡೆ, ವಿಶೇಷವಾಗಿ ಒತ್ತಡದಲ್ಲಿದ್ದಾಗ ಅಥವಾ ಅಡೆತಡೆಗಳನ್ನು ಎದುರಿಸಿದಾಗ, ನನ್ನ ಅಗತ್ಯ ಕರ್ತವ್ಯಗಳು ಮತ್ತು ಜವಾಬ್ದಾರಿಗಳನ್ನು ನಾನು ತ್ಯಜಿಸಿ ಬಿಡುತ್ತೇನೆಯೇ?  ಅಗತ್ಯವಾಗಿ ಮಾಡಲೇಬೇಕಾದ್ದು ಮತ್ತು ಅನಗತ್ಯ ಕೆಲಸಗಳ ಮಧ್ಯೆ ಸಮತೋಲನ ಸಾಧಿಸಲು ಕಲಿಯಬಲ್ಲೆನೇ? ಒಂದು ಆಂತರಿಕ ನಿರ್ಲಿಪ್ತತೆಯನ್ನು ಕಾಪಾಡಿಕೊಂಡು, ಯಾವುದೇ ಕಾರ್ಯಗಳಲ್ಲಿ ಸಂಪೂರ್ಣವಾಗಿ ನನ್ನನ್ನು ನಾನು ತೊಡಗಿಸಿಕೊಳ್ಳುವುದು ಹೇಗೆ?(ಅಂದರೆ, ವಿಷಯಗಳಲ್ಲೂ  ಕಾರ್ಯಗಳಲ್ಲೂ ಮೋಹಗೊಳ್ಳದಿರುವಿಕೆ).
\end{mananam}
\WritingHand\enspace\textbf{ಆತ್ಮ ವಿಮರ್ಶೆ}\\
\begin{inspiration}{\mananamfont ಸ್ಫೂರ್ತಿ}
\small \mananamtext ಚಟುವಟಿಕೆಯ ಬಗ್ಗೆ ಇರುವ ಎರಡು ಆಧ್ಯಾತ್ಮಿಕ ಮನೋವೃತ್ತಿಗಳು - ಅಂದರೆ, ತ್ಯಜಿಸುವಿಕೆ (ಸಂನ್ಯಾಸ) ಮತ್ತು ಪ್ರತಿಫಲಗಳ ನಿರೀಕ್ಷೆಯಿಲ್ಲದ ಕರ್ತವ್ಯ ಪಾಲನೆ – ಇವೆರಡರಲ್ಲಿ ಎರಡನೆಯದು, ಬಿಡಲಾರದ ಜವಾಬ್ದಾರಿಗಳು ಮತ್ತು ಕರ್ತವ್ಯಗಳನ್ನು ಹೊಂದಿರುವ ಬಹುಪಾಲು ಜನರಿಗೆ ಹೆಚ್ಚಿನ ಪ್ರಯೋಜನವನ್ನು ನೀಡುತ್ತದೆ. ಆದಾಗ್ಯೂ, ತನ್ನನ್ನು ತಾನು ಯಾವಾಗ ಕ್ರಿಯೆಯಲ್ಲಿ ತೊಡಗಿಸಿಕೊಳ್ಳಬೇಕು  ಮತ್ತು ತನ್ನ ಆಧ್ಯಾತ್ಮಿಕ ಹಾದಿಯಲ್ಲಿ ಅಗತ್ಯವಿಲ್ಲದ ಅಥವಾ ಸಹಾಯಕಾರಿಯಲ್ಲದ ಚಟುವಟಿಕೆಗಳಿಂದ ಯಾವಾಗ ಹಿಂದೆ ಸರಿಯಬೇಕು ಎಂಬುದನ್ನು ತಿಳಿದುಕೊಳ್ಳುವುದು ಬುದ್ಧಿವಂತ ಮಾರ್ಗವಾಗಿದೆ. ಅದಾಗಿಯೂ,  ಎಲ್ಲರೂ ಯಾವ ಚಟುವಟಿಕೆಗಳಲ್ಲಿ ಕಾರ್ಯ ಪ್ರವೃತ್ತರಾಗಬೇಕು ಮತ್ತು ಅದರಿಂದ ಯಾವಾಗ ಹೊರಬರಬೇಕು ಎಂಬ ಬುದ್ಧಿವಂತಿಕೆಯನ್ನು. ಬೆಳೆಸಿಕೊಳ್ಳಬೇಕು, ಅಗತ್ಯತೆ ಇದ್ದಾಗ ತ್ವರಿತಗತಿಯಲ್ಲಿ ಚಟುವಟಿಕೆಯಲ್ಲಿ ತೊಡಗಿಸಿಕೊಳ್ಳುವುದು ಹಾಗೂ, ಅನಗತ್ಯತೆ ಎಂದೆನಿಸಿದಾಗ ಅಷ್ಟೇ ತ್ವರಿತಗತಿಯಲ್ಲಿ ಅದರಿಂದ ಹಿಂದೆ ಸರಿಯುವ ವಿವೇಕವನ್ನು ಪ್ರತಿಯೊಬ್ಬರೂ ಬೆಳೆಸಿಕೊಳ್ಳಬೇಕು.
\end{inspiration}
\newpage

\begin{mananam}{\mananamfont ಮನನ ಶ್ಲೋಕ - ೩}
\small \mananamtext ನನ್ನ ಇಷ್ಟಗಳು ಮತ್ತು ಅನಿಷ್ಟಗಳ ಬಗ್ಗೆ ನನಗೆ ತಿಳಿದಿದೆಯೇ ಮತ್ತು ಅವು ನನ್ನ ಜೀವನದಲ್ಲಿಯ ಆಯ್ಕೆಗಳು ಮತ್ತು ನಿರ್ಧಾರಗಳನ್ನು ಹೇಗೆ ನಿಬಂಧಿಸುತ್ತವೆ?ಮಾಡಲೇಬೇಕಾದ ಕರ್ತವ್ಯಗಳ ಬಗ್ಗೆ ಸುಪ್ತ ಮನಸ್ಸಿನಲ್ಲಿ ವಿರೋಧವಿದ್ದಾಗ್ಯೂ, ಅಂಥಹ ಕರ್ತವ್ಯಗಳನ್ನು ಪಾಲಿಸಬಲ್ಲೆನೇ? ಹಾಗೂ, ಬಲವಾದ ಇಚ್ಛೆ ಮತ್ತು ಮೋಹವಿದ್ದಾಗ್ಯೂ, ನಿರರ್ಥಕವಾದ ಕೃತ್ಯಗಳನ್ನು ಮಧ್ಯದಲ್ಲೇ ತ್ಯಜಿಸುವ ಸಮರ್ಥತೆ ಇದೆಯೇ? ನಾನು ಮೋಹಕ್ಕೆ ಒಳಗಾಗದೇ ಉತ್ಸಾಹಭರಿತನಾಗಿ ಕಾರ್ಯ ಮಾಡುವುದನ್ನು ಹೇಗೆ ಕಲಿಯಲಿ?
\end{mananam}
\WritingHand\enspace\textbf{ಆತ್ಮ ವಿಮರ್ಶೆ}\\
\begin{inspiration}{\mananamfont ಸ್ಫೂರ್ತಿ}
\small \mananamtext ಓರ್ವ ಸಂನ್ಯಾಸಿಯಾಗಿರಬಹುದು ಅಥವಾ ಸಂಸಾರಿಯೇ ಆಗಿರಬಹುದು, ಇವರಿಬ್ಬರಿಗೂ ಕೂಡ ‘ಆಂತರಿಕ ಸಂನ್ಯಾಸ’ವು ಅತ್ಯಗತ್ಯ ಜೀವನ ಕೌಶಲ್ಯವಾಗಿದೆ; ಅಂತಹವರು ಮನಸ್ಸಿನಲ್ಲಿ ವಿರಕ್ತಿ ಭಾವವನ್ನು ಪರಿಪೂರ್ಣಗೊಳಿಸಿಕೊಂಡಂತಹ  ಸನಾತನ ಸoನ್ಯಾಸಿಯೇ ಆಗಿರುತ್ತಾರೆ; ಹೀಗೆ ಇರುವವನು ಅಥವಾ ಇರುವವಳು, ಸಾಧನೆಗೆ ಸಹಕಾರಿಯಾಗುವಂತಹ ಕೆಲವು ಜೀವನದ ಶೈಲಿ ಅಥವಾ ಅಭ್ಯಾಸಗಳಲ್ಲಿ ಆದ್ಯತೆ ಉಳ್ಳವರಾಗಿರುತ್ತಾರೆ. ಆದರೆ, ಆಂತರಿಕವಾಗಿ ಅವರು ಯಾವುದರ ಬಗ್ಗೆಯೂ ಅಥವಾ, ಯಾರ ಬಗ್ಗೆಯೂ ಇಷ್ಟ ಅಥವಾ ಅನಿಷ್ಟದ ಭಾವನೆಗಳನ್ನು  ಹೊಂದಿರುವುದಿಲ್ಲ. ಸಂದರ್ಭಗಳಿಗೆ ತಕ್ಕಂತೆ ಸಮುಚಿತ ನಿರ್ಧಾರ ಕೈಗೊಂಡು, ಅವರು ತೆಗೆದುಕೊಂಡಿರುವ ಯಾವುದನ್ನಾದರೂ (ಕಾರ್ಯವನ್ನೂ) ತ್ಯಜಿಸಬೇಕಾದಾಗ, ತ್ವರಿತವಾಗಿ ತ್ಯಜಿಸಿ, ಜೀವನದಲ್ಲಿ ಮುಂದೆ ಸಾಗುತ್ತಾರೆ. 
\end{inspiration}
\newpage

\slcol{\Index{ಸಾಂಖ್ಯಯೋಗೌ ಪೃಥಗ್ಬಾಲಾಃ} ಪ್ರವದಂತಿ ನ ಪಂಡಿತಾಃ ।\\
ಏಕಮಪ್ಯಾಸ್ಥಿತಃ ಸಮ್ಯಗುಭಯೋರ್ವಿಂದತೇ ಫಲಮ್ ॥ ೪ ॥}
\cquote{ಜ್ಞಾನಮಾರ್ಗ ಕರ್ಮ ಮಾರ್ಗಗಳು ಬೇರೆ ಬೇರೆ ಎಂಬುದು ತಿಳಿಯದವರ ಮಾತೇ ಹೊರತು ತಿಳಿದವರ ಮಾತಲ್ಲ. ಯಾವ ಒಂದನ್ನು ಚೆನ್ನಾಗಿ ನಡೆಸಿದರೂ ಅವನು ಎರಡರ ಫಲವನ್ನೂ ಪಡೆಯುತ್ತಾನೆ.}
\slcol{\Index{ಯತ್ಸಾಂಖ್ಯೈಃ ಪ್ರಾಪ್ಯತೇ} ಸ್ಥಾನಂ ತದ್ಯೋಗೈರಪಿ ಗಮ್ಯತೇ ।\\
ಏಕಂ ಸಾಂಖ್ಯಂ ಚ ಯೋಗಂ ಚ ಯಃ ಪಶ್ಯತಿ ಸ ಪಶ್ಯತಿ ॥ ೫ ॥}
\cquote{ಯಾವದೇ ಯೋಗಿಗಳಿಗೆ ಲಭಿಸುವ ಫಲವು ಕರ್ಮಯೋಗಿಗಳಿಗೂ ಲಭಿಸುತ್ತದೆ. ಆ ಎರಡನ್ನೂ ಒಂದೇ ಎಂದು ತಿಳಿದವನು ಯಥಾರ್ಥ ಜ್ಞಾನಿ.}
\slcol{\Index{ಸಂನ್ಯಾಸಸ್ತು ಮಹಾಬಾಹೋ} ದುಃಖಮಾಪ್ತುಮಯೋಗತಃ ।\\
ಯೋಗಯುಕ್ತೋ ಮುನಿರ್ಬ್ರಹ್ಮ ನಚಿರೇಣಾಧಿಗಚ್ಛತಿ ॥ ೬ ॥}
\cquote{ಮಹಾಬಾಹೋ, ಕರ್ಮ ಯೋಗದ ನೆರವಿಲ್ಲದೆ ರಾಗ ದ್ವೇಷಗಳನ್ನು ಮೀರಿ ನಿಲ್ಲುವುದು ಕಷ್ಟ. ಫಲದಾಸೆ ಇಲ್ಲದೆ ಕರ್ಮ ಮಾಡುವ ದ್ವಂದ್ವಾತೀತನು ತಡವಿಲ್ಲದೇ ಆತ್ಮ ಜ್ಞಾನವನ್ನು ಹೊಂದುತ್ತಾನೆ.}
\slcol{\Index{ಯೋಗಯುಕ್ತೋ ವಿಶುದ್ಧಾತ್ಮಾ} ವಿಜಿತಾತ್ಮಾ ಜಿತೇಂದ್ರಿಯಃ ।\\
ಸರ್ವಭೂತಾತ್ಮಭೂತಾತ್ಮಾ ಕುರ್ವನ್ನಪಿ ನ ಲಿಪ್ಯತೇ ॥ ೭ ॥}
\cquote{ಕರ್ಮ ಯೋಗದಲ್ಲಿ ನಿರತನಾಗಿ ಶುದ್ಧವಾದ ಮನಸ್ಸಿನಿಂದ ದೇಹವನ್ನು ಹಿಡಿತದಲ್ಲಿ ಇಟ್ಟುಕೊಂಡು ಇಂದ್ರಿಯಗಳನ್ನು ಬಿಗಿ ಹಿಡಿದು ಎಲ್ಲೆಡೆಯೂ ಅಂತರ್ಯಾಮಿಯಾಗಿರುವ ಭಗವಂತನನ್ನೇ ಸದಾ ನೆನೆಸುವವನು ಕರ್ಮ ಮಾಡಿದರೂ ಅದರ ಕಟ್ಟಿಗೊಳಗಾಗುವುದಿಲ್ಲ.}


\newpage
\begin{mananam}{\mananamfont ಮನನ ಶ್ಲೋಕ - ೬}
\small \mananamtext ನಾನು ಹೃದಯಪೂರ್ವಕವಾಗಿ ಅಧ್ಯಾತ್ಮಿಕ ಜೀವನವನ್ನು ಅನುಸರಿಸಲು ಬಯಸುತ್ತೇನೆಯೇ? ನಾನು ಆಧ್ಯಾತ್ಮದ ಅನ್ವೇಷಣೆಗಾಗಿ ಎಲ್ಲಾ ಚಟುವಟಿಕೆಗಳನ್ನೂ ಅಥವಾ ಹೆಚ್ಚಿನ ಇತರ ಚಟುವಟಿಕೆಗಳನ್ನು ಬಿಡಲು ಸಿದ್ಧನಿದ್ದೇನೆಯೇ? ಪ್ರಾಪಂಚಿಕ ಚಟುವಟಿಕೆಗಳನ್ನು ತ್ಯಜಿಸಲು ಅಥವಾ ಯಾವದೇ ಕಾರ್ಯಗಳಿಂದ  ನಿವೃತ್ತಿಯನ್ನು ಪಡೆಯಲು ನನಗೆ ಯೋಗ್ಯತೆ ಇದೆಯೇ – ಅದರ ಅವಧಿಯು ಸಂಪೂರ್ಣವಾಗುವ ಮುನ್ನವೇ ಇರಬಹುದು ಅಥವಾ ನಂತರವೇ ಆಗಿರಬಹುದು? ನನಗೆ ಆಧ್ಯಾತ್ಮಿಕ ಶಿಸ್ತು ಇದೆಯೇ ಅಥವಾ ‘ಸಾಧನೆ’ಯ ಬದ್ಧತೆಯನ್ನು ಹೊಂದಿದ್ದೇನೆಯೇ? ನಾನು ನನ್ನ ಸಮಯವನ್ನು ಸದುಪಯೋಗ ಪಡಿಸಿಕೊಳ್ಳುವುದು ಹೇಗೆ, ಹಾಗೂ, ನನ್ನ ಇಚ್ಛೆಯಂತೆ ಆಧ್ಯಾತ್ಮದ ಹಾದಿಯಲ್ಲಿ ಮುಂದುವರೆಯುವುದು ಹೇಗೆ? 
\end{mananam}
\WritingHand\enspace\textbf{ಆತ್ಮ ವಿಮರ್ಶೆ}\\
\begin{inspiration}{\mananamfont ಸ್ಫೂರ್ತಿ}
\small \mananamtext ಹೃದಯಪೂರ್ವಕವಾಗಿ ಆಧ್ಯಾತ್ಮಿಕ ದಾರಿಯಲ್ಲಿ ಸಾಗಲು  ಒಂದು ಸಾಧನೆಗೆ ಬದ್ಧನಾಗಿರಬೇಕು. ಶಿಸ್ತಿನ ಜೀವನವಿಲ್ಲದಿದ್ದರೆ, ಪ್ರಾಪಂಚಿಕ ಜೀವನವನ್ನು, ಭಾಗಷಃವಾಗಲೀ ಅಥವಾ ಪೂರ್ತಿಯಾಗಲೀ ತ್ಯಜಿಸುವುದು, ಖಂಡಿತವಾಗಿಯೂ ಇಂದ್ರಿಯಲೋಲುಪತೆಗೂ, ಸೋಮಾರಿತನಕ್ಕೂ ಕಾರಣವಾಗುತ್ತದೆ.  (ಇದನ್ನು ಇನ್ನೊಂದು ರೀತಿಯಲ್ಲಿ “ತಾಮಸ ಜೀವನ” ಎನ್ನುತ್ತಾರೆ). ಸಂಪೂರ್ಣವಾಗಿ ಆಧ್ಯಾತ್ಮಿಕ ಜೀವನ ಬಯಸುವವರಿಗೆ, ಯಾವುದೇ ಹಂತದಲ್ಲಿ ಅವರ ಜೀವನ ಇದ್ದರೂ ಕೂಡ ಅಂತಹವರಿಗೆ, ಸಂಪ್ರದಾಯಬದ್ಧವಾದ ಆಧ್ಯಾತ್ಮಿಕ ಪಥ ಎಂದರೆ - ಸೇವೆ, ಆರಾಧನೆ, ಧ್ಯಾನ ಮತ್ತು ಅಧ್ಯಯನಗಳಿಂದ ಕೂಡಿದ ದಾರಿಯೇ ಸರ್ವೋತ್ಕೃಷ್ಟವಾದ ದಾರಿಯಾಗಿದೆ.
\end{inspiration}
\newpage

\slcol{\Index{ನೈವ ಕಿಂಚಿತ್ಕರೋಮೀತಿ} ಯುಕ್ತೋ ಮನ್ಯೇತ ತತ್ತ್ವವಿತ್ ।\\
ಪಶ್ಯನ್‍ ಶೃಣ್ವನ್‍ ಸ್ಪೃಶನ್‍ ಜಿಘ್ರನ್ನಶ್ನನ್‍ ಗಚ್ಛನ್‍ ಸ್ವಪನ್‍ ಶ್ವಸನ್ ॥ ೮ ॥\\
\Index{ಪ್ರಲಪನ್‍ ವಿಸೃಜನ್ ಗೃಹ್ಣನ್ನು}ನ್ಮಿಷನ್ನಿಮಿಷನ್ನಪಿ ।\\
ಇಂದ್ರಿಯಾಣೀಂದ್ರಿಯಾರ್ಥೇಷು ವರ್ತಂತ ಇತಿ ಧಾರಯನ್ ॥ ೯ ॥}
\cquote{ಆತ್ಮ ಸ್ವರೂಪವನ್ನು ತಿಳಿದು ಮನಸ್ಸನ್ನು ಹಿಡಿತದಲ್ಲಿ ಇಟ್ಟುಕೊಂಡಿರುವವನು ನೋಡಿದರೂ ಕೇಳಿದರೂ ಮುಟ್ಟಿದರೂ ಮೂಸಿದರೂ ತಿಂದರೂ ಹೋದರೂ ನಿದ್ರೆ ಮಾಡಿದರರೂ ಉಸಿರುಬಿಟ್ಟರೂ ಮಾತನಾಡಿದರೂ ಮಲಮೂತ್ರಗಳನ್ನು ಬಿಟ್ಟರೂ ತೆಗೆದುಕೊಂಡರೂ ಕಣ್ಣು ಬಿಟ್ಟರೂ ಕಣ್ಣು ಮುಚ್ಚಿದರೂ ಇಂದ್ರಿಯಗಳು ವಿಷಯಗಳೊಡನೆ ವ್ಯವಹರಿಸುತ್ತಿರುವವೆಂದು ಮನಸ್ಸಿನಲ್ಲಿ ಎಣಿಸುತ್ತಾ ತಾನು ಏನನ್ನೂ ಮಾಡುವವನಲ್ಲವೆಂದು ತಿಳಿಯುವನು.}
\slcol{\Index{ಬ್ರಹ್ಮಣ್ಯಾಧಾಯ ಕರ್ಮಾಣಿ} ಸಂಗಂ ತ್ಯಕ್ತ್ವಾ ಕರೋತಿ ಯಃ ।\\
ಲಿಪ್ಯತೇ ನ ಸ ಪಾಪೇನ ಪದ್ಮಪತ್ರಮಿವಾಂಭಸಾ ॥ ೧೦ ॥}
\cquote{ಈಶ್ವರಾರ್ಪಣ ಬುದ್ಧಿಯಿಂದ ಫಲಾಸಕ್ತಿಯನ್ನು ಬಿಟ್ಟು ಕರ್ಮ ಮಾಡುವವನನ್ನು ಕಮಲದ ಎಲೆಯನ್ನು ನೀರು  ಮುಟ್ಟದಂತೆ, ಪಾಪವು ಅಂಟುವುದಿಲ್ಲ.}
\slcol{\Index{ಕಾಯೇನ ಮನಸಾ ಬುದ್ಧ್ಯಾ} ಕೇವಲೈರಿಂದ್ರಿಯೈರಪಿ ।\\
ಯೋಗಿನಃ ಕರ್ಮ ಕುರ್ವಂತಿ ಸಂಗಂ ತ್ಯಕ್ತ್ವಾತ್ಮಶುದ್ಧಯೇ ॥ ೧೧ ॥}
\cquote{ಕರ್ಮಯೋಗಿಗಳು ಆಸಕ್ತಿಯನ್ನು ಬಿಟ್ಟು ಶರೀರದಿಂದ ಮನಸ್ಸಿನಿಂದ ಬುದ್ಧಿಯಿಂದ ಆಯಾ ಇಂದ್ರಿಯಗಳಿಂದ ಆತ್ಮ ಶುದ್ಧಿಗಾಗಿ ಕರ್ಮಗಳನ್ನು ಮಾಡುತ್ತಾರೆ.}
\slcol{\Index{ಯುಕ್ತಃ ಕರ್ಮಫಲಂ ತ್ಯಕ್ತ್ವಾ} ಶಾಂತಿಮಾಪ್ನೋತಿ ನೈಷ್ಠಿಕೀಮ್ ।\\
ಅಯುಕ್ತಃ ಕಾಮಕಾರೇಣ ಫಲೇ ಸಕ್ತೋ ನಿಬಧ್ಯತೇ ॥ ೧೨ ॥}
\cquote{ಕರ್ಮದ ಫಲವನ್ನು ತೊರೆದು ಈಶ್ವರನ ಪ್ರೀತಿಗಾಗಿ ಮಾಡುವವನು ಶಾಶ್ವತವಾದ ಶಾಂತಿಯನ್ನು ಪಡೆಯುತ್ತಾನೆ. ಬಯಕೆಗಳಿಗೆ ಬಲಿಯಾಗಿ ಫಲದಾಸೆಯಿಂದ ಕರ್ಮ ಮಾಡುವವನು ಬಾಳಿನ ಬಂಧನಕ್ಕೊಳಗಾಗುವನು.}

\newpage
\begin{mananam}{\mananamfont {ಮನನ ಶ್ಲೋಕ - ೮, ೯}}
\small \mananamtext ಈ ಪ್ರಪಂಚದಲ್ಲಿ ಕೃತ್ಯಗಳಿಂದಾಗಿ ಗುರುತಿಸಲ್ಪಡದೆಲೇ ನಾನು ಕಾರ್ಯನಿರ್ವಹಿಸುವುದನ್ನು ಹೇಗೆ ಕಲಿಯಬಹುದು? ನನ್ನ ‘ಇಂದ್ರಿಯಗಳ ಕಾರ್ಯನಿರ್ವಹಣೆ’ ಹಾಗೂ ಅವುಗಳ ಬಗ್ಗೆ ನನ್ನ ಒಂದು ‘ಅನಾಸಕ್ತ ಅರಿವು’, ಇವೆರಡರ  ಮಧ್ಯೆ ಅಂತರವನ್ನು ಸೃಷ್ಟಿಸಬಲ್ಲೆನೇ? ಅಥವಾ, ‘ನಾನು ಇದ್ದೇನೆ’ ಎಂಬ ಪ್ರಜ್ಞೆಯನ್ನು  ನಾನು ಗಮನಿಸುತ್ತೇನೆಯೇ? ಮತ್ತು ತದನಂತರ ಹುಟ್ಟಿಕೊಳ್ಳುವ ‘ನಾನೇ ಮಾಡುವವನು’ ಎಂಬ ಪ್ರಜ್ಞೆಯನ್ನು ಗಮನಿಸುತ್ತೇನೆಯೇ? ಎಲ್ಲಿಯವರೆಗೂ ನಾನು ‘ಪ್ರೇಕ್ಷಕ’ನಾಗಿ ಒಂದು ವಸ್ತುವಿನ ಬಗ್ಗೆ ಋಣಾತ್ಮಕ ಅಥವಾ ಧನಾತ್ಮಕವಾಗಿ ಮಾನಸಿಕ ಪ್ರತಿಕ್ರಿಯೆಯನ್ನು ತೋರುವುದಿಲ್ಲವೋ ಅಲ್ಲಿಯವರೆಗೂ, ಆ ಒಂದು ವಸ್ತು ನನಗೆ ತಟಸ್ಥ (ನಿರಾಸಕ್ತ) ವಸ್ತು. ಒಂದು ವಸ್ತು ಅಥವಾ ವಿಷಯವನ್ನು, ‘ಕೇವಲ ನೋಡುವುದು ಮತ್ತು ಪೂರ್ವಗ್ರಹಪೀಡಿತವಾಗಿ ನೋಡುವುದು’ ಇವೆರಡೂ ಹೇಗೆ ಬೇರೆ, ಬೇರೆ ರೀತಿಯ ಅನಿಸಿಕೆಗಳನ್ನು ಮನಸ್ಸಿನ ಮೇಲೆ  ಪ್ರಭಾವ ಬೀರುತ್ತವೆ ಎಂದು ನನ್ನ ನಿತ್ಯ ಜೀವನದಲ್ಲಿ ಗಮನಿಸಬಲ್ಲೆನೇ?
\end{mananam}
\WritingHand\enspace\textbf{ಆತ್ಮ ವಿಮರ್ಶೆ}\\
\begin{inspiration}{\mananamfont ಸ್ಫೂರ್ತಿ}
\small \mananamtext ಸದಾ ಸ್ವಯಂ ಬಗ್ಗೆ ಅರಿವು ಇಟ್ಟುಕೊಂಡೇ (ಆತ್ಮ ಜ್ಞಾನದ ಅರಿವು) ನಮ್ಮ ದೈನಂದಿನ ಎಲ್ಲಾ ಚಟುವಟಿಕೆಗಳನ್ನೂ ನಿರ್ವಹಿಸುವ ಕಲೆಯನ್ನು ಕರಗತಮಾಡಿಕೊಂಡಲ್ಲಿ ನಮಗೆಲ್ಲಾ ಲಾಭದಾಯಕವಾಗುವುದು. ಅನಾವಶ್ಯಕವಾದ ಉದ್ರೇಕ ಅಥವಾ ಒತ್ತಡಕ್ಕೆ ಒಳಗಾಗದೇ ನಮ್ಮ ಕೆಲಸವನ್ನು ಮಾಡುವುದು ಮತ್ತು ನಮ್ಮ ಎಲ್ಲಾ ಅನುಭವಗಳನ್ನು ಎದುರಿಸುವುದೇ ಇದರ ಧ್ಯೇಯವಾಗಿದೆ. ನಾವು ಎಷ್ಟರಮಟ್ಟಿಗೆ ವೈಯಕ್ತಿಕವಾಗಿ ನಮ್ಮ ಚಟುವಟಿಕೆಗಳಿಂದ ಗುರುತಿಸಲ್ಪಡುತ್ತಿದ್ದೇವೆ ಮತ್ತು ಅದಕ್ಕಾಗಿ ಸಮಯ ಮೀಸಲಿಟ್ಟಿದ್ದೇವೆ ಎಂಬುದರ ಬಗ್ಗೆ ನಮಗೆ ಸದಾ ಸ್ವಯಂ ನಿರ್ಧಾರಿತ ಆಯ್ಕೆ ಇರುತ್ತದೆ; ಅಲ್ಲದೇ, ನಮ್ಮ ಚಟುವಟಿಕೆಗಳಿಂದ ಗುರುತಿಸಲ್ಪಡುತ್ತಿದ್ದರೂ ಸಹಿತ, ನಾವು ತ್ವರಿತವಾಗಿ ಹೇಗೆ ನಮ್ಮ ಸ್ವಾಭಾವಿಕ ಶಾಂತಿಯ ಅಸ್ತಿತ್ವಕ್ಕೆ ಹಿಂದಿರುಗುತ್ತೇವೆ ಎಂಬುದಕ್ಕೂ ಸದಾ ಸ್ವಯಂ ನಿರ್ಧಾರಿತ ಆಯ್ಕೆ ಇರುತ್ತದೆ.
\end{inspiration}
\newpage

\newpage
\begin{mananam}{\mananamfont ಮನನ ಶ್ಲೋಕ - ೧೧}
\small \mananamtext ಕೃತ್ಯಗಳಲ್ಲಿ ‘ಅಹಂ’ನ ಸ್ವಭಾವವೇನು?ಅದು ನನ್ನ ಜೀವನದಲ್ಲಿ ಹೇಗೆ ಪ್ರಕಟಿಸಲ್ಪಡುತ್ತದೆ? ಈ ‘ಅಹಂ’ನ ಶುದ್ಧೀಕರಣವನ್ನು ಹೇಗೆ ಸಾಧಿಸಬಹುದು?  ‘ಅಹಂ’ಅನ್ನು ಉತ್ತೇಜಿಸುವ ಎಲ್ಲಾ ಕೃತ್ಯಗಳನ್ನೂ ತ್ಯಾಗ ಮಾಡುವ ಅವಶ್ಯಕತೆ ಇದೆಯೇ? ಅಥವಾ ಸಮತೋಲನವಾದ ಯಾವುದಾದರೂ ಮಾರ್ಗವಿದೆಯೇ? ‘ಯೋಗಿಗಳು ಯಾವುದರ ಬಗ್ಗೆಯೂ ಮೋಹಕ್ಕೊಳಗಾಗದೇ  ಕರ್ತವ್ಯ ನಿರ್ವಹಿಸಲು ಸಮರ್ಥರಾಗಿರುತ್ತಾರೆ’ ಎನ್ನುವುದರ ಅರ್ಥವೇನು? ಅವರು ‘ಅಹಂ’ನಿಂದ ಸಂಪೂರ್ಣ ಸ್ವತಂತ್ರರೇ? ಅಥವಾ  ಕೆಲವು ರೀತಿಯ ಅಂದರೆ, ಕಾರ್ಯ ಮಾಡಲು ಸಹಾಯವಾಗುವಂತಹ ಕ್ರಿಯಾತ್ಮಕತೆಗೆ ಸಂಬಂಧಿಸಿದ ‘ಅಹಂ’ ಅವರಲ್ಲಿದೆಯೇ? ಹಾಗಿದ್ದರೆ, ಅದರ ಸ್ವಭಾವವೇನು?
\end{mananam}
\WritingHand\enspace\textbf{ಆತ್ಮ ವಿಮರ್ಶೆ}\\
\begin{inspiration}{\mananamfont ಸ್ಫೂರ್ತಿ}
\small \mananamtext ಎಲ್ಲಾ ಸನಾತನ ಸಂಪ್ರದಾಯಗಳೂ ಮನಸ್ಸನ್ನು ಪರಿಶುದ್ಧಗೊಳಿಸುವ ಸಲುವಾಗಿ ಸ್ವಾರ್ಥರಹಿತ ಕೃತ್ಯಗಳಿಗೆ ಒತ್ತು ಕೊಟ್ಟಿವೆ. ಒಬ್ಬ ಸಾಧಕನು, ಯಾವುದೇ ರೀತಿಯ ಸಾಧನೆಯನ್ನು ಅನುಸರಿಸುತ್ತಿದ್ದರೂ ಕೂಡ ಅವನಿಗೆ, ‘ಸೇವೆ ಮತ್ತು ಕರ್ಮ ಯೋಗ’ಗಳು   ಅತ್ಯಮೂಲ್ಯ; ಇದರ ವಿನಃ, ಒಬ್ಬ ಸಾಧಕನ ಜೀವನದಲ್ಲಿ, ಬೇರೆ ಬೇರೆ ಹಂತಗಳಲ್ಲಿ, ವಿಧ ವಿಧವಾಗಿ ಪ್ರಕಟವಾಗಿ ಕಾರ್ಯ ನಿರ್ವಹಿಸುವ ಈ ‘ಅಹಂ’ ಅನ್ನು ತೊಡೆದು ಹಾಕಲು ಬಹಳ ಕಠಿಣವಾಗುತ್ತದೆ.
\end{inspiration}
\newpage

\slcol{\Index{ಸರ್ವಕರ್ಮಾಣಿ ಮನಸಾ} ಸಂನ್ಯಸ್ಯಾಸ್ತೇ ಸುಖಂ ವಶೀ ।\\
ನವದ್ವಾರೇ ಪುರೇ ದೇಹೀ ನೈವ ಕುರ್ವನ್ನ ಕಾರಯನ್ ॥ ೧೩ ॥}
\cquote{ಜ್ಞಾನಿಯು ದೇಹೇಂದ್ರಿಯಗಳನ್ನು ಹಿಡಿತದಲ್ಲಿಟ್ಟುಕೊಂಡು ತಾನು ಮಾಡುವವನಲ್ಲವೆಂಬ ಅನುಸಂಧಾನದಿಂದ ಒಂಬತ್ತು ಬಾಗಿಲುಗಳು ಉಳ್ಳ ದೇಹವೆಂಬ ಪಟ್ಟಣದಲ್ಲಿ ಹಾಯಾಗಿರುವನು. ಅವನು ಮಾಡಿದರೂ ಮಾಡಿದಂತಲ್ಲ. ಮಾಡಿಸಿದರೂ ಮಾಡಿಸಿದ್ದಂತಲ್ಲ. }
\slcol{\Index{ನ ಕರ್ತೃತ್ವಂ ನ ಕರ್ಮಾಣಿ} ಲೋಕಸ್ಯ ಸೃಜತಿ ಪ್ರಭುಃ ।\\
ನ ಕರ್ಮಫಲಸಂಯೋಗಂ ಸ್ವಭಾವಸ್ತು ಪ್ರವರ್ತತೇ ॥ ೧೪ ॥}
\cquote{ಜಗತ್ತಿನಲ್ಲಿ ಮಾಡುವಿಕೆ, ಮಾಡಿದ ಕ್ರಿಯೆ ಮತ್ತು ಕ್ರಿಯೆಗೆ ತಕ್ಕ ಪ್ರತಿಫಲ ಇವು ಯಾವುದನ್ನೂ ಆತ್ಮನು ನಿರ್ಮಿಸಲಾರ. ಎಲ್ಲವೂ ಸ್ವಭಾವಕ್ಕೆ ತಕ್ಕಂತೆ ಭಗವಂತನ ಇಚ್ಛೆಯಂತೆ ನಡೆದಿದೆ.}
\slcol{\Index{ನಾದತ್ತೇ ಕಸ್ಯಚಿತ್ಪಾಪಂ} ನ ಚೈವ ಸುಕೃತಂ ವಿಭುಃ ।\\
ಅಜ್ಞಾನೇನಾವೃತಂ ಜ್ಞಾನಂ ತೇನ ಮುಹ್ಯಂತಿ ಜಂತವಃ ॥ ೧೫ ॥}
\cquote{ಎಲ್ಲೆಲ್ಲಿಯೂ ತುಂಬಿರುವ ಭಗವಂತ ಯಾರ ಪಾಪ ಪುಣ್ಯಗಳಿಗೂ ತಾನು ಭಾಗಿಯಲ್ಲ. ಅರಿವನ್ನು ಅಜ್ಞಾನದ ತೆರೆ ಮುಚ್ಚಿದೆ. ಅದರಿಂದ ಜನ ಭ್ರಮಿಸುತ್ತಾರೆ, ಅಷ್ಟೇ.}
\slcol{\Index{ಜ್ಞಾನೇನ ತು ತದಜ್ಞಾನಂ} ಯೇಷಾಂ ನಾಶಿತಮಾತ್ಮನಃ ।\\
ತೇಷಾಮಾದಿತ್ಯವಜ್ಞಾನಂ ಪ್ರಕಾಶಯತಿ ತತ್ಪರಮ್ ॥ ೧೬ ॥}
\cquote{ಆತ್ಮದ ಅರಿವಿನ ಬೆಳಕಿನಿಂದ ಅಜ್ಞಾನದ ಕತ್ತಲನ್ನು ಕಳೆದುಕೊಂಡವರಿಗೆ ಅವರ ಆ ಅರಿವೇ ಸೂರ್ಯನಂತೆ ಆ ಪರತತ್ವವನ್ನು ಬೆಳಗಿಸುತ್ತದೆ.}
\slcol{\Index{ತದ್ಬುದ್ಧಯಸ್ತದಾತ್ಮಾನಸ್ತ}ನ್ನಿಷ್ಠಾಸ್ತತ್ಪರಾಯಣಾಃ ।\\
ಗಚ್ಛಂತ್ಯಪುನರಾವೃತ್ತಿಂ ಜ್ಞಾನನಿರ್ಧೂತಕಲ್ಮಷಾಃ ॥ ೧೭ ॥}
\cquote{ಭಗವಂತನಲ್ಲಿ ಬುದ್ಧಿ ನೆಲೆಗೊಳ್ಳಬೇಕು, ಬಾಳು ಅವನಿಗೇ ಅರ್ಪಿತವಾಗಬೇಕು.ಅವನೇ ಜೀವನದ ಗುರಿ ನೆಲೆಯಾಗಬೇಕು. ಇಂಥಹ ಅರಿವಿನಿಂದ ಪಾಪದ ಕೊಳೆಯನ್ನು ತೊಳೆದುಕೊಂಡವರು ಮತ್ತೆ ಮರಳದ ಶಾಶ್ವತ ಪದವನ್ನು ಪಡೆಯುವರು.}


\begin{mananam}{\mananamfont ಮನನ ಶ್ಲೋಕ - ೧೩}
\small \mananamtext ಕಣ್ಣು ಮತ್ತು ಕಿವಿಗಳಂತಹ ಇತರ ದೇಹದ ದ್ವಾರಗಳು, ಹೊರಗಿನ ಪ್ರಪಂಚದೊಂದಿಗೆ ಮಾಧ್ಯಮಗಳಾಗಿ ಹೇಗೆ ಸಂವಹನ ನಡೆಸುತ್ತವೆ ಎಂದು ನನಗೆ ತಿಳಿದಿದೆಯೇ? ಮೌಖಿಕವಾಗಿ ಹಾಗೂ ದೈಹಿಕವಾಗಿ ಈ ದ್ವಾರಗಳ ಮೂಲಕ ಒಳ ಬರುವ ಮತ್ತು ಹೊರ ಹೋಗುವ ಸ್ಪಂದನಗಳು ಅಥವಾ ಪ್ರತಿಕ್ರಿಯೆಗಳು, ನನ್ನ ಮನಸ್ಸಿನ ಮೇಲೆ ಪ್ರಭಾವ ಉಂಟುಮಾಡುತ್ತವೆ ಎಂದು ನನಗೆ ತಿಳಿದಿದೆಯೇ? ನನ್ನ ‘ಗಮನ’ ಮತ್ತು ‘ಚೈತನ್ಯ’ ಈ ದ್ವಾರಗಳ ಮೂಲಕ ಹೊರಹೋಗುವ ಮೊದಲು ಮತ್ತು ಗಮನವಿಟ್ಟ ವಸ್ತುಗಳೊಂದಿಗೆ ಸಂಪೂರ್ಣವಾಗಿ ಗುರುತಿಸಿಕೊಳ್ಳುವ ಮೊದಲು ಇರುವ ‘ಆಯ್ಕೆಯ ಕ್ಷಣ’ದ ಬಗ್ಗೆ ನನಗೆ ತಿಳಿದಿದೆಯೇ?
\end{mananam}
\WritingHand\enspace\textbf{ಆತ್ಮ ವಿಮರ್ಶೆ}\\
\begin{inspiration}{\mananamfont ಸ್ಫೂರ್ತಿ}
\small \mananamtext ನಮ್ಮೊಳಗಿನ ಆಂತರಿಕ ಅಸ್ತಿತ್ವವು ಉನ್ನತ ಆತ್ಮವಾಗಿದೆ. ಹೊರಗಿನ ಪ್ರಪಂಚದೊಂದಿಗಿನ ಪರಸ್ಪರ ಕ್ರಿಯೆಗಳು ನಮ್ಮ ನಿಜವಾದ ವ್ಯಕ್ತಿತ್ವವನ್ನು ಕಳೆದುಕೊಳ್ಳುವಂತೆ ಮಾಡುತ್ತದೆ, ಇತರರೊಂದಿಗಿನ ನಮ್ಮ ಸಂವಹನವು ವಿವಿಧ ಸವಾಲುಗಳಿಗೆ ಕಾರಣವಾಗುತ್ತದೆ;ಈ ಅಡಚಣೆಯನ್ನು ದಾಟಲು ನಮ್ಮ ನಿತ್ಯ  ಚಟುವಟಿಕೆಗಳಲ್ಲಿ ‘ಆತ್ಮಪ್ರಜ್ಞೆ’ಯನ್ನು ಬೆಳೆಸಿಕೊಳ್ಳಬೇಕು ಅಥವಾ, ಪ್ರಜ್ಞಾಪೂರ್ವಕವಾಗಿ   ಸಾಕ್ಷಿಯಾಗಿರಬೇಕು; ಹೀಗೆ ಸಣ್ಣ ಸಣ್ಣ ಹಂತಗಳೊಂದಿಗೆ ಪ್ರಾರಂಭಿಸಿ: ಉದಾಹರಣೆಗಾಗಿ ಹೇಳುವುದಾದರೆ: ಒಂದು ಹೂವಿನ ಮೇಲೆ ಗಮನ ಇಡುವಿಕೆ, ಆಕಾಶದಲ್ಲಿ ಪಕ್ಷಿಯ ಹಾರಾಡುವಿಕೆಯ ಮೇಲೆ ಗಮನ, ದೂರದಿಂದ ಕೇಳಿ ಬರುತ್ತಿರುವ ಶಬ್ದಗಳನ್ನು ಆಲಿಸುವುದು,ನಿಮ್ಮ ಮನೆಯಲ್ಲಿ ಅಥವಾ ಮನೆಯ ಮಾಳಿಗೆಯಲ್ಲಿ ನಡೆದಾಡುತ್ತಾ ಅದರ ಮೇಲೆ ಗಮನ ಮತ್ತು ನಿಮ್ಮದೇ ಉಸಿರಾಟವನ್ನು ಗಮನಿಸುವುದು; ಹೀಗೆ ಇವೆಲ್ಲಾ ಕ್ಷಣಗಳಲ್ಲಿಯೂ, ಆಂತರಿಕ ಅಸ್ತಿತ್ವದ ಅಭ್ಯಾಸ ಅಥವಾ ಉನ್ನತ ಆತ್ಮದ ಕಡೆಗೆ ಭಾಗಷ: ಪ್ರಜ್ಞೆಯನ್ನು ನಿರ್ವಹಿಸುವ ಅಭ್ಯಾಸ ಮಾಡಬೇಕು.
\end{inspiration}
\newpage

\begin{mananam}{\mananamfont {ಮನನ ಶ್ಲೋಕ - ೧೬, ೧೭}}
\small \mananamtext ನಮ್ಮ ಸನಾತನ ಗ್ರಂಥಗಳು ನಾವು ಅಜ್ಞಾನಿಗಳೆಂದು ಏಕೆ ಹೇಳುತ್ತವೆ? ನನ್ನ ಶೈಕ್ಷಣಿಕ ವಿದ್ಯಾರ್ಹತೆ ಮತ್ತು ಸುಸಂಸ್ಕೃತ ಪಾಲನೆಯ ಹೊರತಾಗಿಯೂ, ಜ್ಞಾನದ ಯಾವ ಕ್ಷೇತ್ರಗಳು ನನ್ನಿಂದ ತಪ್ಪಿಸಿಕೊಳ್ಳುತ್ತಿವೆ? ನನಗೆ ಪ್ರಾಪಂಚಿಕ ಜ್ಞಾನ ಮತ್ತು ಆಧ್ಯಾತ್ಮಿಕ ವಿವೇಕಗಳ  ನಡುವೆ ವ್ಯತ್ಯಾಸವನ್ನು ತಿಳಿಯುವ ವಿವೇಚನಾಶೀಲ ಸಾಮರ್ಥ್ಯವಿದೆಯೇ? ಯಾವ ಶ್ರೇಷ್ಠವಾದ  ಜ್ಞಾನವು ಪರಮ ಸತ್ಯದ ಕಡೆಗೆ ನನ್ನನ್ನು ಕರೆದೊಯ್ಯಬಹುದು? ನನ್ನ ಮನಸ್ಸನ್ನು ಆ ಅಂತಿಮ ಸತ್ಯದೊಂದಿಗೆ ಒಗ್ಗೂಡಿಸಲು ಯಾವ ಅಭ್ಯಾಸಗಳು ಸಹಕಾರಿಯಾಗಿವೆ? ಈ ಅಭ್ಯಾಸಗಳು ನನ್ನ ನಿತ್ಯ ಜೀವನದಲ್ಲಿ ಯಾವ ರೀತಿಯ ಪರಿವರ್ತನೆ ತರಬಹುದು?
\end{mananam}
\WritingHand\enspace\textbf{ಆತ್ಮ ವಿಮರ್ಶೆ}\\
\begin{inspiration}{\mananamfont ಸ್ಫೂರ್ತಿ}
\small \mananamtext ಎಲ್ಲಾ ಆಧ್ಯಾತ್ಮಿಕ ಮಾರ್ಗಗಳ ಪರಾಕಾಷ್ಠೆಯು ಅಜ್ಞಾನದ ನಿರ್ಮೂಲನೆಯೇ ಆಗಿದೆ. ಕತ್ತಲೆಯು ಹೇಗೆ ಬೆಳಕಿಗೆ  ಸೋಲುತ್ತದೆಯೋ ಹಾಗೆಯೇ, ಜ್ಞಾನಮಾತ್ರವೇ ಅಜ್ಞಾನವನ್ನು ಹೋಗಲಾಡಿಸಬಹುದು. ಅಜ್ಞಾನದ ಒಂದು ಕಪಟ  ಪರಿಣಾಮವೆಂದರೆ ಅದು ನಮ್ಮ ನಿಜವಾದ ಅಸ್ತಿತ್ವವನ್ನು ಮುಚ್ಚಿಡುತ್ತದೆ; ಇದರ ಪರಿಣಾಮವಾಗಿ ನಮ್ಮ ಎಲ್ಲಾ ಕ್ರಿಯೆಗಳೂ ಅಜ್ಞಾನದಿಂದ ಹುಟ್ಟುತ್ತವೆ. ನಮ್ಮ ಬುದ್ಧಿಶಕ್ತಿಯನ್ನು ನಿರಂತರ ಪ್ರಯತ್ನದಿಂದ ಅತ್ಯುನ್ನತವಾದ ವಿವೇಕದಲ್ಲಿ ಸ್ಥಾಪಿಸಿದಾಗ ಅದು, ಎಲ್ಲಾ ಪರಿಮಿತಿಗಳನ್ನೂ ಮುರಿಯುತ್ತದೆ. 
\end{inspiration}
\newpage

\slcol{\Index{ವಿದ್ಯಾವಿನಯಸಂಪನ್ನೇ} ಬ್ರಾಹ್ಮಣೇ ಗವಿ ಹಸ್ತಿನಿ ।\\
ಶುನಿ ಚೈವ ಶ್ವಪಾಕೇ ಚ ಪಂಡಿತಾಃ ಸಮದರ್ಶಿನಃ ॥ ೧೮ ॥}
\cquote{ವಿದ್ಯೆ ವಿನಯಗಳ ನೆಲೆಯಾದ ಬ್ರಹ್ಮಜ್ಞಾನಿ, ಹಸು, ಆನೆ, ನಾಯಿ, ನಾಯಮಾಂಸ ತಿಂದು ಬದುಕುವ ಅನಾಗರಿಕ ಈ ಎಲ್ಲರಲ್ಲಿಯೂ ಒಬ್ಬನೇ ಭಗವಂತ ನೆಲೆಸಿದ್ದಾನೆ.ಅವನಿಗೆ ಕೀಳು ಮೇಲೆಂಬುದಿಲ್ಲ ಎಂದು ಜ್ಞಾನಿಗಳು ತಿಳಿಯುತ್ತಾರೆ.}
\slcol{\Index{ಇಹೈವ ತೈರ್ಜಿತಃ ಸರ್ಗೋ} ಯೇಷಾಂ ಸಾಮ್ಯೇ ಸ್ಥಿತಂ ಮನಃ ।\\
ನಿರ್ದೋಷಂ ಹಿ ಸಮಂ ಬ್ರಹ್ಮ ತಸ್ಮಾದ್ಬ್ರಹ್ಮಣಿ ತೇ ಸ್ಥಿತಾಃ ॥ ೧೯ ॥}
\cquote{ನಿರ್ದೋಷವಾದ ಬ್ರಹ್ಮವು ಸಮವಾಗಿದೆ. ಅವರು ಆ ಬ್ರಹ್ಮದಲ್ಲಿಯೇ ಇರುತ್ತಾರೆ. ಸಾಮ್ಯ ಬುದ್ಧಿಯುಳ್ಳ ದೊಡ್ಡವರು ಬದುಕಿರುವಾಗಲೇ ಮುಕ್ತರಾಗುತ್ತಾರೆ.}
\slcol{\Index{ನ ಪ್ರಹೃಷ್ಯೇತ್ಪ್ರಿಯಂ ಪ್ರಾಪ್ಯ} ನೋದ್ವಿಜೇತ್ಪ್ರಾಪ್ಯ ಚಾಪ್ರಿಯಮ್ ।\\
ಸ್ಥಿರಬುದ್ಧಿರಸಂಮೂಢೋ ಬ್ರಹ್ಮವಿದ್ಬ್ರಹ್ಮಣಿ ಸ್ಥಿತಃ ॥ ೨೦ ॥}
\cquote{ಸ್ಥಿರಬುದ್ದಿಯುಳ್ಳವನೂ ಮೋಹರಹಿತನೂ ಆದ ಬ್ರಹ್ಮ ಜ್ಞಾನಿಯು ಬ್ರಹ್ಮದಲ್ಲಿಯೇ ಸ್ಥಿರವಾಗಿರುವುದರಿಂದ ಅವನಿಗೆ ಪ್ರಿಯ ವಸ್ತುವಿನಲ್ಲಿ ಇಚ್ಛೆ ಇಲ್ಲ, ಅಪ್ರಿಯವಾದ ವಸ್ತುವಿನಲ್ಲಿ ದ್ವೇಷವೂ ಇರುವುದಿಲ್ಲ.}
\slcol{\Index{ಬಾಹ್ಯಸ್ಪರ್ಶೇಷ್ವಸಕ್ತಾತ್ಮಾ} ವಿಂದತ್ಯಾತ್ಮನಿ ಯತ್ಸುಖಮ್ ।\\
ಸ ಬ್ರಹ್ಮಯೋಗಯುಕ್ತಾತ್ಮಾ ಸುಖಮಕ್ಷಯಮಶ್ನುತೇ ॥ ೨೧ ॥}
\cquote{ಹೊರಗಣ ವಿಷಯಗಳಿಗೆ ಮನಸೋಲದವನು ತನ್ನೊಳಗೆಯೆ ಯಾವ ಸುಖವನ್ನು ಪಡೆಯುವನೋ ಅದೇ ಸುಖವನ್ನು ಬ್ರಹ್ಮತತ್ವದಲ್ಲಿ ಮನಸ್ಸನ್ನು ನೆಲೆಗೊಳಿಸಿದವನು ಅನಂತವಾಗಿ ಪಡೆಯುತ್ತಾನೆ.}
\slcol{\Index{ಯೇ ಹಿ ಸಂಸ್ಪರ್ಶಜಾ ಭೋಗಾ} ದುಃಖಯೋನಯ ಏವ ತೇ ।\\
ಆದ್ಯಂತವಂತಃ ಕೌಂತೇಯ ನ ತೇಷು ರಮತೇ ಬುಧಃ ॥ ೨೨ ॥}
\cquote{ಕುಂತೀಕುಮಾರ, ವಿಷಯಗಳ ಸಂಬಂಧದಿಂದಾಗುವ ಸುಖಗಳೆಲ್ಲ ದುಃಖಕ್ಕೆ ಮೂಲವಾದುವುಗಳೇ. ಅರ್ಜುನ, ಅವು ಮೊದಲೂ ಕೊನೆಯೂ ಉಳ್ಳವುಗಳು. ವಿವೇಕಿಯು ಅವುಗಳಲ್ಲಿ ಪ್ರೀತಿಯನ್ನಿಡುವುದಿಲ್ಲ.}

\newpage
\begin{mananam}{\mananamfont ಮನನ ಶ್ಲೋಕ - ೧೮}
\small \mananamtext ನಾನು ಯಾವ ಮಾನದಂಡದ ಆಧಾರದ ಮೇಲೆ ಸಮಾಜದ ಜನರ ಮೌಲ್ಯ ಮಾಪನ ಮಾಡಲಿ – ಅವರ ಸ್ಥಾನಮಾನದಿಂದಲೇ, ಅವರ ಶ್ರೀಮಂತಿಕೆಯಿಂದಲೇ, ಅವರ ವಿದ್ಯಾರ್ಹತೆಯಿಂದಲೇ, ಅವರ ಸೌಂದರ್ಯದಿಂದಲೇ ಅಥವಾ ಅವರ ಸಾಧನೆಗಳಿಂದಲೇ? ಯಾರಿಗೆ ಹೆಚ್ಚು ಗೌರವ ಅಥವಾ ಯಾರಿಗೆ ಕಡಿಮೆ ಗೌರವ ತೋರಿಸಲಿ? ನಾನು ಕೆಲವರನ್ನು  ಕಡಿಮೆ ಆದರದಿಂದ ಕಾಣುತ್ತೇನೆಯೇ? ಅಗತ್ಯವಿರುವವರಿಗೆ ನನ್ನ ಲಕ್ಷ್ಯ ಮತ್ತು ಕಾಳಜಿಯನ್ನು ನೀಡುವ ಮೂಲಕ ನಾನು ಹೇಗೆ ಪ್ರಯೋಜನ ಪಡೆಯಬಹುದು?ಯಾರ ಮೌಲ್ಯಗಳು ಮತ್ತು ಜೀವನದ ಅನುಭವಗಳು ನನ್ನ ಬೌದ್ಧಿಕ ಮತ್ತು ಆಧ್ಯಾತ್ಮಿಕ ಬೆಳವಣಿಗೆಯನ್ನು ಉನ್ನತಿಯತ್ತ ಕೊಂಡೊಯ್ಯುತ್ತದೆಯೋ ಅಂಥಹ, ಅನುಸರಣೀಯವಾದ ಉತ್ತಮ ಮಾದರಿಗಳು ನನಗೆ ಯಾರಿರುವರು?
\end{mananam}
\WritingHand\enspace\textbf{ಆತ್ಮ ವಿಮರ್ಶೆ}\\
\begin{inspiration}{\mananamfont ಸ್ಫೂರ್ತಿ}
\small \mananamtext ಒಬ್ಬ ಉತ್ಕೃಷ್ಟ ಮಾನವನ ಪ್ರತಿಕ್ರಿಯೆಯು ಎಲ್ಲರ ಕಡೆಗೆ ಸಮಚಿತ್ತವಾಗಿರುತ್ತದೆ, ಅಂತಹವನು ಒಬ್ಬ ಶ್ರೀಮಂತ, ಶಕ್ತಿಯುತ ವ್ಯಕ್ತಿಯ ಗಮನ ತನ್ನ ಕಡೆಗೆ ಇರುವುದೆಂದು ವಿಸ್ಮಯಗೊಳ್ಳುವುದಿಲ್ಲ ಅಥವಾ ಕಡಿಮೆ ಸಾಮಾಜಿಕ ಸ್ಥಾನಮಾನವನ್ನು ಹೊಂದಿರುವ ವ್ಯಕ್ತಿಯನ್ನು ತಿರಸ್ಕಾರದ ನೋಟದಿಂದ ನೋಡುವುದೂ ಇಲ್ಲ. ಸಮಭಾವವನ್ನು ಉಪೇಕ್ಷೆ ಎಂದು ತಪ್ಪು ಭಾವಿಸಬಾರದು ಬದಲಿಗೆ, ನಮ್ಮ ಗಮನದ ಅವಶ್ಯಕತೆ ಯಾರಿಗಿದೆ ಮತ್ತು ಯಾರು ಗೌರವಕ್ಕೆ ಯೋಗ್ಯರು ಎಂಬುದನ್ನು ಗುರುತಿಸಲು ಕಲಿಯಬೇಕು. ನಮ್ಮ ಮೌಲ್ಯವ್ಯವಸ್ಥೆಯನ್ನು ರೂಪಿಸುವ ಅಭ್ಯಾಸ ಮಾಡುವ ಸಲುವಾಗಿ, ಭೌತಿಕವಾಗಿ ಶ್ರೀಮಂತ ಹಾಗೂ ಶಕ್ತಿಶಾಲಿಗಳಿಗೆ ಮನ್ನಣೆ ನೀಡುವುದರಲ್ಲಿ ಕೇಂದ್ರೀಕೃತವಾಗಿರುವ ನಮ್ಮ ಪ್ರವೃತ್ತಿಯನ್ನು ತಿರುಗಿಸಿ, ಆಧ್ಯಾತ್ಮಿಕವಾಗಿ ವಿವೇಕಯುಕ್ತರಾದವರಿಗೆ ಗೌರವ ನೀಡುವ ಆದ್ಯತೆ ಬೆಳೆಸಿಕೊಳ್ಳಬೇಕು. ಇಂತಹ ಮೌಲ್ಯಯುತ ಪದ್ಧತಿಯು ಆಧ್ಯಾತ್ಮಿಕವಾಗಿ ಮುಂದುವರಿದ ಸಮಾಜದ ಗುರುತಾಗಿದೆ.
\end{inspiration}
\newpage

\newpage
\begin{mananam}{\mananamfont ಮನನ ಶ್ಲೋಕ - ೨೦}
\small \mananamtext ಅಹ್ಲಾದಕರವಾದ ಅನುಭವವಾದಾಗ ನಾನು ಹೇಗೆ  ಸ್ಪಂದಿಸುತ್ತೇನೆ?  ಅಹಿತಕರ ಸಂದರ್ಭಗಳನ್ನು  ನಾನು ಹೇಗೆ ನಿಭಾಯಿಸುತ್ತೇನೆ?ಸಂತೋಷವನ್ನು ಹುಡುಕುವ ಮತ್ತು ನೋವನ್ನು ತಪ್ಪಿಸುವ ಮನೋಸ್ಥಿತಿಯಿಂದ ಮೇಲೇರುವುದು, ನನ್ನ ಮಾನಸಿಕ ಯೋಗಕ್ಷೇಮಕ್ಕೆ ಪ್ರಯೋಜನಕಾರಿ ಎಂದು ನನಗೆ ಮನವರಿಕೆಯಾಗಿದೆಯೇ? ಹಾಗಿದ್ದರೆ, ಆ ಸ್ಥಿತಿಯನ್ನು ತಲುಪುವುದು ಹೇಗೆ? ನನಗೆ, ಜೀವನದ ಎಲ್ಲಾ ಮಜಲುಗಳಲ್ಲಿ, ಶೀಘ್ರವಾಗಿ ಲಭಿಸುವ ಸಂತೋಷ ಮತ್ತು ದೀರ್ಘಕಾಲದ ಒಳಿತುಗಳ ಆಯ್ಕೆಯ ವಿಚಾರದಲ್ಲಿ ತಿಳುವಳಿಕೆ ಇದೆಯೇ? ನಾನು ವಿಶೇಷವಾಗಿ ಮಹತ್ವಪೂರ್ಣ ಪ್ರಯತ್ನಪಟ್ಟು ಮಾಡಿದ ಕೆಲಸದ  ಫಲಿತಾಂಶವನ್ನು ಅಥವಾ ಅದರ ಪರಿಣಾಮವನ್ನು ಪ್ರಶಾಂತ ಭಾವದಿಂದ ಸ್ವೀಕರಿಸುವುದರ ಬೆಲೆ ಏನು?
\end{mananam}
\WritingHand\enspace\textbf{ಆತ್ಮ ವಿಮರ್ಶೆ}\\
\begin{inspiration}{\mananamfont ಸ್ಫೂರ್ತಿ}
\small \mananamtext ಮನುಷ್ಯನಲ್ಲಿ ಆಗಾಗ್ಗೆ ಕಂಡುಬರುವ ‘ಸಹಜವಾದ ಮಾನಸಿಕ ಪ್ರಚೋದನೆ’ಗಳನ್ನು ಋಷಿಗಳು ಮತ್ತು ಯೋಗಿಗಳು ಅಪಾಯಕಾರಿಯಾದದ್ದು ಎಂದು  ಪರಿಗಣಿಸಿದ್ದಾರೆ.  ಬಹುತೇಕ ಎಲ್ಲರೂ ಆಹ್ಲಾದಕರವಾದುದನ್ನು ಹುಡುಕಲು ಮತ್ತು ಅಹಿತಕರವಾದುದನ್ನು ತಪ್ಪಿಸಲು ಧಾವಿಸುತ್ತಾರೆ. ಆದಾಗ್ಯೂ, ಈ ‘ಪ್ರಿಯ - ಅಪ್ರಿಯ’  ವಿಚಾರಧಾರೆಯೇ ಆಂತರ್ಯದ ಅಸ್ಥಿರತೆಗೆ ನಿಖರವಾಗಿ ಕಾರಣವಾಗುತ್ತದೆ. ಆಹ್ಲಾದಕರ ಮತ್ತು ಅಹಿತಕರ ಅನುಭವಗಳೆರಡನ್ನೂ ಸ್ವೀಕರಿಸುವ ಮೂಲಕ ಮಾನಸಿಕ ಸಮಚಿತ್ತತೆಯನ್ನು ಬೆಳೆಸಿಕೊಳ್ಳುವುದು ಹೆಚ್ಚು ಪ್ರಯೋಜನಕಾರಿ ಅಭ್ಯಾಸವಾಗಿದೆ. 
\end{inspiration}
\newpage

\begin{mananam}{\mananamfont {ಮನನ ಶ್ಲೋಕ - ೨೧, ೨೨}}
\small \mananamtext ನನ್ನ ಇಂದ್ರಿಯಗಳು ಮತ್ತು ಇಂದ್ರಿಯಗಳಿಗೆ ಗೋಚರವಾಗುವ ವಸ್ತುಗಳ ನಡುವಿನ ಪರಸ್ಪರ ಕ್ರಿಯೆಯ ಬಗ್ಗೆ ನಾನು ಜಾಗೃತನಾಗಿದ್ದೇನೆಯೇ? ಉದಾಹರಣೆಗೆ, ನನ್ನ ಕಣ್ಣುಗಳು ಆಕರ್ಷಕವಾದದ್ದನ್ನು ಗ್ರಹಿಸಿದಾಗ, ಅದು ಪ್ರಚೋದಿಸುವ ಮಾನಸಿಕ ಪ್ರತಿಕ್ರಿಯೆಗಳು – ಅಂದರೆ, ಆ ವಸ್ತುವನ್ನು ಸ್ವಾಧೀನಪಡಿಸಿಕೊಳ್ಳುವ ಅಥವಾ ಹೊಂದುವ ಬಯಕೆಯ ಬಗ್ಗೆ ನನಗೆ ತಿಳಿದಿದೆಯೇ? ಬಾಹ್ಯ ವಸ್ತುಗಳಾದ ಆಹಾರ, ಮನೋರಂಜನೆ ಅಥವಾ ಪರಸ್ಪರ ಸಾಮಾಜಿಕ ಸಂಪರ್ಕ ನಿಜವಾಗಿಯೂ ಶಾಶ್ವತ ಸಂತೋಷ ನೀಡುತ್ತದೆಯೇ? ಅಥವಾ ಅವುಗಳು ಕೇವಲ ಅಲ್ಪಕಾಲದ ತೃಪ್ತಿ ಕೊಡುವ ಸಾಮರ್ಥ್ಯ ಹೊಂದಿರುವವೇ? ಕೇವಲ ಬಾಹ್ಯ ಮೂಲಗಳಲ್ಲೇ ಸಂತೋಷವನ್ನು ಅರಸಿದಾಗ, ಅದರಲ್ಲಿ ಅಶಾಶ್ವತೆಯು ಅಂತರ್ಗತವಾಗಿದೆಯೆಂಬ ಅರಿವು ನನಗಿದೆಯೇ?
\end{mananam}
\WritingHand\enspace\textbf{ಆತ್ಮ ವಿಮರ್ಶೆ}\\
\begin{inspiration}{\mananamfont ಸ್ಫೂರ್ತಿ}
\small \mananamtext ನಾವು ಆಗಾಗ್ಗೆ ಅಶಾಶ್ವತವಾದ ಬಾಹ್ಯ ವಸ್ತುಗಳಲ್ಲಿ ಸಂತೋಷವನ್ನು ಅರಸುತ್ತೇವೆ; ಪರಿಣಾಮವಾಗಿ, ಆ ವಸ್ತುಗಳು ಅದೃಶ್ಯವಾದಂತೆ, ಅದರೊಟ್ಟಿಗೆ ಸಂತೋಷವೂ ಕೂಡ ಕ್ಷಣಿಕವಾಗುತ್ತದೆ ಹಾಗೂ ಸಂಕಟಕ್ಕೆ ಕಾರಣವಾಗುತ್ತದೆ. ನಮ್ಮ ಒಳಗಿನ ಯೋಚನೆಗಳು ಮತ್ತು ಭಾವನೆಗಳು ಬಾಹ್ಯ  ಪ್ರಪಂಚದೊಂದಿಗೆ ನಿಕಟವಾಗಿ ಬಂಧಿಸಲ್ಪಟ್ಟಿವೆ. ಸಂತೋಷ ಮತ್ತು ದುಃಖಗಳನ್ನು ನಮ್ಮ ಒಳ ಮನಸ್ಸಿನಿಂದ ಅನುಭವಿಸುತ್ತೇವೆ; ಹೀಗೆ ಈ ವಿಚಾರಗಳ ಸ್ಪಷ್ಟತೆಯು, ನಮಗೆ ಅಂತರ್ ಪ್ರಪಂಚದಿಂದ ಬಾಹ್ಯ ಪ್ರಪಂಚವನ್ನು ಬೇರ್ಪಡಿಸುವ ಶಕ್ತಿಯನ್ನು  ಕೊಡುತ್ತದೆ.  ಬಾಹ್ಯ ಸನ್ನಿವೇಶಗಳ ಹೊರತಾಗಿಯೂ (ವಿಪರೀತವಾಗಿದ್ದಾಗಲೂ ಸಹಿತ), ಬೇಕೆಂದಾಗ ಸಂತೋಷವಾಗಿ ಇರಲು ಕಲಿಯುವುದು ಆಧ್ಯಾತ್ಮಿಕ ದಾರಿಯಲ್ಲಿ ಅತ್ಯಾವಶ್ಯಕವಾದ ಜಾಣ್ಮೆಯಾಗಿದೆ.
\end{inspiration}
\newpage

\slcol{\Index{ಶಕ್ನೋತೀಹೈವ ಯಃ ಸೋಢುಂ} ಪ್ರಾಕ್ಶರೀರವಿಮೋಕ್ಷಣಾತ್ ।\\
ಕಾಮಕ್ರೋಧೋದ್ಭವಂ ವೇಗಂ ಸ ಯುಕ್ತಃ ಸ ಸುಖೀ ನರಃ ॥ ೨೩ ॥}
\cquote{ಬದುಕಿ ದೇಹದೊಡನೆ ಇರುವಾಗಲೇ ಕಾಮಕ್ರೋಧಗಳ ವೇಗವನ್ನು ಸಹಿಸಬಲ್ಲವನೇ ಯೋಗಿ, ಅವನೇ ಸುಖವಂತನು}
\slcol{\Index{ಯೋಽಂತಃಸುಖೋಽಂತರಾರಾಮಸ್ತ}ಥಾಂತರ್ಜ್ಯೋತಿರೇವ ಯಃ ।\\
ಸ ಯೋಗೀ  ಬ್ರಹ್ಮನಿರ್ವಾಣಂ ಬ್ರಹ್ಮಭೂತೋಧಿಗಚ್ಛತಿ ॥ ೨೪ ॥}
\cquote{ಆತ್ಮಸುಖದಲ್ಲಿ ಲೀನನಾಗಿ ಭಗವಂತನ ದರ್ಶನದಿಂದ ಆನಂದಗೊಳ್ಳುತ್ತ. ಒಳಗೆ ಆ ಬೆಳಕನ್ನೇ ತುಂಬಿಕೊಂಡಿರುವನೋ ಬ್ರಹ್ಮನಲ್ಲಿಯೇ ನೆಲೆಗೊಂಡ ಆ ಯೋಗಿಯು ಆನಂದರೂಪಿಯಾದ ಬ್ರಹ್ಮನನ್ನೇ ಪಡೆಯುತ್ತಾನೆ.}
\slcol{\Index{ಲಭಂತೇ ಬ್ರಹ್ಮನಿರ್ವಾಣ}ಮೃಷಯಃ ಕ್ಷೀಣಕಲ್ಮಷಾಃ ।\\
ಛಿನ್ನದ್ವೈಧಾ ಯತಾತ್ಮಾನಃ ಸರ್ವಭೂತಹಿತೇ ರತಾಃ ॥ ೨೫ ॥}
\cquote{ಪಾಪರಹಿತರು, ಸಂಶಯ ಶೂನ್ಯರು, ಜಿತೇಂದ್ರಿಯರು, ಸರ್ವಭೂತಗಳ ಹಿತದಲ್ಲಿ ನಿರತರಾದ ಋಷಿಗಳು ಪರಮಮುಕ್ತಿಯನ್ನು ಪ್ರಾಪ್ತಿ ಮಾಡಿಕೊಳ್ಳುತ್ತಾರೆ.}
\slcol{\Index{ಕಾಮಕ್ರೋಧವಿಯುಕ್ತಾನಾಂ} ಯತೀನಾಂ ಯತಚೇತಸಾಮ್ ।\\
ಅಭಿತೋ ಬ್ರಹ್ಮನಿರ್ವಾಣಂ ವರ್ತತೇ ವಿದಿತಾತ್ಮನಾಮ್ ॥ ೨೬ ॥}
\cquote{ಬಯಕೆ ಸಿಟ್ಟುಗಳನ್ನು ತೊರೆದು ಮನಸ್ಸನ್ನು ಬಿಗಿಹಿಡಿದು ಆತ್ಮವನ್ನರಿತ ಸಂನ್ಯಾಸಿಗಳಿಗೆ ಎಲ್ಲೆಡೆಯೂ ಆನಂದ ರೂಪವಾದ ಬ್ರಹ್ಮವೇ ತುಂಬಿದೆ.}
\slcol{\Index{ಸ್ಪರ್ಶಾನ್ಕೃತ್ವಾ ಬಹಿರ್ಬಾಹ್ಯಾಂ}ಶ್ಚಕ್ಷುಶ್ಚೈವಾಂತರೇ ಭ್ರುವೋಃ ।\\
ಪ್ರಾಣಾಪಾನೌ ಸಮೌ ಕೃತ್ವಾ ನಾಸಾಭ್ಯಂತರಚಾರಿಣೌ ॥ ೨೭ ॥\\
\Index{ಯತೇಂದ್ರಿಯಮನೋಬುದ್ಧಿ}ರ್ಮುನಿರ್ಮೋಕ್ಷಪರಾಯಣಃ ।\\
ವಿಗತೇಚ್ಛಾಭಯಕ್ರೋಧೋ ಯಃ ಸದಾ ಮುಕ್ತ ಏವ ಸಃ ॥ ೨೮ ॥}
\cquote{ಹೊರಗಣ ವಿಷಯಗಳನ್ನು ಬಹಿಷ್ಕರಿಸಿ ಕಣ್ಣನ್ನು ಹುಬ್ಬುಗಳ ನಡುವೆ ನೆಲೆಗೊಳಿಸಿ ಮೂಗಿನೊಳಗೆ ಓಡಾಡುವ ಉಸಿರನ್ನು ಕುಂಭಕದಲ್ಲಿ ಬಿಗಿಹಿಡಿದು, ಇಂದ್ರಿಯ, ಮನಸ್ಸು, ಬುದ್ಧಿ ಇವುಗಳನ್ನು ಹಿಡಿತದಲ್ಲಿಟ್ಟುಕೊಂಡು, ಬಯಕೆ, ಭಯ, ಕೋಪಗಳನ್ನು ತೊರೆದು ಭವದ ಬಿಡುಗಡೆಯನ್ನೇ ಬಯಸುವ ಮುನಿ ಯಾವಾಗಲೂ ಮುಕ್ತನೇ.}

\newpage
\begin{mananam}{\mananamfont ಮನನ ಶ್ಲೋಕ - ೨೬}
\small \mananamtext ನನ್ನ ಜೀವನದಲ್ಲಿ ಆಕಾಂಕ್ಷೆಗಳು ಮತ್ತು ಅದಕ್ಕೆ ಸಂಬಂಧಿಸಿದ ಭಾವನೆಗಳು ಯಾವ ಪಾತ್ರವಹಿಸುತ್ತವೆ? ನನ್ನ ದೀರ್ಘಕಾಲದ ಒಳಿತಿಗೆ ನನ್ನ ಆಕಾಂಕ್ಷೆಗಳು, ಆರೋಗ್ಯಕರ ಹಾಗೂ ರಚನಾತ್ಮಕವಾಗಿವೆಯೇ? ನನ್ನ ಜೀವನದಲ್ಲಿನ ಪ್ರೇರಣೆಗಳು ಕೇವಲ ಸ್ವಾರ್ಥದಲ್ಲಿ ಕೇಂದ್ರೀಕೃತವಾಗಿರುವ ಆಕಾಂಕ್ಷೆಗಳಿಂದ ಹೊರ ಹೊಮ್ಮಿವೆಯೇ?  ಇತರರ ಯೋಗಕ್ಷೇಮದ ಉದ್ದೇಶವನ್ನು ಆಧರಿಸಿ ಕಾರ್ಯನಿರ್ವಹಿಸಲು ನಾನು ಪ್ರೇರಣೆಯನ್ನು ಕಂಡುಕೊಳ್ಳಬಹುದೇ? ಬಲವಾದ ಬಾಂಧವ್ಯವು ಕೋಪಕ್ಕೆ ಕಾರಣವಾದ ಸಂದರ್ಭಗಳನ್ನು ನಾನು ನೆನಪಿಸಿಕೊಳ್ಳಬಹುದೇ? ಬಲವಾದ ಮೋಹದಿಂದಾಗಿ “ನನ್ನ ಆಕಾಂಕ್ಷೆಗಳಿಗೆ  ಅಡಚಣೆಯಾಯಿತು” ಎಂಬ ಗ್ರಹಿಕೆಯಿಂದ ಉಂಟಾದ ಕೋಪದ ಸಂದರ್ಭಗಳನ್ನು ಸ್ಮರಿಸಿಕೊಳ್ಳಬಲ್ಲೆನೇ? “ಯಾವುದೇ ಭಾವೋದ್ವೇಗಗಳು ಮತ್ತು ವ್ಯಾಮೋಹಗಳಿಲ್ಲದ ಋಷಿಗಳಂತಹ ಸ್ಥಿತಿಯನ್ನು ಸಾಧಿಸುವುದು” ಎಂದರೆ ನನಗೆ ಏನು ಅರ್ಥ ಕೊಡುತ್ತದೆ?
\end{mananam}
\WritingHand\enspace\textbf{ಆತ್ಮ ವಿಮರ್ಶೆ}\\
\begin{inspiration}{\mananamfont ಸ್ಫೂರ್ತಿ}
\small \mananamtext ಸಾಮಾನ್ಯ ಮನುಷ್ಯನ ದೃಷ್ಟಿಯಲ್ಲಿ ಸ್ವಾತಂತ್ರ್ಯದ ಪರಿಕಲ್ಪನೆ ಎಂದರೆ, ಎಲ್ಲಾ ತರಹದ ಅವಶ್ಯಕತೆಗಳು, ಆಕಾಂಕ್ಷೆಗಳು ಅವನ ಮನಸ್ಸಿನಲ್ಲಿ ಹುಟ್ಟಿದ ಕೂಡಲೇ   ನೆರವೇರಿಸುವುದೇ ಆಗಿದೆ. ಇದಕ್ಕೆ ವ್ಯತಿರಿಕ್ತವಾಗಿ, ಯೋಗಿಗಳು  ಇಂದ್ರಿಯ ತೃಪ್ತಿಗಾಗಿ ಅನಗತ್ಯ ಆಸೆಗಳಿಗೆ ಅಥವಾ ಅಹಂಕಾರದ ದೃಢೀಕರಣಕ್ಕೆ  (ಯಾವುದೇ ಅಹಂ ನಿರ್ಧಾರಿತ ಮೌಲ್ಯೀಕರಣಕ್ಕೆ) ಎಡೆಗೊಡದೆ, ಸದಾ  ಮಾನಸಿಕ ಅರಿವಿನ ಜಾಗರೂಕತೆಯ  ಮೂಲಕ ಸ್ವಾತಂತ್ರ್ಯವನ್ನು ಪಡೆಯುತ್ತಾರೆ. ಒಂದು ವೇಳೆ, ಯಾವುದೇ ಆಕಾಂಕ್ಷೆಗಳು  ಹುಟ್ಟಿಕೊಂಡರೂ ಸಹ, ಅವರು ಕೂಡಲೇ ಅವನ್ನು  ರೂಪಾಂತರಗೊಳಿಸಿ  ಇತರರ ಯೋಗಕ್ಷೇಮದ ಕಡೆಗೆ ಹರಿಬಿಡುತ್ತಾರೆ. ಮನಸ್ಸಿನಲ್ಲಿ ಒಂದು ಬಯಕೆಯು ಏಳುತ್ತಿದ್ದಂತೆಯೇ  ಅದು (ಬಯಕೆ) ದೇಹದೊಂದಿಗೆ ಗುರುತಿಸಿಕೊಂಡು ಪ್ರಾಪಂಚಿಕತೆಯತ್ತ ನಮ್ಮನ್ನು ಎಳೆದು ಬಂಧಿಸುತ್ತದೆ. ನಿಜವಾದ ಸ್ವಾತಂತ್ರ್ಯವೆಂದರೆ, ಆತ್ಮದ ಸಹಜ ಸ್ಥಿತಿಯಾದ  ಪರಮಾನಂದದಲ್ಲಿ   ನೆಲೆಸಿರುವುದು.
\end{inspiration}
\newpage

\slcol{\Index{ಭೋಕ್ತಾರಂ ಯಙ್ಞತಪಸಾಂ} ಸರ್ವಲೋಕಮಹೇಶ್ವರಮ್ ।\\
ಸುಹೃದಂ ಸರ್ವಭೂತಾನಾಂ ಜ್ಞಾತ್ವಾ ಮಾಂ ಶಾಂತಿಮೃಚ್ಛತಿ ॥ ೨೯ ॥}
\cquote{ನನ್ನನ್ನು ಯಜ್ಞದ ಮತ್ತು ತಪಸ್ಸಿನ ಫಲವನ್ನು ಉಣ್ಣುವವನೆಂದೂ ಎಲ್ಲಾ ಲೋಕಗಳಿಗೂ ಒಡೆಯನೆಂದು ಎಲ್ಲ ಪ್ರಾಣಿಗಳಿಗೂ ಗೆಳೆಯನೆಂದು ತಿಳಿದವನಿಗೆ ಮುಕ್ತಿ ಕೈಗಂಟು.}
\chapEndSloka{ಕರ್ಮಸಂನ್ಯಾಸಯೋಗ}