\centerline{\textbf{ಅಥ ದ್ವಿತೀಯೋऽಧ್ಯಾಯಃ ।}\\}
ಸಂಜಯ ಉವಾಚ ।\\
\slcol{\Index{ತಂ ತಥಾ ಕೃಪಯಾವಿಷ್ಟ}ಮಶ್ರುಪೂರ್ಣಾಕುಲೇಕ್ಷಣಮ್ ।\\
ವಿಷೀದಂತಮಿದಂ ವಾಕ್ಯಮುವಾಚ ಮಧುಸೂದನಃ ॥ 1 ॥}
\cquote{ಸಂಜಯನು ಹೇಳಿದನು,  ಧೃತರಾಷ್ಟ್ರ ಮಹಾರಾಜ ಕೇಳು. 
ಈ ಪ್ರಕಾರ ಕಣ್ಣಿನಲ್ಲಿ ನೀರು ತುಂಬಿ ದುಃಖಿಸುತ್ತಿರುವ  ಅರ್ಜುನನನ್ನು ನೋಡಿ ಮಧುಸೂದನನು ಕೃಪೆಯಿಂದ ಈ ವಿಧವಾಗಿ ಹೇಳಿದನು.\\}
\slcol{ಶ್ರೀಭಗವಾನುವಾಚ ।\\
\Index{ಕುತಸ್ತ್ವಾ ಕಶ್ಮಲಮಿದಂ} ವಿಷಮೇ ಸಮುಪಸ್ಥಿತಮ್ ।\\
ಅನಾರ್ಯಜುಷ್ಟಮಸ್ವರ್ಗ್ಯಮಕೀರ್ತಿಕರಮರ್ಜುನ ॥ 2 ॥}
\cquote{ಶ್ರೀ ಭಗವಂತನು ಹೇಳಿದನು, ಅರ್ಜುನ, ಬಲ್ಲವರು ಮೆಚ್ಚಿದ ಸ್ವರ್ಗಕ್ಕೆ ಒಯ್ಯದ ಮತ್ತು ಕೆಟ್ಟ ಹೆಸರನ್ನು ತರುವ ಈ ದೌರ್ಬಲ್ಯ ನಿನಗೆ ಬರಬಾರದ ಕಡೆ ಹೇಗೆ ಬಂದಿತು?\\}
\slcol{\Index{ಕ್ಲೈಬ್ಯಂ ಮಾ ಸ್ಮ ಗಮಃ }ಪಾರ್ಥ ನೈತತ್ತ್ವಯ್ಯುಪಪದ್ಯತೇ ।\\
ಕ್ಷುದ್ರಂ ಹೃದಯದೌರ್ಬಲ್ಯಂ ತ್ಯಕ್ತ್ವೋತ್ತಿಷ್ಠ ಪರಂತಪ ॥ 3 ॥}
\cquote{ಅರ್ಜುನ, ಎದೆಗೆಡಬೇಡ. ನಿನಗೆ ಇದು ತರವಲ್ಲ. ಶತ್ರುಗಳನ್ನು ಗದಗುಟ್ಟಿಸುವ ವೀರನೇ, ಕೀಳಾದ ಅಳುಕನ್ನು ಕೊಡವಿಹಾಕಿ ಎದ್ದೇಳು.\\}
\slcol{ಅರ್ಜುನ ಉವಾಚ ।\\
\Index{ಕಥಂ ಭೀಷ್ಮಮಹಂ} ಸಾಂಖ್ಯೇ ದ್ರೋಣಂ ಚ ಮಧುಸೂದನ ।\\
ಇಷುಭಿಃ ಪ್ರತಿಯೋತ್ಸ್ಯಾಮಿ ಪೂಜಾರ್ಹಾವರಿಸೂದನ ॥ 4 ॥}
\cquote{ಅರ್ಜುನನು ಹೇಳಿದನು, ನನ್ನಿಂದ ಪೂಜೆಗೊಳ್ಳುವುದಕ್ಕೆ ತಕ್ಕವರಾದ ಭೀಷ್ಮನನ್ನು ದ್ರೋಣನನ್ನು ನಾನು ಯುದ್ಧದಲ್ಲಿ ಹೇಗೆ ಬಾಣಗಳಿಂದ  ಹೊಡೆಯಲಿ? ಮಧುಸೂದನ.\\}

\newpage
\begin{mananam}{\kanfont ಮನನ ಶ್ಲೋಕ - \textenglish{2, 3}}
\footnotesize \mananamfont ನಾನು ನನ್ನ ಮೇಲಿನ ಅನುಕಂಪದಿಂದ ಅದರಲ್ಲಿಯೇ ಹೊರಳಾಡುತ್ತಿರುವೆನಾ? ನನ್ನ ಕೆಲವು ಬಾಹ್ಯ ಪರಿಸ್ಥಿತಿಗಳು ಮತ್ತು ಆಂತರಿಕ ಪ್ರಚೋದನೆಗಳು ನನ್ನ ಶಕ್ತಿಯನ್ನು ಕುಂದಿಸಿ ನಾನು ದುರ್ಬಲನಾಗುತ್ತಿರುವೆನಾ? ಗುರುಗಳು ಹಿತೈಷಿಗಳು ಮಾಡಿದ ಸದುಪದೇಶಗಳನ್ನು ಅಳವಡಿಸಿಕೊಳ್ಳಲಾರದಷ್ಟು ಸಂವೇದನಶೀಲನಾಗಿದ್ದೇನಾ?
\end{mananam}
\WritingHand\enspace\textbf{ಆತ್ಮ ವಿಮರ್ಶೆ}
\begin{inspiration}{\kanfont ಸ್ಪೂರ್ತಿ}
\footnotesize \mananamfont ಸವಾಲುಗಳನ್ನು ಎದುರಿಸಲು ನೀವು ಎಂದಿಗೂ ದುರ್ಬಲರಲ್ಲ. ಯಾವುದೇ ತರಹದ ಪ್ರಲೋಭನೆ ಸ್ವಯಂ ಅನುಕಂಪ ಮತ್ತು ಅನುಮಾನಗಳಿಗೆ ಒಳಗಾಗಬೇಡಿ. ಜೀವನದ ಅತ್ಯುನ್ನತ ಗುರಿ ಮತ್ತು ಆಕಾಂಕ್ಷೆಗಳೊಂದಿಗೆ ಮುಂದುವರೆಯಿರಿ. ನೀವು ಒಂದು ದೃಢ ಸಂಕಲ್ಪ ಮಾಡಿದಾಗ ಇಡೀ ಬ್ರಹ್ಮಾಂಡದ ಎಲ್ಲಾ ಸಕಾರಾತ್ಮಕ ಶಕ್ತಿಗಳು ನಿಮ್ಮನ್ನು ಬಲಪಡಿಸಿ ಬೆಂಬಲಿಸುತ್ತವೆ.
\end{inspiration}
\newpage

\slcol{\Index{ಗುರೂನಹತ್ವಾ ಹಿ} ಮಹಾನುಭಾವಾನ್ಶ್ರೇಯೋ \\ಭೋಕ್ತುಂ ಭೈಕ್ಷ್ಯಮಪೀಹ ಲೋಕೇ ।\\
ಹತ್ವಾರ್ಥಕಾಮಾಂಸ್ತು ಗುರುನಿಹೈವ \\ಭುಂಜೀಯ ಭೋಗಾನ್ऽರುಧಿರಪ್ರದಿಗ್ಧಾನ್ ॥ 5 ॥}
\cquote{ಮಹಾತ್ಮರಾದ ಗುರುಗಳನ್ನು ಕೊಲ್ಲುವ ಬದಲು ಈ ಲೋಕದಲ್ಲಿ ತಿರಿದು ತಿನ್ನುವುದಾದರೂ ಮೇಲು. ದುಡ್ಡಿನ ಆಸೆಯಿಂದ ಯುದ್ಧಕ್ಕೆ ನಿಂತಿರುವ ಗುರುಗಳನ್ನು ಕೊಂದು ಅವರ ನೆತ್ತರಿನಿಂದ ತೋಯ್ದ ರಾಜ ಭೋಗವನ್ನು ನಾನಿಲ್ಲಿ  ಹೇಗೆ ಉಣ್ಣಬೇಕು.\\}
\slcol{\Index{ನ ಚೈತದ್ವಿದ್ಮಃ ಕತರನ್ನೋ} ಗರೀಯೋ ಯದ್ವಾ \\ಜಯೇಮ ಯದಿ ವಾ ನೋ ಜಯೇಯುಃ ।\\
ಯಾನೇವ ಹತ್ವಾ ನ ಜಿಜೀವಿಷಾಮಸ್ತೇऽವಸ್ಥಿತಾಃ \\ಪ್ರಮುಖೇ ಧಾರ್ತರಾಷ್ಟ್ರಾಃ ॥ 6 ॥}
\cquote{ಯಾವುದು ಸರಿಯೋ ಯಾವುದು ತಪ್ಪೋ ಗೊತ್ತಿಲ್ಲ. ನಾವು ಗೆಲ್ಲುವೆವೂ ಅಥವಾ ಅವರೇ ನಮ್ಮನ್ನು ಗೆಲ್ಲುವರೋ ಅದು ಸಹ ಗೊತ್ತಿಲ್ಲ. ಯಾರನ್ನು ಕೊಂದು ನಾವು ಬದುಕ ಬಯಸುವುದಿಲ್ಲವೋ ಅಂತ ಕೌರವರೇ ಎದುರಿಗೆ ನಿಂತಿದ್ದಾರೆ.\\}
\slcol{\Index{ಕಾರ್ಪಣ್ಯದೋಷೋಪ}ಹತಸ್ವಭಾವಃ \\ಪೃಚ್ಛಾಮಿ ತ್ವಾಂ ಧರ್ಮಸಂಮೂಢಚೇತಾಃ ।\\
ಯಚ್ಛ್ರೇಯಃ ಸ್ಯಾನ್ನಿಶ್ಚಿತಂ ಬ್ರೂಹಿ ತನ್ಮೇ \\ಶಿಷ್ಯಸ್ತೇऽಹಂ ಶಾಧಿ ಮಾಂ ತ್ವಾಂ ಪ್ರಪನ್ನಮ್ ॥ 7 ॥}
\cquote{ಧೈರ್ಯದಿಂದ ಕಂಗೆಟ್ಟಿದ್ದೇನೆ. ಧರ್ಮದ ಬಗೆಗೆ ಮನಸ್ಸು ನಿರ್ಧರಿಸಲಾರದಾಗಿದೆ. ಆದ್ದರಿಂದ ನಿನ್ನನ್ನು ಕೇಳಿಕೊಳ್ಳುತ್ತೇನೆ, ಯಾವುದು ಸರಿ ಎಂಬುದನ್ನು ನೀನೆ ನನಗೆ ತಿಳಿ ಹೇಳಬೇಕು. ನಾನೇ ನಿನ್ನ ಶಿಷ್ಯ. ಶರಣು ಬಂದಿರುವ ನನಗೆ ನೀನೇ ದಾರಿ ತೋರಬೇಕು.\\}
\slcol{\Index{ನ ಹಿ ಪ್ರಪಶ್ಯಾಮಿ ಮಮಾಪ}ನುದ್ಯಾದ್ಯಚ್ಛೋಕಮು\\ಚ್ಛೋಷಣಮಿಂದ್ರಿಯಾಣಾಮ್ ।\\
ಅವಾಪ್ಯ ಭೂಮಾವಸಪತ್ನಮೃದ್ಧಂ \\ರಾಜ್ಯಂ ಸುರಾಣಾಮಪಿ ಚಾಧಿಪತ್ಯಮ್ ॥ 8 ॥}
\cquote{ಶತ್ರುಗಳಿಲ್ಲದ  ಸಮೃದ್ಧವಾದ ಇಡಿಯ ಭೂಮಂಡಲದ ಒಡೆತನ ಅಥವಾ ದೇವಲೋಕದ ಒಡೆತನವೇ ದೊರೆತರೂ ನನ್ನ ಇಂದ್ರಿಯಗಳನ್ನೆಲ್ಲ ಸೊರಗಿಸುವ ಈ ದುಃಖವನ್ನು ಕಳೆದೀತೆಂದು ನನಗೆ ಕಾಣುವುದಿಲ್ಲ.\\}
\slcol{ಸಂಜಯ ಉವಾಚ ।\\
\Index{ಏವಮುಕ್ತ್ವಾ ಹೃಷೀಕೇಶಂ} ಗುಡಾಕೇಶಃ ಪರಂತಪ ।\\
ನ ಯೋತ್ಸ್ಯ ಇತಿ ಗೋವಿಂದಮುಕ್ತ್ವಾ ತೂಷ್ಣೀಂ ಬಭೂವ ಹ ॥ 9 ॥}
\cquote{ಸಂಜಯನು ಹೇಳಿದನು, ಶತ್ರು ಗಳನ್ನು ಗದಗುಟ್ಟಿಸುವ ಅರ್ಜುನನು ಕೃಷ್ಣನನ್ನು ಕುರಿತು ಹೀಗೆ ಹೇಳಿ, ನಾನು ಕಾದಲಾರೆ ಎಂದು ಸುಮ್ಮನಾದನು.}

\begin{mananam}{\kanfont ಮನನ ಶ್ಲೋಕ - \textenglish{7}}
\footnotesize \mananamfont ನಮ್ಮ ದೌರ್ಬಲ್ಯ ಮತ್ತು ಮಿತಿಗಳ ಬಗ್ಗೆ ಅರಿವು, ಅದನ್ನು ಸರಿಯಾಗಿ ಅಥೈಸಿಕೊಳ್ಳುವುದರಿಂದ ನಮ್ಮ ಜೀವನದಲ್ಲಿ ಅಧ್ಯಾತ್ಮ ಲಕ್ಷಣಗಳಾದ ನಮ್ರತೆ ಗೌರವವನ್ನು ಪಡೆಯಬಹುದು. ನನಗೇನು ಗೊತ್ತಿಲ್ಲ ಎಂದು ತಿಳಿದುಕೊಳ್ಳುವಷ್ಟು ವಿನಮ್ರನಾಗಿದ್ದೇನೆಯೇ?
\end{mananam}
\WritingHand\enspace\textbf{ಆತ್ಮ ವಿಮರ್ಶೆ}
\begin{inspiration}{\kanfont ಸ್ಪೂರ್ತಿ}
\footnotesize \mananamfont ಶಿಷ್ಯನು ಸಿದ್ಧವಾದಾಗ, ಅವನ ಬಳಿ ಗುರು ಬಂದೇ ಬರುತ್ತಾನೆ ಎಂಬುದು ಎಲ್ಲಾ ಅಧ್ಯಾತ್ಮಿಕ ಅನ್ವೇಷಕರಿಗೆ ಹೇಳುವ ಗಾದೆಯಾಗಿದೆ. ಇಲ್ಲಿ ಸಿದ್ಧವಾಗಿರುವುದು ಎಂದರೆ ಸಮಯವನ್ನು ಆಧರಸಿರುವುದಲ್ಲ ಆದರೆ ಶಿಷ್ಯನ ಮಾನಸಿಕ ಸ್ಥಿತಿಗಳಾದ ಮುಕ್ತಮನಸ್ಸು ಮತ್ತು ಗ್ರಹಿಕಾ ಶಕ್ತಿಯಾಗಿದೆ. ನಿಜವಾದ ನಮ್ರತೆಯು ನಮ್ಮ ದೌರ್ಬಲ್ಯವಲ್ಲ ಅದು ಶ್ರೇಷ್ಠತೆ.
\end{inspiration}
\newpage

\slcol{\Index{ತಮುವಾಚ ಹೃಷೀಕೇಶಃ} ಪ್ರಹಸನ್ನಿವ ಭಾರತ ।\\
ಸೇನಯೋರುಭಯೋರ್ಮಧ್ಯೇ ವಿಷೀದಂತಮಿದಂ ವಚಃ ॥ 10 ॥}
\cquote{ದೃತರಾಷ್ಟ್ರನೇ, ಎರಡು ದಂಡುಗಳ ನಡುವೆ ವ್ಯಥೆಗೊಳ್ಳುತ್ತಿರುವ ಅರ್ಜುನನ್ನು ಕುರಿತು ಕೃಷ್ಣನು ಮುಗುಳು ನಗುತ್ತಲೇ ಹೀಗೆ ಹೇಳಿದನು.\\}
\slcol{ಶ್ರೀಭಗವಾನುವಾಚ ।\\
\Index{ಅಶೋಚ್ಯಾನನ್ವಶೋಚಸ್ತ್ವಂ} ಪ್ರಙ್ಞಾವಾದಾಂಶ್ಚ ಭಾಷಸೇ ।\\
ಗತಾಸೂನಗತಾಸೂಂಶ್ಚ ನಾನುಶೋಚಂತಿ ಪಂಡಿತಾಃ ॥ 11 ॥}
\cquote{ಶ್ರೀ ಭಗವಂತನು ಹೇಳಿದನು,\\
ನೀನು ಯಾರಿಗಾಗಿ ಅಳಬಾರದೊ ಅವರಿಗಾಗಿ ಅಳುತ್ತಿ,ಜಾಣನಂತೆ ಮಾತುಗಳನ್ನು ಆಡುತ್ತಿ, ತಿಳಿದವರು ಸತ್ತವರಿಗಾಗಲಿ ಸಾಯುವವರಿಗಾಗಲಿ ಅಳುವುದಿಲ್ಲ.\\}
\slcol{\Index{ನ ತ್ವೇವಾಹಂ ಜಾತು} ನಾಸಂ ನ ತ್ವಂ ನೇಮೇ ಜನಾಧಿಪಾಃ ।\\
ನ ಚೈವ ನ ಭವಿಷ್ಯಾಮಃ ಸರ್ವೇ ವಯಮತಃ ಪರಮ್ ॥ 12 ॥}
\cquote{ನಾನು, ನೀನು, ಈ ಅರಸರು ಹಿಂದೆಂದೂ ಇಲ್ಲವೆಂದಾದುದಿಲ್ಲ. ಮುಂದೆಯೂ ನಾವೆಲ್ಲರೂ ಇಲ್ಲವಾಗಲಾರೆವು.\\}
\slcol{\Index{ದೇಹಿನೋऽಸ್ಮಿನ್ಯಥಾ ದೇಹೇ} ಕೌಮಾರಂ ಯೌವನಂ ಜರಾ ।\\
ತಥಾ ದೇಹಾಂತರಪ್ರಾಪ್ತಿರ್ಧೀರಸ್ತತ್ರ ನ ಮುಹ್ಯತಿ ॥ 13 ॥}
\cquote{ಜೀವನಿಗೆ ಈ ದೇಹದಲ್ಲಿ ಹುಡುಗತನ, ಪ್ರಾಯ, ಮುಪ್ಪು ಹೇಗೆ, ಹಾಗೆ ಬೇರೆ ದೇಹ ಬರುವುದು.ಈ ವಿಷಯದಲ್ಲಿ ಜಾಣರು ಕಂಗಡುವುದಿಲ್ಲ.\\}
\slcol{\Index{ಮಾತ್ರಾಸ್ಪರ್ಶಾಸ್ತು ಕೌಂತೇಯ} ಶೀತೋಷ್ಣಸುಖದುಃಖದಾಃ ।\\
ಆಗಮಾಪಾಯಿನೋऽನಿತ್ಯಾಸ್ತಾಂಸ್ತಿತಿಕ್ಷಸ್ವ ಭಾರತ ॥ 14 ॥}
\cquote{ಕುಂತಿಪುತ್ರ, ಇಂದ್ರಿಯಗಳು ವಿಷಯಗಳೊಂದಿಗೆ ಕೂಡಿದಾಗ ಶೀತೋಷ್ಣ ಸುಖದುಃಖಗಳು ಸಂಭವಿಸುತ್ತವೆ. ಆ ಕೂಡಿಕೆ ಸ್ಥಿರವಲ್ಲ. ಬಂದು ಹೋಗುತ್ತಿರುತ್ತವೆ.ಆದುದರಿಂದ ಎ ಭಾರತವೀರ ಸಹಿಸಿಕೋ.\\}
\slcol{\Index{ಯಂ ಹಿ ನ ವ್ಯಥಯಂತ್ಯೇತೇ} ಪುರುಷಂ ಪುರುಷರ್ಷಭ ।\\
ಸಮದುಃಖಸುಖಂ ಧೀರಂ ಸೋऽಮೃತತ್ವಾಯ ಕಲ್ಪತೇ ॥ 15 ॥}
\cquote{ ಅರ್ಜುನ, ಸುಖ-ದುಃಖಗಳಲ್ಲಿ ಒಂದೇ ತರನಾಗಿರುವ ಯಾವ ಧೀರನನ್ನು ಇವು ವ್ಯಥೆಗೊಳಿಸುವುದಿಲ್ಲವೋ ಅವನು ಮೋಕ್ಷಕ್ಕೆ ಯೋಗ್ಯನಾಗುತ್ತಾನೆ .\\}
\slcol{\Index{ನಾಸತೋ ವಿದ್ಯತೇ ಭಾವೋ} ನಾಭಾವೋ ವಿದ್ಯತೇ ಸತಃ ।\\
ಉಭಯೋರಪಿ ದೃಷ್ಟೋऽಂತಸ್ತ್ವನಯೋಸ್ತತ್ತ್ವದರ್ಶಿಭಿಃ ॥ 16 ॥}
\cquote{ ಇಲ್ಲದ ಅಸತ್ ಪದಾರ್ಥಕ್ಕೆ ಇರುವಿಕೆ ಇಲ್ಲ. ಇರುವ ಸದ್ ವಸ್ತುವಿಗೆ ಇಲ್ಲದಿರುವಿಕೆ ಇಲ್ಲ.  ಈ ಎರಡು ತತ್ವಗಳ ನಿರ್ಣಯವನ್ನು ತತ್ವಜ್ಞಾನಿಗಳು ಬಲ್ಲರು.}

\newpage
\begin{mananam}{\kanfont ಮನನ ಶ್ಲೋಕ - \textenglish{11}}
\footnotesize \mananamfont ನನ್ನ ಅಂತಃಸ್ಪುರಣೆ ಮತ್ತು ಬುದ್ಧಿವಂತಿಕೆಯ ಮೇಲೆ ನಾನು ನಿಲುವುಗಳನ್ನು ತೆಗೆದುಕೊಳ್ಳುತ್ತೇನೆಯೇ? ಅಥವಾ ನನ್ನ ನಿಲುವುಗಳೇ ಸರಿ ಎಂದು ಅದನ್ನು ಅನುಮೋದಿಸಲು ಅಧಿಕೃತ ಬೋಧನೆಗಳನ್ನು ದುರುಪಯೋಗಪಡಿಸಿಕೊಳ್ಳುತ್ತೇನಾ?
\end{mananam}
\WritingHand\enspace\textbf{ಆತ್ಮ ವಿಮರ್ಶೆ}
\begin{inspiration}{\kanfont ಸ್ಪೂರ್ತಿ}
\footnotesize \mananamfont ನನ್ನ ಪ್ರಿಯಜನರ ಸದ್ಗುಣಗಳನ್ನು ಗಮನಿಸಲು ಕಲಿಯುತ್ತೀನಾ? ಅವರ ಮರಣದ ನಂತರ ಅವರ ಚೈತನ್ಯವನ್ನು ಗೌರವಿಸುತ್ತಾ ಅವರ ಜೀವನ ಮೌಲ್ಯಗಳನ್ನು ನನ್ನ ಜೀವನದಲ್ಲಿ ಅಳವಡಿಸಿಕೊಳ್ಳುತ್ತೇನಾ? 
\end{inspiration}
\newpage

\begin{mananam}{\kanfont ಮನನ ಶ್ಲೋಕ - \textenglish{13}}
\footnotesize \mananamfont ನನ್ನ ಜೀವನದಲ್ಲಿ ಬದಲಾವಣೆಗಳನ್ನು ವಿರೋಧಿಸುತ್ತೇನೆಯೇ? ನನ್ನಲ್ಲಿ ಉತ್ತಮ ಆರೋಗ್ಯ ಮತ್ತು ಯೌವ್ವನದ ಶಕ್ತಿ ಇರುವಾಗ ಮನಸ್ಸು ಉಲ್ಲಾಸವಾಗಿರುತ್ತದೆ, ಮತ್ತು ಶಕ್ತಿ ಕುಂದಿದಾಗ ಅನಾರೋಗ್ಯ ಇರುವಾಗ ಮನಸ್ಸು ಅಸಮಾಧಾನ ಗೊಳ್ಳುತ್ತದೆಯೇ? ಈ ಪ್ರಕೃತಿಯಲ್ಲಿ ಎಲ್ಲವೂ ನಿರಂತರ ಬದಲಾವಣೆಗೆ ಒಳಪಟ್ಟಿದೆ ಎಂದು ತಿಳಿದು ಕೂಡ ನನಗೆ ದೈಹಿಕ ಬದಲಾವಣೆಗಳಾದಾಗ ಮನಸ್ಸು ವಿಚಲಿತವಾಗದೇ ದೃಢವಾಗಿಟ್ಟುಕೊಳ್ಳಬಲ್ಲನೆ? ನನ್ನ ದೈಹಿಕ ಲಕ್ಷಣಗಳು ಮತ್ತು ಮನಸ್ಸಿನ ಆಧಾರದ ಮೇಲೆ ನನ್ನ ಜೀವನದ ಮಿತಿಗಳನ್ನು ನನ್ನ ಮೇಲೆ ಹೇರಿಕೊಳ್ಳಲು ಸಾಧ್ಯವೇ?
\end{mananam}
\WritingHand\enspace\textbf{ಆತ್ಮ ವಿಮರ್ಶೆ}
\begin{inspiration}{\kanfont ಸ್ಪೂರ್ತಿ}
\footnotesize \mananamfont ಬಾಲ್ಯ, ಯೌವ್ವನ, ವೃದ್ಯಾಪ್ಯ ಈ ದೇಹದ ನೈಸರ್ಗಿಕ ಪ್ರಕ್ರಿಯೆಯಾಗಿದೆ. ಯೌವನದಲ್ಲಿರುವ ಸೌಂದರ್ಯವೇ ಸತ್ಯ ಎಂದುಕೊಂಡರೆ ದುಃಖ ಖಚಿತ. ಮನಸ್ಸಿನ ನಿರ್ಮಲತೆ, ಶುದ್ಧತೆ ದೈಹಿಕ ಸೌಂದರ್ಯಕ್ಕಿಂತ ಶ್ರೇಷ್ಠವಾಗಿದೆ. ಬದಲಾವಣೆಯಾಗದೆ ಇರುವ ಆತ್ಮವೂ ಈ ದೇಹದೊಂದಿಗಿದೆ. ಈ ಆತ್ಮ ಸೌಂದರ್ಯವೂ ಅತ್ಯಂತ ಶ್ರೇಷ್ಠವಾಗಿದೆ. ಪರಮಾತ್ಮನ ಕೌಶಲದಿಂದ ವಿನ್ಯಾಸವಾಗಿರುವ ಈ ದೇಹವು ಸದಾ ಬದಲಾವಣೆಗೆ ಒಳಪಟ್ಟಿದೆ. ಇದರಿಂದ ನಮ್ಮ ದೈಹಿಕ ಸ್ಥಿತಿಯನ್ನು ಮೀರಿ ಯಾವಾಗಲೂ ಬದಲಾಗದೇ ಇರುವ ಆತ್ಮದ ಬಗ್ಗೆ ಅರಿತುಕೊಳ್ಳಬಹುದಾಗಿದೆ.
\end{inspiration}
\newpage

\begin{mananam}{\kanfont ಮನನ ಶ್ಲೋಕ - \textenglish{14}}
\footnotesize \mananamfont ನಮ್ಮ ದೇಹದಂತೆಯೇ ಮಾನಸಿಕ ಸ್ಥಿತಿಯೂ ಕೂಡಾ ಬದಲಾಗುತ್ತಲೇ ಇರುತ್ತದೆ. ನಾವು ನಮ್ಮ ಮಾನಸಿಕ ಸ್ಥಿತಿಯನ್ನು ಒಳ್ಳೆಯದು ಅಥವಾ ಕೆಟ್ಟದ್ದು ಎಂದು ನಿರ್ಣಯಿಸದೆ, ನಮ್ಮ ಮನಸ್ಸಿನ ಸ್ಥಿತಿಯ ಬಗ್ಗೆ ಅರಿಯಬಹುದೇ? ಮನಸ್ಸಿನಲ್ಲಿ ಬರುವ ಕೋಪ, ಭಯ, ದುಃಖ ಅಥವಾ ಉದ್ವೇಗ, ತೃಪ್ತಿ ಮತ್ತು ಸಂತೋಷ ಇವೆಲ್ಲದರ ಹಿಂದೆ ದೃಢವಾಗಿರುವ ಸೂಕ್ಷ್ಮ ಹಿನ್ನೆಲೆಯನ್ನು  ಗಮನಿಸಬಹುದೇ?\\
ನಾನು ಇನ್ನು ಏನನ್ನೂ ಸಹಿಸಲು ಸಾಧ್ಯವಿಲ್ಲ ಎಂದು ಭಾವಿಸಿದ್ದೇನೆಯೇ? ನಾನು ಸುಲಭವಾಗಿ ತಾಳ್ಮೆ ಕಳೆದುಕೊಳ್ಳುತ್ತೀನಾ? ಸಿಡಿಮಿಡಿ ಗೊಳ್ಳುತ್ತಿದ್ದೀನಾ?  ಹಾಗಾದರೆ ನಾನು ಈ ಜೀವನದಲ್ಲಿ ನನ್ನ ಮನಸ್ಸಿನಲ್ಲಿ ಸುಳಿಯುವ ವಿಚಾರಗಳನ್ನೇ ಶಾಶ್ವತ ಎಂದುಕೊಂಡಿದ್ದೇನೆ.
\end{mananam}
\WritingHand\enspace\textbf{ಆತ್ಮ ವಿಮರ್ಶೆ}
\begin{inspiration}{\kanfont ಸ್ಪೂರ್ತಿ}
\footnotesize \mananamfont ಈ ನಮ್ಮ ಜೀವನವು ಹಳೆಯ ನೆನಪುಗಳನ್ನು ಸ್ಮರಿಸುತ್ತಾ ಬದುಕಲು ವಿನ್ಯಾಸವಾಗಿಲ್ಲ. ಇದು ನಾವು ಮರೆತುಹೋಗಿರುವ ನಮ್ಮ ಆತ್ಮದ ಸಹಜ ಸ್ವಭಾವದ ಬಗ್ಗೆ ಸ್ಮರಿಸಬೇಕಾಗಿದೆ. ಕ್ಷಣಿಕ ಸಂತೋಷ ಕೊಡುವ ಸಣ್ಣಪುಟ್ಟ ಅನುಭವಗಳಿಗೋಸ್ಕರ ನಿಮ್ಮ ಸಮಯವನ್ನು ವ್ಯರ್ಥಮಾಡಬೇಡಿ.  ಆನಂದವು ನಿಮ್ಮ ಜನ್ಮ ಸಿದ್ಧ ಹಕ್ಕು.
\end{inspiration}
\newpage

\begin{mananam}{\kanfont ಮನನ ಶ್ಲೋಕ - \textenglish{15}}
\footnotesize \mananamfont ನಾನು ಸಂತೋಷವನ್ನು ಎಲ್ಲಿ ಅನುಭವಿಸುತ್ತೇನೆ?ನಾನು ದುಃಖವನ್ನು ಎಲ್ಲಿ ಅನುಭವಿಸುತ್ತೇನೆ? ಇವೆಲ್ಲವೂ ಕೇವಲ ಮಾನಸಿಕ ಸ್ಥಿತಿಗಳು ಎಂದು ನನ್ನ ಒಳನೋಟದಲ್ಲಿ ನಾನು ನೋಡಬಹುದೇ? ನಿಮ್ಮ ಜೀವನದಲ್ಲಿ ಕೆಲವೊಂದು ಸನ್ನಿವೇಶಗಳಲ್ಲಿ ಒಂದು ವಸ್ತು ನಿಮಗೆ ದುಃಖವನ್ನು ತರುತ್ತದೆ ಆದರೆ ಅದೇ ವಸ್ತುವು ಬೇರೆಯವರಿಗೆ ಸಂತೋಷವನ್ನು ತರುತ್ತದೆ.ಅದು ಆಹಾರ, ಹವಾಮಾನ ಅಥವಾ ಇನ್ಯಾವುದೇ ಆಗಿರಬಹುದು ಎಂಬುದನ್ನು ಗಮನಿಸಿ. ಇದಲ್ಲದೆ ನಿಮ್ಮ ಮಾನಸಿಕ ಸ್ಥಿತಿಗಳ ಅರಿವನ್ನು ನೀವು ಗಮನಿಸಬಹುದೇ? ಅದು ಬಂದು ಹೋಗುತ್ತದೆಯೇ?
\end{mananam}
\WritingHand\enspace\textbf{ಆತ್ಮ ವಿಮರ್ಶೆ}
\begin{inspiration}{\kanfont ಸ್ಪೂರ್ತಿ}
\footnotesize \mananamfont ಜೀವನದಲ್ಲಿ ಪ್ರತಿಯೊಂದು ಜೀವಿಗೂ ಸಾವಿನ ಭಯವಿರುವಂತೆಯೇ ಅಮರತ್ವವೂ ಕೂಡ ರಹಸ್ಯ ಬಯಕೆಯಾಗಿದೆ. ಆತ್ಮವು ಈ ದೇಹವನ್ನು ಮನಸ್ಸನ್ನು ಮೀರಿದುದಾಗಿದೆ ಎಂದು ಯಾರು ಅರಿತುಕೊಳ್ಳುವರೋ ಅವರು ಬದಲಾಗುವ ಸ್ಥಿತಿಗಳಿಂದ ತೊಂದರೆಗೆ ಒಳಗಾಗುವುದಿಲ್ಲ. ಈ ತತ್ವವನ್ನು ತಿಳಿದ ವ್ಯಕ್ತಿಯು ಅಮರನಾಗುತ್ತಾನೆ. ಅವನು ಅಥವಾ ಅವಳು ತನ್ನ ಚೈತನ್ಯವು ಎಂದೆಂದಿಗೂ ಅಮರ ಎಂದು ಅರಿತುಕೊಳ್ಳುತ್ತಾನೆ.
\end{inspiration}

\newpage
\begin{mananam}{\kanfont ಮನನ ಶ್ಲೋಕ - \textenglish{16}}
\footnotesize \mananamfont ಈ ಜೀವನದಲ್ಲಿ ಪ್ರತಿಯೊಂದು ಸನ್ನಿವೇಶದಲ್ಲೂ ಯಾವುದು ವಾಸ್ತವ ಮತ್ತು ಯಾವುದು ಅವಾಸ್ತವ ಎಂದು ತಿಳಿಯುವಷ್ಟು ಧೈರ್ಯವಿದೆಯೇ? ನನ್ನ ಹಿಂದಿನ ಅನುಭವಗಳು, ಸಾಮಾಜಿಕ ನಂಬಿಕೆಗಳು, ಸಾಂಸ್ಕೃತಿಕ ರೂಡಿಗಳು ಇತ್ಯಾದಿಗಳಿಂದ ಜನರು ಮತ್ತು ಸನ್ನಿವೇಶಗಳ ಬಗ್ಗೆ ನನ್ನ ಗ್ರಹಿಕೆ ರೂಪಗೊಂಡಿದೆಯೇ? ನನ್ನ ನಂಬಿಕೆ ಮತ್ತು ತಾರ್ಕಿಕ ನಿಲುವುಗಳನ್ನು ಪ್ರಶ್ನಿಸಲು ಒಪ್ಪುತ್ತೇನೆಯೆ? ಋಷಿಗಳು ಕಂಡುಕೊಂಡ ಪರಮ ಸತ್ಯವನ್ನು, ನನ್ನನ್ನು ನಾನು ಅರಿಯಲು ಜೀವನದ ಉನ್ನತ ವಾಸ್ತವವನ್ನು ಅರಿತುಕೊಂಡು ಜೀವನ್ಮುಕ್ತನಾಗಲು ಅವರು ಹೇಳಿದ ಪರಮ ಸತ್ಯವನ್ನು ಅಳಪಡಿಸಿಕೊಳ್ಳಲು ಒಪ್ಪಿಕೊಂಡಿರುವೆನಾ? ಸತ್ಯ ಮತ್ತು ವಾಸ್ತವದ ಅನ್ವೇಷಣೆಗೆ ನನ್ನ ಜೀವನವನ್ನು ಮುಡಿಪಾಗಿಡಲು ನಾನು ಸಿದ್ಧನಿದ್ದೇನಾ?
\end{mananam}
\WritingHand\enspace\textbf{ಆತ್ಮ ವಿಮರ್ಶೆ}
\begin{inspiration}{\kanfont ಸ್ಪೂರ್ತಿ}
\footnotesize \mananamfont ಸಾಫೇಕ್ಷೆ ಸತ್ಯಗಳ ವಿವಿಧ ಹಂತಗಳಿವೆ. ಆದರೆ ಒಂದೇ ಒಂದು ಅಂತಿಮ ಸತ್ಯವಿದೆ. ಪರಮ ಸತ್ಯವನ್ನು ಕಂಡುಕೊಂಡವರು ಮಾತ್ರ ಮುಕ್ತರಾಗುವರು. ಉಳಿದವರೆಲ್ಲಾ ಅದರ ಸ್ವರೂಪವನ್ನು ಅರಿಯದೇ ಆ ಸ್ವಾತಂತ್ರ್ಯವನ್ನು ಬಯಸುತ್ತಿರುವವರು ಒಮ್ಮೆ ನೀವು ನಿಮ್ಮ ಜೀವನವನ್ನು ಈ ಸತ್ಯದ ಅನ್ವೇಷಣೆಗೆ ಮುಡಿಪಾಗಿಟ್ಟರೆ, ರಾಜ ಹರಿಶ್ಚಂದ್ರ ಮತ್ತು ಭಗವಾನ್ ರಾಮನಂತೆ ಮಾನಸಿಕ ಬಲ ಮತ್ತು ಅತ್ಯುನ್ನತ ಶಕ್ತಿಯನ್ನು ಪಡೆಯುವಿರಿ.
\end{inspiration}
\newpage

\slcol{\Index{ಅವಿನಾಶಿ ತು ತದ್ವಿದ್ಧಿ} ಯೇನ ಸರ್ವಮಿದಂ ತತಮ್ ।\\
ವಿನಾಶಮವ್ಯಯಸ್ಯಾಸ್ಯ ನ ಕಶ್ಚಿತ್ಕರ್ತುಮರ್ಹತಿ ॥ 17 ॥}
\cquote{ಈ ಸಮಸ್ತ ವಿಶ್ವವನ್ನು ತುಂಬಿಕೊಂಡಿರುವ ಆತ್ಮನು ನಾಶವಾಗುವವನಲ್ಲ. ಆ ಆತ್ಮನನ್ನು ನಾಶಗೊಳಿಸಲು ಯಾರೂ ಸಮರ್ಥರಲ್ಲ.\\}
\slcol{\Index{ಅಂತವಂತ ಇಮೇ ದೇಹಾ} ನಿತ್ಯಸ್ಯೋಕ್ತಾಃ ಶರೀರಿಣಃ ।\\
ಅನಾಶಿನೋऽಪ್ರಮೇಯಸ್ಯ ತಸ್ಮಾದ್ಯುಧ್ಯಸ್ವ ಭಾರತ ॥ 18 ॥}
\cquote{ಈ ದೇಹಗಳು ಒಂದಲ್ಲ ಒಂದು ದಿನ ಹೋಗಲೇಬೇಕು. ಒಳಗಿರುವ ಜೀವ ಮಾತ್ರ ನಿತ್ಯ. ಪೂರ್ಣನಾದ ಭಗವಂತನಂತೆ ಅವನಿಗೂ ನಾಶವಿಲ್ಲ. ಆದ್ದರಿಂದ ನಿತ್ಯಪೂರ್ಣನಾದ ಭಗವಂತನ ಪೂಜೆಯೆಂದು ಓ ಅರ್ಜುನ ಯುದ್ಧ ಮಾಡು.\\}
\slcol{\Index{ಯ ಏನಂ ವೇತ್ತಿ ಹಂತಾರಂ} ಯಶ್ಚೈನಂ ಮನ್ಯತೇ ಹತಮ್ ।\\
ಉಭೌ ತೌ ನ ವಿಜಾನೀತೋ ನಾಯಂ ಹಂತಿ ನ ಹನ್ಯತೇ ॥ 19 ॥}
\cquote{ಜೀವನನ್ನು ಯಾರಾದರೂ ಕೊಲ್ಲಬಹುದು ಎಂದಾಗಲಿ ಆಗ ಜೀವ ಸಾಯುತ್ತಾನೆ ಎಂದಾಗಲಿ ತಿಳಿದವರು ಏನನ್ನು ತಿಳಿದಿಲ್ಲ. ಏಕೆಂದರೆ ಜೀವ ಕೊಲ್ಲಲೂ ಆರ ಸಾಯಲು ಆರ.\\}
\slcol{\Index{ನ ಜಾಯತೇ ಮ್ರಿಯತೇ ವಾ} ಕದಾಚಿನ್ನಾಯಂ \\ಭೂತ್ವಾ ಭವಿತಾ ವಾ ನ ಭೂಯಃ ।\\
ಅಜೋ ನಿತ್ಯಃ ಶಾಶ್ವತೋऽಯಂ ಪುರಾಣೋ \\ನ ಹನ್ಯತೇ ಹನ್ಯಮಾನೇ ಶರೀರೇ ॥ 20 ॥}
\cquote{ಈ ಆತ್ಮ ಎಂದಿಗೂ ಹುಟ್ಟುವುದು ಇಲ್ಲ, ಸಾಯುವುದು ಇಲ್ಲ. ಈ ಆತ್ಮನು ಮೊದಲು ಇದ್ದವನೆನಿಸಿಕೊಂಡು, ಆಮೇಲೆ ಇಲ್ಲದವನು ಆಗುವುದಿಲ್ಲ. ಈ ಆತ್ಮನು ಹುಟ್ಟಿಲ್ಲದವನು,ಸಾವಿಲ್ಲದವನು,ಬೇರೆ ಬಗೆಗಳಾಗದವನು, ಪುರಾತನನು,ದೇಹವನ್ನು ಕೊಂದರು ಅವನು ಸಾಯುವುದಿಲ್ಲ.\\}
\slcol{\Index{ವೇದಾವಿನಾಶಿನಂ ನಿತ್ಯಂ} ಯ ಏನಮಜಮವ್ಯಯಮ್ ।\\
ಅಥಂ ಸ ಪುರುಷಃ ಪಾರ್ಥ ಕಂ ಘಾತಯತಿ ಹಂತಿ ಕಮ್ ॥ 21॥}
\cquote{ಅರ್ಜುನ, ಈ ಜೀವ ಯಾವ ಕಾರಣಕ್ಕು ನಾಶವಾಗದ, ಹುಟ್ಟದ, ರೂಪಾಂತಗೊಳ್ಳದ ನಿತ್ಯ ವಸ್ತು ಎಂದು ತಿಳಿದ ಮನುಷ್ಯ ಯಾರನ್ನಾದರೂ ಹೇಗೆ ಕೊಲ್ಲಿಸುವುದು? ಹೇಗೆ ಕೊಲ್ಲುವುದು?\\}
\slcol{\Index{ವಾಸಾಂಸಿ ಜೀರ್ಣಾನಿ ಯಥಾ} ವಿಹಾಯ ನವಾನಿ ಗೃಹ್ಣಾತಿ ನರೋऽಪರಾಣಿ ।\\
ತಥಾ ಶರೀರಾಣಿ ವಿಹಾಯ ಜೀರ್ಣಾನ್ಯನ್ಯಾನಿ ಸಂಯಾತಿ ನವಾನಿ ದೇಹೀ ॥ 22 ॥}
\cquote{ಮನುಷ್ಯ ಉಟ್ಟ ಬಟ್ಟೆ ಹಳತಾದಾಗ ಅವುಗಳನ್ನು ಬಿಟ್ಟು ಹೊಸತನ್ನು ತೊಟ್ಟುಕೊಳ್ಳುತ್ತಾನೆ. ಹಾಗೆಯೇ ಜೀವನವು ತನ್ನ ದೇಹ ಜೀರ್ಣವಾದಾಗ ಅದನ್ನು ತೊರೆದು ಇನ್ನೊಂದು ದೇಹವನ್ನು ಸೇರುತ್ತಾನೆ.\\}

\newpage
\begin{mananam}{\kanfont ಮನನ ಶ್ಲೋಕ - \textenglish{20, 21}}
\footnotesize \mananamfont ನನ್ನ ಬಗ್ಗೆ ನನ್ನ ಭಾವನೆಯು ಈ ಬೌತಿಕ ದೇಹಕ್ಕೆ ಮಾತ್ರ ಸೀಮಿತವಾಗಿದೆಯೇ? ನಾನು ಇತರರ ಬಗ್ಗೆ ನನ್ನ ಗ್ರಹಿಕೆಯನ್ನು ಅವರ ಭೌತಿಕ ಅಸ್ತಿತ್ವಕ್ಕೆ ಮಾತ್ರ ಸೀಮಿತಗೊಳಿಸುತ್ತೇನೆಯೇ? ಈಗಾಗಲೇ ಕಾಲವಾಗಿ ಹೋಗಿರುವ ಮಹಾಪುರುಷರ ಚೈತನ್ಯ ಮತ್ತು ಯೋಚನೆಗಳಿಗೆ ಸ್ಪಂದಿಸಲು ಪ್ರಯತ್ನಿಸಬಹುದೇ? ನಿತ್ಯ ಬದಲಾಗುವ ದೇಹ ಮತ್ತು ಮನಸ್ಸನ್ನು ಮೀರಿ ಯಾವಾಗಲೂ ಬದಲಾಗದೇ ಇರುವ ತತ್ವವನ್ನು ಗ್ರಹಿಸಬಹುದೇ?
\end{mananam}
\WritingHand\enspace\textbf{ಆತ್ಮ ವಿಮರ್ಶೆ}
\begin{inspiration}{\kanfont ಸ್ಪೂರ್ತಿ}
\footnotesize \mananamfont ಹುಟ್ಟಿದವರೆಲ್ಲ ಮರಣ ಹೊಂದುತ್ತಾರೆ ಎಂಬುದು ನಮ್ಮೆಲ್ಲರ ಭಾವನೆಯಾಗಿದೆ. ನಮ್ಮಲ್ಲಿ ಪರಿಪೂರ್ಣ ಜ್ಞಾನ ಉದಯಿಸಿದಾಗ ನಮ್ಮ ಈ ಅಂತಃಸತ್ವವು ಆತ್ಮವು ಹುಟ್ಟು ಸಾವಿಗೆ ಮೀರಿದ್ದಾಗಿದೆ ಮತ್ತು ಅದು ಎಂದೂ ಜನಿಸಿಲ್ಲ ಎಂಬ ವಾಸ್ತವಿತೆ ಕತೆ ಅರಿವಾಗುತ್ತದೆ.
\end{inspiration}
\newpage

\begin{mananam}{\kanfont ಮನನ ಶ್ಲೋಕ - \textenglish{22}}
\footnotesize \mananamfont ನಮ್ಮ ಸಮಾಜದಲ್ಲಿ ಒಬ್ಬ ವ್ಯಕ್ತಿಯನ್ನು ಭಾಹ್ಯ ರೂಪದಿಂದ ಅಳೆಯುತ್ತೀವಿ. ಒಬ್ಬ ನಟನಾದವನು ತನ್ನ ಪಾತ್ರಕ್ಕೋಸ್ಕರ ವಿಧವಿಧವಾದ ಉಡುಪುಗಳನ್ನು ಧರಿಸಿಕೊಂಡರೂ ಅವನು ಒಳಗಿನಿಂದ ಅದೇ ವ್ಯಕ್ತಿಯಾಗಿರುತ್ತಾನೆ. ಒಬ್ಬ ವ್ಯಕ್ತಿಯನ್ನು ಅವನು ಹೇಗಿದ್ದಾನೆಯೋ ಹಾಗೆ ನೋಡಲು ಕಲಿಯಬಹುದಾ? ಎಲ್ಲರಲ್ಲೂ ಏನೇ ನ್ಯೂನ್ಯತೆಗಳು ದೌರ್ಬಲ್ಯಗಳಿದ್ದರೂ ಅವರಲ್ಲಿರುವ ಶುದ್ಧ ಆತ್ಮ ತತ್ವವನ್ನು ನೋಡಲು ಕಲಿಯಬಹುದಾ?
\end{mananam}
\WritingHand\enspace\textbf{ಆತ್ಮ ವಿಮರ್ಶೆ}
\begin{inspiration}{\kanfont ಸ್ಪೂರ್ತಿ}
\footnotesize \mananamfont ಸೂಕ್ಷ್ಮದೃಷ್ಟಿ ಇರುವವನು ಒಬ್ಬ ವ್ಯಕ್ತಿಯನ್ನೇ ಅವನ ಭಾಹ್ಯರೂಪದಿಂದ ಅಳೆಯುವುದಿಲ್ಲ. ಅವರು ಅವನ ಅಥವಾ ಅವಳ ವ್ಯಕ್ತಿತ್ವದಿಂದ ಅಳೆಯುತ್ತಾರೆ. ಒಬ್ಬ ಸಂತನು ಪ್ರತಿಯೊಬ್ಬನನ್ನು ತನ್ನ ಆತ್ಮದ ಪ್ರತಿರೂಪ ಎಂದು ನೋಡುತ್ತಾನೆ.
\end{inspiration}
\newpage

\slcol{\Index{ನೈನಂ ಛಿಂದಂತಿ ಶಸ್ತ್ರಾಣಿ} ನೈನಂ ದಹತಿ ಪಾವಕಃ ।\\
ನ ಚೈನಂ ಕ್ಲೇದಯಂತ್ಯಾಪೋ ನ ಶೋಷಯತಿ ಮಾರುತಃ ॥ 23 ॥}
\cquote{ಇವನನ್ನು ಆಯುಧಗಳು ತುಂಡರಿಸಲಾರವು. ಬೆಂಕಿ ಸುಡಲಾರದು. ನೀರು ನೆನೆಸಲಾರದು.ಗಾಳಿ ಒಣಗಿಸಲು ಆರದು.\\}
\slcol{\Index{ಅಚ್ಛೇದ್ಯೋऽಯಮದಾಹ್ಯೋऽಯಮ}ಕ್ಲೇದ್ಯೋऽಶೋಷ್ಯ ಏವ ಚ ।\\
ನಿತ್ಯಃ ಸರ್ವಗತಃ ಸ್ಥಾಣುರಚಲೋऽಯಂ ಸನಾತನಃ ॥ 24 ॥}
\cquote{ಏಕೆಂದರೆ ಇವನು ತುಂಡಾಗದವನು, ಬೇಯದವನು,ನೆನೆಯದವನು,ಮತ್ತು ಒಣಗದವನು.ಏಕೆಂದರೆ ಈ ಜೀವ ಎಂದೆಂದೂ ಎಲ್ಲೆಡೆಯೂ ತುಂಬಿರುವ ನಿರ್ವಿಕಾರನೂ,ಅಚಲನೂ ಸನಾತನನೂ ಆದ ಭಗವಂತನ ಪಡಿನೆಳಲು.\\}
\slcol{\Index{ಅವ್ಯಕ್ತೋऽಯಮಚಿಂತ್ಯೋऽಯಮ}ವಿಕಾರ್ಯೋऽಯಮುಚ್ಯತೇ ।\\
ತಸ್ಮಾದೇವಂ ವಿದಿತ್ವೈನಂ ನಾನುಶೋಚಿತುಮರ್ಹಸಿ ॥ 25 ॥}
\cquote{ಇಂದ್ರಿಯಗಳಿಗೆ ಈ ಆತ್ಮ ಕಾಣಬರುವುದಿಲ್ಲ.ಮನಸ್ಸಿಗೆ ದೊರಕದು, ವಿಕಾರಕ್ಕೆ ಒಳಪಡದೆಂದು ಶಾಸ್ತ್ರಗಳು ಹೇಳುತ್ತಿವೆ. ಇವೆಲ್ಲವುಗಳನ್ನು ತಿಳಿದ ನೀನು ದುಃಖಿಸುವುದು ಯೋಗ್ಯವಲ್ಲ.\\}
\slcol{\Index{ಅಥ ಚೈನಂ ನಿತ್ಯಜಾತಂ} ನಿತ್ಯಂ ವಾ ಮನ್ಯಸೇ ಮೃತಮ್ ।\\
ತಥಾಪಿ ತ್ವಂ ಮಹಾಬಾಹೋ ನೈವಂ ಶೋಚಿತುಮರ್ಹಸಿ ॥ 26 ॥}
\cquote{ಒಂದು ವೇಳೆ ದೇಹದ ಮೂಲಕವಾದರೂ ಈ ಜೀವ ನಿರಂತರವಾಗಿ ಹುಟ್ಟುತ್ತಾನೆ, ಸಾಯುತ್ತಾನೆ ಎಂದು ತಿಳಿದರೂ ಅದಕ್ಕಾಗಿ ಹೀಗೆ ದುಃಖಿಸಬೇಕಿಲ್ಲ.\\}
\slcol{\Index{ಜಾತಸ್ಯ ಹಿ ಧ್ರುವೋ ಮೃತ್ಯು}ರ್ಧ್ರುವಂ ಜನ್ಮ ಮೃತಸ್ಯ ಚ ।\\
ತಸ್ಮಾದಪರಿಹಾರ್ಯೇऽರ್ಥೇ ನ ತ್ವಂ ಶೋಚಿತುಮರ್ಹಸಿ ॥ 27 ॥}
\cquote{ ಹುಟ್ಟಿದವನು ಸಾಯುವುದು ನಿಜ ಸತ್ತವನಿಗೆ ಜನ್ಮ ತಪ್ಪದು. ತಪ್ಪಿಸಲಾರದ ವಿಷಯಕ್ಕೆ ಚಿಂತಿಸಿ ಲಾಭವೇನು?\\}
\slcol{\Index{ಅವ್ಯಕ್ತಾದೀನಿ ಭೂತಾನಿ} ವ್ಯಕ್ತಮಧ್ಯಾನಿ ಭಾರತ ।\\
ಅವ್ಯಕ್ತನಿಧನಾನ್ಯೇವ ತತ್ರ ಕಾ ಪರಿದೇವನಾ ॥ 28 ॥}
\cquote{ ಅರ್ಜುನ, ಪ್ರಾಣಿಗಳೆಲ್ಲ ಕಾಣದ ಕಡೆಯಿಂದ ಬಂದಿವೆ.ನಡುವೆ ಒಂದಿಷ್ಟು ಕಾಲ ಕಾಣುತ್ತವೆ. ಮತ್ತೆ ಪುನಃ ಕಾಣದ ಕಡೆಗೆ ತರುಳುತ್ತವೆ.ಈ ವಿಷಯದಲ್ಲಿ ದುಃಖವೇಕೆ?\\}

\newpage
\begin{mananam}{\kanfont ಮನನ ಶ್ಲೋಕ - \textenglish{23, 24}}
\footnotesize \mananamfont ನನ್ನ ಬಗ್ಗೆ ನನಗಿರುವ ದೃಷ್ಟಿಕೋನದಂತೆಯೇ ನನ್ನ ನಡವಳಿಕೆ ಜೀವನ ಮತ್ತು ಅದರಾಚೆಗೂ ಇರುತ್ತದೆ. ಸವಾಲುಗಳು, ತೊಂದರೆಗಳು, ನೋವು ಮತ್ತು ಅಂತಿಮವಾಗಿ ಸಾವಿನ ಭಯದಿಂದ ಜೀವನದ ಸನ್ನಿವೇಶಗಳನ್ನು ಎದುರಿಸಲು ನಾನು ಹೆದರುತ್ತೀನೆಯೇ? ಅಥವಾ ನನ್ನ ಮತ್ತು ಇತರರ ಬಗ್ಗೆ ನನ್ನ ದೃಷ್ಟಿಕೋನ ಅಚಲವಾಗಿದೆಯೇ? ನಾನು ನನ್ನ ದೇಹಕ್ಕೆ ಮಾತ್ರ ಸೀಮಿತ ಎಂದು ಗುರುತಿಸಿಕೊಳ್ಳುತ್ತೇನಾ?
\end{mananam}
\WritingHand\enspace\textbf{ಆತ್ಮ ವಿಮರ್ಶೆ}
\begin{inspiration}{\kanfont ಸ್ಪೂರ್ತಿ}
\footnotesize \mananamfont ಸ್ಥೂಲ ವಸ್ತುಗಳು ಸ್ಥೂಲಅಂಶಗಳ ಮೇಲೆ ಪರಿಣಾಮ ಬೀರಬಹುದು ಆದರೆ ಸೂಕ್ಷ್ಮಅಂಶಗಳಿಗಲ್ಲ. ಹಾದು ಹೋಗುವ ಗಾಳಿ ಅಥವಾ ಬೆಂಕಿಯು ಆಕಾಶದ ಮೇಲೆ ಯಾವುದೇ ಪರಿಣಾಮ ಬೀರುವುದಿಲ್ಲ. ದೈಹಿಕ ಅಥವಾ ಮಾನಸಿಕ ರೂಪಾಂತರಗಳಿಂದ ಆತ್ಮಕ್ಕೆ ಏನೂ ಪರಿಣಾಮವಾಗುವುದಿಲ್ಲ.
\end{inspiration}
\newpage

\begin{mananam}{\kanfont ಮನನ ಶ್ಲೋಕ - \textenglish{27}}
\footnotesize \mananamfont ವಯಸ್ಸಾಗುತ್ತಿರುವುದು ನನಗೆ ಭಯ ಉಂಟು ಮಾಡುತ್ತಿದೆಯಾ? ಬದಲಾವಣೆಗಳು ನನಗೆ ಆತಂಕ ತರುತ್ತಿದೆಯಾ? ಜೀವನದ ಬದಲಾವಣೆಗಳನ್ನು ನಾನು ಶಾಂತ ರೀತಿಯಿಂದ ಸ್ವೀಕರಿಸಬಹುದೇ? ಬದಲಾಗುತ್ತಿರುವ ಈ ದೇಹದ ಲಕ್ಷಣ ಮತ್ತು ಸುತ್ತಲಿನ ಪರಿಸರದ ಬದಲಾವಣೆಗೆ ನಾನು ಸ್ವೀಕಾರ ಮನೋಭಾವ ತರಬಹುದಾ? 
\end{mananam}
\WritingHand\enspace\textbf{ಆತ್ಮ ವಿಮರ್ಶೆ}
\begin{inspiration}{\kanfont ಸ್ಪೂರ್ತಿ}
\footnotesize \mananamfont ಈ ಸಮಯವೂ ಕಳೆದು ಹೋಗವುದು ಎಂದು ಒಂದು ಪುರಾತನ ಗಾದೆ ಇದೆ. ಒಳ್ಳೆಯ ಸಮಯಗಳಲ್ಲಾಗಲಿ ಅಥವಾ ಕಷ್ಟದ ಸಮಯದಲ್ಲಾಗಲಿ, ಈ ಗಾದೆಯನ್ನು ಅಳಪಡಿಸಿಕೊಳ್ಳುವುದರಿಂದ ನಮ್ಮಲ್ಲಿ ಸ್ವೀಕಾರ ಮನೋಭಾವ ಮತ್ತು ಎಲ್ಲವೂ ಒಳ್ಳೆಯದಾಗುವುದು ಎಂಬ ಮನೋಭಾವ ಉಂಟಾಗುವುದು.
\end{inspiration}
\newpage

\slcol{\Index{ಆಶ್ಚರ್ಯವತ್ಪಶ್ಯತಿ ಕಶ್ಚಿದೇನ}ಮಾಶ್ಚರ್ಯ-\\ವದ್ವದತಿ ತಥೈವ ಚಾನ್ಯಃ ।\\
ಆಶ್ಚರ್ಯವಚ್ಚೈನಮನ್ಯಃ ಶೃಣೋತಿ \\ಶ್ರುತ್ವಾಪ್ಯೇನಂ ವೇದ ನ ಚೈವ ಕಶ್ಚಿತ್ ॥ 29 ॥}
\cquote{ಈ ಆತ್ಮನನ್ನು ಒಬ್ಬಾನೊಬ್ಬನು ಆಶ್ಚರ್ಯವಾಗಿ ನೋಡುತ್ತಾನೆ.ಮತ್ತೊಬ್ಬನು ಆಶ್ಚರ್ಯವಾಗಿ ಹೇಳುತ್ತಾನೆ. ಮತ್ತೊಬ್ಬನು ಆಶ್ಚರ್ಯವಾಗಿ ಕೇಳುತ್ತಾನೆ.ಕೇಳಿದರೂ ಈ ಆತ್ಮನನ್ನು ಯಾರೂ ತಿಳಿಯಲಾರರು.\\}
\slcol{\Index{ದೇಹೀ ನಿತ್ಯಮವಧ್ಯೋऽಯಂ} ದೇಹೇ zಸರ್ವಸ್ಯ ಭಾರತ ।\\
ತಸ್ಮಾತ್ಸರ್ವಾಣಿ ಭೂತಾನಿ ನ ತ್ವಂ ಶೋಚಿತುಮರ್ಹಸಿ ॥ 30 ॥}
\cquote{ಅರ್ಜುನ,ಎಲ್ಲರ ದೇಹದಲ್ಲಿರುವ ಈ ಆತ್ಮ ತತ್ವ ಕೊಲ್ಲಬರುವಂಥ ವಸ್ತುವಲ್ಲ. ಆದ್ದರಿಂದ ಯಾವ ಪ್ರಾಣಿಯ ಬಗೆಗೂ ನೀನು ವ್ಯಥೆಪಡುವ ಕಾರಣವಿಲ್ಲ.\\}
\slcol{\Index{ಸ್ವಧರ್ಮಮಪಿ ಚಾವೇಕ್ಷ್ಯ} ನ ವಿಕಂಪಿತುಮರ್ಹಸಿ ।\\
ಧರ್ಮ್ಯಾದ್ಧಿ ಯುದ್ಧಾಚ್ಛ್ರೇಯೋऽನ್ಯತ್ಕ್ಷತ್ರಿಯಸ್ಯ ನ ವಿದ್ಯತೇ ॥ 31 ॥}
\cquote{ಯುದ್ಧವು ನಿನ್ನ ಸಹಜ ಧರ್ಮವೆಂಬುದನ್ನು ನೋಡಿಯಾದರೂ ನೀನು ಕಂಗೆಡಬಾರದು. ಕ್ಷತ್ರಿಯನಿಗೆ ಧರ್ಮಯುದ್ಧಕ್ಕಿಂತ ಬೇರೆ ಶ್ರೇಯಸ್ ಇಲ್ಲ.\\}
\slcol{\Index{ಯದೃಚ್ಛಯಾ ಚೋಪಪನ್ನಂ} ಸ್ವರ್ಗದ್ವಾರಮಪಾವೃತಮ್ ।\\
ಸುಖಿನಃ ಕ್ಷತ್ರಿಯಾಃ ಪಾರ್ಥ ಲಭಂತೇ ಯುದ್ಧಮೀದೃಶಮ್ ॥ 32 ॥}
\cquote{ಅರ್ಜುನ, ತಾನಾಗಿ ಒದಗಿ ಬಂದ ಇಂತಹ ಯುದ್ಧವೆಂದರೆ ತೆರೆದಿಟ್ಟ ಸ್ವರ್ಗದ ಬಾಗಿಲು. ಇಂಥ ಯುದ್ಧವನ್ನು ಪುಣ್ಯಶಾಲಿಗಳಾದ ಕ್ಷತ್ರಿಯರು ಪಡೆಯುತ್ತಾರೆ.\\}
\slcol{\Index{ಅಥ ಚೇತ್ತ್ವಮಿಮಂ ಧರ್ಮ್ಯಂ} ಸಂಗ್ರಾಮಂ ನ ಕರಿಷ್ಯಸಿ ।\\
ತತಃ ಸ್ವಧರ್ಮಂ ಕೀರ್ತಿಂ ಚ ಹಿತ್ವಾ ಪಾಪಮವಾಪ್ಸ್ಯಸಿ ॥ 33 ॥}
\cquote{ನೀನು ಈ ಧರ್ಮ ಯುದ್ಧವನ್ನು ಮಾಡದೆ ಬಿಟ್ಟರೆ ಸ್ವಧರ್ಮಭ್ರಷ್ಟನೂ ಕೀರ್ತಿಭ್ರಷ್ಟನೂ ಆಗಿ ಪಾಪಕ್ಕೆ ಗುರಿಯಾಗುವೆ.\\}
\slcol{\Index{ಅಕೀರ್ತಿಂ ಚಾಪಿ ಭೂತಾನಿ} ಕಥಯಿಷ್ಯಂತಿ ತೇऽವ್ಯಯಾಮ್ ।\\
ಸಂಭಾವಿತಸ್ಯ ಚಾಕೀರ್ತಿರ್ಮರಣಾದತಿರಿಚ್ಯತೇ ॥ 34 ॥}
\cquote{ನಿನ್ನ ಅಪಕೀರ್ತಿಯನ್ನು ಜನರು ಅನಂತಕಾಲದ ವರೆಗೆ ಆಡಿಕೊಳ್ಳುತ್ತಾರೆ.ಮರ್ಯಾದಸ್ತನಿಗೆ ಅಪನಿಂದನೆಯು ಮರಣಕ್ಕಿಂತ ಕೀಳಾದದ್ದು.\\}
\slcol{\Index{ಭಯಾದ್ರಣಾದುಪರತಂ} ಮಂಸ್ಯಂತೇ ತ್ವಾಂ ಮಹಾರಥಾಃ ।\\
ಯೇಷಾಂ ಚ ತ್ವಂ ಬಹುಮತೋ ಭೂತ್ವಾ ಯಾಸ್ಯಸಿ ಲಾಘವಮ್ ॥ 35 ॥}
\cquote{ನಿನ್ನನ್ನು, ಭಯದಿಂದ ಯುದ್ಧವನ್ನು ಬಿಟ್ಟವನೆಂದು ಈ ಕ್ಷತ್ರಿಯ ವೀರರು ತಿಳಿಯುತ್ತಾರೆ. ಇಲ್ಲಿಯವರೆಗೆ ನಿನ್ನನ್ನು ಗೌರವದಿಂದ ನೋಡಿದವರೇ ಈಗ ಹಗುರಾಗಿ ನೋಡುವರು.\\}

\begin{mananam}{\kanfont ಮನನ ಶ್ಲೋಕ - \textenglish{29}}
\footnotesize \mananamfont ನಾನು ನಮ್ಮ ಜೀವನವನ್ನು ಹೊಸ ದೃಷ್ಟಿಕೋನದಿಂದ ನೋಡಬಹುದಾ? ನಾನು ಪಕ್ಷಪಾತಿಯಾಗಿದ್ದೀನಾ? ಎಲ್ಲದರ ಬಗ್ಗೆ ವಿಮರ್ಶಾತ್ಮಕವಾಗಿ ನಿರ್ಣಯ ತೆಗೆದುಕೊಳ್ಳುತ್ತೇನಾ?ನನ್ನ ಈ ಕ್ಷಣದ ಅನುಭವಕ್ಕೂ ಹಳೆಯ  ವಿಮರ್ಶಾತ್ಮಕ ಬಣ್ಣ ಬಳೆಯುತ್ತೀನಾ?
\end{mananam}
\WritingHand\enspace\textbf{ಆತ್ಮ ವಿಮರ್ಶೆ}
\begin{inspiration}{\kanfont ಸ್ಪೂರ್ತಿ}
\footnotesize \mananamfont ಮಗು ಎಲ್ಲವನ್ನೂ ಆಶ್ಚರ್ಯ ಮತ್ತು ಕುತೂಹಲದಿಂದ ನೋಡುತ್ತದೆ. ಅದು ಎಲ್ಲವನ್ನು ಹೊಸತನ, ತಾಜಾತನ ಮತ್ತು ಸಂತೋಷದಿಂದ ಅನುಭವಿಸುತ್ತದೆ. ಆದರೆ ಅದು ಬೆಳೆದಂತೆ ತಾನು ಏನನ್ನು ಮಾಡಬೇಕು, ಕೇಳಬೇಕು ಎಂದು ಬಯಸುವ ಕಡೆಗೆ ಪಕ್ಷಪಾತಭಾವನೆ ಬೆಳೆಸಿಕೊಳ್ಳುವುದು. ಅನಂತರ ಶೀಘ್ರದಲ್ಲಿಯೇ ಜೀವನವು ಬೇಸರ, ಮಂದ ನಿರಾಶದಾಯಕವಾಗುತ್ತದೆ. ನಿಮ್ಮೊಳಗಿನ ಮಗುವನ್ನು ಪುನರ್ಜೀವಗೊಳಿಸಿ. ಜೀವನದ ಪ್ರತಿ ಕ್ಷಣವನ್ನು ತಾಜಾತನ ಮತ್ತು ನಿಷ್ಕಲ್ಮಶ ಮನಸ್ಸಿನಿಂದ ಅನುಭವಿಸಿ.
\end{inspiration}
\newpage

\begin{mananam}{\kanfont ಮನನ ಶ್ಲೋಕ - \textenglish{31}}
\footnotesize \mananamfont ನನ್ನ ಜೀವನದಲ್ಲಿ ಅನ್ಯಾಯವಾಗಿ ನನ್ನ ಮೇಲೆ ಕೆಲವು ಕರ್ತವ್ಯಗಳು ಹೇರಲಾಗಿದೆ ಎಂದು ಭಾವಿಸುತ್ತೇನಾ? ಆದರೆ ಈ ಕರ್ತವ್ಯಗಳನ್ನು ನಿಭಾಯಿಸಲು ಕಲಿತರೆ ನನ್ನ ಜೀವನ ಪ್ರಕಾಶಮಯವಾಗಬಹುದಾ? ನನ್ನ ಜೀವನದಲ್ಲಿ ಬರುವ ಕರ್ತವ್ಯಗಳನ್ನು ಗೊಣಗದೆ,ದೂರದೆ, ಸಂಪೂರ್ಣ ಸ್ವೀಕಾರ ಮನೋಭಾವದಿಂದ ಮಾಡಲು ಕಲಿತರೆ ಒಳ್ಳೆಯದಲ್ಲವೇ?
\end{mananam}
\WritingHand\enspace\textbf{ಆತ್ಮ ವಿಮರ್ಶೆ}
\begin{inspiration}{\kanfont ಸ್ಪೂರ್ತಿ}
\footnotesize \mananamfont ನಮಗೆ ಇಷ್ಟವಾಗದ ಕರ್ತವ್ಯಗಳು ನಮಗೆ ಕೆಲವು ಅವಕಾಶಗಳು ಮತ್ತು ಕೌಶಲಗಳೊಂದಿಗೆ ಸಜ್ಜುಗೊಳಿಸುತ್ತದೆ. ಪ್ರಕೃತಿಯ ನಿಯಮವೆಂದರೆ ಎಂದಿಗೂ ಯಾರಿಗೂ ಅನ್ಯಾಯವಾಗುವುದಿಲ್ಲ ಮತ್ತು ನಮ್ಮ ಯಾವುದೇ ಪ್ರಯತ್ನಕ್ಕೆ ಪ್ರತಿಫಲ ದೊರಕದೇ ಇರುವುದಿಲ್ಲ.
\end{inspiration}
\newpage

\begin{mananam}{\kanfont ಮನನ ಶ್ಲೋಕ - \textenglish{33}}
\footnotesize \mananamfont ಪರಮಸತ್ಯವನ್ನು ಪಡೆಯಲು ???
\end{mananam}
\begin{inspiration}{\kanfont ಸ್ಪೂರ್ತಿ}
\footnotesize \mananamfont ಪರಮಸತ್ಯವನ್ನು ಪಡೆಯಲು ಪೂರ್ಣ ಹೃದಯದಿಂದ ಬದ್ಧನಾಗಿರುವವನು ತ್ಯಾಗ ಮಾಡಿದರೆ ಸ್ವೀಕಾರಾರ್ಹವಾಗುವುದು. ಒಬ್ಬನು ತನ್ನ ರಾಷ್ಟ್ರಕ್ಕೋಸ್ಕರ ಅಥವಾ ಒಂದು ದೊಡ್ಡ ಸಮುದಾಯಕೋಸ್ಕರ ಸೇವೆ ಸಲ್ಲಿಸುವುದರಲ್ಲಿ ನಿರತನಾಗಿದ್ದಕ್ಕೆ ತನ್ನ ಕುಟುಂಬದ ಜವಾಬ್ದಾರಿಯನ್ನು ಹೊರಲು ವಿಫಲನಾದರೆ ಇದು ಸ್ವೀಕಾರಾರ್ಹವಾಗಿದೆ. ಆದರೆ ಸೋಮಾರಿತನ, ತನ್ನ ಆಕಾಂಕ್ಷೆ ಗೋಸ್ಕರ ಮತ್ತು ನಕರಾತ್ಮಕತೆ ಮತ್ತು ಭಯದಿಂದ ಹೊರಬರಲು ತನ್ನ ಕುಟುಂಬ, ಸಮಾಜದಿಂದ ಆತ್ಮಕ್ಕೆ ಕರ್ತವ್ಯ ಮತ್ತು ಜವಾಬ್ದಾರಿ ಯನ್ನು ತ್ಯಜಿಸುವುದು ತನಗೂ ಮತ್ತು ಬೇರೆಯವರಿಗೂ ಮಾಡುವ ಅತ್ಯಂತ ಕೆಟ್ಟ ಕೆಲಸ ವಾಗುವುದು.
\end{inspiration}
\newpage


\slcol{\Index{ಅವಾಚ್ಯವಾದಾಂಶ್ಚ ಬಹೂನ್ವ}ದಿಷ್ಯಂತಿ ತವಾಹಿತಾಃ ।\\
ನಿಂದಂತಸ್ತವ ಸಾಮರ್ಥ್ಯಂ ತತೋ ದುಃಖತರಂ ನು ಕಿಮ್ ॥ 36 ॥}
\cquote{ಶತ್ರುಗಳು ನಿನ್ನ ಪರಾಕ್ರಮವನ್ನು ನಿಂದಿಸಿ ಮಾತನಾಡುವರು.ಇದಕ್ಕಿಂತ ಹೆಚ್ಚಿನ ದುಃಖ ಯಾವುದು?\\}
\slcol{\Index{ಹತೋ ವಾ ಪ್ರಾಪ್ಸ್ಯಸಿ} ಸ್ವರ್ಗಂ ಜಿತ್ವಾ ವಾ ಭೋಕ್ಷ್ಯಸೇ ಮಹೀಮ್ ।\\
ತಸ್ಮಾದುತ್ತಿಷ್ಠ ಕೌಂತೇಯ ಯುದ್ಧಾಯ ಕೃತನಿಶ್ಚಯಃ ॥ 37 ॥}
\cquote{ಸತ್ತರೆ ಸ್ವರ್ಗವನ್ನು ಸೇರುವೆ, ಗೆದ್ದರೆ ಭೂಮಿಯನ್ನು ಆಳುವೆ. ಆದ್ದರಿಂದ ಅರ್ಜುನ ಕಾದುವುದಕ್ಕೆ ಮನಸ್ಸು ಗಟ್ಟಿಮಾಡಿಕೊಂಡು ಏಳು.\\}
\slcol{\Index{ಸುಖದುಃಖೇ ಸಮೇ ಕೃತ್ವಾ} ಲಾಭಾಲಾಭೌ ಜಯಾಜಯೌ ।\\
ತತೋ ಯುದ್ಧಾಯ ಯುಜ್ಯಸ್ವ ನೈವಂ ಪಾಪಮವಾಪ್ಸ್ಯಸಿ ॥ 38 ॥}
\cquote{ಸುಖದುಃಖಗಳನ್ನು, ಲಾಭನಷ್ಟಗಳನ್ನು, ಜಯಾಪಜಯಗಳನ್ನು ಸಮನವಾಗಿ ತಿಳಿದು ಯುದ್ಧವನ್ನು ಮಾಡು. ಹಾಗಾದರೆ ಪಾಪಗಳು ನಿನ್ನನ್ನು ಅಂಟಲಾರವು.\\}

\newpage
\begin{mananam}{\kanfont ಮನನ ಶ್ಲೋಕ - \textenglish{38}}
\footnotesize \mananamfont ಜೀವನದಲ್ಲಿ ನಾನು ಎದುರಿಸುವ ಯಾವುದೇ ಸವಾಲಿನ ಬಗ್ಗೆ ನನ್ನ ವರ್ತನೆ ಏನು? ನಾನು ಯಾವುದೇ ತರಹದ ಫಲಿತಾಂಶದ ಕಡೆಗೆ ಸಮಚಿತ್ತನಾಗಿದ್ದೇನೆಯೇ? ನನ್ನ ಅತ್ಯುತ್ತಮ ಪ್ರಯತ್ನದ ಹೊರತಾಗಿಯೂ ಒಳ್ಳೆಯ ಫಲಿತಾಂಶ ಬಂದಾಗ ಅದರ ಬಗ್ಗೆ ಮೋಹವು ಮತ್ತು ಫಲಿತಾಂಶಗಳಿಗೆ ಮಾನಸಿಕವಾಗಿ ವಿಮುಕನಾಗಿದ್ದೇನೆಯೇ? ಗೆಲುವನ್ನು ಬಯಸದೆ ಸೋಲನ್ನು ತಿರಸ್ಕರಿದೆ ಲಾಭವನ್ನು ಹುಡುಕುವ ಮತ್ತು ನಷ್ಟವನ್ನು ತಪ್ಪಿಸುವ ಪ್ರೇರಣೆ ಇಲ್ಲದೆ ಜೀವನದಲ್ಲಿ  ಕಾರ್ಯನಿರ್ವಹಿಸಲು ಕಲಿಯಬಹುದೇ?
\end{mananam}
\begin{inspiration}{\kanfont ಸ್ಪೂರ್ತಿ}
\footnotesize \mananamfont ಯಾವಾಗಲೂ ಗೆಲುವನ್ನು ಬಯಸುವುದು ಮತ್ತು ಸಂತೋಷವಾಗಿರಲು ಬಯಸುವುದು ಎಲ್ಲರಲ್ಲಿ ಸಹಜವಾಗಿರುವ ಒಲವು. ಹಾಗೆಯೇ ಅಹಿತಕರವಾದದ್ದನ್ನು ತಪ್ಪಿಸುವುದು ಮತ್ತು ಎಂದಿಗೂ ಸೋಲನ್ನು ಬಯಸದೇ ಇರುವುದು ಪ್ರವೃತ್ತಿಯಾಗಿದೆ.ಆದರೆ ನಿಜವಾದ ಸ್ವಾತಂತ್ರ್ಯ ಹೊಂದಿದ ವ್ಯಕ್ತಿಗೆ ಎಲ್ಲಾ ಕ್ರಿಯೆಗಳು ನೀರಿನಲ್ಲಿ ರೇಖೆಯನ್ನು ಎಳೆಯುವಂತಿದೆ. ಅವನು ಮಾನಸಿಕವಾಗಿ ಕಳಂಕರಹಿತ ಮತ್ತು ಯಾವುದೇ ತರಹದ ಕರ್ಮವು ಅವನನ್ನು ಬಂಧಿಸುವುದಿಲ್ಲ.
\end{inspiration}
\newpage

\slcol{\Index{ಏಷಾ ತೇऽಭಿಹಿತಾ ಸಾಂಖ್ಯೇ} ಬುದ್ಧಿರ್ಯೋಗೇ ತ್ವಿಮಾಂ ಶೃಣು ।\\
ಬುದ್ಧ್ಯಾ ಯುಕ್ತೋ ಯಯಾ ಪಾರ್ಥ ಕರ್ಮಬಂಧಂ ಪ್ರಹಾಸ್ಯಸಿ ॥ 39 ॥}
\cquote{ಅರ್ಜುನ, ಆತ್ಮನ ವಿಚಾರವಾಗಿ ಈ ತಿಳುವಳಿಕೆಯನ್ನು ನಿನಗೆ ಹೇಳಿದ್ದಾಯಿತು ಈಗ ಅದರ ಉಪಾಯದ ಬಗ್ಗೆ ಕೇಳು. ಅರ್ಜುನ ನೀನು ಇದನ್ನು ತಿಳಿಯುವುದರಿಂದ ಕರ್ಮದ ಕಟ್ಟನ್ನು ಕಳಚಿಕೊಳ್ಳುತ್ತಿ.\\}
\slcol{\Index{ನೇಹಾಭಿಕ್ರಮನಾಶೋऽಸ್ತಿ} ಪ್ರತ್ಯವಾಯೋ ನ ವಿದ್ಯತೇ ।\\
ಸ್ವಲ್ಪಮಪ್ಯಸ್ಯ ಧರ್ಮಸ್ಯ ತ್ರಾಯತೇ ಮಹತೋ ಭಯಾತ್ ॥ 40 ॥}
\cquote{ಇದರ ಆರಂಭ ಮಾತ್ರವೂ ವ್ಯರ್ಥವಲ್ಲ. ಇದರಲ್ಲಿ ದೋಷ ಉಂಟಾಗುವುದಿಲ್ಲ. ಈ ಧರ್ಮದ ಅಲ್ಪಾಚರಣೆ ಕೂಡ ಹಿರಿಯ ಪಾತಕದಿಂದ ಪಾರು ಮಾಡುತ್ತದೆ.\\}
\slcol{\Index{ವ್ಯವಸಾಯಾತ್ಮಿಕಾ ಬುದ್ಧಿ}ರೇಕೇಹ ಕುರುನಂದನ ।\\
ಬಹುಶಾಖಾ ಹ್ಯನಂತಾಶ್ಚ ಬುದ್ಧಯೋऽವ್ಯವಸಾಯಿನಾಮ್ ॥ 41 ॥}
\cquote{ಅರ್ಜುನ,ಈ ಸಾಧನಗಳಲ್ಲಿ ನೆಲೆಗೆ ನಿಂತ ಬುದ್ಧಿಯು ಒಂದೇ ಮುಖವಾಗಿರುವುದು. ನೆಲೆಗೆ ನಿಲ್ಲದವರ ಬುದ್ಧಿಯು ಅನೇಕ ಕೊಂಬೆಗಳುಳ್ಳದಾಗಿ ಬಗೆಬಗೆಯಾಗಿರುವುದು. \\}
\slcol{\Index{ಯಾಮಿಮಾಂ ಪುಷ್ಪಿತಾಂ} ವಾಚಂ ಪ್ರವದಂತ್ಯವಿಪಶ್ಚಿತಃ ।\\
ವೇದವಾದರತಾಃ ಪಾರ್ಥ ನಾನ್ಯದಸ್ತೀತಿ ವಾದಿನಃ ॥ 42 ॥}
\cquote{ಅರ್ಜುನ, ದಡ್ಡರು ವೇದದ ಮೇಲ್ನೋಟಕ್ಕೆ ಕಾಣುವ ಹೂವಿನಂತ ಮಾತಿಗೆ ಮರುಳಾಗುತ್ತಾರೆ. ಅದರ ಆಚೆಗಿರುವ ಭಗವತತ್ವವೆಂಬ ಹಣ್ಣು ಅವರಿಗೆ ಕಾಣಿಸದು. ಅದಕ್ಕೆಂದೇ ಅವರು ಅದನ್ನು ನಿರಾಕರಿಸಿಬಿಡುತ್ತಾರೆ.\\}
\slcol{\Index{ಕಾಮಾತ್ಮಾನಃ ಸ್ವರ್ಗಪರಾ} ಜನ್ಮಕರ್ಮಫಲಪ್ರದಾಮ್ ।\\
ಕ್ರಿಯಾವಿಶೇಷಬಹುಲಾಂ ಭೋಗೈಶ್ವರ್ಯಗತಿಂ ಪ್ರತಿ ॥ 43 ॥}
\cquote{ಅವರು ಬಯಕೆಯ ಬೆನ್ನು ಹತ್ತಿದವರು. ಸ್ವರ್ಗವೇ ಪುರುಷಾರ್ಥ ಎಂದು ಭ್ರಮಿಸಿದವರು. ನಮ್ಮನ್ನು ಹುಟ್ಟು ಸಾವುಗಳ ಸುಳಿಯಲ್ಲಿ ಸಿಕ್ಕಿಸುವ ಕರ್ಮಕಾಂಡದ ಕ್ಷಣಿಕ ಭೋಗಭಾಗ್ಯಗಳಿಗೆ ಮರುಳಾದವರು.\\}
\slcol{\Index{ಭೋಗೈಶ್ವರ್ಯಪ್ರಸಕ್ತಾನಾಂ} ತಯಾಪಹೃತಚೇತಸಾಮ್ ।\\
ವ್ಯವಸಾಯಾತ್ಮಿಕಾ ಬುದ್ಧಿಃ ಸಮಾಧೌ ನ ವಿಧೀಯತೇ ॥ 44 ॥}
\cquote{ಇಂದ್ರಿಯ ಬೋಗ ಮತ್ತು ಸಂಪತ್ತುಗಳಲ್ಲಿ ಆಸಕ್ತರಾದ ಇಂಥವರು ಫಲಸ್ತುತಿಗಳ ಮಾತಿನ ಸೆಳೆತಕ್ಕೆ ಮರುಳಾಗುತ್ತಾರೆ. ಅಂತವರ ಮನಸ್ಸಿನಲ್ಲಿ ನೆಲೆ ನಿಂತ ತತ್ವದ ತಿಳುವಳಿಕೆ ಉಂಟಾಗುವುದಿಲ್ಲ.\\}
\slcol{\Index{ತ್ರೈಗುಣ್ಯವಿಷಯಾ ವೇದಾ} ನಿಸ್ತ್ರೈಗುಣ್ಯೋ ಭವಾರ್ಜುನ ।\\
ನಿರ್ದ್ವಂದ್ವೋ ನಿತ್ಯಸತ್ತ್ವಸ್ಥೋ ನಿರ್ಯೋಗಕ್ಷೇಮ ಆತ್ಮವಾನ್ ॥ 45 ॥}
\cquote{ಅರ್ಜುನಾ, ವೇದಗಳು ತ್ರಿಗುಣ ರೂಪವಾದ ಸಂಸಾರವನ್ನು ಹೇಳುತ್ತವೆ. ನೀನು ತ್ರಿಗುಣತೀತನು ದ್ವಂದ್ವರಹಿತನು ಆಗು. ಶುದ್ಧ ಸತ್ವವನ್ನು ಆಶ್ರಯಿಸುವವನಾಗಿಯೂ ಯೋಗ ಕ್ಷೇಮಗಳ ಚಿಂತೆ ಇಲ್ಲದವನಾಗಿ  ಆಗು. ಆತ್ಮನಿಷ್ಟನಾಗಿರು.}

\newpage
\begin{mananam}{\kanfont ಮನನ ಶ್ಲೋಕ - \textenglish{45}}
\footnotesize \mananamfont ಒಳ್ಳೆಯದು, ಕೆಟ್ಟದ್ದು, ಸುಂದರ, ಕೊಳಕು, ಬಿಳಿ ಮತ್ತು ಕಪ್ಪು ಈ ಜೋಡಿ ವಿರುದ್ಧ ಪದಗಳಿಂದ ನನ್ನ ಜೀವನದ ದೃಷ್ಟಿಕೋನವು ಕಳಂಕಿತವಾಗಿದೆಯೇ? ನಾನು ಯಾವಾಗಲೂ ಇತರರ ಬಗ್ಗೆ ಮತ್ತು ನನ್ನ ಬಗ್ಗೆ ವಿಮರ್ಶಾತ್ಮಕವಾಗಿದ್ದೇನೆಯೇ?ನನ್ನೊಳಗೆ ಮತ್ತು ಇತರರೊಂದಿಗೆ ಸಂಘರ್ಷದ ಮೂಲವಾಗಿರುವ ವಿಪರೀತ ದೃಷ್ಟಿಕೋನಗಳಿಗೆ ನಾನು ಅಂಟಿಕೊಂಡಿದ್ದೇನೆಯೇ? ಈ ಎಲ್ಲಾ ಅನಿಸಿಕೆಗಳು ಮತ್ತು ವಿಮರ್ಶೆಗಳಿಂದ ಮೇಲೆ ಬರುವಷ್ಟು ಧೈರ್ಯವಿದೆಯೇ? ಎಲ್ಲದರಲ್ಲೂ ತಟಸ್ಥವಾಗಿರದೆ ತೆರೆದ ಮನಸ್ಸು ಮತ್ತು ಸ್ವೀಕಾರ ಮನೋಭಾವದಿಂದ ಹೆಚ್ಚಿನ ಆಯಾಮಕ್ಕೆ ತೆರೆದುಕೊಳ್ಳಬಲ್ಲೆನೇ?
\end{mananam}
\begin{inspiration}{\kanfont ಸ್ಪೂರ್ತಿ}
\footnotesize \mananamfont ಎಲ್ಲಾ ಒತ್ತಡದ ಮತ್ತು ಆತಂಕಗಳಿಂದ ನಿಮ್ಮನ್ನು ಮುಕ್ತಿಗೊಳಿಸಲು ತಕ್ಷಣದ ಮಾರ್ಗವೆಂದರೆ ನಿಮ್ಮ ದಿನಚರಿಯದಲ್ಲಿ ಕೆಲವು ನಿಮಿಷಗಳ ಕಾಲ ದೇಹಕ್ಕೆ ವಿಶ್ರಾಂತಿ ಕೊಟ್ಟು ಮನಸ್ಸಿಗೆ ಅದರ ನೈಸರ್ಗಿಕ ಸ್ಥಿತಿಯಲ್ಲಿರುವುದು ಅಂದರೆ ಏನನ್ನು ಮಾಡಲು ಬಯಸದೇ ಇರುವುದು ಮತ್ತು ಏನನ್ನು ಮಾಡದೇ ಇರುವುದು. ಈ ಸ್ಥಿತಿಯು ಧ್ಯಾನವೂ ಅಲ್ಲ ಅಥವಾ ನಿದ್ರಿಸುವುದು ಅಲ್ಲ.ಇದು ನಿಮ್ಮ ಆತ್ಮದೊಂದಿಗೆ ನೀವು ಇರುವ ಸ್ಥಿತಿಯು ಸಹಜವು ಮತ್ತು ಆಳವಾಗಿ ತೃಪ್ತವಾಗಿರುವುದಾಗಿದೆ. ಇದು ತ್ರಿಗುಣಗಳಾದ ಸತ್ವ, ರಜಸ್,ತಮಸ್ ನಿಂದ ಮುಕ್ತವಾಗಿದೆ. ಇವು ನಮ್ಮನ್ನು ಯಾವಾಗಲೂ ಹೊರಗೆ ತೊಡಗಿಸಿಟ್ಟುಕೊಂಡಿರುತ್ತದೆ.
\end{inspiration}
\newpage

\slcol{\Index{ಯಾವಾನರ್ಥ ಉದಪಾನೇ} ಸರ್ವತಃ ಸಂಪ್ಲುತೋದಕೇ ।\\
ತಾವಾನ್ಸರ್ವೇಷು ವೇದೇಷು ಬ್ರಾಹ್ಮಣಸ್ಯ ವಿಜಾನತಃ ॥ 46 ॥}
\cquote{ಬಾವಿಯಿಂದ ಆಗುವ ಪ್ರಯೋಜನ ಎಲ್ಲೆಡೆಯೂ ತುಂಬಿ ಹರಿಯುವ ಸಮುದ್ರದಿಂದ ಆಗಿಯೇ ಆಗುತ್ತದೆ. ಹಾಗೆಯೇ ವೇದದಲ್ಲಿ ಹೇಳಿರುವ ಎಲ್ಲಾ ಫಲಗಳು ಬ್ರಹ್ಮ ಜ್ಞಾನಿಗೆ ಸಿಕ್ಕೇ ಸಿಗುವುದು.\\}
\slcol{\Index{ಕರ್ಮಣ್ಯೇವಾಧಿಕಾರಸ್ತೇ ಮಾ} ಫಲೇಷು ಕದಾಚನ ।\\
ಮಾ ಕರ್ಮಫಲಹೇತುರ್ಭೂರ್ಮಾ ತೇ ಸಂಗೋऽಸ್ತ್ವಕರ್ಮಣಿ ॥ 47 ॥}
\cquote{ಕರ್ಮ ಮಾಡುವುದಷ್ಟೇ ನಿನ್ನ ಹಕ್ಕು. ಕರ್ಮಫಲದ ಮೇಲೆ ಹಕ್ಕು ಸಾಧಿಸಬೇಡ. ಫಲದ ಆಸೆಯಿಂದ ಕರ್ಮ ಮಾಡಲು ಬೇಡ. ಹಾಗೆಯೇ ಕರ್ಮ ತ್ಯಾಗದ ಕಡೆಗೂ ನಿನ್ನ ಒಲವು ಹರಿಯದಿರಲಿ.\\}
\slcol{\Index{ಯೋಗಸ್ಥಃ ಕುರು ಕರ್ಮಾಣಿ} ಸಂಗಂ ತ್ಯಕ್ತ್ವಾ ಧನಂಜಯ ।\\
ಸಿದ್ಧ್ಯಸಿದ್ಧ್ಯೋಃ ಸಮೋ ಭೂತ್ವಾ ಸಮತ್ವಂ ಯೋಗ ಉಚ್ಯತೇ ॥ 48 ॥}
\cquote{ಅರ್ಜುನ,ಯೋಗ ನಿಷ್ಠನಾಗಿ ಫಲಕ್ಕಾಗಿ ಆಸೆ ಮಾಡದೆ ಫಲ ದೊರೆತರೆ ಹಿಗ್ಗದೆ ಸಿಗದಿದ್ದರೆ ಕುಗ್ಗದೇ ಒಂದೇ ಭಾವದಿಂದ ಕರ್ಮವನ್ನು ಮಾಡು. ಈ ಸಮದೃಷ್ಟಿಯೇ ನಿಜವಾದ ಯೋಗ.\\}

\newpage
\begin{mananam}{\kanfont ಮನನ ಶ್ಲೋಕ - \textenglish{47,48}}
\footnotesize \mananamfont ಏನನ್ನಾದರೂ ಮಾಡುವಾಗ ಫಲಿತಾಂಶವನ್ನು ಎದುರು ನೋಡದೆ ಇರುವುದನ್ನು ಹಾಗೂ ಅಹಿತಕರ ಫಲಿತಾಂಶ ಬಂದರೂ ಕೂಡ ಸಮತ್ವ ಸ್ಥಿತಿಯಲ್ಲಿ ಇರಲು ಕಲಿಯಬಹುದೇ? ಹೆಚ್ಚಿನವರಂತೆ ನಾನು ಕೂಡ ಎಲ್ಲ ಸಂಬಂಧಗಳನ್ನು ವ್ಯವಹಾರಿಕ ದೃಷ್ಟಿಕೋನದಿಂದ ನೋಡುತ್ತೇನೆಯೇ? ಯಾರಿಂದಲೂ ಏನನ್ನು ತೆಗೆದುಕೊಳ್ಳಲು ಬಯಸದೆ ನಾನು ಕೊಡುವುದನ್ನು ಮಾತ್ರ ಕಲಿಯಬಹುದೇ? ತೆಗೆದುಕೊಳ್ಳದೆ ಕೊಡುವ ಈ ತತ್ವವನ್ನು ದಿನನಿತ್ಯ ಜೀವನದಲ್ಲಿ ಸಣ್ಣ ವಿಷಯಗಳಿಗೂ ಅಳವಡಿಸಲು ಅಭ್ಯಾಸ ಮಾಡಿಕೊಳ್ಳಬಹುದೇ? ಇದರಿಂದ ಈ ತತ್ವವನ್ನು ನನ್ನ ಜೀವನದಲ್ಲಿ ದೊಡ್ಡ ದೊಡ್ಡ ವಿಷಯಗಳಿಗೆ ಅಳವಡಿಸಿಕೊಳ್ಳಲು ಸಾಧ್ಯವಾಗಬಹುದೇ?
\end{mananam}
\begin{inspiration}{\kanfont ಸ್ಪೂರ್ತಿ}
\footnotesize \mananamfont ಮನಸ್ಸಿನ ಸಮತ್ವವನ್ನು ಸಾಧಿಸಲು ಎಲ್ಲಾ ನಿರೀಕ್ಷೆಗಳಿಂದ ಮನಸ್ಸನ್ನು ಶುದ್ಧೀಕರಿಸುವುದು ಅತ್ಯಗತ್ಯ. ದಿನನಿತ್ಯದ ಜೀವನದಲ್ಲಿ ಮನಸ್ಸಿನ ಈ ಸಮುತ್ವವನ್ನು ಕಾಪಾಡಿಕೊಳ್ಳುವುದು ಯೋಗ ಮತ್ತು ಈ ಸ್ಥಿತಿಯನ್ನು ಯಾರು ಸಾಧಿಸುತ್ತಾರೋ ಅವನೇ ಯೋಗಿ.
\end{inspiration}
\newpage

\slcol{\Index{ದೂರೇಣ ಹ್ಯವರಂ ಕರ್ಮ} ಬುದ್ಧಿಯೋಗಾದ್ಧನಂಜಯ ।\\
ಬುದ್ಧೌ ಶರಣಮನ್ವಿಚ್ಛ ಕೃಪಣಾಃ ಫಲಹೇತವಃ ॥ 49 ॥}
\cquote{ಅರ್ಜುನ, ಇಂತಹ ಜ್ಞಾನಮಾರ್ಗಕ್ಕಿಂತ ಫಲವನ್ನು ಬಯಸಿ ಮಾಡುವ ಕರ್ಮವು ಬಹು ಕೀಳು. ಅದರಿಂದ ಜ್ಞಾನ ಯೋಗವನ್ನು ಆಶ್ರಯಿಸು, ಫಲಕ್ಕಾಗಿ ಕರ್ಮ ಮಾಡುವವರು ಶೋಚನೀಯರು.\\}
\slcol{\Index{ಬುದ್ಧಿಯುಕ್ತೋ ಜಹಾತೀಹ} ಉಭೇ ಸುಕೃತದುಷ್ಕೃತೇ ।\\
ತಸ್ಮಾದ್ಯೋಗಾಯ ಯುಜ್ಯಸ್ವ ಯೋಗಃ ಕರ್ಮಸು ಕೌಶಲಮ್ ॥ 50 ॥}
\cquote{ಸಮತ್ವ ಬುದ್ಧಿಯುಕ್ತನು ಬದುಕಿರುವಾಗಲೇ ಪುಣ್ಯ, ಪಾಪ ಎರಡಕ್ಕೂ ಅತಿತನಾಗಬಲ್ಲನು. ಆದ್ದರಿಂದ ಆ ಯೋಗವನ್ನು ಆಶ್ರಯಿಸುವುದಕ್ಕೆ ಯತ್ನ 
ಮಾಡು.\\}

\newpage
\begin{mananam}{\kanfont ಮನನ ಶ್ಲೋಕ - \textenglish{50}}
\footnotesize \mananamfont ನನ್ನ ಬಾಹ್ಯಜೀವನದಲ್ಲಿ ಮಾತ್ರವಲ್ಲ ನನ್ನ ಆಂತರಿಕ ಜೀವನದಲ್ಲಿಯೂ ಯಶಸ್ವಿಯಾಗಲು ಅಗತ್ಯವಾದ ಕೌಶಲ್ಯಗಳನ್ನು ಹೊಂದಿದ್ದೇನೆಯೇ? ವಿಶೇಷವಾಗಿ ಕೆಲಸದ ಒತ್ತಡ ಇರುವಾಗ ನನ್ನ ಭಾವನೆಗಳನ್ನು ನಿಭಾಯಿಸುವ ಸಾಮರ್ಥ್ಯವಿದೆಯೇ? ನನ್ನ ಎಲ್ಲಾ ಸಂಬಂಧಗಳಲ್ಲಿ ನಾನು  ಸಮರಸ್ಯದಿಂದ ಇರಲು ಸಾಧ್ಯವೇ? ನನ್ನ ಸಮತೋಮುಖ ಯೋಗಕ್ಷೇಮಕ್ಕಾಗಿ ಮಾಡುವ ಪ್ರಯತ್ನಗಳಲ್ಲಿ ತಾಳ್ಮೆ ಮತ್ತು ನಿರಂತರತೆಯನ್ನು ಹೊಂದಿರಲು ಸಾಧ್ಯವೇ? ಗೀತೆಯ ದೃಷ್ಟಿಕೋನದಂತೆ ಈ ಲೌಕಿಕ ಲಾಭ ನಷ್ಟವನ್ನು ಮೀರಿ ಮಾನಸಿಕ ಸಮತ್ವವು ನನಗಿದೆಯೇ?
\end{mananam}
\begin{inspiration}{\kanfont ಸ್ಪೂರ್ತಿ}
\footnotesize \mananamfont ಯೋಗಭ್ಯಾಸದ ಉದ್ದೇಶವು ಜೀವನಕ್ಕೆ ಅಗತ್ಯವಾದ ಕೌಶಲ್ಯಗಳೊಂದಿಗೆ ನಮ್ಮನ್ನು ಸಜ್ಜುಗೊಳಿಸುವುದು. ಒಂದು ವಾಹನವನ್ನು ಓಡಿಸಲು ಕೌಶಲ್ಯಗಳು ಹೇಗೆ ಬೇಕೋ ಹಾಗೆಯೇ ಜೀವನ ನಿರ್ವಹಿಸಲು ನಮಗೆ ಜೀವನ ಕೌಶಲ್ಯಗಳು ಬೇಕಾಗುತ್ತವೆ.ಕೌಶಲ್ಯದಿಂದ ಕೆಲಸ ಮಾಡುವುದರಿಂದ, ಕರ್ಮಯೋಗಿ ಕ್ರಿಯೆಗಳಿಂದ ಪ್ರೇರಿತವಾದ ಬಂಧನದಿಂದ ಮುಕ್ತನಾಗುತ್ತಾನೆ.
\end{inspiration}
\newpage

\slcol{\Index{ಕರ್ಮಜಂ ಬುದ್ಧಿಯುಕ್ತಾ} ಹಿ ಫಲಂ ತ್ಯಕ್ತ್ವಾ ಮನೀಷಿಣಃ ।\\
ಜನ್ಮಬಂಧವಿನಿರ್ಮುಕ್ತಾಃ ಪದಂ ಗಚ್ಛಂತ್ಯನಾಮಯಮ್ ॥ 51 ॥}
\cquote{ಜ್ಞಾನಿಗಳು ಕರ್ಮದ ಫಲವನ್ನು ಬಯಸದೆ ಜ್ಞಾನಮಾರ್ಗದಲ್ಲಿ ನಿರತರಾಗಿ ಬಾಳಬಂಧನವನ್ನು ಕಳಚಿಕೊಂಡು ದೋಷದೂರವಾದ ಪರಮ ಪದವಿಯನ್ನು ಪಡೆಯುತ್ತಾರೆ.\\}
\slcol{\Index{ಯದಾ ತೇ ಮೋಹಕಲಿಲಂ} ಬುದ್ಧಿರ್ವ್ಯತಿತರಿಷ್ಯತಿ ।\\
ತದಾ ಗಂತಾಸಿ ನಿರ್ವೇದಂ ಶ್ರೋತವ್ಯಸ್ಯ ಶ್ರುತಸ್ಯ ಚ ॥ 52 ॥}
\cquote{ನಿನ್ನ ಮನಸ್ಸು ತಪ್ಪು ತಿಳಿವೆಂಬ ಹೊಲಸನ್ನು ಕಳೆದುಕೊಂಡಾಗ ನೀನು ಕೇಳಿದ, ಕೇಳಲಿರುವ ಎಲ್ಲ ಉಪದೇಶ ಸಾರ್ಥಕವಾಗುತ್ತದೆ.\\}
\slcol{\Index{ಶ್ರುತಿವಿಪ್ರತಿಪನ್ನಾ ತೇ} ಯದಾ ಸ್ಥಾಸ್ಯತಿ ನಿಶ್ಚಲಾ ।\\
ಸಮಾಧಾವಚಲಾ ಬುದ್ಧಿಸ್ತದಾ ಯೋಗಮವಾಪ್ಸ್ಯಸಿ ॥ 53 ॥}
\cquote{ವೇದವಾದಗಳಿಂದ ಚಂಚಲವಾಗಿರುವ ನಿನ್ನ ಬುದ್ಧಿಯು ವಿಷಯಗಳಿಗೆರಗದೆ,ಅಲುಗಾಡದೆ ಆತ್ಮನಲ್ಲಿ ನೆಲೆಯಾಗಿ ನಿಂತಾಗ ಆತ್ಮದೊಡನೆ ಕೂಡಿದವನಾಗಿರುವೆ.\\}

\newpage
\begin{mananam}{\kanfont ಮನನ ಶ್ಲೋಕ - \textenglish{52,53}}
\footnotesize \mananamfont ನನ್ನ ಮನಸ್ಸು ಯೋಗ ಮಾರ್ಗದಲ್ಲಿ ಸಲ್ಪ ಮಟ್ಟಿಗಿನ ಸ್ಥಿರತೆ ಸಾಧಿಸಿದೆಯಾ? ಈ ಮಾರ್ಗದಲ್ಲಿ ಇನ್ನೂ ನಾನು ಅಸ್ಪೃಷ್ಟ  ಮತ್ತು ಗೊಂದಲದಲ್ಲಿದ್ದೇನೆಯೇ? ಅನುಮಾನಗಳನ್ನು ಪರಿಹರಿಸಿಕೊಳ್ಳಲು ನನ್ನ ಬಳಿ ಯಾವುದಾದರೂ ಮಾರ್ಗವಿದೆಯೇ? ಯವುದು ಆದಾರ? ಬೋಧನೆಯ ಅಥವಾ ಶಿಕ್ಷೇಕಾರ? ಈ ಭಾಹ್ಯ ಆಶ್ರಯಗಳ ಮೂಲಕ ನನ್ನ ಒಳಗಿರುವ ಗುರುತತ್ವದೊಂದಿಗೆ ಸಂಪರ್ಕ ಸಾಧಿಸಬಹುದೇ?
\end{mananam}
\begin{inspiration}{\kanfont ಸ್ಪೂರ್ತಿ}
\footnotesize \mananamfont ಭ್ರಮೆಯಿಂದ ನಮ್ಮನ್ನು ಹೊರ ತರುವುದೇ ಗುರುಗಳ ಮತ್ತು ಪುರಾಣ ಗ್ರಂಥಗಳ ಉದ್ದೇಶ. ಗೊಂದಲಗಳು ಮತ್ತು ಸವಾಲುಗಳು ಅಧ್ಯಾತ್ಮಿ ಪ್ರಯಾಣದ ಒಂದು ಭಾಗವಾಗಿದೆ. ಲೌಕಿಕ ಚಿಂತನೆಗಳನ್ನು ದೂರಸರಿಸಿದರೆ ಉನ್ನತ ವಾಸ್ತವದಲ್ಲಿ ಆಶ್ರಯ ಪಡೆಯಬಹುದು. ಹೀಗೆ ನಾವು ಪ್ರಗತಿ ಹೊಂದಿದಾಗ ಈ ಮಾರ್ಗದಲ್ಲಿ ಸ್ಪಷ್ಟತೆಯನ್ನು ಪಡೆಯುತ್ತೇವೆ.
\end{inspiration}
\newpage

\slcol{ಅರ್ಜುನ ಉವಾಚ ।\\
\Index{ಸ್ಥಿತಪ್ರಙ್ಞಸ್ಯ ಕಾ ಭಾಷಾ} ಸಮಾಧಿಸ್ಥಸ್ಯ ಕೇಶವ ।\\
ಸ್ಥಿತಧೀಃ ಕಿಂ ಪ್ರಭಾಷೇತ ಕಿಮಾಸೀತ ವ್ರಜೇತ ಕಿಮ್ ॥ 54 ॥}
\cquote{ಅರ್ಜುನನ್ನು ಹೇಳಿದನು, ಕೇಶವ ಸಮಾಧಿನಿಷ್ಠನಾದ ಸ್ಟಿತಪ್ರಜ್ಞನ ಲಕ್ಷಣವೇನು? ಅವನು ಹೇಗೆ ಮಾತನಾಡುತ್ತಾನೆ?ಹೇಗೆಇರುತ್ತಾನೆ? ಹೇಗೆ 
ವ್ಯವರಿಸುತ್ತಾನೆ?\\}

\newpage
\begin{mananam}{\kanfont ಮನನ ಶ್ಲೋಕ - \textenglish{54}}
\footnotesize \mananamfont ಸಾಕ್ಷಾತ್ಕಾರದ ಸ್ಥಿತಿ ಯಾವುದು ಎಂದು ನಾನು ಅರ್ಥ ಮಾಡಿಕೊಂಡಿದ್ದೇನೆಯೇ? ಸಂತರ ಸಾಕ್ಷಾತ್ಕಾರದ  ವಿವಿಧ ಹಂತಗಳ ಬಗ್ಗೆ ನನಗೆ ತಿಳಿದಿದೆಯೇ? ನನ್ನ ಸ್ವಂತ ಅಧ್ಯಾತ್ಮಿಕ ವಿಕಾಸದ ಮುಂದಿನ ಹಂತ ಯಾವುದು? ನಾನು ಯಾವ ಹಂತವನ್ನು ಪ್ರಾಮಾಣಿಕವಾಗಿ ಬಯಸಬಹುದು? ಅಂತಿಮ ವಿಮೋಚನೆ ಮತ್ತು ಸ್ವಾತಂತ್ರದ ಬಗ್ಗೆ ನನ್ನ ತಿಳುವಳಿಕೆಯು ನನ್ನ ಜೀವನದಲ್ಲಿ ಮಾನಸಿಕ ಮತ್ತು ಅಧ್ಯಾತ್ಮಿಕ ಪ್ರಗತಿಯನ್ನು ಮಾಡಲು  ನನ್ನನ್ನು ಪ್ರೇರೇಪಿಸುತ್ತದೆಯೇ?
\end{mananam}
\begin{inspiration}{\kanfont ಸ್ಪೂರ್ತಿ}
\footnotesize \mananamfont
 ಜೀವನದ ಪ್ರತಿಯೊಂದು ಕ್ಷೇತ್ರದಲ್ಲೂ ಯಶಸ್ವಿಯಿಂದ ಜನರಿಂದ ನಾವು ಪ್ರೇರೇರಿತರಾಗಿದ್ದೇವೆ. ಹಾಗೆಯೇ ಸಾದು ಸಂತರು ಸಾಧಿಸಿದ ಭಾಹ್ಯಸ್ಥಿತಿ ಮಾತ್ರವಲ್ಲ ಆಂತರಿಕ ಸ್ಥಿತಿಯ ಬಗ್ಗೆ ನಮಗೆ ಅರ್ಥೈಸಿಕೊಳ್ಳಲು ಕಷ್ಟವಾಗುತ್ತದೆ. ಅವರ ಈ ಆಂತರಿಕ ಸ್ಥಿತಿಯನ್ನು ಚೆನ್ನಾಗಿ ಅರ್ಥೈಸಿಕೊಳ್ಳುವುದರಿಂದ ತಪ್ಪು ತಿಳುವಳಿಕೆ ಮತ್ತು ತಪ್ಪು ನಿರ್ಧಾರಗಳಿಂದ ದೂರವಿರಲು ಸಹಾಯ ಮಾಡುತ್ತದೆ.
\end{inspiration}
\newpage

\slcol{ಶ್ರೀಭಗವಾನುವಾಚ ।\\
\Index{ಪ್ರಜಹಾತಿ ಯದಾ ಕಾಮಾನ್ಸ}ರ್ವಾನ್ಪಾರ್ಥ ಮನೋಗತಾನ್ ।\\
ಆತ್ಮನ್ಯೇವಾತ್ಮನಾ ತುಷ್ಟಃ ಸ್ಥಿತಪ್ರಙ್ಞಸ್ತದೋಚ್ಯತೇ ॥ 55 ॥}
\cquote{ಅರ್ಜುನಾ, ಮನಸ್ಸಿನಲ್ಲಿರುವ ಬಯಕೆಗಳನ್ನೆಲ್ಲ ಬಿಟ್ಟಾಗ ಅವನು ತನ್ನಿಂದಲೇ ತನ್ನಲ್ಲಿ ತೃಪ್ತನಾಗಿ ಆತ್ಮದಲ್ಲಿ ಸ್ಥಿರವಾದ ಬುದ್ಧಿಯುಳ್ಳವನಾಗುತ್ತಾನೆ.}

\newpage
\begin{mananam}{\kanfont ಮನನ ಶ್ಲೋಕ - \textenglish{55}}
\footnotesize \mananamfont ಅಧ್ಯಾತ್ಮಿಕ ಪ್ರಗತಿಯನ್ನು ಮಾಡದಂತೆ ನನ್ನನ್ನು ತಡೆಯುತ್ತಿರುವ ಕೆಳಮಟ್ಟದ ಆಸೆಗಳು ಯಾವುವು? ನನ್ನನ್ನು ಕೆಳಮಟ್ಟಕ್ಕೆ ಎಳೆಯುತ್ತಿರುವ ಮಾನಸಿಕ ಅಭ್ಯಾಸಗಳು ಮತ್ತು ಬಾವೋದ್ರೇಕಗಳನ್ನು   ನಾನು ಹೇಗೆ ತ್ಯಜಿಸಬಹುದು. ಆನಂದ ಮತ್ತು ತೃಪ್ತಿಯನ್ನು ನನ್ನ ಸ್ವಂತ ಆತ್ಮದಲ್ಲಿಯೇ ಕಂಡುಕೊಳ್ಳುವುದು ಎಂಬುದರ ಅರ್ಥವೇನು?
\end{mananam}
\begin{inspiration}{\kanfont ಸ್ಪೂರ್ತಿ}
\footnotesize \mananamfont  ಒಬ್ಬ ನಿಜವಾದ ಯೋಗಿ ಅಥವಾ ಸನ್ಯಾಸಿಯು ತನ್ನ ಸಂತೋಷಕ್ಕಾಗಿ ಯಾವುದರ ಮೇಲೆಯೂ ಯಾರ ಮೇಲೆಯೂ ಅವಲಂಬಿತನಾಗುವುದಿಲ್ಲ. ತಮ್ಮ ಸ್ವಂತ ಆತ್ಮದಲ್ಲಿಯೇ ಸುಖ ಮತ್ತು ಸಂತೋಷ ಕಂಡುಕೊಂಡಿರುವ ಇವನು ಲೌಕಿಕ ಆಸೆಗಳಿಗೆ ಹಾತೊರೆಯುವುದಿಲ್ಲ.
\end{inspiration}
\newpage

\slcol{\Index{ದುಃಖೇಷ್ವನುದ್ವಿಗ್ನಮನಾಃ} ಸುಖೇಷು ವಿಗತಸ್ಪೃಹಃ ।\\
ವೀತರಾಗಭಯಕ್ರೋಧಃ ಸ್ಥಿತಧೀರ್ಮುನಿರುಚ್ಯತೇ ॥ 56 ॥}
\cquote{ದುಃಖಗಳು ಬಂದಾಗ ತಳವಳಗೊಳ್ಳದೆ, ಸುಖಗಳು ಬಂದಾಗ ಬಾಯಿನೀರು ಸುರಿಸದೆ, ಒಲವು, ಹೆದರಿಕೆ,ಸಿಟ್ಟು ಇಂಥ ಭಾವಗಳಿಗೆ ಬಲಿಯಾಗದೆ ಆತ್ಮವಿಚಾರವನ್ನೇ ಹಚ್ಚಿಕೊಂಡಿರುವವನು ಸ್ಥಿತಪ್ರಜ್ಞ ಎನಿಸಿಕೊಳ್ಳುತ್ತಾನೆ.}
\slcol{\Index{ಯಃ ಸರ್ವತ್ರಾನಭಿಸ್ನೇಹ}ಸ್ತತ್ತತ್ಪ್ರಾಪ್ಯ ಶುಭಾಶುಭಮ್ ।\\
ನಾಭಿನಂದತಿ ನ ದ್ವೇಷ್ಟಿ ತಸ್ಯ ಪ್ರಙ್ಞಾ ಪ್ರತಿಷ್ಠಿತಾ ॥ 57 ॥}
\cquote{ಯಾವುದನ್ನು ಅತಿಯಾಗಿ ಹಚ್ಚಿಕೊಳ್ಳದೆ, ಒಳ್ಳೆಯದೂ,ಕೆಟ್ಟದ್ದೂ ಒದಗಿ ಬಂದಾಗ ಹಿಗ್ಗದೆ ಕುಗ್ಗದ ಸಮವಾಗಿ ಕಾಣಬಲ್ಲವನ ಪ್ರಜ್ಞೆ ಸ್ಥಿರವಾಗಿರುತ್ತದೆ.}

\newpage
\begin{mananam}{\kanfont ಮನನ ಶ್ಲೋಕ - \textenglish{56, 57}}
\footnotesize \mananamfont ವೈಫಲ್ಯಗಳು ಮತ್ತು ನಷ್ಟಗಳು ನನ್ನನ್ನು ಮಾನಸಿಕವಾಗಿ  ಕುಗ್ಗಿಸುತ್ತವೆಯೇ? ಅದು ನನ್ನ ಆತ್ಮವಿಶ್ವಾಸದ ಮೇಲೆ ಪರಿಣಾಮ ಬೀರಲು ನಾನು ಬಿಡುತ್ತೇನೆಯೇ? ಯಶಸ್ಸು ಮತ್ತು ಲಾಭ ಬಂದಾಗ ಹೇಗಿರುತ್ತದೆ? ನಾನು ಅತಿಯಾಗಿ , ಉತ್ಸುಕನಾಗುತ್ತೇನೆಯೇ? ಅದು ನನ್ನನ್ನು ಅಹಂಕಾರಿಯಾಗಿ ಮತ್ತು ಇತರರ ಬಗ್ಗೆ ನಿರಾಕರಣೆ ಭಾವ ಹೊಂದುತ್ತೇನೆಯೇ?ಜೀವನದ ಸನ್ನಿವೇಶದಲ್ಲಿ ಇದು ಒಳ್ಳೆಯದು ಕೆಟ್ಟದ್ದು ಎಂದು ನಾನು ನಿರಂತರವಾಗಿ ನಿರ್ಣಯಿಸುತ್ತಾ ಇರುತ್ತೇನೆಯೇ?
\end{mananam}
\begin{inspiration}{\kanfont ಸ್ಪೂರ್ತಿ}
\footnotesize \mananamfont ಒಬ್ಬ ಸಾಮಾನ್ಯ ಮನುಷ್ಯನಿಗೆ ಜೀವನದಲ್ಲಿ ಬರುವ ಸಂದರ್ಭಗಳಿಗೆ ತಕ್ಕಂತೆ ಮಾನಸಿಕ ಸ್ಥಿತಿಯು ಬದಲಾಗುತ್ತಿರುತ್ತದೆ. ಹೀಗಾಗಿ ಜೀವನದಲ್ಲಿ ಧನಾತ್ಮಕ ಫಲಿತಾಂಶಗಳ ಕಡೆಗೆ ಒಲವು ಮತ್ತು ಋಣಾತ್ಮಕ ಫಲಿತಾಂಶಗಳ ವಿಮುಖತೆ ಯಾಗುತ್ತದೆ. ಆದರೆ ಒಬ್ಬ ಯೋಗಿಗೆ ತನ್ನ ಎಲ್ಲಾ ಅಧ್ಯಾತ್ಮಿಕ ಅಭ್ಯಾಸದ ಗುರಿ ಜೀವನದಲ್ಲಿ ಯಾವಾಗಲೂ ಅನುಕೂಲಕರ ಸಂದರ್ಭವನ್ನು ಪಡೆಯುವುದು ಅಲ್ಲ, ಎಂತಹ ಸಂದರ್ಭದಲ್ಲಿ ಮಾನಸಿಕ ಸ್ಥಿರತೆಯನ್ನು ಪಡೆಯುವುದೇ ಆಗಿದೆ.
\end{inspiration}
\newpage

\slcol{\Index{ಯದಾ ಸಂಹರತೇ ಚಾಯಂ} ಕೂರ್ಮೋऽಂಗಾನೀವ ಸರ್ವಶಃ ।\\
ಇಂದ್ರಿಯಾಣೀಂದ್ರಿಯಾರ್ಥೇಭ್ಯಸ್ತಸ್ಯ ಪ್ರಙ್ಞಾ ಪ್ರತಿಷ್ಠಿತಾ ॥ 58 ॥}
\cquote{ಆಮೆಯು ತನ್ನ ಅವಯವಗಳನ್ನು ಎಲ್ಲ ಕಡೆಯಿಂದಲೂ ಒಳ ಸೆಳೆದುಕೊಳ್ಳುವಂತೆ ಹೊರಗಣ ವಿಷಯಗಳಿಂದ ಇಂದ್ರಿಯಗಳನ್ನು ಅಂತರ್ಮುಖಗೊಳಿಸಬಲ್ಲವನ ಪ್ರಜ್ಞೆ ಸ್ಥಿರವಾಗಿರುತ್ತದೆ.}

\newpage
\begin{mananam}{\kanfont ಮನನ ಶ್ಲೋಕ - \textenglish{58}}
\footnotesize \mananamfont ಇಂದ್ರಿಯ ನಿಯಂತ್ರಣಗಳನ್ನು ಎಷ್ಟರ ಮಟ್ಟಿಗೆ ನಾನು ಹೊಂದಿದ್ದೇನೆ? ಇಂದ್ರಿಯ ಸುಖದಲ್ಲಿ ನಾನು ಅತಿಯಾಗಿ ತೊಡಗಿಸಿಕೊಳ್ಳುತ್ತೇನೆಯೇ? ನನ್ನ ಮನಸ್ಸನ್ನು ಇಂದ್ರಿಯಗಳ ಕಡೆಗೆ ಸೆಳೆಯುವ ಬಾಹ್ಯ ಪ್ರಚೋದಗಳ ಬಗ್ಗೆ ನನಗೆ ಅರಿವಿದೆಯೇ? ಆಕರ್ಷಕ ವಸ್ತುಗಳಿಂದ ನಮ್ಮ ಯೋಚನೆಗಳು ಮತ್ತು ನೆನಪುಗಳನ್ನು ಪ್ರಚೋದಿಸುವ ಆಂತರಿಕ ಪ್ರಚೋದನೆಗಳ ಬಗ್ಗೆ ನನಗೆ ಅರಿವಿದೆಯೇ? ಇಂದ್ರಿಯಗಳಿಂದ ನನ್ನ ಮನಸ್ಸನ್ನು ಹಿಂತೆಗೆದುಕೊಳ್ಳುವ ಯಾವುದಾದರೂ ವಿಧಾನವನ್ನು ನಾನು ಪ್ರಯತ್ನಿಸಿದ್ದೇನೆಯೇ? ಇಲ್ಲವಾದಲ್ಲಿ ಅದನ್ನು ಹೇಗೆ ಅಭಿವೃದ್ಧಿಗೊಳಿಸುವುದು?
\end{mananam}
\begin{inspiration}{\kanfont ಸ್ಪೂರ್ತಿ}
\footnotesize \mananamfont ಸ್ವಾಭಾವಿಕವಾಗಿ ಹೊರಗೆ ಹೋಗುವ ಇಂದ್ರಿಯಗಳನ್ನು ಒಳಕ್ಕೆ ಹಿಂತೆಗೆದುಕೊಳ್ಳಲು ಪ್ರಜ್ಞಪೂರ್ವಕ ತರಬೇತಿಯನ್ನು ಪಡೆಯಬೇಕಾಗುತ್ತದೆ. ವ್ಯಸನಗಳ ಹಾನಿಯ ಬಗ್ಗೆ ಕೇವಲ ಜ್ಞಾನ ಮತ್ತು ಆಶಯ ಸಾಕಾಗುವುದಿಲ್ಲ. ಧನಾತ್ಮಕ ಅಭ್ಯಾಸಗಳನ್ನು ಬೆಳೆಸಿಕೊಳ್ಳಲು ಸತತವಾಗಿ ಹೆಚ್ಚು ಮಾಡುತ್ತಾ ಹೋಗುವ ಕ್ರಮಗಳು ಉತ್ತಮವಾಗಿರುತ್ತವೆ.
\end{inspiration}
\newpage

\slcol{\Index{ವಿಷಯಾ ವಿನಿವರ್ತಂತೇ} ನಿರಾಹಾರಸ್ಯ ದೇಹಿನಃ ।\\
ರಸವರ್ಜಂ ರಸೋऽಪ್ಯಸ್ಯ ಪರಂ ದೃಷ್ಟ್ವಾ ನಿವರ್ತತೇ ॥ 59 ॥}
\cquote{ಆಹಾರ ನಿಗ್ರಹದಿಂದ ಜೀವನಿಗೆ ವಿಷಯ ಭೋಗದ ಶಕ್ತಿ ಕುಂದುವುದೇ ಹೊರತು ಭೋಗದ ಬಯಕೆ ಕುಂದುವುದಿಲ್ಲ. ಭಗವಂತನ ದರ್ಶನವಾದಾಗಲೇ ಈ ಬಯಕೆಯನ್ನೂ ನಿಗ್ರಹಿಸುವುದು ಸಾಧ್ಯ.\\}
\slcol{\Index{ಯತತೋ ಹ್ಯಪಿ ಕೌಂತೇಯ} ಪುರುಷಸ್ಯ ವಿಪಶ್ಚಿತಃ ।\\
ಇಂದ್ರಿಯಾಣಿ ಪ್ರಮಾಥೀನಿ ಹರಂತಿ ಪ್ರಸಭಂ ಮನಃ ॥ 60 ॥}
\cquote{ಹತ್ತು ಕಡೆಗೂ ಎಳೆಯುವಂತ ಇಂದ್ರಿಯಗಳು ಪ್ರಯತ್ನಿಶೀಲನಾದ ಜ್ಞಾನಿಯ ಮನಸ್ಸನ್ನೂ ಬಲಾತ್ಕಾರವಾಗಿ ಅಪಹರಿಸಿಬಿಡುವವು.\\}
\slcol{\Index{ತಾನಿ ಸರ್ವಾಣಿ ಸಂಯಮ್ಯ} ಯುಕ್ತ ಆಸೀತ ಮತ್ಪರಃ ।\\
ವಶೇ ಹಿ ಯಸ್ಯೇಂದ್ರಿಯಾಣಿ ತಸ್ಯ ಪ್ರಙ್ಞಾ ಪ್ರತಿಷ್ಠಿತಾ ॥ 61 ॥}
\cquote{ಅವೆಲ್ಲವನ್ನೂ ಬಿಗಿಹಿಡಿದು ನನ್ನನ್ನೇ ಗತಿಯೆಂದು ನನ್ನಲ್ಲಿಯೇ ಮನಸಿಡಬೇಕು. ಯಾರ ಇಂದ್ರಿಯಗಳು ಹಿಡಿತದಲ್ಲಿರುವವೋ ಅವನ ಪ್ರಜ್ಞೆ ಸ್ಥಿರವಾಗಿರುತ್ತದೆ.\\}
\slcol{\Index{ಧ್ಯಾಯತೋ ವಿಷಯಾನ್ಪುಂಸಃ} ಸಂಗಸ್ತೇಷೂಪಜಾಯತೇ ।\\
ಸಂಗಾತ್ಸಂಜಾಯತೇ ಕಾಮಃ ಕಾಮಾತ್ಕ್ರೋಧೋऽಭಿಜಾಯತೇ ॥ 62 ॥}
\cquote{ಸುಖ ಸಾಧನಗಳನ್ನೇ ಹಂಬಲಿಸುತ್ತಿರುವ ಅವನಿಗೆ ಅವುಗಳಲ್ಲಿ ಆಸಕ್ತಿ ಹುಟ್ಟುತ್ತದೆ. ಆಸಕ್ತಿಯಿಂದ ಬಯಕೆ ಹುಟ್ಟುತ್ತದೆ.ಬಯಕೆ ಈಡೇರದಾಗ ಸಿಟ್ಟು ತಲೆ ಹಾಕುತ್ತದೆ.\\}
\slcol{\Index{ಕ್ರೋಧಾದ್ಭವತಿ ಸಂಮೋಹಃ} ಸಂಮೋಹಾತ್ಸ್ಮೃತಿವಿಭ್ರಮಃ ।\\
ಸ್ಮೃತಿಭ್ರಂಶಾದ್ಬುದ್ಧಿನಾಶೋ ಬುದ್ಧಿನಾಶಾತ್ಪ್ರಣಶ್ಯತಿ ॥ 63 ॥}
\cquote{ಸಿಟ್ಟಿನ ಮರಿ ಅವಿವೇಕ.ಅವಿವೇಕದಿಂದ ಧರ್ಮ ಅಧರ್ಮಗಳ ಮರೆವು. ಇಂತ ಮರೆವಿನಿಂದ ಬುದ್ಧಿ ಕೆಡುತ್ತದೆ. ಬುದ್ದಿ ಕೆಡುವುದೇ ಎಲ್ಲ ಅನರ್ಥದ ಮೂಲ.\\}

\newpage
\begin{mananam}{\kanfont ಮನನ ಶ್ಲೋಕ - \textenglish{62, 63}}
\footnotesize \mananamfont ನನ್ನ ದೈನೆಂದಿನ ಜೀವನದ ಏರಿಳಿತಗಳ ಸರಣಿಯ ಬಗ್ಗೆ ನನಗೆ ಅರಿವಿದೆಯೇ? ಜನರು ಮತ್ತು ವಸ್ತುಗಳ ಬಗ್ಗೆ ಅತಿಯಾಗಿ ಚಿಂತಿಸುವುದರಿಂದ ಮೋಹಕ್ಕೆ ಕಾರಣವಾಗುತ್ತದೆ. ಆಸೆಗಳಿಗೆ  ಅಡ್ಡಿಯಾದಾಗ ಕ್ರೋದದ ಘಟನೆಗಳು ಹೇಗೆ ಸಂಭವಿಸುತ್ತದೆ ಎಂದು ನನಗೆ ಅರಿವಿದೆಯೇ? ನನಗೆ ಈ ಕೋಪದಿಂದ ಉಂಟಾದ ಘಟನೆಗಳುನ್ನು ಹಿಂತಿರುಗಿ ನೋಡಿದಾಗ ಅವು ಹೇಗೆ ನಮ್ಮ ಒಳ್ಳೆಯ ಉದ್ದೇಶವನ್ನು ಮರೆಯಿಸಿ ಭ್ರಮೆಯನ್ನು ಉಂಟುಮಾಡುತ್ತದೆ ಎಂದು ನೋಡಬಹುದೇ?ಮತ್ತು  ಇದರಿಂದಾಗಿ ಒಳ್ಳೆಯದು ಕೆಟ್ಟದ್ದು ಸರಿ ತಪ್ಪುಗಳನ್ನು ವಿವೇಚಿಸುವ ಸಾಮರ್ಥ್ಯವನ್ನು ಕಳೆದುಕೊಳ್ಳುತ್ತೇವೆ. ಇತರರು ತಮ್ಮ ಜೀವನದಲ್ಲಿ ಈ ಕ್ರೋದ ಘಟನೆಗಳಿಂದ ಕೆಳಕ್ಕೆ ಬೀಳುವುದನ್ನು ಗಮನಿಸಿ ನಾವು ಕಲಿಯಬಹುದೇ? 
\end{mananam}
\begin{inspiration}{\kanfont ಸ್ಪೂರ್ತಿ}
\footnotesize \mananamfont ಜೀವನದಲ್ಲಿ ವಿಷಯದ ಮೋಹಕ್ಕೆ ಒಳಗಾಗಿ ಕ್ರೋಧ; ಕೋಪದಿಂದ ಹೇಗೆ ಕೆಳಗೆ ಬೀಳುತ್ತೇವೆ ಎಂಬುದರ ಬಗ್ಗೆ ಗೀತೆಯಲ್ಲಿ ಸ್ಪಷ್ಟವಾಗಿ  ಹೇಳಲಾಗಿದೆ. ಈ ತರಹ ಘಟನೆಗಳಿಗೆ ಜನರು ಬಲಿಯಾದ ಉದಾಹರಣೆಗಳನ್ನು ನಾವು ನಮ್ಮ ಸುತ್ತಲೂ ನೋಡಬಹುದು. ಈ ತರಹ ಕ್ರೋದ ಸರಮಾಲೆಗೆ ಸಿಲುಕಿಕೊಂಡರೆ ಅದರಿಂದ ಹೊರಬರುವುದು ಕಷ್ಟವಾಗುತ್ತದೆ. 
\end{inspiration}
\newpage

\slcol{\Index{ರಾಗದ್ವೇಷವಿಮುಕ್ತೈಸ್ತು} ವಿಷಯಾನಿಂದ್ರಿಯೈಶ್ಚರನ್ ।\\
ಆತ್ಮವಶ್ಯೈರ್ವಿಧೇಯಾತ್ಮಾ ಪ್ರಸಾದಮಧಿಗಚ್ಛತಿ ॥ 64 ॥}
\cquote{ಮನಸ್ಸನ್ನು ಹಿಡಿತದಲ್ಲಿಟ್ಟುಕೊಂಡು ಆಸಕ್ತಿ ಆಗಲಿ ದ್ವೇಷವಾಗಲಿ ಇಲ್ಲದೆ ತನ್ನ ಅಂಕಿತದಲ್ಲಿರುವ ಇಂದ್ರಿಯಗಳಿಂದ ವಿಷಯಗಳನ್ನು ಬಳಸುವವರ ಮನಸ್ಸು ತಿಳಿಯಾಗುತ್ತದೆ.\\}
\slcol{\Index{ಪ್ರಸಾದೇ ಸರ್ವದುಃಖಾನಾಂ} ಹಾನಿರಸ್ಯೋಪಜಾಯತೇ ।\\
ಪ್ರಸನ್ನಚೇತಸೋ ಹ್ಯಾಶು ಬುದ್ಧಿಃ ಪರ್ಯವತಿಷ್ಠತೇ ॥ 65 ॥}
\cquote{ಮನಸ್ಸು ತಿಳಿಯಾದಾಗ ದುಃಖಗಳೆಲ್ಲ ದೂರವಾಗುತ್ತದೆ.ತಿಳಿಯಾದ ಮನಸ್ಸಿನವರ ಬುದ್ಧಿ ಬೇಗ ಭಗವಂತನಲ್ಲಿ ನೆಲೆಗೊಳ್ಳುತ್ತದೆ.\\}
\slcol{\Index{ನಾಸ್ತಿ ಬುದ್ಧಿರಯುಕ್ತಸ್ಯ} ನ ಚಾಯುಕ್ತಸ್ಯ ಭಾವನಾ ।\\
ನ ಚಾಭಾವಯತಃ ಶಾಂತಿರಶಾಂತಸ್ಯ ಕುತಃ ಸುಖಮ್ ॥ 66 ॥}
\cquote{ಮನಸ್ಸು ಹಿಡಿತದಲ್ಲಿರದವನಿಗೆ ಜ್ಞಾನ ಸಿದ್ದಿ ಇಲ್ಲ. ಧ್ಯಾನವು ಸಿದ್ಧಿಸುವುದಿಲ್ಲ. ಧ್ಯಾನ ಇಲ್ಲದೆ ಶಾಂತಿ ಇಲ್ಲ. ಶಾಂತಿ ಇಲ್ಲದವನಿಗೆ ಸುಖವೆಲ್ಲಿಯದು!\\}
\slcol{\Index{ಇಂದ್ರಿಯಾಣಾಂ ಹಿ ಚರತಾಂ} ಯನ್ಮನೋऽನುವಿಧೀಯತೇ ।\\
ತದಸ್ಯ ಹರತಿ ಪ್ರಙ್ಞಾಂ ವಾಯುರ್ನಾವಮಿವಾಂಭಸಿ ॥ 67 ॥}
\cquote{ವಿಷಯಗಳತ್ತ ಹರಿಯುವ ಇಂದ್ರಿಯಗಳ ಜೊತೆಗೆ ಮನಸ್ಸನ್ನು ಹೋಗಗೊಟ್ಟರೆ ಅದು ನಡು ನೀರಿನಲ್ಲಿರುವ ಹಡಗನ್ನು ಬಿರುಗಾಳಿ ಹೇಗೆ ಹಾಗೆ ಸಾಧಕನ ಪ್ರಜ್ಞೆಯನ್ನು ಹಾರಿಸಿಬಿಡುತ್ತದೆ.}
\slcol{\Index{ತಸ್ಮಾದ್ಯಸ್ಯ ಮಹಾಬಾಹೋ} ನಿಗೃಹೀತಾನಿ ಸರ್ವಶಃ ।\\
ಇಂದ್ರಿಯಾಣೀಂದ್ರಿಯಾರ್ಥೇಭ್ಯಸ್ತಸ್ಯ ಪ್ರಙ್ಞಾ ಪ್ರತಿಷ್ಠಿತಾ ॥ 68 ॥}
\cquote{ಆದ್ದರಿಂದ ಅರ್ಜುನ ಯಾವ ಇಂದ್ರಿಯಗಳು ಎಲ್ಲ ಬಗೆಯ ವಿಷಯಗಲಿಂದಲೂ ಪಾರಾಗಿ ಅಂತರ್ಮುಖವಾಗಿದೆಯೋ ಅವನ ಪ್ರಜ್ಞೆ ಸ್ಥಿರವಾಗಿರುತ್ತದೆ.}

\newpage
\begin{mananam}{\kanfont ಮನನ ಶ್ಲೋಕ - \textenglish{67, 68}}
\footnotesize \mananamfont ದೀರ್ಘಾವಧಿಯಲ್ಲಿ ನನಗೆ ಒಳ್ಳೆಯದು ಮತ್ತು ಅಲ್ಪಾವಧಿಯಲ್ಲಿ ಮಾತ್ರ ಒಳ್ಳೆಯದು, ಇವೆರಡರ ನಡುವೆ ವಿವೇಚಿಸುವ ನನ್ನ ಸಾಮರ್ಥ್ಯ ಎಷ್ಟು ಉತ್ತಮವಾಗಿದೆ? ನನ್ನ ಜೀವನದಲ್ಲಿ ನನಗೆ ಸಹಾಯ ಮಾಡುವ ಅಭ್ಯಾಸಗಳನ್ನು ಗುರುತಿಸಲು ಮತ್ತು ಅಂತಹ ಅಭ್ಯಾಸಗಳನ್ನೇ ಬಲಪಡಿಸಲು ಪ್ರಜ್ಞಪೂರ್ವಕವಾಗಿ ಕೆಲಸ ಮಾಡಲು ಸಾಧ್ಯವೇ?\\
ಯಾವುದೇ ಭೋಗದಲ್ಲಿ ಎರಡು ಪ್ರಕ್ರಿಯೆಗಳಿವೆ. ಮನಸ್ಸು ವಸ್ತುಗಳ ಮೇಲೆಯೇ ಅವಲಂಬನೆಯಗಿರುವುದು ಮತ್ತು ಬೌದ್ಧಿಕ ಇಂದ್ರಿಯಗಳು ಅವುಗಳನ್ನು ಗ್ರಹಿಸುವುದು ಇವುಗಳು ತುಂಬಾ ವೇಗವಾಗಿ ಸಂಭವಿಸುವುದರಿಂದ ಹೆಚ್ಚಿನವರಿಗೆ ಅವುಗಳನ್ನು ಪ್ರತ್ಯೇಕಿಸಲು ಸಾಧ್ಯವಾಗುವುದಿಲ್ಲ.ಈ ಎರಡು ಪ್ರಕ್ರಿಯೆಗಳನ್ನು ನಿಮ್ಮ ಜೀವನದ ಕೆಲವೊಂದು ಸನ್ನಿವೇಶಗಳಲ್ಲಿ ನೋಡಬಹುದು ನೀವು ಯಾವುದರ ಮೇಲೆ ಗಮನಹರಿಸಬೇಕು?
\end{mananam}
\begin{inspiration}{\kanfont ಸ್ಪೂರ್ತಿ}
\footnotesize \mananamfont ಈ ಜಗತ್ತಿನಲ್ಲಿ ಅನೇಕರು ತಮ್ಮ ಇಂದ್ರಿಯಗಳಿಗೆ ದಾಸರಾಗಿರುತ್ತಾರೆ. ಆಧುನಿಕ ಜಗತ್ತು ಇವತ್ತು ಇದನ್ನು ಸಹಜ ಎಂದು ಪರಿಗಣಿಸುವುದಲ್ಲದೆ ಅಂತಹ ಜೀವನವನ್ನು ವೈಭವಿಕರಿಸುತ್ತದೆ. ಆದರೆ ಇಂತಹ ಬದುಕನ್ನು ಆಶಿಸುವವರು ದೀರ್ಘಾವಧಿಯಲ್ಲಿ ಸೂಚನೆಯ ಆಗುವುದು ಖಚಿತ. ಹೆಚ್ಚಿನ ಧರ್ಮಗಳು ಇಂದ್ರಿಯ ಸಂಯಮವನ್ನು ಹೊಸಬರಿಗೆ ಸಲಹೆ ಕೊಡುತ್ತದೆ ಯಾರು ತಮ್ಮ ಪ್ರಗತಿಗೆ ಖಂಡಿತವಾಗಿ ಬದ್ಯವಾಗಿರುವ ರೋ ಅವರಿಗೆ ಮಾನಸಿಕ ಸಂಯಮವು ಅತ್ಯುತ್ತಮ ಅಭ್ಯಾಸವಾಗಿದೆ.
\end{inspiration}
\newpage

\slcol{\Index{ಯಾ ನಿಶಾ ಸರ್ವಭೂತಾನಾಂ} ತಸ್ಯಾಂ ಜಾಗರ್ತಿ ಸಂಯಮೀ ।\\
ಯಸ್ಯಾಂ ಜಾಗ್ರತಿ ಭೂತಾನಿ ಸಾ ನಿಶಾ ಪಶ್ಯತೋ ಮುನೇಃ ॥ 69 ॥}
\cquote{ಸಾಧಾರಣ ಮನುಷ್ಯರಿಗೆ ರಾತ್ರಿಯಂತಿರುವ ಜ್ಞಾನ ದೆಶೆಯಲ್ಲಿ ಯೋಗಿ ಎಚ್ಚೆತ್ತಿರುವನು. ಅವರಿಗೆ ಹಗಲಿನಂತಿರುವ ಭೋಗೇಚ್ಚಾ ವಿಷಯದಲ್ಲಿ ಆತ್ಮಜ್ಞಾನಿ ನಿದ್ರಿಸುತ್ತಾನೆ . (ಅಂದರೆ ಭೋಗಿಯ ರಾತ್ರಿ ಯೋಗಿಗೆ ಹಗಲು ಯೋಗಿಯ ರಾತ್ರಿ ಬೋಗಿಗೆ ಹಗಲು.)\\}

\newpage
\begin{mananam}{\kanfont ಮನನ ಶ್ಲೋಕ - \textenglish{69}}
\footnotesize \mananamfont ದೈನಂದಿನ ಜೀವನದ ಜಂಜಾಟವನ್ನು ತಪ್ಪಿಸಿಕೊಳ್ಳುವ ಸಾಧನವಾಗಿ ನಾನು ನಿದ್ರೆಯನ್ನು ಎದುರು ನೋಡುತ್ತಿದ್ದೇನೆಯೇ? ನನಗೆ ಜೀವನದ ಉದ್ದೇಶದ ಅರಿವಿದೆಯೇ? ಅದರಿಂದ ಉತ್ಸಾಹದಿಂದ ಎಚ್ಚರವಾಗಲು ಸಾಧ್ಯವಾಗುತ್ತಿದೆಯೇ? ಸೋಮಾರಿತನ ಮತ್ತು ಬೇಸರದ ಪ್ರವೃತ್ತಿಗಳು ನನ್ನ ದಿನನಿತ್ಯದ ಜೀವನದ ಮೇಲೆ ಪ್ರಾಬಲ್ಯ ಹೊಂದಿದೆಯೇ?\\
 ನಾನು ಎಚ್ಚರದ ಸ್ಥಿತಿಯನ್ನು ಅಂತಿಮ ವಾಸ್ತವವೆಂದು ಪರಿಗಣಿಸುತ್ತೇನೆಯೇ? ಅಥವಾ ಇದಕ್ಕಿಂತ ಹೆಚ್ಚಿನದನ್ನು ಗ್ರಹಿಸಬಹುದೇ? ಸಮಾಜದ ಮಾದರಿಗಳಿಂದ ಪ್ರಭಾವಿತವಾಗಿರುವ ಜೀವನದ ನಾಗಾಲೋಟದಲ್ಲಿ ಸಿಕ್ಕಿ ಬಿದ್ದಿದ್ದೇನೆಯೇ? ಎಚ್ಚರ ಮತ್ತು ನಿದ್ದೆಯ ಈ ಎರಡು ಸ್ಥಿತಿಗಲ್ಲು ನಾನು ಹೇಗೆ ಋಷಿಗಳ ತರಹ ಉನ್ನತ ಅರಿವನ್ನು ಬೆಳೆಸಿಕೊಳ್ಳಬಹುದು.
\end{mananam}
\begin{inspiration}{\kanfont ಸ್ಪೂರ್ತಿ}
\footnotesize \mananamfont ನಮ್ಮಲ್ಲಿ ಅನೇಕರು ತಮ್ಮ ಇಂದ್ರಿಯ ಮನಸ್ಸು ಮತ್ತು ಪ್ರಚೋದನೆಗಳಿಂದ ಸೆಳೆಯಲ್ಪಟ್ಟ ಪ್ರಜ್ಞಾಹೀನ ನಿರರ್ಥಕ   ಜೀವನವನ್ನು ನಡೆಸುತ್ತಾರೆ. ಅಂತಹ ಜೀವನವು ನಿದ್ರೆಯಲ್ಲಿ  ಮುಳುಗಿರುವ ಹಾಗಿರುತ್ತದೆ. ಇನ್ನು ಕೆಲವರು ತಮ್ಮ ಆಸೆ ಮತ್ತು ಬಾವೋದ್ರೆಕಗಳಿಂದ ಪ್ರೇರೇಪಿಸಲ್ಪಟ್ಟಿರುತ್ತಾರೆ. ಆದರೆ ಅವರಿಗೆ ತಮ್ಮ ಅಂತಿಮ ಸಂತೋಷವೂ ಎಲ್ಲಿ ಅಡಗಿದೆ ಎಂದು ವಿವೇಚಿಸುವ ಶಕ್ತಿಯು ಇರುವುದಿಲ್ಲ. ನಿಜವಾಗಿಯೂ ಎಚ್ಚರವಾಗಿರುವುದೇನೆಂದರೆ ಜೀವನವನ್ನು ಚೆನ್ನಾಗಿ ಜೀವಿಸುವ ಕಲೆಯನ್ನು ಕಲಿಯುವುದು.
\end{inspiration}
\newpage

\slcol{\Index{ಆಪೂರ್ಯಮಾಣಮಚಲಪ್ರತಿಷ್ಠಂ} ಸಮುದ್ರಮಾಪಃ ಪ್ರವಿಶಂತಿ ಯದ್ವತ್ ।\\
ತದ್ವತ್ಕಾಮಾ ಯಂ ಪ್ರವಿಶಂತಿ ಸರ್ವೇ ಸ ಶಾಂತಿಮಾಪ್ನೋತಿ ನ ಕಾಮಕಾಮೀ ॥ 70 ॥}
\cquote{ಎಲ್ಲ ಕಡೆಯಿಂದಲೂ ನೀರು ಬರುತ್ತಿದ್ದರೂ ಅಲ್ಲಾಡದೆ ನೆಲೆಯಾಗಿರುವ ಸಮುದ್ರವನ್ನು ಹೊರಗಣ ನೀರುಗಳು ಹೇಗೆ ಸೇರಿ ಹೋಗುವುವೂ ಹಾಗೆ ಬಯಕೆಗಳೆಲ್ಲ ಯಾವನೊಳಗೆ ಸೇರಿ ಹೋಗುವುವೋ ಅವನು ಶಾಂತಿಯನ್ನು ಪಡೆಯುತ್ತಾನೆ. ಬಯಕೆಗಳ ಬೆನ್ನು ಹತ್ತುವನಿಗೆ ಎಂದೂ ಶಾಂತಿ ಇಲ್ಲ.\\}
\slcol{\Index{ವಿಹಾಯ ಕಾಮಾನ್ಯಃ ಸರ್ವಾ}ನ್ಪುಮಾಂಶ್ಚರತಿ ನಿಃಸ್ಪೃಹಃ ।\\
ನಿರ್ಮಮೋ ನಿರಹಂಕಾರಃ ಸ ಶಾಂತಿಮಧಿಗಚ್ಛತಿ ॥ 71 ॥}
\cquote{ಎಲ್ಲ ಕಾಮನೆಗಳನ್ನು ಬಿಟ್ಟು ಆಸೆಯೂ ಮಮಕಾರವೂ ಅಹಂಕಾರವೂ ಇಲ್ಲದ ಪುರುಷನು ಮುಕ್ತಿಯನ್ನು ಪಡೆಯಬಲ್ಲನು.}

\newpage
\begin{mananam}{\kanfont ಮನನ ಶ್ಲೋಕ - \textenglish{70, 71}}
\footnotesize \mananamfont ನನ್ನ ಜೀವನದಲ್ಲಿ ಗುರಿಗಳು ಮತ್ತು ಆಸೆಗಳಿಗೆ ಹೇಗೆ ಸಂಬಂಧಿಸುತ್ತೇನೆ. ನನ್ನ ಜೀವನದ ಉದಾತ್ತ   ಗುರಿಗಳನ್ನು ಪೂರ್ಣಗೊಳಿಸಲು ಕೆಲಸ ಮಾಡಬಹುದೇ? ಮತ್ತು ಅದರಿಂದ ಬರುವ ಅಂತಿಮ ಫಲಿತಾಂಶಗಳಿಗೆ ವಿಚಲಿತ ನಾಗದೇ ಇರಬಹುದೇ? ನನ್ನ ಆಸೆಗಳ ಸ್ವರೂಪವೇನು?ಅದು ಸ್ವಾರ್ಥ ಹಾನಿಕಾರಕ ಮತ್ತು ಅಹಂಕಾರದಿಂದ ಕೂಡಿದೆಯೇ?\\
ನನ್ನ ಈ ಗುರಿಗಳ ಹುಡುಕಾಟದಲ್ಲಿ ನಿಜವಾಗಿಯೂ ಎಲ್ಲವನ್ನೂ ಒಪ್ಪಿಕೊಳ್ಳುವುದು, ಆತ್ಮ ಗೌರವ, ಸ್ವಯಂ  ಅಂಗೀಕಾರ……. ಈ ತರಹ ಎಲ್ಲವನ್ನೂ ಪಡೆಯಬಹುದೇ? ನನ್ನ ಜೀವನದ ಸನ್ನಿವೇಶಗಳಲ್ಲಿ ಭಾಹ್ಯ ವಸ್ತುಗಳಿಂದ ಸಿಗುವ ತೃಪ್ತಿಗಿಂತ ಆಂತರಿಕವಾಗಿ ಕೃತಜ್ಞತೆ ಮನೋಭಾವವನ್ನು ಹೇಗೆ ಕಲಿಯಬಹುದು? ನನ್ನ ಉದ್ದೇಶವನ್ನು ಹೇಗೆ ವಿಶ್ವಕ್ಕೆ ಒಳ್ಳೆಯದಾಗುವ ರೀತಿ ತಿರುಗಿಸಬಹುದು?
\end{mananam}
\begin{inspiration}{\kanfont ಸ್ಪೂರ್ತಿ}
\footnotesize \mananamfont ಜೀವನದ ವಿವಿಧ ಅಗತ್ಯಗಳು ಮತ್ತು ಬೇಡಿಕೆಗಳು ಪ್ರತಿಯೊಬ್ಬರ ಮೇಲೆ ಹೇರುತ್ತೇವೆ. ಆದರೆ ಬುದ್ಧಿವಂತರು ಇಂದ್ರಿಯ ತೃಪ್ತಿ ಮತ್ತು ವಸ್ತುಗಳ ಸ್ವಾಧೀನಕೋಸ್ಕರ  ತಮ್ಮ ಕಾರ್ಯಗಳ ಮೇಲೆ ಕೇಂದ್ರೀಕರಿಸುವುದಿಲ್ಲ. ಏಕೆಂದರೆ ಅಂತಹ ಹಂಬಲಗಳಿಗೆ ಎಂದಿಗೂ ಕೊನೆಯಿಲ್ಲ ಎಂದು ಅವರಿಗೆ ತಿಳಿದಿದೆ. ತಮ್ಮ ಅಸ್ತಿತ್ವವು ಅನಂತ ಎಂದು ಯಾರೂ ಅರಿತಿರುವರೋ ಅವರಿಗೆ ಮಾತ್ರ ನಿಜವಾದ ಸಂತೃಪ್ತಿ.
\end{inspiration}
\newpage

\slcol{\Index{ಏಷಾ ಬ್ರಾಹ್ಮೀ ಸ್ಥಿತಿಃ ಪಾರ್ಥ} ನೈನಾಂ ಪ್ರಾಪ್ಯ ವಿಮುಹ್ಯತಿ ।\\
ಸ್ಥಿತ್ವಾಸ್ಯಾಮಂತಕಾಲೇऽಪಿ ಬ್ರಹ್ಮನಿರ್ವಾಣಮೃಚ್ಛತಿ ॥ 72 ॥}
\cquote{ಅರ್ಜುನ,ಇದು ಭಗವಂತನಲ್ಲಿ ನೆಲೆಗೊಂಡವನ ಬದುಕಿನ ರೀತಿ. ಈ ಸ್ಥಿತಿಯನ್ನು ಪಡೆದವರು ಮತ್ತೆ ದಾರಿ ತಪ್ಪುವುದಿಲ್ಲ. ಜೀವನದ ಕೊನೆಯ ಕ್ಷಣದ ತನಕ ಇದನ್ನು ಉಳಿಸಿಕೊಂಡವನು ಆನಂದಮಯವಾದ ಭಗವಂತನನ್ನು ಪಡೆಯುತ್ತಾನೆ.\\}
\begin{center}
{\tiny\color{brown}
ಓಂ ತತ್ಸದಿತಿ ಶ್ರೀಮದ್ಭಗವದ್ಗೀತಾಸೂಪನಿಷತ್ಸು \\
ಬ್ರಹ್ಮವಿದ್ಯಾಯಾಂ ಯೋಗಶಾಸ್ತ್ರೇ ಶ್ರೀಕೃಷ್ಣಾರ್ಜುನಸಂವಾದೇ\\
ಸಾಂಖ್ಯಯೋಗೋ ನಾಮ ದ್ವಿತೀಯೋऽಧ್ಯಾಯಃ ॥2 ॥\\
ಇತಿ ಶ್ರೀಮದ್ಭಗತಗೀತಾ ರೂಪೀ ,ಉಪನಿಷತ್,ಬ್ರಾಹ್ಮವಿದ್ಯಾ, ಯೋಗಾಶಾಸ್ತ್ರ ವಿಷಯವಾಗಿ ಶ್ರೀಕೃಷ್ಣ ಹಾಗು ಅರ್ಜುನರ ಸಂವಾದದಲ್ಲಿ ಸಂಖ್ಯಾಯೋಗ ಎಂಬ ಎರಡನೆಯ ಅಧ್ಯಾಯ ಸಂಪೂರ್ಣ.}
\end{center}
