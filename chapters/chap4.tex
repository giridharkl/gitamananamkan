\slcol{ಶ್ರೀಭಗವಾನುವಾಚ ।\\
\Index{ಇಮಂ ವಿವಸ್ವತೇ ಯೋಗಂ} ಪ್ರೋಕ್ತವಾನಹಮವ್ಯಯಮ್ ।\\
ವಿವಸ್ವಾನ್ಮನವೇ ಪ್ರಾಹಮನುರಿಕ್ಷ್ವಾಕವೇಽಬ್ರವೀತ್ ॥ ೧ ॥}
\cquote{ಶ್ರೀ ಭಗವಂತನು ಹೇಳಿದನು,\\
 ಅರ್ಜುನ, ಸನಾತನವಾದ ಈ ಯೋಗ ವಿದ್ಯೆಯನ್ನು ನಾನು ಮೊದಲು ಸೂರ್ಯನಿಗೆ ಹೇಳಿದೆನು ಸೂರ್ಯ ಮನುವಿಗೆ ಹೇಳಿದನು, ಮನು ಇಕ್ಷ್ವಾಕುವಿಗೆ  ಹೇಳಿದನು.}
\slcol{\Index{ಏವಂ ಪರಂಪರಾಪ್ರಾಪ್ತ}ಮಿಮಂ ರಾಜರ್ಷಯೋ ವಿದುಃ ।\\
ಸ ಕಾಲೇನೇಹ ಮಹತಾ ಯೋಗೋ ನಷ್ಟಃ ಪರಂತಪ ॥ ೨ ॥}
\cquote{ಅರ್ಜುನ, ಹೀಗೆ ಒಬ್ಬರಿಂದ ಒಬ್ಬರಿಗೆ ಬಂದ ಇದನ್ನು ರಾಜಶ್ರೀಗಳು ತಿಳಿದುಕೊಂಡರು. ಅಲ್ಲಿಂದ ಬಹಳ ಕಾಲ ಸಂದು ಈ ಯೋಗ ಪರಂಪರೆ ಕಾಲಗರ್ಭದಲ್ಲಿ ಮರೆಯಾಯಿತು.}
\slcol{\Index{ಸ ಏವಾಯಂ ಮಯಾ ತೇಽದ್ಯ} ಯೋಗಃ ಪ್ರೋಕ್ತಃ ಪುರಾತನಃ ।\\
ಭಕ್ತೋಽಸಿ ಮೇ ಸಖಾ ಚೇತಿ ರಹಸ್ಯಂ ಹ್ಯೇತದುತ್ತಮಮ್ ॥ ೩ ॥}
\cquote{ಬಹು ಹಿಂದಿನಿಂದ ಬಂದ ಈ ಯೋಗವನ್ನೇ ನೀನು ಭಕ್ತನೂ ಗೆಳೆಯನೂ ಆಗಿರುವುದರಿಂದ ನಿನಗೆ ಈಗ ಹೇಳಿದೆನು. ಇದು ಬಹು ರಹಸ್ಯವಾದ ವಿದ್ಯೆ.}


\newpage
\begin{mananam}{\mananamfont {ಮನನ ಶ್ಲೋಕ - ೧, ೨}}
\small \mananamtext ಪ್ರಾಚೀನ ಸಂಪ್ರದಾಯಗಳ ಬಗ್ಗೆ ನನ್ನ ತಿಳುವಳಿಕೆ ಏನು? ತಮ್ಮ ಜೀವನವನ್ನೇ   ಮುಡಿಪಾಗಿಟ್ಟ ಋಷಿ - ಆಚಾರ್ಯರ ಪರಂಪರೆಯಿಂದ ಅರ್ಪಿಸಲ್ಪಟ್ಟ ಈ ಬೋಧನೆಗಳ ಸಂರಕ್ಷಣೆಯ ಮಹತ್ತಿಗೆ ನಾನು ಇನ್ನಷ್ಟು ಮೆಚ್ಚುಗೆ ಹೇಗೆ ವ್ಯಕ್ತಪಡಿಸಲಿ?  ನನ್ನ ಜೀವನದಲ್ಲಿ ಈ ಬೋಧನೆಗಳನ್ನು  ಅಳವಡಿಸಿಕೊಳ್ಳುವ ಮೂಲಕ ಯಾವ ಪ್ರಯೋಜನಗಳನ್ನು ಪಡೆಯಬಹುದು ಅಥವಾ ನೋಡಲು ಆಶಿಸಬಹುದು. 
\end{mananam}
\WritingHand\enspace\textbf{ಆತ್ಮ ವಿಮರ್ಶೆ}\\
\begin{inspiration}{\mananamfont ಸ್ಫೂರ್ತಿ}
\small \mananamtext ಆಧುನಿಕ ಕಾಲದಲ್ಲಿ ಎಲ್ಲಕ್ಕೂ ಪುರಾವೆಯ ಅವಶ್ಯಕತೆ ಇರುವ ಸಮಾಜಕ್ಕೆ, ನಮ್ಮ ಪ್ರಾಚೀನ ಗ್ರಂಥಗಳಲ್ಲಿ ಕಾಣಿಸಿಕೊಳ್ಳುವ ಅನೇಕ ಪಾತ್ರಗಳ ಐತಿಹಾಸಿಕ ಸತ್ಯತೆಯನ್ನು ನಾವು ಪ್ರಶ್ನಿಸಬಹುದು. ಗೀತೆಗೆ ಪುರಾವೆಯನ್ನು ಇತಿಹಾಸ ಅಥವಾ ವಿಜ್ಞಾನದಲ್ಲಿ ಹುಡುಕಬಾರದು. ಬದಲಿಗೆ, ಗೀತೆಯ ಬೋಧನೆಗಳನ್ನು ನಮ್ಮ ನಿತ್ಯ ಜೀವನದಲ್ಲಿ ಅಳವಡಿಸಿಕೊಂಡು, ಅವು ನಮಗೆ ಮಾನಸಿಕ ಶಾಂತಿ ಹಾಗೂ ಸ್ಪಷ್ಟತೆ ನೀಡುವವೇ ಎಂಬುದರಲ್ಲಿ ಅದರ ಪುರಾವೆಯನ್ನು ಕಂಡುಕೊಳ್ಳಬೇಕು. ಅಪ್ರಮುಖ ಬಾಹ್ಯ ವಿಚಾರಗಳಲ್ಲಿ ನಮ್ಮ ಗಮನವನ್ನು ಕೇಂದ್ರೀಕರಿಸಿದರೆ, ನಮಗೆ ಹಸ್ತಾಂತರಿಸಲ್ಪಟ್ಟ ಅಂತರಂಗದ ಪರಿವರ್ತನೆಯ ಸಂದೇಶವನ್ನು ಕಳೆದುಕೊಳ್ಳುತ್ತೇವೆ.
\end{inspiration}
\newpage

\slcol{ಅರ್ಜುನ ಉವಾಚ ।\\
\Index{ಅಪರಂ ಭವತೋ ಜನ್ಮ} ಪರಂ ಜನ್ಮ ವಿವಸ್ವತಃ ।\\
ಕಥಮೇತದ್ವಿಜಾನೀಯಾಂ ತ್ವಮಾದೌ ಪ್ರೋಕ್ತವಾನಿತಿ ॥ ೪ ॥}
\cquote{ಅರ್ಜುನನು ಹೇಳಿದನು,\\
ನೀನು ಇತ್ತೀಚಿಗೆ ಜನಿಸಿದವನು, ಸೂರ್ಯನು ತುಂಬಾ ಪುರಾತನದವನು. ಹೀಗಿರುವಾಗ ನೀನು ಇದನ್ನು ಮೊದಲು ಹೇಳಿದವನೆಂದು ನಾನು ಹೇಗೆ ತಿಳಿಯಲಿ?}
\slcol{ಶ್ರೀಭಗವಾನುವಾಚ ।\\
\Index{ಬಹೂನಿ ಮೇ ವ್ಯತೀತಾನಿ} ಜನ್ಮಾನಿ ತವ ಚಾರ್ಜುನ ।\\
ತಾನ್ಯಹಂ ವೇದ ಸರ್ವಾಣಿ ನ ತ್ವಂ ವೇತ್ಥ ಪರಂತಪ ॥ ೫ ॥}
\cquote{ಶ್ರೀ ಭಗವಂತನು ಹೇಳಿದನು,\\
 ಅರ್ಜುನ, ನಿನಗೂ ನನಗೂ ಅನೇಕ ಜನ್ಮಗಳು ಆಗಿ ಹೋದವು, ನಾನು ಆ ಎಲ್ಲವನ್ನು ಬಲ್ಲೆನು. ನಿನಗೆ ಮಾತ್ರ ತಿಳಿದಿಲ್ಲ.}
\slcol{\Index{ಅಜೋಽಪಿ ಸನ್ನವ್ಯಯಾತ್ಮಾ} ಭೂತಾನಾಮೀಶ್ವರೋಽಪಿ ಸನ್ ।\\
ಪ್ರಕೃತಿಂ ಸ್ವಾಮಧಿಷ್ಠಾಯ ಸಂಭವಾಮ್ಯಾತ್ಮಮಾಯಯಾ ॥ ೬ ॥}
\cquote{ನನಗೆ ಹುಟ್ಟಿಲ್ಲ ನನ್ನ ದೇಹಕ್ಕೆ ಅಳಿವಿಲ್ಲ ನಾನು ಎಲ್ಲ ಪ್ರಾಣಿಗಳ ಒಡೆಯ ನನ್ನ ಮಾಯಾ ಶಕ್ತಿಯನ್ನು ಮುಂದಿಟ್ಟುಕೊಂಡು ಸ್ವೇಚ್ಛೆಯಿಂದ  ನಾನು ಹುಟ್ಟುತ್ತೇನೆ.}
\slcol{\Index{ಯದಾ ಯದಾ ಹಿ ಧರ್ಮಸ್ಯ} ಗ್ಲಾನಿರ್ಭವತಿ ಭಾರತ ।\\
ಅಭ್ಯುತ್ಥಾನಮಧರ್ಮಸ್ಯ ತದಾತ್ಮಾನಂ ಸೃಜಾಮ್ಯಹಮ್ ॥ ೭ ॥}
\cquote{ಭಾರತ, ಯಾವ ಯಾವ ಕಾಲದಲ್ಲಿ ಧರ್ಮದ ಇಳುಗಡೆಯೂ ಅಧರ್ಮದ ಏರಿಕೆಯೂ ಆಗುತ್ತದೋ ಆಗೆಲ್ಲ ನಾನು ಅವತರಿಸಿ ಬರುತ್ತೇನೆ.}
\slcol{\Index{ಪರಿತ್ರಾಣಾಯ ಸಾಧೂನಾಂ} ವಿನಾಶಾಯ ಚ ದುಷ್ಕೃತಾಮ್ ।\\
ಧರ್ಮಸಂಸ್ಥಾಪನಾರ್ಥಾಯ ಸಂಭವಾಮಿ ಯುಗೇ ಯುಗೇ ॥ ೮ ॥}
\cquote{ಒಳ್ಳೆಯವರ ಉದ್ಧಾರಕ್ಕಾಗಿ ಕೆಟ್ಟವರ ನಿರ್ನಾಮಕ್ಕಾಗಿ ಮತ್ತು ಧರ್ಮವನ್ನು ನೆಲೆಗೊಳಿಸುವುದಕ್ಕಾಗಿ ಪ್ರತಿಯೊಂದು ಯುಗದಲ್ಲಿಯೂ ನಾನು ಹುಟ್ಟಿ ಬರುತ್ತೇನೆ.}


\newpage
\begin{mananam}{\mananamfont ಮನನ ಶ್ಲೋಕ - ೫}
\small \mananamtext ಪುನರ್ಜನ್ಮದ  ಸಿದ್ಧಾಂತವನ್ನು ನಾನು ಹೇಗೆ ಅರ್ಥ ಮಾಡಿಕೊಂಡಿದ್ದೇನೆ? ಅದರ ಬಗ್ಗೆ ನನ್ನ ಅನುಮಾನಗಳೇನು?  ಪುನರ್ಜನ್ಮ ಮತ್ತು ಕರ್ಮಸಿದ್ಧಾಂತವನ್ನು ಅರ್ಥ ಮಾಡಿಕೊಂಡರೆ, ಈ ಜೀವನದಲ್ಲಿ ನನ್ನ ಕ್ರಿಯೆಗಳನ್ನು ಮಾರ್ಗದರ್ಶಿಸಲು ಅದು ಉಪಯುಕ್ತವಾಗಬಹುದೇ? ಸಮಾಜದಲ್ಲಿ ಅಡಕವಾದ ಸಮತೋಲನ ಮತ್ತು ನ್ಯಾಯವನ್ನು ನೋಡಲು ಈ ಸಿದ್ಧಾಂತವು ನನಗೆ ಸಹಾಯ ಮಾಡಬಹುದೇ?
\end{mananam}
\WritingHand\enspace\textbf{ಆತ್ಮ ವಿಮರ್ಶೆ}\\
\begin{inspiration}{\mananamfont ಸ್ಫೂರ್ತಿ}
\small \mananamtext ಯಾವುದೇ ಕ್ಷೇತ್ರದಲ್ಲಿರುವ ಬಾಲ ಪ್ರತಿಭೆಗಳನ್ನು ನೋಡಿದಾಗ ಪುನರ್ಜನ್ಮವನ್ನು ನಂಬುವುದು ಕಷ್ಟವೇನಲ್ಲ. ಅವರ ಪ್ರತಿಭೆ ಮತ್ತು ಕೌಶಲ್ಯಗಳು ಹೇಗೆ ಉಂಟಾದವು? ಪ್ರತಿಯೊಂದು ಜನ್ಮದಲ್ಲಿಯೂ ನಮ್ಮ ಸ್ಮರಣೆಯು ಅಳಿಸಿರುವುದರಿಂದ ಈ ಜೀವನವನ್ನು ಹೊಸದಾಗಿ ಪ್ರಾರಂಭಿಸುವ ಮತ್ತು ನಕರಾತ್ಮಕ ಪ್ರವೃತ್ತಿಗಳನ್ನು ಬದಲಾಯಿಸಿಕೊಳ್ಳಲು ಸಹಾಯವಾಗುತ್ತದೆ. ಭೂತಕಾಲವು ವರ್ತಮಾನದ ನೀಲನಕ್ಷೆಯಾಗಿದ್ದರೆ, ಆಗ ವರ್ತಮಾನವು ಉತ್ತಮ ಭವಿಷ್ಯವನ್ನು ರೂಪಿಸಲು ನಮಗೆ ಅವಕಾಶವನ್ನು ನೀಡುತ್ತದೆ.\\
\end{inspiration}
\newpage
\newpage
\begin{mananam}{\mananamfont {ಮನನ ಶ್ಲೋಕ - ೭, ೮}}
\small \mananamtext ಈ ಸಮಾಜದಲ್ಲಿ ನಡೆಯುವ ಅನ್ಯಾಯಗಳಿಗೆ ನನ್ನ ಪ್ರತಿಕ್ರಿಯೆ ಏನು? ಒಬ್ಬ ವ್ಯಕ್ತಿ ಅಥವಾ ಒಂದು ಗುಂಪು ಅನ್ಯಾಯ ಮಾಡುತ್ತಿದ್ದರೂ ನಿರ್ಭಯವಾಗಿ ತಪ್ಪಿಸಿಕೊಳ್ಳುವುದನ್ನು ನೋಡಿದಾಗ ನನಗೆ ಹೇಗೆ ಅನ್ನಿಸುತ್ತದೆ? ಭಗವಂತನ  ವಾಗ್ದಾನದಲ್ಲಿ ನಾನು ಸಮಾಧಾನವನ್ನು ಕಂಡುಕೊಳ್ಳಬಹುದೇ? ಬಾಹ್ಯಬದಲಾವಣೆಯು ಸಾಧ್ಯವಾಗದಂತಹ ಪರಿಸ್ಥಿತಿಗಳಲ್ಲಿ ಅಂತರಂಗದಲ್ಲಿ ಸ್ವೀಕಾರದ ಭಾವವನ್ನು ಕಂಡುಕೊಳ್ಳಲು ನಾನು ಇದನ್ನು ಬಳಸಬಹುದೇ?
\end{mananam}
\WritingHand\enspace\textbf{ಆತ್ಮ ವಿಮರ್ಶೆ}\\
\begin{inspiration}{\mananamfont ಸ್ಫೂರ್ತಿ}
\small \mananamtext ನಾವು ನಂಬುವ ಉನ್ನತ ಶಕ್ತಿಯ ಪರಿಕಲ್ಪನೆ ಪ್ರೇಮಮಯ ಮತ್ತು ದಯಾಮಯವಲ್ಲದೇ, ಶಕ್ತಿಶಾಲಿ ಮತ್ತು ನ್ಯಾಯ ನಿಷ್ಠವಾಗಿಯೂ ಇರಬೇಕು. ನಮ್ಮೊಳಗೆ ಭರವಸೆ ಮತ್ತು ಶಕ್ತಿಯನ್ನು ತುಂಬುವ ಅಂಥಹ ದೇವರು, ಆಶ್ರಯದ ಆದರ್ಶ ಮೂಲವಾಗಬಹುದು. 
\end{inspiration}
\newpage

\slcol{\Index{ಜನ್ಮ ಕರ್ಮ ಚ ಮೇ ದಿವ್ಯ}ಮೇವಂ ಯೋ ವೇತ್ತಿ ತತ್ತ್ವತಃ ।\\
ತ್ಯಕ್ತ್ವಾ ದೇಹಂ ಪುನರ್ಜನ್ಮ ನೈತಿ ಮಾಮೇತಿ ಸೋಽರ್ಜುನ ॥ ೯ ॥}
\cquote{ದಿವ್ಯವಾದ ನನ್ನ ಹುಟ್ಟನ್ನೂ ಕರ್ಮವನ್ನೂ ಹೀಗೆ ಯಥಾವತ್ತಾಗಿ ತಿಳಿದವನು ದೇಹವನ್ನು ಬಿಟ್ಟ ಮೇಲೆ ಮತ್ತೆ ಹುಟ್ಟನ್ನು ಹೊಂದುವುದಿಲ್ಲ. ಅವನು ನನ್ನನ್ನು ಪಡೆಯುತ್ತಾನೆ.}
\slcol{\Index{ವೀತರಾಗಭಯಕ್ರೋಧಾ} ಮನ್ಮಯಾ ಮಾಮುಪಾಶ್ರಿತಾಃ ।\\
ಬಹವೋ ಜ್ಞಾನತಪಸಾ ಪೂತಾ ಮದ್ಭಾವಮಾಗತಾಃ ॥ ೧೦ ॥}
\cquote{ಪ್ರೀತಿ ಹೆದರಿಕೆ ಸಿಟ್ಟು ಇವುಗಳಿಲ್ಲದವನಾಗಿ ನನ್ನಲ್ಲಿಯೇ ಮನಸ್ಸಿಟ್ಟು ನನಗೆ ಶರಣು ಬಂದ ಅನೇಕರು ಜ್ಞಾನವೆಂಬ ತಪಸ್ಸಿನಿಂದ ಶುದ್ಧರಾಗಿ ನನ್ನನ್ನು ಪಡೆದಿರುತ್ತಾರೆ.}
\slcol{\Index{ಯೇ ಯಥಾ ಮಾಂ ಪ್ರಪದ್ಯಂತೇ} ತಾಂಸ್ತಥೈವ ಭಜಾಮ್ಯಹಮ್ ।\\
ಮಮ ವರ್ತ್ಮಾನುವರ್ತಂತೇ ಮನುಷ್ಯಾಃ ಪಾರ್ಥ ಸರ್ವಶಃ ॥ ೧೧ ॥}
\cquote{ಅರ್ಜುನ, ನನ್ನನ್ನು ಯಾರು ಹೇಗೆ ಶರಣೆಂದರೆ ನಾನು ಅವರಿಗೆ ಹಾಗೆ ಅನುಗ್ರಹಿಸುತ್ತೇನೆ. ಮನುಷ್ಯರು ಎಲ್ಲಿ ಯಾವ ದಾರಿಯಲ್ಲಿ ಹೋದರೂ ಕೊನೆಗೆ ಬಂದು ಸೇರುವುದು ನನ್ನತ್ತವೆ.}
\slcol{\Index{ಕಾಂಕ್ಷಂತಃ ಕರ್ಮಣಾಂ} ಸಿದ್ಧಿಂ ಯಜಂತ ಇಹ ದೇವತಾಃ ।\\
ಕ್ಷಿಪ್ರಂ ಹಿ ಮಾನುಷೇ ಲೋಕೇ ಸಿದ್ಧಿರ್ಭವತಿ ಕರ್ಮಜಾ ॥ ೧೨ ॥}
\cquote{ಕರ್ಮಗಳಿಗೆ ಫಲವನ್ನು ಬಯಸುವವರಾಗಿ ದೇವತೆಗಳನ್ನು ಆರಾಧಿಸುತ್ತಾರೆ. ಏಕೆಂದರೆ ಮನುಷ್ಯ ಲೋಕದಲ್ಲಿ ಕರ್ಮದಿಂದ ಸಿದ್ಧಿ ಬಹುಬೇಗ ಆಗುತ್ತದೆ.}
\slcol{\Index{ಚಾತುರ್ವರ್ಣ್ಯಂ ಮಯಾ} ಸೃಷ್ಟಂ ಗುಣಕರ್ಮವಿಭಾಗಶಃ ।\\
ತಸ್ಯ ಕರ್ತಾರಮಪಿ ಮಾಂ ವಿದ್ಧ್ಯಕರ್ತಾರಮವ್ಯಯಮ್ ॥ ೧೩ ॥}
\cquote{ಗುಣಕ್ಕೂ ಕರ್ಮಕ್ಕೂ ತಕ್ಕಂತೆ ನಾಲ್ಕು ವರ್ಣಗಳನ್ನು ನಾನು ಸೃಷ್ಟಿಸಿದ್ದೇನೆ. ನಾನು ಅದನ್ನು ಮಾಡಿದವನಾದರೂ ನಾನು ಕರ್ತನಲ್ಲ. ಏಕೆಂದರೆ ನನಗೆ ಕರ್ತೃತ್ವದ  ಲೇಪವಿಲ್ಲ.}


\newpage
\begin{mananam}{\mananamfont ಮನನ ಶ್ಲೋಕ - ೧೦}
\small \mananamtext ಭಗವಂತನನ್ನು ಪಡೆಯುವುದು ದೈವೀಕ ಸ್ಥಿತಿಯಲ್ಲಿರುವುದನ್ನು ಸಂಕೇತಿಸುತ್ತದೆಯೇ? ಭಯ, ಕೋಪ ಮತ್ತು ಇಂತಹ ಭಾವನೆಗಳು ಕಾಡದ ಮನೋಸ್ಥಿತಿಗೆ  ನಾನು ಸ್ಪಂದಿಸಬಹುದೇ? ಜ್ಞಾನದ ಅಗ್ನಿಯಿಂದ ಶುದ್ಧವಾಗುವುದು ಎಂದರೆ ಏನು? ಬಲವಾದ ಭಾವನೆಗಳು ನನ್ನ ವ್ಯಕ್ತಿತ್ವದ ಕರಾಳ ಮುಖವನ್ನು ತೋರಿಸಿದಾಗ ಜ್ಞಾನವು ನನಗೆ ಹೇಗೆ ಸಹಾಯ ಮಾಡಬಲ್ಲದು?
\end{mananam}
\WritingHand\enspace\textbf{ಆತ್ಮ ವಿಮರ್ಶೆ}\\
\begin{inspiration}{\mananamfont ಸ್ಫೂರ್ತಿ}
\small \mananamtext ಸುಪ್ತ ಕ್ಲೇಷಗಳು (ಅಸ್ಮಿತಾ, ರಾಗ, ದ್ವೇಷ, ಅಭಿನಿವೇಶ) ಜಾಗೃತವಾದಾಗ, ಒಂದು ಬೇರೆಯೇ ರೀತಿಯಂತಹ ವ್ಯಕ್ತಿತ್ವದ ಹಿಡಿತಕ್ಕೆ ನಾವು ಒಳಗಾಗುತ್ತೇವೆ. ಉನ್ನತ ವಿವೇಕದಿಂದ ಕಾರ್ಯ ನಿರ್ವಹಿಸುವ ನಮ್ಮ ಸಾಮರ್ಥ್ಯವು ಸುಪ್ತವಾಗುತ್ತದೆ, ಇದು ನಮ್ಮದೇ ನೈಜ ಸ್ಥಿತಿಯ ಬಗ್ಗೆ ನಮಗೆ ಪ್ರಜ್ಞೆ ಇಲ್ಲವೇನೋ ಎಂಬಂತೆ ಮಾಡುತ್ತದೆ. ಪ್ರಜ್ಞೆಯೇ ಜ್ಞಾನದ ಮೊದಲ ಬೆಳಕು; ಅದು  ನಿಜವಾದ ಮನೆಯಾದ ನಮ್ಮ ನೈಜ ಸ್ವರೂಪಕ್ಕೆ ಮರಳಿ ಮಾರ್ಗದರ್ಶಿಸುತ್ತದೆ.
\end{inspiration}
\newpage

\newpage
\begin{mananam}{\mananamfont ಮನನ ಶ್ಲೋಕ - ೧೧}
\small \mananamtext ದೇವರು ಅಥವಾ ಪರಮ ಅಸ್ತಿತ್ವದ ಬಗ್ಗೆ ನನ್ನ ಪರಿಕಲ್ಪನೆ ಏನು? ಬ್ರಹ್ಮಾಂಡದ ಒಂದು ಉನ್ನತ ಶಕ್ತಿಯಲ್ಲಿ ನನಗೆ ನಂಬಿಕೆ ಹೊಂದಲು ಸಾಧ್ಯವೇ ಅಥವಾ,  ನನ್ನೊಳಗೇ ಇರುವ ಪರಿಶುದ್ಧ ಆತ್ಮನಲ್ಲಿಯೇ? ಪ್ರಪಂಚದ ಪ್ರತಿಯೊಬ್ಬರೂ ದೇವರ ಪರಿಕಲ್ಪನೆಯಲ್ಲಿ ಹೇಗೆ ತಮ್ಮದೇ ಆದ ನಂಬಿಕೆಯನ್ನು ಹೊಂದಿದ್ದಾರೆ ಎಂಬುದನ್ನು ನಾನು ಅವಲೋಕನದಿಂದ ಕಲಿಯಬಹುದೇ?ಯಾವುದೇ ಪರಿಕಲ್ಪನೆಯಲ್ಲಿ ನಂಬಿಕೆಯು ಹೇಗೆ ಅಭಿವೃದ್ಧಿ ಹೊಂದಿ, ಉಳಿದು, ಬಲವಾಗಿ ಬೆಳೆಯುತ್ತದೆ ಎಂಬುದನ್ನು ನನ್ನದೇ  ಜೀವನದಲ್ಲಿ  ಗಮನಿಸಬಹುದೇ?
\end{mananam}
\WritingHand\enspace\textbf{ಆತ್ಮ ವಿಮರ್ಶೆ}\\
\begin{inspiration}{\mananamfont ಸ್ಫೂರ್ತಿ}
\small \mananamtext ಪರಮಾತ್ಮನ ಯಾವುದೇ ಪರಿಕಲ್ಪನೆಯನ್ನು ಆಶ್ರಯಿಸಿದರೂ, “ಪ್ರತಿಫಲ ನೀಡುವ ಮೂಲಕ ಪ್ರತಿಯೊಬ್ಬರ ನಂಬಿಕೆಯನ್ನೂ ಬಲಪಡಿಸುತ್ತೇನೆ “ ಎಂಬ ಭಗವಂತನ ಘೋಷಣೆ ವಿರೋಧಾಭಾಸವಾಗಿ ತೋರಲೂಬಹುದು.ಸನಾತನ ಧರ್ಮದ ದೇವರು ಅಸೂಯಾಪರನಲ್ಲ; ತಮ್ಮದೇ ನಂಬಿಕೆಯನ್ನು ಅನುಸರಿಸಲು ಆತನು ಪ್ರತಿಯೊಬ್ಬರಿಗೂ ಶಕ್ತಿ ನೀಡುತ್ತಾನೆ. ಈ ಪರಮ ಸತ್ಯದ ಅನುಯಾಯಿಗಳಾಗಿ, ಭಿನ್ನ ರೀತಿಯ ನಂಬಿಕೆಗಳನ್ನು ಹೊಂದಿರುವವರತ್ತ ನಾವು, ತೆರೆದ-ಹೃದಯದವರಗಿಯೂ ಸಹಿಷ್ಣುಗಳಾಗಿಯೂ ಇರಬೇಕು.
\end{inspiration}
\newpage

\slcol{\Index{ನ ಮಾಂ ಕರ್ಮಾಣಿ ಲಿಂಪಂತಿ} ನ ಮೇ ಕರ್ಮಫಲೇ ಸ್ಪೃಹಾ ।\\
ಇತಿ ಮಾಂ ಯೋಽಭಿಜಾನಾತಿ ಕರ್ಮಭಿರ್ನ ಸ ಬಧ್ಯತೇ ॥ ೧೪ ॥}
\cquote{ನನ್ನನ್ನು ಕರ್ಮಗಳು ಸೋಂಕುವುದಿಲ್ಲ, ನನಗೆ ಕರ್ಮದ ಫಲದಾಸೆ ಇಲ್ಲ, ನನ್ನನ್ನು ಹೀಗೆ ಯಾವನು ತಿಳಿಯುತ್ತಾನೋ ಅವನು ಕರ್ಮದ ಕಟ್ಟಿಗೆ ಒಳಗಾಗುವುದಿಲ್ಲ.}
\slcol{\Index{ಏವಂ ಜ್ಞಾತ್ವಾ ಕೃತಂ} ಕರ್ಮ ಪೂರ್ವೈರಪಿ ಮುಮುಕ್ಷುಭಿಃ ।\\
ಕುರು ಕರ್ಮೈವ ತಸ್ಮಾತ್ತ್ವಂ ಪೂರ್ವೈಃ ಪೂರ್ವತರಂ ಕೃತಮ್ ॥ ೧೫ ॥}
\cquote{ಮೋಕ್ಷವನ್ನು ಬಯಸಿದ ಹಿಂದಿನವರು ಕೂಡ ಹೀಗೆ ತಿಳಿದು ಕರ್ಮವನ್ನು ಮಾಡಿದರು. ಆದ್ದರಿಂದ ಹಿಂದಿನವರು ಕೂಡ ಮಾಡಿದ ಪುರಾತನವಾದ ಕರ್ಮ ಮಾರ್ಗವನ್ನೇ ನೀನು ಹಿಡಿ.}
\slcol{\Index{ಕಿಂ ಕರ್ಮ ಕಿಮಕರ್ಮೇತಿ} ಕವಯೋಽಪ್ಯತ್ರ ಮೋಹಿತಾಃ ।\\
ತತ್ತೇ ಕರ್ಮ ಪ್ರವಕ್ಷ್ಯಾಮಿ ಯಜ್ಜ್ಞಾತ್ವಾ ಮೋಕ್ಷ್ಯಸೇಽಶುಭಾತ್ ॥ ೧೬ ॥}
\cquote{ಕರ್ಮವು ಯಾವುದು ಮತ್ತು ಅಕರ್ಮವು ಯಾವುದು? ಈ ಪ್ರಕಾರವಾಗಿ ಇದರ ನಿರ್ಣಯ ಮಾಡುವಲ್ಲಿ ಬುದ್ಧಿವಂತರಾದ ಪುರುಷರು ಸಹ ಮೋಹಿತರಾಗುತ್ತಾರೆ. ಆದ್ದರಿಂದ ಆ ಕರ್ಮತತ್ವವನ್ನು ನಾನು ನಿನಗೆ ಚೆನ್ನಾಗಿ ತಿಳಿಯ ಹೇಳುವೆನು.ಅದನ್ನು ತಿಳಿದು ನೀನು ಅಶುಭದಿಂದ ಅರ್ಥಾತ್ ಕರ್ಮ ಬಂಧನದಿಂದ ಮುಕ್ತನಾಗಿಬಿಡುವೆ.}
\slcol{\Index{ಕರ್ಮಣೋ ಹ್ಯಪಿ ಬೋದ್ಧವ್ಯಂ} ಬೋದ್ಧವ್ಯಂ ಚ ವಿಕರ್ಮಣಃ ।\\
ಅಕರ್ಮಣಶ್ಚ ಬೋದ್ಧವ್ಯಂ ಗಹನಾ ಕರ್ಮಣೋ ಗತಿಃ ॥ ೧೭ ॥}
\cquote{ಕರ್ಮದ ಸ್ವರೂಪವನ್ನು ಕೂಡ ತಿಳಿಯಬೇಕು ಮತ್ತು ಅಕರ್ಮದ ಸ್ವರೂಪವನ್ನೂ ಕೂಡ ತಿಳಿಯಬೇಕು. ಹಾಗೆಯೇ ವಿಕರ್ಮದ ಸ್ವರೂಪವನ್ನು ಸಹ ತಿಳಿಯಬೇಕು. ಏಕೆಂದರೆ ಕರ್ಮದ ಗತಿಯು ಗಹನವಾಗಿದೆ.}


\newpage
\begin{mananam}{\mananamfont ಮನನ ಶ್ಲೋಕ - ೧೪}
\small \mananamtext ನನ್ನ ಕರ್ತವ್ಯಗಳನ್ನು ಮಾಡುವಾಗ, ನನ್ನ ಮನಸ್ಸಿನಲ್ಲಿ, ಆ ಕಾರ್ಯಗಳ ಸಣ್ಣ ಚಿಹ್ನೆಯೂ  ಉಳಿಯದಂತೆ ಎಚ್ಚರದಿಂದಿರುವ ಮಹತ್ವವನ್ನು ನಾನು ಅರಿತುಕೊಳ್ಳುವೆನೇ?ನಾನು ಮಾಡುವ ಯಾವುದೇ ಕಾರ್ಯಗಳ ಪರಿಣಾಮಗಳಿಂದ ಮುಕ್ತನಾಗಿದ್ದೇನೆ’ ಎಂಬುದು ನನಗೆ ಏನು ಅರ್ಥ ಕೊಡುತ್ತದೆ?ನನ್ನ ಎಲ್ಲಾ ಕಾಮನೆಗಳಲ್ಲೂ ವೈಯುಕ್ತಿಕ ತೃಪ್ತಿಗಾಗಿ ಎಡೆಗೊಡುತ್ತೇನೆಯೇ? ವೈಯಕ್ತಿಕ ತೃಪ್ತಿಯನ್ನು ಅರಸುವುದರಲ್ಲಿ ಬೀಸಿರುವ ಬಲೆ ಏನು?
\end{mananam}
\WritingHand\enspace\textbf{ಆತ್ಮ ವಿಮರ್ಶೆ}\\
\begin{inspiration}{\mananamfont ಸ್ಫೂರ್ತಿ}
\small \mananamtext ಭಗವಂತನು ತಾನು ಅನುಸರಿಸುವ ಸ್ವಾತಂತ್ರ್ಯದ ಸೂತ್ರವನ್ನೇ ನಮ್ಮೊಂದಿಗೂ ಹಂಚಿಕೊಂಡಿದ್ದಾನೆ. ಒಬ್ಬನು ಫಲದ ಮೋಹದಿಂದ ಮುಕ್ತನಾದಾಗ, ಅವನು ಕೈಗೊಳ್ಳುವ ಯಾವುದೇ ಕ್ರಿಯೆಯ ಚಿಹ್ನೆಯೂ ಅದರಿಂದಾಗುವ ಪರಿಣಾಮಗಳೂ, ಅವನ ಮನಸ್ಸಿನಲ್ಲಿ ಅಥವಾ ಕರ್ಮದ ನೀಲನಕ್ಷೆಯಲ್ಲಾಗಲೀ ಉಳಿಯುವುದಿಲ್ಲ.
\end{inspiration}
\newpage

\newpage
\begin{mananam}{\mananamfont {ಮನನ ಶ್ಲೋಕ - ೧೬, ೧೭}}
\small \mananamtext ಸರಿ ಮತ್ತು ತಪ್ಪು ಕ್ರಿಯೆಗಳ ನಡುವಿನ ವ್ಯತ್ಯಾಸವನ್ನು ನಾನು ಹೇಗೆ ಗುರುತಿಸಬಹುದು?ಅದಕ್ಕೆ ಸಂಬಂಧಿಸಿದ ನನ್ನ ಮಾಪಕಗಳು ಯಾವುವು?ಅವು ಕೇವಲ ನನ್ನ ಮತ್ತು ನನ್ನ ಪ್ರೀತಿಪಾತ್ರರ ಹಿತವನ್ನಷ್ಟೇ  ಆಧರಿಸಿವೆಯೇ ಅಥವಾ ಸಂಪೂರ್ಣ ವಿಶ್ವದ ಮೌಲ್ಯಗಳನ್ನು ಆಧರಿಸಿವೆಯೇ? ನನ್ನ ದೈನಂದಿನ ಕಾರ್ಯವಿಧಾನಗಳನ್ನು, ಆಧ್ಯಾತ್ಮಿಕ ಮೌಲ್ಯಗಳಾಧಾರಿತ ಕಾರ್ಯಗಳಾಗಿ ಹೇಗೆ ರೂಪಿಸಬಲ್ಲೆ? ನನ್ನ ಜೀವನದಲ್ಲಿ ಕಠಿಣ ನಿರ್ಧಾರಗಳನ್ನು ಕೈಗೊಳ್ಳಬೇಕಾದಾಗ ನಾನು ನನ್ನ ಅಂತಃಚೇತನವನ್ನು ಆಲಿಸುತ್ತೇನೆಯೇ?
\end{mananam}
\WritingHand\enspace\textbf{ಆತ್ಮ ವಿಮರ್ಶೆ}\\
\begin{inspiration}{\mananamfont ಸ್ಫೂರ್ತಿ}
\small \mananamtext ತಪ್ಪು ಕ್ರಿಯೆಗಳ ಫಲಿತಾಂಶಗಳು – ವ್ಯಸನಗಳು, ಒತ್ತಡ, ನಕಾರಾತ್ಮಕತೆ ಹಾಗೂ ಸಂಘರ್ಷಗಳನ್ನು ಇದೇ ಜೀವನದಲ್ಲಿಯೇ ಉಂಟುಮಾಡುತ್ತವೆ. ಸರಿಯಾದ ಕ್ರಿಯೆಗಳ ಫಲಿತಾಂಶಗಳು, ಈ ಜೀವನದಲ್ಲಿಯೂ, ಇರಬಹುದಾದ ಭವಿಷ್ಯದ ಜೀವನಗಳಲ್ಲಿಯೂ, ಮಾನಸಿಕವಾಗಿ ಉನ್ನತಿಗೇರಿಸುತ್ತವೆ.
\end{inspiration}
\newpage

\slcol{\Index{ಕರ್ಮಣ್ಯಕರ್ಮ ಯಃ ಪಶ್ಯೇದ}ಕರ್ಮಣಿ ಚ ಕರ್ಮ ಯಃ ।\\
ಸ ಬುದ್ಧಿಮಾನ್ಮನುಷ್ಯೇಷು ಸ ಯುಕ್ತಃ ಕೃತ್ಸ್ನಕರ್ಮಕೃತ್ ॥ ೧೮ ॥}
\cquote{ಯಾವ ಮನುಷ್ಯನು ಕರ್ಮದಲ್ಲಿ ಅಕರ್ಮವನ್ನು ನೋಡುತ್ತಾನೋ ಮತ್ತು ಯಾರು ಅಕರ್ಮದಲ್ಲಿ ಕರ್ಮವನ್ನು ನೋಡುತ್ತಾನೋ ಅವನು ಮನುಷ್ಯರಲ್ಲಿ ಬುದ್ಧಿವಂತನಾಗಿದ್ದಾನೆ. ಮತ್ತು ಆ ಯೋಗಿಯು ಸಮಸ್ತ ಕರ್ಮಗಳನ್ನು ಮಾಡುವವನಾಗಿದ್ದಾನೆ.}
\slcol{\Index{ಯಸ್ಯ ಸರ್ವೇ ಸಮಾರಂಭಾಃ} ಕಾಮಸಂಕಲ್ಪವರ್ಜಿತಾಃ ।\\
ಜ್ಞಾನಾಗ್ನಿದಗ್ಧಕರ್ಮಾಣಂ ತಮಾಹುಃ ಪಂಡಿತಂ ಬುಧಾಃ ॥ ೧೯ ॥}
\cquote{ಫಲದಾಸೆಯನ್ನೂ ಅಭಿಮಾನವನ್ನೂ ತೊರೆದು ಕರ್ಮ ಮಾಡುವವನು ತಿಳುವಿನ ಬೆಂಕಿಯಿಂದ ಕರ್ಮಗಳನ್ನು ಸುಟ್ಟುಕೊಂಡವನು. ತಿಳಿದವರು ಅವನನ್ನು ಆತ್ಮಜ್ಞಾನಿ ಎನ್ನುವರು.}
\slcol{\Index{ತ್ಯಕ್ತ್ವಾ ಕರ್ಮಫಲಾಸಂಗಂ} ನಿತ್ಯತೃಪ್ತೋ ನಿರಾಶ್ರಯಃ ।\\
ಕರ್ಮಣ್ಯಭಿಪ್ರವೃತ್ತೋಽಪಿ ನೈವ ಕಿಂಚಿತ್ಕರೋತಿ ಸಃ ॥ ೨೦ ॥}
\cquote{ಫಲದ ಆಸೆಯನ್ನು ತೊರೆದು ತೃಪ್ತನಾಗಿ ಯಾವ ಹಂಗೂ ಇಲ್ಲದೆ ಕರ್ಮಗಳನ್ನು ಮಾಡಿದರೂ ಅವನು ಕರ್ಮದ ಬಂಧನದಿಂದ ದೂರವಾಗಿರುತ್ತಾನೆ.}
\slcol{\Index{ನಿರಾಶೀರ್ಯತಚಿತ್ತಾತ್ಮಾ} ತ್ಯಕ್ತಸರ್ವಪರಿಗ್ರಹಃ ।\\
ಶಾರೀರಂ ಕೇವಲಂ ಕರ್ಮ ಕುರ್ವನ್ನಾಪ್ನೋತಿ ಕಿಲ್ಬಿಷಮ್ ॥ ೨೧ ॥}
\cquote{ಬಯಕೆ ಇಲ್ಲದೆ ಮನಸ್ಸನ್ನು ದೇಹವನ್ನು ಹಿಡಿತದಲ್ಲಿಟ್ಟುಕೊಂಡು ಎಲ್ಲವನ್ನು ತೊರೆದು ಬರಿ ಬದುಕಿರುವಷ್ಟರ ಮಟ್ಟಿಗೆ ಕರ್ಮವನ್ನು ಮಾಡುವವನಿಗೆ ಪಾಪವಿಲ್ಲ.}
\slcol{\Index{ಯದೃಚ್ಛಾಲಾಭಸಂತುಷ್ಟೋ} ದ್ವಂದ್ವಾತೀತೋ ವಿಮತ್ಸರಃ ।\\
ಸಮಃ ಸಿದ್ಧಾವಸಿದ್ಧೌ ಚ ಕೃತ್ವಾಪಿ ನ ನಿಬಧ್ಯತೇ ॥ ೨೨ ॥}
\cquote{ತಾನಾಗಿ ದೊರಕಿದಲ್ಲಿ ತೃಪ್ತನಾಗಿ ಸುಖ ದುಃಖ ಮೊದಲಾದ ದ್ವಂದ್ವಗಳನ್ನು ದಾಟಿದವನಾಗಿ ಮತ್ತೊಬ್ಬರ ಏಳಿಗೆಗೆ ಅಸೂಯೆ ಪಡದೆ ಕಾರ್ಯ ಫಲಿಸಿದರೂ ಫಲಿಸದಿದ್ದರೂ  ಒಂದೇ ರೀತಿಯಲ್ಲಿರುವವನು ಕರ್ಮ ಮಾಡಿದರೂ ಅದರ ಕಟ್ಟಿಗೊಳಪಡುವುದಿಲ್ಲ.}

\newpage
\begin{mananam}{\mananamfont ಮನನ ಶ್ಲೋಕ - ೧೮}
\small \mananamtext ಸರಿ ಮತ್ತು ತಪ್ಪು ಕ್ರಿಯೆಗಳು ಯಾವುವೆಂಬ ನಡುವೆ  ಗೊಂದಲಕ್ಕೀಡಾದಾಗ, ಯಾವದೇ ಕ್ರಿಯೆ ಕೈಗೊಳ್ಳಲು ಹೆದರಿ, ನಿಶ್ಕ್ರಿಯನಾಗುವತ್ತ ಒಲವು ತೋರುತ್ತೇನೆಯೇ? ನಾನು ನಿಷ್ಕ್ರಿಯತೆಯನ್ನು, ಸೋಮಾರಿಯಾಗಿರುವುದರಿಂದ ಆರಿಸಿಕೊಳ್ಳುತ್ತೇನೆಯೋ ಅಥವಾ ಪರಿಣಾಮಗಳ ಹೆದರಿಕೆಯಿಂದಲೋ? ನನ್ನ ಕರ್ತವ್ಯಗಳು, ಸ್ವಯಂ ಸುಧಾರಣೆ ಮತ್ತು ಆಧ್ಯಾತ್ಮಿಕ ಅಭ್ಯಾಸಗಳಿಗೆ, ನನ್ನ ಪ್ರತಿರೋಧಕ್ಕೆ ಕಾರಣವೇನು ಎಂದು ನಾನು ಆತ್ಮಾವಲೋಕನ ಮಾಡುತ್ತೇನೆಯೇ? ಅತೀ ಸುಲಭವಾಗಿ ಕೆಲಸ ಮಾಡುವುದೆಂದರೆ ಏನೆಂಬುದನ್ನು ನಾನು ಅನುಭವಿಸಿದ್ದೇನೆಯೇ? ಹಾಗಿದ್ದಲ್ಲಿ, ಅದು ಹೇಗಿತ್ತು ಮತ್ತು ಅಂತಹ ಸ್ಥಿತಿಗೆ ಕಾರಣವಾದ ಅಂಶಗಳು ಯಾವುವು?‘ಅಹಂ’ನಿಂದ ಮುಕ್ತಗೊಂಡ ಕ್ರಿಯೆಗಳು ನನ್ನನ್ನು ಸಬಲೀಕರಣಗೊಳಿಸುವ ಮತ್ತು ಉನ್ನತಿಗೇರಿಸುವ ಸ್ಥಿತಿಯನ್ನು ನಾನು ಅರಿಯಬಲ್ಲೆನೇ?
\end{mananam}
\WritingHand\enspace\textbf{ಆತ್ಮ ವಿಮರ್ಶೆ}\\
\begin{inspiration}{\mananamfont ಸ್ಫೂರ್ತಿ}
\small \mananamtext ಅತೀ ಸುಲಭವಾಗಿ ಕ್ರಿಯೆ ನಡೆಸುವ ರಹಸ್ಯ, ಒಬ್ಬನು ತನ್ನ ಕರ್ತವ್ಯಗಳಲ್ಲಿ ಸಂಪೂರ್ಣವಾಗಿ ತಲ್ಲೀನನಾದ ಸ್ಥಿತಿಯಲ್ಲಿ ಕಾರ್ಯ ನಡೆಸುವುದರಲ್ಲಿ ಅಡಗಿದೆ. ‘ಅಹಂ’ ಇಲ್ಲದ ಸ್ಥಳದಲ್ಲಿ, ದೈವೀಕ ಶಕ್ತಿಯು ಪೂರ್ಣ ರಭಸದಿಂದ ಹರಿಯುತ್ತದೆ.ಅತ್ಯುತ್ತವಾಗಿ ಜೀವಿಸುವ ಕಲೆ, ದೈವದ ಹರಿವನ್ನು ಪ್ರತಿರೋಧಿಸದೇ ಜೀವನವನ್ನು ಸಾಗಿಸುವುದರಲ್ಲಿ ಇದೆ.
\end{inspiration}
\newpage


\newpage
\begin{mananam}{\mananamfont ಮನನ ಶ್ಲೋಕ - ೧೯}
\small \mananamtext ನನ್ನ ದೈನಂದಿನ ಜೀವನದಲ್ಲಿ, ವಿವೇಕಕ್ಕೂ ಕ್ರಿಯೆಗಳಿಗೂ ಇರುವ ಸಂಬಂಧವೇನು? ಗತಜೀವನದ ಬಗ್ಗೆ ಚಿಂತನೆ ನಡೆಸಿದಾಗ, ನನ್ನ ಜೀವನದ ಯಾವುದೇ ಕ್ಷಣದಲ್ಲಾದರೂ, ನನ್ನ ಕ್ರಿಯೆಗಳ ಸ್ವಭಾವವು, ಆ ಕ್ಷಣದಲ್ಲಿ ನನ್ನಲ್ಲಿದ್ದ ವಿವೇಕದ ಪ್ರಮಾಣವನ್ನು ಆಧರಿಸಿತ್ತು ಎಂಬುದನ್ನು ನಾನು ಗಮನಿಸಬಲ್ಲೆನೇ? ನನ್ನ ಕ್ರಿಯೆಗಳನ್ನು ಮಾರ್ಗದರ್ಶಿಸಲು, ಒಂದು ಉನ್ನತ ಶ್ರೇಣಿಯ ವಿವೇಕದಿಂದ ನಾನು ಹೇಗೆ ಕಾರ್ಯ ನಿರ್ವಹಿಸಬಹುದು? ವೈಯಕ್ತಿಕ ಕಾಮನೆಗಳಿಂದ ಮುಕ್ತನಾಗಿರುವುದು, ನನ್ನ ಎಲ್ಲಾ ಕಾರ್ಯಗಳ ಆಧ್ಯಾತ್ಮಿಕ ಮತ್ತು ಮಾನಸಿಕ ಗುಣಮಟ್ಟವನ್ನು ಸುಧಾರಿಸುತ್ತದೆಯೇ?
\end{mananam}
\WritingHand\enspace\textbf{ಆತ್ಮ ವಿಮರ್ಶೆ}\\
\begin{inspiration}{\mananamfont ಸ್ಫೂರ್ತಿ}
\small \mananamtext ನಾವು ನಮ್ಮ ಕಾಮನೆಗಳ ಗೀಳಿನಲ್ಲಿ ಮುಳುಗಿದಾಗ, ಒತ್ತಡಕ್ಕೆ ಮಣಿದು ನಮ್ಮದೇ ಮಾರ್ಗವನ್ನು ನಾವೇ ನಿರ್ಬಂಧಿಸುತ್ತೇವೆ. ನಮ್ಮ ಜೀವನದಲ್ಲಿ, ಹಾನಿರಹಿತವೂ ಅರ್ಥಪೂರ್ಣವೂ ಆದದ್ದನ್ನು ನಿಜವಾಗಿ ನಾವು ಬಯಸಿದಾಗ, ಅದನ್ನು ಪಡೆಯುವ ಅತ್ಯುತ್ತಮ ಮಾರ್ಗವೆಂದರೆ – ನಾವು ಯಾವುದಕ್ಕೆ ಯೋಗ್ಯರೋ ಅದನ್ನು ಕರುಣಿಸುವರೆಂದು  ಭಗವಂತನನ್ನೂ ಬ್ರಹ್ಮಾಂಡದ ಶಕ್ತಿಗಳನ್ನೂ ನಂಬಿ, ಹಾರೈಕೆ ಮತ್ತು ಪ್ರಾರ್ಥನೆ ಸಲ್ಲಿಸಿ, ಮನಸ್ಸನ್ನು ವಿಶ್ರಾಂತಗೊಳಿಸುವುದು.
\end{inspiration}
\newpage

\begin{mananam}{\mananamfont ಮನನ ಶ್ಲೋಕ - ೨೨}
\small \mananamtext ಯಶಸ್ಸು ಮತ್ತು ವೈಫಲ್ಯಕ್ಕೆ ನಾನು ಹೇಗೆ ಪ್ರತಿಕ್ರಿಸುತ್ತೇನೆ? ಯಶಸ್ಸು ನನ್ನನ್ನು ಉಲ್ಲಸಿತನಾಗಿ ಮಾಡುತ್ತದೆಯೇ? ವಿಫಲತೆ ನನ್ನನ್ನು ನಿರಾಶೆಗೊಳಿಸುತ್ತದೆಯೇ? ಇತರರು ಹೊಂದಿರುವುದನ್ನು ಮನಸ್ಸಿನಲ್ಲಿ ಆಕಾಂಕ್ಷಿಸುವುದಕ್ಕಿಂತ, ಜೀವನದಲ್ಲಿ ಸಹಜವಾಗಿ ಬರುವುದನ್ನು ಸ್ವೀಕರಿಸುವುದರಲ್ಲಿ ನಾನು ತೃಪ್ತನಾಗಿದ್ದೀನೆಯೇ? ಇತರರ ಯಶಸ್ಸಿನ ಆಧಾರದ ಮೇಲೆ ನನ್ನನ್ನು ನಾನು ಮೌಲ್ಯಮಾಪನ  ಮಾಡಿಕೊಳ್ಳುತ್ತೇನೆಯೇ? ತನ್ನ ಕರ್ತವ್ಯಗಳಲ್ಲಿ ಸಕ್ರಿಯನಾಗಿದ್ದೂ ಜೀವನದಲ್ಲಿ ತೃಪ್ತನಾಗಿರಲು ಸಾಧ್ಯವೇ?
\end{mananam}
\WritingHand\enspace\textbf{ಆತ್ಮ ವಿಮರ್ಶೆ}\\
\begin{inspiration}{\mananamfont ಸ್ಫೂರ್ತಿ}
\small \mananamtext ಆಧ್ಯಾತ್ಮಿಕ ಅರಿವಿನ ಒಂದು ಲಕ್ಷಣವೆಂದರೆ — ಭೌತಿಕ ಯಶಸ್ಸು ಮತ್ತು ಸಾಧನೆಗಳ ಆಧಾರದ ಮೇಲೆ ತನ್ನನ್ನು ತಾನು ಮೌಲ್ಯಮಾಪನ ಮಾಡದಿರುವುದು. ತಾನು ಜೀವನದಲ್ಲಿ ಪಡೆದುದರಲ್ಲಿಯೇ ತೃಪ್ತಿಯುಳ್ಳವನಾಗಿ, ಇತರರೊಂದಿಗೆ ಹೋಲಿಸದೇ ತನ್ನ ಎಲ್ಲಾ ಕರ್ತವ್ಯಗಳನ್ನು ಪೂರ್ಣಗೊಳಿಸುವುದು ನಿಜವಾದ ಸ್ವಾತಂತ್ರ್ಯ. ಇಂತಹ ಆಂತರಿಕ ಸ್ವಾತಂತ್ರ್ಯದ ಸ್ಥಿತಿಯನ್ನು ಉಂಟುಮಾಡುವುದೇ ಆಧ್ಯಾತ್ಮಿಕ ಅರಿವು.
\end{inspiration}
\newpage

\slcol{\Index{ಗತಸಂಗಸ್ಯ ಮುಕ್ತಸ್ಯ} ಜ್ಞಾನಾವಸ್ಥಿತಚೇತಸಃ ।\\
ಯಜ್ಞಾಯಾಚರತಃ ಕರ್ಮ ಸಮಗ್ರಂ ಪ್ರವಿಲೀಯತೇ ॥ ೨೩ ॥}
\cquote{ಅಭಿಮಾನವನ್ನು ತೊರೆದು ಕರ್ಮಗಳ ಕಟ್ಟನ್ನು ಕಳಚಿಕೊಂಡು ಜ್ಞಾನದಲ್ಲಿ ಮನಸ್ಸನ್ನು ನೆಲೆಗೊಳಿಸಿ, ಈಶ್ವರನಿಗೆ ಪ್ರೀತಿಯಾಗಲೆಂದು ಮಾಡುವ ಕರ್ಮವೆಲ್ಲವೂ ಲಯವಾಗುತ್ತವೆ. }
\slcol{\Index{ಬ್ರಹ್ಮಾರ್ಪಣಂ ಬ್ರಹ್ಮ} ಹವಿರ್ಬ್ರಹ್ಮಾಗ್ನೌ ಬ್ರಹ್ಮಣಾ ಹುತಮ್ ।\\
ಬ್ರಹ್ಮೈವ ತೇನ ಗಂತವ್ಯಂ ಬ್ರಹ್ಮಕರ್ಮಸಮಾಧಿನಾ ॥ ೨೪ ॥}
\cquote{ಹೋಮವು ಬ್ರಹ್ಮಮಯ, ಹವಿಸ್ಸೂ ಬ್ರಹ್ಮಮಯ, ಅಗ್ನಿಯು ಬ್ರಹ್ಮಮಯ, ಹೋಮಿಸುವವನೂ ಬ್ರಹ್ಮಮಯ. ಹೀಗೆ ತಾನು ಮಾಡುವ ಕ್ರಿಯೆ ಎಲ್ಲಾ ಸರ್ವಗತವಾದ ಬ್ರಹ್ಮತತ್ವವನ್ನೇ ಕಾಣುವವನು ಬ್ರಹ್ಮನನ್ನೇ ಹೋಗಿ ಸೇರುತ್ತಾನೆ.}
\slcol{\Index{ದೈವಮೇವಾಪರೇ ಯಜ್ಞಂ} ಯೋಗಿನಃ ಪರ್ಯುಪಾಸತೇ ।\\
ಬ್ರಹ್ಮಾಗ್ನಾವಪರೇ ಯಜ್ಞಂ ಯಜ್ಞೇನೈವೋಪಜುಹ್ವತಿ ॥ ೨೫ ॥}
\cquote{ಕೆಲವು ಯೋಗಿಗಳು ಭಗವಂತನ ಉಪಾಸನೆಯನ್ನೇ ಯಜ್ಞವೆಂದು ಆಚರಿಸುತ್ತಾರೆ. ಇನ್ನು ಕೆಲವರು ಬ್ರಹ್ಮವೆಂಬ ಅಗ್ನಿಯಲ್ಲಿ ಭಗವತ್ ಪೂಜಾರೂಪವಾಗಿ ಯಜ್ಞ ಯಾಗಾದಿಗಳಿಂದ ಹೋಮಿಸುವರು.}
\slcol{\Index{ಶ್ರೋತ್ರಾದೀನೀಂದ್ರಿಯಾಣ್ಯನ್ಯೇ} ಸಂಯಮಾಗ್ನಿಷು ಜುಹ್ವತಿ ।\\
ಶಬ್ದಾದೀನ್ವಿಷಯಾನನ್ಯ ಇಂದ್ರಿಯಾಗ್ನಿಷು ಜುಹ್ವತಿ ॥ ೨೬ ॥}
\cquote{ಅನ್ಯ ಯೋಗಿ ಜನರು ಶ್ರವಣ ಇಂದ್ರಿಯಾದಿ ಸಮಸ್ತ ಇಂದ್ರಿಯಗಳನ್ನು ಸಂಯಮ ರೂಪೀ ಅಗ್ನಿಗಳಲ್ಲಿ ಹೋಮ ಮಾಡುತ್ತಾರೆ ಮತ್ತು ಬೇರೆ ಯೋಗಿ ಜನರು ಶಬ್ಧಾದಿ ಸಮಸ್ತ ವಿಷಯಗಳನ್ನು ಇಂದ್ರಿಯರೂಪೀ ಅಗ್ನಿಗಳಲ್ಲಿ ಹವನ ಮಾಡುತ್ತಾರೆ.}
\slcol{\Index{ಸರ್ವಾಣೀಂದ್ರಿಯಕರ್ಮಾಣಿ} ಪ್ರಾಣಕರ್ಮಾಣಿ ಚಾಪರೇ ।\\
ಆತ್ಮಸಂಯಮಯೋಗಾಗ್ನೌ ಜುಹ್ವತಿ ಜ್ಞಾನದೀಪಿತೇ ॥ ೨೭ ॥}
\cquote{ಇನ್ನೂ ಕೆಲವರು ಎಲ್ಲ ಇಂದ್ರಿಯಗಳ ವಿಷಯಗಳನ್ನೂ ಪ್ರಾಣಗಳ ವಿಷಯಗಳನ್ನೂ ಆತ್ಮ ಸಂಯಮವೆಂಬ ಬೆಂಕಿಯಲ್ಲಿ ಹೋಮ ಮಾಡುತ್ತಾರೆ.}

\newpage
\begin{mananam}{\mananamfont ಮನನ ಶ್ಲೋಕ - ೨೪}
\small \mananamtext ಈ ನನ್ನ ಜೀವನವು ಭಗವಂತನಿಗೆ ಸಮರ್ಪಿತವಾಗುವುದು ಹೇಗೆ? ನನ್ನ ಎಲ್ಲ ಚಿಂತನೆ, ಮಾತು ಮತ್ತು ಕ್ರಿಯೆಗಳು ಸಮರ್ಪಣಾ ಭಾವದಿಂದ ತುಂಬಿರಬಹುದೇ? ಸಣ್ಣತನದ ಈ ‘ಅಹಂ’ನನ್ನು ತ್ಯಜಿಸುವ ಮನೋಭಾವದಿಂದ ಎಲ್ಲ ಕ್ರಿಯೆಗಳನ್ನೂ ಮಾಡುವಂತೆ, ಪ್ರತಿದಿನ ನನಗೆ ನಾನೇ ನೆನಪಿಸಿಕೊಳ್ಳುವುದರ ಮಹತ್ವವನ್ನು ನಾನು ಅರಿಯಬಹುದೇ? ನನ್ನ ನಿತ್ಯ ಜೀವನದಲ್ಲಿ ಅಂತಹ ಆಂತರಿಕ ಅಭ್ಯಾಸವನ್ನು ಮಾಡುವ ವಿಧಾನಗಳು ಯಾವುವು?
\end{mananam}
\WritingHand\enspace\textbf{ಆತ್ಮ ವಿಮರ್ಶೆ}\\
\begin{inspiration}{\mananamfont ಸ್ಫೂರ್ತಿ}
\small \mananamtext ತನ್ನ ಎಲ್ಲಾ ಕ್ರಿಯೆಗಳನ್ನು ಪರಮಾತ್ಮನಿಗೆ ಅರ್ಪಿಸುವುದೂ, ಎಲ್ಲವನ್ನೂ ಆ ಏಕ ಚೈತನ್ಯದಿಂದ ವಿಭಿನ್ನವಲ್ಲವೆಂದು ನೋಡುವುದೂ, ಮಾನಸಿಕವಾಗಿ ಅತ್ಯುನ್ನತ ಸ್ಥಿತಿ. ಇದು ಜೀವನದ ಎಲ್ಲಾ ಒತ್ತಡಗಳಿಂದ ನಮ್ಮನ್ನು ಮುಕ್ತಗೊಳಿಸಿ, ತಕ್ಷಣವೇ ಮನಶಾಂತಿಯನ್ನು ನೀಡುತ್ತದೆ.
\end{inspiration}
\newpage


\slcol{\Index{ದ್ರವ್ಯಯಜ್ಞಾಸ್ತಪೋಯಜ್ಞಾ} ಯೋಗಯಜ್ಞಾಸ್ತಥಾಪರೇ ।\\
ಸ್ವಾಧ್ಯಾಯಜ್ಞಾನಯಜ್ಞಾಶ್ಚ ಯತಯಃ ಸಂಶಿತವ್ರತಾಃ ॥ ೨೮ ॥}
\cquote{ಕಠಿಣ ವ್ರತನಿಷ್ಠರಾದ ಕೆಲವು ಸಾಧಕರು ಭಗವತ್ ಪೂಜಾ ರೂಪವಾಗಿ ಹಣವನ್ನು ವೆಚ್ಚ ಮಾಡುವರು, ಉಪವಾಸ ಮೊದಲಾದದ್ದನ್ನು ನಡೆಸುವರು, ಹಾಗೆಯೇ ಇನ್ನು ಕೆಲವರು ಯೋಗಾಭ್ಯಾಸ ನಡೆಸುವರು, ವೇದಾಭ್ಯಾಸಗಳನ್ನು ಮಾಡುವರು ಮತ್ತು ಶಾಸ್ತ್ರಾರ್ಥವನ್ನು ತಿಳಿದುಕೊಳ್ಳುವರು.}
\slcol{\Index{ಅಪಾನೇ ಜುಹ್ವತಿ ಪ್ರಾಣಂ} ಪ್ರಾಣೇಽಪಾನಂ ತಥಾಪರೇ ।\\
ಪ್ರಾಣಾಪಾನಗತೀ ರುದ್ಧ್ವಾಪ್ರಾಣಾಯಾಮಪರಾಯಣಾಃ ॥ ೨೯ ॥}
\cquote{ಪ್ರಾಣಯಾಮದಲ್ಲಿ ಪರಿಣತರಾದ ಇನ್ನೂ ಕೆಲವರು ಪ್ರಾಣದಲ್ಲಿ ಅಪಾನವನ್ನು ಅಪಾನದಲ್ಲಿ ಪ್ರಾಣವನ್ನು ಹೋಮಿಸಿ ಪ್ರಾಣಾಪಾನಗಳ ಚಲನವನ್ನು ನಿಯಂತ್ರಿಸಿ ಕುಂಭಕದಲ್ಲಿ ನೆಲೆಗೊಳಿಸುತ್ತಾರೆ.}
\slcol{\Index{ಅಪರೇ ನಿಯತಾಹಾರಾಃ} ಪ್ರಾಣಾನ್ಪ್ರಾಣೇಷು ಜುಹ್ವತಿ ।\\
ಸರ್ವೇಽಪ್ಯೇತೇ ಯಜ್ಞವಿದೋ ಯಜ್ಞಕ್ಷಪಿತಕಲ್ಮಷಾಃ ॥ ೩೦ ॥}
\cquote{ಮತ್ತೆ ಕೆಲವರು ಮಿತಾಹಾರದ ಮೂಲಕ ಪ್ರಾಣವನ್ನು ಶೋಷಿಸಿ ಪ್ರಾಣಗಳಲ್ಲಿ ಪ್ರಾಣಗಳನ್ನು ಹೋಮಿಸುತ್ತಾರೆ. ಇವರೆಲ್ಲರೂ ಯಜ್ಞಗಳನ್ನು ತಿಳಿದವರು ಮತ್ತು ಯಜ್ಞದಿಂದ ತಮ್ಮ ಪಾಪವನ್ನು ಕಳೆದುಕೊಂಡವರು.}
\slcol{\Index{ಯಜ್ಞಶಿಷ್ಟಾಮೃತಭುಜೋ} ಯಾಂತಿ ಬ್ರಹ್ಮ ಸನಾತನಮ್ ।\\
ನಾಯಂ ಲೋಕೋಽಸ್ತ್ಯಯಜ್ಞಸ್ಯ ಕುತೋಽನ್ಯಃ ಕುರುಸತ್ತಮ ॥ ೩೧ ॥}
\cquote{ಯಜ್ಞದಲ್ಲಿ  ಉಳಿದ ಅಮೃತವನ್ನು ಭೋಜನ ಮಾಡಿ ಇವರು ಅನಂತ ಬ್ರಹ್ಮನನ್ನು ಸೇರುವರು. ಎಲೈ  ಕುರುಶ್ರೇಷ್ಠನೆ, ಯಜ್ಞಮಾಡದಿರುವವರಿಗೆ ಈ ಲೋಕವೇ ಇಲ್ಲ, ಇನ್ನು ಪರಲೋಕವೆಲ್ಲಿಯದು?}

\newpage
\slcol{\Index{ಏವಂ ಬಹುವಿಧಾ ಯಜ್ಞಾ} ವಿತತಾ ಬ್ರಹ್ಮಣೋ ಮುಖೇ ।\\
ಕರ್ಮಜಾನ್ವಿದ್ಧಿ ತಾನ್ಸರ್ವಾನೇವಂ ಜ್ಞಾತ್ವಾ ವಿಮೋಕ್ಷ್ಯಸೇ ॥ ೩೨ ॥}
\cquote{ಹೀಗೆ ಬಹುವಿಧವಾದ ಯಜ್ಞಗಳು ವೇದದಲ್ಲಿ ಹೇಳಲ್ಪಟ್ಟಿವೆ. ಅವುಗಳೆಲ್ಲಾ ಕರ್ಮದಿಂದ ಜನಿಸಿದವುಗಳು ಎಂಬುದನ್ನು ತಿಳಿ. ಹೀಗೆ ನೀನು ತಿಳಿಯುವುದರಿಂದ ಬಿಡುಗಡೆಯನ್ನು ಪಡೆಯುತ್ತೀಯೆ.}
\slcol{\Index{ಶ್ರೇಯಾಂದ್ರವ್ಯಮಯಾದ್ಯ}ಜ್ಞಾತ್‍ ಜ್ಞಾನಯಜ್ಞಃ ಪರಂತಪ ।\\
ಸರ್ವಂ ಕರ್ಮಾಖಿಲಂ ಪಾರ್ಥ ಜ್ಞಾನೇ ಪರಿಸಮಾಪ್ಯತೇ ॥ ೩೩ ॥}
\cquote{ಅರ್ಜುನ, ವಸ್ತುಗಳನ್ನು ಹೋಮಿಸಿ ಮಾಡುವ ಯಜ್ಞಕ್ಕಿಂತಲೂ  ಜ್ಞಾನರೂಪವಾದ ಯಜ್ಞವು ಹೆಚ್ಚಿನದು. ಅರ್ಜುನ, ಎಲ್ಲಾ ಕರ್ಮಗಳು ಜ್ಞಾನದ ಮೂಲಕವೇ ಪೂರ್ಣ ಫಲಪ್ರದವಾಗುತ್ತವೆ.}
\slcol{\Index{ತದ್ವಿದ್ಧಿ ಪ್ರಣಿಪಾತೇನ} ಪರಿಪ್ರಶ್ನೇನ ಸೇವಯಾ ।\\
ಉಪದೇಕ್ಷ್ಯಂತಿ ತೇ ಜ್ಞಾನಂ ಜ್ಞಾನಿನಸ್ತತ್ತ್ವದರ್ಶಿನಃ ॥ ೩೪ ॥}
\cquote{ಜ್ಞಾನಿಗಳೂ ತತ್ವದರ್ಶಿಗಳೂ ಆದ ಮಹಾತ್ಮರನ್ನು ವಿನಮ್ರಭಾವದಿಂದ ಸೇವಿಸುವುದರಿಂದ, ಜ್ಞಾನ ಭಿಕ್ಷೆಯನ್ನು ಬೇಡು. ಅವರು ನಿನಗೆ ಜ್ಞಾನೋಪದೇಶವನ್ನು ಮಾಡುವರು.}
\slcol{\Index{ಯಜ್ಞಾತ್ವಾ ನ ಪುನರ್ಮೋಹ}ಮೇವಂ ಯಾಸ್ಯಸಿ ಪಾಂಡವ ।\\
ಯೇನ ಭೂತಾನ್ಯಶೇಷೇಣ ದ್ರಕ್ಷ್ಯಸ್ಯಾತ್ಮನ್ಯಥೋ ಮಯಿ ॥ ೩೫ ॥}
\cquote{ಎಲೈ ಪಾಂಡವನೇ, ಅದನ್ನು ಅರಿತ ಮೇಲೆ ಹೀಗೆ ನೀನು, ಮೋಹಗೊಳ್ಳುವುದಿಲ್ಲ ಎಲ್ಲಾ ಜೀವಿಗಳನ್ನು ನೀನು, ನಿನ್ನ ಆತ್ಮನಾದ ನನ್ನಲ್ಲೇ ಕಾಣುವೆ.}
\slcol{\Index{ಅಪಿ ಚೇದಸಿ ಪಾಪೇಭ್ಯಃ} ಸರ್ವೇಭ್ಯಃ ಪಾಪಕೃತ್ತಮಃ ।\\
ಸರ್ವಂ ಜ್ಞಾನಪ್ಲವೇನೈವ ವೃಜಿನಂ ಸಂತರಿಷ್ಯಸಿ ॥ ೩೬ ॥}
\cquote{ನೀನು ಪಾಪಿಗಳಲ್ಲೆಲ್ಲ ಹೆಚ್ಚಿನ ಪಾಪಿಯಾಗಿದ್ದರೂ ಆ ಪಾಪಗಳನ್ನೆಲ್ಲ ಜ್ಞಾನರೂಪವಾದ ಹಡಗಿನಿಂದಲೇ ದಾಟುತ್ತಿ.}

\newpage
\begin{mananam}{\mananamfont ಮನನ ಶ್ಲೋಕ - ೩೨}
\small \mananamtext ವೈಯಕ್ತಿಕ ಸವಲತ್ತು ಪಡೆಯುವ ಉದ್ದೇಶದಿಂದ ಪ್ರಾಪಂಚಿಕ ಅಧಿಕಾರಿಗಳನ್ನು ತೃಪ್ತಿಪಡಿಸಲು ನಾನು ಯಾವ ಯಾವ ರೀತಿಯಲ್ಲಿ ಪ್ರಯತ್ನಿಸುತ್ತೇನೆ? ಅದರಂತೆಯೇ, ಜೀವನದಲ್ಲಿ ಭೌತಿಕ ಲಾಭಗಳಿಗಾಗಿ ದೇವರುಗಳನ್ನು ಮತ್ತು ದೇವತೆಗಳನ್ನು ಪೂಜಿಸುತ್ತೇನೆಯೇ? “ನಶ್ವರ ಸ್ವಭಾವದ ನೂರಾರು ಭೌತಿಕ ಗುರಿಗಳನ್ನು ಹಿಂಬಾಲಿಸುವುದಕ್ಕಿಂತ, ಪರಮಾತ್ಮನನ್ನು ಅರಸುವುದೂ ದೈವೀ ಸ್ಥಿತಿಯಲ್ಲಿ ಸ್ಥಿರವಾಗುವುದೂ ಶ್ರೇಷ್ಠವಾದುದು” ಎಂದು ಶಾಸ್ತ್ರಗಳು ಮತ್ತು ಋಷಿಗಳು ಹೇಳಿರುವರು. ನನ್ನ ಜೀವನದಲ್ಲಿ ಇಂತಹ ಮೌಲ್ಯಗಳು ಮತ್ತು ದೃಷ್ಟಿಕೋನವನ್ನು ಅಳವಡಿಸಿಕೊಳ್ಳಲು ನಾನು ಅಗತ್ಯವಾಗಿ ಮಾಡಬೇಕಾದದ್ದೇನು?ಅಂತರಂಗವನ್ನು ಉನ್ನತಿಗೇರಿಸುವ ಹಾಗೂ ಇತರರಿಗೂ ಉಪಯುಕ್ತವಾಗುವಂತಹ, ದೀರ್ಘಾವಧಿಯ ಗುರಿಗಳನ್ನು ಅರಸುವುದಕ್ಕೆ ನಾನು ಹೇಗೆ ಪ್ರಾಮುಖ್ಯತೆ ನೀಡಬಲ್ಲೆ?
\end{mananam}
\WritingHand\enspace\textbf{ಆತ್ಮ ವಿಮರ್ಶೆ}\\
\begin{inspiration}{\mananamfont ಸ್ಫೂರ್ತಿ}
\small \mananamtext ಯಜ್ಞಗಳು, ಪೂಜೆಗಳು ಹಾಗೂ ಇತರ ವಿಧಿ-ವಿಧಾನಗಳ ನಿಜವಾದ ಮೌಲ್ಯವೆಂದರೆ — ಅವು ನಮ್ಮನ್ನು ಅತಿಯಾದ ಭೌತಿಕ ಆಸಕ್ತಿಗಳಿಂದ ಮೇಲಕ್ಕೆ ಎತ್ತುವುದು. ಉನ್ನತ ಆಧ್ಯಾತ್ಮಿಕ ಶಕ್ತಿಗಳಿಗೆ ಶರಣಾಗತಿಯ ಮೂಲಕ ಗಳಿಸಿದ ಮನಸ್ಸಿನ ಶುದ್ಧಿಯಿಂದ- ನಾವು ಆಧ್ಯಾತ್ಮಿಕ ವಿವೇಕವನ್ನು ಅರಸುವ ಮೌಲ್ಯವನ್ನು ಅರಿತು, ನಮ್ಮ ಆತ್ಮಸ್ವರೂಪವನ್ನು ಅರಿಯುವತ್ತ ಕಾರ್ಯದಲ್ಲಿ ತೊಡಗುತ್ತೇವೆ. 
\end{inspiration}
\newpage

\begin{mananam}{\mananamfont ಮನನ ಶ್ಲೋಕ - ೩೪}
\small \mananamtext ನಾನು ಜ್ಞಾನ ಮತ್ತು ಆಧ್ಯಾತ್ಮಿಕ ಸತ್ಯವನ್ನು ಅರಸಲು ಗುರುಗಳ ಬಳಿ ಹೋಗುತ್ತೇನೆಯೇ? ನಮ್ರತೆಯಿಂದ ಅವರ ಬಳಿ ಹೋಗುವುದನ್ನು ನಾನು ಪ್ರತಿರೋಧಿಸುತ್ತೇನೆಯೇ? ನಾನು ನನ್ನ ಪ್ರಶ್ನೆಗಳನ್ನು ಪೂಜ್ಯಭಾವದಿಂದ ಕೇಳಲು ಸಿದ್ಧನೋ? ನಾನು ಉದಾರವಾಗಿ ಅವರಿಗೆ ಉಪಯೋಗವಾಗುವ ಯಾವುದನ್ನಾದರೂ ಅರ್ಪಿಸುವೆನೇ? ನಾನು ಅವರಿಗಾಗಿ ಅಥವಾ ಮಾನವಕುಲಕ್ಕಾಗಿ ಸೇವೆ ಸಲ್ಲಿಸಲು ಸಮಯ ಮೀಸಲಿಡಬಲ್ಲೆನೇ ಮತ್ತು ಅದರಿಂದ, ನನ್ನ ಆಧ್ಯಾತ್ಮಿಕ ಜೀವನದ ಪ್ರಾಮುಖ್ಯತೆಯನ್ನು ದೃಢಪಡಿಸಬಹುದೇ?
\end{mananam}
\WritingHand\enspace\textbf{ಆತ್ಮ ವಿಮರ್ಶೆ}\\
\begin{inspiration}{\mananamfont ಸ್ಫೂರ್ತಿ}
\small \mananamtext ಆಧ್ಯಾತ್ಮಿಕ ಸತ್ಯವನ್ನು ಅರಸುವ ಸಾಧಕನು,  ಸರಿಯಾದ ರೀತಿಯಲ್ಲಿ ಗುರುವನ್ನು ಸಮೀಪಿಸುವುದರಲ್ಲೂ, ಅವರಿಗೆ ಸೇವೆ ಸಲ್ಲಿಸುವ ರೀತಿಯಲ್ಲೂ, ಮೊದಲು ಕೆಲವು ಕೌಶಲ್ಯಗಳನ್ನು ಬೆಳೆಸಿಕೊಳ್ಳಬೇಕು. ಇವು ಕೇವಲ ಸಾಂಸ್ಕೃತಿಕ ಮತ್ತು ಸಾಮಾಜಿಕ ರೂಢಿಗಳಲ್ಲ, ಆದರೆ ಇಂತಹ ಮಾನಸಿಕ ವರ್ತನೆ ಜ್ಞಾನವನ್ನು ಪ್ರಸರಿಸಲು ಸಾಧ್ಯಮಾಡುತ್ತದೆ.
\end{inspiration}
\newpage


\slcol{\Index{ಯಥೈಧಾಂಸಿ ಸಮಿದ್ಧೋಽಗ್ನಿ}ರ್ಭಸ್ಮಸಾತ್ಕುರುತೇಽರ್ಜುನ ।\\
ಜ್ಞಾನಾಗ್ನಿಃ ಸರ್ವಕರ್ಮಾಣಿ ಭಸ್ಮಸಾತ್ಕುರುತೇ ತಥಾ ॥ ೩೭ ॥}
\cquote{ಅರ್ಜುನ, ಉರಿಗೊಳಿಸಿದ ಬೆಂಕಿಯು ಕಟ್ಟಿಗೆಗಳನ್ನು ಬೂದಿಮಾಡಿಬಿಡುವಂತೆ ಜ್ಞಾನರೂಪವಾದ ಅಗ್ನಿಯು ಕರ್ಮಗಳನ್ನೆಲ್ಲ ಬೂದಿಮಾಡಿಬಿಡುತ್ತದೆ.}
\slcol{\Index{ನ ಹಿ ಜ್ಞಾನೇನ ಸದೃಶಂ} ಪವಿತ್ರಮಿಹ ವಿದ್ಯತೇ ।\\
ತತ್ಸ್ವಯಂ ಯೋಗಸಂಸಿದ್ಧಃ ಕಾಲೇನಾತ್ಮನಿ ವಿಂದತಿ ॥ ೩೮ ॥}
\cquote{ಆತ್ಮ ಜ್ಞಾನಕ್ಕೆ ಸಮಾನವಾದ ಪವಿತ್ರಕರ ವಸ್ತುವು ಈ ಜಗತ್ತಿನಲ್ಲಿ ಮತ್ತೊಂದಿಲ್ಲ. ಯೋಗಸಿದ್ಧಿಯನ್ನು ಪಡೆದವನು ಕಾಲಕ್ರಮದಲ್ಲಿ ಸ್ವಯಂ ಇದನ್ನು ಪಡೆದುಕೊಳ್ಳುತ್ತಾನೆ.}
\slcol{\Index{ಶ್ರದ್ಧಾವಾಂಲ್ಲಭತೇ ಜ್ಞಾನಂ} ತತ್ಪರಃ ಸಂಯತೇಂದ್ರಿಯಃ ।\\
ಜ್ಞಾನಂ ಲಬ್ಧ್ವಾಪರಾಂ ಶಾಂತಿಮಚಿರೇಣಾಧಿಗಚ್ಛತಿ ॥ ೩೯ ॥}
\cquote{ಶ್ರದ್ಧಾವಂತನೂ, ಇಂದ್ರಿಯಗಳನ್ನು ನಿಗ್ರಹಿಸಿದವನೂ, ಈ ಜ್ಞಾನವನ್ನು ಪಡೆಯುತ್ತಾನೆ. ಜ್ಞಾನ ದೊರಕಿದ ಮೇಲೆ ಹೆಚ್ಚು ತಡವಿಲ್ಲದೆ ಪರಮಶಾಂತಿಯನ್ನು ಗಳಿಸುತ್ತಾನೆ.}
\slcol{\Index{ಅಜ್ಞಶ್ಚಾಶ್ರದ್ದಧಾನಶ್ಚ} ಸಂಶಯಾತ್ಮಾ ವಿನಶ್ಯತಿ ।\\
ನಾಯಂ ಲೋಕೋಽಸ್ತಿ ನ ಪರೋ ನ ಸುಖಂ ಸಂಶಯಾತ್ಮನಃ ॥ ೪೦ ॥}
\cquote{ಈ ಮಾತಿನಲ್ಲಿ ಶ್ರದ್ಧೆ ಇಲ್ಲದ ಮೂಢನು ಸಂಶಯಪಟ್ಟು, ತನ್ನ ಸರ್ವನಾಶವನ್ನು ಮಾಡಿಕೊಳ್ಳುತ್ತಾನೆ. ಸಂಶಯಾತ್ಮನಿಗೆ ಈ ಲೋಕವು ಇಲ್ಲ, ಪರಲೋಕವು ಇಲ್ಲ, ಸುಖವು ಇಲ್ಲ.}
\slcol{\Index{ಯೋಗಸಂನ್ಯಸ್ತಕರ್ಮಾಣಂ} ಜ್ಞಾನಸಂಛಿನ್ನಸಂಶಯಮ್ ।\\
ಆತ್ಮವಂತಂ ನ ಕರ್ಮಾಣಿ ನಿಬಧ್ನಂತಿ ಧನಂಜಯ ॥ ೪೧ ॥}
\cquote{ಧನಂಜಯ, ಭಗವದರ್ಪಣ ಬುದ್ಧಿಯಿಂದ ಕರ್ಮಗಳನ್ನು ಮಾಡುತ್ತಾ, ತತ್ವಜ್ಞಾನದ ಮೂಲಕ ಸಂಶಯವನ್ನು ಕಳೆದುಕೊಂಡ ವಿವೇಕಿಯನ್ನು ಕರ್ಮಗಳು ಕಟ್ಟಿ ಹಾಕುವುದಿಲ್ಲ.}

\newpage
\begin{mananam}{\mananamfont {ಮನನ ಶ್ಲೋಕ - ೩೬, ೩೭}}
\small \mananamtext ನಾನು ಹಿಂದೆ ಮಾಡಿದ ಕೆಲವು ಕ್ರಿಯೆಗಳ ಬಗ್ಗೆ ಪಶ್ಚಾತ್ತಾಪ ಮತ್ತು ಅಪರಾಧೀ ಭಾವನೆಗಳು ನನ್ನಲ್ಲಿ ಅಡಗಿವೆಯೇ? ನನಗೆ ಆ ಭಾವನೆಗಳನ್ನು ಅದುಮಿಡುವ ಪ್ರವೃತ್ತಿ ಇದೆಯೇ? ಅವುಗಳನ್ನು ಸಮರ್ಪಕವಾಗಿ ಎದುರಿಸುವ ಆರೋಗ್ಯಕರ ಮಾರ್ಗವಿರಬಹುದೇ? ಜೀವನದಲ್ಲಿ ಇಂತಹ ಭಾವನೆಗಳನ್ನು ಪರಿಹರಿಸಲು ನಾನು ಆಧ್ಯಾತ್ಮಿಕ ವಿವೇಕವನ್ನು ಹೇಗೆ ಬಳಸಬಹುದು? ಹಳೆಯ ನಕಾರಾತ್ಮಕ ಆಲೋಚನೆಗಳು ಹಾಗೂ ಕ್ರಿಯೆಗಳ ಮಾದರಿಗಳಿಂದ ಮುಕ್ತನಾಗಿ, ನವೀನ ಧನಾತ್ಮಕ ಕ್ರಿಯೆಗಳನ್ನು ಅಳವಡಿಸಿಕೊಳ್ಳಲು  ಆಧ್ಯಾತ್ಮಿಕ ವಿವೇಕವನ್ನು ಅನ್ವಯಿಸುವ ಮಾರ್ಗವನ್ನು ಕಂಡುಕೊಂಡಿದ್ದೇನೆಯೆ?  
\end{mananam}
\WritingHand\enspace\textbf{ಆತ್ಮ ವಿಮರ್ಶೆ}\\
\begin{inspiration}{\mananamfont ಸ್ಫೂರ್ತಿ}
\small \mananamtext ನಾವು ನಮ್ಮ ನಿಜವಾದ ಸ್ವರೂಪವನ್ನು ಅರಿತಾಗ ಮತ್ತು ಕೀಳಾದ ಪಶುಗಳ ತರಹ ಪ್ರಚೋದನೆಗಳಿಗೆ ಮಣಿಯುವುದನ್ನು ನಿಲ್ಲಿಸಿದಾಗ, ನಮ್ಮ ಜೀವನದ ಒಟ್ಟಾರೆ ಮಾನಸಿಕ ಮತ್ತು ಆಧ್ಯಾತ್ಮಿಕ ಗುಣಮಟ್ಟವು ಉನ್ನತಿಗೇರುತ್ತದೆ. ಆಗ, ನಮ್ಮ ಭೂತಕಾಲವು ಮತ್ತೆಂದೂ ನಮ್ಮನ್ನು ಕಾಡುವುದಿಲ್ಲ ಹಾಗೂ ನಮ್ಮನ್ನು ಕೆಳಮಟ್ಟಕ್ಕೆ ಸೆಳೆಯಲಾರದು.
\end{inspiration}
\newpage

\begin{mananam}{\mananamfont ಮನನ ಶ್ಲೋಕ - ೩೮}
\small \mananamtext ಪುಸ್ತಕಗಳು, ಗುರುಗಳು ಮತ್ತು ಇತರ ಮಾಧ್ಯಮಗಳ ಮೂಲಕ ನಾನು ಹೇಗೆ ಜ್ಞಾನವನ್ನು ಅರಸುತ್ತೇನೆಯೋ ಹಾಗೆಯೇ, ನನ್ನೊಳಗೇ ಜ್ಞಾನವನ್ನು ಅರಸಬಹುದು ಎಂಬ ಅರಿವು ನನಗಿದೆಯೇ? ಜ್ಞಾನವನ್ನು ಆಂತರಿಕವಾಗಿ ಅರಸುವ ಪ್ರಕ್ರಿಯೆ ಹಾಗೂ ಬಾಹ್ಯವಾಗಿ ಅರಸುವ ಪ್ರಕ್ರಿಯೆಗಳ ಬಗ್ಗೆ ನನ್ನ ತಿಳುವಳಿಕೆ ಏನು? ಇದರ ಅರ್ಥ ನಾನು ಬರಿದೇ ನನ್ನ ದೇಹ ಮತ್ತು ಅದರ ವಿವಿಧ ಪ್ರಕ್ರಿಯೆಗಳ ಬಗ್ಗೆ ತಿಳಿದುಕೊಳ್ಳುವುದೆಂದೇ ಅಥವಾ ಅದು ಮನಸ್ಸು ಮತ್ತು ವಿವಿಧ ಮಾನಸಿಕ ಸ್ಥಿತಿಗಳನ್ನು  ತಿಳಿದುಕೊಳ್ಳುವುದೇ? ಅಥವಾ, ನಾನು ಇನ್ನೂ ಪತ್ತೆಹಚ್ಚದ, ಹೆಚ್ಚು ಗಾಢವಾದ ವಿಷಯವನ್ನು ಇದು ಒಳಗೊಂಡಿದೆಯೇ? 
\end{mananam}
\WritingHand\enspace\textbf{ಆತ್ಮ ವಿಮರ್ಶೆ}\\
\begin{inspiration}{\mananamfont ಸ್ಫೂರ್ತಿ}
\small \mananamtext ಆತ್ಮಜ್ಞಾನವು ಶ್ರೇಷ್ಠವಾದ ಶುದ್ಧಿಕಾರಕವಾಗಿದೆ. ಎಲ್ಲಾ ಆಧ್ಯಾತ್ಮಿಕ ಅಭ್ಯಾಸಗಳು ನಮ್ಮನ್ನು ಆತ್ಮಜ್ಞಾನದತ್ತ ಕರೆತರಲು ಇರುವ ಒಂದು ಸಾಧನ. ಈ ಆಧ್ಯಾತ್ಮಿಕ ವಿವೇಕದಲ್ಲಿ ಒಬ್ಬನು ಸ್ಥಿರವಾಗುವವರೆಗೂ, ಅವನು ಅಕಾಲಿಕವಾಗಿ ತನ್ನ ಜೀವನದಲ್ಲಿ ಕರ್ತವ್ಯಗಳನ್ನೂ ಆಧ್ಯಾತ್ಮಿಕ ಅಭ್ಯಾಸಗಳನ್ನೂ  ತ್ಯಜಿಸಬಾರದು. 
\end{inspiration}

\newpage
\begin{mananam}{\mananamfont {ಮನನ ಶ್ಲೋಕ - ೩೯, ೪೦}}
\small \mananamtext ನಂಬಿಕೆ ಮತ್ತು ಜ್ಞಾನದ ಮಧ್ಯೆ ಇರುವ ಸಂಬಂಧವೇನು? ಅನುಮಾನಗಳು ಮತ್ತು ಆತ್ಮವಿಶ್ವಾಸದ ಕೊರತೆ ಹೇಗೆ ವಿನಾಶಕಾರಿಯಾಗಿವೆ? ಪ್ರಶ್ನೆಗಳು ಮತ್ತು ಸಂದೇಹಗಳು ವಿನಾಶಕಾರಿ ಮತ್ತು ಹಾನಿಕಾರಕವಾಗುವ ಮೊದಲೇ ಅವುಗಳನ್ನು ಹೇಗೆ ಪರಿಹರಿಸಿಕೊಳ್ಳಬೇಕು? ಜ್ಞಾನವು ನನಗೆ ಪರಮ ಶಾಂತಿಯ ಸ್ಥಿತಿಗೆ ಹೇಗೆ ತರಬಲ್ಲದು? ಇಂದ್ರಿಯಗಳನ್ನು ನಿಗ್ರಹಪಡಿಸಲು ಬೇಕಾದಂತಹ ಜ್ಞಾನದ ಸ್ವಭಾವವೇನು?  ಇಂದ್ರಿಯಗಳನ್ನು ತರಬೇತಿಗೊಳಿಸಿ ನನ್ನನ್ನು ಶಾಂತಿಯತ್ತ ಕೊಂಡೊಯ್ಯಲು, ಧ್ಯಾನವು ಒಂದು ಸಾಧನವಾಗಬಲ್ಲದೇ?
\end{mananam}
\WritingHand\enspace\textbf{ಆತ್ಮ ವಿಮರ್ಶೆ}\\
\begin{inspiration}{\mananamfont ಸ್ಫೂರ್ತಿ}
\small \mananamtext ಮಾನವ ಕುಲದ ದೊಡ್ಡ ಖಾಯಿಲೆ ಎಂದರೆ ನಮ್ಮ ಆತ್ಮವನ್ನೇ ನಾವು ಅನುಮಾನಿಸುವುದು. ದೇವತೆಗಳು ಕೂಡ ಇಂಥವರಿಗೆ ಸಹಾಯ ಮಾಡಲಾರರು. ಸಂದೇಹಗಳು ಒಬ್ಬರನ್ನು ವಿಚಾರಕ್ಕೆ ಪ್ರೇರೇಪಿಸಿದಾಗ ಉಪಯುಕ್ತವಾಗುತ್ತವೆ, ಆದರೆ ಅವು ಶಂಕಾಶೀಲತೆ ಮತ್ತು ನಕಾರಾತ್ಮಕತೆಯ ಕಡೆಗೆ ತಳ್ಳಿದರೆ ಹಾನಿಕಾರಕವಾಗುತ್ತವೆ. ಆತ್ಮ ಜ್ಞಾನದ ಕೊರತೆ ಇರುವವರೆಗೂ, ಒಬ್ಬನು ಭ್ರಮೆಯಲ್ಲಿಯೇ ಉಳಿಯುತ್ತಾನೆ ಮತ್ತು ಈ ಭ್ರಮೆಯು ಸಂಘರ್ಷ ಮತ್ತು ಹತಾಶೆಯ ಜೀವನದತ್ತ ಕೊಂಡೊಯ್ಯುತ್ತದೆ. 
\end{inspiration}
\newpage

\slcol{\Index{ತಸ್ಮಾದಜ್ಞಾನಸಂಭೂತಂ} ಹೃತ್ಸ್ಥಂ ಜ್ಞಾನಾಸಿನಾತ್ಮನಃ ।\\
ಛಿತ್ತ್ವೈನಂ ಸಂಶಯಂ ಯೋಗಮಾತಿಷ್ಠೋತ್ತಿಷ್ಠ ಭಾರತ ॥ ೪೨ ॥}
\cquote{ಆದ್ದರಿಂದ ಭಾರತ, ಅಜ್ಞಾನದಿಂದ ಮನಸ್ಸನ್ನು  ಕಾಡುತ್ತಿರುವ ಈ ಸಂಶಯವನ್ನು ಜ್ಞಾನ ರೂಪವಾದ ಕತ್ತಿಯಿಂದ ಕತ್ತರಿಸಿ ಪರದಾಸೆಯನ್ನು ಬಿಟ್ಟು ಕರ್ಮವನ್ನು ಮಾಡು, ಏಳು.}

\chapEndSloka{ಜ್ಞಾನಕರ್ಮಸನ್ಯಾಸಯೋಗ}
