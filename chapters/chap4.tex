\slcol{ಶ್ರೀಭಗವಾನುವಾಚ ।\\
\Index{ಇಮಂ ವಿವಸ್ವತೇ ಯೋಗಂ} ಪ್ರೋಕ್ತವಾನಹಮವ್ಯಯಮ್ ।\\
ವಿವಸ್ವಾನ್ಮನವೇऽऽಮನುರಿಕ್ಷ್ವಾಕವೇऽಬ್ರವೀತ್ ॥ 1 ॥}
\cquote{ಶ್ರೀ ಭಗವಂತನು ಹೇಳಿದನು,\\
 ಸನಾತನವಾದ ಈ ಯೋಗ ವಿದ್ಯೆಯನ್ನು ನಾನು ಮೊದಲು ಸೂರ್ಯನಿಗೆ ಹೇಳಿದನು ಸೂರ್ಯ ಮನುವಿಗೆ ಹೇಳಿದನು, ಮನು ಇಕ್ಷ್ಶ್ವಕುವಿಗೆ  ಹೇಳಿದನು.\\}
\slcol{\Index{ಏವಂ ಪರಂಪರಾಪ್ರಾಪ್ತ}ಮಿಮಂ ರಾಜರ್ಷಯೋ ವಿದುಃ ।\\
ಸ ಕಾಲೇನೇಹ ಮಹತಾ ಯೋಗೋ ನಷ್ಟಃ ಪರಂತಪ ॥ 2 ॥}
\cquote{ಅರ್ಜುನ, ಹೀಗೆ ಒಬ್ಬರಿಂದ ಒಬ್ಬರಿಗೆ ಬಂದ ಇದನ್ನು ರಾಜಶ್ರೀಗಳು ತಿಳಿದುಕೊಂಡರು. ಅಲ್ಲಿಂದ ಬಹಳ ಕಾಲ ಸಂದು ಈ ಯೋಗ ಪರಂಪರೆ ಕಾಲ ಗರ್ಭದಲ್ಲಿ ಮರೆಯಾಯಿತು.\\}
\slcol{\Index{ಸ ಏವಾಯಂ ಮಯಾ ತೇऽದ್ಯ} ಯೋಗಃ ಪ್ರೋಕ್ತಃ ಪುರಾತನಃ ।\\
ಭಕ್ತೋऽಸಿ ಮೇ ಸಖಾ ಚೇತಿ ರಹಸ್ಯಂ ಹ್ಯೇತದುತ್ತಮಮ್ ॥ 3 ॥}
\cquote{ಬಹು ಹಿಂದಿನಿಂದ ಬಂದ ಈ ಯೋಗವನ್ನೇ ನೀನು ಭಕ್ತನು ಗೆಳೆಯನೂ ಆಗಿರುವುದರಿಂದ ನಿನಗೆ ಈಗ ಹೇಳಿದೆನು. ಇದು ಬಹು ರಹಸ್ಯವಾದ ವಿದ್ಯೆ.\\}
\slcol{ಅರ್ಜುನ ಉವಾಚ ।\\
\Index{ಅಪರಂ ಭವತೋ ಜನ್ಮ} ಪರಂ ಜನ್ಮ ವಿವಸ್ವತಃ ।\\
ಕಥಮೇತದ್ವಿಜಾನೀಯಾಂ ತ್ವಮಾದೌ ಪ್ರೋಕ್ತವಾನಿತಿ ॥ 4 ॥}
\cquote{ಅರ್ಜುನನು ಹೇಳಿದನು,\\
ನೀನು ಇತ್ತೀಚಿಗೆ ಜನಿಸಿದವನು, ಸೂರ್ಯನು ತುಂಬಾ ಪುರಾತನದವನು. ಹೀಗಿರುವಾಗ ನೀನು ಇದನ್ನು ಮೊದಲು ಹೇಳಿದವನೆಂದು ನಾನು ಹೇಗೆ ತಿಳಿಯಲಿ?\\}


\newpage
\begin{mananam}{\mananamfont ಮನನ ಶ್ಲೋಕ - \textenglish{1,2}}
\footnotesize \mananamtext ಪ್ರಾಚೀನ ಸಂಪ್ರದಾಯಗಳ ಬಗ್ಗೆ ನನ್ನ ತಿಳುವಳಿಕೆ ಏನು?ಇದಕ್ಕಾಗಿ ತಮ್ಮ ಜೀವನವನ್ನೇ ಮುಡಿಪಾಗಿಟ್ಟಿರುವ ಋಷಿಗಳು ಮತ್ತು ಗುರುಗಳು, ಪರಂಪರೆಯಿಂದ ಬಂದ ಬೋಧನೆಗಳನ್ನು ಸಂರಕ್ಷಿಸಿದವರಿಗೆ ಹೇಗೆ ತಾನೇ ಮೆಚ್ಚುಗೆ ವ್ಯಕ್ತಪಡಿಸಬಹುದು? ನನ್ನ ಜೀವನದಲ್ಲಿ ಈ ಬೋಧನೆಗಳನ್ನು  ಅಳವಡಿಸಿಕೊಳ್ಳುವ ಮೂಲಕ ಯಾವ ಪ್ರಯೋಜನಗಳನ್ನು ಪಡೆಯಬಹುದು ಅಥವಾ ನೋಡಲು ಆಶಿಸಬಹುದು. 
\end{mananam}
\WritingHand\enspace\textbf{ಆತ್ಮ ವಿಮರ್ಶೆ}\\
\begin{inspiration}{\mananamfont ಸ್ಪೂರ್ತಿ}
\footnotesize \mananamtext ಆಧುನಿಕ ಕಾಲದಲ್ಲಿ ಸಮಾಜಕ್ಕೆ ಎಲ್ಲದಕ್ಕೂ ವೈಜ್ಞಾನಿಕ ಪುರಾವೆಗಳು ಬೇಕಾಗುತ್ತದೆ. ನಮ್ಮ ಹಳೆಯ ಪಠ್ಯಪುಸ್ತಕಗಳಲ್ಲಿ ಬರುವ ಐತಿಹಾಸಿಕ ಪಾತ್ರಗಳ ವಾಸ್ತವಿಕತೆಯನ್ನು ಪ್ರಶ್ನಿಸಬಹುದು. ಗೀತೆಗೆ ಪುರಾವೆಯನ್ನು ಇತಿಹಾಸ ಅಥವಾ ವಿಜ್ಞಾನದಲ್ಲಿ ಹುಡುಕಬಾರದು. ಬದಲಾಗಿ ಗೀತೆಯ ಬೋಧನೆಗಳನ್ನು ನಮ್ಮ ಜೀವನದಲ್ಲಿ ಅಳವಡಿಸಿಕೊಳ್ಳಬೇಕು.  ನಮಗೆ ಇದರಿಂದ ಮಾನಸಿಕ ಶಾಂತಿ ಮತ್ತು ಮಾನಸಿಕ ಸ್ಪಷ್ಟತೆಯನ್ನು ನೀಡುತ್ತದೆಯೇ ಎಂದು ಪರಿಶೀಲಿಸಬೇಕು. ಸಣ್ಣಪುಟ್ಟ ಭಾಹ್ಯವಿಚಾರಗಳಿಗೆ ಗಮನಕೊಟ್ಟರೆ ನಮಗೆ ಅರಿವಾಗುವ ಆಂತರಿಕ ಪರಿವರ್ತನೆಯ ಸಂದೇಶವನ್ನು ಕಳೆದುಕೊಳ್ಳುತ್ತೇವೆ.
\end{inspiration}
\newpage

\slcol{ಶ್ರೀಭಗವಾನುವಾಚ ।\\
\Index{ಬಹೂನಿ ಮೇ ವ್ಯತೀತಾನಿ} ಜನ್ಮಾನಿ ತವ ಚಾರ್ಜುನ ।\\
ತಾನ್ಯಹಂ ವೇದ ಸರ್ವಾಣಿ ನ ತ್ವಂ ವೇತ್ಥ ಪರಂತಪ ॥ 5 ॥}
\cquote{ಶ್ರೀ ಭಗವಂತನು ಹೇಳಿದನು,\\
 ಅರ್ಜುನ, ನಿನಗೂ ನನಗೂ ಅನೇಕ ಜನ್ಮಗಳು ಆಗಿ ಹೋದವು, ನಾನು ಆ ಎಲ್ಲವನ್ನು ಬಲ್ಲೆನು. ನಿನಗೆ ಮಾತ್ರ ತಿಳಿದಿಲ್ಲ.\\}
\slcol{\Index{ಅಜೋऽಪಿ ಸನ್ನವ್ಯಯಾತ್ಮಾ} ಭೂತಾನಾಮೀಶ್ವರೋऽಪಿ ಸನ್ ।\\
ಪ್ರಕೃತಿಂ ಸ್ವಾಮಧಿಷ್ಠಾಯ ಸಂಭವಾಮ್ಯಾತ್ಮಮಾಯಯಾ ॥ 6 ॥}
\cquote{ನನಗೆ ಹುಟ್ಟಿಲ್ಲ ನನ್ನ ದೇಹಕ್ಕೆ ಅಳಿಲಿಲ್ಲ ನಾನು ಎಲ್ಲ ಪ್ರಾಣಿಗಳ ಒಡೆಯ ನನ್ನ ಮಾಯಾ ಶಕ್ತಿಯನ್ನು ಮುಂದಿಟ್ಟುಕೊಂಡು ಸ್ವೇಚ್ಚೆಯಿಂದ  ನಾನು ಹುಟ್ಟುತ್ತೇನೆ.\\}
\slcol{\Index{ಯದಾ ಯದಾ ಹಿ ಧರ್ಮಸ್ಯ} ಗ್ಲಾನಿರ್ಭವತಿ ಭಾರತ ।\\
ಅಭ್ಯುತ್ಥಾನಮಧರ್ಮಸ್ಯ ತದಾತ್ಮಾನಂ ಸೃಜಾಮ್ಯಹಮ್ ॥ 7 ॥}
\cquote{ಅರ್ಜುನ,ಯಾವ ಯಾವ ಕಾಲದಲ್ಲಿ ಧರ್ಮದ ಇಲುಗಡೆಯೂ ಅಧರ್ಮದ ಏರಿಕೆಯೂ ಆಗುತ್ತದೋ ಆಗೆಲ್ಲ ನಾನು ಅವತರಿಸಿ ಬರುತ್ತೇನೆ.\\}
\slcol{\Index{ಪರಿತ್ರಾಣಾಯ ಸಾಧೂನಾಂ} ವಿನಾಶಾಯ ಚ ದುಷ್ಕೃತಾಮ್ ।\\
ಧರ್ಮಸಂಸ್ಥಾಪನಾರ್ಥಾಯ ಸಂಭವಾಮಿ ಯುಗೇ ಯುಗೇ ॥ 8 ॥}
\cquote{ಒಳ್ಳೆಯವರ ಉದ್ಧಾರಕ್ಕಾಗಿ ಕೆಟ್ಟವರ ನಿರ್ನಾಮಕ್ಕಾಗಿ ಮತ್ತು ಧರ್ಮವನ್ನು ನೆಲೆಗೊಳಿಸುವುದಕ್ಕಾಗಿ ಪ್ರತಿಯೊಂದು ಯುಗದಲ್ಲಿಯೂ ನಾನು ಹುಟ್ಟಿ ಬರುತ್ತೇನೆ.\\}
\slcol{\Index{ಜನ್ಮ ಕರ್ಮ ಚ ಮೇ ದಿವ್ಯ}ಮೇವಂ ಯೋ ವೇತ್ತಿ ತತ್ತ್ವತಃ ।\\
ತ್ಯಕ್ತ್ವಾ ದೇಹಂ ಪುನರ್ಜನ್ಮ ನೈತಿ ಮಾಮೇತಿ ಸೋऽರ್ಜುನ ॥ 9 ॥}
\cquote{ದಿವ್ಯವಾದ ನನ್ನ ಹುಟ್ಟನ್ನೂ ಕರ್ಮವನ್ನೂ ಹೀಗೆ ಯಥಾವತ್ತಾಗಿ ತಿಳಿದವನು ದೇಹವನ್ನು ಬಿಟ್ಟ ಮೇಲೆ ಮತ್ತೆ ಹುಟ್ಟನ್ನು ಹೊಂದುವುದಿಲ್ಲ. ಅವನು ನನ್ನನ್ನು ಪಡೆಯುತ್ತಾನೆ.\\}
\slcol{\Index{ವೀತರಾಗಭಯಕ್ರೋಧಾ} ಮನ್ಮಯಾ ಮಾಮುಪಾಶ್ರಿತಾಃ ।\\
ಬಹವೋ ಙ್ಞಾನತಪಸಾ ಪೂತಾ ಮದ್ಭಾವಮಾಗತಾಃ ॥ 10 ॥}
\cquote{ಪ್ರೀತಿ ಹೆದರಿಕೆ ಸಿಟ್ಟು ಇವುಗಳಿಲ್ಲದವನಾಗಿ ನನ್ನಲ್ಲಿಯೇ ಮನಸ್ಸಿಟ್ಟು ನನಗೆ ಶರಣು ಬಂದ ಅನೇಕರು ಜ್ಞಾನವೆಂಬ ತಪಸ್ಸಿನಿಂದ ಶುದ್ದರಾಗಿ ನನ್ನನ್ನು ಪಡೆದಿರುತ್ತಾರೆ.\\}
\slcol{\Index{ಯೇ ಯಥಾ ಮಾಂ ಪ್ರಪದ್ಯಂತೇ} ತಾಂಸ್ತಥೈವ ಭಜಾಮ್ಯಹಮ್ ।\\
ಮಮ ವರ್ತ್ಮಾನುವರ್ತಂತೇ ಮನುಷ್ಯಾಃ ಪಾರ್ಥ ಸರ್ವಶಃ ॥ 11 ॥}
\cquote{ಅರ್ಜುನ, ನನ್ನನ್ನು ಯಾರು ಹೇಗೆ ಶರಣೆಂದರೆ ನಾನು ಅವರಿಗೆ ಹಾಗೆ ಅನುಗ್ರಹಿಸುತ್ತೇನೆ. ಮನುಷ್ಯರು ಎಲ್ಲಿ ಯಾವ ದಾರಿಯಲ್ಲಿ ಹೋದರು ಕೊನೆಗೆ ಬಂದು ಸೇರುವುದು ನನ್ನತ್ತವೆ.\\}

\newpage
\begin{mananam}{\mananamfont ಮನನ ಶ್ಲೋಕ - \textenglish{5}}
\footnotesize \mananamtext ಪುನರ್ಜನ್ಮದ ಸಿದ್ದಾಂತವನ್ನು ನಾನು ಹೇಗೆ ಅರ್ಥ ಮಾಡಿಕೊಂಡಿದ್ದೇನೆ ? ಅದರಲ್ಲಿ ನನ್ನ ಅನುಮಾನಗಳೇನು?  ಪುನರ್ಜನ್ಮ ಮತ್ತು ಕರ್ಮದ ಸಿದ್ದಾಂತವನ್ನು ಅರ್ಥ ಮಾಡಿಕೊಂಡರೆ ಈ ಜೀವನ ನಡೆಸಲು ಉಪಯುಕ್ತವಾಗಬಹುದೇ? ಸಮಾಜದಲ್ಲಿ ಆಧಾರವಾಗಿರುವ ಸಮತೋಲನ ಮತ್ತು ನ್ಯಾಯವನ್ನು ನೋಡಲು ಈ ಸಿದ್ಧಾಂತವು ನನಗೆ ಸಹಾಯ ಮಾಡಬಹುದೇ?
\end{mananam}
\WritingHand\enspace\textbf{ಆತ್ಮ ವಿಮರ್ಶೆ}\\
\begin{inspiration}{\mananamfont ಸ್ಪೂರ್ತಿ}
\footnotesize \mananamtext ಯಾವುದೇ ಕ್ಷೇತ್ರದಲ್ಲಿರುವ ಬಾಲ ಪ್ರತಿಭೆಗಳನ್ನು ನೋಡಿದಾಗ ಪುನರ್ಜನ್ಮವನ್ನು ನಂಬುವುದು ಕಷ್ಟವೇನಲ್ಲ. ಅವರ ಪ್ರತಿಭೆ ಮತ್ತು ಕೌಶಲ್ಯಗಳು ಹೇಗೆ ಬಂದಿವೆ. ಪ್ರತಿಯೊಂದು ಜನ್ಮದಲ್ಲಿಯೂ ನಮ್ಮ ಸ್ಮರಣೆಯು ಅಳಿಸಿರುವುದರಿಂದ ಈ ಜೀವನವನ್ನು ಹೊಸದಾಗಿ ಪ್ರಾರಂಭಿಸುವ ಮತ್ತು ನಕರಾತ್ಮಕ ಪ್ರವೃತ್ತಿಗಳನ್ನು ಬದಲಾಯಿಸಿಕೊಳ್ಳಲು ಸಹಾಯವಾಗುತ್ತದೆ. ಭೂತಕಾಲವು ವರ್ತಮಾನದ ನೀಲನಕ್ಷೆಯಾಗಿದ್ದರೆ ವರ್ತಮಾನವು ಉತ್ತಮ ಭವಿಷ್ಯವನ್ನು ರೂಪಿಸಲು ನಮಗೆ ಅವಕಾಶವನ್ನು ನೀಡುತ್ತದೆ.\\
\end{inspiration}
\newpage
\newpage
\begin{mananam}{\mananamfont ಮನನ ಶ್ಲೋಕ - \textenglish{7,8}}
\footnotesize \mananamtext ಈ ಸಮಾಜದಲ್ಲಿ ನಡೆಯುವ ಅನ್ಯಾಯಗಳಿಗೆ ನನ್ನ ಪ್ರತಿಕ್ರಿಯೆ ಏನು? ಒಬ್ಬ ವ್ಯಕ್ತಿ ಅಥವಾ ಒಂದು ಗುಂಪು ಅನ್ಯಾಯ ಮಾಡುತ್ತಿದ್ದರೂ ನಿರ್ಭಯವಾಗಿ ತಪ್ಪಿಸಿಕೊಳ್ಳುವುದನ್ನು ನೋಡಿದಾಗ ನನಗೆ ಹೇಗೆ ಅನ್ನಿಸುತ್ತದೆ? ಭಗವಂತನ  ವಾಗ್ದಾನದಲ್ಲಿ ನಾನು ಸಮಾಧಾನವನ್ನು ಕಂಡುಕೊಳ್ಳಬಹುದೇ? ಯಾವಾಗ ಬಾಹ್ಯಬದಲಾವಣೆಯು ಸಾಧ್ಯವಾಗದಂತಹ ಪರಿಸ್ಥಿತಿಯಲ್ಲಿ ಇದನ್ನು ನಾನು ಆಂತರಿಕ ಸ್ವೀಕಾರಕ್ಕಾಗಿ ಉಪಯೋಗಿಸಬಹುದೇ?
\end{mananam}
\WritingHand\enspace\textbf{ಆತ್ಮ ವಿಮರ್ಶೆ}\\
\begin{inspiration}{\mananamfont ಸ್ಪೂರ್ತಿ}
\footnotesize \mananamtext ನಾವು ನಂಬುವ ಮತ್ತು ಉನ್ನತ ಶಕ್ತಿಯ ಪರಿಕಲ್ಪನೆಯು ಕೇವಲ ಪ್ರೀತಿ ಮತ್ತು ದಯೆಯಿಂದ ಕೂಡಿರದೆ ಶಕ್ತಿಯುತ ಮತ್ತು ನ್ಯಾಯಯುತವಾಗಿ ಕೂಡ ಇರಬೇಕು. ಆಗ ಅಂತಹ ದೇವರಲ್ಲಿ ಆಶ್ರಯ ಪಡೆಯಲು  ಸೂಕ್ತ, ಅಂತಹ ದೇವರು ನಮ್ಮಲ್ಲಿ ಪ್ರೀತಿ ಮತ್ತು ಶಕ್ತಿಯನ್ನು ತುಂಬುತ್ತಾರೆ.
\end{inspiration}
\newpage

\begin{mananam}{\mananamfont ಮನನ ಶ್ಲೋಕ - \textenglish{10}}
\footnotesize \mananamtext ಭಗವಂತನನ್ನು ಪಡೆಯುವುದು ದೈವಿಕ ಸ್ಥಿತಿಯಲ್ಲಿರುವುದನ್ನು ಸಂಕೇತಿಸುತ್ತದೆಯೇ? ಭಯ ಕೋಪ ಇನ್ನೂ ಹಲವಾರು ಭಾವನೆಗಳು ನನ್ನ ಮನಸ್ಸಿಗೆ ತೊಂದರೆಗೊಳಿಸದಂತಹ ಸ್ಥಿತಿಯನ್ನು ಪಡೆಯಬಹುದೇ? ಜ್ಞಾನದ ಅಗ್ನಿಯಿಂದ ಶುದ್ಧ ವಾಗುವುದು ಎಂದರೆ ಏನು? ಬಲವಾದ ಭಾವನೆಗಳು ನನ್ನ ವ್ಯಕ್ತಿತ್ವದ ಕರಾಳ ಮುಖವನ್ನು ತೋರಿಸಿದಾಗ ಜ್ಞಾನವು ನನಗೆ ಹೇಗೆ ಸಹಾಯ ಮಾಡಬಲ್ಲದು?
\end{mananam}
\WritingHand\enspace\textbf{ಆತ್ಮ ವಿಮರ್ಶೆ}\\
\begin{inspiration}{\mananamfont ಸ್ಪೂರ್ತಿ}
\footnotesize \mananamtext ಯಾವಾಗ ಸುಪ್ತ ಭಾವನೆಗಳು [ಕ್ಲೇಷಗಳು] ನಮ್ಮಲ್ಲಿ ಸಕ್ರಿಯವಾದಾಗ ನಾವು ಸ್ವಲ್ಪ ಮಟ್ಟಿಗೆ ವಿಭಿನ್ನ ವ್ಯಕ್ತಿತ್ವದವರಾಗುತ್ತೇವೆ. ನಮ್ಮ ಉನ್ನತ ಬುದ್ಧಿವಂತಿಕೆಯಿಂದ ಕಾರ್ಯ ನಿರ್ವಹಿಸುವ ನಮ್ಮ  ಸಾಮರ್ಥ್ಯವು ಸುಪ್ತವಾಗುತ್ತದೆ.
\end{inspiration}
\newpage

\begin{mananam}{\mananamfont ಮನನ ಶ್ಲೋಕ - \textenglish{11}}
\footnotesize \mananamtext ದೇವರು ಅಥವಾ ಸರ್ವೋಚ್ಚ ಅಸ್ತಿತ್ವದ ಬಗ್ಗೆ ನನ್ನ ಪರಿಕಲ್ಪನೆ ಏನು? ನಾನು ವಿಶ್ವದ ಉನ್ನತ ಶಕ್ತಿ ಅಥವಾ ನನ್ನೊಳಗಿನ ಶುದ್ಧ ಆತ್ಮನಲ್ಲಿ ನಂಬಿಕೆ ಹೊಂದಲು ಸಾಧ್ಯವೇ? ಪ್ರಪಂಚದ ಪ್ರತಿಯೊಬ್ಬರೂ ದೇವರ ಪರಿಕಲ್ಪನೆಯಲ್ಲಿ ತಮ್ಮದೇ ಆದ ನಂಬಿಕೆಯನ್ನು ಹೊಂದಿದ್ದಾರೆ ಎಂಬುದನ್ನು ನಾನು ಅವಲೋಕನದಿಂದ ಕಲಿಯಬಹುದೇ? ಯಾವುದೇ ಪರಿಕಲ್ಪನೆಯ ನಂಬಿಕೆಯು ಹೇಗೆ ಅಭಿವೃದ್ಧಿಗೊಳಿಸುತ್ತದೆ, ಸುಸ್ಥಿರವಾಗಿದೆ ಮತ್ತು ಬಲಗೊಳ್ಳುತ್ತದೆ ಎಂಬುದನ್ನು ನಾನು ನನ್ನ ಸ್ವಂತ ಜೀವನದಲ್ಲಿ ಗಮನಿಸಬಹುದೇ?
\end{mananam}
\WritingHand\enspace\textbf{ಆತ್ಮ ವಿಮರ್ಶೆ}\\
\begin{inspiration}{\mananamfont ಸ್ಪೂರ್ತಿ}
\footnotesize \mananamtext ಪ್ರತಿಯೊಬ್ಬರೂ ಯಾವ ದೇವರ ಪರಿಕಲ್ಪನೆಯನ್ನು ಹೊಂದಿದ್ದರೂ ಅವರಿಗೆ ಪ್ರತಿಫಲ ನೀಡುವ ಮೂಲಕ ಪ್ರತಿಯೊಬ್ಬರ ನಂಬಿಕೆಯನ್ನು ದೇವರು ಬಲಪಡಿಸುತ್ತಾರೆ ಎಂದು ಹೇಳುವುದು ವಿರೋಧಾಭಾಸವಾಗಿ ಕಾಣಿಸಬಹುದು. ಸನಾತನ ಧರ್ಮದ ದೇವರು ಅಸೂಯೆ ಪಡುವ ದೇವರಲ್ಲ, ಅವರು ಪ್ರತಿಯೊಬ್ಬರಿಗೂ ಅವರವರ ನಂಬಿಕೆಯನ್ನು ಅನುಸರಿಸಲು ಅಧಿಕಾರ ನೀಡುತ್ತಾರೆ. ಈ ಶಾಶ್ವತ ಸತ್ಯದ ಅನುಯಾಯಿಗಳಾಗಿ ನಾವು ಇತರರ ವಿಭಿನ್ನ ನಂಬಿಕೆಗಳ ಬಗ್ಗೆ ತೆರೆದ ಮನಸ್ಸು ಮತ್ತು ಸಹಿಷ್ಣುಗಳಾಗಿರಬೇಕು.
\end{inspiration}
\newpage

\slcol{\Index{ಕಾಂಕ್ಷಂತಃ ಕರ್ಮಣಾಂ} ಸಿದ್ಧಿಂ ಯಜಂತ ಇಹ ದೇವತಾಃ ।\\
ಕ್ಷಿಪ್ರಂ ಹಿ ಮಾನುಷೇ ಲೋಕೇ ಸಿದ್ಧಿರ್ಭವತಿ ಕರ್ಮಜಾ ॥ 12 ॥}
\cquote{ಕರ್ಮಗಳಿಗೆ ಫಲವನ್ನು ಬಯಸುವವರಾಗಿ ದೇವತೆಗಳನ್ನು ಆರಾಧಿಸುತ್ತಾರೆ. ಏಕೆಂದರೆ ಮನುಷ್ಯ ಲೋಕದಲ್ಲಿ ಕರ್ಮದಿಂದ ಸಿದ್ದಿ ಬಹುಬೇಗ ಆಗುತ್ತದೆ.\\}
\slcol{\Index{ಚಾತುರ್ವರ್ಣ್ಯಂ ಮಯಾ} ಸೃಷ್ಟಂ ಗುಣಕರ್ಮವಿಭಾಗಶಃ ।\\
ತಸ್ಯ ಕರ್ತಾರಮಪಿ ಮಾಂ ವಿದ್ಧ್ಯಕರ್ತಾರಮವ್ಯಯಮ್ ॥ 13 ॥}
\cquote{ಗುಣಕ್ಕೂ ಕರ್ಮಕ್ಕೂ ತಕ್ಕಂತೆ ನಾಲ್ಕು ವರ್ಣಗಳನ್ನು ನಾನು ಹುಟ್ಟಿದ್ದೇನೆ. ನಾನು ಅದನ್ನು ಮಾಡಿದವನಾದರೂ ನಾನು ಕರ್ತಾರನಲ್ಲ. ಏಕೆಂದರೆ ನನಗೆ ಕರ್ತೃತ್ವದ  ಲೇಪವಿಲ್ಲ.\\}
\slcol{\Index{ನ ಮಾಂ ಕರ್ಮಾಣಿ ಲಿಂಪಂತಿ} ನ ಮೇ ಕರ್ಮಫಲೇ ಸ್ಪೃಹಾ ।\\
ಇತಿ ಮಾಂ ಯೋऽಭಿಜಾನಾತಿ ಕರ್ಮಭಿರ್ನ ಸ ಬಧ್ಯತೇ ॥ 14 ॥}
\cquote{ನನ್ನನ್ನು ಕರ್ಮಗಳು ಸೋಂಕುವುದಿಲ್ಲ, ನನಗೆ ಕರ್ಮದ ಫಲದಾಸೆ ಇಲ್ಲ, ನನ್ನನ್ನು ಹೀಗೆ ಯಾವನು ತಿಳಿಯುತ್ತಾನೋ ಅವನು ಕರ್ಮದ ಕಟ್ಟಿಗೆ ಒಳಗಾಗುವುದಿಲ್ಲ.\\}
\slcol{\Index{ಏವಂ ಙ್ಞಾತ್ವಾ ಕೃತಂ} ಕರ್ಮ ಪೂರ್ವೈರಪಿ ಮುಮುಕ್ಷುಭಿಃ ।\\
ಕುರು ಕರ್ಮೈವ ತಸ್ಮಾತ್ತ್ವಂ ಪೂರ್ವೈಃ ಪೂರ್ವತರಂ ಕೃತಮ್ ॥ 15 ॥}
\cquote{ಮೋಕ್ಷವನ್ನು ಬಯಸಿದ ಹಿಂದಿನವರು ಕೂಡ ಹೀಗೆ ತಿಳಿದು ಕರ್ಮವನ್ನು ಮಾಡಿದರು. ಆದ್ದರಿಂದ ಹಿಂದಿನವರು ಕೂಡ ಮಾಡಿದ ಪುರಾತನವಾದ ಕರ್ಮ ಮಾರ್ಗವನ್ನೇ ನೀನು ಹಿಡಿ. \\}
\slcol{\Index{ಕಿಂ ಕರ್ಮ ಕಿಮಕರ್ಮೇತಿ} ಕವಯೋऽಪ್ಯತ್ರ ಮೋಹಿತಾಃ ।\\
ತತ್ತೇ ಕರ್ಮ ಪ್ರವಕ್ಷ್ಯಾಮಿ ಯಜ್ಙ್ಞಾತ್ವಾ ಮೋಕ್ಷ್ಯಸೇऽಶುಭಾತ್ ॥ 16 ॥}
\cquote{ಕರ್ಮವು ಯಾವುದು ಮತ್ತು ಅಕರ್ಮವು ಯಾವುದು? ಈ ಪ್ರಕಾರವಾಗಿ ಇದರ ನಿರ್ಣಯ ಮಾಡುವಲ್ಲಿ ಬುದ್ಧಿವಂತರಾದ ಪುರುಷರು ಸಹ ಮೋಹಿತರಾಗುತ್ತಾರೆ.ಆದ್ದರಿಂದ ಆ ಕರ್ಮತತ್ವವನ್ನು ನಾನು ನಿನಗೆ ಚೆನ್ನಾಗಿ ತಿಳಿಯ ಹೇಳುವೆನು.ಅದನ್ನು ತಿಳಿದು ನೀನು ಅಶುಭದಿಂದ ಅರ್ಥತ್ ಕರ್ಮ ಬಂಧನದಿಂದ ಮುಕ್ತನಾಗಿ ಬಿಡುವೆ.\\}
\slcol{\Index{ಕರ್ಮಣೋ ಹ್ಯಪಿ ಬೋದ್ಧವ್ಯಂ} ಬೋದ್ಧವ್ಯಂ ಚ ವಿಕರ್ಮಣಃ ।\\
ಅಕರ್ಮಣಶ್ಚ ಬೋದ್ಧವ್ಯಂ ಗಹನಾ ಕರ್ಮಣೋ ಗತಿಃ ॥ 17 ॥}
\cquote{ಕರ್ಮದ ಸ್ವರೂಪವನ್ನು ಕೂಡ ತಿಳಿಯಬೇಕು ಮತ್ತು ಅಕರ್ಮದ ಸ್ವರೂಪವನ್ನೂ ಕೂಡ ತಿಳಿಯಬೇಕು. ಹಾಗೆಯೇ ವಿಕರ್ಮದ ಸ್ವರೂಪವನ್ನು ಸಹ ತಿಳಿಯಬೇಕು. ಏಕೆಂದರೆ ಕರ್ಮದ ಗತಿಯು ಗಹನವಾಗಿದೆ.\\}
\slcol{\Index{ಕರ್ಮಣ್ಯಕರ್ಮ ಯಃ ಪಶ್ಯೇದ}ಕರ್ಮಣಿ ಚ ಕರ್ಮ ಯಃ ।\\
ಸ ಬುದ್ಧಿಮಾನ್ಮನುಷ್ಯೇಷು ಸ ಯುಕ್ತಃ ಕೃತ್ಸ್ನಕರ್ಮಕೃತ್ ॥ 18 ॥}
\cquote{ಯಾವ ಮನುಷ್ಯನು ಕರ್ಮದಲ್ಲಿ ಅಕರ್ಮವನ್ನು ನೋಡುತ್ತಾನೋ ಮತ್ತು ಯಾರು ಅಕರ್ಮದಲ್ಲಿ ಕರ್ಮವನ್ನು ನೋಡುತ್ತಾನೋ ಅವನು ಮನುಷ್ಯರಲ್ಲಿ ಬುದ್ಧಿವಂತನಾಗಿದ್ದಾನೆ. ಮತ್ತು ಆ ಯೋಗಿಯು ಸಮಸ್ತ ಕರ್ಮಗಳನ್ನು ಮಾಡುವವನಾಗಿದ್ದಾನೆ.}

\begin{mananam}{\mananamfont ಮನನ ಶ್ಲೋಕ - \textenglish{14}}
\footnotesize \mananamtext ನನ್ನ ಕರ್ತವ್ಯಗಳನ್ನು ಮಾಡಿದ ನಂತರ ಅದರ ಕುರುಹುಗಳು ನನ್ನ ಮನಸ್ಸಿನಲ್ಲಿ ಉಳಿಯದೆ ಇರುವ ಮಹತ್ವವನ್ನು ನಾನು ಅರಿತುಕೊಳ್ಳುವೆನೇ? ನಾನು ಮಾಡುವ ಯಾವುದೇ ಕೆಲಸಗಳ ಪರಿಣಾಮಗಳಿಂದ ಮುಕ್ತನಾಗಿರುವುದರ ಅರ್ಥವೇನು? ನನ್ನ ಎಲ್ಲ ಕೆಲಸಕ್ಕೂ ನಾನು ವೈಯಕ್ತಿಕ ತೃಪ್ತಿಯ ಬಯಕೆ ಹೊಂದಿದ್ದೇನೆಯೇ? ವೈಯಕ್ತಿಕ ತೃಪ್ತಿಯನ್ನು ಹುಡುಕುವ ಉದ್ದೇಶ ಏನು?
\end{mananam}
\WritingHand\enspace\textbf{ಆತ್ಮ ವಿಮರ್ಶೆ}\\
\begin{inspiration}{\mananamfont ಸ್ಪೂರ್ತಿ}
\footnotesize \mananamtext ಭಗವಂತನು ಸ್ವಾತಂತ್ರ್ಯದ ಸೂತ್ರವನ್ನು ನಮ್ಮೊಂದಿಗೆ ಹಂಚಿಕೊಂಡಿದ್ದಾನೆ. ಅದನ್ನು ಅವನು ಸ್ವತಃ ಬಳಸುತ್ತಾನೆ. ಯಾವಾಗ ಒಬ್ಬನು  ಫಲಿತಾಂಶದ ಆಸೆಯಿಂದ ಮುಕ್ತಾರಾದಾಗ ಯಾವುದೇ ಕೆಲಸದ ಕುರುಹು ಮತ್ತು ಪರಿಣಾಮಗಳು  ಅವನ ಮನಸ್ಸಿನಲ್ಲಿ ನಾಟುವುದಿಲ್ಲ.
\end{inspiration}
\newpage

\begin{mananam}{\mananamfont ಮನನ ಶ್ಲೋಕ - \textenglish{16,17}}
\footnotesize \mananamtext ಸರಿಯಾದ ಕೆಲಸ ಮತ್ತು ತಪ್ಪು ಕೆಲಸದ ನಡುವೆ ನಾನು ಹೇಗೆ ವ್ಯತ್ಯಾಸವನ್ನು ಕಂಡುಹಿಡಿಯುವುದು. ಅದಕ್ಕೆ ಇರುವ ಮಾಪನ ಯಾವುದು? ಅವು ನನಗೆ ಮತ್ತು ನನ್ನ ಕುಟುಂಬಕ್ಕೆ ಯಾವುದು ಒಳ್ಳೆಯದು ಅಥವಾ ವಿಶ್ವಕ್ಕೆ ಯಾವುದು ಒಳ್ಳೆಯದು ಎಂಬ ಮೌಲ್ಯಗಳ ಮೇಲೆ ಆಧರಿಸಿದೆಯೇ? ನನಗೆ ಅಭ್ಯಾಸವಾಗಿರುವ ದಿನನಿತ್ಯದ ಕಾರ್ಯ ಚರಣಿಯಲ್ಲಿ ಅಧ್ಯಾತ್ಮಿಕ ಮೌಲ್ಯಗಳ ಆಧಾರದ ಮೇಲೆ ಹೇಗೆ ಬದಲಾಯಿಸಬಹುದು. ನನ್ನ ಜೀವನದಲ್ಲಿ ಕಠಿಣ ನಿರ್ಧಾರಗಳನ್ನು ತೆಗೆದುಕೊಳ್ಳಬೇಕಾದಾಗ ನಾನು ನನ್ನ ಆತ್ಮಸಾಕ್ಷಿಯನ್ನು ಅನುಸರಿಸುತ್ತೇನೆಯೇ?
\end{mananam}
\WritingHand\enspace\textbf{ಆತ್ಮ ವಿಮರ್ಶೆ}\\
\begin{inspiration}{\mananamfont ಸ್ಪೂರ್ತಿ}
\footnotesize \mananamtext ಈ ಜೀವನದಲ್ಲಿ ಮಾಡುವ ತಪ್ಪು ಕೆಲಸಗಳ ಫಲಿತಾಂಶವು ವ್ಯಸನಗಳು,ಒತ್ತಡ, ನಕರಾತ್ಮಕತೆ ಮತ್ತು ಸಂಘರ್ಷಗಳನ್ನು ಸೃಷ್ಟಿಸುತ್ತವೆ. ಸರಿಯಾದ ಕ್ರಿಯೆಗಳ ಫಲಿತಾಂಶಗಳು ಈ ಜೀವನ ಮತ್ತು ಮುಂದಿನ ಜೀವನದಲ್ಲೂ ಮಾನಸಿಕ ಉನ್ನತಿಯನ್ನು ಸೃಷ್ಟಿಸುತ್ತವೆ.
\end{inspiration}
\newpage

\begin{mananam}{\mananamfont ಮನನ ಶ್ಲೋಕ - \textenglish{18}}
\footnotesize \mananamtext ಯಾವ ಕೆಲಸ ಸರಿ ಮತ್ತು ಯಾವ ಕೆಲಸ ತಪ್ಪು ಎಂದು ಗೊಂದಲಕ್ಕೊಳಗಾದಾಗ ನಾನು ಯಾವುದೇ ಕ್ರಮ ಕೈಗೊಳ್ಳಲು ಹೆದರುತ್ತೇನೆಯೇ? ನಾನು  ಸೋಮಾರಿಯಾಗಿರುವುದರಿಂದ ಅಥವಾ ಪರಿಣಾಮಗಳ ಭಯದಿಂದ ನಾನು ನಿಷ್ಕ್ರಿಯತೆಯನ್ನು ಆರಿಸಿಕೊಳ್ಳುತ್ತೇನೆಯೇ? ನನ್ನ ಕರ್ತವ್ಯಗಳು ಮತ್ತು ಸ್ವಯಂ ಸುಧಾರಣೆ ಮತ್ತು ಅಧ್ಯಾತ್ಮಿಕ ಅಭ್ಯಾಸಗಳಿಗೆ ನನ್ನ ಪ್ರತಿರೋಧಕ್ಕೆ ಕಾರಣವೇನು ಎಂದು ನಾನು ಆತ್ಮಾವಲೋಕನ ಮಾಡುತ್ತೇನೆಯೇ?\\
 ಸುಲಭವಾಗಿ ಕೆಲಸ ಮಾಡುವುದು ಹೇಗೆ ಎಂದು ನಾನು ಅನುಭವಿಸಿದ್ದೇನೆಯೇ? ಹಾಗಿದ್ದಲ್ಲಿ ಅದು ಹೇಗಿತ್ತು ಮತ್ತು ಅಂತಹ ಸ್ಥಿತಿಗೆ ಕಾರಣವಾದ ಅಂಶಗಳು ಯಾವುದು? ನಾನು ಎಂಬುದರಿಂದ ಮುಕ್ತವಾದ ಕರ್ಮಗಳು ನಮ್ಮನ್ನು ಸಶಕ್ತಗೊಳಿಸಬಹುದೇ ಮತ್ತು ಉನ್ನತಿಗೇರಿಸಬಹುದೇ?
\end{mananam}
\WritingHand\enspace\textbf{ಆತ್ಮ ವಿಮರ್ಶೆ}\\
\begin{inspiration}{\mananamfont ಸ್ಪೂರ್ತಿ}
\footnotesize \mananamtext ಕರ್ಮಗಳನ್ನು ಅತ್ಯಂತ ಸುಲಭವಾಗಿ ಮುಗಿಸುವ ರಹಸ್ಯವೇನೆಂದರೆ, ತನ್ನ ಕರ್ತವ್ಯಗಳಲ್ಲಿ ಪೂರ್ಣವಾಗಿ ಮುಳುಗುವುದು. ಎಲ್ಲಿ ಅಹಂ ಇರುವುದಿಲ್ಲವೋ ಅಲ್ಲಿ ದೈವಿಕ ಶಕ್ತಿಯು ಪೂರ್ಣಪ್ರಮಾಣದಲ್ಲಿ ಹರಿಯುತ್ತದೆ. ಇಂತಹ ದೈವಿಕ ಶಕ್ತಿಯ ಹರಿವನ್ನು ವಿರೋಧಿಸದೆ ಇರುವುದು ಜೀವನ ನಡೆಸಲು ಇರುವ ಉತ್ತಮ ಮಾರ್ಗವಾಗಿದೆ.
\end{inspiration}
\newpage

\slcol{\Index{ಯಸ್ಯ ಸರ್ವೇ ಸಮಾರಂಭಾಃ} ಕಾಮಸಂಕಲ್ಪವರ್ಜಿತಾಃ ।\\
ಙ್ಞಾನಾಗ್ನಿದಗ್ಧಕರ್ಮಾಣಂ ತಮಾಹುಃ ಪಂಡಿತಂ ಬುಧಾಃ ॥ 19 ॥}
\cquote{ಫಲದಾಸೆಯನ್ನೂ ಅಭಿಮಾನವನ್ನೂ ತೊರೆದು ಕರ್ಮ ಮಾಡುವವನು ತಿಳುವಿನ ಬೆಂಕಿಯಿಂದ ಕರ್ಮಗಳನ್ನು ಸುಟ್ಟುಕೊಂಡವನು. ತಿಳಿದವರು ಅವನನ್ನು ಆತ್ಮಜ್ಞಾನಿ ಎನ್ನುವರು.\\}
\slcol{\Index{ತ್ಯಕ್ತ್ವಾ ಕರ್ಮಫಲಾಸಂಗಂ} ನಿತ್ಯತೃಪ್ತೋ ನಿರಾಶ್ರಯಃ ।\\
ಕರ್ಮಣ್ಯಭಿಪ್ರವೃತ್ತೋऽಪಿ ನೈವ ಕಿಂಚಿತ್ಕರೋತಿ ಸಃ ॥ 20 ॥}
\cquote{ಫಲದ ಆಸೆಯನ್ನು ತೊರೆದು ತೃಪ್ತನಾಗಿ ಯಾವ ಹಂಗೂ ಇಲ್ಲದೆ ಕರ್ಮಗಳನ್ನು ಮಾಡಿದರೂ ಅವನು ಕರ್ಮದ ಬಂದದಿಂದ ದೂರವಾಗಿರುತ್ತಾನೆ.\\}
\slcol{\Index{ನಿರಾಶೀರ್ಯತಚಿತ್ತಾತ್ಮಾ} ತ್ಯಕ್ತಸರ್ವಪರಿಗ್ರಹಃ ।\\
ಶಾರೀರಂ ಕೇವಲಂ ಕರ್ಮ ಕುರ್ವನ್ನಾಪ್ನೋತಿ ಕಿಲ್ಬಿಷಮ್ ॥ 21 ॥}
\cquote{ಬಯಕೆ ಇಲ್ಲದೆ ಮನಸ್ಸನ್ನು ದೇಹವನ್ನು ಹಿಡಿತದಲ್ಲಿಟ್ಟುಕೊಂಡು ಎಲ್ಲವನ್ನು ತೊರೆದು ಬರಿ ಬದುಕಿರುವಷ್ಟು ಮಟ್ಟಿಗೆ ಕರ್ಮವನ್ನು ಮಾಡುವವನಿಗೆ ಪಾಪವಿಲ್ಲ.\\}
\slcol{\Index{ಯದೃಚ್ಛಾಲಾಭಸಂತುಷ್ಟೋ} ದ್ವಂದ್ವಾತೀತೋ ವಿಮತ್ಸರಃ ।\\
ಸಮಃ ಸಿದ್ಧಾವಸಿದ್ಧೌ ಚ ಕೃತ್ವಾಪಿ ನ ನಿಬಧ್ಯತೇ ॥ 22 ॥}
\cquote{ತಾನಾಗಿ ದೊರಕದಲ್ಲಿ ತೃಪ್ತನಾಗಿ ಸುಖ ದುಃಖ ಮೊದಲಾದ ದ್ವಂದ್ವಗಳನ್ನು ದಾಟಿದವನಾಗಿ ಮತ್ತೊಬ್ಬರ ಏಳಿಗೆಗೆ ಅಸೂಯೆ ಪಡದೆ ಕಾರ್ಯ ಫಲಿಸಿದರೂ ಫಲಿಸದಿದ್ದರೂ  ಒಂದೇ ರೀತಿಯಲ್ಲಿರುವವನು ಕರ್ಮ ಮಾಡಿದರೂ ಅದರ ಕಟ್ಟಿಗೊಳಪಡುವುದಿಲ್ಲ.\\}
\slcol{\Index{ಗತಸಂಗಸ್ಯ ಮುಕ್ತಸ್ಯ} ಙ್ಞಾನಾವಸ್ಥಿತಚೇತಸಃ ।\\
ಯಙ್ಞಾಯಾಚರತಃ ಕರ್ಮ ಸಮಗ್ರಂ ಪ್ರವಿಲೀಯತೇ ॥ 23 ॥}
\cquote{ಅಭಿಮಾನವನ್ನು ತೊರೆದು ಕರ್ಮಗಳ ಕಟ್ಟನ್ನು ಕಳಚಿಕೊಂಡು ಜ್ಞಾನದಲ್ಲಿ ಮನಸ್ಸನ್ನು ನೆಲೆಗೊಳಿಸಿ, ಈಶ್ವರನಿಗೆ ಪ್ರೀತಿಯಾಗಲೆಂದು ಮಾಡುವ ಕರ್ಮವೆಲ್ಲವೂ ಲಯವಾಗುತ್ತವೆ. \\}
\slcol{\Index{ಬ್ರಹ್ಮಾರ್ಪಣಂ ಬ್ರಹ್ಮ} ಹವಿರ್ಬ್ರಹ್ಮಾಗ್ನೌ ಬ್ರಹ್ಮಣಾ ಹುತಮ್ ।\\
ಬ್ರಹ್ಮೈವ ತೇನ ಗಂತವ್ಯಂ ಬ್ರಹ್ಮಕರ್ಮಸಮಾಧಿನಾ ॥ 24 ॥}
\cquote{ಹೋಮವು ಬ್ರಹ್ಮಮಯ, ಹವಿಸ್ಸೂ ಬ್ರಹ್ಮಮಯ, ಅಗ್ನಿಯು ಬ್ರಹ್ಮಮಯ, ಹೋಮಿಸುವವನೂ ಬ್ರಹ್ಮಮಯ. ಹೀಗೆ ತಾನು ಮಾಡುವ ಕ್ರಿಯೆ ಎಲ್ಲಾ ಸರ್ವಗತವಾದ ಬ್ರಹ್ಮತತ್ವವನ್ನೇ ಕಾಣುವವನು ಬ್ರಹ್ಮನನ್ನೇ ಹೋಗಿ ಸೇರುತ್ತಾನೆ.\\}
\slcol{\Index{ದೈವಮೇವಾಪರೇ ಯಙ್ಞಂ} ಯೋಗಿನಃ ಪರ್ಯುಪಾಸತೇ ।\\
ಬ್ರಹ್ಮಾಗ್ನಾವಪರೇ ಯಙ್ಞಂ ಯಙ್ಞೇನೈವೋಪಜುಹ್ವತಿ ॥ 25 ॥}
\cquote{ಕೆಲವು ಯೋಗಿಗಳು ಭಗವಂತನ ಉಪಾಸನೆಯನ್ನೇ ಯಜ್ಞವೆಂದು ಆಚರಿಸುತ್ತಾರೆ. ಇನ್ನು ಕೆಲವರು ಬ್ರಹ್ಮವೆಂಬ ಅಗ್ನಿಯಲ್ಲಿ ಭಗವತ್ ಪೂಜಾರೂಪವಾಗಿ ಯಜ್ಞ ಯಾಗಾದಿಗಳಿಂದ ಹೋಮಿಸುವರು.\\}

\newpage
\begin{mananam}{\mananamfont ಮನನ ಶ್ಲೋಕ - \textenglish{19}}
\footnotesize \mananamtext ದಿನನಿತ್ಯದ ಜೀವನದಲ್ಲಿ ನನ್ನ ಬುದ್ಧಿವಂತಿಕೆಗೂ ಮತ್ತು ಕೆಲಸಗಳಿಗೂ ಏನಾದರೂ ಸಂಬಂಧವಿದೆಯೇ? ನನ್ನ ಹಿಂದಿನ ಜೀವನವನ್ನು ಗಮನಿಸಿದಾಗ ಎಲ್ಲಾ ಕೆಲಸಗಳು ಬುದ್ಧಿವಂತಿಕೆಯಿಂದ ಮಾಡಿದ್ದೀನಾ? ನನ್ನ ಎಲ್ಲ ಕೆಲಸಗಳಿಗೂ ಮಾರ್ಗದರ್ಶನ ನೀಡಲು ಉನ್ನತ ಮಟ್ಟದ ಬುದ್ದಿವಂತಿಕೆಯಿಂದ ಹೇಗೆ ಕಾರ್ಯನಿರ್ವಹಿಸಬಹುದು? ವೈಯಕ್ತಿಕ ಆಸೆಗಳಿಂದ ಮುಕ್ತರಾಗಿರುವುದು ನನ್ನ ಎಲ್ಲ ಕೆಲಸಗಳ ಆಧ್ಯಾತ್ಮಿಕ ಮತ್ತು ಮಾನಸಿಕ ಗುಣಮಟ್ಟವನ್ನು ಸುಧಾರಿಸುತ್ತದೆಯೇ?
\end{mananam}
\WritingHand\enspace\textbf{ಆತ್ಮ ವಿಮರ್ಶೆ}\\
\begin{inspiration}{\mananamfont ಸ್ಪೂರ್ತಿ}
\footnotesize \mananamtext ನಾವು ನಮ್ಮ ಬಯಕೆಗಳ ಬಗ್ಗೆ ಗೀಳನ್ನು ಹೊಂದಿರುವಾಗ ಒತ್ತಡಕ್ಕೆ ಒಳಗಾಗುತ್ತೇವೆ ಮತ್ತು ನಮ್ಮ ದಾರಿಯನ್ನು ನಿರ್ಬಂಧಿಸುತ್ತೇವೆ. ನಮ್ಮ ಜೀವನಕ್ಕೆ ತೊಂದರೆಯಾಗದಂತಹ ಮತ್ತು ಅರ್ಥಪೂರ್ಣವಾದದ್ದನ್ನು ಬಯಸಿದಾಗ, ಅದು ಫಲಿಸಲು ಮಾಡುವ ಉತ್ತಮ ಮಾರ್ಗವೆಂದರೆ, ಹಾರೈಸುವುದು ಮತ್ತು ಪ್ರಾರ್ಥನೆಯನ್ನು ಮಾಡಿ ಮಾನಸಿಕ ವಿಶ್ರಾಂತಿ ಪಡೆಯುವುದು. ನಮಗೆ ಯಾವುದು ಸೂಕ್ತವಾಗಿರುವುದೋ ಅದು ಫಲಿಸುವುದು ಎಂದು ದೇವರನ್ನು ಮತ್ತು ಬ್ರಹ್ಮಾಂಡದ ಶಕ್ತಿಯನ್ನು ನಂಬುವುದು.
\end{inspiration}
\newpage

\begin{mananam}{\mananamfont ಮನನ ಶ್ಲೋಕ - \textenglish{22}}
\footnotesize \mananamtext ಯಶಸ್ಸು ಮತ್ತು ವೈಫಲ್ಯಕ್ಕೆ ನಾನು ಹೇಗೆ ಪ್ರತಿಕ್ರಿಸುತ್ತೇನೆ? ಯಶಸ್ಸು ನನ್ನನ್ನು ಉಲ್ಲಸಿತನಾಗಿ ಮಾಡುತ್ತದೆಯೇ? ಸೋಲು ನನ್ನನ್ನು ನಿರಾಶೆಗೊಳಿಸುತ್ತದೆಯೇ? ಬೇರೆಯವರಿಗೆ ಏನು ದೊರಕಿದೆ ಎಂದು ಮಾನಸಿಕವಾಗಿ ಹಾತೊರೆಯುವುದಕ್ಕಿಂತ ಹೆಚ್ಚಾಗಿ ನನಗೆ ಸ್ವಾಭಾವಿಕವಾಗಿ ಜೀವನದಿಂದ ಏನು ದೊರೆಯುತ್ತದೆಯೋ ಅದರಲ್ಲಿ ತೃಪ್ತಿ ಹೊಂದಿದ್ದೇನೆಯೇ? ಇತರರ ಯಶಸ್ವಿನ ಆಧಾರದ ಮೇಲೆ ನಾನು ನನ್ನನ್ನು ಮಾಪನ ಮಾಡಿಕೊಳ್ಳುತ್ತೇನೆಯೇ? ಜೀವನದಲ್ಲಿ ತೃಪ್ತರಾಗಿರಲು ಮತ್ತು ತನ್ನ ಕರ್ತವ್ಯವನ್ನು ಉತ್ಸಾಹದಿಂದ ಮಾಡಲು ಸಾಧ್ಯವೇ?
\end{mananam}
\WritingHand\enspace\textbf{ಆತ್ಮ ವಿಮರ್ಶೆ}\\
\begin{inspiration}{\mananamfont ಸ್ಪೂರ್ತಿ}
\footnotesize \mananamtext ಅಧ್ಯಾತ್ಮಿಕ ಅರಿವುಳ್ಳವನು ಈ ಲೌಕಿಕ ಬದುಕಿನ ಯಶಸ್ಸು ಮತ್ತು ಸಾಧನೆಗಳ ಮೇಲೆ ತನ್ನನ್ನು ತಾನು ಅಳೆದುಕೊಳ್ಳುವುದಿಲ್ಲ. ಒಬ್ಬನು ತನ್ನ ಜೀವನದಲ್ಲಿ ದೊರಕಿರುವುದರಲ್ಲಿ ತೃಪ್ತಿ ಹೊಂದುವುದು ಮತ್ತು ತನ್ನ ಎಲ್ಲಾ ಕರ್ತವ್ಯಗಳನ್ನು ಇತರರೊಂದಿಗೆ ಹೋಲಿಸಿಕೊಳ್ಳದೆ ಪೂರೈಸುವುದೇ ನಿಜವಾದ ಸ್ವಾತಂತ್ರ್ಯ.  ಅಧ್ಯಾತ್ಮಿಕ  ಅರಿವು ಉಂಟಾದಾಗ ಅದು ನಮ್ಮಲ್ಲಿ ಆಂತರಿಕ ಸ್ವಾತಂತ್ರ್ಯದ ಸ್ಥಿತಿಯನ್ನು ಉಂಟುಮಾಡುತ್ತದೆ. 
\end{inspiration}
\newpage

\begin{mananam}{\mananamfont ಮನನ ಶ್ಲೋಕ - \textenglish{24}}
\footnotesize \mananamtext ಈ ನನ್ನ ಜೀವನವನ್ನು ದೇವರಿಗೆ ಹೇಗೆ ತಾನೇ ಅರ್ಪಿಸಬಹುದು? ನನ್ನ ಎಲ್ಲಾ ಆಲೋಚನೆ, ಮಾತು ಮತ್ತು ಕೆಲಸವನ್ನು ಸಮರ್ಪಣಭಾವದಿಂದ ಮಾಡಬಹುದೇ? ಪ್ರತಿನಿತ್ಯ ನೆನಪಿಸಿಕೊಂಡು ಸ್ವಲ್ಪ ಅಹಂಕಾರವನ್ನು ತ್ಯಜಿಸಿ ದಿನನಿತ್ಯದ ಎಲ್ಲಾ ಕೆಲಸ ಮಾಡಬಹುದೇ? ನನ್ನ ದೈನಂದಿನ ಜೀವನದಲ್ಲಿ ಅಂತಹ ಆಂತರಿಕ ಅಭ್ಯಾಸವನ್ನು ಮಾಡುವ ವಿಧಾನಗಳು ಯಾವುವು?
\end{mananam}
\WritingHand\enspace\textbf{ಆತ್ಮ ವಿಮರ್ಶೆ}\\
\begin{inspiration}{\mananamfont ಸ್ಪೂರ್ತಿ}
\footnotesize \mananamtext ತನ್ನ ಎಲ್ಲ ಕೆಲಸಗಳನ್ನು ಪರಮಾತ್ಮನಿಗೆ ಅರ್ಪಿಸುವುದು ಮತ್ತು ಎಲ್ಲವೂ ಆ ಉನ್ನತ ಪ್ರಜ್ಞೆಯ ಭಾಗವಾಗಿದೆ ಎಂದು ನೋಡುವುದು ಒಂದು ಉನ್ನತ ಮನಸ್ಥಿತಿಯಾಗಿದೆ. ಇದು ಜೀವನದ ಎಲ್ಲಾ ಒತ್ತಡಗಳಿಂದ ಮುಕ್ತಗೊಳಿಸುತ್ತದೆ ಮತ್ತು ಮಾನಸಿಕ ಶಾಂತಿಯನ್ನು ತಕ್ಷಣವೇ ನೀಡುತ್ತದೆ.
\end{inspiration}
\newpage

\slcol{\Index{ಶ್ರೋತ್ರಾದೀನೀಂದ್ರಿಯಾಣ್ಯನ್ಯೇ} ಸಂಯಮಾಗ್ನಿಷು ಜುಹ್ವತಿ ।\\
ಶಬ್ದಾದೀನ್ವಿಷಯಾನನ್ಯ ಇಂದ್ರಿಯಾಗ್ನಿಷು ಜುಹ್ವತಿ ॥ 26 ॥}
\cquote{ಅನ್ಯ ಯೋಗಿ ಜನರು ಶ್ರಾವಣಿಂದ್ರಿಯಾದಿ ಸಮಸ್ತ ಇಂದ್ರಿಯಗಳನ್ನು ಸಂಯಮ ರೂಪೀ ಅಗ್ನಿಗಳಲ್ಲಿ ಹೋಮ ಮಾಡುತ್ತಾರೆ ಮತ್ತು ಬೇರೆ ಯೋಗಿ ಜನರು ಶಬ್ದಾದಿ ಸಮಸ್ತ ವಿಷಯಗಳನ್ನು ಇಂದ್ರಿಯರೂಪೀ ಅಗ್ನಿಗಳಲ್ಲಿ ಹವನ ಮಾಡುತ್ತಾರೆ.\\}
\slcol{\Index{ಸರ್ವಾಣೀಂದ್ರಿಯಕರ್ಮಾಣಿ} ಪ್ರಾಣಕರ್ಮಾಣಿ ಚಾಪರೇ ।\\
ಆತ್ಮಸಂಯಮಯೋಗಾಗ್ನೌ ಜುಹ್ವತಿ ಙ್ಞಾನದೀಪಿತೇ ॥ 27 ॥}
\cquote{ಇನ್ನು ಕೆಲವರು ಎಲ್ಲ ಇಂದ್ರಿಯಗಳ ವಿಷಯಗಳನ್ನೂ ಪ್ರಾಣಗಳ ವಿಷಯಗಳನ್ನೂ ಆತ್ಮ ಸಂಯಮವೆಂಬ ಬೆಂಕಿಯಲ್ಲಿ ಹೋಮ ಮಾಡುತ್ತಾರೆ.\\}
\cquote{ಇನ್ನು ಕೆಲವರು ಎಲ್ಲ ಇಂದ್ರಿಯಗಳ ವಿಷಯಗಳನ್ನೂ ಪ್ರಾಣಗಳ ವಿಷಯಗಳನ್ನೂ ಆತ್ಮ ಸಂಯಮವೆಂಬ ಬೆಂಕಿಯಲ್ಲಿ ಹೋಮ ಮಾಡುತ್ತಾರೆ.\\}
\slcol{\Index{ದ್ರವ್ಯಯಙ್ಞಾಸ್ತಪೋಯಙ್ಞಾ} ಯೋಗಯಙ್ಞಾಸ್ತಥಾಪರೇ ।\\
ಸ್ವಾಧ್ಯಾಯಙ್ಞಾನಯಙ್ಞಾಶ್ಚ ಯತಯಃ ಸಂಶಿತವ್ರತಾಃ ॥ 28 ॥}
\cquote{ಕಠಿನ ವ್ರತನಿಷ್ಠರಾದ ಕೆಲವು ಸಾಧಕರು ಭಗವತ್ ಪೂಜಾ ರೂಪವಾಗಿ ಹಣವನ್ನು ವೆಚ್ಚ ಮಾಡುವವರು, ಉಪವಾಸ ಮೊದಲಾದದ್ದನ್ನು ನಡೆಸುವರು, ಹಾಗೆಯೇ ಇನ್ನು ಕೆಲವರು ಯೋಗಭ್ಯಾಸ ನಡೆಸುವರು, ವೇದಾಭ್ಯಾಸಗಳನ್ನು ಮಾಡುವವರು ಮತ್ತು ಶಾಸ್ತ್ರಾರ್ಥವನ್ನು ತಿಳಿದುಕೊಳ್ಳುವರು.\\}
\slcol{\Index{ಅಪಾನೇ ಜುಹ್ವತಿ ಪ್ರಾಣಂ} ಪ್ರಾಣೇऽಪಾನಂ ತಥಾಪರೇ ।\\
ಪ್ರಾಣಾಪಾನಗತೀ ರುದ್ಧ್ವಾ ಪ್ರಾಣಾಯಾಮಪರಾಯಣಾಃ ॥ 29 ॥}
\cquote{ಪ್ರಾಣಯಾಮದಲ್ಲಿ ಪರಿಣತರಾದ ಇನ್ನೂ ಕೆಲವರು ಪ್ರಾಣದಲ್ಲಿ ಅಪಾನವನ್ನು ಅಪಾನದಲ್ಲಿ ಪ್ರಾಣವನ್ನು ಹೋಮಿಸಿ ಪ್ರಾಣಪಾನಗಳ ಚಲನವನ್ನು ನಿಯಂತ್ರಿಸಿ ಕುಂಭಕದಲ್ಲಿ ನೆಲೆಗೊಳಿಸುತ್ತಾರೆ.\\}
\slcol{\Index{ಅಪರೇ ನಿಯತಾಹಾರಾಃ} ಪ್ರಾಣಾನ್ಪ್ರಾಣೇಷು ಜುಹ್ವತಿ ।\\
ಸರ್ವೇऽಪ್ಯೇತೇ ಯಙ್ಞವಿದೋ ಯಙ್ಞಕ್ಷಪಿತಕಲ್ಮಷಾಃ ॥ 30 ।}
\cquote{ಮತ್ತೆ ಕೆಲವರು ಮಿತಹಾರದ ಮೂಲಕ ಪ್ರಾಣವನ್ನು ಶೋಷಿಸಿ ಪ್ರಾಣಗಳಲ್ಲಿ ಪ್ರಾಣಗಳನ್ನು ಹೋಮಿಸುತ್ತಾರೆ. ಇವರೆಲ್ಲರೂ ಯಜ್ಞಗಳನ್ನು ತಿಳಿದವರು ಮತ್ತು ಯಜ್ಞದಿಂದ ತಮ್ಮ ಪಾಪವನ್ನು ಕಳೆದುಕೊಂಡವರು. \\}
\slcol{\Index{ಯಙ್ಞಶಿಷ್ಟಾಮೃತಭುಜೋ} ಯಾಂತಿ ಬ್ರಹ್ಮ ಸನಾತನಮ್ ।\\
ನಾಯಂ ಲೋಕೋऽಸ್ತ್ಯಯಙ್ಞಸ್ಯ ಕುತೋऽನ್ಯಃ ಕುರುಸತ್ತಮ ॥ 31 ॥}
\cquote{ಯಜ್ಞದಲ್ಲಿ  ಉಳಿದ ಅಮೃತವನ್ನು ಭೋಜನ ಮಾಡಿ ಇವರು ಅನಂತ ಬ್ರಹ್ಮನನ್ನು ಸೇರುವರು. ಎಲೈ  ಕುರುಶ್ರೇಷ್ಠನೆ, ಯಜ್ಞಮಾಡದಿರುವವರಿಗೆ ಈ ಲೋಕವೇ ಇಲ್ಲ, ಇನ್ನು ಪರಲೋಕವೆಲ್ಲಿಯದು?}
\slcol{\Index{ಏವಂ ಬಹುವಿಧಾ ಯಙ್ಞಾ} ವಿತತಾ ಬ್ರಹ್ಮಣೋ ಮುಖೇ ।\\
ಕರ್ಮಜಾನ್ವಿದ್ಧಿ ತಾನ್ಸರ್ವಾನೇವಂ ಙ್ಞಾತ್ವಾ ವಿಮೋಕ್ಷ್ಯಸೇ ॥ 32 ॥}
\cquote{ಹೀಗೆ ಬಹುವಿಧವಾದ ಯಜ್ಞಗಳು ವೇದದಲ್ಲಿ ಹೇಳಲ್ಪಟ್ಟಿವೆ. ಅವುಗಳೆಲ್ಲಾ ಕರ್ಮದಿಂದ ಜನಿಸಿದವುಗಳು ಎಂಬುದನ್ನು ತಿಳಿ. ಹೀಗೆ ನೀನು ತಿಳಿಯುವುದರಿಂದ ಬಿಡುಗಡೆಯನ್ನು ಪಡೆಯುತ್ತೀಯೆ.\\}
\slcol{\Index{ಶ್ರೇಯಾಂದ್ರವ್ಯಮಯಾದ್ಯ}ಙ್ಞಾಜ್ಙ್ಞಾನಯಙ್ಞಃ ಪರಂತಪ ।\\
ಸರ್ವಂ ಕರ್ಮಾಖಿಲಂ ಪಾರ್ಥ ಙ್ಞಾನೇ ಪರಿಸಮಾಪ್ಯತೇ ॥ 33 ॥}
\cquote{ಅರ್ಜುನ, ವಸ್ತುಗಳನ್ನು ಹೋಮಿಸಿ ಮಾಡುವ ಯಜ್ಞಕ್ಕಿಂತಲೂ  ಜ್ಞಾನರೂಪವಾದ ಯಜ್ಞವು ಹೆಚ್ಚಿನದು. ಅರ್ಜುನ, ಎಲ್ಲ ಕರ್ಮಗಳು ಜ್ಞಾನದ ಮೂಲಕವೇ ಪೂರ್ಣ ಫಲಪ್ರದವಾಗುತ್ತವೆ.\\}
\slcol{\Index{ತದ್ವಿದ್ಧಿ ಪ್ರಣಿಪಾತೇನ} ಪರಿಪ್ರಶ್ನೇನ ಸೇವಯಾ ।\\
ಉಪದೇಕ್ಷ್ಯಂತಿ ತೇ ಙ್ಞಾನಂ ಙ್ಞಾನಿನಸ್ತತ್ತ್ವದರ್ಶಿನಃ ॥ 34 ॥}
\cquote{ಜ್ಞಾನಿಗಳಿಗೂ ತತ್ವದರ್ಶಿಗಳಿಗೂ ಆದ ಮಹಾತ್ಮರನ್ನು ವಿನಮ್ರಭಾವದಿಂದ ಸೇವಿಸುವುದರಿಂದ, ಜ್ಞಾನ ಭಿಕ್ಷೆಯನ್ನು ಬೇಡು. ಅವರು ನಿನಗೆ ಜ್ಞಾನೋಪದೇಶವನ್ನು ಮಾಡುವರು.\\}
\slcol{\Index{ಯಜ್ಙ್ಞಾತ್ವಾ ನ ಪುನರ್ಮೋಹ}ಮೇವಂ ಯಾಸ್ಯಸಿ ಪಾಂಡವ ।\\
ಯೇನ ಭೂತಾನ್ಯಶೇಷೇಣ ದ್ರಕ್ಷ್ಯಸ್ಯಾತ್ಮನ್ಯಥೋ ಮಯಿ ॥ 35 ॥}
\cquote{ಎಲೈ ಪಾಂಡವನೇ, ಅದನ್ನು ಅರಿತ ಮೇಲೆ ಹೀಗೆ ನೀನು ಮೋಹಗೊಳ್ಳುವುದಿಲ್ಲ ಎಲ್ಲಾ ಜೀವಿಗಳನ್ನು ನೀನು ನಿನ್ನ ಆತ್ಮನಾದ ನನ್ನಲ್ಲೇ ಕಾಣುವೆ.\\}
\slcol{\Index{ಅಪಿ ಚೇದಸಿ ಪಾಪೇಭ್ಯಃ} ಸರ್ವೇಭ್ಯಃ ಪಾಪಕೃತ್ತಮಃ ।\\
ಸರ್ವಂ ಙ್ಞಾನಪ್ಲವೇನೈವ ವೃಜಿನಂ ಸಂತರಿಷ್ಯಸಿ ॥ 36 ॥}
\cquote{ನೀನು ಪಾಪಿಗಳಲ್ಲೆಲ್ಲ ಹೆಚ್ಚಿನ ಪಾಪಿಯಾಗಿದ್ದರೂ ಆ ಪಾಪಗಳನ್ನೆಲ್ಲ ಜ್ಞಾನರೂಪವಾದ ಹಡಗಿನಿಂದಲೇ ದಾಟುತ್ತಿ.\\}
\slcol{\Index{ಯಥೈಧಾಂಸಿ ಸಮಿದ್ಧೋऽಗ್ನಿ}ರ್ಭಸ್ಮಸಾತ್ಕುರುತೇऽರ್ಜುನ ।\\
ಙ್ಞಾನಾಗ್ನಿಃ ಸರ್ವಕರ್ಮಾಣಿ ಭಸ್ಮಸಾತ್ಕುರುತೇ ತಥಾ ॥ 37 ॥}
\cquote{ಅರ್ಜುನ,ಊರಿಗೊಳಿಸಿದ ಬೆಂಕಿಯು ಕಟ್ಟಿಗೆಗಳನ್ನು ಬೂದಿಮಾಡಿಬಿಡುವಂತೆ ಜ್ಞಾನರೂಪವಾದ ಅಗ್ನಿಯು ಕರ್ಮಗಳನ್ನೆಲ್ಲ ಬೂದಿಮಾಡಿಬಿಡುತ್ತದೆ.\\}
\slcol{\Index{ನ ಹಿ ಙ್ಞಾನೇನ ಸದೃಶಂ} ಪವಿತ್ರಮಿಹ ವಿದ್ಯತೇ ।\\
ತತ್ಸ್ವಯಂ ಯೋಗಸಂಸಿದ್ಧಃ ಕಾಲೇನಾತ್ಮನಿ ವಿಂದತಿ ॥ 38 ॥}
\cquote{ಆತ್ಮ ಜ್ಞಾನಕ್ಕೆ ಸಮಾನವಾದ ಪವಿತ್ರಕರ ವಸ್ತುವು ಈ ಜಗತ್ತಿನಲ್ಲಿ ಮತ್ತೊಂದಿಲ್ಲ. ಯೋಗಸಿದ್ಧಿಯನ್ನು ಪಡೆದವನು ಕಾಲಕ್ರಮದಲ್ಲಿ ಸ್ವಯಂ ಇದನ್ನು ಪಡೆದುಕೊಳ್ಳುತ್ತಾನೆ.\\}

\newpage
\begin{mananam}{\mananamfont ಮನನ ಶ್ಲೋಕ - \textenglish{32}}
\footnotesize \mananamtext ನನ್ನ ಸ್ವಂತ ಲಾಭಕ್ಕೋಸ್ಕರ ಲೌಕಿಕದಲ್ಲಿ  ಅಧಿಕಾರಿಗಳನ್ನು ಸಂತೋಷಪಡಿಸಲು  ವಿವಿಧ ಮಾರ್ಗಗಳು ಯಾವುವು? ಅದೇ ರೀತಿ ನಾನು ಜೀವನದಲ್ಲಿ ಭೌತಿಕ ಪ್ರಯೋಜನ ಪಡೆಯಲು ದೇವರು ಮತ್ತು ದೇವತೆಗಳನ್ನು ಪೂಜಿಸುತ್ತೇನೆಯೇ? ಪರಮಾತ್ಮನನ್ನು ಹುಡುಕುವುದು ಮತ್ತು ದೈವಿಕ ಸ್ಥಿತಿಯಲ್ಲಿರುವುದು, ಆಶಾಶ್ವತವಾದ ನೂರಾರು ಲೌಕಿಕ ಗುರಿಗಳನ್ನು ಪಡೆಯುವುದಕ್ಕಿಂತ ಶ್ರೇಷ್ಠವೆಂದು ಪುರಾಣಗಳು ಮತ್ತು ಋಷಿಗಳು ಹೇಳುತ್ತಾರೆ. ನನ್ನ ಜೀವನದಲ್ಲಿ ಇಂತಹ ಮೌಲ್ಯಗಳು ಮತ್ತು ದೃಷ್ಟಿಕೋನವನ್ನು ಅಳವಡಿಸಿಕೊಳ್ಳಲು ನಾನು ಏನು ಮಾಡಬೇಕು? ದೀರ್ಘಾವಧಿಯ ಆಂತರಿಕವಾಗಿ ಉನ್ನತಿಗೇರಿಸುವ ಮತ್ತು ಇತರರಿಗೆ ಪ್ರಯೋಜನಕಾರಿಯಾದ ಗುರಿಗಳನ್ನು ಹುಡುಕುವುದಕ್ಕೆ ನಾನು ಹೇಗೆ ಆದ್ಯತೆ ನೀಡಬಲ್ಲೆ?
\end{mananam}
\WritingHand\enspace\textbf{ಆತ್ಮ ವಿಮರ್ಶೆ}\\
\begin{inspiration}{\mananamfont ಸ್ಪೂರ್ತಿ}
\footnotesize \mananamtext ಯಾಗ,ಪೂಜೆ ಮತ್ತು ಆಚರಣೆಗಳ ನಿಜವಾದ ಮೌಲ್ಯವೆಂದರೆ ನಮ್ಮನ್ನು ಲೌಕಿಕ ಅನ್ವೇಷಣೆಗಳ ಅತಿಯಾದ ಗೀಳಿನಿಂದ ಮೇಲಕ್ಕೆ ಎತ್ತುವುದು ಉನ್ನತ ಅಧ್ಯಾತ್ಮಿಕ ಶಕ್ತಿಗಳಿಗೆ ಶರಣಾಗುವ ಮೂಲಕ ಮನಸ್ಸಿನ ಶುದ್ಧತೆಯನ್ನು ಪಡೆಯುತ್ತೇವೆ. ಇದರಿಂದ ನಾವು ಬುದ್ದಿವಂತಿಕೆಯಿಂದ ನಮ್ಮ ಆತ್ಮವನ್ನು ಅರಿತುಕೊಳ್ಳಲು ಕೆಲಸ ಮಾಡುತ್ತೇವೆ.\\
\end{inspiration}
\newpage

\begin{mananam}{\mananamfont ಮನನ ಶ್ಲೋಕ - \textenglish{34}}
\footnotesize \mananamtext ಜ್ಞಾನ ಮತ್ತು ಆಧ್ಯಾತ್ಮಿಕ ಸತ್ಯವನ್ನು ಹುಡುಕುವುದಕ್ಕಾಗಿ ನಾನು ಗುರುಗಳನ್ನು ಸಂಪರ್ಕಿಸುತ್ತೇನೆಯೇ? ಅವರನ್ನು ನಾನು ನಮ್ರತೆಯಿಂದ ಸಂಪರ್ಕಿಸುವುದಕ್ಕೆ ನನ್ನಲ್ಲಿ ಪ್ರತಿರೋಧವಿದೆಯೇ? ಪೂಜ್ಯ ಭಾವನೆಯಿಂದ ಪ್ರಶ್ನೆಗಳನ್ನು ಕೇಳಲು ನಾನು ಸಿದ್ಧನಾಗಿದ್ದೇನೆಯೇ? ಅವರಿಗೆ ಉಪಯೋಗವಾಗುವಂತವುನ್ನು ನಾನು ನೀಡುವಷ್ಟು ಉದಾರನಾಗಿದ್ದೀನಾ? ನನ್ನ ಆಧ್ಯಾತ್ಮಿಕ ಜೀವನದಲ್ಲಿ ಅವರಿಗೆ ಅಥವಾ ಮಾನವೀಯತೆಗೆ ಸೇವೆ ಸಲ್ಲಿಸಲು ನಾನು ಸಮಯ ಮಾಡಿಕೊಳ್ಳುವುದು ನನ್ನ ಆಧ್ಯಾತ್ಮಿಕ ಜೀವನದ ಮೊದಲ ಆದ್ಯತೆ ಆಗಬಹುದೇ?
\end{mananam}
\WritingHand\enspace\textbf{ಆತ್ಮ ವಿಮರ್ಶೆ}\\
\begin{inspiration}{\mananamfont ಸ್ಪೂರ್ತಿ}
\footnotesize \mananamtext ಅಧ್ಯಾತ್ಮಿಕ ಸತ್ಯದ ಹಾದಿಯಲ್ಲಿರುವ ಅನ್ವೇಷಕನು ಮೊದಲು ಗುರುಗಳ ಜೊತೆ ಸಂಪರ್ಕಿಸಲು ಮತ್ತು ಅವರ ಸೇವೆ ಮಾಡಲು ಕೆಲವು ಕೌಶಲ್ಯಗಳನ್ನು ಬೆಳೆಸಿಕೊಳ್ಳಬೇಕು. ಇವು ಕೇವಲ ಸಾಂಸ್ಕೃತಿಕ ಮತ್ತು ಸಾಮಾಜಿಕ ರೂಡಿಗಳಲ್ಲ, ಆದರೆ ಇಂತಹ ಮಾನಸಿಕ ವರ್ತನೆ ಜ್ಞಾನವನ್ನು ಪ್ರಸರಿಸಲು ಸಾಧ್ಯಮಾಡುತ್ತದೆ.
\end{inspiration}

\newpage
\begin{mananam}{\mananamfont ಮನನ ಶ್ಲೋಕ - \textenglish{36,37}}
\footnotesize \mananamtext ನಾನು ಹಿಂದೆ ಮಾಡಿದ ಕೆಲವು ಕರ್ಮಗಳಿಗೆ ಪ್ರಶ್ಚಾತಾಪ  ಮತ್ತು ಅಪರಾಧ ಭಾವನೆಗಳನ್ನು ಅಡಗಿಸಿಕೊಂಡಿದ್ದೀನಾ? ನಾನು ಆ ಭಾವನೆಗಳನ್ನು ಒಳಗೆ ಅದು ಮಿಡಲು ಪ್ರಯತ್ನಿಸುತ್ತೇನಾ? ಇವುಗಳನ್ನು ಪರಿಹರಿಸಲು ಆರೋಗ್ಯಕರವಾದ ವಿಧಾನಗಳಿವೆಯೇ? ನನ್ನ ಜೀವನದಲ್ಲಿ ಈ ತರಹ ಭಾವನೆಗಳನ್ನು ಆಧ್ಯಾತ್ಮಿಕ ಚಿಂತನೆಯಿಂದ ಹೇಗೆ ಪರಿಹರಿಸಬಹುದು. ಅಧ್ಯಾತ್ಮಿಕ ಚಿಂತನೆಗಳಿಂದ ಈ ಜೀವನದಲ್ಲಿ ಬರುವ ಹಳೆಯ ನಕರಾತ್ಮಕ ಚಿಂತನೆಗಳಿಂದ ಮಾಡುವ ಕೆಲಸಗಳಿಂದ ಹೊರಬಂದು ಹೊಸದಾದ ಸಕರಾತ್ಮಕ ಕ್ರಿಯೆಗಳನ್ನು ಅಳವಡಿಸಿಕೊಳ್ಳಲು ಸಾಧ್ಯವೇ?\\
\end{mananam}
\WritingHand\enspace\textbf{ಆತ್ಮ ವಿಮರ್ಶೆ}\\
\begin{inspiration}{\mananamfont ಸ್ಪೂರ್ತಿ}
\footnotesize \mananamtext ನಾವು ನಮ್ಮ ನಿಜವಾದ ಸ್ವರೂಪವನ್ನು ಅರಿತುಕೊಂಡಾಗ ಮತ್ತು ಕೀಳುಮಟ್ಟದ ಪಶುಗಳ ತರಹ ಪ್ರಚೋದನೆಗಳಿಗೆ ಮಣಿಯುವುದನ್ನು ನಿಲ್ಲಿಸಿದಾಗ, ನಮ್ಮ ಜೀವನದ ಒಟ್ಟಾರೆ ಮಾನಸಿಕ ಮತ್ತು ಅಧ್ಯಾತ್ಮಿಕ ಗುಣಮಟ್ಟವನ್ನು ಉನ್ನತಿಕೇರಿಸುತ್ತೇವೆ. ಇದರಿಂದ ನಮಗೆ ನಮ್ಮ ಭೂತಕಾಲವು ಕಾಡುವುದಿಲ್ಲ ಮತ್ತು ನಮ್ಮನ್ನು ಕೆಳಕ್ಕೆ ಎಳೆಯುವುದಿಲ್ಲ.
\end{inspiration}
\newpage

\begin{mananam}{\mananamfont ಮನನ ಶ್ಲೋಕ - \textenglish{38}}
\footnotesize \mananamtext ಪುಸ್ತಕಗಳು, ಗುರುಗಳು ಮತ್ತು ಇತರ ಮಾಧ್ಯಮದ ಮೂಲಕ ನಾನು ಹೇಗೆ ಜ್ಞಾನವನ್ನು ಹುಡುಕುತ್ತೇನೆ, ಹಾಗೆಯೇ ನನ್ನೊಳಗೆ ಜ್ಞಾನವನ್ನು ಹುಡುಕಬಹುದು ಎಂದು ನನಗೆ ಅರಿವಿದೆಯೇ? ಜ್ಞಾನವನ್ನು ಆಂತರಿಕವಾಗಿ ಹುಡುಕುವ ಮತ್ತು ಭಾಹ್ಯವಾಗಿ ಹುಡುಕುವ ಪ್ರಕ್ರಿಯೆಯ ಬಗ್ಗೆ ನನ್ನ ತಿಳುವಳಿಕೆ ಏನು? ಇದರ ಅರ್ಥ ನಾನು ಬರಿದೆ ನನ್ನ ದೇಹ ಮತ್ತು ಅದರ ವಿವಿಧ ಪ್ರಕ್ರಿಯೆಗಳ ಬಗ್ಗೆ ತಿಳಿದುಕೊಳ್ಳುವುದು ಇಂದು ಅರ್ಥವೆ ಅಥವಾ ಅದು ಮನಸ್ಸು ಮತ್ತು ವಿವಿಧ ಮಾನಸಿಕ ಸ್ಥಿತಿಗಳನ್ನು  ತಿಳಿದುಕೊಳ್ಳುವುದೇ?    ಅಥವಾ ಇದೆಲ್ಲಕ್ಕಿಂತ ಹೆಚ್ಚಿನದನ್ನು ನಾನು ಇನ್ನೂ ಕಂಡುಕೊಳ್ಳಬೇಕಾಗಿದೆಯೇ?\\
\end{mananam}
\WritingHand\enspace\textbf{ಆತ್ಮ ವಿಮರ್ಶೆ}\\
\begin{inspiration}{\mananamfont ಸ್ಪೂರ್ತಿ}
\footnotesize \mananamtext ಆತ್ಮ ಜ್ಞಾನವು ಶ್ರೇಷ್ಠವಾದ ಶುದ್ಧಿಕಾರಕವಾಗಿದೆ. ಎಲ್ಲಾ ಆಧ್ಯಾತ್ಮಿಕ ಅಭ್ಯಾಸಗಳು ನಮ್ಮನ್ನು ಶುದ್ಧೀಕರಿಸಲು ಇರುವ ಒಂದು ಸಾಧನೆ. ಆತ್ಮ ಜ್ಞಾನವು ಸಿದ್ಧಿಸುವ ತನಕ  ಜೀವನದಲ್ಲಿ ನಮ್ಮ ಕರ್ತವ್ಯ ಮತ್ತು ಅಭ್ಯಾಸಗಳನ್ನು ಬಿಡಬಾರದು.
\end{inspiration}
\newpage
 
\slcol{\Index{ಶ್ರದ್ಧಾವಾಂಲ್ಲಭತೇ ಙ್ಞಾನಂ} ತತ್ಪರಃ ಸಂಯತೇಂದ್ರಿಯಃ ।\\
ಙ್ಞಾನಂ ಲಬ್ಧ್ವಾ ಪರಾಂ ಶಾಂತಿಮಚಿರೇಣಾಧಿಗಚ್ಛತಿ ॥ 39 ॥}
\cquote{ಶ್ರದ್ಧಾವಂತನೂ, ಇಂದ್ರಿಯಗಳನ್ನು ನಿಗ್ರಹಿಸಿದವನೂ, ಈ ಜ್ಞಾನವನ್ನು ಪಡೆಯುತ್ತಾನೆ.ಜ್ಞಾನ ದೊರಕಿದ ಮೇಲೆ ಹೆಚ್ಚು ತಡವಿಲ್ಲದೆ ಪರಮಶಾಂತಿಯನ್ನು ಗಳಿಸುತ್ತಾನೆ.\\}
\slcol{\Index{ಅಙ್ಞಶ್ಚಾಶ್ರದ್ದಧಾನಶ್ಚ} ಸಂಶಯಾತ್ಮಾ ವಿನಶ್ಯತಿ ।\\
ನಾಯಂ ಲೋಕೋऽಸ್ತಿ ನ ಪರೋ ನ ಸುಖಂ ಸಂಶಯಾತ್ಮನಃ ॥ 40 ॥}
\cquote{ಈ ಮಾತಿನಲ್ಲಿ ಶ್ರದ್ದೆ ಇಲ್ಲದ ಮೂಡನು ಸಂಶಯಪಟ್ಟು, ತನ್ನ ಸರ್ವನಾಶವನ್ನು ಮಾಡಿಕೊಳ್ಳುತ್ತಾನೆ. ಸಂಶಯ್ಯಾತ್ಮನಿಗೆ ಈ ಲೋಕವು ಇಲ್ಲ, ಪರಲೋಕವು ಇಲ್ಲ, ಸುಖವು ಇಲ್ಲ.\\}
\slcol{\Index{ಯೋಗಸಂನ್ಯಸ್ತಕರ್ಮಾಣಂ} ಙ್ಞಾನಸಂಛಿನ್ನಸಂಶಯಮ್ ।\\
ಆತ್ಮವಂತಂ ನ ಕರ್ಮಾಣಿ ನಿಬಧ್ನಂತಿ ಧನಂಜಯ ॥ 41 ॥}
\cquote{ಅರ್ಜುನ, ಭಗವದರ್ಪಣ ಬುದ್ಧಿಯಿಂದ ಕರ್ಮಗಳನ್ನು ಮಾಡುತ್ತಾ, ತತ್ವ ಜ್ಞಾನದ ಮೂಲಕ ಸಂಶಯವನ್ನು ಕಳೆದುಕೊಂಡ ವಿವೇಕಿಯನ್ನು ಕರ್ಮಗಳು ಕಟ್ಟಿ ಹಾಕುವುದಿಲ್ಲ.\\}
\slcol{\Index{ತಸ್ಮಾದಙ್ಞಾನಸಂಭೂತಂ} ಹೃತ್ಸ್ಥಂ ಙ್ಞಾನಾಸಿನಾತ್ಮನಃ ।\\
ಛಿತ್ತ್ವೈನಂ ಸಂಶಯಂ ಯೋಗಮಾತಿಷ್ಠೋತ್ತಿಷ್ಠ ಭಾರತ ॥ 42 ॥}
\cquote{ಆದ್ದರಿಂದ ಅರ್ಜುನ, ಅಜ್ಞಾನದಿಂದ ಮನಸ್ಸನ್ನು  ಕಾಡುತ್ತಿರುವ ಈ ಸಂಶಯವನ್ನು ಜ್ಞಾನ ರೂಪವಾದ ಕತ್ತೆಯಿಂದ ಕತ್ತರಿಸಿ ಪರದಾಸೆಯನ್ನು ಬಿಟ್ಟು ಕರ್ಮವನ್ನು ಮಾಡು, ಎಳು.\\}

\begin{center}
ಓಂ ತತ್ಸದಿತಿ ಶ್ರೀಮದ್ಭಗವದ್ಗೀತಾಸೂಪನಿಷತ್ಸು\\ಯೋಗಶಾಸ್ತ್ರೇ ಶ್ರೀಕೃಷ್ಣಾರ್ಜುನಸಂವಾದೇ\\
ಙ್ಞಾನಕರ್ಮಸಂನ್ಯಾಸಯೋಗೋ ನಾಮ ಚತುರ್ಥೋऽಧ್ಯಾಯಃ ॥4 ॥\\
\end{center}

\newpage
\begin{mananam}{\mananamfont ಮನನ ಶ್ಲೋಕ - \textenglish{39,40}}
\footnotesize \mananamtext ಜ್ಞಾನ ಮತ್ತು ನಂಬಿಕೆಯ ಮಧ್ಯೆ ಇರುವ ಸಂಬಂಧವೇನು? ಅನುಮಾನ ಮತ್ತು ಆತ್ಮವಿಶ್ವಾಸದ ಕೊರತೆ ಹೇಗೆ ವಿನಾಶಕಾರಿಯಾಗಿದೆ? ಪ್ರಶ್ನೆಗಳು ಮತ್ತು ಸಂದೇಹಗಳನ್ನು  ಅವು ವಿನಾಶಕಾರಿಯಾಗುವ ಮೊದಲೇ ಹೇಗೆ ಪರಿಹರಿಸಿಕೊಳ್ಳಬಹುದು. ಜ್ಞಾನವು ನನ್ನೊಳಗೆ ಪರಮ ಶಾಂತಿಯನ್ನು ಹೇಗೆ ತರಬಲ್ಲದು.ಇಂದ್ರಿಯಗಳ ನಿಗ್ರಹಕ್ಕಿರುವ ಜ್ಞಾನದ ಸ್ವರೂಪವೇನು? ಧ್ಯಾನವು ನಮ್ಮನ್ನು  ಇಂದ್ರಿಯಗಳ ನಿಗ್ರಹ ಮತ್ತು ಶಾಂತಿಯೆತ್ತ ಕೊಂಡಯ್ಯ ಬಲ್ಲದೆ?\\
\end{mananam}
\WritingHand\enspace\textbf{ಆತ್ಮ ವಿಮರ್ಶೆ}\\
\begin{inspiration}{\mananamfont ಸ್ಪೂರ್ತಿ}
\footnotesize \mananamtext ಮಾನವ ಕುಲದ ದೊಡ್ಡ ರೋಗವೆಂದರೆ ನಮ್ಮ ಆತ್ಮವನ್ನೇ ನಾವು ಅನುಮಾನಿಸುವುದು. ಇಂಥವರಿಗೆ ದೇವರು ಕೂಡ ಸಹಾಯ ಮಾಡಲಾರರು. ಅನುಮಾನಗಳು ನಮ್ಮನ್ನು ವಿಚಾರಣೆಗೆ ಒಳಪಡಿಸುವಂತಿರಬೇಕು.ಅದು ನಮ್ಮನ್ನು ನಕರಾತ್ಮಕವಾಗಿಮಾಡಿದರೆ ಅದು ಹಾನಿಕಾರಕ.\\
 ಆತ್ಮ ಜ್ಞಾನದ ಕೊರತೆ ಇರುವವರಿಗೆ,ಒಬ್ಬನು ಭ್ರಮೆಯಲ್ಲಿ ಉಳಿಯುತ್ತಾನೆ ಮತ್ತು ಈ ಭ್ರಮೆಯು ಹೋರಾಟ ಮತ್ತು ಹತಾಶೆಯ ಜೀವಕ್ಕೆ ಕಾರಣವಾಗುತ್ತದೆ. 
\end{inspiration}
\newpage


