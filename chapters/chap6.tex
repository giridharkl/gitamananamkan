\slcol{ಶ್ರೀಭಗವಾನುವಾಚ ।\\
\Index{ಅನಾಶ್ರಿತಃ ಕರ್ಮಫಲಂ} ಕಾರ್ಯಂ ಕರ್ಮ ಕರೋತಿ ಯಃ ।\\
ಸ ಸಂನ್ಯಾಸೀ ಚ ಯೋಗೀ ಚ ನ ನಿರಗ್ನಿರ್ನ ಚಾಕ್ರಿಯಃ ॥ ೧ ॥}
\cquote{ಶ್ರೀ ಭಗವಂತನು ಹೇಳಿದನು,\\
ಕರ್ಮದ ಫಲಕ್ಕಾಗಿ ಹಂಬಲಿಸದೆ ಕರ್ತವ್ಯವೆಂದು ಕರ್ಮವನ್ನು ಮಾಡುವವನೇ ನಿಜವಾದ ಸಂನ್ಯಾಸಿ ಮತ್ತು ಕರ್ಮಯೋಗಿ ಹೊರತು ಸುಮ್ಮನೆ ಅಗ್ನಿಹೋತ್ರ ಮೊದಲಾದ ಕರ್ಮಗಳನ್ನು ಬಿಟ್ಟವನೂ ಯಾವ ಸತ್ಕರ್ಮಗಳನ್ನೂ ಮಾಡದವನೂ ಅಲ್ಲ.}
\slcol{\Index{ಯಂ ಸಂನ್ಯಾಸಮಿತಿ} ಪ್ರಾಹುರ್ಯೋಗಂ ತಂ ವಿದ್ಧಿ ಪಾಂಡವ ।\\
ನ ಹ್ಯಸಂನ್ಯಸ್ತಸಂಕಲ್ಪೋ ಯೋಗೀ ಭವತಿ ಕಶ್ಚನ ॥ ೨ ॥}
\cquote{ಅರ್ಜುನ, ಯಾವ ಸ್ಥಿತಿಯನ್ನು ಕರ್ಮಸಂನ್ಯಾಸವೆನ್ನುವರೋ ಅದನ್ನೇ ಕರ್ಮಯೋಗವೆಂದೂ ತಿಳಿ. ಏಕೆಂದರೆ ಫಲದ ಬಯಕೆಯ ಸಂನ್ಯಾಸವಿಲ್ಲದೆ ಕರ್ಮ ಮಾಡುವವನು ಕರ್ಮ ಯೋಗಿಯೂ ಆಗಲಾರ.}

\newpage
\begin{mananam}{\mananamfont{ಮನನ ಶ್ಲೋಕ - ೧, ೨}}
\small \mananamtext ಈ ಜೀವನ ಪಯಣದಲ್ಲಿ ನನಗೆ, ಏನನ್ನಾದರೂ ತ್ಯಾಗ ಮಾಡುವ ಅಥವಾ, ಯಾವುದೇ ಹೊಣೆಗಾರಿಕೆಯಿಂದ ಮುಕ್ತನಾಗುವ ಬಯಕೆ ಏಕೆ ಬರುತ್ತದೆ? ಇದರ ಹಿಂದಿರುವ ಪ್ರೇರಣೆಯಾದರೂ ಯಾವುದು? ಚಿಂತಾಮುಕ್ತವಾದ ಜೀವನ ಶೈಲಿಯ ಬಯಕೆಯೇ? ಅಥವಾ, ಭವಿಷ್ಯದಲ್ಲಿ ದೊರೆಯಬಹುದಾದಂತಹ, ಭೌತಿಕ ಯಾ ಆಧ್ಯಾತ್ಮಿಕ ಪ್ರತಿಫಲಾಪೇಕ್ಷಯೇ? ಯಾವುದೇ ಕಾರ್ಯವನ್ನೂ, ‘ಇದು ನನ್ನ ಶುದ್ಧ ಕರ್ತವ್ಯ ಮಾತ್ರ’ ಎಂದು ತಿಳಿದು, ಅನಿರ್ಬಂಧಿತವಾಗಿ ಮತ್ತು ಯಾವುದೇ ಪ್ರತಿಫಲಾಪೇಕ್ಷೆ ಇಲ್ಲದೆಯೇ ಸಹಾಯವನ್ನು (ಜೀವಕೋಟಿಗೆ) ಮಾಡುವ ಇಚ್ಛೆ ನನಗೆ ಇದೆಯೇ? ಇದಕ್ಕಾಗಿ ಸ್ವತಃ ಪ್ರೇರಿತನಾಗಬಲ್ಲೆನೇ? ‘ನಾನು ನಿಜವಾಗಿಯೂ ಅರ್ಹನಾದಾಗ ಒಳ್ಳೆಯ ಪ್ರತಿಫಲ ಖಂಡಿತ ದೊರಕುತ್ತದೆ’ ಎಂಬ ಒಂದು ಉನ್ನತ ನಂಬಿಕೆಯನ್ನು (ದೈವೀಕ ಶಕ್ತಿಯಲ್ಲಿ)  ಬೆಳೆಯಿಸಿಕೊಳ್ಳಬಲ್ಲೆನೇ?
\end{mananam}
\WritingHand\enspace\textbf{ಆತ್ಮ ವಿಮರ್ಶೆ}\\
\begin{inspiration}{\mananamfont ಸ್ಫೂರ್ತಿ}
ನಮ್ಮ ಪ್ರಾಚೀನ ಪರಿಕಲ್ಪನೆಯಲ್ಲಿ ಸoನ್ಯಾಸಿ ಎಂದರೆ ಎಲ್ಲಾ ವೈದಿಕ ಆಚರಣೆಗಳನ್ನು ತ್ಯಜಿಸಿದವನು. ಅವರು ಈ ಆಚರಣೆಗಳನ್ನು ಏಕೆ ತ್ಯಜಿಸುತ್ತಾರೆ? ಏಕೆಂದರೆ, ಇನ್ನು ಅವರು ಯಾವುದೇ ಕೋರಿಕೆಯ ಬೆನ್ನಟ್ಟುವುದಿಲ್ಲ, ಆದ್ದರಿಂದ, ಯಾವುದೇ ಲೌಕಿಕ, ಅಲೌಕಿಕ ವಿಷಯದ ಮೇಲಿನ ಆಸಕ್ತಿಗಳನ್ನೂ  ಕೂಡ ಮೆಟ್ಟಿ ನಿಂತಿರುತ್ತಾರೆ. “ಮನುಷ್ಯನು ಪ್ರಯತ್ನಪಟ್ಟು ತನ್ನ ಎಲ್ಲಾ ಕಾರ್ಯಗಳಲ್ಲೂ ವೈಯಕ್ತಿಕ ನಿರೀಕ್ಷೆಯನ್ನು ತ್ಯಜಿಸಬೇಕು ಆದರೆ,ಸ್ವತಃ ಕ್ರಿಯೆಗಳನ್ನಲ್ಲ”, ಎಂಬುದು ಅತ್ಯುನ್ನತ ಧರ್ಮವಾದ ಸoನ್ಯಾಸದ ಸಾರವಾಗಿದೆ ಎಂದು ಭಗವಂತನು ಸ್ಪಷ್ಟ ಪಡಿಸಿದ್ದಾನೆ. ಕ್ರಿಯೆಗಳು ಸ್ವತಃ ಆಧ್ಯಾತ್ಮಿಕ ಪ್ರಗತಿಗೆ ಅಡ್ಡಿಯಾಗುವುದಿಲ್ಲ,ಆದರೆ ಅದರ (ಕ್ರಿಯೆಗಳ) ಫಲಿತಾಂಶಗಳ ಚಿಂತೆ ಮತ್ತು ನಿರೀಕ್ಷೆಗಳು, ತದನಂತರದ ಮೋಹ, ಅಡೆತಡೆ ಉಂಟುಮಾಡುತ್ತವೆ
\end{inspiration}
\newpage


\slcol{\Index{ಆರುರುಕ್ಷೋರ್ಮುನೇರ್ಯೋಗಂ} ಕರ್ಮ ಕಾರಣಮುಚ್ಯತೇ ।\\
ಯೋಗಾರೂಢಸ್ಯ ತಸ್ಯೈವ ಶಮಃ ಕಾರಣಮುಚ್ಯತೇ ॥ ೩ ॥}
\cquote{ಸಾಧನೆಯ ದಾರಿಯಲ್ಲಿ ಮೇಲೇರ ಬಯಸುವ ಸಾಧಕನಿಗೆ ಲೋಕಸೇವಾರೂಪವಾದ ಕರ್ಮವೂ ಉನ್ನತಿಯ ಸಾಧನ. ಸಿದ್ಧಿ ಪಡೆದು ಭಗವಂತನಲ್ಲಿ ನೆಲೆಗೊಂಡ ಜ್ಞಾನಿಗೆ ಲೌಕಿಕ ಕರ್ಮದ ನಿಯತಿಯಿಲ್ಲ.}
\slcol{\Index{ಯದಾ ಹಿ ನೇಂದ್ರಿಯಾರ್ಥೇಷು} ನ ಕರ್ಮಸ್ವನುಷಜ್ಜತೇ ।\\
ಸರ್ವಸಂಕಲ್ಪಸಂನ್ಯಾಸೀ ಯೋಗಾರೂಢಸ್ತದೋಚ್ಯತೇ ॥ ೪ ॥}
\cquote{ವಿಷಯಗಳಲ್ಲಿ, ಕರ್ಮಗಳಲ್ಲಿ ಮಮತೆ ತೊರೆದು ಎಲ್ಲ ಕಾಮನೆಗಳನ್ನೂ ಮೀರಿ ನಿಂತವನು ಸಾಧನೆಯ ಮೆಟ್ಟಲೇರಿ ಸಿದ್ಧಿ ಪಡೆದಂತವನು. ಅವನನ್ನು ಯೋಗಾರೂಢನೆಂದು ಕರೆಯುತ್ತಾರೆ.}
\slcol{\Index{ಉದ್ಧರೇದಾತ್ಮನಾತ್ಮಾನಂ}
ನಾತ್ಮಾನಮವಸಾದಯೇತ್ ।\\
ಆತ್ಮೈವ ಹ್ಯಾತ್ಮನೋ ಬಂಧುರಾತ್ಮೈವ ರಿಪುರಾತ್ಮನಃ ॥ ೫ ॥}
\cquote{ತನ್ನನ್ನು ತನ್ನ ಮನೋಬಲದಿಂದಲೇ ಮೇಲಕ್ಕೆತ್ತಬೇಕು. ಪತನದ ದಾರಿಗೆ ತನ್ನನ್ನು ತಳ್ಳಬಾರದು. ಏಕೆಂದರೆ ನಮ್ಮ ಮನಸ್ಸೇ ನಮಗೆ ನೆಂಟ ನಮ್ಮ ಮನಸ್ಸೇ  ನಮಗೆ ಶತ್ರು.}
\slcol{\Index{ಬಂಧುರಾತ್ಮಾತ್ಮನಸ್ತಸ್ಯ} ಯೇನಾತ್ಮೈವಾತ್ಮನಾ ಜಿತಃ ।\\
ಅನಾತ್ಮನಸ್ತು ಶತ್ರುತ್ವೇ ವರ್ತೇತಾತ್ಮೈವ ಶತ್ರುವತ್ ॥ ೬ ॥}
\cquote{ಮನೋಬಲದಿಂದ ತನ್ನನ್ನು ತಾನೇ ಗೆದ್ದವನಿಗೆ ಅವನ ಮನಸ್ಸು ನೆಂಟ. ಯಾರು ಮನಸ್ಸನ್ನು ಗೆದೆಯಲಾರ ಅವನನ್ನು ಮನಸ್ಸೇ ಶತ್ರುವಾಗಿ ಕಾಡುತ್ತದೆ.}
\slcol{\Index{ಜಿತಾತ್ಮನಃ ಪ್ರಶಾಂತಸ್ಯ} ಪರಮಾತ್ಮಾ ಸಮಾಹಿತಃ ।\\
ಶೀತೋಷ್ಣಸುಖದುಃಖೇಷು ತಥಾ ಮಾನಾಪಮಾನಯೋಃ ॥ ೭ ॥}
\cquote{ಶೀತೋಷ್ಣ, ಸುಖದುಃಖ,ಮಾನ ಅಪಮಾನಗಳಲ್ಲಿ ಚಿತ್ತವನ್ನು ಜಯಿಸಿದ ಪ್ರಶಾಂತನಿಗೆ ಪರಮಾತ್ಮನ ಅನುಭವ ಉಂಟಾಗುವುದು.}


\newpage
\begin{mananam}{\mananamfont{ಮನನ ಶ್ಲೋಕ - ೩, ೪}}
\small \mananamtext ಧ್ಯಾನಕ್ಕೆ ಕುಳಿತುಕೊಳ್ಳಲು ನನ್ನ ಮನಸ್ಸು ಸಿದ್ಧವಾಗಿದೆಯೇ? ಸಹಕರಿಸುತ್ತದೆಯೇ? ಅನ್ಯಮನಸ್ಕನಾಗಿ ಅಥವಾ ಅತಿಯಾದ ಪ್ರಕ್ಷುಬ್ಧತೆಯ   ಭಾವನೆ ಇಲ್ಲದೆ ನಿತ್ಯವೂ ನಾನು ಎಷ್ಟು ಸಮಯವನ್ನು ಧ್ಯಾನಕ್ಕಾಗಿ ಮೀಸಲಿಡಬಹುದು? ಧ್ಯಾನ ಮಾಡಲು ಸಾಧ್ಯವಾಗದಿದ್ದಲ್ಲಿ ನನ್ನನ್ನು ನಾನು ಆ ಸಮಯದಲ್ಲಿ ನಿಸ್ವಾರ್ಥವಾದ  ಸೇವೆಯಲ್ಲಿ ತೊಡಗಿಸಿಕೊಳ್ಳಬಲ್ಲೆನೇ? ನನ್ನ ದಿನನಿತ್ಯದ ಕಾರ್ಯ ಮತ್ತು ಧ್ಯಾನದ ಮಧ್ಯೆ ಸಮತೋಲನವನ್ನು ಹೇಗೆ ಕಾಪಾಡಬಲ್ಲೆ? ನನ್ನ ಧ್ಯಾನದ ಸಮಯವನ್ನು ರಾಜಿ ಮಾಡಿಕೊಳ್ಳದಂತೆ ಖಚಿತಪಡಿಸಿಕೊಳ್ಳಲು, ನನ್ನ ಚಟುವಟಿಕೆಗಳನ್ನು ಹೇಗೆ ನಿಯಂತ್ರಿಸಬೇಕು?
\end{mananam}
\WritingHand\enspace\textbf{ಆತ್ಮ ವಿಮರ್ಶೆ}\\
\begin{inspiration}{\mananamfont ಸ್ಫೂರ್ತಿ}
\small \mananamtext ಒಬ್ಬ ನಿಜವಾದ ಆಧ್ಯಾತ್ಮಿಕ ಗುರು ಅಥವಾ ಆಧ್ಯಾತ್ಮಿಕ ಬೋಧನೆಯು ಪ್ರತಿಯೊಬ್ಬ ವಿದ್ಯಾರ್ಥಿಗೂ, ಅವರವರ ಆಧ್ಯಾತ್ಮಿಕ ಪ್ರಯಾಣದ ವಿಕಾಸದ ಆಧಾರದ ಮೇಲೆ ಸೂಚನೆಗಳನ್ನು ಒದಗಿಸುತ್ತದೆ.  ನಾವು ಮನಸ್ಸನ್ನು ಶಾಂತಮಾಡಿ, ಚಂಚಲತೆ ಮತ್ತು ಲೌಕಿಕ ಆಲೋಚನೆಗಳಿಂದ ಮುಕ್ತಮಾಡಿದಾಗ ಮಾತ್ರವೇ ಧ್ಯಾನದ ನಿಶ್ಚಲತೆಯನ್ನು ತಲುಪಲು ಸಾಧ್ಯ. ಅಂತಹ ಧ್ಯಾನದ ನಿಶ್ಚಲತೆಯನ್ನು ಆನಂದಿಸಲು ಅಂತತಃ, ಧ್ಯಾನಸ್ಥಿತಿಗೆ ಕೊಂಡೊಯ್ಯುವ  ಎಲ್ಲಾ ದೈಹಿಕ ಮತ್ತು ಮಾನಸಿಕ  ಪ್ರಯತ್ನಗಳನ್ನು ಕೂಡ ನಿಲ್ಲಿಸಬೇಕು. ದೀರ್ಘಾವಧಿಯ ಧ್ಯಾನಕ್ಕಾಗಿ ಕುಳಿತುಕೊಳ್ಳಲು ದೇಹ  ಮತ್ತು ಮನಸ್ಸು ಸಿದ್ಧವಾಗಿರದಿದ್ದಲ್ಲಿ, ನಿಸ್ವಾರ್ಥವಾದ, ನಿಷ್ಕಲ್ಮಶ ಸೇವೆಗಳಲ್ಲಿಯಾದರೂ ನಮ್ಮನ್ನು ನಾವು ತೊಡಗಿಸಿಕೊಳ್ಳಬೇಕು.
\end{inspiration}
\newpage

\newpage
\begin{mananam}{\mananamfont{ಮನನ ಶ್ಲೋಕ - ೫, ೬}}
\small \mananamtext ನನ್ನ ದಿನನಿತ್ಯದ ಜೀವನದಲ್ಲಿ ನಾನು, ನನ್ನ ನಿಜವಾದ ಸ್ನೇಹಿತ ಮತ್ತು ಮಾರ್ಗದರ್ಶಿಯಾಗಿರುವ ನನ್ನ ಆತ್ಮಸಾಕ್ಷಿಯ ಜೊತೆಯಲ್ಲಿ ಸಾಮರಸ್ಯದಿಂದಿರುವೆನೇ?  ನನ್ನ ಸಕರಾತ್ಮಕ ಉದ್ದೇಶಗಳು ಮತ್ತು ನಿರ್ಣಯಗಳಿಗೆ ನಾನು ಬದ್ಧನಾಗಿದ್ದೇನೆಯೇ? ಅಥವಾ ಮಾರ್ಗದರ್ಶನ ನೀಡುವ ನನ್ನ ಆಂತರಿಕ ಧ್ವನಿಯನ್ನು ನಾನು ವಿರೋಧಿಸುತ್ತೇನೆಯೇ? 
ನನ್ನ ನಿಜವಾದ ಹಿತೈಷಿಗಳಾದ ಒಳ್ಳೆಯ ಜನರನ್ನು ನಾನು ಬಾಲಿಶವಾಗಿ ವಿರೋಧಿಸುತ್ತೇನೆಯೇ? ಈ ರೀತಿಯಾಗಿ ಗುರುಗಳ ಮತ್ತು ಧರ್ಮ ಗ್ರಂಥಗಳ ಮಾರ್ಗದರ್ಶನದ ವಿರೋಧ, ನನ್ನ ಸ್ವತಃ ಆಧ್ಯಾತ್ಮಿಕ ಪ್ರಗತಿಗೆ ಅಡ್ಡಿಯಾಗುತ್ತದೆ ಎಂಬುದನ್ನು ಗ್ರಹಿಸಬಲ್ಲೆನೇ? ನನ್ನ ಜೀವನದಲ್ಲಿ ಇರುವ ಅಹಂಕಾರದ ಶಕ್ತಿಯ ಪಾತ್ರ ಮತ್ತು ಅದರ ಪರಿಣಾಮದ ಬಗ್ಗೆ ನನಗೆ ಅರಿವಿದೆಯೇ?
\end{mananam}
\WritingHand\enspace\textbf{ಆತ್ಮ ವಿಮರ್ಶೆ}\\
\begin{inspiration}{\mananamfont ಸ್ಫೂರ್ತಿ}
ಶುದ್ಧ ಚೈತನ್ಯ ಮತ್ತು ಆನಂದದ ಸ್ವರೂಪವೇ ಆದ ಆತ್ಮನಲ್ಲಿ ಗಮನ ಕೇಂದ್ರೀಕರಿಸಿದಾಗ, ನಮ್ಮ ಮನಸ್ಸು,  ಧನಾತ್ಮಕ ಮತ್ತು ಆಧ್ಯಾತ್ಮಕ ಉನ್ನತಿಯತ್ತ ಸಾಗುತ್ತದೆ. ಅಂತಹ ಮನಸ್ಸು ನಮ್ಮ ಬುದ್ಧಿಯೊಂದಿಗೆ ಸಹಕರಿಸಿ ಸಮಗ್ರ ಯೋಗಕ್ಷೇಮವನ್ನು ಪೋಷಿಸುತ್ತದೆ. ಆದರೆ, ಮನಸ್ಸು ಇಂದ್ರಿಯಗಳ ಸೆಳೆತಕ್ಕೊಳಗಾದಾಗ ಅದು, ಕ್ಷಣಿಕ ಲೌಕಿಕ ಸಂತೋಷ ಮತ್ತು ತೃಪ್ತಿಯನ್ನು ಬೆನ್ನಟ್ಟುತ್ತದೆ. ಹಾಗೆ ಮಾಡಿದಾಗ ಅದು, ಪ್ರಜ್ಞೆಯ ಜೊತೆ ಮಿಳಿತವಾದ ಬುದ್ಧಿಯ (ವಿವೇಕದ)   ವಿರುದ್ಧ ಬಂಡಾಯವೆದ್ದು, ‘ಅಜ್ಞಾನವೇ ಆನಂದ’ ಎಂಬ ಗಾದೆಯಂತಾಗುತ್ತದೆ. ದುರದೃಷ್ಟವಶಾತ್ ಅಂತಹ ಜೀವನ ಮಾರ್ಗವು ದುಃಖದಿಂದ ತುಂಬಿದೆ. 
\end{inspiration}
\newpage

\slcol{\Index{ಜ್ಞಾನವಿಜ್ಞಾನತೃಪ್ತಾತ್ಮಾ} ಕೂಟಸ್ಥೋ ವಿಜಿತೇಂದ್ರಿಯಃ ।\\
ಯುಕ್ತ ಇತ್ಯುಚ್ಯತೇ ಯೋಗೀ ಸಮಲೋಷ್ಟಾಶ್ಮಕಾಂಚನಃ ॥ ೮ ॥}
\cquote{ಜ್ಞಾನ-ವಿಜ್ಞಾನದಿಂದ ತೃಪ್ತ ಅಂತ:ಕರಣವುಳ್ಳವನ ಸ್ಥಿತಿಯು ವಿಕಾರರಹಿತವಾಗಿರುತ್ತದೆ. ಇಂದ್ರಿಯಗಳನ್ನು ಪೂರ್ಣವಾಗಿ ಗೆದ್ದುಕೊಂಡು ಕಲ್ಲು, ಮಣ್ಣು ಮತ್ತು ಬಂಗಾರವನ್ನು ಸಮಾನವಾಗಿ ಕಾಣುವ ಯೋಗಿಯು ಯುಕ್ತನು, ಅರ್ಥಾತ್, ಭಗವತ್ ಪ್ರಾಪ್ತನಾಗಿದ್ದಾನೆ ಎಂದು ಹೇಳಲಾಗುತ್ತದೆ.}
\slcol{\Index{ಸುಹೃನ್ಮಿತ್ರಾರ್ಯುದಾಸೀನಮಧ್ಯ}ಸ್ಥದ್ವೇಷ್ಯಬಂಧುಷು ।\\
ಸಾಧುಷ್ವಪಿ ಚ ಪಾಪೇಷು ಸಮಬುದ್ಧಿರ್ವಿಶಿಷ್ಯತೇ ॥ ೯ ॥}
\cquote{ಅಕಾರಣ ಬಂಧುಗಳು, ಆಪತ್ತಿಗೆ ಒದಗದ ಗೆಳೆಯರು, ಹಗೆಗಳು, ಯಾವ ಪಕ್ಷವನ್ನೂ ಬಯಸದವರು, ಎರಡೂ ಪಕ್ಷಗಳ ಹಿತವನ್ನು ಬಯಸುವವರು, ಅಪ್ರಿಯವಾದದ್ದನ್ನೇ ಮಾಡುವವರು, ನೆಂಟರು, ಸಜ್ಜನರು, ದುರ್ಜನರು-ಎಂಬ ವಿಧವಿಧ ವರ್ಗಗಳಲ್ಲಿಯೂ ಸಮಬುದ್ಧಿಯುಳ್ಳವನು ಅಂದರೆ, ಇವರೆಲ್ಲರನ್ನು ನಿರ್ವಿಕಾರ ಬುದ್ಧಿಯಿಂದ ಕಾಣುವವನು ಯೋಗಿಗಳಲ್ಲಿ ಹೆಚ್ಚಿನವನು.}
\slcol{\Index{ಯೋಗೀ ಯುಂಜೀತ} ಸತತಮಾತ್ಮಾನಂ ರಹಸಿ ಸ್ಥಿತಃ ।\\
ಏಕಾಕೀ ಯತಚಿತ್ತಾತ್ಮಾ ನಿರಾಶೀರಪರಿಗ್ರಹಃ ॥ ೧೦ ॥}
\cquote{ಯೋಗಿಯಾದವನು ಏಕಾಂತದಲ್ಲಿ ಒಂಟಿಯಾಗಿದ್ದು ದೇಹವನ್ನು ಮನಸ್ಸನ್ನು ನಿಯಂತ್ರಿಸಿಕೊಂಡು, ಏನನ್ನೂ ಬಯಸದೆ, ಪರರ ಸ್ವತ್ತಿಗೆ ಕೈಯೊಡ್ಡದೆ, ಯಾವಾಗಲೂ ಆತ್ಮವನ್ನು ಭಗವಂತನಲ್ಲಿಯೇ ನೆಲೆಗೊಳಿಸಬೇಕು.}


\newpage
\begin{mananam}{\mananamfont{ಮನನ ಶ್ಲೋಕ - ೮, ೯}}
\small \mananamtext ನಾನು, ‘ಒಂದೇ ದೈವೀಕ ಸಾರವೇ ಎಲ್ಲರಲ್ಲೂ ಇರುವುದು’ ಎಂದು ಗ್ರಹಿಸುತ್ತೇನೆಯೇ? ಅಥವಾ ಸಂಪೂರ್ಣವಾಗಿ, ನನ್ನ ಮಾನಸಿಕ ಒಲವು ಮತ್ತು ಹಿಂದಿನ ಅನುಭವಗಳ ಮಸೂರದ ಮೂಲಕ ವಸ್ತು ಮತ್ತು ಜನರನ್ನು ಪರಕಿಸುತ್ತೇನೆಯೇ?  ಇತರರೊಂದಿಗಿನ ನನ್ನ ನಡೆ, ನುಡಿಗಳ ಮೇಲೆ, ಅವರ ಬಗ್ಗೆ ಇರುವ ನನ್ನ ಪೂರ್ವಭಾವೀ ಗ್ರಹಿಕೆಗಳು ಎಷ್ಟರಮಟ್ಟಿಗೆ   ಪ್ರಭಾವ ಬೀರುತ್ತವೆ? ಮತ್ತು ಬಣ್ಣಗಟ್ಟುತ್ತವೆ? ಜೀವನದ ಪ್ರತಿಯೊಂದೂ ಅನುಭವಗಳೆಡೆಗೆ ಹೊಸ ದೃಷ್ಟಿಕೋನವನ್ನು ಅಳವಡಿಸಿಕೊಳ್ಳುವುದರಿಂದ ಯಾವ ಪ್ರಯೋಜನಗಳಿವೆ? ಆ ಹೊಸ ದೃಷ್ಟಿಕೋನವನ್ನು ಕಾರ್ಯರೂಪಕ್ಕೆ ತರಲು ನನಗಿರುವ ಅಡೆತಡೆಗಳು ಯಾವುವು?
\end{mananam}
\WritingHand\enspace\textbf{ಆತ್ಮ ವಿಮರ್ಶೆ}\\
\begin{inspiration}{\mananamfont ಸ್ಫೂರ್ತಿ}
\small \mananamtext ಆತ್ಮಸಾಕ್ಷಾತ್ಕಾರ ಹೊಂದಿದ ಯೋಗಿಗಳು ಎಲ್ಲದರಲ್ಲಿಯೂ ಅಡಗಿರುವ ಚೈತನ್ಯವನ್ನು ಗ್ರಹಿಸುತ್ತಾರೆ. ಅವರಿಗೆ ಪ್ರತಿಯೊಬ್ಬರೂ, ಆ ಏಕ ದೈವಿಕ  ಚೈತನ್ಯದ ಒಂದು ಕಣದಂತೆ ಕಾಣುತ್ತಾರೆ. ಪರಿಣಾಮವಾಗಿ, ಸರ್ವರಲ್ಲಿಯೂ ಅವರು ವ್ಯವಹರಿಸುವ ರೀತಿ ಪೂರ್ವಾಗ್ರಹ ಮತ್ತು ಪಕ್ಷಪಾತಗಳಿಂದ ಮುಕ್ತವಾಗಿರುತ್ತವೆ. ಇದಕ್ಕೆ ವ್ಯತಿರಿಕ್ತವಾಗಿ  ಇತರರಿಗೆ, ಈ ಜಗತ್ತು ಹಣೆಪಟ್ಟಿಗಳು, ಹೆಸರುಗಳು, ವರ್ಗಗಳು, ಮತ್ತು ನಿರ್ದಿಷ್ಟ ರೂಪಗಳಿಂದಷ್ಟೇ ಗುರುತಿಸಲ್ಪಟ್ಟು,  ಭಿನ್ನ ಭಿನ್ನವಾದಂತೆ ತೋರುತ್ತದೆ.
\end{inspiration}
\newpage

\newpage
\begin{mananam}{\mananamfont{ಮನನ ಶ್ಲೋಕ - ೧೦}}
\small \mananamtext ನನ್ನ ದೈನಂದಿನ ಜೀವನದಲ್ಲಿ ನನ್ನೊಂದಿಗೆ ನಾನು ಸುಮ್ಮನೆ ಇರಲು ಸಮಯವನ್ನು ಮೀಸಲಿಡುತ್ತೇನೆಯೇ? ಆಗಾಗ್ಗೆ  ಏಕಾಂತದಲ್ಲಿರುವುದು ನನಗೆ ಆರಾಮದಾಯಕವೆಂದು ಅನ್ನಿಸುತ್ತದೆಯೇ? ಅಥವಾ ವಿಚಲಿತತೆ ಉಂಟುಮಾಡುತ್ತದೆಯೇ? ಏಕಾಂತವಾಗಿರುವಾಗ ನನಗೆ ಆತಂಕ, ಭಯ, ಬೇಸರ ಮತ್ತು ಅರ್ಥವಿಲ್ಲದ ಒಂಟಿತನ ಕಾಡುತ್ತದೆಯೇ? ಬಾಹ್ಯ ಜಗತ್ತಿನ ಸಂಪರ್ಕಕ್ಕಾಗಿ ಚಡಪಡಿಸುತ್ತೇನೆಯೇ?\\        
ಏಕಾಂತದಲ್ಲಿರುವಾಗ, ರಂಜನೆಗಾಗಿ ಸಂಗೀತ,ಟಿವಿ, ಅಥವಾ ಇತರ ಉಪಕರಣಗಳನ್ನು ಬಳಸಲೇಬೇಕೆಂಬ ಬಲವಂತದ ಭಾವನೆಗೆ ಮೊರೆಹೋಗುತ್ತೇನೆಯೇ? ಇತರರ ಜೊತೆಗಿನ ಸಂಪರ್ಕದ ಬದಲಿಗೆ ಹೀಗೆ ಮಾಡುತ್ತೇನೆಯೇ? ಒಂದು ಗಂಟೆಯ ಏಕಾಂತತೆ ಮತ್ತು ಚಿಂತನೆಯು ನನಗೆ ಯಾವ ಆಧ್ಯಾತ್ಮಿಕ ಮತ್ತು ಮಾನಸಿಕ ಪ್ರಯೋಜನಗಳನ್ನು ನೀಡುತ್ತದೆ?
\end{mananam}
\WritingHand\enspace\textbf{ಆತ್ಮ ವಿಮರ್ಶೆ}\\
\begin{inspiration}{\mananamfont ಸ್ಫೂರ್ತಿ}
\small \mananamtext ಪ್ರಾಚೀನ ಭಾರತದ ಬಗ್ಗೆ ವರ್ಣಿಸಿರುವ ಜೀವನ ಶೈಲಿಗಿಂತ, ಆಧುನಿಕ ಪ್ರಪಂಚದ ಜೀವನ ಶೈಲಿಯು ಗಮನಾರ್ಹವಾಗಿ ಭಿನ್ನವಾಗಿದೆ.ಇಂದಿನ ದಿನಗಳಲ್ಲಿ ತಂತ್ರಜ್ಞಾನ ಮತ್ತು ಸಾಮಾಜಿಕ  ಮಾಧ್ಯಮಗಳು ಮನುಷ್ಯರ ಏಕಾಂತದ ಅವಕಾಶವನ್ನು ಕಸಿದುಕೊಂಡಿವೆ. ಏಕಾಂತವು ಆತ್ಮಾವಲೋಕನೆ, ಆಂತರಿಕ ಶಾಂತಿಗೆ ಮತ್ತು ಮಾನಸಿಕ ಪುನರುಜ್ಜೀವನಕ್ಕೆ ಅವಕಾಶ ನೀಡುತ್ತದೆ; ಸೃಜನಶೀಲತೆ ಬೆಳೆಸುತ್ತದೆ, ಚಿಂತನೆಯಲ್ಲಿ ಸ್ಪಷ್ಟತೆ ಹೆಚ್ಚಿಸುತ್ತದೆ, ಸ್ವಯಂ ಬಗ್ಗೆ  ಆಳವಾದ ಅರಿವುಂಟುಮಾಡುತ್ತದೆ ಮತ್ತು ಮನೋಬಲ ಹೆಚ್ಚಿಸುತ್ತದೆ. ನಮ್ಮೊಂದಿಗೆ ನಾವು ಸಕರಾತ್ಮಕವಾಗಿ ಇರಲು ಕಲಿಯುವುದು ಒಂದು ಅಮೂಲ್ಯವಾದ ಕೌಶಲ್ಯವಾಗಿದೆ. ಈ ಕಲೆಯಲ್ಲಿ ಪ್ರಾವೀಣ್ಯತೆಯನ್ನು ಪಡೆಯುವುದರಿಂದ ಉತ್ತಮ ಪ್ರಯೋಜನವಿದೆ. ಈ ಕೌಶಲ್ಯವನ್ನು ನಿರಂತರವಾಗಿ ಅಭ್ಯಾಸ ಮಾಡುವುದು ವಿಜ್ಞಾನವೇ ಆಗಿದೆ. ನಮ್ಮ ಪ್ರಾಚೀನ ಋಷಿಗಳು ಈ ಕಲೆ ಮತ್ತು ವಿಜ್ಞಾನವನ್ನು “ಧ್ಯಾನ” ಎಂದು ಕರೆದರು.
\end{inspiration}
\newpage

\slcol{\Index{ಶುಚೌ ದೇಶೇ ಪ್ರತಿಷ್ಠಾಪ್ಯ} ಸ್ಥಿರಮಾಸನಮಾತ್ಮನಃ ।\\
ನಾತ್ಯುಚ್ಛ್ರಿತಂ ನಾತಿನೀಚಂ ಚೈಲಾಜಿನಕುಶೋತ್ತರಮ್ ॥ ೧೧ ॥\\
ತತ್ರೈಕಾಗ್ರಂ ಮನಃ ಕೃತ್ವಾ ಯತಚಿತ್ತೇಂದ್ರಿಯಕ್ರಿಯಾಃ ।\\
ಉಪವಿಶ್ಯಾಸನೇ ಯುಂಜ್ಯಾದ್ಯೋಗಮಾತ್ಮವಿಶುದ್ಧಯೇ ॥ ೧೨ ॥}
\cquote{ಶುದ್ಧವಾದ ಕಡೆ ಹೆಚ್ಚು ಎತ್ತರವೂ ಹೆಚ್ಚು ತಗ್ಗೂ ಆಗದಂತೆ ದರ್ಭೆ, ಹುಲಿಯ ಅಥವಾ ಜಿಂಕೆಯ ಚರ್ಮ, ಬಟ್ಟೆ ಇವುಗಳನ್ನು ಕ್ರಮವಾಗಿ ಒಂದರ ಮೇಲೊಂದಾಗಿ ಹಾಸಿ ತನಗೆ ಭದ್ರವಾದ ಆಸನವನ್ನು ಸಿದ್ಧಪಡಿಸಿಕೊಂಡು, ಆ ಪೀಠದ ಮೇಲೆ ಕುಳಿತು ಮನಸ್ಸಿನ ಮತ್ತು ಇಂದ್ರಿಯಗಳ ಚಲನವನ್ನು ತಡೆದು, ಮನಸ್ಸನ್ನು ಒಮ್ಮುಖವಾಗಿ  ಮಾಡಿ ಮನಸ್ಸಿನ ಶುದ್ಧಿಗಾಗಿ ಮನೋನಿಗ್ರಹವನ್ನು ಅಭ್ಯಾಸ ಮಾಡಬೇಕು.}
\slcol{\Index{ಸಮಂ ಕಾಯಶಿರೋಗ್ರೀವಂ} ಧಾರಯನ್ನಚಲಂ ಸ್ಥಿರಃ ।\\
ಸಂಪ್ರೇಕ್ಷ್ಯ ನಾಸಿಕಾಗ್ರಂ ಸ್ವಂ ದಿಶಶ್ಚಾನವಲೋಕಯನ್ ॥ ೧೩ ॥\\
ಪ್ರಶಾಂತಾತ್ಮಾ ವಿಗತಭೀರ್ಬ್ರಹ್ಮಚಾರಿವ್ರತೇ ಸ್ಥಿತಃ ।\\
ಮನಃ ಸಂಯಮ್ಯ ಮಚ್ಚಿತ್ತೋ ಯುಕ್ತ ಆಸೀತ ಮತ್ಪರಃ ॥ ೧೪ ॥}
\cquote{ಮೈ, ತಲೆ, ಕೊರಳು - ಇವುಗಳು ಅಲ್ಲಾಡದಂತೆ ಸ್ಥಿರವಾಗಿ ನೆಟ್ಟಗೆ ಕುಳಿತು ಅತ್ತಿತ್ತ ದಿಕ್ಕುಗಳ ಕಡೆ ನೋಡದೆ ದೃಷ್ಟಿಯನ್ನು ಮೂಗಿನ ತುದಿಯಲ್ಲಿ ಕೇಂದ್ರೀಕರಿಸಿ ತಿಳಿ ಮನಸ್ಸಿನಿಂದ ನಿರ್ಭಯನಾಗಿ ಗುರು ಸೇವೆ, ಮೊದಲಾದ ವ್ರತವನ್ನು ನಡೆಸುತ್ತಾ ಮನಸ್ಸನ್ನು ಬಿಗಿಹಿಡಿದು ನನ್ನಲ್ಲಿಟ್ಟು ನನ್ನನ್ನೇ ಧ್ಯಾನ ಮಾಡುತ್ತಾ ಇರಬೇಕು.}
\slcol{\Index{ಯುಂಜನ್ನೇವಂ ಸದಾತ್ಮಾನಂ} ಯೋಗೀ ನಿಯತಮಾನಸಃ ।\\
ಶಾಂತಿಂ ನಿರ್ವಾಣಪರಮಾಂ ಮತ್ಸಂಸ್ಥಾಮಧಿಗಚ್ಛತಿ ॥ ೧೫ ॥}
\cquote{ಹೀಗೆ ಯಾವಾಗಲೂ ಮನಸ್ಸನ್ನು ಬಿಗಿ ಹಿಡಿದು ತನ್ನನ್ನು ಧ್ಯಾನದಲ್ಲಿ ತೊಡಗಿಸಿದ ಯೋಗಿಯು ಈ ದೇಹ ತೊರೆದ ಮೇಲೆ ನನ್ನಲ್ಲಿ ನೆಲೆಯಾಗಿ ಸುಖವನ್ನು ಪಡೆಯುತ್ತಾನೆ.}

\newpage
\begin{mananam}{\mananamfont{ಮನನ ಶ್ಲೋಕ - ೧೨}}
\small \mananamtext ನಿಯಮಿತ ಅಭ್ಯಾಸವನ್ನು ಆಕಾಂಕ್ಷಿಸುವವರಿಗೆ ಧ್ಯಾನ ಮತ್ತು ಆತ್ಮಪರಿಶೀಲನೆಗಾಗಿ ಸೂಚಿಸಿರುವ ಕ್ಷೇತ್ರಗಳು:\\ 
\textbf{ಸ್ಥಿರ ಅಭ್ಯಾಸ:} ನನ್ನ ಜೀವನದಲ್ಲಿ ನಿಯಮಿತವಾಗಿ ಧ್ಯಾನದ ಅಭ್ಯಾಸವನ್ನು ಮಾಡುತ್ತೇನೆಯೇ?\\
\textbf{ಇಂದ್ರಿಯ ನಿಗ್ರಹ:} ಧ್ಯಾನದ ಸಂಪೂರ್ಣ ಸಮಯ, ಕಣ್ಣುಗಳನ್ನು ಮುಚ್ಚಿಕೊಂಡು, ಬೇರೆ  ಇಂದ್ರಿಯಗಳನ್ನು ಅಂತರ್ಮುಖವಾಗಿಸಿ, ದೇಹದಲ್ಲಿ ಸ್ಥಿರತೆಯನ್ನು ಕಾಪಾಡಿಕೊಳ್ಳಬಲ್ಲೆನೇ?\\
\textbf{ಗಮನ:}  ಏಕಾಗ್ರತೆಯ ಸ್ಥಿತಿಯನ್ನು ನಾನು ಎಷ್ಟು ಬೇಗ ಸಾಧಿಸಬಲ್ಲೆ? ಅಲ್ಪ ಸಮಯದ ಧ್ಯಾನದಲ್ಲೂ ತೀವ್ರತೆಯನ್ನು ತರಬಲ್ಲೆನೇ?\\
\textbf{ಮನಸ್ಸಿನ ಸ್ಥಿತಿ:} ನನ್ನ ಮನಸ್ಸು ಆಗಾಗ್ಯೆ, ಮಂದ ಅಥವಾ ಪ್ರಕ್ಷುಬ್ದವಾಗಿದೆಯೇ?\\
\textbf{ಪ್ರಕ್ರಿಯೆ:} ನಾನು ಧ್ಯಾನ ಪ್ರಕ್ರಿಯೆಯನ್ನು ಸಂಪೂರ್ಣವಾಗಿ ಗ್ರಹಿಸಿದ್ದೇನೆಯೇ? ಗುರುಗಳಿಂದ ಮಾರ್ಗದರ್ಶನ ಪಡೆಯಲು ನನ್ನ ಮನಸ್ಸು ತೆರೆದಿದೆಯೇ?
\end{mananam}
\WritingHand\enspace\textbf{ಆತ್ಮ ವಿಮರ್ಶೆ}\\
\begin{inspiration}{\mananamfont ಸ್ಫೂರ್ತಿ}
\small \mananamtext ದೇಹವನ್ನು ಪ್ರತಿದಿನ ಸ್ನಾನದಿಂದ ಶುದ್ಧೀಕರಿಸಿದಂತೆಯೇ – ‘ಧ್ಯಾನ’, ಮನಸ್ಸಿನ ಆಂತರಿಕ ಶುದ್ಧೀಕರಣಕ್ಕೆ ಒಂದು ಅತ್ಯಾವಶ್ಯಕ ಸಾಧನವೇ ಆಗಿದೆ. ಶಾಂತಿ ಮತ್ತು ಅಂತರಂಗದ ಸಂತೋಷದಿಂದಾಗುವ ಪ್ರಯೋಜನಗಳ ದ್ವಾರ ತೆರೆಯಲು ಹಾಗೂ ಪಾರಮಾರ್ಥಿಕದತ್ತ ಸಾಗಲು, ಔಪಚಾರಿಕ ಧ್ಯಾನದ ಪ್ರಕ್ರಿಯೆಯನ್ನು ಕರಗತ ಮಾಡಿಕೊಳ್ಳುವುದು ಅತ್ಯಗತ್ಯ. ಭಗವದ್ಗೀತೆಯ ಪ್ರಕಾರ ಒಬ್ಬ ಆದರ್ಶಯೋಗಿ ಎಂದರೆ, ಪ್ರಕೃತಿಯನ್ನು ತನ್ನ ಇಚ್ಛೆಗೆ ತಕ್ಕಂತೆ ಬದಲಾಯಿಸುವ ಅಥವಾ ಪವಾಡಗಳನ್ನು ಮಾಡುವ ವ್ಯಕ್ತಿ ಅಲ್ಲ, ಬದಲಾಗಿ, ಇಂದ್ರಿಯಗಳು ಮತ್ತು ಮನಸ್ಸನ್ನು ನಿಗ್ರಹಿಸಿ, ಜಗತ್ತಿನ ವ್ಯವಹಾರಗಳಲ್ಲಿ ತನ್ನ ಪ್ರತಿಕ್ರಿಯೆಗಳನ್ನು ನಿಯಂತ್ರಿಸಲು ಕಲಿಯುವವನೇ  ಆದರ್ಶ ಯೋಗಿ.
\end{inspiration}
\newpage


\slcol{\Index{ನಾತ್ಯಶ್ನತಸ್ತು ಯೋಗೋಽಸ್ತಿ} ನ ಚೈಕಾಂತಮನಶ್ನತಃ ।\\
ನ ಚಾತಿಸ್ವಪ್ನಶೀಲಸ್ಯ ಜಾಗ್ರತೋ ನೈವ ಚಾರ್ಜುನ ॥ ೧೬ ॥}
\cquote{ಅರ್ಜುನ, ಅಳತೆಗೆಟ್ಟು ತಿನ್ನುವವನಿಗೆ ಯೋಗದ ಮಾರ್ಗ ಮೈಗೂಡುವುದಿಲ್ಲ, ಎಷ್ಟು ಮಾತ್ರವೂ ಉಣ್ಣದೇ ಇರುವವನಿಗೂ ಇಲ್ಲ, ತುಂಬಾ ನಿದ್ದೆ ಮಾಡುವವನಿಗೂ ಇಲ್ಲ, ನಿದ್ದೆಗೇಡಿಗೂ ಇಲ್ಲ.}
\slcol{\Index{ಯುಕ್ತಾಹಾರವಿಹಾರಸ್ಯ} ಯುಕ್ತಚೇಷ್ಟಸ್ಯ ಕರ್ಮಸು ।\\
ಯುಕ್ತಸ್ವಪ್ನಾವಬೋಧಸ್ಯ ಯೋಗೋ ಭವತಿ ದುಃಖಹಾ ॥ ೧೭ ॥}
\cquote{ಮಿತವಾದ ಊಟ, ಮಿತವಾದ ಓಡಾಟ, ಮಿತವಾದ ಕೆಲಸ, ಮಿತವಾದ ನಿದ್ದೆ, ಮಿತವಾದ ಎಚ್ಚರ ಇರುವವರಿಗೆ ಧ್ಯಾನವು ಅಡಚಣೆಗಳನ್ನೆಲ್ಲ ದಾಟಬಲ್ಲದ್ದಾಗುತ್ತದೆ.}
\slcol{\Index{ಯದಾ ವಿನಿಯತಂ ಚಿತ್ತಮಾ}ತ್ಮನ್ಯೇವಾವತಿಷ್ಠತೇ ।\\
ನಿಃಸ್ಪೃಹಃ ಸರ್ವಕಾಮೇಭ್ಯೋ ಯುಕ್ತ ಇತ್ಯುಚ್ಯತೇ ತದಾ ॥ ೧೮ ॥}
\cquote{ಯಾವಾಗ ಒಮ್ಮುಖವಾದ ಮನಸ್ಸು ಭಗವಂತನಲ್ಲಿಯೇ ಇರುವುದೋ ಆಗ ಅವನು ಯಾವ ಬಯಕೆಯೂ ಇಲ್ಲದವನಾಗಿ ಧ್ಯಾನನಿಷ್ಠನೆನಿಸುವನು.}
\slcol{\Index{ಯಥಾ ದೀಪೋ ನಿವಾತಸ್ಥೋ} ನೇಂಗತೇ ಸೋಪಮಾ ಸ್ಮೃತಾ ।\\
ಯೋಗಿನೋ ಯತಚಿತ್ತಸ್ಯ ಯುಂಜತೋ ಯೋಗಮಾತ್ಮನಃ ॥ ೧೯ ॥}
\cquote{ಗಾಳಿ ಇಲ್ಲದ ಕಡೆಯ ದೀಪವು ಅಲ್ಲಾಡದೆ ಇರುವ ಸ್ಥಿತಿಯು ಧ್ಯಾನವನ್ನು ಅಭ್ಯಾಸ ಮಾಡುವ ಮನೋನಿಗ್ರಹವುಳ್ಳ ಸಾಧಕನಿಗೆ ಹೋಲಿಕೆ ಎಂದು ದೊಡ್ಡವರು ಹೇಳಿರುವರು.}
\slcol{\Index{ಯತ್ರೋಪರಮತೇ ಚಿತ್ತಂ} ನಿರುದ್ಧಂ ಯೋಗಸೇವಯಾ ।\\
ಯತ್ರ ಚೈವಾತ್ಮನಾತ್ಮಾನಂ ಪಶ್ಯನ್ನಾತ್ಮನಿ ತುಷ್ಯತಿ ॥ ೨೦ ॥}
\cquote{ಯೋಗಾಭ್ಯಾಸದಿಂದ ವಶಪಡಿಸಿಕೊಂಡಿರುವ ಚಿತ್ತವು ಯಾವ ಅವಸ್ಥೆಯಲ್ಲಿ ಉಪರತಿ ಉಂಟಾಗುತ್ತದೆ ಮತ್ತು ಯಾವ ಅವಸ್ಥೆಯಲ್ಲಿ ಪರಮಾತ್ಮನ ಧ್ಯಾನದಿಂದ ಶುದ್ಧವಾದ, ಸೂಕ್ಷ್ಮ ಬುದ್ಧಿಯ ಮೂಲಕ ಪರಮಾತ್ಮನನ್ನು ಸಾಕ್ಷಾತ್ಕರಿಸಿಕೊಳ್ಳುತ್ತಾ ಸಚ್ಚಿದಾನಂದಘನನಾದ ಪರಮಾತ್ಮನಲ್ಲಿಯೇ ಸಂತುಷ್ಟನಾಗಿರುತ್ತಾನೋ -}


\newpage
\begin{mananam}{\mananamfont{ಮನನ ಶ್ಲೋಕ - ೧೭}}
\small \mananamtext ದೈನಂದಿನ ಜೀವನದಲ್ಲಿ ಸಮತೋಲನದಿಂದ ಆತ್ಮಪರಿಶೀಲನೆಯನ್ನು ಸಾಧಿಸಲು ಸೂಚಿಸಲಾದ ಕ್ಷೇತ್ರಗಳು.\\
\textbf{ಆಹಾರ ಸಮತೋಲನ:} ನನ್ನ ಸಮತೋಲನಯುಕ್ತ ಆಹಾರ ಕ್ರಮ ಏನು ಒಳಗೊಂಡಿರಬೇಕು? ನಾನು ಕೆಲವೊಮ್ಮೆ ಅತಿಯಾದ ಆಹಾರ ಸೇವಿಸಿ ಮತ್ತೊಮ್ಮೆ ಅದರ ಪರಿಹಾರವಾಗಿ, ಅತಿಯಾದ ಉಪವಾಸಮಾಡಿ ಸರಿದೂಗಿಸಲು ಪ್ರಯತ್ನಿಸುತ್ತೇನೆಯೇ?\\
\textbf{ಚಟುವಟಿಕೆಯ ಮಟ್ಟಗಳು:} ದೈಹಿಕ ಚಟುವಟಿಕೆಯ ವಿಷಯದಲ್ಲಿ ನಾನು  ಸಮತೋಲದಲ್ಲಿದ್ದೇನೆಯೇ? ಸೋಮಾರಿಯಾಗಿರುವುದು, ಎಲ್ಲಾ ಕೆಲಸಗಳನ್ನು ಮುಂದೂಡುವುದು ತದನಂತರ, ಪ್ರಕ್ಷುಬ್ಧತೆ ಮತ್ತು ಅತಿಯಾದ ಚಟುವಟಿಕೆಯ ನಡುವೆ ಓಲಾಡುತ್ತಿದ್ದೇನೆಯೇ? \\
\textbf{ನಿದ್ರೆಯ ಮಾದರಿಗಳು:} ನನ್ನ ನಿದ್ರೆಯ ಅಭ್ಯಾಸಗಳು ಎಷ್ಟರ ಮಟ್ಟಿಗೆ ಸ್ಥಿರವಾಗಿವೆ? ಅಂದರೆ, ನಾನು ಕೆಲವೊಮ್ಮೆ ಅತಿಯಾಗಿ ನಿದ್ರಿಸುವುದು ಮಗದೊಮ್ಮೆ, ಸಾಕಷ್ಟು ವಿಶ್ರಾಂತಿಯನ್ನು ಪಡೆಯದಿರುವುದು ಮಾಡುತ್ತೇನೆಯೇ?
\end{mananam}
\WritingHand\enspace\textbf{ಆತ್ಮ ವಿಮರ್ಶೆ}\\
\begin{inspiration}{\mananamfont ಸ್ಫೂರ್ತಿ}
\small \mananamtext ಆಧ್ಯಾತ್ಮಿಕ ಪಥದಲ್ಲಿ ದೀರ್ಘಾವಧಿ ಯಶಸ್ಸಿಗಾಗಿ ಜೀವನವದಲ್ಲಿ  ಸಮತೋಲನವನ್ನು ಕಾಪಾಡಬೇಕು. ಸಾಧಕರು ಇಂದ್ರಿಯಲೋಲುಪರಾಗದಂತೆ ಎಚ್ಚರವಹಿಸಬೇಕು ಅಲ್ಲದೇ, ಅತಿ ಕಡಿಮೆ ಆಹಾರ ಸೇವನೆ, ಅತಿ ಕಡಿಮೆ  ನಿದ್ರೆಮಾಡುವುದು – ಇವನ್ನು ದೂರಮಾಡುವುದು ಅತ್ಯಂತ ಮುಖ್ಯ, ಇವು ದೇಹವನ್ನು ದುರ್ಬಲಗೊಳಿಸುವುವು. ಆಧ್ಯಾತ್ಮಿಕ ಸಾಧನೆಗಳಿಗೆ ಆರೋಗ್ಯಕರ ದೇಹವು ಒಂದು ನಿರ್ಣಾಯಕ ಮಾಧ್ಯಮವಾಗಿ ಕಾರ್ಯನಿರ್ವಹಿಸುತ್ತದೆ. ಭಗವದ್ಗೀತೆಯು ನಮಗೆ ಅತಿಯಾಗಿರುವುದನ್ನು ಬಿಟ್ಟು, ಮಧ್ಯಮ ಮಾರ್ಗವನ್ನು ಅಳವಡಿಸಿಕೊಳ್ಳಲು ಪ್ರತಿಪಾದಿಸುತ್ತದೆ.
\end{inspiration}
\newpage

\slcol{\Index{ಸುಖಮಾತ್ಯಂತಿಕಂ ಯತ್ತದ್ಬುದ್ಧಿ}ಗ್ರಾಹ್ಯಮತೀಂದ್ರಿಯಮ್ ।\\
ವೇತ್ತಿ ಯತ್ರ ನ ಚೈವಾಯಂ ಸ್ಥಿತಶ್ಚಲತಿ ತತ್ತ್ವತಃ ॥ ೨೧ ॥}
\cquote{ಇಂದ್ರಿಯಗಳಿಗೆ ನಿಲುಕದೆ ಬುದ್ಧಿಗೆ ಮಾತ್ರ ನಿಲುಕುವಂಥ ಪರಮ ಸುಖವನ್ನು ಅನುಭವಿಸುತ್ತಾನೋ, ಯಾವ ಸ್ಥಿತಿಯಲ್ಲಿ ನೆಲೆಗೊಂಡ ಯೋಗಿಯು ಅದರಿಂದ ವಿಚಲಿತನಾಗಲಾರನೋ,}
\slcol{\Index{ಯಂ ಲಬ್ಧ್ವಾ ಚಾಪರಂ} ಲಾಭಂ ಮನ್ಯತೇ ನಾಧಿಕಂ ತತಃ ।\\
ಯಸ್ಮಿನ್ಸ್ಥಿತೋ ನ ದುಃಖೇನ ಗುರುಣಾಪಿ ವಿಚಾಲ್ಯತೇ ॥ ೨೨ ॥}
\cquote{ಯಾವ ಸ್ಥಿತಿಯನ್ನು ಪಡೆದ ಮೇಲೆ ಬೇರೆ ಲಾಭವನ್ನು ಅದಕ್ಕಿಂತ ಹೆಚ್ಚಿನದೆಂದು ತಿಳಿಯುವುದಿಲ್ಲವೋ, ಯಾವದೇ ಸ್ಥಿತಿಯಲ್ಲಿರುವವನು ಎಂಥಹ ದೊಡ್ಡ ವಿಪತ್ತಿನಿಂದಲೂ ಕಂಗೆಡುವುದಿಲ್ಲವೋ}
\slcol{\Index{ತಂ ವಿದ್ಯಾದ್ದುಃಖಸಂಯೋಗ}ವಿಯೋಗಂ ಯೋಗಸಂಙ್ಞಿತಮ್ ।\\
ಸ ನಿಶ್ಚಯೇನ ಯೋಕ್ತವ್ಯೋ ಯೋಗೋಽನಿರ್ವಿಣ್ಣಚೇತಸಾ ॥ ೨೩ ॥}
\cquote{ದುಃಖದ ಸೋಂಕನ್ನು ದೂರಮಾಡುವ ಆ ಸ್ಥಿತಿಗೆ ಯೋಗವೆಂಬ ಹೆಸರೆಂದು ತಿಳಿ. ಆ ಯೋಗವನ್ನು ನಿರಾಶನಾಗದೆ ಭರವಸೆತಾಳಿ ಅಭ್ಯಾಸ ಮಾಡತಕ್ಕದ್ದು.}
\slcol{\Index{ಸಂಕಲ್ಪಪ್ರಭವಾನ್ಕಾಮಾಂ}ಸ್ತ್ಯಕ್ತ್ವಾ ಸರ್ವಾನಶೇಷತಃ ।\\
ಮನಸೈವೇಂದ್ರಿಯಗ್ರಾಮಂ ವಿನಿಯಮ್ಯ ಸಮಂತತಃ ॥ ೨೪ ॥\\
\Index{ಶನೈಃ ಶನೈರುಪರಮೇದ್ಬುದ್ಧ್ಯಾ} ಧೃತಿಗೃಹೀತಯಾ ।\\
ಆತ್ಮಸಂಸ್ಥಂ ಮನಃ ಕೃತ್ವಾ ನ ಕಿಂಚಿದಪಿ ಚಿಂತಯೇತ್ ॥ ೨೫ ॥}
\cquote{ಮನಸ್ಸಿನ ಗುಣಾಕಾರದಿಂದ ಹುಟ್ಟುವ ಬಯಕೆಗಳನ್ನೆಲ್ಲ ಸಂಪೂರ್ಣವಾಗಿ ಬಿಟ್ಟು ಮನಸ್ಸಿನಿಂದಲೇ ಇಂದ್ರಿಯಗಳನ್ನು ಎಲ್ಲ ಕಡೆಯಿಂದಲೂ ನಿಯಂತ್ರಿಸಿ ಧೈರ್ಯ ಪಡೆದ ಬುದ್ಧಿಬಲದಿಂದ ಮೆಲ್ಲಮೆಲ್ಲನೆ ವಿಷಯಗಳಿಂದ ನಿವೃತ್ತನಾಗಬೇಕು. ಮನಸ್ಸನ್ನು ಆತ್ಮನಲ್ಲಿ ನೆಲಗೊಳಿಸಿ ಬೇರೆ ಏನನ್ನೂ ಚಿಂತಿಸಬಾರದು.}


\newpage
\begin{mananam}{\mananamfont{ಮನನ ಶ್ಲೋಕ - ೨೨, ೨೩}}
\small \mananamtext ಯೋಗ ಮತ್ತು ಇತರ ಯೋಗಕ್ಷೇಮ ಉಂಟು ಮಾಡುವ ಸಾಧನೆಗಳಲ್ಲಿ ನಾನು ಏಕೆ ತೊಡಗಿಸಿಕೊಳ್ಳುತ್ತೇನೆ? ಜೀವನೋಪಾಯ, ಸಾಮಾಜಿಕ ಸ್ಥಾನಮಾನ, ಮನ್ನಣೆ ಮತ್ತು ಹೆಮ್ಮೆ ಇವುಗಳಿಗೋಸ್ಕರವೇ? ಅಥವಾ ದೈಹಿಕ, ಮಾನಸಿಕ ಮತ್ತು ಭಾವನಾತ್ಮಕವಾದ ನೋವು, ದುಃಖ ಮುಂತಾದವುಗಳಿಂದ ಪಾರಾಗುವುದಕ್ಕೋಸ್ಕರವೇ? ಇತರರನ್ನು ಮೆಚ್ಚಿಸಲು ಅಥವಾ ನನ್ನ ಯೋಗಕ್ಷೇಮಕ್ಕಾಗಿ  ಇದನ್ನು ಮಾಡುತ್ತಿದ್ದೇನೆಯೇ? ಈ ಮಾರ್ಗವು  ಶ್ರೇಷ್ಠ ಮಟ್ಟದ ಯೋಗಕ್ಷೇಮದತ್ತ ನನ್ನನ್ನು ಕೊಂಡೊಯ್ಯುತ್ತದೆ ಎಂದು ನನಗೆ ಮನವರಿಕೆಯಾಗಿದೆಯೇ? 
\end{mananam}
\WritingHand\enspace\textbf{ಆತ್ಮ ವಿಮರ್ಶೆ}\\
\begin{inspiration}{\mananamfont ಸ್ಫೂರ್ತಿ}
\small \mananamtext ಎಲ್ಲ ಮಾನಸಿಕ ಕಷ್ಟಗಳ ಅಂತ್ಯ ಮತ್ತು ಅಂತರ್ಗತವಾದ ಆನಂದವನ್ನು ಅನುಭವಿಸುವುದು ನಮ್ಮ ಆಧ್ಯಾತ್ಮಿಕ ಸಾಧನೆಗಳ ಮತ್ತು ಅಧ್ಯಯನದ ಗುರಿಯಾಗಿದೆ. ಈ ಉದ್ದೇಶದ ಸ್ಪಷ್ಟ ನಿರ್ಣಯವಿಲ್ಲದೆ, ಪ್ರಾಪಂಚಿಕ ಲಾಭಕ್ಕಾಗಿ ಈ ಬೋಧನೆಗಳನ್ನು ಅಭ್ಯಾಸ ಮತ್ತು ಅಧ್ಯಯನ ಮಾಡುವುದು, ಅಂತಿಮವಾಗಿ ಅತೃಪ್ತಿ ಮತ್ತು ನಿರಾಶೆಗೆ ಕಾರಣವಾಗುತ್ತದೆ. ಈ ಸಾಧನೆಗಳನ್ನು ಕೈಗೊಳ್ಳುವಾಗ ಸ್ಪಷ್ಟತೆಯ ಜೊತೆಗೇ ಧೃಡತೆಯೂ ಇದ್ದಲ್ಲಿ, ದುಃಖವು ಅಂತ್ಯಗೊಳ್ಳುತ್ತದೆ. ಇದು ಯೋಗ ನೀಡುವ ಭರವಸೆ.
\end{inspiration}
\newpage

\slcol{\Index{ಯತೋ ಯತೋ ನಿಶ್ಚರತಿ} ಮನಶ್ಚಂಚಲಮಸ್ಥಿರಮ್ ।\\
ತತಸ್ತತೋ ನಿಯಮ್ಯೈತದಾತ್ಮನ್ಯೇವ ವಶಂ ನಯೇತ್ ॥ ೨೬ ॥}
\cquote{ಒಂದು ಕಡೆ ನಿಲ್ಲದೆ ಹರಿದಾಡುವ ಮನಸ್ಸು ಯಾವ ಯಾವ ವಿಷಯಗಳ ಬೆನ್ನು ಹಿಡಿದು ಹೊರಕ್ಕೆ ಹೊರಡುತ್ತದೋ ಆಯಾ ವಿಷಯಗಳಿಂದ ಅದನ್ನು ಹಿಮ್ಮೆಟ್ಟಿಸಿ ಪರಮಾತ್ಮನಲ್ಲಿಯೇ ನೆಲೆಗೊಳಿಸಬೇಕು.}
\slcol{\Index{ಪ್ರಶಾಂತಮನಸಂ ಹ್ಯೇನಂ} ಯೋಗಿನಂ ಸುಖಮುತ್ತಮಮ್ ।\\
ಉಪೈತಿ ಶಾಂತರಜಸಂ ಬ್ರಹ್ಮಭೂತಮಕಲ್ಮಷಮ್ ॥ ೨೭ ॥}
\cquote{ಶಾಂತವಾದ ಮನಸ್ಸುಳ್ಳ,  ರಜೋಗುಣದ ದೋಷಗಳನ್ನು ದೂರೀಕರಿಸಿಕೊಂಡ, ಪಾಪಗಳನ್ನು ಕಳೆದುಕೊಂಡು ಬ್ರಹ್ಮನಲ್ಲಿ ನೆಲೆಗೊಂಡ ಇಂತಹ ಯೋಗಿ, ಉತ್ತಮವಾದ ಸುಖವನ್ನು ಪಡೆಯುತ್ತಾನೆ.}
\slcol{\Index{ಯುಂಜನ್ನೇವಂ ಸದಾತ್ಮಾನಂ} ಯೋಗೀ ವಿಗತಕಲ್ಮಷಃ ।\\
ಸುಖೇನ ಬ್ರಹ್ಮಸಂಸ್ಪರ್ಶಮತ್ಯಂತಂ ಸುಖಮಶ್ನುತೇ ॥ ೨೮ ॥}
\cquote{ಪಾಪಗಳನ್ನೆಲ್ಲ ಕಳೆದುಕೊಂಡ ಯೋಗಿಯೂ ಹೀಗೆಯೇ ಯಾವಾಗಲೂ ಮನಸ್ಸನ್ನು ಭಗವಂತನಲ್ಲಿ ನೆಲೆಗೊಳಿಸಿದಾಗ ಅನಾಯಾಸವಾಗಿ ಬ್ರಹ್ಮನೊಡನೆ ಕೂಡುವುದೆಂಬ ಶಾಶ್ವತವಾದ ಸುಖವನ್ನು ಅನುಭವಿಸುವನು.}
\slcol{\Index{ಸರ್ವಭೂತಸ್ಥಮಾತ್ಮಾನಂ} ಸರ್ವಭೂತಾನಿ ಚಾತ್ಮನಿ ।\\
ಈಕ್ಷತೇ ಯೋಗಯುಕ್ತಾತ್ಮಾ ಸರ್ವತ್ರ ಸಮದರ್ಶನಃ ॥ ೨೯ ॥}
\cquote{ಈ ಬಗೆಯ ಯೋಗದಿಂದ ಮಾಗಿದ ಮನಸ್ಸುಳ್ಳವನು ಎಲ್ಲ ವಿಷಯಗಳನ್ನೂ ಒಂದೇ ರೀತಿಯಲ್ಲಿ ಕಾಣುತ್ತಾ,ಭಗವಂತನು ಎಲ್ಲ ಪ್ರಾಣಿಗಳಲ್ಲಿ ತುಂಬಿರುವಂತೆಯೂ, ಎಲ್ಲ ಪ್ರಾಣಿಗಳೂ ಭಗವಂತನನ್ನು ಆಶ್ರಯಿಸಿಕೊಂಡಿರುವಂತೆಯೂ ಕಾಣುತ್ತಾನೆ.}
\slcol{\Index{ಯೋ ಮಾಂ ಪಶ್ಯತಿ ಸರ್ವತ್ರ} ಸರ್ವಂ ಚ ಮಯಿ ಪಶ್ಯತಿ ।\\
ತಸ್ಯಾಹಂ ನ ಪ್ರಣಶ್ಯಾಮಿ ಸ ಚ ಮೇ ನ ಪ್ರಣಶ್ಯತಿ ॥ ೩೦ ॥}
\cquote{ಯಾವನು ಹೀಗೆ ನನ್ನನ್ನು ಎಲ್ಲಡೆಯೂ ಕಾಣುತ್ತಾನೋ ಮತ್ತು ಎಲ್ಲವನ್ನೂ ನನ್ನಲ್ಲಿ ಕಾಣುತ್ತಾನೋ ಅವನನ್ನು ನಾನು ಕೈ ಬಿಡಲಾರೆ.ಅವನು ನನ್ನನ್ನು ಮರೆಯಲಾರ.}

\newpage
\begin{mananam}{\mananamfont{ಮನನ ಶ್ಲೋಕ - ೨೫, ೨೬}}
\small \mananamtext ನನ್ನ ಧ್ಯಾನಸಾಧನೆಗಳಲ್ಲಿ, ಕೇಂದ್ರಿತ ವಸ್ತುವಿನ ಬಗ್ಗೆ ಸ್ಪಷ್ಟವಾಗಿ ನಿರ್ಧರಿಸಿದ್ದೇನೆಯೇ?  ಗೊಂದಲಗಳು ಉಂಟಾದಾಗಲೂ ನನಗೆ, ನನ್ನ ಮನಸ್ಸಿನ ಉದ್ದೇಶದ ಬಗ್ಗೆ ಸತತ ಅರಿವು ಇರುತ್ತದೆಯೇ? ನಂತರ,  ಗೊಂದಲಗೊಂಡ ಮನಸ್ಸಿಗೆ ಧ್ಯಾನದ ವಸ್ತುವಿನ ಮೇಲೆ ಪುನಃ ಕೇಂದ್ರೀಕರಿಸಲು ಮೃದುವಾಗಿ ನೆನಪಿಸುತ್ತೇನೆಯೇ? ಧ್ಯಾನ ಮಾಡುವಾಗ ಮನಸ್ಸು ಚಂಚಲವಾದರೆ ನಾನು ತಾಳ್ಮೆಯಿಂದ, ಮನಸ್ಸು ಪುನಃ ಶಾಂತವಾಗುವವರೆಗೂ ಕಾಯುತ್ತೇನೆಯೇ? ಅಥವಾ ಧ್ಯಾನಕ್ಕೆ ಮೀಸಲಿಟ್ಟ ಅವಧಿ ಪೂರ್ಣವಾಗುವ ಮೊದಲೇ ಧ್ಯಾನವನ್ನು  ಬಿಟ್ಟು ಬಿಡುತ್ತೇನೆಯೇ?
ಧ್ಯಾನದ ಉಪರಾಂತವೇ,ಮತ್ತಷ್ಟು ಶಾಂತವಾದ ಮನೋಸ್ಥಿತಿಯನ್ನು ಸಾಧಿಸಿರುವಂತಹ – ಅಂದರೆ, ಮನಸ್ಸಿನ ತರಬೇತಿಯಿಂದಾಗುವ ಇಂತಹ ತಕ್ಷಣದ ಪ್ರಯೋಜನಗಳ ಬಗ್ಗೆ ನನ್ನ ಗಮನವಿದೆಯೇ? ದಿನದಲ್ಲಿ, ನಂತರದ ಸಮಯಗಳಲ್ಲೂ ಈ ಪ್ರಯೋಜನಗಳನ್ನು ಅನುಭವಿಸಬಲ್ಲೆನೇ?
\end{mananam}
\WritingHand\enspace\textbf{ಆತ್ಮ ವಿಮರ್ಶೆ}\\
\begin{inspiration}{\mananamfont ಸ್ಫೂರ್ತಿ}
\small \mananamtext ಎಲ್ಲಾ ಧ್ಯಾನದ ತಂತ್ರಗಳೂ ಸರಳವಾದ ಸೂತ್ರವನ್ನು ಒಳಗೊಂಡಿರುತ್ತವೆ. ಧ್ಯಾನ ಮಾಡುವಾಗ ಮನಸ್ಸು ವಿಚಲಿತವಾದರೆ, ನಮ್ಮ ಗಮನವನ್ನು ಪುನಃ ಕೇಂದ್ರೀಕರಿಸಲು  ಮನಸ್ಸಿಗೆ ಮೃದುವಾಗಿ ನೆನಪಿಸಬೇಕು; ಇಂದ್ರಿಯಗಳು ಶಾಂತವಾದಾಗ ಸಂವೇದನೆಗಳು ಏಳುವುದಿಲ್ಲ.ಈ ಧ್ಯಾನಾಭ್ಯಾಸವು, ನಮ್ಮ ಪೂರ್ವಿಕರ ವಿವೇಕದೊಂದಿಗೆ ಆಧುನಿಕ ವಿಜ್ಞಾನದ, ಗಮನ ಮತ್ತು ನಿಯಂತ್ರಣದ ಸಮ್ಮಿಳಿತವೇ ಆಗಿದೆ.  ತಾಳ್ಮೆಯಿಂದ ಕೂಡಿದ ಅವಿರತ ಯತ್ನವೇ ಈ ಸೂತ್ರದ ಸಾರ, ಇದರಿಂದ ಖಚಿತವಾದ ಯಶಸ್ಸಿನ ಹಾದಿ ಸಾಧಕನಿಗೆ ದೊರಕುತ್ತದೆ.
\end{inspiration}
\newpage

\newpage
\begin{mananam}{\mananamfont{ಮನನ ಶ್ಲೋಕ - ೩೦}}
\small \mananamtext ನನ್ನ ದಿನನಿತ್ಯ ಜೀವನದಲ್ಲಿ ದೇವರ  ಪಾತ್ರವೇನು? ಅವನಿಗೆ ಪ್ರಾಮುಖ್ಯತೆ ಕೊಡುತ್ತೇನೆಯೇ? ಎಲ್ಲಾ ಅನುಕೂಲ, ಅನಾನುಕೂಲ ಪರಿಸ್ಥಿತಿಗಳಲ್ಲೂ ಕೂಡ ನಾನು ದೇವರನ್ನು ನೆನೆದು ಅವನ ಆವಾಹನೆ ಮಾಡಬಲ್ಲೆನೇ?\\
ದೇವರ ಅಸ್ಥಿತ್ವವನ್ನು ನೆನಪಿಸುವ ವಿಗ್ರಹ, ಪಟ ಅಥವಾ ಯಾವುದೇ ಒಂದು ಮೂರ್ತ ಸ್ವರೂಪವಿಲ್ಲದಿದ್ದಲ್ಲಿ, ದೇವರೊಂದಿಗೆ ಸಂಪರ್ಕ ಕಡಿತದ ಭಾವನೆಯನ್ನು ಅನುಭವಿಸುತ್ತೇನೆಯೇ?\\
ಯಾವುದೇ ಬಾಹ್ಯ ಆಧಾರವಿಲ್ಲದಿದ್ದಾಗ, ನಾನು ದೇವರ ಯಾವ ಪರಿಕಲ್ಪನೆಯ ಅಳವಡಿಕೆಯಿಂದ ಅವನೊಂದಿಗೆ  ನಿರಂತರ ಸಂಪರ್ಕ ಮತ್ತು ಆಶ್ರಯ ಪಡೆಯಬಹುದು? ಯಾವ ರೀತಿ ದೇವರ ಇರುವಿಕೆಯನ್ನು ಅನುಭವಿಸಬಹುದು?
\end{mananam}
\WritingHand\enspace\textbf{ಆತ್ಮ ವಿಮರ್ಶೆ}\\
\begin{inspiration}{\mananamfont ಸ್ಫೂರ್ತಿ}
\small \mananamtext ದೇವರ ಸಾನಿಧ್ಯವನ್ನು ಸತತವಾಗಿ ಅನುಭವಿಸುವುದರಿಂದ ಮನಸ್ಸು ಉತ್ಕೃಷ್ಟವಾಗುತ್ತದೆ. ಇದು ಒಂಟಿತನದ ಭಾವನೆ ಮತ್ತು  ಅಸಹಾಯಕ ಭಾವನೆಗಳನ್ನು ಹೊರ ಹಾಕುತ್ತಾ, ಈ ಭೌತಿಕ ಜಗತ್ತನ್ನು ಒಂದು ದೈವಿಕ ಜಗತ್ತಾಗಿ ಪರಿವರ್ತಿಸುತ್ತದೆ. ಸಮರ್ಪಣಾ ಭಾವವಿರುವ ಆತ್ಮಕ್ಕೆ ಜೀವನದ ಪ್ರತಿಯೊಂದು ಅಂಗದಲ್ಲಿಯೂ ಭಗವಂತನ  ಸಾನಿಧ್ಯದ ಸುಖಕ್ಕಿಂತ, ಸುಖದ ಅನುಭವ ಬೇರಾವುದೂ ಇಲ್ಲ.
\end{inspiration}
\newpage

\slcol{\Index{ಸರ್ವಭೂತಸ್ಥಿತಂ ಯೋ ಮಾಂ} ಭಜತ್ಯೇಕತ್ವಮಾಸ್ಥಿತಃ ।\\
ಸರ್ವಥಾ ವರ್ತಮಾನೋಽಪಿ ಸ ಯೋಗೀ ಮಯಿ ವರ್ತತೇ ॥ ೩೧ ॥}
\cquote{ಎಲ್ಲೆಡೆಯೂ ಇರುವ ಭಗವಂತನೊಬ್ಬನೇ ಎಂದು ತಿಳಿದು ನನ್ನನ್ನು ಭಜಿಸುವ ಜ್ಞಾನಿಯು ಹೇಗಿದ್ದರೂ ನನ್ನಲ್ಲಿರುತ್ತಾನೆ.}
\slcol{ಆತ್ಮೌಪಮ್ಯೇನ ಸರ್ವತ್ರ ಸಮಂ ಪಶ್ಯತಿ ಯೋಽರ್ಜುನ ।\\
ಸುಖಂ ವಾ ಯದಿ ವಾ ದುಃಖಂ ಸ ಯೋಗೀ ಪರಮೋ ಮತಃ ॥ ೩೨ ॥}
\cquote{ಅರ್ಜುನ, ಯಾವನು ಎಲ್ಲ ಪ್ರಾಣಿಗಳಲ್ಲಿಯೂ ಸುಖವಾಗಲೀ ದುಃಖವಾಗಲೀ ತನ್ನಂತೆಯೆ ಎಂದು ತಿಳಿಯುತ್ತಾನೋ ಆ ಜ್ಞಾನಿಯು ಎಲ್ಲರಿಗಿಂತ ಮೇಲೆನಿಸುವನು.}
\slcol{ಅರ್ಜುನ ಉವಾಚ ।\\
\Index{ಯೋಽಯಂ ಯೋಗಸ್ತ್ವಯಾ} ಪ್ರೋಕ್ತಃ ಸಾಮ್ಯೇನ ಮಧುಸೂದನ ।\\
ಏತಸ್ಯಾಹಂ ನ ಪಶ್ಯಾಮಿ ಚಂಚಲತ್ವಾತ್ಸ್ಥಿತಿಂ ಸ್ಥಿರಾಮ್ ॥ ೩೩ ॥}
\cquote{ಅರ್ಜುನನು ಹೇಳಿದನು,\\
ಕೃಷ್ಣ, ಸಮದೃಷ್ಟಿಯಿಂದ ನೋಡಬೇಕೆಂದು ನೀನು ಈ ಮನಸ್ಸನ್ನು ಬಿಗಿ ಹಿಡಿಯುವ ಸಾಧನವನ್ನು ಹೇಳಿದೆ ಅಷ್ಟೇ. ಮನಸ್ಸು ಚಂಚಲವಾದ್ದರಿಂದ ಅದು ಹೀಗೆ ಒಂದು ಕಡೆ ನಿಲ್ಲುವಂತೆ ನನಗೆ ಕಾಣುವುದಿಲ್ಲ.}
\slcol{\Index{ಚಂಚಲಂ ಹಿ ಮನಃ} ಕೃಷ್ಣ ಪ್ರಮಾಥಿ ಬಲವದ್ದೃಢಮ್ ।\\
ತಸ್ಯಾಹಂ ನಿಗ್ರಹಂ ಮನ್ಯೇ ವಾಯೋರಿವ ಸುದುಷ್ಕರಮ್ ॥ ೩೪ ॥}
\cquote{ಕೃಷ್ಣ, ಮನಸ್ಸು ಹರಿದಾಡುವಂಥಾದ್ದು ನಮ್ಮನ್ನು ಘಾಸಿಗೊಳಿಸಿ ತನ್ನ ಪಟ್ಟನ್ನು ಬಿಡದೆ ಸಾಧಿಸಬಲ್ಲ ಬಲಶಾಲಿ. ಅದನ್ನು ಬಿಗಿಯುವುದು ಎಂದರೆ ಗಾಳಿಯನ್ನು ಕಟ್ಟಿ ಹಾಕಿದಂತೆ ಎಂದು ನನಗನ್ನಿಸುತ್ತದೆ.}
\slcol{ಶ್ರೀ ಭಗವಾನುವಾಚ ।\\
\Index{ಅಸಂಶಯಂ ಮಹಾಬಾಹೋ} ಮನೋ ದುರ್ನಿಗ್ರಹಂ ಚಲಮ್ ।\\
ಅಭ್ಯಾಸೇನ ತು ಕೌಂತೇಯ ವೈರಾಗ್ಯೇಣ ಚ ಗೃಹ್ಯತೇ ॥ ೩೫ ॥}
\cquote{ಶ್ರೀ ಭಗವಂತನು ಹೇಳಿದನು, ಅರ್ಜುನ ಮನಸ್ಸು ತಡೆಯುವುದಕ್ಕೆ ಆಗದು ಎಂಬುದು ಮತ್ತು ಹರಿದಾಟದ ಸ್ವಭಾವವೆಂಬುದು ನಿಶ್ಚಯ. ಆದರೆ ಅರ್ಜುನ, ಪ್ರಯತ್ನದಿಂದಲೂ, ವೈರಾಗ್ಯದಿಂದಲೂ ಅದನ್ನು ಬಿಗಿ ಹಿಡಿಯುವುದಕ್ಕಾಗುತ್ತದೆ.}

\newpage
\begin{mananam}{\mananamfont{ಮನನ ಶ್ಲೋಕ - ೩೧, ೩೨}}
\small \mananamtext ನನ್ನ ಸುತ್ತಲಿರುವ ಪ್ರತಿಯೊಬ್ಬರಲ್ಲೂ ನನ್ನದೇ ಆತ್ಮತತ್ವವಿರುವುದನ್ನು ಗುರುತಿಸಿ ಅನುಭವಿಸಬಲ್ಲೆನೇ? ವ್ಯಾವಹಾರಿಕ ಉದ್ದೇಶಗಳ ಹೊರತಾಗಿ ಜನರನ್ನು, ಅವರ ಪ್ರಾಪಂಚಿಕ ಪಾತ್ರಕ್ಕೆ ಅಥವಾ ಅವರ ಸಾಧನೆಗಳ ಆಧಾರದ ಮೇಲೆ ಗೌರವಿಸುತ್ತೇನೆಯೇ? ಅಥವಾ ಅವರಲ್ಲಿ ಅಂತರ್ಯಾಮಿ ಆಗಿರುವ ದೈವಸತ್ವದ ಆಧಾರದ ಮೇಲೆ ಗೌರವಿಸುತ್ತೇನೆಯೇ?\\
ನಾನು, ನನಗಿರುವ ಬಾಹ್ಯಪಾತ್ರಗಳು, ಕೌಶಲ್ಯಗಳು ಮತ್ತು ವ್ಯಕ್ತಿತ್ವದ ಹಣೆಪಟ್ಟಿಗಳನ್ನು ಮೀರಿ, ನನ್ನ ಆತ್ಮತತ್ವದ ಜೊತೆ ಗುರುತಿಸಿಕೊಳ್ಳಬಲ್ಲೆನೇ? ನನ್ನಲ್ಲೂ ಮತ್ತು ಎಲ್ಲರಲ್ಲೂ ಅಸ್ತಿತ್ವದಲ್ಲಿರುವ ಈ ದೈವಸತ್ವದ ಅರಿವಿನಿಂದಾಗಿ, ನನ್ನ ದೈನಂದಿನ ಜೀವನದಲ್ಲಿ ನಾನು ಇತರರೊಡನೆ ವ್ಯವಹರಿಸುವ ರೀತಿಯು ಪರಿವರ್ತಿತವಾಗುತ್ತದೆಯೇ?
\end{mananam}
\WritingHand\enspace\textbf{ಆತ್ಮ ವಿಮರ್ಶೆ}\\
\begin{inspiration}{\mananamfont ಸ್ಫೂರ್ತಿ}
\small \mananamtext ಪಾರಮಾರ್ಥಿಕ ವಿವೇಕದ ಹಾದಿಯಲ್ಲಿರುವವರಿಗೆ, ‘ನಮ್ಮಲ್ಲಿರುವ ದೈವಿಕ ಚೈತನ್ಯದ -ಅಂದರೆ ಆತ್ಮದ – ಸಾರವೇ, ಸಕರಲರಲ್ಲಿಯೂ ಇರುವ ಸಾರ’, ಎಂದು ಗುರುತಿಸುವುದೇ ಅತ್ಯಮೂಲ್ಯ  ಅಭ್ಯಾಸವಾಗಿದೆ; ಇಂತಹ ವಿವೇಕದಿಂದ, ಯಾರೂ ಶತ್ರುಗಳಿಲ್ಲವೆಂಬ ಅರಿವಾಗಿ ಯಾವದೇ ಭಯವು ಕರಗುತ್ತದೆ – ಏಕೆಂದರೆ, ಇಡೀ ಪ್ರಪಂಚವೇ ನಮ್ಮ ಅಸ್ತಿತ್ವದ ವಿಸ್ತರಣೆಯಲ್ಲದೇ ಬೇರೇನಲ್ಲ!
\end{inspiration}
\newpage

\newpage
\begin{mananam}{\mananamfont{ಮನನ ಶ್ಲೋಕ - ೩೪, ೩೫}}
\small \mananamtext ನನ್ನ ಮನಸ್ಸಿನ ಚಂಚಲ  ಸ್ವಭಾವದ ಬಗ್ಗೆ ನನಗೆ ಅರಿವಿದೆಯೇ? ಅವುಗಳ ಸೆಳೆತಕ್ಕೆ ಒಳಗಾಗುವ ಬದಲಾಗಿ, ನಾನು ಆಗಿಂದಾಗ್ಗೆ ಮನಸ್ಸಿನ ಹರಿದಾಟವನ್ನು ನಿಲ್ಲಿಸಿ, ಅದರ ಪ್ರಚೋದನೆಗಳು ಮತ್ತು ಪ್ರಕ್ಷೇಪಣಗಳ ಬಗ್ಗೆ  ತಿಳಿಯುತ್ತೇನೆಯೇ? ದಿನನಿತ್ಯದ ಚಟುವಟಿಕೆಯ ಮಧ್ಯದಲ್ಲಿ ನಾನು ಆಗಾಗ್ಗೆ, ಪ್ರಜ್ಞಾಪೂರ್ವಕವಾಗಿ,  ಮನಸ್ಸಿನ ಏರಿಳಿತಗಳನ್ನು ತಡೆದು ವಿರಾಮಕೊಡುವ ಅಭ್ಯಾಸವನ್ನು ದಿನದಲ್ಲಿ ಮಾಡಿದ್ದೇನೆಯೇ?ಮನಸ್ಸನ್ನು ಶಾಂತಗೊಳಿಸುವಲ್ಲಿ ನನ್ನ ನಿಯಮಿತ ಅಭ್ಯಾಸ (ಯೋಗ, ಧ್ಯಾನ ಇತ್ಯಾದಿ) ಎಷ್ಟು ಪರಿಣಾಮಕಾರಿಯಾಗಿದೆ? \\
ನಿರಾಸಕ್ತಿಯ  ಬಗ್ಗೆ ನನ್ನ ತಿಳುವಳಿಕೆ ಏನು? ಇಂದ್ರಿಯಗಳ, ಮನಸ್ಸಿನ ನಿಗ್ರಹಮಾಡಲು ಹಾಗೂ ಚಿತ್ತ ಚಾoಚಲ್ಯ ತಪ್ಪಿಸಲು ಮತ್ತು ನಿರ್ದಿಷ್ಟ ಚಟುವಟಿಕೆಗಳಲ್ಲಿ ಗಮನವನ್ನು ಕೇಂದ್ರೀಕರಿಸಲು ನಾನು ನಿರಾಸಕ್ತಿಯನ್ನು( ಅಂದರೆ, ಅನುಪಯುಕ್ತವಾದ ಕೆಲಸ, ಚಿಂತೆ, ಭಾವನೆಗಳಲ್ಲಿ) ಹೇಗೆ ಅನ್ವಯಿಸಬಹುದು?
\end{mananam}
\WritingHand\enspace\textbf{ಆತ್ಮ ವಿಮರ್ಶೆ}\\
\begin{inspiration}{\mananamfont ಸ್ಫೂರ್ತಿ}
\small \mananamtext ಮನಸ್ಸಿನ ಅಂತರ್ಗತ ಸ್ವಭಾವವೇ ಪ್ರಕ್ಷುಬ್ಧತೆ; ನಾವು ಇದನ್ನು ಪ್ರಜ್ಞಾಪೂರ್ವಕವಾಗಿ  ಒಪ್ಪಿಕೊಂಡಾಗ, ವಿಶೇಷವಾಗಿ, ಅದರ ಪ್ರಕ್ಷುಬ್ದ ಕ್ಷಣಗಳಲ್ಲಿ ನಾವು ಅದರಿಂದ ದೂರವಿರುತ್ತೇವೆ. ಮನಸ್ಸನ್ನು ನಿಯಂತ್ರಿಸಲು ಇಚ್ಛಿಸುವವರಿಗೆ -  ಅನಗತ್ಯ ವಿಷಯ, ಭಾವನೆ, ಚಿಂತೆಗಳೆಡೆಗೆ ತೋರುವ ನಿರಾಸಕ್ತಿ ಮತ್ತು ನಿರ್ಲಿಪ್ತತೆಯು ಅಮೂಲ್ಯವಾದ ಸಾಧನಗಳಾಗಿ ಕಾರ್ಯನಿರ್ವಹಿಸುತ್ತವೆ. ತನ್ನ ಅಧ್ಯನಗಳ ಮೇಲೆ ಗಮನ ಕೇಂದ್ರೀಕರಿಸುವ ಗುರಿಹೊಂದಿದ ಒಬ್ಬ ವಿದ್ಯಾರ್ಥಿಯೂ ಸಹ, ತಾತ್ಕಾಲಿಕವಾಗಿಯಾದರೂ, ಇತರ ಮನೋರಂಜನೆಗಳನ್ನೂ ವಿಚಲಿತಗೊಳಿಸುವ ಆಲೋಚನೆ – ಕೆಲಸಗಳನ್ನೂ ಬದಿಗಿರಿಸಲೇಬೇಕು.\\
ಶ್ರದ್ಧಾಪೂರ್ವಕ ಅಭ್ಯಾಸದಿಂದ ಒಬ್ಬನು ಯೋಗಿಯಾಗುತ್ತಾನೆ- ತನ್ನ ಮನದ ಮೇಲೆಯೂ ಪ್ರಪಂಚದ ಮೇಲೆಯೂ ಪ್ರಭುತ್ವ ಸಾಧಿಸಿರುವವನೇ ಯೋಗಿ. 
\end{inspiration}
\newpage

\slcol{\Index{ಅಸಂಯತಾತ್ಮನಾ ಯೋಗೋ} ದುಷ್ಪ್ರಾಪ ಇತಿ ಮೇ ಮತಿಃ ।\\
ವಶ್ಯಾತ್ಮನಾ ತು ಯತತಾ ಶಕ್ಯೋಽವಾಪ್ತುಮುಪಾಯತಃ ॥ ೩೬ ॥}
\cquote{ಮನಸ್ಸನ್ನು ಬಿಗಿಹಿಡಿಯಲಾರದವನಿಗೆ ಧ್ಯಾನವು ದಕ್ಕುವ  ಮಾತಲ್ಲವೆಂದು ನನ್ನ ಅಭಿಪ್ರಾಯ. ಪ್ರಯತ್ನದಿಂದ ಮನಸ್ಸನ್ನು ಹಿಡಿತಕ್ಕೆ ತಂದುಕೊಂಡವನಿಗೆ ಉಪಾಯದಿಂದ ಅದನ್ನು ಪಡೆಯುವುದಕ್ಕಾಗುತ್ತದೆ.}
\slcol{ಅರ್ಜುನ ಉವಾಚ ।\\
\Index{ಅಯತಿಃ ಶ್ರದ್ಧಯೋಪೇತೋ} ಯೋಗಾಚ್ಚಲಿತಮಾನಸಃ ।\\
ಅಪ್ರಾಪ್ಯ ಯೋಗಸಂಸಿದ್ಧಿಂ ಕಾಂ ಗತಿಂ ಕೃಷ್ಣ ಗಚ್ಛತಿ ॥ ೩೭ ॥}
\cquote{ಅರ್ಜುನನ್ನು ಹೇಳಿದನು,\\
ಕೃಷ್ಣ, ಶಾಸ್ತ್ರದಲ್ಲಿಯೂ ಗುರುವಿನಲ್ಲಿಯೂ ಶ್ರದ್ಧೆಯುಳ್ಳವನು ಪ್ರಯತ್ನ ಮಾಡಲಾರದೇ ಧ್ಯಾನದಿಂದ ಕದಲಿದ ಮನಸುಳ್ಳವನಾದರೆ ಯೋಗದ ಫಲವಾದ ಜ್ಞಾನವನ್ನು ಪಡೆಯದೆ ಅವನು ಮತ್ತೆ ಯಾವ ಗತಿಯನ್ನು ಪಡೆಯುತ್ತಾನೆ?}
\slcol{\Index{ಕಚ್ಚಿನ್ನೋಭಯವಿಭ್ರಷ್ಟ}ಶ್ಛಿನ್ನಾಭ್ರಮಿವ ನಶ್ಯತಿ ।\\
ಅಪ್ರತಿಷ್ಠೋ ಮಹಾಬಾಹೋ ವಿಮೂಢೋ ಬ್ರಹ್ಮಣಃ ಪಥಿ ॥ ೩೮ ॥}
\cquote{ಗುರಿತಪ್ಪಿ ಮೋಕ್ಷ ಮಾರ್ಗದಿಂದ ಜಾರಿದ ಯೋಗ ಭ್ರಷ್ಟನು ಎರಡಕ್ಕೂ ತಪ್ಪಿದವನಾಗಿ ಭಿನ್ನ-ಭಿನ್ನವಾದ ಮೋಡದಂತೆ ಹಾಳಾಗಿ ಹೋಗುವುದಿಲ್ಲವೇ?}
\slcol{\Index{ಏತನ್ಮೇ ಸಂಶಯಂ ಕೃಷ್ಣ} ಛೇತ್ತುಮರ್ಹಸ್ಯಶೇಷತಃ ।\\
ತ್ವದನ್ಯಃ ಸಂಶಯಸ್ಯಾಸ್ಯ ಛೇತ್ತಾ ನ ಹ್ಯುಪಪದ್ಯತೇ ॥ ೩೯ ॥}
\cquote{ಕೃಷ್ಣ, ನನ್ನ ಈ ಸಂಶಯವನ್ನು ಪೂರ್ಣವಾಗಿ ನೀನು ಕಳೆಯಬೇಕು, ಏಕೆಂದರೆ ನಿನ್ನ ಹೊರತು ಯಾರೂ ಈ ಸಂದೇಹವನ್ನು ಪರಿಹರಿಸಲಾರರು.}


\newpage
\begin{mananam}{\mananamfont{ಮನನ ಶ್ಲೋಕ - ೩೭, ೩೮}}
\small \mananamtext ನನ್ನ ಮನಸ್ಸಿನ ಮೇಲೆ ನಿಯಂತ್ರಣ ಹೊಂದಲು ನನಗೆ ಪ್ರೇರೇಪಣೆ ಕೊಡುವುದು ಯಾವುದು? ಪ್ರಕ್ಷುಬ್ಧವಾದ ಮತ್ತು ಅಸಂತೋಷಗೊಂಡ ಮನಸ್ಸೇ  ಜೀವನದ ಎಲ್ಲಾ ಸಮಸ್ಯೆಗಳಿಗೂ ಮೂಲ ಕಾರಣ ಎಂಬ ಅರಿವು ನನಗಿದೆಯೇ? \\
ನನಗಿರುವ ದೌರ್ಬಲ್ಯಗಳು, ಅಹಿತಕರ ಲಕ್ಷಣಗಳನ್ನು (ಕೀಳು ಗುಣಗಳನ್ನು) ಗುರುತಿಸಬಲ್ಲೆನೇ? ಮತ್ತು ಅವುಗಳನ್ನು ಜಯಿಸಲು, ಸಕ್ರಿಯವಾಗಿ ಮತ್ತು ಸoಪೂರ್ಣವಾಗಿ ಪ್ರಯತ್ನಿಸದೆಯೇ ಅಕಾಲಿಕವಾಗಿ (ಅಂದರೆ,ಮಧ್ಯದಲ್ಲಿಯೇ) ಬಿಟ್ಟು ಬಿಡುತ್ತೇನೆಯೇ?\\
ಅಪೇಕ್ಷಿತವಾದ, ಉತ್ತಮ ಮತ್ತು ಅದಮ್ಯವಾದ ಆಂತರಿಕ ಸ್ಥಿತಿಯನ್ನು (ದೌರ್ಬಲ್ಯತೆ ಮತ್ತು ನಿಮ್ನ ಗುಣಗಳನ್ನು ಅಂತ್ಯ ಮಾಡಿ, ಪಾರಾಮಾರ್ಥಿಕದೆಡೆ ಕೊಂಡೊಯ್ಯುವ ಸ್ಥಿತಿ) ಸಾಧಿಸುವುದು ಬಹು ದುಸ್ತರ ಎಂದು ಭಯಪಡುತ್ತೇನೆಯೇ? ಅದನ್ನು ಸಾಧಿಸಲು ಒಂದಕ್ಕಿಂತ ಹೆಚ್ಚು ಜೀವತಾವಧಿಯ ಅವಶ್ಯಕತೆ ಇರಬುಹುದೆಂದು ಚಿಂತಿಸುತ್ತೇನೆಯೇ?
\end{mananam}
\WritingHand\enspace\textbf{ಆತ್ಮ ವಿಮರ್ಶೆ}\\
\begin{inspiration}{\mananamfont ಸ್ಫೂರ್ತಿ}
\small \mananamtext ಜೀವನದಲ್ಲಿ ಎಲ್ಲಾ ರೀತಿಯ ಪ್ರತಿಕೂಲ ಸನ್ನಿವೇಶಗಳನ್ನೂ ಅಡೆತಡೆಗಳನ್ನೂ ಮೀರಿ ಜೀವನ ನಡೆಸಿದಂತಹ ಋಷಿಮುನಿಗಳ ಜೀವನ  ಸ್ಪೂರ್ತಿದಾಯಕವಾಗಿದೆ. ಆದಾಗ್ಯೂ, ಆಧ್ಯಾತ್ಮಿಕ ಮಾರ್ಗವನ್ನು ಪ್ರಾರಂಭಿಸುವವರಿಗೆ ಇದು (ಋಷಿಮುನಿಗಳoತೆ ಜೀವಿಸುವುದು) ಬಹಳ ಕಠಿಣವೆನಿಸಬಹುದು. ಸ್ವಯಂ ಮೇಲಿನ ನಂಬಿಕೆ ಮತ್ತು ಶ್ರದ್ಧೆಯ ಕೊರತೆಯು ಕೆಲವರಿಗೆ, ತಮ್ಮ ಉದಾತ್ತ ಗುರಿಗಳನ್ನು ತಲುಪುವ ಪ್ರಯತ್ನ ತ್ಯಜಿಸಲು ಕಾರಣವಾಗುತ್ತದೆ. ಆದರೂ, ಸಂಭವನೀಯ ಆಂತರಿಕ ಸ್ವಾತಂತ್ರ್ಯವನ್ನು ಕಿಂಚಿತ್ತಾದರೂ ಅನುಭವಿಸಿದವರಿಗೆ ಕೇವಲ ಲೌಕಿಕ ಸಂತೋಷದಿಂದ ತೃಪ್ತರಾಗಲು ಸಾಧ್ಯವಿಲ್ಲ.
\end{inspiration}
\newpage


\slcol{ಶ್ರೀಭಗವಾನುವಾಚ ।\\
\Index{ಪಾರ್ಥ ನೈವೇಹ ನಾಮುತ್ರ} ವಿನಾಶಸ್ತಸ್ಯ ವಿದ್ಯತೇ ।\\
ನ ಹಿ ಕಲ್ಯಾಣಕೃತ್ಕಶ್ಚಿದ್ದುರ್ಗತಿಂ ತಾತ ಗಚ್ಛತಿ ॥ ೪೦ ॥}
\cquote{ಭಗವಂತನು ಹೀಗೆಂದನು, ಹೇ ಪಾರ್ಥ, ಯೋಗಭ್ರಷ್ಟನಿಗೆ  ಇಹಪರಗಳಲ್ಲಿ ಎಲ್ಲಿಯೂ ಒಳಿತಿಲ್ಲ. ಅಯ್ಯ! ಒಳ್ಳೆಯದನ್ನು  ಮಾಡಿದವನು ಯಾವಾಗಲೂ ಕೆಡುವುದಿಲ್ಲ.\\}
\slcol{\Index{ಪ್ರಾಪ್ಯ ಪುಣ್ಯಕೃತಾಂ ಲೋಕಾ}ನುಷಿತ್ವಾ ಶಾಶ್ವತೀಃ ಸಮಾಃ ।\\
ಶುಚೀನಾಂ ಶ್ರೀಮತಾಂ ಗೇಹೇ ಯೋಗಭ್ರಷ್ಟೋಽಭಿಜಾಯತೇ ॥ ೪೧ ॥}
\cquote{ಅಂತಹ ಯೋಗ ಭ್ರಷ್ಟನು ಪುಣ್ಯವಂತರು ಪಡೆಯುವ ಲೋಕಗಳನ್ನು ಪಡೆದು ಅನೇಕ ವರ್ಷಗಳು ಅಲ್ಲಿದ್ದು ಧರ್ಮಶ್ರದ್ಧೆಯುಳ್ಳ ಭಾಗ್ಯವಂತರ ಮನೆಯಲ್ಲಿ ಹುಟ್ಟುತ್ತಾನೆ. }
\slcol{\Index{ಯೋಗಿನಾಮೇವ ಕುಲೇ} ಭವತಿ ಧೀಮತಾಮ್ ।\\
ಏತದ್ಧಿ ದುರ್ಲಭತರಂ ಲೋಕೇ ಜನ್ಮ ಯದೀದೃಶಮ್ ॥ ೪೨ ॥}
\cquote{ಅಥವಾ ಅವನು ಜ್ಞಾನಿಗಳಾದ ಯೋಗಾಭ್ಯಾಸಿಗಳ ಕುಲದಲ್ಲಿಯೇ ಹುಟ್ಟುತ್ತಾನೆ. ಜಗತ್ತಿನಲ್ಲಿ ಈ ಬಗೆಯ ಹುಟ್ಟು ದೊರೆಯುವುದು ಬಹು ದುರ್ಲಭ.}
\slcol{\Index{ತತ್ರ ತಂ ಬುದ್ಧಿಸಂಯೋಗಂ} ಲಭತೇ ಪೌರ್ವದೇಹಿಕಮ್ ।\\
ಯತತೇ ಚ ತತೋ ಭೂಯಃ ಸಂಸಿದ್ಧೌ ಕುರುನಂದನ ॥ ೪೩ ॥}
\cquote{ಕುರುನಂದನ, ಅವನು ಅಲ್ಲಿ ಆ ಹಿಂದಿನ ಜನ್ಮದ ಬುದ್ಧಿಯನ್ನು ಪಡೆಯುತ್ತಾನೆ ಮತ್ತು ಜ್ಞಾನವನ್ನು ಮೈಗೂಡಿಸಿಕೊಳ್ಳುವುದಕ್ಕೆ ಇನ್ನೂ ಹೆಚ್ಚು ಪ್ರಯತ್ನವನ್ನು ಮಾಡುತ್ತಾನೆ.}
\slcol{\Index{ಪೂರ್ವಾಭ್ಯಾಸೇನ ತೇನೈವ} ಹ್ರಿಯತೇ ಹ್ಯವಶೋಽಪಿ ಸಃ ।\\
ಜಿಜ್ಞಾಸುರಪಿ ಯೋಗಸ್ಯ ಶಬ್ದಬ್ರಹ್ಮಾತಿವರ್ತತೇ ॥ ೪೪ ॥}
\cquote{ಏಕೆಂದರೆ ಅವನು ಹೆಚ್ಚು ಪ್ರಯತ್ನವಿಲ್ಲದೇ ಹಿಂದಿನ ಜನ್ಮದ ಅಭ್ಯಾಸದ ಕಡೆಗೆ ಕೊಚ್ಚಿಕೊಂಡು ಹೋಗುತ್ತಾನೆ. ಯೋಗದ ಸ್ವರೂಪವನ್ನು ತಿಳಿಯಲಪೇಕ್ಷಿಸುವವನೂ ಕೂಡ ಎಲ್ಲಾ ಕರ್ಮಗಳ ಫಲವನ್ನು ಮೀರಿದವನಾಗುತ್ತಾನೆ.}


\newpage
\begin{mananam}{\mananamfont{ಮನನ ಶ್ಲೋಕ - ೪೦}}
\footnotesize \mananamtext ನನ್ನ ಸಾಧನೆಗಳು ಅಥವಾ ವೈಫಲ್ಯಗಳನ್ನು ಅತಿಯಾಗಿ ವಿಮರ್ಶೆ ಮಾಡಿಕೊಳ್ಳುತ್ತೇನೆಯೇ? ಉದಾಹರಣೆಗೆ, ನಾನು ವಿಷಯಗಳನ್ನು ಸಂಪೂರ್ಣ ಯಶಸ್ಸು ಅಥವಾ ಸಂಪೂರ್ಣ ವೈಫಲ್ಯ ಎಂದು ನೋಡುತ್ತೇನೆಯೇ?(ವಿಪರೀತ ದೃಷ್ಟಿಕೋನ) ನನ್ನನ್ನು ಬೇರೆಯವರೊಂದಿಗೆ ಹೋಲಿಸಿಕೊಂಡು ಮೌಲ್ಯ ಮಾಪನೆ ಮಾಡಿಕೊಳ್ಳುತ್ತೇನೆಯೇ ಅಥವಾ, ನನಗಾಗಿ ಯುಕ್ತವಾದ ಗುರಿಗಳನ್ನು ನಿಗದಿಪಡಿಸಿದ್ದೇಯೇ?\\
ಹಲವಾರು ವರ್ಷಗಳಿಂದ ನಾನು ಮಾಡಿದ ಪ್ರಯತ್ನದ ಫಲವಾಗಿ ವರ್ತಮಾನದಲ್ಲಿ, ಸಾಧನೆಗಳು, ಸಿದ್ಧಿ ಮತ್ತು ಕೌಶಲ್ಯಗಳ ಉಗಮವಾಗಿದೆ ಎಂದು  ಗುರುತಿಸಬಲ್ಲೆನೇ? ಹಿಂದೆ ಪಟ್ಟ ಶ್ರಮಗಳು ಹೇಗೆ ಪ್ರಸ್ತುತ ಸಾಮರ್ಥ್ಯಗಳಿಗೆ ಅವಕಾಶ ನೀಡುತ್ತವೆಯೋ ಹಾಗೆಯೇ, ನನ್ನ ವರ್ತಮಾನದ ಈ ಪ್ರಯತ್ನಗಳು ಭವಿಷ್ಯದ ಬೆಳವಣಿಗೆಗೆ ಮತ್ತು ಹೆಚ್ಚಾದ ಮಾನಸಿಕ ಕೌಶಲ್ಯಗಳಿಗೆ ಸಂಭವನೀಯ ಕಾರಣಗಳಾಗಬಹುದು. ಈ ಎಲ್ಲಾ ಕಾರಣಗಳಿಂದ  ‘ಧ್ಯಾನ, ಅಧ್ಯಾತ್ಮಿಕ ಪ್ರಯತ್ನಗಳು ಮತ್ತು ಎಲ್ಲಾ ಪ್ರಯತ್ನಗಳೂ ಕೂಡ ಎಂದಿಗೂ ವ್ಯರ್ಥವಾಗುವುದಿಲ್ಲ, ಖಂಡಿತ ಫಲ ಕೊಟ್ಟೇ ಕೊಡುತ್ತವೆ’ ಎಂದು ನಾನು ವಿವೇಚನೆಯಿಂದ ತರ್ಕ ಮಾಡಬಲ್ಲೆನೇ? 
\end{mananam}
\WritingHand\enspace\textbf{ಆತ್ಮ ವಿಮರ್ಶೆ}\\
\begin{inspiration}{\mananamfont ಸ್ಫೂರ್ತಿ}
\footnotesize \mananamtext ಪ್ರಪಂಚದ ದೃಷ್ಟಿಕೋನದ ಆಧಾರದಿಂದ ಮಾತ್ರವೇ ಜೀವನದ ಗೆಲುವು ಮತ್ತು ವೈಫಲ್ಯಗಳನ್ನು ಅಳೆಯಲಾಗುವುದಿಲ್ಲ. ಅದೇ ರೀತಿ, ನಮಗೆ ಯಾವುದೇ ಒಳ್ಳೆಯ ಗುರಿಗಳನ್ನು ಈ ಜೀವನಾವಧಿಯಲ್ಲಿ ಸಾಧಿಸಲು ಸಾಧ್ಯವಾಗಿಲ್ಲ ಎಂಬ ಮಾತ್ರಕ್ಕೆ ನಮ್ಮನ್ನು ನಾವು ಸುಧಾರಿಸಿಕೊಳ್ಳುವ ಪ್ರಯತ್ನವನ್ನು ಬಿಟ್ಟುಬಿಡಬಾರದು. ನಮ್ಮನ್ನು ನಾವು ಒಳ್ಳೆಯದಕ್ಕೆ ಪರಿವರ್ತನೆಗೊಳಿಸಿಕೊಳ್ಳುವ (ಅಂದರೆ, ದೈಹಿಕವಾಗಿ, ಮಾನಸಿಕವಾಗಿ, ಪಾರಮಾರ್ಥಿಕವಾಗಿ ಮತ್ತು ಭೌತಿಕವಾಗಿ ಸುಧಾರಣೆ) ಮತ್ತು ಪ್ರತೀಜೀವಿಗೆ ನಿಸ್ವಾರ್ಥ ಸೇವೆ ಮಾಡುವ ಪ್ರತಿಯೊಂದು ಪ್ರಯತ್ನವೂ ಅಮೂಲ್ಯ ಮತ್ತು ಅದು ಸೂಕ್ತ ಸಮಯದಲ್ಲಿ ಫಲ ನೀಡೇ ನೀಡುತ್ತದೆ; ಭಾರತೀಯ ವೇದ ಮತ್ತು ಪುರಾಣ ಶಾಸ್ತ್ರಗಳ ಪ್ರಕಾರ, ಒಂದು ಜನ್ಮದಲ್ಲಿ ನಮ್ಮ ಪ್ರಯತ್ನವನ್ನು ಎಲ್ಲಿ ನಿಲ್ಲಿಸಿರುತ್ತೇವೆಯೋ ಅಲ್ಲಿಂದ, ಅದು ಇನ್ನೊಂದು ಜನ್ಮಕ್ಕೆ ಮುಂದುವರಿಯುತ್ತದೆ. ಆದ್ದರಿಂದ, ನಮ್ಮ ಕೊನೆಯ ಉಸಿರಿರುವರೆಗೂ ನಮ್ಮ ಅಂತರಂಗದ ಪರಿಪೂರ್ಣತೆಗೆ ಪ್ರಯತ್ನಿಸೋಣ.
\end{inspiration}
\newpage

\slcol{\Index{ಪ್ರಯತ್ನಾದ್ಯತಮಾನಸ್ತು} ಯೋಗೀ ಸಂಶುದ್ಧಕಿಲ್ಬಿಷಃ ।\\
ಅನೇಕಜನ್ಮಸಂಸಿದ್ಧಸ್ತತೋ ಯಾತಿ ಪರಾಂ ಗತಿಮ್ ॥ ೪೫ ॥}
\cquote{ಪ್ರಯತ್ನಪೂರ್ವಕವಾಗಿ ಯೋಗ ಸಾಧನೆಯಲ್ಲಿ ನಿರತನಾದ ಯೋಗಿಯಂತೂ ಪಾಪದ ಕೊಳೆಯನ್ನೆಲ್ಲ ಕಳೆದುಕೊಂಡು ಅನೇಕ ಜನ್ಮಗಳ ಸಾಧನೆಯ ಸಿದ್ಧಿಯ ಫಲವಾಗಿ ಕೊನೆಗೆ ಪರಮ ಪದವನ್ನು ಪಡೆಯುತ್ತಾನೆ.}
\slcol{\Index{ತಪಸ್ವಿಭ್ಯೋಽಧಿಕೋ ಯೋಗೀ} ಜ್ಞಾನಿಭ್ಯೋಽಪಿ ಮತೋಽಧಿಕಃ ।\\
ಕರ್ಮಿಭ್ಯಶ್ಚಾಧಿಕೋ ಯೋಗೀ ತಸ್ಮಾದ್ಯೋಗೀ ಭವಾರ್ಜುನ ॥ ೪೬ ॥}
\cquote{ಈ ಯೋಗಿಯು ತಪಸ್ವಿಗಳಿಗಿಂತಲೂ, ಪರೋಕ್ಷ ಜ್ಞಾನಿಗಳಿಗಿಂತಲೂ, ಕರ್ಮಿಗಳಿಗಿಂತಲೂ ಶ್ರೇಷ್ಠನು.ಆದುದರಿಂದ ಹೇ ಅರ್ಜುನ, ನೀನು ಯೋಗಿಯಾಗು.}
\slcol{\Index{ಯೋಗಿನಾಮಪಿ ಸರ್ವೇಷಾಂ} ಮದ್ಗತೇನಾಂತರಾತ್ಮನಾ ।\\
ಶ್ರದ್ಧಾವಾನ್ಭಜತೇ ಯೋ ಮಾಂ ಸ ಮೇ ಯುಕ್ತತಮೋ ಮತಃ ॥ ೪೭ ॥}
\cquote{ಎಲ್ಲ ಬಗೆಯ ಯೋಗಿಗಳಲ್ಲಿ ನನ್ನಲ್ಲೇ ಮನಸ್ಸನ್ನಿಟ್ಟು ನನ್ನನ್ನೇ ಶ್ರದ್ಧೆಯಿಂದ ಯಾವನು ಧ್ಯಾನ ಮಾಡುತ್ತಾನೋ ಅವನು ಬಹಳ ಹೆಚ್ಚಿನವನೆಂದು ನನ್ನ ಅಭಿಪ್ರಾಯ.}

\newpage
\begin{mananam}{\mananamfont{ಮನನ ಶ್ಲೋಕ - ೪೫, ೪೬}}
\small \mananamtext ನಾನು, “ಆಂತರಿಕ ಪರಿಪೂರ್ಣತೆ” ಎಂದರೆ ಏನೆಂದು ಅರ್ಥೈಸಿಕೊಂಡಿದ್ದೇನೆ? ಕಲೆ, ಕ್ರೀಡೆ, ಕೆಲಸಗಳಲ್ಲಿ ಕಂಡು ಬರುವ ಬಾಹ್ಯ   ಪರಿಪೂರ್ಣತೆಗಿಂತ ಇದು ಹೇಗೆ  ಭಿನ್ನವಾಗುವುದು? ಆಧ್ಯಾತ್ಮಿಕ ಸಾಧನೆಯಲ್ಲಿ, ಮನಸ್ಸು ಮತ್ತು ಹೃದಯವನ್ನು ಶುದ್ಧೀಕರಿಸುವುದು ಏಕೆ ನಿರ್ಣಾಯಕವಾಗಿದೆ?  ಇದು ನಮ್ಮ ಬೆಳವಣಿಗೆಯ ಮೇಲೆ ಹೇಗೆ ಪ್ರಭಾವ ಬೀರುವುದು?
ಅಧ್ಯಾತ್ಮಿಕ ಪ್ರಯಾಣದಲ್ಲಿ, ಅನೇಕ ಜನ್ಮಗಳಲ್ಲಿ ನಮ್ಮನ್ನು ನಾವು ಹೇಗೆ ಅಭಿವೃದ್ಧಿಪಡಿಸಿಕೊಳ್ಳಬಹುದು? ಈ ವಿಕಾಸಕ್ಕೆ ಯಾವ ಅಭ್ಯಾಸಗಳು ಸಹಕಾರಿಯಾಗುವುವು?\\
ನಾನು ಮಾಡಿದ ಪ್ರಯತ್ನಗಳನ್ನು ಗುರುತಿಸಿ ನನ್ನಲ್ಲಿ ನಾನು ಪ್ರೇರಣೆಯನ್ನು ಉಳಿಸಿಕೊಳ್ಳುವುದು ಹೇಗೆ? ಎಂಥಹ ಮನಸ್ಥಿತಿಯು ನನ್ನ ನಿರಂತರ ಪ್ರಗತಿಗೆ ಸಹಾಯಕವಾಗಬಲ್ಲದು?
\end{mananam}
\WritingHand\enspace\textbf{ಆತ್ಮ ವಿಮರ್ಶೆ}\\
\begin{inspiration}{\mananamfont ಸ್ಫೂರ್ತಿ}
\small \mananamtext ಆಧ್ಯಾತ್ಮಿಕ  ಬೋಧನೆಗಳ ಪ್ರಕಾರ, ಸ್ವಯಂ ನಿಯಂತ್ರಣ ಮತ್ತು ಅಂತರಂಗದ ಶುದ್ಧೀಕರಣದ ಯೋಗವು ಅತ್ಯುನ್ನತ ಸ್ಥಾನವನ್ನು ಹೊಂದಿದೆ. ಸ್ವಯಂ ನಿಯಂತ್ರಣವನ್ನು ಕರಗತ ಮಾಡಿಕೊಂಡವನು ಅಪಾರ ಸಾಮರ್ಥ್ಯಗಳನ್ನು ಪಡೆಯುತ್ತಾನೆ ಮತ್ತು ಏನನ್ನಾದರೂ ಸುಲಭವಾಗಿ ಸಾಧಿಸಬಲ್ಲನು. ಶುದ್ಧ ಹೃದಯ ಮತ್ತು ಕೇಂದ್ರೀಕೃತ ಮನಸ್ಸು ಆಧ್ಯಾತ್ಮಿಕ ಹಾದಿಯಲ್ಲಿರುವ ಸಾಧಕನಿಗೆ ಅಮೂಲ್ಯವಾದ ಆಸ್ತಿಗಳಾಗಿವೆ. ಅಂತಹ ಆಧ್ಯಾತ್ಮಿಕ ಗುಣಗಳು ಒಬ್ಬ ವ್ಯಕ್ತಿಯನ್ನು, ಭೌತಿಕವಾಗಿ ಅತ್ಯಂತ ಶ್ರೀಮಂತನಾಗಿರುವ ವ್ಯಕ್ತಿಗಿಂತಲೂ ಶ್ರೀಮಂತನನ್ನಾಗಿ ಮಾಡುತ್ತವೆ.
\end{inspiration}
\newpage

\chapEndSloka{ಆತ್ಮಸಂಯಮಯೋಗ}
\newpage