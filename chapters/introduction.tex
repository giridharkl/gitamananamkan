\indent ನಾವೆಲ್ಲರೂ ಜೀವನದ ಹೋರಾಟಗಳನ್ನು ಎದುರಿಸಬೇಕು. ಕುರುಕ್ಷೇತ್ರ ಯುದ್ಧದಲ್ಲಿ ಶ್ರೀ ಕೃಷ್ಣ ಪರಮಾತ್ಮ ತನ್ನ ವೇದನೆಯುಕ್ತ ಶಿಷ್ಯ ಅರ್ಜುನನಿಗೆ ಅಧ್ಯಾತ್ಮಿಕ ಹಾದಿಯಲ್ಲಿ ಆಚರಣೆಗೆ ತರುವಂತ ಬಹಳ ಪವಿತ್ರವಾದ ಬೋಧನೆಗಳನ್ನು ಕೊಟ್ಟಿದ್ದಾನೆ. ಈ ಶ್ರೇಷ್ಠವಾದ ಉಪನಿಷತ್ತುಗಳ ಸತ್ವಗಳನೊಳಗೊಂಡ  ಬೋಧನೆಗಳನ್ನು ಪವಿತ್ರವಾದ ಭಗವದ್ಗೀತೆಯನ್ನು ಸಂತ ವೇದವ್ಯಾಸರು ನಮ್ಮ ಕೈಗೆ ನೀಡಿದ್ದಾರೆ.
 ಅರ್ಜುನನು ಇದ್ದ ಪರಿಸ್ಥಿತಿಗೂ ನಾವು ಇರುವ ಪರಿಸ್ಥಿತಿ ಮತ್ತು ಸಂಘರ್ಷಗಳಿಗೂ ವ್ಯತ್ಯಾಸಗಳಿರಬಹುದು. ಸಾರ್ವರ್ತಿಕ ಉಪದೇಶಗಳು ಸತ್ಯಾನ್ವೇಷಣೆ ಮಾಡಲು ಬಯಸುವ ಎಲ್ಲರಿಗೂ ಆತ್ಮೋನತಿ  ಮತ್ತು ಅಧ್ಯಾತ್ಮಿಕ ಪ್ರಗತಿ ಸಾಧಿಸಲು ಬೇಕಾಗುವ ಮಾದರಿಯಾಗಿದೆ.
 ಭಗವದ್ಗೀತೆಯ ಉಪದೇಶಗಳು ಕೇವಲ ಆಧ್ಯಾತ್ಮಿಕ ಅನ್ವೇಷಣೆ ಮಾಡುವವರಿಗೆ ಸಮರ್ಪಿತವಾದದ್ದು ಮಾತ್ರವಲ್ಲದೆ ಜೀವನದಲ್ಲಿ ಬೇಕಾಗುವ ಅತ್ಯಮೂಲ್ಯವಾದ ಕೈಪಿಡಿಯಾಗಿದೆ. ಯಾರು ಕೆಲಸದ ಸಮತೋಲನ ಮತ್ತು ಕೌಟುಂಬಿಕ ಜವಾಬ್ದಾರಿಗಳನ್ನು ಮಾನಸಿಕ ನೆಮ್ಮದಿ ಮತ್ತು ಒತ್ತಡ ರಹಿತವಾಗಿ ಮಾಡಲು ಬಯಸುತ್ತಾರೋ ಅವರಿಗೆ ಈ ಬೋಧನೆಗಳು ತುಂಬಾ ಮಹತ್ವದ್ದಾಗಿ ಕಾಣುತ್ತದೆ.
 ಅನೇಕ ಗುರುಗಳು ಮತ್ತು ವಿದ್ವಾಂಸರು ಆಗಲೇ ಮಾಡಿರುವಂತೆ ಈ ದಿನಚರಿ ಪುಸ್ತಕ ಮತ್ತು ನಿಯತಕಾಲಿಕವು, ಗೀತೆಯ ಬೋಧನೆಗಳನ್ನು ತಿಳಿಸುವ ಪ್ರಯತ್ನ ಅಥವಾ ವ್ಯಾಖ್ಯಾನ ಕೊಡುವುದಾಗಿಲ್ಲ. ಈ ಗೀತಾ ಮನನವು ಬೋಧನೆಗಳ ಚಿಂತನೆ ಮಾಡುವುದು ಮತ್ತು ಅದನ್ನು ನಮ್ಮ ಸ್ವಂತದ್ದನ್ನಾಗಿ ಮಾಡಿಕೊಳ್ಳುವುದಾಗಿದೆ. ದೇವ ನಾಗರಿಯಲ್ಲಿರುವ `ಮನನ` ಎಂಬ ಪದವು ಆಗಲೇ ಕೇಳಿದ್ದನ್ನು ಅಥವಾ ಓದಿದ್ದನ್ನು ಚಿಂತನೆ ಮಾಡುವ ಕಾರ್ಯವಿಧಾನವನ್ನು ಅನ್ವಯಿಸುವುದಾಗಿದೆ.\\
\\
 ಈ ದಿನಚರಿ ಪುಸ್ತಕವನ್ನು ನೀವು ನಿಮ್ಮ ಮನಸ್ಸಿನ ಇಂಗಿತವನ್ನು ಸ್ವಾತಂತ್ರ್ಯವಾಗಿ ವ್ಯಕ್ತಪಡಿಸಲು ದಾರಿ ಮಾಡಿಕೊಳ್ಳುವುದಕ್ಕೆ ಮತ್ತು ನಿಮ್ಮ ಜೀವನದಲ್ಲಿ ಅಳವಡಿಸಿಕೊಳ್ಳಲು ಉದ್ದೇಶದಿಂದ ರೂಪಿಸಲಾಗಿದೆ. ಗೀತೆಯಲ್ಲಿರುವ ಶ್ಲೋಕಗಳಲ್ಲಿನ ಪ್ರಶ್ನೆಗಳು ಈ ಬೋಧನೆಗಳ ಸನ್ನಿವೇಶಕ್ಕೆ ಸಂಬಂಧಿಸಿದಂತೆ, ನಿಮ್ಮ ವೈಯಕ್ತಿಕ ಅರ್ಥಗಳನ್ನು ಹುಡುಕಲು ಮತ್ತು ಅದರಿಂದ ಜೀವನದ ಸಂದರ್ಭದೊಳಗೆ ನಿಶ್ಚಲವಾದ ಸ್ಪಷ್ಟನೆ ಹುಡುಕಲು ರೂಪಿಸಲಾಗಿದೆ.ಶ್ರೀ ಕೃಷ್ಣ ಪರಮಾತ್ಮನು ಹೇಗೆ ಅರ್ಜುನನಿಗೆ ನ್ಯಾಯವಾದ ಯುದ್ಧವನ್ನು ಮಾಡಲು ಪ್ರೇರೇಪಿಸಿದಂತೆ ನಿಮ್ಮ ಜೀವನದ ಕರ್ತವ್ಯಗಳನ್ನು ಈ ಗೀತೆ ಎಂಬ ಕೆಲಸದಿಂದ,ಆ ಭಗವಂತ ನಿಮಗೂ ಪ್ರೇರೇಪಿಸಲಿ ಎಂದು ನಂಬುತ್ತೇನೆ. ನಿಮ್ಮ ಕರ್ತವ್ಯಗಳನ್ನು ಕುಶಲತೆಯಿಂದ ಯಶಸ್ವಿಯಾಗಿ ನಿರ್ವಹಿಸಲು ನಿಮ್ಮ ಅಂತರಂಗದ ಶಾಂತಿಯನ್ನು ಉಪಯೋಗಿಸದೆ ವೈಯಕ್ತಿಕ ಪ್ರಗತಿ ಮತ್ತು ದೈವತ್ವಕ್ಕೆ ನಂಬಿಕೆಯಿಂದ ಇರುವ ನಿರಂತರ ಉದ್ದೇಶದಿಂದ ಕೂಡಿರುವುದೇ ಈ ದಿವ್ಯವಾದ ಗೀತೆ.
