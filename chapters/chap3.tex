\slcol{ಅರ್ಜುನ ಉವಾಚ ।\\
\Index{ಜ್ಯಾಯಸೀ ಚೇತ್ಕರ್ಮಣಸ್ತೇ} ಮತಾ ಬುದ್ಧಿರ್ಜನಾರ್ದನ ।\\
ತತ್ಕಿಂ ಕರ್ಮಣಿ ಘೋರೇ ಮಾಂ ನಿಯೋಜಯಸಿ ಕೇಶವ ॥ 1 ॥}
\cquote{ಅರ್ಜುನನು ಹೇಳಿದನು,
ಕೇಶವ, ನಿನಗೆ ಕರ್ಮಕಿಂತಲೂ ಜ್ಞಾನವು ಹೆಚ್ಚೆಂದು ಅಭಿಪ್ರಾಯವಾದರೆ ನನ್ನನ್ನು ಭಯಂಕರವಾದ ಕರ್ಮದಲ್ಲೇಕೆ ತೊಡಗಿಸುತ್ತಿ?\\}
\slcol{\Index{ಕೋವ್ಯಾಮಿಶ್ರೇಣಿವ ವಾಕ್ಯನಾ} ಬುದ್ಧಿಂ ಮೋಹಯಸೀವ ಮೇ । \\
ತದೇಕಂ ವದ ನಿಶ್ಚಿತ್ಯ ಯೇನ ಶ್ರೇಯೋಹ ಮಾಪ್ನುಯಮ್ ॥ 2 ॥}
\cquote{ಬೆರಕೆಯಾಗಿ ತೋರುವ ಮಾತುಗಳಿಂದ ನನ್ನ ಅಂತರಂಗವನ್ನು ಮೋಹೋಗೊಳಿಸುವಂತೆ ಆಡುತ್ತಿರುವೆ. ಯಾವುದರಿಂದ ನನಗೆ ಒಳ್ಳೆಯದಾಗುವುದೋ ಆ ಒಂದನ್ನು ನಿಶ್ಚಿತವಾಗಿ ಹೇಳು.\\}
\slcol{\Index{ನಿಷ್ಠಾ ಪುರಾ ಪ್ರೋಕ್ತಾ} ಮಯಾನಘ ।\\
ಙ್ಞಾನಯೋಗೇನ ಸಾಂಖ್ಯಾನಾಂ ಕರ್ಮಯೋಗೇನ ಯೋಗಿನಾಮ್ ॥ 3 ॥}
\cquote{ಭಗವಂತನು ಹೇಳಿದನು,\\
 ಅರ್ಜುನ, ಈ ಲೋಕದಲ್ಲಿ ನಾನು ಹಿಂದೆ ಎರಡು ಸ್ಥಿತಿಗಳನ್ನು ಹೇಳಿದನು. ಜ್ಞಾನಿಗಳಿಗೆ ಜ್ಞಾನ ಯೋಗ, ಸಾಧಕರಿಗೆ ಕರ್ಮ ಯೋಗ.\\}
\slcol{\Index{ನ ಕರ್ಮಣಾಮನಾರಂಭಾ}ನ್ನೈಷ್ಕರ್ಮ್ಯಂ ಪುರುಷೋऽಶ್ನುತೇ ।\\
ನ ಚ ಸಂನ್ಯಸನಾದೇವ ಸಿದ್ಧಿಂ ಸಮಧಿಗಚ್ಛತಿ ॥ 4 ॥}
\cquote{ಕೆಲಸ ಮಾಡದೆ ಇರುವುದರಿಂದ ಕರ್ಮ ಸಂಬಂಧವನ್ನು ತೊಡೆದುಹಾಕಲಾಗುವುದಿಲ್ಲ. ಕರ್ಮಗಳನ್ನು ಬಿಟ್ಟ ಮಾತ್ರದಿಂದಲೇ ಸಿದ್ಧಿಯನ್ನು ಪಡೆಯುವುದು ಸಾಧ್ಯವಿಲ್ಲ.\\}

\newpage
\begin{mananam}{\mananamfont ಮನನ ಶ್ಲೋಕ - \textenglish{1}}
\footnotesize \mananamtext ಜೀವನದಲ್ಲಿ ನಾನು ಚಟುವಟಿಕೆಯಿಂದ ಇರಬೇಕೋ ಬೇಡವೋ ಎನ್ನುವ ಅನುಮಾನದಲ್ಲಿದ್ದೇನೆ. ಪ್ರಾಪಂಚಿಕ ವಿಷಯಗಳಲ್ಲಿ ನಾನು ಅರೆ ಮನಸ್ಸಿನಿಂದ ತೊಡಗಿಸಿಕೊಂಡಿದ್ದೇನೆಯೇ? ನಾನು ವೈಯಕ್ತಿಕ ಲಾಭಗಳಿಗೆ ಎಳೆಯಪಟ್ಟಿದ್ದೇನೆಂಬ ಭಾವನೆಯಲ್ಲಿ ಇದ್ದೇನೆಯೇ? ಅಥವಾ ಜೀವನದ ಎಲ್ಲಾ ಚಟುವಟಿಕೆಗಳಿಂದ ದೂರವಾಗಿ ಆಧ್ಯಾತ್ಮಿಕವಾಗಿ ತೊಡಗಿಸಿಕೊಳ್ಳಬೇಕೆಂಬ ಆಳವಾದ ಭಾವನೆಯಲ್ಲಿದ್ದೇನೆಯೇ?  ಹಾಗಿದ್ದಲ್ಲಿ, ಈ ತರಹದ ಇಚ್ಛೆ ಜೀವನದ ಕಷ್ಟ ಮತ್ತು ಜವಾಬ್ದಾರಿಗಳಿಂದ ಪಲಾಯನ ಮಾಡುವುದಾಗಿದೆಯೇ? ಈ ನಿಷ್ಕ್ರಿಯ ಪ್ರಲೋಭನೆಯು ದೈಹಿಕ ಮತ್ತು ಮಾನಸಿಕ ನಿರುತ್ಸಾಹವೇ?
\end{mananam}
\WritingHand\enspace\textbf{ಆತ್ಮ ವಿಮರ್ಶೆ}\\
\begin{inspiration}{\mananamfont ಸ್ಪೂರ್ತಿ}
\footnotesize \mananamtext ನಿಜವಾದ ಜ್ಞಾನ ನಮಗೆ ಸರಿಯಾದ ದಾರಿಯಲ್ಲಿ ನಡೆಯಲು ಶಕ್ತಿ ತುಂಬುತ್ತದೆ. ನಮ್ಮ  ಆತ್ಮವಿದ್ಯೆಯು ನಮ್ಮನ್ನು ಅಹಂನಿಂದ ದೂರ ತರಬಲ್ಲದು. ಈ ಕ್ರಿಯೆಯು ಯಾವ ಅಡಚಣೆ ಇಲ್ಲದೆ ನಡೆಯಬಹುದು. ಈ ತರಹದ ಕ್ರಿಯೆಯು ತುಂಬಾ ಉತ್ಸಾಹದಾಯಕವಾಗಿದ್ದು ಜೀವನದಲ್ಲಿಯ ಯಾವ ಕ್ಷೇತ್ರದಲ್ಲಾದರೂ ಜಯದ ಕಡೆಗೆ ದಾರಿ ತೋರಿಸಬಹುದು.
\end{inspiration}
\newpage

\newpage
\begin{mananam}{\mananamfont ಮನನ ಶ್ಲೋಕ - \textenglish{3}}
\footnotesize \mananamtext ನಾನು ನನ್ನ ಪ್ರಧಾನ ವ್ಯಕ್ತಿತ್ವವನ್ನು ಆಧ್ಯಾತ್ಮಿಕ ಅಧಿಪತ್ಯದೊಳಗೆ ಹೇಗೆ ವರ್ಗಿಕರಿಸಲಿ? ನಾನು ಅಧಿಕವಾಗಿ ಬಹಿರ್ಮುಖಿಯಾಗಿ ಕಾರ್ಯನಿರ್ವಹಿಸುವವನೆ? ನಾನು ಬೇರೆಯವರಿಗೆ ಸೇವೆ ಮಾಡುವಂಥವನಾಗಬೇಕೇ? ನನ್ನ ಒಳಗಿನ  ಸ್ವಂತ ಅಭಿವೃದ್ಧಿಗೆ ಪ್ರಪಂಚದಲ್ಲಿ ಬದಲಾವಣೆ ತರುವಂತೆ ಯೋಚಿಸುತ್ತಿದ್ದೇನೆಯೇ? ಅಥವಾ ನಾನು ಅಂತರ್ಮುಖಿಯಾಗಿ ಅಥವಾ ಧ್ಯಾನ ಪರತೆಯಲ್ಲಿರುವೆನೇ? ನನ್ನೊಳಗಿನ ಹುಡುಗಾಟ ಮತ್ತು ದೈವಿಕವಾದ ವಿಷಯಗಳ ಬಗ್ಗೆ ನಿಶ್ಚಯವಾಗಿ ಆಳವಾದ ಶಿಕ್ಷಣ ಪಡೆಯಲು ಮತ್ತು ನನ್ನೊಳಗೆ ಮುಳುಗಿ ನೋಡಲು ತಯಾರಿದ್ದೇನೆಯೆ? ನಾನು ಈ ಎರಡರಲ್ಲಿ ಸಮಚಿತ್ತವನ್ನು ಸಾಧಿಸಬೇಕಾದರೆ ಅದು ಹೇಗೆ ಕಾಣಬಹುದು?
\end{mananam}
\WritingHand\enspace\textbf{ಆತ್ಮ ವಿಮರ್ಶೆ}\\
\begin{inspiration}{\mananamfont ಸ್ಪೂರ್ತಿ}
\footnotesize \mananamtext ಅಧ್ಯಾತ್ಮಕ ದಾರಿಯಲ್ಲಿ ವಿಶಾಲವಾದ ವರ್ಗೀಕರಣವು ನಮ್ಮ ಜೀವನ ಪಥವನ್ನು ಕೆತ್ತುವಲ್ಲಿ ಸಹಾಯ ಮಾಡುವುದಕ್ಕಾಗಿ ಇದೆ. ನಮ್ಮ ಪ್ರತಿ ಒಬ್ಬರೊಳಗು ವಿವಿಧ ರೀತಿಯ ವ್ಯಕ್ತಿತ್ವಗಳಿದ್ದು ನಮ್ಮ ಪ್ರಗತಿಗೆ ಅಗತ್ಯವಾದ ಸಮಚಿತ್ತ ಬೇಕಾಗುತ್ತದೆ. ಒಬ್ಬರಿಗೆ ಏನು ಸಮಚಿತ್ತವಿರುತ್ತದೆಯೋ, ಬೇರೆಯವರಿಗೂ ಅದೇ ಆಗುವುದಿಲ್ಲ.
\end{inspiration}
\newpage


\slcol{\Index{ನ ಹಿ ಕಶ್ಚಿತ್ಕ್ಷಣಮಪಿ} ಜಾತು ತಿಷ್ಠತ್ಯಕರ್ಮಕೃತ್ ।\\
ಕಾರ್ಯತೇ ಹ್ಯವಶಃ ಕರ್ಮ ಸರ್ವಃ ಪ್ರಕೃತಿಜೈರ್ಗುಣೈಃ ॥ 5 ॥} 
\cquote{ಯಾವನು ಒಂದು ಕ್ಷಣ ಎಂದಿಗೂ ಕೆಲಸವನ್ನು ಮಾಡದೆ ಇರಲಾರನು. ಎಲ್ಲರೂ ತಮ್ಮ ಹುಟ್ಟು ಗುಣಗಳಿಂದ ತಮ್ಮಗರಿವಿಲ್ಲದಂತೆ ಕರ್ಮ ಮಾಡುತ್ತಿರುವರು.\\}
\slcol{\Index{ಕರ್ಮೇಂದ್ರಿಯಾಣಿ ಸಂಯಮ್ಯ} ಯ ಆಸ್ತೇ ಮನಸಾ ಸ್ಮರನ್ ।\\
ಇಂದ್ರಿಯಾರ್ಥಾನ್ವಿಮೂಢಾತ್ಮಾ ಮಿಥ್ಯಾಚಾರಃ ಸ ಉಚ್ಯತೇ ॥ 6 ॥}
\cquote{ಯಾವನು ಕರ್ಮೀನ್ದ್ರಿಯಗಳನ್ನು ಬಿಗಿಹಿಡಿದು ಇಂದ್ರಿಯಗಳು ಬಯಸುವ ವಿಷಯಗಳನ್ನು ಮನಸ್ಸಿನಲ್ಲಿ ಹಂಬಲಿಸುತ್ತಿರುವನೋ ಆ ಮೂಡನು ಸುಳ್ಳು ನಟನೆಯ ಡಂಬಾಚಾರಿಯೆನಿಸುವನು.\\}
\slcol{\Index{ಯಸ್ತ್ವಿಂದ್ರಿಯಾಣಿ ಮನಸಾ} ನಿಯಮ್ಯಾರಭತೇऽರ್ಜುನ ।\\
ಕರ್ಮೇಂದ್ರಿಯೈಃ ಕರ್ಮಯೋಗಮಸಕ್ತಃ ಸ ವಿಶಿಷ್ಯತೇ ॥ 7 ॥}
\cquote{ಅರ್ಜುನ,ಯಾವನು ಇಂದ್ರಿಯಗಳನ್ನು ಮನಸ್ಸಿನಿಂದ ಬಿಗಿಹಿಡಿದು ಆಸಕ್ತಿ ಇಲ್ಲದೆ ಕರ್ಮೇಂದ್ರಿಯಗಳಿಂದ ಕೆಲಸದಲ್ಲಿ ತೊಡಗಿರುವನೋ ಅವನು ಹೆಚ್ಚಿನವನು.\\}
\slcol{\Index{ನಿಯತಂ ಕುರು ಕರ್ಮ ತ್ವಂ} ಕರ್ಮ ಜ್ಯಾಯೋ ಹ್ಯಕರ್ಮಣಃ ।\\
ಶರೀರಯಾತ್ರಾಪಿ ಚ ತೇ ನ ಪ್ರಸಿದ್ಧ್ಯೇದಕರ್ಮಣಃ ॥ 8 ॥}
\cquote{ನೀನು ಮಾಡತಕ್ಕದ್ದೆಂದು ಗೊತ್ತಾಗಿರುವ ಕೆಲಸವನ್ನು ಮಾಡು. ಏನು ಮಾಡದೆ ಇರುವುದಕ್ಕಿಂತ ಮಾಡುವುದು ಮೇಲು. ನೀನು ಯಾವ ಕರ್ಮವನ್ನು ಮಾಡದೆ ಇದ್ದರೆ ಬದುಕುವುದೇ  ಆಗಲಾರದು.\\}
\slcol{\Index{ಯಙ್ಞಾರ್ಥಾತ್ಕರ್ಮಣೋऽನ್ಯತ್ರ} ಲೋಕೋऽಯಂ ಕರ್ಮಬಂಧನಃ ।\\
ತದರ್ಥಂ ಕರ್ಮ ಕೌಂತೇಯ ಮುಕ್ತಸಂಗಃ ಸಮಾಚರ ॥ 9 ॥}
\cquote{ಈಶ್ವರನಿಗೆ ಪ್ರೀತಿಯಾಗಲೆಂದಲ್ಲದೆ ಸ್ವಾರ್ಥಕ್ಕಾಗಿ ಮಾಡುವ ಕರ್ಮಗಳಿಂದ ಈ ಲೋಕದ ಕಟ್ಟಿಗೋಳಪಡುವದು. ಅರ್ಜುನ, ಫಲವನ್ನು ಬಯಸದೆ ಈಶ್ವರನಿಗೆ ಪ್ರೀತಿಯಾಗಲೆಂದು ಕರ್ಮವನ್ನು ನಡೆಸು.\\}
\slcol{\Index{ಸಹಯಙ್ಞಾಃ ಪ್ರಜಾಃ} ಸೃಷ್ಟ್ವಾ ಪುರೋವಾಚ ಪ್ರಜಾಪತಿಃ ।\\
ಅನೇನ ಪ್ರಸವಿಷ್ಯಧ್ವಮೇಷ ವೋऽಸ್ತ್ವಿಷ್ಟಕಾಮಧುಕ್ ॥ 10 ॥} 
\cquote{ಯಜ್ಞಗಳೊಡನೆ ಜನರನ್ನು ಹುಟ್ಟಿಸಿ ಹಿಂದೆ ಬ್ರಹ್ಮನು ಇದರಿಂದ ಏಳಿಗೆಯನ್ನು ಹೊಂದಿರಿ, ಇದು ನಿಮಗೆ ಬಯಸಿದ್ದನ್ನೆಲ್ಲಾ ಕೊಡುವಂತದ್ದಾಗಲಿ.\\}
\slcol{\Index{ದೇವಾನ್ಭಾವಯತಾನೇನ ತೇ} ದೇವಾ ಭಾವಯಂತು ವಃ ।\\
ಪರಸ್ಪರಂ ಭಾವಯಂತಃ ಶ್ರೇಯಃ ಪರಮವಾಪ್ಸ್ಯಥ ॥ 11 ॥}
\cquote{ಇದರಿಂದ ದೇವತೆಗಳನ್ನು ಸಂತೋಷಗೊಳಿಸಿರಿ. ಆ ದೇವತೆಗಳು ನಿಮ್ಮನ್ನು ಸಂತೋಷಗೊಳಿಸಲಿ. ಒಬ್ಬರನ್ನೊಬ್ಬರು ಸಂತೋಷಗೊಳಿಸುವರಾಗಿ ಹೆಚ್ಚಿನ ಯಶಸ್ಸನ್ನು ಪಡೆಯಿರಿ.\\}

\newpage
\begin{mananam}{\mananamfont ಮನನ ಶ್ಲೋಕ - \textenglish{6}}
\footnotesize \mananamtext ನೀವೊಬ್ಬ ವಿದ್ಯಾಭ್ಯಾಸ ಬಿಟ್ಟು ಪ್ರೌಢ ವಯಸ್ಕನಾಗಿದ್ದರೆ ಉದ್ಯೋಗವನ್ನು ಹುಡುಕುವ ಯುವಕನಾಗಿಲ್ಲದಿದ್ದರೆ ವಯಸ್ಕನಾಗಿ ಸಂಸಾರದ ಜವಾಬ್ದಾರಿಗಳನ್ನು ಪೂರ್ಣಗೊಳಿಸುವವನಾಗಿಲ್ಲದಿದ್ದರೆ, ನೀವು ಒಬ್ಬ ಪ್ರಗತಿ ಸಹಜ ಕೆಲಸಗಳಿಗೆ ವಿರುದ್ಧವಾಗಿದ್ದೀರಿ ಎಂದು ತಿಳಿಯಬೇಕು. ಹಾಗಿದ್ದರೆ ನೀವು ಹೇಗೆ ನಿಮ್ಮ ಶಕ್ತಿಯನ್ನು ಸರಿಯಾದ ದಾರಿಯಲ್ಲಿ ಹರಿಸುತ್ತೀರಿ? ಇದು ನಿಮ್ಮ ಮಾನಸಿಕ ಮತ್ತು ಭಾವನೆಗಳ ಉದ್ವೇಗಕ್ಕೆ ಕಾರಣವಾಗಬಹುದೇ? ನೀವು ಮಾನಸಿಕ ಬ್ರಾಂತಿಯಲ್ಲಿ ಕಳೆದು ಹೋಗಿದ್ದೀರಿಯೇ?ನೀವು ಏನಾದರೂ ಸೃಜನಾತ್ಮಕವಾದ
ಕಾರ್ಯವನ್ನು ಮಾಡಲು ನಿರ್ಧರಿಸಿದ್ದೀರಿಯೇ?
\end{mananam}
\WritingHand\enspace\textbf{ಆತ್ಮ ವಿಮರ್ಶೆ}\\
\begin{inspiration}{\mananamfont ಸ್ಪೂರ್ತಿ}
\footnotesize \mananamtext ಒಬ್ಬರು ಸುಪ್ತಾವಸ್ಥೆಯಲ್ಲಿ ಅಥವಾ ಸೋಮಾರಿತನದಲ್ಲಿದ್ದಾರೆಂದರೆ ಅದರ ಅರ್ಥ ಅವರು ಇಂದ್ರೀಯ ಜ್ಞಾನವನ್ನು ನಿರ್ಬಂಧಿಸಿದ್ದಾರೆಂದು ಅರ್ಥವಲ್ಲ. ತರಬೇತಿ ಪಡೆಯದ ಮನಸ್ಸು ನಿಜವಾಗಿಯೂ ಇಂದ್ರಿಯ ಜ್ಞಾನವನ್ನು ಹತೋಟಿಯಲ್ಲಿಡಲು ಸಾಧ್ಯವಿಲ್ಲ. ಇಂತಹ ಸಂದರ್ಭಗಳಲ್ಲಿ ನಿಸ್ಕ್ರೀಯವಾಗಿರುವುದಕ್ಕಿಂತ ಫಲದಾಯಕವಾದ ಕ್ರಿಯೆ ಮಾಡುವುದಕ್ಕೆ ಇಂದ್ರಿಯಗಳನ್ನು ತರಬೇತಿಗೆ ಒಳಪಡಿಸುವುದು ಸೂಕ್ತ.
\end{inspiration}
\newpage

\begin{mananam}{\mananamfont ಮನನ ಶ್ಲೋಕ - \textenglish{8}}
\footnotesize \mananamtext ನಾನು ಮಾಡುವ ಕೆಲಸದ ರೀತಿಯಿಂದ ನಾನು ಕರ್ಮ ಯೋಗವನ್ನು ಹೇಗೆ ಅಭ್ಯಾಸ ಮಾಡಲಿ? [ಫಲಿತಾಂಶದ ಮೇಲಿನ ಮೋಹವನ್ನು ಬಿಟ್ಟ ನಮ್ಮ ಕ್ರಿಯೆಗಳು] ನಾನು ಪ್ರತಿಫಲದ ಪ್ರೇರಣೆ ಇಲ್ಲದೆ ಕೆಲಸ ಮಾಡಬಲ್ಲೆನೆ? ಅಥವಾ ಯಾವ ನಿರ್ದಿಷ್ಟ ಗುರಿ ಇಲ್ಲದ ಕೆಲಸದಲ್ಲಿ, ನನಗೆ ಚೈತನ್ಯ ಮತ್ತು ಉತ್ಸಾಹದ ಕೊರತೆ ಇದೆ ಎಂಬ ಭಾವನೆ ಇದೆಯೇ? ನನಗೆ ಆ ಕ್ಷಣದ ಬೇಡಿಕೆಗಳನ್ನು ಒಂದು ಸಮಯದಲ್ಲಿ ಒಂದು ಕೆಲಸದಂತೆ ವಿಂಗಡಿಸುವುದು ತಿಳಿದಿದೆಯೇ?
\end{mananam}
\WritingHand\enspace\textbf{ಆತ್ಮ ವಿಮರ್ಶೆ}\\
\begin{inspiration}{\mananamfont ಸ್ಪೂರ್ತಿ}
\footnotesize \mananamtext ಕರ್ಮ ಯೋಗದ ಹೃದಯಭಾಗದಲ್ಲಿ ಸಾಧನೆಯ ಬಗ್ಗೆ ಕೇಂದಿಕೃತವಾಗಿರುತ್ತದೆಯೇ ಹೊರತು ಪ್ರತಿಫಲದ ಕಡೆಗೆ ಇರುವುದಿಲ್ಲ. ಒಬ್ಬರು, ಕಾರ್ಯವಿಧಾನದ ಬಗ್ಗೆ ಸರಿಯಾದ ದಾರಿಯಲ್ಲಿ ನಡೆದರೆ ಕೊನೆಯಲ್ಲಿ ಸಿಗುವ ಪ್ರತಿಫಲ ತನ್ನಷ್ಟಕ್ಕೆ ತಾನೇ ಸಿಗುವಂತಾಗುವುದು. ನಿರ್ದಿಷ್ಟ ಕಾರಣವಿಲ್ಲದೆ ಕೆಲಸವನ್ನು ಕೆಲಸದ ಸಲುವಾಗಿಯಷ್ಟೇ ಮಾಡುವುದು ತುಂಬಾ ಆರೋಗ್ಯಕರ ಮತ್ತು ಉತ್ತಮ ಮಟ್ಟದ ಕಾರ್ಯವಾಗಿರುತ್ತದೆ.
\end{inspiration}
\newpage

\begin{mananam}{\mananamfont ಮನನ ಶ್ಲೋಕ - \textenglish{9}}
\footnotesize \mananamtext ನಾನು ಮಾಡುವ ಎಲ್ಲ ಕೆಲಸಗಳನ್ನು ಒಂದು ತ್ಯಾಗಯುಕ್ತವಾದ ಯಜ್ಞವೆಂದು ಭಾವಿಸಿ ಮಾಡಲು ಕಲಿಯಬಲ್ಲೆನೇ? ನನ್ನ ಮನಸ್ಸಿನ ಆವೇಗವು ಫಲಿತಾಂಶ ಮತ್ತು ಪ್ರತಿಫಲಗಳ ಯೋಚನೆಯಿಂದ ಕೆಲಸಗಳಲ್ಲಿ ತೊಡಗಿಸಿಕೊಳ್ಳಲು ಪ್ರೇರೇಪಿಸುತ್ತಿದೆಯೇ? ನಾನು ನನ್ನ ಕೆಲಸದಲ್ಲಿ ತೊಡಗಿಸಿಕೊಂಡಾಗ ನಾನು ಆ ಕ್ಷಣದ ಬಗ್ಗೆ ಕೇಂದ್ರೀಕರಿಸುತ್ತೇನೆಯೇ ಅಥವಾ ಮುಂದಿನ ಫಲಿತಾಂಶದ ಬಗ್ಗೆಯೇ? ನಾನು ನನ್ನ ಕೆಲಸವನ್ನು ಪೂರ್ತಿಗೊಳಿಸಿದ ಮೇಲೆ ನಾನು ನನ್ನ ಕೆಲಸದ ಫಲಿತಾಂಶದಿಂದ ಸಂತೋಷ ಅಥವಾ ದುಃಖದ ಭಾವನೆಯನ್ನು ಅನುಭವಿಸುತ್ತಿದ್ದೇನೆಯೇ?
\end{mananam}
\WritingHand\enspace\textbf{ಆತ್ಮ ವಿಮರ್ಶೆ}\\
\begin{inspiration}{\mananamfont ಸ್ಪೂರ್ತಿ}
\footnotesize \mananamtext ಪ್ರಾಚೀನ ಕಾಲದ ವೇದಗಳಲ್ಲಿ ತಿಳಿಸಿರುವ ಯಜ್ಞವು, ಆರಾಧಕರು ನೆಲೆಗೊಳಿಸಿದಂತೆ, ನಾನು ಕೊಡುವ ಪ್ರಾಪಂಚಿಕ ವಸ್ತುಗಳ ಆಹುತಿಗಳು, ಮೌಲ್ಯಯುತವಾದ ಲಾಭ ಪಡೆಯುವುದಕ್ಕೋಸ್ಕರ ಉತ್ತಮವಾದ ಕರ್ಮ ಮತ್ತು ದಿವ್ಯಾನಂದದ ಫಲವನ್ನು ಪಡೆಯುವುದಕ್ಕೋಸ್ಕರ ಇರುವುದಾಗಿದೆ. ಈ ತರಹದ ಯಜ್ಞವು ಒಂದನ್ನು ಪಡೆಯಲು ಇನ್ನೊಂದನ್ನು ಕೊಡುವ ಲೇವಾದೇವಿ ವ್ಯಾಪಾರವಾಗುತ್ತದೆ. ಕರ್ಮ ಯೋಗವು ಈ ಮಾನಸಿಕ ಸ್ಥಿತಿಯನ್ನು ಕೇವಲ ಕೊಡುವುದರ ಬೆಲೆಯನ್ನು ತಿಳಿಸಿ ಕೊಡುವ ಬಗ್ಗೆ ರೂಪಾಂತರಗೊಳ್ಳುವಂತೆ ಮಾಡುತ್ತದೆ.
\end{inspiration}
\newpage


\slcol{\Index{ಇಷ್ಟಾನ್ಭೋಗಾನ್ಹಿ ವೋ} ದೇವಾ ದಾಸ್ಯಂತೇ ಯಙ್ಞಭಾವಿತಾಃ ।\\
ತೈರ್ದತ್ತಾನಪ್ರದಾಯೈಭ್ಯೋ ಯೋ ಭುಂಕ್ತೇ ಸ್ತೇನ ಏವ ಸಃ ॥ 12 ॥}
\cquote{ಯಜ್ಞಗಳಿಂದ ಸಂತೋಷಗೊಂಡ ದೇವತೆಗಳು ನಿಮಗೆ ಬಯಸಿದ್ದನ್ನೆಲ್ಲಾ ಕೊಡುತ್ತಾರೆ. ಅವರು ಕೊಟ್ಟಿದ್ದನ್ನು ಅವರಿಗೆ ಕೊಡದೆ ಯಾವನು ಉಣ್ಣುತ್ತಾನೋ ಅವನು ಕಳ್ಳನೇ ಸರಿ.\\}
\slcol{\Index{ಯಙ್ಞಶಿಷ್ಟಾಶಿನಃ ಸಂತೋ} ಮುಚ್ಯಂತೇ ಸರ್ವಕಿಲ್ಬಿಷೈಃ ।\\
ಭುಂಜತೇ ತೇ ತ್ವಘಂ ಪಾಪಾ ಯೇ ಪಚಂತ್ಯಾತ್ಮಕಾರಣಾತ್ ॥ 13 ॥}
\cquote{ಯಜ್ಞಗಳನ್ನು ಮಾಡಿ ಉಳಿದದ್ದನ್ನು ಉಣ್ಣುವವರು ಎಲ್ಲ ಪಾಪಗಳಿಂದಲೂ ಬಿಡುಗಡೆಯನ್ನು ಹೊಂದುತ್ತಾರೆ. ಯಾರು ತಮ್ಮ ಹೊಟ್ಟೆಗಾಗಿ ಭೇಯಿಸುವರೋ, ಆ ಪಾಪಿಗಳು ಪಾಪವನ್ನು ಉಣ್ಣುತ್ತಾರೆ.\\}
\slcol{\Index{ಅನ್ನಾದ್ಭವಂತಿ ಭೂತಾನಿ} ಪರ್ಜನ್ಯಾದನ್ನಸಂಭವಃ ।\\
ಯಙ್ಞಾದ್ಭವತಿ ಪರ್ಜನ್ಯೋ ಯಙ್ಞಃ ಕರ್ಮಸಮುದ್ಭವಃ ॥ 14 ॥} 
\cquote{ಅನ್ನದಿಂದ ಜೀವಿಗಳು ಹುಟ್ಟುತ್ತವೆ. ಮಳೆಯಿಂದ ಅನ್ನ ಹುಟ್ಟುತ್ತದೆ. ಯಜ್ಞದಿಂದ ಮಳೆ ಉಂಟಾಗುತ್ತದೆ. ಯಜ್ಞವು ಕರ್ಮದಿಂದ ನಡೆಯುವುದು.\\}
\slcol{\Index{ಕರ್ಮ ಬ್ರಹ್ಮೋದ್ಭವಂ} ವಿದ್ಧಿ ಬ್ರಹ್ಮಾಕ್ಷರಸಮುದ್ಭವಮ್ ।\\
ತಸ್ಮಾತ್ಸರ್ವಗತಂ ಬ್ರಹ್ಮ ನಿತ್ಯಂ ಯಙ್ಞೇ ಪ್ರತಿಷ್ಠಿತಮ್ ॥ 15 ॥}
\cquote{ಎಲ್ಲ ಕರ್ಮಗಳ ಮೂಲ ಭಗವಂತ.ವೇದಾಕ್ಷರಗಳಿಂದ ಭಗವಂತನ ಅಭಿವ್ಯಕ್ತಿ. ವೇದಾಕ್ಷರಗಳನ್ನು ಜೀವಿಗಳು ಉಚ್ಚರಿಸುತ್ತವೆ. ಆದ್ದರಿಂದ ಎಲ್ಲೆಡೆಯೂ ತುಂಬಿರುವ ಭಗವಂತನು ವಿಶೇಷತಃ ಯಜ್ಞದಲ್ಲಿ ಸದಾ ಸನ್ಹಿತನಾಗಿದ್ದಾನೆ.\\}
\slcol{\Index{ಏವಂ ಪ್ರವರ್ತಿತಂ ಚಕ್ರಂ} ನಾನುವರ್ತಯತೀಹ ಯಃ ।\\
ಅಘಾಯುರಿಂದ್ರಿಯಾರಾಮೋ ಮೋಘಂ ಪಾರ್ಥ ಸ ಜೀವತಿ ॥ 16 ॥}
\cquote{ಅರ್ಜುನ,ಹೀಗೆ ಪ್ರವೃತ್ತವಾದ ಈ ಜೀವನ ಚಕ್ರವನ್ನು ಯಾವನು ಮುಂದುವರಿಸುವುದಿಲ್ಲವೋ ಅವನು ಪಾಪದ ಬಾಳಿನವನೂ ಇಂದ್ರಿಯಗಳೊಡನೆ ವಿನೋದಿಸುವವನೂ ಆಗುವುದರಿಂದ ಅವನ ಬದುಕು ವ್ಯರ್ಥ.\\}
\slcol{\Index{ಯಸ್ತ್ವಾತ್ಮರತಿರೇವ} ಸ್ಯಾದಾತ್ಮತೃಪ್ತಶ್ಚ ಮಾನವಃ ।\\
ಆತ್ಮನ್ಯೇವ ಚ ಸಂತುಷ್ಟಸ್ತಸ್ಯ ಕಾರ್ಯಂ ನ ವಿದ್ಯತೇ ॥ 17 ॥}
\cquote{ಯಾವ ಮನುಷ್ಯನು ಪರಮಾತ್ಮನಲ್ಲಿಯೇ ಪ್ರೇಮವುಳ್ಳವನಾಗಿ ಪರಮಾತ್ಮನಿಂದ ತೃಪ್ತನಾಗಿ ಆತನಲ್ಲಿಯೇ ಸಂತೋಷಗೊಳ್ಳುತ್ತಿರುವನೋ ಅವನು ಮಾಡಬೇಕಾದದ್ದೇನೂ ಇಲ್ಲ.\\}

\newpage
\begin{mananam}{\mananamfont ಮನನ ಶ್ಲೋಕ - \textenglish{15}}
\footnotesize \mananamtext ಈ ಜಗತ್ತು ಈ ದೇಹವನ್ನು ಸಹಿಸಿಕೊಳ್ಳುವುದಾದಲ್ಲಿ ನಾನು ಈ ಜಗತ್ತಿಗೆ ಎಲ್ಲವನ್ನೂ ಹಿಂತಿರುಗಿಸಿ ಕೊಡುವುದು ಸರಿಯಲ್ಲ. ನಾನು ಹೇಗೆ ತ್ಯಾಗದ ಬುದ್ಧಿಯನ್ನು ಬೆಳೆಸಿಕೊಳ್ಳಲಿ ಮತ್ತು ನ್ಯಾಯಯುತವಾದ ಕಾರ್ಯಕ್ಕೆ ಹೇಗೆ ತೊಡಗಿಸಿಕೊಳ್ಳಲಿ? ನಾನು ಮಾಡುವ ಎಲ್ಲ ಕೆಲಸಗಳನ್ನು ದೇವರಿಗೆ ಕಾಣಿಕೆಯಾಗಿ ನೀಡುವುದನ್ನು ಕಲಿಯಲೇ ಅಥವಾ ನನ್ನ ಸಂಪೂರ್ಣ ಅಸ್ತಿತ್ವವನ್ನು ಪೋಷಿಸುವ ಪ್ರಕೃತಿಗೆ ನೀಡುವುದನ್ನು ಕಲಿಯಲೇ?
\end{mananam}
\WritingHand\enspace\textbf{ಆತ್ಮ ವಿಮರ್ಶೆ}\\
\begin{inspiration}{\mananamfont ಸ್ಪೂರ್ತಿ}
\footnotesize \mananamtext ಯಜ್ಞದ ನಿಜವಾದ ಅರ್ಥ ಅಥವಾ ತ್ಯಾಗವು ಏನನ್ನು ಮರುಳಿ ನಿರೀಕ್ಷಿಸದೆ ಬಿಟ್ಟು ಕೊಡುವುದಾಗಿದೆ. ಈ ಕಾರ್ಯವು ನಮ್ಮೊಳಗಿನ ಶ್ರದ್ಧೆ [ನಂಬಿಕೆಯನ್ನು] ಎಚ್ಚರಿಸಿದಾಗ ಮಾತ್ರ ಸಾಧ್ಯವಾಗುವುದು. ಸಂಪೂರ್ಣ ವಿಶ್ವದ ಎಲ್ಲ ಆಗು ಹೋಗುಗಳು ಈ ಶ್ರದ್ದೆಯಿಂದ [ನಂಬಿಕೆಯಿಂದ] ನಡೆಯುತ್ತಿರುತ್ತದೆ ಮತ್ತು ಇದುವೇ ಜೀವನದ ಮೂಲತತ್ವವಾಗಿದೆ.ಯಾರು ಏನನ್ನು ಹಿಡಿದಿಟ್ಟುಕೊಳ್ಳದೆ ಎಲ್ಲವನ್ನೂ ತ್ಯಾಗ ಮಾಡುತ್ತಾರೋ ಅವರನ್ನು ಪ್ರಕೃತಿಯು ಪೋಷಿಸುತ್ತದೆ.
\end{inspiration}
\newpage

\begin{mananam}{\mananamfont ಮನನ ಶ್ಲೋಕ - \textenglish{16}}
\footnotesize \mananamtext
 ಭೌತಿಕ ವಿಷಯಗಳ ಅಧಿಕ ಕೇಂದ್ರೀಕರಣದಿಂದ, ನಾನು, ಈ ಜೀವನ ಚಕ್ರವನ್ನು ನೋಡಿ ಜಿಗುಪ್ಸೆ ಪಟ್ಟುಕೊಳ್ಳುತ್ತಿದ್ದೇನೆಯೇ? ಇದು ಸೋತವನು ‘ದ್ರಾಕ್ಷಿ ಹುಳಿ’ ಎನ್ನುವ ಮನೋಭಾವವೇ ಅಥವಾ ಸೋಮಾರಿತನವೇ?ಹಾಗಿದ್ದರೆ ನಾನು ಸಮಾಜದ ಪದ್ಧತಿಗಳಲ್ಲಿ ಭಾಗವಹಿಸುವುದಿಲ್ಲವಾದರೆ ಅದರ ಲಾಭವನ್ನು ನಾನೇ ತೆಗೆದುಕೊಳ್ಳುತ್ತಿದ್ದೇನೆಯೇ? ನಾನು ಸಮಾಜಕ್ಕೆ ಧನಾತ್ಮಕವಾದ ಕೊಡುಗೆಯನ್ನು ನೀಡುತ್ತಿದ್ದೇನೆಯೆ? ನಾನು ಯಾವ ಸ್ವಾರ್ಥಯುಕ್ತವಾದ ತೃಪ್ತಿ ಪಡೆಯುವ ಅವಶ್ಯಕತೆ ಇಲ್ಲದೆ ಕಾರ್ಯಕ್ಕೋಸ್ಕರ ಕಾರ್ಯ ಮಾಡಲು ಸಮರ್ಥನಾಗಿದ್ದೇನೆಯೇ?
\end{mananam}
\WritingHand\enspace\textbf{ಆತ್ಮ ವಿಮರ್ಶೆ}\\
\begin{inspiration}{\mananamfont ಸ್ಪೂರ್ತಿ}
\footnotesize \mananamtext ಆಧುನಿಕ ಸಮಾಜದ ಬೌದ್ಧಿಕವಾದ ಮತ್ತು ಅತಿಯಾದ ಚಟುವಟಿಕೆಗಳು ರಾಜಸ ಸ್ವಭಾವ ಉಳ್ಳದ್ದಾಗಿದೆ. ರಾಜಸ ಸ್ವಭಾವವನ್ನು ಸರಿದೂಗಿಸಲು ತಮಸ್ಸನ್ನು ವರ್ಜಿಸಿ ಸತ್ವವನ್ನು ಸ್ವೀಕರಿಸಬೇಕು. ರಾಜಸ ಸ್ವಭಾವವು ತಾಮಸ ಸ್ವಭಾವಕ್ಕಿಂತ ಉತ್ತಮ ಎಂದು ಗೀತೆಯು ತೋರಿಸಿಕೊಡುತ್ತದೆ. ಹಾಗೆಯೇ ರಾಜಸ ಗುಣವು ಸತ್ವ ಗುಣದೊಂದಿಗೆ ಬೆರೆತಿದ್ದರೆ ಆದರ್ಶಪ್ರಾಯವಾಗಿರುತ್ತದೆ.
\end{inspiration}
\newpage

\slcol{\Index{ನೈವ ತಸ್ಯ ಕೃತೇನಾರ್ಥೋ} ನಾಕೃತೇನೇಹ ಕಶ್ಚನ ।\\
ನ ಚಾಸ್ಯ ಸರ್ವಭೂತೇಷು ಕಶ್ಚಿದರ್ಥವ್ಯಪಾಶ್ರಯಃ ॥ 18 ॥}
\cquote{ಅವನಿಗೆ ಮಾಡಿದ್ದರಿಂದಲೂ ಪ್ರಯೋಜನವಿಲ್ಲ, ಬಿಟ್ಟಿದ್ದರಿಂದ ಈ ಲೋಕದಲ್ಲಿ ಯಾವ ಹಾನಿಯೂ ಇಲ್ಲ.ಅವನಿಗೆ ಪ್ರಪಂಚದ ಯಾವ ಜೀವಿಯಿಂದಲೂ ಯಾವ ಪ್ರಯೋಜನದ ಅಪೇಕ್ಷೆಯೂ ಇಲ್ಲ.\\}
\slcol{\Index{ತಸ್ಮಾದಸಕ್ತಃ ಸತತಂ} ಕಾರ್ಯಂ ಕರ್ಮ ಸಮಾಚರ ।\\
ಅಸಕ್ತೋ ಹ್ಯಾಚರನ್ಕರ್ಮ ಪರಮಾಪ್ನೋತಿ ಪೂರುಷಃ ॥ 19 ॥ }
\cquote{ಆದ್ದರಿಂದ ಯಾವಾಗಲೂ ನಿನ್ನ ಕೆಲಸವನ್ನು ಫಲದಾಸೆ ಇಲ್ಲದೆ ಮಾಡು. ಫಲದಾಸೆ ಇಲ್ಲದೆ ಕರ್ಮವನ್ನು ಮಾಡುವವನು ಪರಮಾತ್ಮನನ್ನು ಪಡೆಯುತ್ತಾನೆ.\\}
\slcol{\Index{ಕರ್ಮಣೈವ ಹಿ} ಸಂಸಿದ್ಧಿಮಾಸ್ಥಿತಾ ಜನಕಾದಯಃ ।\\
ಲೋಕಸಂಗ್ರಹಮೇವಾಪಿ ಸಂಪಶ್ಯನ್ಕರ್ತುಮರ್ಹಸಿ ॥ 20 ॥}
\cquote{ಜನಕನೇ ಮೊದಲಾದವರು ಕರ್ಮದಿಂದಲೇ ಜ್ಞಾನ ಸಿದ್ದಿಯನ್ನು ಹೊಂದಿದರು. ಜನರಿಗೆ ದಾರಿಯನ್ನು ತೋರಿಸಬೇಕೆಂಬುದನ್ನಾದರೂ ಮನಸ್ಸಿಗೆ ತಂದು ನೀನು ಕರ್ಮವನ್ನು ಮಾಡತಕ್ಕದ್ದು.\\}
\slcol{\Index{ಯದ್ಯದಾಚರತಿ ಶ್ರೇಷ್ಠ}ಸ್ತತ್ತದೇವೇತರೋ ಜನಃ ।\\
ಸ ಯತ್ಪ್ರಮಾಣಂ ಕುರುತೇ ಲೋಕಸ್ತದನುವರ್ತತೇ ॥ 21 ॥}
\cquote{ದೊಡ್ಡವನೆನಿಸಿಕೊಂಡವನು ಏನೇನು ಮಾಡುತ್ತಾನೋ ಉಳಿದವರು ಅದನ್ನೇ ಮಾಡುತ್ತಾರೆ. ಅವನು ಯಾವುದನ್ನು ಸರಿ ಎಂದು ತಿಳಿದು ಮಾಡುತ್ತಾನೋ ಜನರು ಅದನ್ನು ಹಿಂಬಾಲಿಸುತ್ತಾರೆ.\\}
\slcol{\Index{ನ ಮೇ ಪಾರ್ಥಾಸ್ತಿ ಕರ್ತವ್ಯಂ} ತ್ರಿಷು ಲೋಕೇಷು ಕಿಂಚನ ।\\
ನಾನವಾಪ್ತಮವಾಪ್ತವ್ಯಂ ವರ್ತ ಏವ ಚ ಕರ್ಮಣಿ ॥ 22 ॥}
\cquote{ಅರ್ಜುನ,ನಾನು ಮಾಡಬೇಕಾದದ್ದೆಂಬುದು ಮೂರು ಲೋಕದಲ್ಲಿಯೂ ಏನೇನೂ ಇಲ್ಲ. ನಾನು ಕರ್ಮದಿಂದ ಪಡೆಯಬೇಕಾದ್ದು ಏನೂ ಇಲ್ಲ. ಆದರೂ ನಾನು ಕರ್ಮದಲ್ಲಿ ತೊಡಗಿಕೊಂಡೇ ಇದ್ದೇನೆ.\\}
\slcol{\Index{ಯದಿ ಹ್ಯಹಂ ನ} ವರ್ತೇಯಂ ಜಾತು ಕರ್ಮಣ್ಯತಂದ್ರಿತಃ ।\\
ಮಮ ವರ್ತ್ಮಾನುವರ್ತಂತೇ ಮನುಷ್ಯಾಃ ಪಾರ್ಥ ಸರ್ವಶಃ ॥ 23 ॥}
\cquote{ಅರ್ಜುನ,ನಾನು ಎಂದಿಗೂ ಆಲಸ್ಯವಿಲ್ಲದೆ ಕರ್ಮದಲ್ಲಿ ತೊಡಗದೆ ಇದ್ದರೆ ಜನರು ಎಲ್ಲ ಬಗೆಯಿಂದಲೂ ನನ್ನ ದಾರಿಯನ್ನು ಹಿಂಬಾಲಿಸಿ ಆಲಸರಾಗಿ ಬಿಡುತ್ತಾರೆ.\\}

\newpage
\begin{mananam}{\mananamfont ಮನನ ಶ್ಲೋಕ - \textenglish{18}}
\footnotesize \mananamtext ನಾನು ಈ ಪ್ರಪಂಚದ ಸಹಜ ಮತ್ತು ಅನಿರ್ವಾರ್ಯವಾದ ಕೆಲಸ ಕಾರ್ಯಗಳಿಗೆ ಮತ್ತು ನಿರೀಕ್ಷೆಗಳಿಗೆ ತಕ್ಕಂತೆ ಕೆಲಸ ಮಾಡಲು ಸಮರ್ಥನೆ? ಹಾಗಿದ್ದರೆ ಇದು ಆಂತರಿಕ ದಾಸ್ಯಕ್ಕೆ ಕಾರಣವಾಗಬಹುದೇ? ನಾನು ವಿಶೇಷವಾಗಿ ಬೇರೆ ವ್ಯಕ್ತಿಗಳ ಮೇಲಿನ ಅವಲಂಬನೆಗಳಿಂದ ಮಾನಸಿಕವಾಗಿ ಮತ್ತು ಭಾವನಾತ್ಮಕವಾಗಿ ಸ್ವಾತಂತ್ರ್ಯನಾಗಿದ್ದೇನೆಯೇ? ನಾನು ಬೇರೆಯವರಿಂದ ಯಾವ ರೀತಿಯ ವೈಯಕ್ತಿಕ ನಿರೀಕ್ಷೆಗಳಿಗೆ ಆಶ್ರಯವಾಗುತ್ತಿದ್ದೇನೆ?
\end{mananam}
\WritingHand\enspace\textbf{ಆತ್ಮ ವಿಮರ್ಶೆ}\\
\begin{inspiration}{\mananamfont ಸ್ಪೂರ್ತಿ}
\footnotesize \mananamtext ಒಬ್ಬ ಜ್ಞಾನಿಯಾದವನು ಯಾರಿಂದಲೂ ಯಾವುದರಿಂದಲೂ ಏನನ್ನು ನಿರೀಕ್ಷಿಸದೆ ಅವನೊಳಗೆ ಸಂತೃಪ್ತಿ ಹೊಂದಿರುತ್ತಾನೆ. ಇಂತಹ ಒಬ್ಬ ಸ್ವತಂತ್ರ ಜ್ಯಾನಿಯು ಸಮಾಜದ ಎಲ್ಲ ತರಹದ ಋಣಗಳಿಂದ ಮುಕ್ತನಾಗಿರುತ್ತಾನೆ. ಆದರೆ ಈ ಸಮಾಜವೇ ಅವನ ವ್ಯಕ್ತಿತ್ವವನ್ನು ಸಹಿಸಿಕೊಳ್ಳುವ ಋಣದಲ್ಲಿರುತ್ತದೆ ಮತ್ತು ಸಮಾಜದ ಎಲ್ಲರ ಕ್ಷೇಮಾಭಿವೃದ್ಧಿಗಾಗಿ ಅವನ ಅವಶ್ಯಕತೆ ಇರುತ್ತದೆ.
\end{inspiration}
\newpage

\begin{mananam}{\mananamfont ಮನನ ಶ್ಲೋಕ - \textenglish{19}}
\footnotesize \mananamtext ನನ್ನ ಕೆಲಸ ಮತ್ತು ಬೇರೆ ಕ್ರಿಯೆಗಳು ಮೆಚ್ಚಿಗೆ ಮತ್ತು ಮೆಚ್ಚದೇ ಇರುವ ಜ್ಞಾನದಿಂದ ಉಗಮವಾಗಿದೆಯೇ? ಕೆಲವು ಸಾರಿ ನನ್ನ ಕೆಲಸಗಳ ಮತ್ತು ಫಲಿತಾಂಶಗಳ ಮೇಲಿನ ನನ್ನ ಪ್ರೇಮ ಹೇಗೆ ಒತ್ತಡ ಮತ್ತು ವ್ಯಾಕುಲತೆಗಳಿಗೆ ಕಾರಣವಾಗುವುದನ್ನು ನೋಡುತ್ತೇನೆಯೇ? ಇದು ವಿಶ್ರಾಂತಿ ರಹಿತವಾದ ನಿದ್ರೆ ಹೀನ ಸ್ಥಿತಿಗೂ ಅಥವಾ ಬುದ್ಧಿ ಸ್ತಿಮಿತ ಕಳೆದುಕೊಳ್ಳುವ ವರೆಗೆ ಕರೆದುಕೊಂಡುಹೋಗುತ್ತಿದೆಯಾ? ನಾನು ಕೆಲಸದ ರೀತಿ ಮತ್ತು ಫಲಿತಾಂಶಗಳಿಗೆ ಅಂಟಿಕೊಳ್ಳದೆ ಕೆಲಸದಲ್ಲಿ ತೊಡಗಿಸಿಕೊಳ್ಳುವ ಮೂಲತತ್ವವನ್ನು ಹೇಗೆ ಅನ್ವಯಿಸಲಿ.
\end{mananam}
\WritingHand\enspace\textbf{ಆತ್ಮ ವಿಮರ್ಶೆ}\\
\begin{inspiration}{\mananamfont ಸ್ಪೂರ್ತಿ}
\footnotesize \mananamtext ಪ್ರತಿಯೊಬ್ಬ ಯಶಸ್ವಿ ವ್ಯಕ್ತಿ ತಾತ್ಕಾಲಿಕವಾಗಿಯಾದರೂ ಅನುಗ್ರಹಹಿತವಾದ ಸ್ಥಿತಿಯನ್ನು ಅನುಭವಿಸುತ್ತಾನೆ. ಇಲ್ಲಿ ಮನಸ್ಸು ವಿಶೇಷವಾದ ಒಂದೇ ಕೆಲಸದಲ್ಲಿ ಅಥವಾ ಅಧ್ಯಯನದಲ್ಲಿ ತೊಡಗಿಸುತ್ತದೆ. ಆಮೇಲೆ ಯಾರು ಅಥವಾ ಏನು ಉಳಿದಿರುವುದಿಲ್ಲ. ಕತೃ, ಕಾರ್ಯಾಚರಣೆ ಮತ್ತು ಮಾಡಬೇಕಾದ ಕೆಲಸ ಎಲ್ಲಾ ಒಂದೇ ಆಗಿರುತ್ತದೆ. ಈ ಅನುಗ್ರಹ ಹಿತವಾದ ಸ್ಥಿತಿ ಸಂತೋಷದ  ಉಚ್ಛ್ರಾಯ ಸ್ಥಿತಿ ಮತ್ತು ಜೀವನದ ಶಾಂತತೆಯನ್ನು ಒಳಗೊಂಡ ನಿರ್ವಾಣದ ತಾತ್ಕಾಲಿಕ ಅನುಭವವಾಗಿರುತ್ತದೆ.
\end{inspiration}
\newpage

\begin{mananam}{\mananamfont ಮನನ ಶ್ಲೋಕ - \textenglish{21}}
\footnotesize \mananamtext ನನ್ನ ಮಾದರಿ ವ್ಯಕ್ತಿ ಯಾರು? ಅವರು ನನಗೆ ಶಾಂತ ಮತ್ತು ಸ್ವಾತಂತ್ರ್ಯವಾಗಿರುವಂತೆ ಪ್ರೇರೇಪಿಸುತ್ತಾರೆಯೇ? ನನ್ನನ್ನು ಕರ್ತವ್ಯದಲ್ಲಿ ತೊಡಗಿಸಿಕೊಳ್ಳುವಂತೆ ಪ್ರೇರೇಪಿಸುವಂತ ಯಾವ ಮಹತ್ವವಾದದ್ದು ನನ್ನಲ್ಲಿ ಇದೆ? ಇದರಲ್ಲಿ ಇನ್ನೊಬ್ಬರ ಏಳಿಗೆಗೋಸ್ಕರ ಕೆಲಸ ಮಾಡುವಂಥದ್ದು ಒಳಗೊಂಡಿದೆಯೇ? ಜನರ ಒಳಿತನ್ನು ಬಯಸುವ ದೃಷ್ಟಿಕೋನವನ್ನು ಇಟ್ಟುಕೊಳ್ಳುವಂತಹ ಶಕ್ತಿಯನ್ನು ಬೆಳೆಸಿಕೊಳ್ಳುವ ಉಪಾಯವನ್ನು ಕಲ್ಪಿಸಿಕೊಳ್ಳಬಲ್ಲೆನೇ?
\end{mananam}
\WritingHand\enspace\textbf{ಆತ್ಮ ವಿಮರ್ಶೆ}\\
\begin{inspiration}{\mananamfont ಸ್ಪೂರ್ತಿ}
\footnotesize \mananamtext ಜನರಿಗೋಸ್ಕರ ಒಳಿತನ್ನು ಬಯಸಿ ಮಾಡಿದ ತ್ಯಾಗಗಳಿಂದಲೇ ಜನರು ಅವರು ಅಳಿದ ನಂತರ ಅವರ ಆತ್ಮ ಜೀವಿಸುತ್ತದೆ. ಸ್ವಾರ್ಥಯುತ ಅಭಿಲಾಷೆಗಳು ಜನರಲ್ಲಿ ತುಂಬಾ ಒತ್ತಡವನ್ನು ಸೃಷ್ಟಿಸುತ್ತದೆ ಯಾವಾಗಲೂ ಜನರಿಗಾಗಿ ಕೆಲಸ ಮಾಡುವಂತಹ ಬಯಕೆಗಳು ವ್ಯಕ್ತಿಯನ್ನು ಮತ್ತು ಸಮಾಜವನ್ನು ಉದ್ದಾರ ಮಾಡುತ್ತದೆ.
\end{inspiration}
\newpage

\begin{mananam}{\mananamfont ಮನನ ಶ್ಲೋಕ - \textenglish{22}}
\footnotesize \mananamtext ಈ ಸೃಷ್ಟಿಯನ್ನು ನಡೆಸುವವರು ಯಾರು? ಅವನ ಅಥವಾ ಅವಳ ಉದ್ದೇಶ ಏನಿರಬಹುದು. ನಾನು ಪ್ರಭಾವಿ ಶಕ್ತಿಗಳಾದ ಸೂರ್ಯ, ಚಂದ್ರ, ವಾಯು ಮತ್ತು ಸಮುದ್ರ ಇತ್ಯಾದಿಗಳಿಂದ ನಿಸ್ವಾರ್ಥವಾಗಿ ಕರ್ತವ್ಯ ಮಾಡುವುದನ್ನು ಕಲಿಯಬಹುದೇ? ಇವರನ್ನೆಲ್ಲ ನಡೆಸಿಕೊಂಡು ಹೋಗುತ್ತಿರುವುದು ಯಾವುದು? ಅವುಗಳ ಮೂಲ ಯಾವುದು ಮತ್ತು ಅವುಗಳ ಪೋಷಣೆ ಯಾವುದರಿಂದ ಆಗುತ್ತಿದೆ?
\end{mananam}
\WritingHand\enspace\textbf{ಆತ್ಮ ವಿಮರ್ಶೆ}\\
\begin{inspiration}{\mananamfont ಸ್ಪೂರ್ತಿ}
\footnotesize \mananamtext ಈ ಜಗತ್ತಿಗೆ ಚಾಲನೆ ಕೊಟ್ಟು ನಡೆಸುತ್ತಿರುವುದು ಪ್ರೀತಿ ಮತ್ತು ಕರುಣೆ ಎಂಬ ದೊಡ್ಡ ಶಕ್ತಿಗಳು. ನಿಜವಾದ ಪ್ರೀತಿ ಎಂದರೆ ಎಲ್ಲರನ್ನೂ ಎಲ್ಲವನ್ನು ತನ್ನದೇ ಭಾಗವೆಂದು ನೋಡುವುದು. ನಮ್ಮ ಪ್ರೀತಿ ಪಾತ್ರರ ಒಳಿತಿಗಾಗಿ ಕಾರ್ಯನಿರ್ವಹಿಸುವುದನ್ನು ಯಾವತ್ತೂ ಹೊರೆ ಎಂದು ಭಾವಿಸಬಾರದು. ಯಾವಾಗ ನಾವು ಮಾಡುವ ಕಾರ್ಯಗಳು ಪ್ರೀತಿ ಎಂಬ ಜಾಗದಿಂದ ಬರುತ್ತದೆಯೋ ಆಗ ನಮ್ಮ ಜೀವನವು ಸಂಪೂರ್ಣ ಸಮರಸದಿಂದ ಕೂಡಿರುತ್ತದೆ.
\end{inspiration}
\newpage


\slcol{\Index{ಉತ್ಸೀದೇಯುರಿಮೇ ಲೋಕಾ} ನ ಕುರ್ಯಾಂ ಕರ್ಮ ಚೇದಹಮ್ ।\\
ಸಂಕರಸ್ಯ ಚ ಕರ್ತಾ ಸ್ಯಾಮುಪಹನ್ಯಾಮಿಮಾಃ ಪ್ರಜಾಃ ॥ 24 ॥}
\cquote{ನಾನು ಕರ್ಮವನ್ನು ಮಾಡದೆ ಹೋದರೆ ಈ ಲೋಕಗಳು ಹಾಳಾದವು, ನಾನು ಧರ್ಮದ   ಕಲಬರಕೆಗೂ ಕಾರಣನಾದನು, ಈ ಜನರ ಪತನಕ್ಕೆ ಕಾರಣನಾದೇನು.\\}
\slcol{\Index{ಸಕ್ತಾಃ ಕರ್ಮಣ್ಯವಿದ್ವಾಂಸೋ} ಯಥಾ ಕುರ್ವಂತಿ ಭಾರತ ।\\
ಕುರ್ಯಾದ್ವಿದ್ವಾಂಸ್ತಥಾಸಕ್ತಶ್ಚಿಕೀರ್ಷುರ್ಲೋಕಸಂಗ್ರಹಮ್ ॥ 25 ॥} 
\cquote{ಅರ್ಜುನ, ತಿಳಿಯದವರು ಫಲದಾಸೆಯಿಂದ ಕರ್ಮವನ್ನು ಹೇಗೆ ಮಾಡುತ್ತಾರೋ ಹಾಗೆ ತಿಳಿದವರು ಜನರಿಗೆ ಮಾದರಿಯಾಗಬೇಕೆಂದು ಫಲದಾಸೆ ಇಲ್ಲದೆ ಮಾಡಬೇಕು. ತಿಳುವಳಿಕೆ ಇಲ್ಲದೆ ಕರ್ಮದ ಫಲವನ್ನು ಬಯಸುವವರಿಗೆ ಬುದ್ಧಿ ಭೇದ ಮಾಡಿ ಅವರನ್ನು ಗೊಂದಲಗೊಳಿಸಬಾರದ. ತಿಳಿದವನು ಫಲದಾಸೆ ಇಲ್ಲದೆ ಮಾಡುತ್ತಾ ಎಲ್ಲ ಕರ್ಮಗಳನ್ನೂ ಮಾಡಿಸಬೇಕು.\\}
\slcol{\Index{ನ ಬುದ್ಧಿಭೇದಂ ಜನಯೇದ}ಙ್ಞಾನಾಂ ಕರ್ಮಸಂಗಿನಾಮ್ ।\\
ಜೋಷಯೇತ್ಸರ್ವಕರ್ಮಾಣಿ ವಿದ್ವಾನ್ಯುಕ್ತಃ ಸಮಾಚರನ್ ॥ 26 ॥}
\cquote{\textenglish missing?}
\slcol{\Index{ಪ್ರಕೃತೇಃ ಕ್ರಿಯಮಾಣಾನಿ} ಗುಣೈಃ ಕರ್ಮಾಣಿ ಸರ್ವಶಃ ।\\
ಅಹಂಕಾರವಿಮೂಢಾತ್ಮಾ ಕರ್ತಾಹಮಿತಿ ಮನ್ಯತೇ ॥ 27 ॥}
\cquote{ಅಹಂಕಾರದಿಂದ ತಲೆಕೆಡಿಸಿಕೊಂಡವನು ಮಾಯೆಯ ಅಧೀನವಾಗಿ ಇಂದ್ರಿಯಗಳಿಗಳಿಂದಾಗುವ ಕರ್ಮಗಳನ್ನು ತಾನೇ ಮಾಡುವುದೆಂದು ತಿಳಿಯುತ್ತಾನೆ.\\}
\slcol{\Index{ಮಹಾಬಾಹೋ ಗುಣ}ಕರ್ಮವಿಭಾಗಯೋಃ ।\\
ಗುಣಾ ಗುಣೇಷು ವರ್ತಂತ ಇತಿ ಮತ್ವಾ ನ ಸಜ್ಜತೇ ॥ 28 ॥}
\cquote{ಅರ್ಜುನ, ಗುಣಗಳ ಮತ್ತು ಕರ್ಮಗಳ ವಿಂಗಡದ ನಿಜವನ್ನರಿತವನಾದರೋ ಇಂದ್ರಿಯ ಮತ್ತು ವಿಷಯಗಳ ಸಂಬಂಧದ ತಿರುಳನ್ನು ತಿಳಿದು ನಾನು ಮಾಡುವವನೆಂದು ಅಭಿಮಾನಕೊಳ್ಳುವುದಿಲ್ಲ.\\}
\slcol{\Index{ಪ್ರಕೃತೇರ್ಗುಣಸಂಮೂಢಾಃ} ಸಜ್ಜಂತೇ ಗುಣಕರ್ಮಸು ।\\
ತಾನಕೃತ್ಸ್ನವಿದೋ ಮಂದಾನ್ಕೃತ್ಸ್ನವಿನ್ನ ವಿಚಾಲಯೇತ್ ॥ 29 ॥}
\cquote{ಇಂದ್ರಿಯಗಳ ಮಾಯೆಗೆ ಒಳಗಾದವರು ವಿಷಯ ಮೋಹದಲ್ಲಿ ಮುಳುಗಿಬಿಡುತ್ತಾರೆ. ತತ್ವದ ತಿರುಳು ತಿಳಿದಿಲ್ಲದ ಆ ದಡ್ಡರನ್ನು ಚೆನ್ನಾಗಿ ತಿಳಿದವರು ಕದಲಗೊಡಬಾರದು.\\}

\newpage
\begin{mananam}{\mananamfont ಮನನ ಶ್ಲೋಕ - \textenglish{25}}
\footnotesize \mananamtext ನನ್ನ ಜೀವನದಲ್ಲಿ ಜ್ಞಾನದ ಪಾತ್ರವೇನು? ಹೇಗೆ ಪ್ರಬುದ್ಧತೆಯು ನಾನು ಜೀವನದಲ್ಲಿ ನೋಡುವ ರೀತಿಯನ್ನು ಬದಲಾಯಿಸಿತು. ನನಗೆ ಉತ್ತಮ ಶ್ರೇಣಿಯಲ್ಲಿ ಕಾರ್ಯನಿರ್ವಹಿಸಲು ಮಾರ್ಗದರ್ಶನ ಮಾಡುತ್ತಿರುವ ನಂಬಿಕಸ್ಥ ಮೂಲ ಯಾರು ಅಥವಾ ಏನು? ನಾನು ಹೇಗೆ ಪ್ರಾಪಂಚಿಕ ಜ್ಞಾನದಿಂದ ಋಷಿಗಳ ಮತ್ತು ಬುದ್ಧಿವಂತರ ಜ್ಞಾನಕ್ಕೆ ರೂಪಾಂತರಗೊಳ್ಳಲಿ?
\end{mananam}
\WritingHand\enspace\textbf{ಆತ್ಮ ವಿಮರ್ಶೆ}\\
\begin{inspiration}{\mananamfont ಸ್ಪೂರ್ತಿ}
\footnotesize \mananamtext ಒಬ್ಬರು ಅವರಿಗೋಸ್ಕರ ಜೀವಿಸುವುದು ಮತ್ತು ಉಳಿವಿಗಾಗಿ ಹೋರಾಡುವುದು ಸಹಜ ಪ್ರವೃತ್ತಿ. ಆದರೆ ಮನುಷ್ಯರಾಗಿ ಬುದ್ಧಿವಂತೆಯ ಜೊತೆಗೆ ನಡೆಯುತ್ತಿದ್ದೇವೆ. ನಮ್ಮ ಬುದ್ಧಿವಂತೆಯನ್ನು ಉಪಯೋಗಿಸಿಕೊಂಡು ನಾವು ನಮಗೋಸ್ಕರ ಮಾತ್ರ ಕೆಲಸ ಮಾಡದೆ, ಮಾನವೀಯತೆಗಾಗಿ ಕೆಲಸ ಮಾಡಬೇಕು.
\end{inspiration}
\newpage

\begin{mananam}{\mananamfont ಮನನ ಶ್ಲೋಕ - \textenglish{26}}
\footnotesize \mananamtext ಯಾರು ಆಧ್ಯಾತ್ಮಿಕ ದಾರಿಯಲ್ಲಿ ಪ್ರಗತಿ ಕಾಣುತ್ತಾರೋ ಅಥವಾ ಉನ್ನತವಾದ ಆಶ್ರಮಗಳಿಗೆ ಹೋಗುತ್ತಾರೋ ಅವರು ಹೊಸ ಆಕಾಂಕ್ಷಿಗಳಿಗೆ ಕೊಟ್ಟ ಉಪದೇಶ ಮತ್ತು ಅಭ್ಯಾಸಗಳನ್ನು ಕೀಳಾಗಿ ಕಾಣದೆ ಇರುವ ಜವಾಬ್ದಾರಿಯನ್ನು ಹೊಂದಿರುತ್ತಾರೆ. ಬೇರೆ ಬೇರೆ ಆಕಾಂಕ್ಷಿಗಳಿಗೆ ಬೇರೆಬೇರೆ ಹಂತದಲ್ಲಿ ಅಧ್ಯಾತ್ಮಿಕ ಬೋಧನೆಗಳನ್ನು ನೀಡಲಾಗುತ್ತದೆ. ಅಧ್ಯಾತ್ಮದಲ್ಲಿ ಮೇಲೇರಿದವರಿಗೆ ಮಾತ್ರ ಎಲ್ಲಾ ತರಹದ ಸಹಾಯ ಮತ್ತು ಆಶ್ರಯವನ್ನು ಬಿಡಲು ಸಾಧ್ಯವಾಗುತ್ತದೆ.
\end{mananam}
\WritingHand\enspace\textbf{ಆತ್ಮ ವಿಮರ್ಶೆ}\\
\begin{inspiration}{\mananamfont ಸ್ಪೂರ್ತಿ}
\footnotesize \mananamtext ಗುರಿಯನ್ನು ಮಾತ್ರ ಕೇಂದ್ರೀಕರಿಸಿದರೆ ಒತ್ತಡಕ್ಕೆ ಒಳಗಾಗುವುದು ಖಂಡಿತ. ಕಾರ್ಯ ವಿಧಾನದ ಬಗ್ಗೆ ಕೇಂದ್ರೀಕರಿಸಿದಾಗ ನಮ್ಮ ಮುಂದೆ ಇರುವುದನ್ನು ನಿರ್ವಹಿಸಲು ನಮಗೆ ಸಾಧ್ಯವಾಗುತ್ತದೆ. ನಾನು ನನ್ನದೆಂಬ ಪ್ರವೃತ್ತಿಯಿಂದ ಹೊರಗೆ ಬರುವುದು, ಇಷ್ಟಪಡುವುದು ಮತ್ತು ಇಷ್ಟಪಡದೇ ಇರುವುದರಿಂದ ಹೊರಗೆ ಬಂದಾಗ ಮಾತ್ರ ಶ್ರೇಷ್ಠ ಗುಣಗಳ ಕಡೆಗೆ ಮತ್ತು ದೈವ ಸ್ವರೂಪದ ಕಡೆಗೆ ಹೋಗಬಹುದು.
\end{inspiration}
\newpage

\begin{mananam}{\mananamfont ಮನನ ಶ್ಲೋಕ - \textenglish{27}}
\footnotesize \mananamtext ಕೆಲಸ ಮಾಡಲು ಬೇಕಾಗುವ ಕೆಲವು ಒಳಗಿನ ಅಂಗಗಳಾದ ಉಸಿರಾಟ ಮತ್ತು ಪಚನಕ್ರಿಯೆಯ ಅಂಗಗಳನ್ನು ಹತೋಟಿಯಲ್ಲಿಟ್ಟುಕೊಳ್ಳುತ್ತಿದ್ದೇಯೇ? ನನಗೆ ಮಾತನಾಡಲು ಸಶಕ್ತನಾಗುವಂತೆ ಪ್ರಕೃತಿಯು ಉತ್ತೇಜಿಸುವ ಶರೀರ ಶಾಸ್ತ್ರದ ಪದ್ಧತಿಗಳಿಲ್ಲವೇ? ಪ್ರಕೃತಿಯ ಭೌತಿಕ ವಸ್ತುಗಳ ಪೋಷಣೆ ಇಲ್ಲದೆ ಮತ್ತು ಪ್ರಕೃತಿಯ ಉತ್ತೇಜನವಿಲ್ಲದೆ ಯೋಚಿಸಲು ಮತ್ತು ತರ್ಕಿಸಲು ನನ್ನ ಮನಸ್ಸಿಗೆ ಶಕ್ತಿ ಇದೆಯೇ? ನಾನು ನನ್ನದೆಂದು ಹೇಳುವ ಯಾರೆಲ್ಲಾ ಅಥವಾ ಏನೆಲ್ಲಾ ಕೌಶಲ್ಯಗಳು ಮತ್ತು ಸಾಮರ್ಥ್ಯಗಳು ಎಲ್ಲವೂ ಅದರ ಅಭಿವೃದ್ಧಿಗೆ ಹೋಗಿವೆ. ಇದನ್ನೆಲ್ಲಾ ಪರ್ಯಾಲೋಚಿಸಿದರೆ, ಜೀವನದಲ್ಲಿ ಕೆಲವು ಸಾಧನೆಗಳು ಮತ್ತು ಸಿದ್ದಿಗಳು ನನ್ನದೆಂದು ಹೇಳುವ ನಾನು ಒಬ್ಬ ಕಾರ್ಯ ಪ್ರವರ್ತಕನೇ?
\end{mananam}
\WritingHand\enspace\textbf{ಆತ್ಮ ವಿಮರ್ಶೆ}\\
\begin{inspiration}{\mananamfont ಸ್ಪೂರ್ತಿ}
\footnotesize \mananamtext ನಾನು ಮತ್ತು ನನಗೆ ಎನ್ನುವುದನ್ನು ಒಳಗೊಂಡಿರುವುದನ್ನು ನಾನು ಯಾವಾಗ ನಿಷ್ಕಪಟವಾಗಿ ಅನ್ವೇಶಿಸುತ್ತೇವೆಯೋ,ಆಗ ನಾವು ಯಾವ ಅಸ್ತಿತ್ವದಲ್ಲಿ ಇಲ್ಲದಿರುವುದನ್ನು ಕಾಣಬಹುದು. ನಾವೇ ಎಲ್ಲಾ ಎಂಬ ತಪ್ಪು ಕಲ್ಪನೆಯು ಕುಸಿದಾಗ ನಾವು ಏನೂ ಅಲ್ಲ ಎನ್ನುವ ವಾಸ್ತಿವಕಥೆಯನ್ನು ಅರಿತಾಗ ಅದುವೇ ನಮ್ಮ ಉನ್ನತವಾದ ಸ್ವಾತಂತ್ರತತ್ತ್ವವಿತ್ತು 
\end{inspiration}
\newpage

\slcol{\Index{ಮಯಿ ಸರ್ವಾಣಿ ಕರ್ಮಾಣಿ} ಸಂನ್ಯಸ್ಯಾಧ್ಯಾತ್ಮಚೇತಸಾ ।\\
ನಿರಾಶೀರ್ನಿರ್ಮಮೋ ಭೂತ್ವಾ ಯುಧ್ಯಸ್ವ ವಿಗತಜ್ವರಃ ॥ 30 ॥}
\cquote{ಎಲ್ಲರೊಳಗೂ ನಾನು ಇರುವೆನೆಂದರಿತು ಎಲ್ಲ ಕರ್ಮಗಳನ್ನು ನನಗೊಪ್ಪಿಸಿ ಫಲದ ಬಯಕೆಯನ್ನು ನನ್ನದೆಂಬ ಅಭಿಮಾನವನ್ನು ತೊರೆದು ನಿಶ್ಚಿಂತನಾಗಿ ಯುದ್ಧ ಮಾಡು.\\}
\slcol{\Index{ಯೇ ಮೇ ಮತಮಿದಂ} ನಿತ್ಯಮನುತಿಷ್ಠಂತಿ ಮಾನವಾಃ ।\\
ಶ್ರದ್ಧಾವಂತೋऽನಸೂಯಂತೋ ಮುಚ್ಯಂತೇ ತೇऽಪಿ ಕರ್ಮಭಿಃ ॥ 31 ॥}
\cquote{ಈ ನನ್ನ ಅಭಿಪ್ರಾಯವನ್ನು ಯಾರು ಅಸೂಯೆ ತಾಳದೆ ಯಾವಾಗಲೂ ನನ್ನ ಮೇಲಿನ ವಿಶ್ವಾಸದಿಂದ ಆಚರಿಸುತ್ತಾರೋ ಅವರು ಕೂಡ ಕರ್ಮ ಬಂಧನದಿಂದ ಬಿಡುಗಡೆಯನ್ನು ಹೊಂದುತ್ತಾರೆ.\\}
\slcol{\Index{ಯೇ ತ್ವೇತದಭ್ಯಸೂಯಂತೋ} ನಾನುತಿಷ್ಠಂತಿ ಮೇ ಮತಮ್ ।\\
ಸರ್ವಙ್ಞಾನವಿಮೂಢಾಂಸ್ತಾನ್ವಿದ್ಧಿ ನಷ್ಟಾನಚೇತಸಃ ॥ 32 ॥}
\cquote{ಅಸೂಯೆಯಿಂದ ನನ್ನ ಅಭಿಪ್ರಾಯವನ್ನು ಆಚರಣೆಗೆ ತರದೆ ತಿರಸ್ಕರಿಸುವವರು ಜ್ಞಾನದ ಮಾರ್ಗವನ್ನೇ ಅರಿಯದ ಅವಿವೇಕಿಗಳು.ಅವರು ತಮ್ಮ ನಾಶವನ್ನು ತಾವೇ ಮಾಡಿಕೊಳ್ಳುವರೆಂದು ತಿಳಿ.\\}
\slcol{\Index{ಸದೃಶಂ ಚೇಷ್ಟತೇ ಸ್ವಸ್ಯಾಃ} ಪ್ರಕೃತೇರ್ಙ್ಞಾನವಾನಪಿ ।\\
ಪ್ರಕೃತಿಂ ಯಾಂತಿ ಭೂತಾನಿ ನಿಗ್ರಹಃ ಕಿಂ ಕರಿಷ್ಯತಿ ॥ 33 ॥}
\cquote{ಬಲ್ಲವನು ಕೂಡ ತನ್ನ ಸ್ವಭಾವಕ್ಕೆ ಸರಿಯಾಗಿ ನಡೆಯುವನು.ಎಲ್ಲ ಪ್ರಾಣಿಗಳು ಹುಟ್ಟುಗುಣವನ್ನು ಹಿಂಬಾಲಿಸುತ್ತವೆ. ನಿಗ್ರಹದಿಂದ ಏನು ನಡೆಯದು.\\}
\slcol{\Index{ಇಂದ್ರಿಯಸ್ಯೇಂದ್ರಿಯಸ್ಯಾರ್ಥೇ} ರಾಗದ್ವೇಷೌ ವ್ಯವಸ್ಥಿತೌ ।\\
ತಯೋರ್ನ ವಶಮಾಗಚ್ಛೇತ್ತೌ ಹ್ಯಸ್ಯ ಪರಿಪಂಥಿನೌ ॥ 34 ॥}
\cquote{ಪ್ರತಿಯೊಂದು ಇಂದ್ರಿಯದ ವಿಷಯಗಳಲ್ಲೂ ರಾಗ ದ್ವೇಷಗಳು ನೆಲೆಸಿವೆ. ಅವುಗಳಿಗೆ ಅಡಿಯಾಳಾಗಬಾರದು. ಈ ರಾಗ ದ್ವೇಷಗಳೆ ಸಾಧಕನಿಗೆ ಶತ್ರುಗಳು.\\}
\slcol{\Index{ಶ್ರೇಯಾನ್ಸ್ವಧರ್ಮೋ ವಿಗುಣಃ} ಪರಧರ್ಮಾತ್ಸ್ವನುಷ್ಠಿತಾತ್ ।\\
ಸ್ವಧರ್ಮೇ ನಿಧನಂ ಶ್ರೇಯಃ ಪರಧರ್ಮೋ ಭಯಾವಹಃ ॥ 35 ॥} 
\cquote{ನಗೆ ಅಸಹಜವಾದ ಧರ್ಮವನ್ನು ಚೆನ್ನಾಗಿ ನಡೆಸುವುದಕ್ಕಿಂತ ಕಿಂಚಿದೂನವಾದರೂ ಸಹಜ ಧರ್ಮವನೆ ಆಚರಿಸುವುದೆ ಮೇಲು. ತನ್ನ ಧರ್ಮದಲ್ಲಿ ಸಾಯುವುದಾದರೂ ಮೇಲು. ಪರಧರ್ಮವು ಆಪತ್ತಿಗೆ ಆಹ್ವಾನ.}

\newpage
\begin{mananam}{\mananamfont ಮನನ ಶ್ಲೋಕ - \textenglish{30}}
\footnotesize \mananamtext ನಾನು ಯಾವಾಗ ನನ್ನ ಉತ್ಕೃಷ್ಟ ಪ್ರಯತ್ನ ಹಾಕುತ್ತೇನೆಯೋ ಆಗ ನಾನು ಉದ್ವೇಗಕ್ಕೆ ಒಳಗಾಗುತ್ತೇನೆಯೇ? ನಾನು ಒತ್ತಡ ಮತ್ತು ಆಯಾಸಕ್ಕೆ ಒಳಗಾಗುತ್ತೇನೆಯೇ?  ನಾನು ಸಹೋದ್ಯೋಗಿಗಳೊಡನೆ ಕುಟುಂಬದ ಸದಸ್ಯರೊಡನೆ ಘರ್ಷಣೆ ಮತ್ತು ಮಾನಸಿಕ ಹಿಂಸೆಗೆ ಒಳಗಾಗುತ್ತಿದ್ದೇನೆಯೇ? ಎಲ್ಲವನ್ನೂ ಬಿಟ್ಟು ಬಿಡುವುದರ ಅರ್ಥವೇನು? ನಾನು ತೀವ್ರವಾದ ಪ್ರಯತ್ನಗಳನ್ನು ರಾಜಿಮಾಡಿಕೊಳ್ಳುವ ಬದಲು ಫಲಿತಾಂಶದ ಮೇಲಿನ ಮೋಹವನ್ನು ತ್ಯಜಿಸಲು ಸಾಧ್ಯವಿದೆಯೇ?
\end{mananam}
\WritingHand\enspace\textbf{ಆತ್ಮ ವಿಮರ್ಶೆ}\\
\begin{inspiration}{\mananamfont ಸ್ಪೂರ್ತಿ}
\footnotesize \mananamtext ಯಾರಲ್ಲಿ ಶ್ರದ್ದೆ ಇರುತ್ತದೆಯೋ ಅವರಿಗೆ ಉದ್ವೇಗ ರಹಿತ ತೀಕ್ಷ್ಣ  ಮಟ್ಟದ ಪ್ರಯತ್ನದ ಗುಟ್ಟು ಗೊತ್ತಿರುತ್ತದೆ. ಯಾವಾಗ ನಾವು ನಮ್ಮ ಜೀವನವನ್ನು ಮಾನಸಿಕವಾಗಿ ಹೆಚ್ಚಿನ ಬಲದಲ್ಲಿ ಒಪ್ಪಿಕೊಳ್ಳುತ್ತಿದ್ದೇವೆಯೋ, ಆಗ ದೊಡ್ಡ ಭಾರವೂಂದು ನಮ್ಮಿಂದ ದೂರವಾಗುತ್ತದೆ. ಆಗ ನಾವು ನಮಗೆ ಅರ್ಹತೆ ಇರುವುದನ್ನು ಯಾರಿಂದಲೂ ನಿರಾಕರಿಸಲು ಸಾಧ್ಯವಿಲ್ಲವೆಂದು ತಿಳಿದುಕೊಂಡು ನಮ್ಮ ಉತ್ತಮವಾದ ಪ್ರಯತ್ನವನ್ನು ಹಾಕಿ ಕಾರ್ಯ ಮಾಡಬೇಕು.
\end{inspiration}
\newpage

\begin{mananam}{\mananamfont ಮನನ ಶ್ಲೋಕ - \textenglish{31,32}}
\footnotesize \mananamtext ನಾನು ಯಾವ ವರ್ಗಕ್ಕೆ ಸೇರುತ್ತೇನೆ? ಈ ಏಳಿಗೆ ತರುವಂತ ಉಪದೇಶಗಳಿಗೆ ಮನಸ್ಸು ಮತ್ತು ಆಸಕ್ತಿ ಇರುವವರ ಗುಂಪಿಗೋ ಅಥವಾ ಇದರ ಪರಿವರ್ತನೆಯ ಪರಿಣಾಮದಿಂದ ದೂರವಿರುವವರ ಅಥವಾ ನಂಬಿಕೆ ಇಲ್ಲದವರ ಗುಂಪಿಗೋ? ನಾನು ಸ್ವಲ್ಪವಾದರೂ ಈ ಬೋಧನೆಗಳನ್ನು ಅಧ್ಯಯನ ಮಾಡಿ ಅರ್ಥಮಾಡಿಕೊಳ್ಳಲು ಪ್ರಯತ್ನಿಸುತ್ತಿದ್ದೇನೆಯೇ? ಅಳವಡಿಸಿಕೊಳ್ಳಲು ಪ್ರಯತ್ನಿಸುತ್ತಿದ್ದೇನೆಯೇ? ಅಥವಾ ಇದರ ವಿರುದ್ಧವಾಗಿ ಪಕ್ಷಪಾತ ಮಾಡುತ್ತಿದ್ದೇನೆಯೇ?
\end{mananam}
\WritingHand\enspace\textbf{ಆತ್ಮ ವಿಮರ್ಶೆ}\\
\begin{inspiration}{\mananamfont ಸ್ಪೂರ್ತಿ}
\footnotesize \mananamtext ಈ ತರಹದ ಬೋಧನೆಗಳು ನಾವು ನಮ್ಮ ಜೀವನವನ್ನು ಸರಿಯಾದ ಕ್ರಮದಲ್ಲಿ ನಡೆಸಲು ನಮಗೆ ಒಂದು ಅನುಗ್ರಹ ರೂಪದಲ್ಲಿ ವಂಶಪಾರಂಪರ್ಯವಾಗಿ ಬಂದಂತಹ ನಂಬಿಕೆ. ಈ ತರಹದ ನಂಬಿಕೆಯು ವಂಶಪರಂಪರ್ಯವಾಗಿ ಬಂದಿರದಿದ್ದಲ್ಲಿ ಈ ಸನಾತನ ಗ್ರಂಥಗಳು ಇದನ್ನು ಪರೀಕ್ಷಿಸಲು ನಮಗೆ ದಾರಿ ಮಾಡಿಕೊಡುತ್ತವೆ. ಆದರೂ ನಾವು ಇದಾವುದನ್ನು ಮಾಡಲು ವಿಫಲರಾದರೆ ಆಮೇಲೆ ನಾವೇ ಜೀವನದ ಪಾಠಗಳನ್ನು ಕಷ್ಟಕರ ರೀತಿಯಲ್ಲಿ ಕಲಿಯಬೇಕಾಗುತ್ತದೆ. 
\end{inspiration}
\newpage

\begin{mananam}{\mananamfont ಮನನ ಶ್ಲೋಕ - \textenglish{33}}
\footnotesize \mananamtext ನನಗೆ ಯಾವ ಅಭ್ಯಾಸಗಳನ್ನು ಬಿಡಲು ತುಂಬಾ ಕಷ್ಟಕರವಾಗಿರುವಂತೆ ಕಾಣುವುದು ಯಾವುದು? ನನಗೆ ಈ ಅಭ್ಯಾಸಗಳ ಋಣಾತ್ಮಕವಾದ ಮುಖದ ಬಗ್ಗೆ ನನಗೆ ತಿಳಿದಿದೆಯೇ? ಈ ಅಭ್ಯಾಸಗಳ ಮೇಲೆ ಒಂದು ಭಾವನಾತ್ಮಕವಾದ ಅವಲಂಬನೆಯೂ ಇದೆಯೇ? ಈ ಅಭ್ಯಾಸಗಳಿಗೆ ವಿರಾಮ ಕೊಡಲು ಸತತ ಪ್ರಯತ್ನ ಮಾಡುವ ನಿರ್ಧಾರ ಕೈಗೊಳ್ಳುತ್ತಿದ್ದೇನೆಯೇ? ಪ್ರಲೋಭೆನೆಗಳ ಆಕ್ರಮಣಗಳನ್ನು ನಿರ್ವಹಿಸುವಲ್ಲಿ ನಾನು ಎಷ್ಟರಮಟ್ಟಿಗೆ ಮನಸ್ಸು ಮಾಡಬಹುದು. ಈ ಅಭ್ಯಾಸಗಳನ್ನು ಸುಡಿಲಗೊಳಿಸಲು ಬೇರೆ ಬೇರೆ ಹಂತಗಳಲ್ಲಿ ಕೆಲಸ ಮಾಡುವ, ಯೋಗದ ಜೀವನಕ್ರಮ ಮತ್ತು ಅಧ್ಯಾತ್ಮಿಕ ಅಭ್ಯಾಸಗಳನ್ನು ನಾನು ಹೇಗೆ ಅಪ್ಪಿಕೊಳ್ಳಲಿ.
\end{mananam}
\WritingHand\enspace\textbf{ಆತ್ಮ ವಿಮರ್ಶೆ}\\
\begin{inspiration}{\mananamfont ಸ್ಪೂರ್ತಿ}
\footnotesize \mananamtext ಪ್ರಲೋಭನೆಗಳ ಮುಂದೆ ಎಲ್ಲಾ ಸಕರಾತ್ಮಕ ಬುದ್ಧಿವಂತಿಕೆಗಳು ಮತ್ತು ಆಸೆಗಳು ಬಿದ್ದು ಹೋಗುತ್ತವೆ. ವ್ಯಸನದ ಶಕ್ತಿ ಎಷ್ಟೆಂದರೆ ಅಪಾಯಕಾರಿ ಪರಿಣಾಮಗಳ ಬಗ್ಗೆ ಗೊತ್ತಿದ್ದರೂ ಕೂಡ ವಸ್ತುಗಳ ಆಕರ್ಷಣೀಯ ಉಪಸ್ಥಿತಿಯಲ್ಲಿ ಒಬ್ಬ ವ್ಯಸೆನಿಯು ತನ್ನನ್ನು ತಾನು ತಡೆಹಿಡಿದುಕೊಳ್ಳಲು ಸಾಧ್ಯವಾಗುವುದಿಲ್ಲ. ನಿಯಮಿತವಾದ ಅಧ್ಯಾತ್ಮದ ಅಭ್ಯಾಸಗಳಿಂದ ಮತ್ತು ದೃಢವಾದ ಮನಸ್ಸಿನಿಂದ ಮಾತ್ರ ಪ್ರಲೋಭನೆಗಳಿಂದ ಮುಕ್ತವಾಗಲು ಸಾಧ್ಯ.
\end{inspiration}
\newpage

\begin{mananam}{\mananamfont ಮನನ ಶ್ಲೋಕ - \textenglish{35}}
\footnotesize \mananamtext ಈ ಜೀವನದಲ್ಲಿ ನನ್ನ ಜಾಣ್ಮೆಗಳು ಮತ್ತು ಸಾಮರ್ಥ್ಯಗಳು ಏನೇನು? ನನ್ನ ಕೆಲಸ ಮತ್ತು ಚಟುವಟಿಕೆಯ ದಾರಿಯಲ್ಲಿ ಅವುಗಳನ್ನು ಯಶಸ್ವಿಯಾಗಿ ಉಪಯೋಗಿಸಲು ಸಾಧ್ಯವೇ? ನನ್ನ ವಿಶೇಷಗಳು ಮತ್ತು ಸಾಮರ್ಥ್ಯಗಳನ್ನು ನಾನು ಹೇಗೆ ಕೆಲಸದ ವ್ಯಾಪ್ತಿಯೊಳಗೆ ಉಪಯೋಗಿಸಬಹುದು? ನಾನು ಬೇರೆಯವರ ಪಾತ್ರವನ್ನು ನಿರ್ವಹಿಸಲು ಹಂಬಲಿಸುತ್ತೇನೆಯೇ? ನಾನು ಬೇರೆಯವರ ಕೆಲಸದಲ್ಲಿ ತುಂಬಾ ಯಶಸ್ವಿಯಾಗುತ್ತೇನೆಂದು ತಿಳಿಯುತ್ತಿದ್ದೇನೆಯೇ? ಆ ಕೆಲಸದ ವ್ಯಾಪ್ತಿಯು ನನ್ನ ಕೈಗೆ ಏಟುಕುವಂತಿದೆಯೇ?
\end{mananam}
\WritingHand\enspace\textbf{ಆತ್ಮ ವಿಮರ್ಶೆ}\\
\begin{inspiration}{\mananamfont ಸ್ಪೂರ್ತಿ}
\footnotesize \mananamtext ಈ ಜೀವನದಲ್ಲಿ ನಮ್ಮ ಕರ್ತವ್ಯವನ್ನು ನಿರ್ವಹಿಸುವುದು ನಮ್ಮ ಜೀವನದ ಮಾನಸಿಕ ಸ್ಥಿತಿಯನ್ನು ಗೆಲ್ಲಲು ಬಹಳ ಉತ್ತಮವಾದ ಮಾರ್ಗ. ಪ್ರವೃತ್ತಿಗಳನ್ನು ಆರೋಗ್ಯಕರವಾದ ರೀತಿಯಲ್ಲಿ ಸುಡಲು ಬೇರೆಯವರ ಪಾತ್ರಗಳನ್ನು ಹಂಬಲಿಸುವುದಕ್ಕಿಂತ ನಮ್ಮ ಜವಾಬ್ದಾರಿಗಳನ್ನು ನಡೆಸಿಕೊಂಡು ಹೋಗುವುದಾಗಿದೆ.
\end{inspiration}
\newpage

\slcol{ಅರ್ಜುನ ಉವಾಚ ।\\
\Index{ಅಥ ಕೇನ ಪ್ರಯುಕ್ತೋऽಯಂ} ಪಾಪಂ ಚರತಿ ಪೂರುಷಃ ।\\
ಅನಿಚ್ಛನ್ನಪಿ ವಾರ್ಷ್ಣೇಯ ಬಲಾದಿವ ನಿಯೋಜಿತಃ ॥ 36 ॥} 
\cquote{ಅರ್ಜುನ ಹೇಳಿದನು,\\
ಕೃಷ್ಣ,ಹಾಗಾದರೆ ಈ ಮನುಷ್ಯನು ತನಗೆ ಬೇಡವಾದರೂ ಬಲವಂತಕ್ಕೆ ಬಲಿಯಾದವನಂತೆ ಯಾರಿಂದ ಪ್ರೇರಿತನಾಗಿ ಪಾಪವನ್ನು ಮಾಡುತ್ತಾನೆ?\\}
\slcol{ಶ್ರೀಭಗವಾನುವಾಚ  ।\\
\Index{ಕಾಮ ಏಷ ಕ್ರೋಧ ಏಷ} ರಜೋಗುಣಸಮುದ್ಭವಃ ।\\
ಮಹಾಶನೋ ಮಹಾಪಾಪ್ಮಾ ವಿದ್ಧ್ಯೇನಮಿಹ ವೈರಿಣಮ್ ॥ 37 ॥} 
\cquote{ಶ್ರೀ ಭಗವಂತನು ಹೇಳಿದನು,\\
ರಾಜೋಗುಣದಿಂದ ಹುಟ್ಟಿದ, ಸಿಟ್ಟಿಗೂ ತವರಾದ ಈ ಬಯಕೆ ಇದಕ್ಕೆಲ್ಲ ಕಾರಣ.ಎಷ್ಟು ತಿಳಿಸಿದರು ಇನ್ನಷ್ಟು ಬೇಕೆನ್ನುವ ಮಹಾ ಪಾಪಿ. ಈ ಬಯಕೆಯನ್ನು ಬಾಳಿನಲ್ಲಿ ದೊಡ್ಡ ಶತ್ರು ಎಂದು ತಿಳಿ.\\}
\slcol{\Index{ಧೂಮೇನಾವ್ರಿಯತೇ} ವಹ್ನಿರ್ಯಥಾದರ್ಶೋ ಮಲೇನ ಚ ।\\
ಯಥೋಲ್ಬೇನಾವೃತೋ ಗರ್ಭಸ್ತಥಾ ತೇನೇದಮಾವೃತಮ್ ॥ 38 ॥}
\cquote{ ಬೆಂಕಿಗೆ ಹೊಗೆಯ ಮುಸುಕು. ಕನ್ನಡಿಗೆ ಕೊಳೆಯ ಮುಸುಕು. ಗರ್ಭಕ್ಕೆ ಕೋಶದ ಮುಸುಕು. ಹಾಗೆ ಜಗತ್ತಿಗೆಲ್ಲ ಕಾಮದ ಮುಸುಕು.\\}
\slcol{\Index{ಆವೃತಂ ಙ್ಞಾನಮೇತೇನ} ಙ್ಞಾನಿನೋ ನಿತ್ಯವೈರಿಣಾ ।\\
ಕಾಮರೂಪೇಣ ಕೌಂತೇಯ ದುಷ್ಪೂರೇಣಾನಲೇನ ಚ ॥ 39 ॥} 
\cquote{ಅರ್ಜುನ,ತಿಳಿದವರ ನಿತ್ಯಶತ್ರುವಾದ, ಎಷ್ಟು ತಿನಿಸಿದರು ಸಾಕೆನಿಸದ ಈ ಕಾಮದ ಮುಸುಕಿನಿಂದ ತಿಳಿವು ಮರೆಯಾಗಿದೆ.\\}
\slcol{\Index{ಇಂದ್ರಿಯಾಣಿ ಮನೋ} ಬುದ್ಧಿರಸ್ಯಾಧಿಷ್ಠಾನಮುಚ್ಯತೇ ।\\
ಏತೈರ್ವಿಮೋಹಯತ್ಯೇಷ ಙ್ಞಾನಮಾವೃತ್ಯ ದೇಹಿನಮ್ ॥ 40 ॥}
\cquote{ಇಂದ್ರಿಯಗಳು,ಮನಸ್ಸು,ಬುದ್ದಿ ಇವೇ ಕಾಮದ ವಾಸಸ್ಥಾನ.ಇದು ಇವುಗಳ ಮೂಲಕ ತಿಳಿವನ್ನು ಮರೆಮಾಡಿ ಮನುಷ್ಯನನ್ನು ಮಂಕು ಗೊಳಿಸುತ್ತದೆ.\\}

\newpage
\begin{mananam}{\mananamfont ಮನನ ಶ್ಲೋಕ - \textenglish{36}}
\footnotesize \mananamtext ನನ್ನ ಒಳ್ಳೆಯ ಉದ್ದೇಶಗಳ ವಿರುದ್ಧವಾಗಿ ನಡೆದುಕೊಳ್ಳುವಂತೆ ಈ ಬಲಗಳು ನನ್ನನ್ನು ಬಲವಂತ ಪಡಿಸುತ್ತವೆ ಎಂಬುದನ್ನು ನಾನು 
 ಅರಿತಿದ್ದೇನೆಯೇ? ನನಗೆ ಪ್ರವೃತ್ತಿಯ ಸುಪ್ತಾವಸ್ಥೆಯ    ಸ್ಥಿತಿ ಮತ್ತು ಪ್ರದರ್ಶಿಸುತ್ತಿರುವ ಸ್ಥಿತಿಗಳ ಬಗ್ಗೆ ತಿಳಿದಿದೆಯೇ? ನನಗೆ ಮಾನಸಿಕ ನಿರ್ಬಂಧ ಮತ್ತು ಜ್ಞಾನಗಳ ನಿರ್ಬಂಧಗಳ ಬಗ್ಗೆ ಅರಿವಿದೆಯೇ? ಈ ಎರಡರಲ್ಲಿ ಯಾವ ಅವಸ್ಥೆಯಲ್ಲಿ ಯಾವುದರ ಪ್ರಾಬಲ್ಯ ಜಾಸ್ತಿ? ನನಗೆ ಜ್ಞಾನದ ಹಂತದಲ್ಲಿ ಪ್ರಲೋಬನೆಗಳ ಶಕ್ತಿಯನ್ನು ಹತೋಟಿಯಲ್ಲಿಡುವ ಸಾಮರ್ಥ್ಯವಿದೆಯೇ? ನನಗೆ ಯೋಚನೆಗಳ ಹಂತದಲ್ಲಿ ಅವುಗಳನ್ನು ಹತೋಟಿಯಲ್ಲಿಡುವ ಸಾಮರ್ಥ್ಯವಿದೆಯೇ?
\end{mananam}
\WritingHand\enspace\textbf{ಆತ್ಮ ವಿಮರ್ಶೆ}\\
\begin{inspiration}{\mananamfont ಸ್ಪೂರ್ತಿ}
\footnotesize \mananamtext ಬೇರೆ ಬೇರೆ ಹಾನಿಕಾರಕ ಪ್ರವೃತ್ತಿಗಳು ಜೀವಮಾನದ ಉದ್ಧಕ್ಕೂ ಒಟ್ಟಿಗೆ ಸೇರಿ ನಮ್ಮೊಳಗೆ ಯಾವಾಗಲೂ ಘರ್ಷಣೆ ಉಂಟು ಮಾಡುತ್ತಿರುತ್ತವೆ. ಸಣ್ಣ ಮಟ್ಟದಲ್ಲಿಯೂ ಪ್ರಚೋದನೆಗಳನ್ನು ಕೊಡುವ ಜ್ಞಾನದ ಪ್ರಲೋಭನೆಗಳನ್ನು ಅತಿಯಾದ ಆತ್ಮವಿಶ್ವಾಸದಿಂದ ಇರುವುದು ಯಾವಾಗಲೂ ಬುದ್ಧಿವಂತಿಕೆಯಲ್ಲ.
\end{inspiration}
\newpage

\begin{mananam}{\mananamfont ಮನನ ಶ್ಲೋಕ - \textenglish{37}}
\footnotesize \mananamtext ನನ್ನ ಇಚ್ಛೆ ಮತ್ತು ಹಂಬಲಗಳ ಲಕ್ಷಣಗಳೇನು? ನನ್ನ ಇಚ್ಛೆಗಳಿಗೆ ಅಡಚಣೆಯಾದಾಗ ಏನಾಗುತ್ತದೆ? ನನಗೆ ಕೋಪದ ಭಾವನೆ ಬರುತ್ತದೆಯೇ?ನನಗೆ ಭಯ ಅಥವಾ ವ್ಯಾಕುಲತೆಯ ಆಗುತ್ತದೆಯೇ? ನನ್ನ ಇಚ್ಛೆಗಳು ನನಗೆ ನಿಬಂಧನೆ ಮತ್ತು ಮೋಹದ ಭಾವನೆ ಕೊಡುತ್ತದೆಯೇ? ಅವುಗಳು ಕೇವಲ ನನ್ನ ವೈಯಕ್ತಿಕ ಸಂತೋಷಕ್ಕೆ ಮಾತ್ರ ನಿರ್ದಿಷ್ಟವಾಗಿದೆಯೆ ಅಥವಾ ಬೇರೆಯವರ ಒಳಿತನ್ನು ಒಳಗೊಂಡಿದೆಯೇ? ಅಲ್ಲಿ ಆರೋಗ್ಯಕರ ಇಚ್ಛೆಗಳು ಅಥವಾ ಅನಾರೋಗ್ಯಕರ ಇಚ್ಛೆಗಳಂತೆ ಕಾಣುವಂತಹವು ಇರಲು ಸಾಧ್ಯವೇ?
\end{mananam}
\WritingHand\enspace\textbf{ಆತ್ಮ ವಿಮರ್ಶೆ}\\
\begin{inspiration}{\mananamfont ಸ್ಪೂರ್ತಿ}
\footnotesize \mananamtext ನಮ್ಮ ನಿಜವಾದ ಗುಣ ಅಗಾಧವಾದದ್ದು.ನಾವು ಅದನ್ನು ಮರೆತು,ನಮ್ಮನ್ನು ಸಂಪೂರ್ಣವಾಗಿಸಲು, ಪ್ರಾಪಂಚಿಕ ವಸ್ತುಗಳ ಹಿಂದೆ ಹೋಗುತ್ತಿದ್ದೇವೆ. ನಾವು ಯಾವಾಗ ಮಿತಿಯೆಂದು ಭಾವಿಸುತ್ತಿದ್ದೇವೆಯೋ ಆಗ, ದೊಡ್ಡದಾದ ಮತ್ತು ಉತ್ತಮವಾದ ಭಾವನೆ ಎಂದು ಭರವಸೆ ಮೂಡಿಸುವ, ಸ್ವಾರ್ಥಯುತ ಇಚ್ಛೆಗಳು ಆವಿರ್ಭಸುತ್ತವೆ. 
\end{inspiration}
\newpage

\begin{mananam}{\mananamfont ಮನನ ಶ್ಲೋಕ - \textenglish{39,40}}
\footnotesize \mananamtext ಯಾವುದು ನನ್ನ ಆಳವಾದ ಜ್ಞಾನವನ್ನು ಮತ್ತು ಮನೋ ಇಂಗಿತವನ್ನು ಮುಚ್ಚಿಸುತ್ತಿದೆ?ನನ್ನ ಅಭಿವೃದ್ಧಿಯ ಮಹತ್ವಾಕಾಂಕ್ಷೆಯು ಇದ್ದಾಗ್ಯೂ ಕೂಡ ಏಕೆ ನಾನು ಪ್ರಲೋಭನೆ ಮತ್ತು ಸೋಮಾರಿತನಕ್ಕೆ ಈಡಾಗುತ್ತಿದ್ದೇನೆ? ನನ್ನ ಉನ್ನತ ಗುರಿಗಳ ಕಡೆಗೆ ಇರುವ ನನ್ನ ಹತ್ತಿರವಾದ ನೆನಪನ್ನು ನಾನು ಏಕೆ ಕಳೆದುಕೊಳ್ಳುತ್ತಿದ್ದೇನೆ? ಹರ್ಷಭರಿತವಾದ ಪ್ರಯತ್ನಗಳು ಇಲ್ಲವಾಗಿದೆಯೇ? ನನ್ನ ದೇಹ ಮತ್ತು ಮನಸ್ಸಿನ ಮೇಲೆ ತೃಪ್ತಿ ಆದ ಮೇಲಿನ ಭಾವ ಯಾವ ರೀತಿಯ ಋಣಾತ್ಮಕ ಪರಿಣಾಮವನ್ನು ಉಳಿಸಿ ಬಿಡಬಹುದು?
\end{mananam}
\WritingHand\enspace\textbf{ಆತ್ಮ ವಿಮರ್ಶೆ}\\
\begin{inspiration}{\mananamfont ಸ್ಪೂರ್ತಿ}
\footnotesize \mananamtext ಪ್ರಜ್ಞೆಯ ಬುದ್ಧಿವಂತಿಕೆಯ ಹಂತದಲ್ಲಿ ಮಾತ್ರವೇ ಪರಿವರ್ತನೆಯನ್ನು ತರುವುದಿಲ್ಲ. ಅದು ನಮ್ಮ ಹೃದಯದ ಮೂಲಕ ಸೊಸಿ ಆಮೇಲೆ ನಮ್ಮ ಜ್ಞಾನೇಂದ್ರಿಯಗಳ ಮೂಲಕ  ವ್ಯಕ್ತಪಡಿಸಬೇಕು. ನಾವು ಮಾಡುವ ಕೃತ್ಯದಲ್ಲಿ ಪಾವಿತ್ರ್ಯತೆ ಇದ್ದಲ್ಲಿ ಅದು ನಮ್ಮ ತಪ್ಪು ಪ್ರವೃತ್ತಿಯ ಹೃದಯವನ್ನು ಶುದ್ಧೀಕರಿಸುತ್ತದೆ.
\end{inspiration}
\newpage

\slcol{\Index{ತಸ್ಮಾತ್ತ್ವಮಿಂದ್ರಿಯಾಣ್ಯಾದೌ} ನಿಯಮ್ಯ ಭರತರ್ಷಭ ।\\
ಪಾಪ್ಮಾನಂ ಪ್ರಜಹಿ ಹ್ಯೇನಂ ಙ್ಞಾನವಿಙ್ಞಾನನಾಶನಮ್ ॥ 41 ॥} 
\cquote{ಅರ್ಜುನ, ಆದುದರಿಂದ ನೀನು ಮೊದಲು ಇಂದ್ರಿಯಗಳನ್ನು ಬಿಗಿಹಿಡಿದು ಜ್ಞಾನವನ್ನೂ ಅನುಭವವನ್ನೂ ಹಾಳು ಮಾಡುವ ಈ ಪಾಪಿಯನ್ನು ಗೆಲ್ಲು.\\}
\slcol{\Index{ಇಂದ್ರಿಯಾಣಿ ಪರಾಣ್ಯಾಹು}ರಿಂದ್ರಿಯೇಭ್ಯಃ ಪರಂ ಮನಃ ।\\
ಮನಸಸ್ತು ಪರಾ ಬುದ್ಧಿರ್ಯೋ ಬುದ್ಧೇಃ ಪರತಸ್ತು ಸಃ ॥ 42 ॥}
\cquote{ಸ್ತೂಲ ದೇಹಕ್ಕಿಂತ ಇಂದ್ರಿಯಗಳು ಹೆಚ್ಚಿನವು ಎನ್ನುವರು. ಇಂದ್ರಿಯಗಳಿಗಿಂತಲೂ ಮನಸ್ಸು ಹೆಚ್ಚಿನದು.ಮನಸ್ಸಿಗಿಂತಲೂ ಬುದ್ಧಿ ಹೆಚ್ಚಿನದು. ಬುದ್ಧಿಗೂ ನಿಲುಕದೆ ಅದರಾಚೆ ಇರುವವನೆ  ಭಗವಂತ.\\}
\slcol{\Index{ಏವಂ ಬುದ್ಧೇಃ ಪರಂ} ಬುದ್ಧ್ವಾ ಸಂಸ್ತಭ್ಯಾತ್ಮಾನಮಾತ್ಮನಾ ।\\
ಜಹಿ ಶತ್ರುಂ ಮಹಾಬಾಹೋ ಕಾಮರೂಪಂ ದುರಾಸದಮ್ ॥ 43 ॥}
\cquote{ಅರ್ಜುನ, ಹೀಗೆ ಬುದ್ದಿಗೂ ನಿಲುಕದ ಹಿರಿಯ ತತ್ವವನ್ನು ತಿಳಿದು ಪ್ರಯತ್ನಿದಿಂದ ಮನಸ್ಸನ್ನು ನಿಯಂತ್ರಿಸಿ ಕಾಮವೆಂಬ ಕೆಟ್ಟ ಶತ್ರುವನ್ನು ಹೋಗಲಾಡಿಸು.\\}
\begin{center}
ಓಂ ತತ್ಸದಿತಿ ಶ್ರೀಮದ್ಭಗವದ್ಗೀತಾಸೂಪನಿಷತ್ಸು \\ಬ್ರಹ್ಮವಿದ್ಯಾಯಾಂ ಯೋಗಶಾಸ್ತ್ರೇ ಶ್ರೀಕೃಷ್ಣಾರ್ಜುನಸಂವಾದೇ\\
ಕರ್ಮಯೋಗೋ ನಾಮ ತೃತೀಯೋऽಧ್ಯಾಯಃ ॥ 3 ॥
\end{center}

\newpage
\begin{mananam}{\mananamfont ಮನನ ಶ್ಲೋಕ - \textenglish{41}}
\footnotesize \mananamtext ನನ್ನ ಜೀವನ ಯಾವ ಕ್ಷೇತ್ರದಲ್ಲಿ ಆತ್ಮ ನಿರ್ಬಂಧನೆ ಬೇಕಾಗಿದೆ? ನಾನು ನೋಡುವುದು, ಕೇಳುವುದು, ಸ್ವಾದಿಸುವುದು ಮುಂತಾದವುಗಳನ್ನು ತಿಳಿಯುವುದರಲ್ಲಿ ನನ್ನ ಜ್ಞಾನೇಂದ್ರಿಯಗಳನ್ನು ಹತೋಟಿಯಲ್ಲಿಡುವ ಸಾಮರ್ಥ್ಯ ನನಗಿದೆಯೇ? ನನಗೆ ಮಾತನಾಡುವುದು, ನಡೆಯುವುದು, ಲೈಂಗಿಕತೆ ಮುಂತಾದವುಗಳನ್ನು ಜ್ಞಾನೇಂದ್ರಿಯಗಳ ಮೂಲಕ ಹತೋಟಿಯಲ್ಲಿಡುವ ಸಾಮರ್ಥ್ಯ ನನಗಿದೆಯೇ?ಈ ನಿರ್ಬಂಧನೆಗಳನ್ನು ಅನುಷ್ಠಾನಗೊಳಿಸಲು ನಾನು ಯಾವ ಮೊದಲ ಹೆಜ್ಜೆಯನ್ನು ತೆಗೆದುಕೊಳ್ಳಲಿ?
\end{mananam}
\WritingHand\enspace\textbf{ಆತ್ಮ ವಿಮರ್ಶೆ}\\
\begin{inspiration}{\mananamfont ಸ್ಪೂರ್ತಿ}
\footnotesize \mananamtext ತುಂಟತನದಿಂದ ಕೂಡಿದ ಜ್ಞಾನವನ್ನು ಹತೋಟಿಯಲ್ಲಿಡಲು ಯೋಗ ಒಂದು ಉತ್ತಮ ವಿದ್ಯೆಯಾಗಿದೆ. ಯಾವಾಗ ಇಂದ್ರಿಯ ಜ್ಞಾನವನ್ನು ನಿರ್ಬಂಧಿಸುತ್ತೇವೆಯೋ ಆಗ ಅರಿವು ಸುತ್ತುವರಿಯಪಟ್ಟಿರುವುದಿಲ್ಲ. ಅದು ನಮ್ಮ ಉತ್ತಮ  ಕೃತ್ಯಗಳಿಗೆ ಬೇಕಾದ ಮಾರ್ಗದರ್ಶನ ಮಾಡಲು ಸಿಗುವಂತಿರುತ್ತದೆ.
\end{inspiration}
\newpage



\newpage
\begin{mananam}{\mananamfont ಮನನ ಶ್ಲೋಕ - \textenglish{42,43}}
\footnotesize \mananamtext ದಟ್ಟವಾದುದನ್ನು ನಿಯಂತ್ರಿಸುವುದಕ್ಕಿಂತ ಸೂಕ್ಷ್ಮವಾಗಿರುವುದು ಉತ್ಕೃಷ್ಟ ಎನ್ನುವುದನ್ನು ನೋಡಿದಾಗ ನಾನು ಏನೆಂದು ಅರ್ಥೈಸಿಕೊಳ್ಳಬೇಕು. ನನ್ನ ಜ್ಞಾನೇಂದ್ರಿಯಗಳು ಹೇಗೆ ನನ್ನ ದೇಹವನ್ನು ನಿರ್ದೇಶಿಸುತ್ತದೆ ಎಂದು ನಾನು ನೋಡಬಹುದೇ? ಮನಸ್ಸು ಮತ್ತು ಜ್ಞಾನದ ಸಂಬಂಧವನ್ನು ಪ್ರಮುಖವಾಗಿ ನಿದ್ರಾವಸ್ಥೆಯಲ್ಲಿ ಯಾವಾಗ ಮನಸ್ಸು ಚುರುಕಾಗಿರುವುದಿಲ್ಲವೋ ಮತ್ತು ಜ್ಞಾನೇಂದ್ರಿಯದ ಪ್ರಭಾವ  ಇರುವುದಿಲ್ಲವೋ ಆಗ ನಾನು ಇವುಗಳಿಗೆ ಸಂಬಂಧ ಕಲ್ಪಿಸಬಲ್ಲೆನೇ? ನಾನು ನನ್ನ ನಿಜವಾದ ಗುಣ ಏನೆಂದು ತಿಳಿದುಕೊಳ್ಳುವುದರ ಅರ್ಥವೇನು? ಅದು ಹೇಗೆ ಜೀವನದ ದೃಷ್ಟಿಕೋನದ ಬದಲಾವಣೆ ಮತ್ತು ಪ್ರಪಂಚದಲ್ಲಿನ ಕರ್ತವ್ಯಗಳಿಗೆ ಕಾರಣವಾಗುತ್ತದೆ?
\end{mananam}
\WritingHand\enspace\textbf{ಆತ್ಮ ವಿಮರ್ಶೆ}\\
\begin{inspiration}{\mananamfont ಸ್ಪೂರ್ತಿ}
\footnotesize \mananamtext ನಮ್ಮ ಪ್ರಜ್ಞೆಯು ನಮ್ಮಲ್ಲಿರುವ ನೇರವಾದ ಮತ್ತು ಉತ್ತಮವಾದ ಅಸ್ತಿತ್ವವಾಗಿದೆ. ಅದುವೇ ನಮ್ಮ ಸ್ವಯಂ ಅಧೀನದಲ್ಲಿ ಇರುವುದಾಗಿದೆ. ಮತ್ತೆಲ್ಲವೂ [ಬುದ್ಧಿಶಕ್ತಿ, ಮನಸ್ಸು, ಜ್ಞಾನ ಮತ್ತು ದೇಹ] ಇದಕ್ಕೆ ವಸ್ತುಗಳಾಗಿವೆ. ಇಚ್ಛೆಗಳ ಪ್ರವೃತ್ತಿಯು ನಮ್ಮ ವಾಸ್ತವಿಕ ಹೆಗುರುತಾಗಿದ್ದು ಅದು ನಮ್ಮ ನಿಜ ಸ್ವಭಾವವನ್ನು ದೂರಕ್ಕೆ ತೆಗೆದುಕೊಂಡು ಹೋಗುತ್ತದೆ. ನಮ್ಮ ನಿಜವಾದ ಅಸ್ತಿತ್ವವಾದ ಎಲ್ಲಾ ಸಂತೋಷಗಳಿಗೂ ಮೂಲವಾದ ಇದನ್ನು ಸರಿಯಾಗಿ ಅರ್ಥ ಮಾಡಿಕೊಂಡು, ಇಚ್ಛೆಗಳ ಹಿಡಿತವನ್ನು ಸಡಿಲಗೊಳಿಸುತ್ತೆವೆ.
\end{inspiration}
\newpage


