\begin{center}
ದಿಶಂತು ಶಂ ಮೇ ಗುರುಪಾದಪಾಂಸವಃ ॥\\
\end{center}
\footnotesize \mananamtext{ಶ್ರಿ\!\char"0CD5ಮದ್ ಭಗವದ್ಗೀತೆಯು ಎಲ್ಲಾ ಧರ್ಮಗ್ರಂಥಗಳಲ್ಲಿ ಒಂದು ಅನನ್ಯ ಮತ್ತು ಸಾಟಿಯಿಲ್ಲದ ರತ್ನವಾಗಿದೆ.  ಇದು ಪರಿತ್ಯಾಗ ಮಾಡುವವರಿಗೆ ಮಾತ್ರವಲ್ಲದೆ ಲೌಕಿಕ ಜವಾಬ್ದಾರಿಗಳನ್ನು ಹೊತ್ತಿರುವವರಿಗೂ ಮಾರ್ಗದರ್ಶಿ ಬೆಳಕಾಗಿ ಕಾರ್ಯನಿರ್ವಹಿಸುತ್ತದೆ, ಆಧ್ಯಾತ್ಮಿಕದೊಂದಿಗೆ ಪ್ರಾಪಂಚಿಕತೆಯನ್ನು ಸಮತೋಲನಗೊಳಿಸಲು ಪ್ರಯತ್ನಿಸುತ್ತದೆ; ಇದು, ಅಜ್ಞಾನದಿಂದ ಅವರು (ಸಾಮಾನ್ಯ ಜನ) ತಮ್ಮನ್ನು ತಾವು,   ದೇಹ ಮತ್ತು ಮನಸ್ಸಿನ ವ್ಯವಹಾರಗಳ  ಜೊತೆ ಗುರುತಿಸಿಕೊಂಡಾಗ, ಅಂತಹ ವ್ಯವಹಾರಗಳ ಬಗ್ಗೆ ನಿಷ್ಪಕ್ಷಪಾತವಾಗಿರುವಂತೆ ಪ್ರತಿಪಾದಿಸುತ್ತದೆ.\\
ಜೀವನದಲ್ಲಿ ಒಬ್ಬರ ಕರ್ತವ್ಯಗಳನ್ನು ಸಾಧಿಸಲು, ಬಾಂಧವ್ಯದ ಅಥವಾ, ಮೋಹದ ಭಾವನೆ ಇರಬೇಕು ಎಂಬುದು ಒಂದು ತಪ್ಪು ಕಲ್ಪನೆ.  ಭಗವಾನ್ ಕೃಷ್ಣ, ಎಲ್ಲರಿಗಿಂತಲೂ  ದೊಡ್ಡ ಸಂಸಾರಿ ಹಾಗೂ, ಪರಿಪೂರ್ಣವಾದ ಮೋಹರಹಿತನಾದ ಅಸಂಸಾರಿ; ದಿವ್ಯವಾದ ಆನಂದದಲ್ಲಿ ನೆಲೆಗೊಂಡು, ಈ ಮೂರ್ತ, ಭೌತಿಕ ಜಗತ್ತಿನಲ್ಲಿ ಹೇಗೆ ಕಾರ್ಯನಿರ್ವಹಿಸಬೇಕು ಎಂಬುದನ್ನು ಅವನು ತನ್ನ ಕಾರ್ಯಗಳು, ಭಾವ ಮತ್ತು ಅವನು ಉಚ್ಛರಿಸುವ ಪ್ರತಿಯೊಂದೂ ಪರಮಪದದ  ಮೂಲಕ ಪ್ರದರ್ಶಿಸುತ್ತಾನೆ. ಆರಂಭದಲ್ಲಿ ‘ಸಂಘರ್ಷ ಮತ್ತು ಸವಾಲುಗಳಿಂದ ತುಂಬಿರುವ ಮಾರ್ಗ’ ಎಂದು ಕಂಡುಬoದರೂ ಸಹ, ಈ ಸ್ಥಿತಿಯನ್ನು ಸಾಧಿಸಲು (ದಿವ್ಯವಾದ ಆನಂದದಲ್ಲಿ ನೆಲೆಗೊಂಡು, ಈ ಮೂರ್ತ, ಭೌತಿಕ ಜಗತ್ತಿನಲ್ಲಿ  ಕಾರ್ಯನಿರ್ವಹಿಸುವುದು) ಸಮರ್ಥ ಶಿಕ್ಷಕರಿಂದ ಸರಿಯಾದ ಮಾರ್ಗದರ್ಶನದ ಅಗತ್ಯವಿದೆ. \\
ಈ ‘ನಿಪುಣ ಮಾರ್ಗದರ್ಶಿ ಕೈಪಿಡಿ’ಯಲ್ಲಿ, ಲೇಖಕರ ಆಳವಾದ ಒಳನೋಟದಿಂದ ಅಧ್ಯಾಯ 2 ರ 52-53 ಪದ್ಯಗಳಲ್ಲಿ ಸೂಚಿಸಿದಂತೆ: “ಶಿಕ್ಷಕರ ಮತ್ತು ಧರ್ಮಗ್ರಂಥಗಳ ಉದ್ದೇಶವು ನಮ್ಮನ್ನು ಭ್ರಮೆಯಿಂದ ಜಗ್ಗಿಸಿ, ಮುಕ್ತರನ್ನಾಗಿ ಮಾಡುವುದೇ ಆಗಿದೆ. ನಾವು ನಮ್ಮ ಲೌಕಿಕ ಚಿಂತನೆಯ ಮಾದರಿಗಳನ್ನು ಬಿಟ್ಟು, ಒಂದು ಉನ್ನತ ಸತ್ಯದಲ್ಲಿ (ಪಾರಮಾರ್ಥಿಕದಲ್ಲಿ) ಆಶ್ರಯ ಪಡೆಯಲು ಪ್ರಾರಂಭಿಸುತ್ತಿದ್ದಂತೆಯೇ, ಆಧ್ಯಾತ್ಮಿಕ ಪ್ರಯಾಣದ ಒಂದು ಭಾಗವಾದ ಗೊಂದಲಗಳು ಮತ್ತು ಸವಾಲುಗಳು ಏಳುತ್ತವೆ; ಆದರೆ, ನಾವು ಹೀಗೆ ಈ ಹಾದಿಯಲ್ಲಿ ಪ್ರಗತಿ ಹೊಂದುತ್ತಿದ್ದಂತೆ, ಸ್ಪಷ್ಟತೆ ಪಡೆಯಲು ಪ್ರಾರಂಭಿಸುತ್ತೇವೆ”.\\
ಈ ಪುಸ್ತಕವು ಲೇಖಕರ ಕ್ರಾಂತಿಕಾರಿ ಚಿಂತನೆಗಳ ಮೂಲಕ ಓದುಗರನ್ನು ದೇಹದಿಂದ, ಮನಸ್ಸಿಗೆ ಮತ್ತು ಮನಸ್ಸಿನಿಂದ ಪ್ರಜ್ಞೆಗೆ (ಚೈತನ್ಯಕ್ಕೆ), ಒಬ್ಬರ ಅಸ್ತಿತ್ವದ ಪದರಗಳನ್ನು ಭೇದಿಸುವಂತೆ ಮಾಡುತ್ತದೆ. \\
ಪ್ರತಿ ವಿಭಾಗದಲ್ಲಿ, ‘ಮನನಂ’ ಶೀರ್ಷಿಕೆಯಡಿಯಲ್ಲಿರುವ ಆತ್ಮಾವಲೋಕನದ ಪ್ರಶ್ನೆಗಳು, ಸ್ವಯಂ ಸಮಾಧಾನ ಮತ್ತು ಆತ್ಮವಂಚನೆಯಲ್ಲಿ (ಆತ್ಮ ಪ್ರವಂಚನ) ತೊಡಗಿರುವವರಿಗೆ ಯಾವುದೇ ವಿರಾಮ ನೀಡುವುದಿಲ್ಲ ಮತ್ತು ಪ್ರಾಮಾಣಿಕವಾದ ಸ್ವಯಂ ಮೌಲ್ಯಮಾಪನ ಮಾಡಲು ಅವರಿಗೆ ಸವಾಲು ಒಡ್ಡುತ್ತವೆ .  ಮತ್ತೊಂದೆಡೆ, ‘ಸ್ಫೂರ್ತಿ’ಯ ಅಡಿಯಲ್ಲಿರುವ ಪದಗಳು, ಪ್ರಯಾಸಕರ ಮತ್ತು ಗೊಂದಲಮಯ ಆಧ್ಯಾತ್ಮಿಕ ಮಾರ್ಗವನ್ನು ತುಲನಾತ್ಮಕವಾಗಿ ಸುಲಭಗೊಳಿಸಿ, ಸಕಾರಾತ್ಮಕತೆಯ ಒಂದು, ಅಕ್ಷಯವಾದ ಮೂಲವಾಗಿ ಕಾರ್ಯನಿರ್ವಹಿಸುತ್ತವೆ.\\
ಈ ಕೈಪಿಡಿಯು, ಆಕಾಂಕ್ಷಿಗಳಿಗೆ ಜೀವನದ ವಿವಿಧ ಸಮಸ್ಯೆಗಳಿಗೆ ಪರಿಹಾರವನ್ನು ಕಂಡುಕೊಳ್ಳಲು ಸಾಕಷ್ಟು ಸಮಾಧಾನಗಳನ್ನು  ನೀಡುತ್ತದೆ, ಆದರೆ ಬಾಹ್ಯವಾಗಿ ಅಲ್ಲ, ಆಂತರಿಕವಾಗಿ.\\
ಈ ಪುಸ್ತಕದ ವಿಷಯವು, ‘ಮಾನವ ಮನೋವಿಜ್ಞಾನ’ದ ಬಗ್ಗೆ ಲೇಖಕರ ಆಳವಾದ ತಿಳುವಳಿಕೆಯನ್ನು ತೋರಿಸುತ್ತದೆ, ಮೊದಲು ಸಕಾರಾತ್ಮಕ ಮಾನಸಿಕ ಸ್ಥಿತಿಯ ಕಡೆಗೆ ಮತ್ತು ನಂತರ ಅದನ್ನು ಮೀರಿ, ಆಂತರಿಕ ಶಾಶ್ವತವಾದ ಆತ್ಮದೆಡೆಗೆ, ಸರಳವಾಗಿ ಮುಂದುವರಿಯುತ್ತದೆ. \\
ಈ ಪುಸ್ತಕದ ಓದುಗರು ಈ ಪ್ರಯಾಣವನ್ನು ಪ್ರಾರಂಭಿಸಿದಾಗ, ಅವರು ಖಂಡಿತವಾಗಿಯೂ ಮಾನವ ಮನಸ್ಸಿನ ಕ್ಷೇತ್ರವನ್ನು ಮೀರುತ್ತಾರೆ ಮತ್ತು ಈ ಸ್ವರ್ಗೀಯ ಗೀತೆಯಾದ  ‘ಭಗವದ್ಗೀತೆ’ ಗಾಯಕನ ಕೃಪೆಯಿಂದ ‘ಅಧಿಷ್ಠಾನ ಚೈತನ್ಯಮ್’ (ಚೇತನಾತ್ಮಕದ ಅಂತಿಮ ಮೂಲತತ್ವ) ಅನ್ನು ತಲುಪುತ್ತಾರೆ. ಸಾಧಕರ ಅನುಕೂಲಕ್ಕಾಗಿ ತಮ್ಮ ಅಮೂಲ್ಯವಾದ ಆಲೋಚನೆಗಳನ್ನು ಲೇಖಿಸಿದ, ಸ್ವಾಮಿ ನಿರ್ಗುಣಾನಂದ ಗಿರಿ, ಇವರ ನಿಸ್ವಾರ್ಥ ಪ್ರಯತ್ನವನ್ನು,  ಸನಾತನ ಗುರುವಾದ, ಆ ಭಗವಂತ ಶ್ರೀಕೃಷ್ಣನು ಆಶೀರ್ವದಿಸಲಿ.\\\\
\begin{center}
ಮಂಗಳಂ ಸರ್ವಂ
\end{center}

{\kanBold ಸ್ವಾಮಿ ಸ್ವಾನಂದ ತೀರ್ಥ} \\
ಆಚಾರ್ಯ, ಕೈಲಾಸ್ ಆಶ್ರಮ\\
ಋಷಿಕೇಶ – ಉತ್ತರಖಂಡ\\
}