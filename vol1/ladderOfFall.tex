
\newpage
\begin{center}
  {\Large ಪತನದ ಹಂತಗಳು} \\(ಶ್ಲೋಕ 62, 63)
\end{center}

{\footnotesize{
\tikzstyle{startstop} = [rectangle, rounded corners, 
minimum width=4cm, 
minimum height=1cm,
text centered, 
draw=black,
align=center,
fill=white!30]

\tikzstyle{io} = [trapezium, 
trapezium stretches=true, % A later addition
trapezium left angle=70, 
trapezium right angle=110, 
minimum width=3cm, 
minimum height=1cm, text centered, 
draw=black, fill=white!30]

\tikzstyle{process} = [rectangle, 
minimum width=4cm, 
minimum height=1cm, 
text centered, 
text width=4cm, 
draw=black, 
fill=white!30]

\tikzstyle{etbox} = [rectangle, 
minimum width=4cm, 
minimum height=1cm, 
text centered, 
text width=3cm, 
draw=white, 
fill=none]

\tikzstyle{lbox} = [rectangle, 
minimum width=3cm, 
minimum height=1cm, 
align=left, 
text width=3cm, 
draw=white, 
fill=none]

\tikzstyle{decision} = [diamond, 
minimum width=3cm, 
minimum height=1cm, 
text centered, 
draw=black, 
fill=green!30]
\tikzstyle{arrow} = [thick,->,>=stealth]
\begin{center}
\begin{tikzpicture}[node distance=2cm]
\node (start) [startstop] {ಇಂದ್ರಿಯ ವಸ್ತುಗಳ ಬಗ್ಗೆ\\ಸದಾ ಚಿಂತಿಸುವುದು}; \node(l1)[lbox, left of =start,xshift=-2.0cm]{ಮನಸ್ಸಿನ ಸ್ಥಿತಿಗಳು};\node(r1)[lbox, right of =start,xshift=2.0cm]{ಧ್ಯಾಯತೋ ವಿಷಯಾನ್ಪುಂಸಃ};
\node(pro1) [process, below of =start] {ಅವುಗಳ ಮೇಲೆ ವ್ಯಾಮೋಹ ಉಂಟಾಗುತ್ತದೆ}; \node(l2)[lbox,left of =pro1,xshift=-2.0cm]{ಇಂದ್ರಿಯ-ಮನಸ್ಸುಗಳಲ್ಲಿ ಮಗ್ನನಾಗಿರುವುದು};\node(r2)[lbox, right of =pro1,xshift=2.0cm]{ಸಂಗಸ್ತೇಷೂಪಜಾಯತೇ};
\node(pro2) [process, below of =pro1] {ಮೋಹದಿಂದ ಕಾಮನೆಗಳು ಹುಟ್ಟುತ್ತವೆ}; \node(l3)[lbox, left of =pro2,xshift=-2.0cm]{ಕಾಮನೆಗಳಿಂದ ವ್ಯಾಮೋಹಕ್ಕೆ ಒಳಗಾಗಿರುವುದು};\node(r3)[lbox, right of =pro2,xshift=2.0cm]{ಸಂಗಾತ್ಸಂಜಾಯತೇ ಕಾಮಃ};
\node(pro3) [process, below of =pro2] {ಅಡಚಣೆಗೊಳಗಾದ ಕಾಮನೆಗಳಿಂದ ಕ್ರೋಧ ಹುಟ್ಟುತ್ತದೆ}; \node(l4)[lbox, left of =pro3,xshift=-2.0cm]{ಈ ಮಟ್ಟದಿಂದ ನಿಯಂತ್ರಣ ಹೆಚ್ಚು ಕಷ್ಟ};\node(r3)[lbox, right of =pro3,xshift=2.0cm]{ಕಾಮಾತ್ಕ್ರೋಧೋಭಿ\\ಜಾಯತೇ};
\node(pro4) [process, below of =pro3] {ಕ್ರೋಧದಿಂದ ಸಮ್ಮೋಹ (ಭ್ರಮೆ) ಉಂಟಾಗುತ್ತದೆ}; \node(l5)[lbox, left of =pro4,xshift=-2.0cm]{ಪತನವು ವೇಗವಾಗಿ ಸಂಭವಿಸುತ್ತದೆ};\node(r4)[lbox, right of =pro4,xshift=2.0cm]{ಕ್ರೋಧಾತ್ಭವತಿಸಮ್ಮೋಹಃ};
\node(pro5) [process, below of =pro4] {ಸಮ್ಮೋಹದಿಂದ ಸ್ಮೃತಿಹೀನತೆ (ಮರೆವು)}; \node(l6)[lbox, left of =pro5,xshift=-2.0cm]{ತನ್ನನ್ನು ತಾನು ರಕ್ಷಿಸಿಕೊಳ್ಳಲು ಹೆಚ್ಚಿನ ಪ್ರಯತ್ನ ಬೇಕಾಗುತ್ತದೆ};\node(r5)[lbox, right of =pro5,xshift=2.0cm]{ಸಂಮೋಹಾತ್ಸ್ಮೃತಿವಿಭ್ರಮಃ};
\node(pro6) [process, below of =pro5] {ಮರೆವಿನಿಂದ ವಿವೇಕ ನಾಶ};\node(r6)[lbox, right of =pro6,xshift=2.0cm]{ಸ್ಮೃತಿಭ್ರಂಶಾತ್ಬುದ್ಧಿನಾಶಃ};
\node (stop) [startstop, below of=pro6] {ವಿವೇಕ ನಾಶವು 'ಸ್ವಯಂ'\\ನಾಶಕ್ಕೆ ಕಾರಣವಾಗುತ್ತದೆ};\node(r1)[lbox, right of =stop,xshift=2.0cm]{ಬುದ್ಧಿನಾಶಾತ್ಪ್ರಣಶ್ಯತಿ};

\draw [arrow] (start) -- (pro1);
\draw [arrow] (pro1) -- (pro2);
\draw [arrow] (pro2) -- (pro3);
\draw [arrow] (pro3) -- (pro4);
\draw [arrow] (pro4) -- (pro5);
\draw [arrow] (pro5) -- (pro6);
\draw [arrow] (pro6) -- (stop);

\draw [arrow] (l1) -- (l2);
\draw [arrow] (l2) -- (l3);
\draw [arrow] (l3) -- (l4);
\draw [arrow] (l4) -- (l5);
\draw [arrow] (l5) -- (l6);

\end{tikzpicture}
\end{center}
}}
