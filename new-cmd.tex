%\newcommand{\slcol}[1]{{\color{MidnightBlue}{#1}}}
%\newcommand{\cquote}[1]{\begin{quoting}{\color{darkgray}{#1}}\end{quoting}}
%\newcommand\Chapter[2]{
%  \chapter[#1]{#1\\\large#2}
%}
\newcommand\Chapter[2]{
  \chapter[#1 #2]{#1\\\Large{#2}}
  \vspace{-10mm}
  \newpgfornamentfamily{vectorian}
  \begin{center}
    
    \pgfornament[anchor=center,ydelta=0pt,width=5cm, color=black]{84}
  \end{center}
  %print different header for each chapter
  \fancyhead[RO]{\footnotesize \kanfont ಗೀತಾ ಮನನಂ} %Odd page Right
  \fancyhead[LO]{\footnotesize #1} % Odd page Left
  \fancyhead[LE]{\footnotesize \kanfont ಗೀತಾ ಮನನಂ} %Even page Left
  \fancyhead[RE]{\footnotesize #1} %Even page Right
}

\newcommand\NewHeader[1]{
  %print different header for each chapter
  \fancyhead[RO]{\footnotesize \kanfont ಗೀತಾ ಮನನಂ} %Odd page Right
  \fancyhead[LO]{\footnotesize #1} % Odd page Left
  \fancyhead[LE]{\footnotesize \kanfont ಗೀತಾ ಮನನಂ} %Even page Left
  \fancyhead[RE]{\footnotesize #1} %Even page Right
}
%
%\addto\captionskannada{\renewcommand{\contentsname}{\color{blue}{ವಿಷಯ ಸೂಚಿ}}}
%
% Font definitions
%
%\newfontfamily\mananamfont[Script=Kannada]{NudiUni08k}
%\newfontfamily\mananamtext[Script=Kannada]{AdishilaVedic}
%\newfontfamily\engfont[Script=Kannada]{Arial Unicode MS}
%
%
%\newcommand{\chapindexpage}{\thechapter-\arabic{page}}
%\renewcommand{\thepage}{\chapindexpage}
%\newcommand{\chapIndex}{\thechapter-\thepage}
%\newcommand\Index[1]{#1\index{#1, \thechapter}}
%\newcommand{\Index}[1]{#1\index{#1|chapIndex}}
\newcommand\chapEnd{
  \begin{center}
    \newpgfornamentfamily{vectorian}
    \pgfornament[anchor=center,ydelta=0pt,width=5cm, color=black]{85}  
  \end{center}
}
\newcommand{\chapEndSloka}[1]{
  \begin{center}
    {\color{black}\small
    ಇತಿ ಶ್ರೀಮದ್ಭಗವದ್ಗೀತಾ  ರೂಪೀ ಉಪನಿಷತ್, ಬ್ರಹ್ಮವಿದ್ಯಾ, ಯೋಗಶಾಸ್ತ್ರ\\ ವಿಷಯವಾಗಿ ಶ್ರೀಕೃಷ್ಣ ಹಾಗೂ ಅರ್ಜುನರ ಸಂವಾದದಲ್ಲಿ, \\‘#1’ ಎಂಬ ಅಧ್ಯಾಯ  ಸಂಪೂರ್ಣ}
  \end{center}
  \chapEnd
}