‘ಮನನಂ’ ಪುಸ್ತಕದ ಬಳಕೆಗೆ ಮಾರ್ಗದರ್ಶಿಗಳು:
ಇದು ಸಾಮಾನ್ಯತಃ ಇರುವ ಗೀತಾ ಅಧ್ಯಯನ ಪುಸ್ತಕವಲ್ಲದ ಕಾರಣ,  ಕೆಳಗಿನ ವಿಧಾನವು ಇದರ ಲಾಭವನ್ನು ಹೆಚ್ಚಾಗಿ ಪಡೆಯಲು ಸಹಾಯ ಮಾಡುತ್ತದೆ.
\begin{description}
\item[ಶ್ಲೋಕಗಳ ಕ್ರಮಬದ್ಧ ಓದು:] 
ಈ ಪುಸ್ತಕವು ಗೀತೆಯ ಎಲ್ಲಾ ಶ್ಲೋಕಗಳನ್ನು ಅರ್ಥ ಸಹಿತವಾಗಿ ನೀಡುತ್ತದೆ. ‘ಮನನಮ್’ ಮತ್ತು ‘ಸ್ಫೂರ್ತಿ’ ವಿಭಾಗಗಳನ್ನು ಓದುವ ಮೊದಲು, ನೀಡಲಾದ ಕ್ರಮದಲ್ಲಿ ಶ್ಲೋಕಗಳನ್ನು ಓದಿರಿ.
\item[ಸಾಂದರ್ಭಿಕ ಅರಿವು:] ಮನನಮ್ ಪ್ರಶ್ನೆಗಳು ನಿರ್ದಿಷ್ಟ ಶ್ಲೋಕಗಳ ಮೇಲೆ ಕೇಂದ್ರಿತವಾಗಿದ್ದರೂ, ಹಿಂದಿನ ಶ್ಲೋಕಗಳನ್ನೂ ಓದುವುದರಿಂದ ವಿಶಾಲವಾದ ಸಂದರ್ಭವನ್ನು ಅರ್ಥ ಮಾಡಿಕೊಳ್ಳಲು ಸಹಾಯವಾಗುತ್ತದೆ.
\item[ಭಗವದ್ಗೀತಾ ಶ್ರವಣದಲ್ಲಿ ಪೂರ್ವಭೂಮಿಕೆ:] ಆದರ್ಶವಾಗಿ ಹೇಳುವುದಾದರೆ, ಒಬ್ಬ ಸಂಪ್ರದಾಯಬದ್ಧ ವೇದಾಂತ ಗುರುವಿನ ಮುಖೇನ  ಭಗವದ್ಗೀತೆಯ ಕ್ರಮಬದ್ಧ ವಿವರಣೆಯ ಶ್ರವಣ (ಶ್ರವಣಮ್) ಮಾಡಿಕೊಂಡಿದ್ದರೆ ಉತ್ತಮ;  ಆದಾಗ್ಯೂ, ಈ ಪುಸ್ತಕವನ್ನು ಬಳಸಲು ಅಂತಹ ಶ್ರವಣದ   ಔಪಚಾರಿಕ ಅಧ್ಯಯನ ಕಡ್ಡಾಯವಲ್ಲ.
\item[ಮನನಮ್ ಪ್ರಶ್ನೆಗಳ ಮೇಲಿನ ಆಳವಾದ ಚಿಂತನ:] 
ಶ್ಲೋಕಗಳನ್ನು ಓದಿದ ನಂತರ ಮನನಮ್ ಪ್ರಶ್ನೆಗಳನ್ನು ಆಳವಾಗಿ ಪರಾಮರ್ಶಸಿ; ಅವುಗಳನ್ನು ಹಲವಾರು ಬಾರಿ ಓದಿ, ನಿಮ್ಮ ಜೀವನಕ್ಕೆ ಅವು ಹೇಗೆ ಸಂಬಂಧಿಸುತ್ತವೆ ಎಂಬುದರ ಕುರಿತು ಚಿಂತನೆ ಮಾಡಿ. ಆ ಜ್ಞಾನವನ್ನು ನಿತ್ಯ ಜೀವನದಲ್ಲಿ ಪ್ರಾಯೋಗಿಕವಾಗಿ ಹೇಗೆ ಅಭ್ಯಾಸಿಸಬಹುದು ಅಥವಾ ಅಳವಡಿಸಿಕೊಳ್ಳಬಹುದು ಎಂಬುದನ್ನು ಕಂಡುಹಿಡಿಯಿರಿ.
\item[ಲೇಖನ ಮತ್ತು ಮರು ಪರಿಶೀಲನೆ:] 
ಪ್ರಶ್ನೆಗಳ ಬಗ್ಗೆ ಚಿಂತನೆ ಮಾಡಿ, ಅಗತ್ಯವಿದ್ದರೆ ಶ್ಲೋಕಗಳನ್ನು ಮತ್ತೊಮ್ಮೆ ಓದಿ. ಪುಸ್ತಕದಲ್ಲಿ ನೀಡಿರುವ ಸ್ಥಳದಲ್ಲಿ ಅಥವಾ ಬೇರೆ ನೋಟು ಪುಸ್ತಕದಲ್ಲಿ ನಿಮ್ಮ ಮನಸ್ಸಿಗೆ ಹೊಳೆದ ಅಂತರ್ದೃಷ್ಟಿಗಳು ಹಾಗೂ ಚಿಂತನೆಯನ್ನು ಲಿಖಿತವಾಗಿ ದಾಖಲಿಸಿ.
\item[‘ಜಿಜ್ಞಾಸೆ’ಯನ್ನು ಬೆಳೆಸುವುದು:] 
ಚಿಂತನೆಯ ಸಮಯದಲ್ಲಿ ಹೆಚ್ಚುವರಿ ಪ್ರಶ್ನೆಗಳು ಉದ್ಭವಿಸಿದರೆ ಅವುಗಳನ್ನೂ ಕೂಡ ಬರೆದುಕೊಳ್ಳಿ. ಈ ಅರಿವಿನ ಹಸಿವು (ಜಿಜ್ಞಾಸೆ )  ಸಹಜವಾಗಿ  ಆಳವಾದ ಅರ್ಥದ ಅನ್ವೇಷಣೆಗೆ 
ಉತ್ತೇಜನವಾಗುತ್ತದೆ.
\item[ದೃಷ್ಟಿಕೋನವನ್ನು ಪಡೆಯುವುದು:] 
‘ಸ್ಫೂರ್ತಿ’ದಾಯಕ ಟಿಪ್ಪಣಿಗಳನ್ನು ಓದಿ   
ಮುಕ್ತಾಯಗೊಳಿಸಿ; ಅವು ನಿಮ್ಮ ಒಳನೋಟಗಳೊಂದಿಗೆ ಹೊಂದಿಕೆಯಾಗುತ್ತವೆಯೇ ಅಥವಾ ಹೊಸ ದೃಷ್ಟಿಕೋನಗಳನ್ನು ಒದಗಿಸುತ್ತವೆಯೇ ಎಂಬುದನ್ನು ನಿಮ್ಮ ವ್ಯಾಖ್ಯಾನಗಳೊಂದಿಗೆ ಹೋಲಿಸಿ.
\item[ಅನೌಪಚಾರಿಕ ಚಿಂತನೆ:] ಕ್ರಮಬದ್ಧ ಅಧ್ಯಯನದ ಹೊರತಾಗಿ, ಚಿಂತನೆಗೆ ನಿಮ್ಮ ದಿನಚರಿಯಲ್ಲಿ ಸ್ಥಾನ ನೀಡಿ.  ನಿಮ್ಮೊಂದಿಗೆ ಸ್ಪಂದಿಸುವ ಮನನಮ್ ಪ್ರಶ್ನೆಗಳನ್ನು ಆಯ್ಕೆಮಾಡಿ ಮತ್ತು ದಿನದ ಮಧ್ಯೆ ಸಿಗುವ  ವಿರಾಮ ಕ್ಷಣಗಳಲ್ಲಿ ಅವುಗಳ ಬಗ್ಗೆ ಚಿಂತನೆ ನಡೆಸಿ.
\item[ಪ್ರಮಾಣಕ್ಕಿಂತ ಗಾಢತೆಯ ಪ್ರಾಮುಖ್ಯತೆ:]  ಈ ಪುಸ್ತಕದ ಪರಿಣಾಮವು “ಎಷ್ಟನ್ನು ಓದಿದ್ದೀರಿ” ಎಂಬುದಲ್ಲ, ಬದಲಾಗಿ ಚಿಂತನೆಯ ಗುಣಮಟ್ಟ ಮತ್ತು ಗಾಢತೆಯಲ್ಲಿ ಇರುತ್ತದೆ. ನೀವು ಐದು ನಿಮಿಷ ಓದಿದರೆ, ಕನಿಷ್ಠ 10-15 ನಿಮಿಷಗಳ ಕಾಲ ಪ್ರಾಮಾಣಿಕ ಚಿಂತನೆಗೆ ಅರ್ಪಿಸಿ; ಈ ಮೂಲಕ ಆ ಉಪದೇಶಗಳು ದಿನವಿಡೀ ನಿಮ್ಮ ಆಲೋಚನೆಗಳನ್ನು
ಅನೌಪಚಾರಿಕವಾಗಿ ವ್ಯಾಪಿಸಿಕೊಳ್ಳುತ್ತವೆ.
\item[ಬೋಧನೆಗಳನ್ನು ಜೀವಿಸುವುದು:] ಭಗವದ್ಗೀತೆಯ ಪರಿಮಳವು ನಿಮ್ಮ ಜೀವನದ ಪ್ರತಿಯೊಂದು ಅಂಶದಲ್ಲಿಯೂ ಸಮಾವೇಶವಾಗಲಿ; ಅದು ನಿಮ್ಮ ಸಂವಹನಗಳನ್ನು
ಸಮೃದ್ಧಿಗೊಳಿಸಿ, ನಿಮ್ಮ ಸುತ್ತಲೂ ಇರುವವರಿಗೆ ವಿವೇಕವನ್ನು ಪಸರಿಸಲಿ!
\end{description}