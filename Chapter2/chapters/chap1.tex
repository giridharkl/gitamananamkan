\newpage
\thispagestyle{empty}
\vspace*{10px}
\begin{center}
\begin{tikzpicture}[baseline={(0,0)}, color=Maroon, every node/.style={inner sep=0pt}] 
%\draw[help lines] (-5,-5) grid (5,5); 
\node[rectangle, draw=none, fill=none, thick, minimum width=10.6cm,
                        minimum height = 2cm] (vecbox) {}; 
\node[anchor=north west] at (vecbox.north west) {\pgfornament[width=2cm]{61}}; 
\node[anchor=north east] at (vecbox.north east) {\pgfornament[width=2cm,symmetry=v]{61}}; %\node[anchor=south west] at (vecbox.south west) {\pgfornament[width=2cm,symmetry=h]{61}}; %\node[anchor=south east] at (vecbox.south east) {\pgfornament[width=2cm,symmetry=c]{61}}; 
\end{tikzpicture}
\normalsize\textbf{\color{blue}ಧೃತರಾಷ್ಟ್ರ ಉವಾಚ ।\\
ಧರ್ಮಕ್ಷೇತ್ರೇ ಕುರುಕ್ಷೇತ್ರೇ ಸಮವೇತಾ ಯುಯುತ್ಸವಃ।\\
ಮಾಮಕಾಃ ಪಾಂಡವಾಶ್ಚೈವ ಕಿಮಕುರ್ವತ ಸಂಜಯ॥}
\end{center}
{\mananatext ಧೃತರಾಷ್ಟ್ರನು ಹೇಳಿದನು,\\
ಸಂಜಯನೇ, ಯುದ್ಧದ ಬಯಕೆಯಿಂದ ಧರ್ಮ ಭೂಮಿಯಾದ ಕುರುಕ್ಷೇತ್ರದಲ್ಲಿ ಕಲೆತ ನನ್ನ ಮಕ್ಕಳೂ ಪಾಂಡವರೂ ಏನು ಮಾಡಿದರು?\\}
\begin{mananam}{\mananamfont ಮನನ ಶ್ಲೋಕ - ೧}
{\small \mananatext ನನ್ನ ನಿತ್ಯ ಕರ್ಮದಲ್ಲಿ ದೇಹೆಂದ್ರೀಯ ಪ್ರವೃತ್ತಿಗಳಾದ ಆಸೆ, ಕೋಪ, ಭಯ, ಮತ್ಸರ ಇತ್ಯಾದಿಗಳೆಲ್ಲವನ್ನೂ ನನ್ನ ಅಂತರಂಗದಿಂದ ಪುಟ್ಟಿದೆದ್ದ, ತೀವ್ರತರ ಮುಕ್ತಿಯ ಹಂಬಲವು ಪ್ರತಿಭಟಿಸಿದಾಗ  ಇದಕ್ಕಾಗಿ, ಸನಾತನ ಗ್ರಂಥ ಮತ್ತು ಬೋಧಕರಿಂದ ಪಡೆದ ವಿವೇಕವನ್ನು ಅನುಸರಿಸಿದೆನೇ ? (ದೇಹೇಂದ್ರೀಯ ಪ್ರವೃತ್ತಿಗಳನ್ನು ದಮನಿಸಲು);  ಯಾವ ಶಕ್ತಿ ಅಥವಾ ಮಾರ್ಗವನ್ನು ಆಯ್ಕೆ ಮಾಡಿ ಅನುಸರಿಸಿದೆ?\\
ನನ್ನ ಶ್ರೇಷ್ಠವಾದ ಹಂಬಲ ಮತ್ತು ಸಂಕಲ್ಪಗಳನ್ನು ತಳ್ಳಿಹಾಕುವ ನನ್ನ ದುರಭ್ಯಾಸಗಳು ಮತ್ತು ಅಪಾಯಕಾರಿ ನಮೂನೆಗಳನ್ನು ನನ್ನ ನಿತ್ಯ ಜೀವನದ ಕದನದಲ್ಲಿ ( ಜಂಝಾಟಗಳಲ್ಲಿ) ಗುರುತಿಸಬಲ್ಲೆನೇ?
}
\end{mananam}
\begin{inspiration}{\mananamfont ಸ್ಪೂರ್ತಿ}
{\small \mananatext ನಿನ್ನಲ್ಲಿ ನೈಜಿಕತೆ (ಅಂದರೆ, ನಿನ್ನ ಮೌಲ್ಯಕ್ಕೆ ಅನುಗುಣವಾಗಿ ಬದುಕುವುದು) ಇರಲಿ,  ನೀನು ಉನ್ನತಿಯತ್ತ ಬದಲಾಗುವೆ.  ಪಕ್ಷಪಾತ ರಹಿತ ಅವಲೋಕನ, ಜೀವನ ಕೌಶಲ್ಯಕ್ಕೆ ಅತ್ಯಗತ್ಯ. ನಮ್ಮನ್ನು ನಾವು ಬದಲಾಯಿಸಿಕೊಳ್ಳಲು ಕೇವಲ ಬಯಕೆ ಇದ್ದರೆ ಮಾತ್ರ ಸಾಲದು, ಜ್ಞಾನಿಗಳ, ಋಷಿಗಳ ಮಹತ್ವದ ಉನ್ನತವಾದ ಬೋಧನೆಗಳಿಂದ ನಮ್ಮ ಯೋಚನೆಗಳು, ಮಾತುಗಳು ಮತ್ತು ಕೃತಿಗಳನ್ನು ತಹಬಂದಿಗೆ ತಂದು, ಪ್ರತಿದಿನವೂ ನಮ್ಮನ್ನು ನಾವು ಆತ್ಮ ವಿಮರ್ಶೆ ಮಾಡಿಕೊಳ್ಳಲೇಬೇಕು.}
\end{inspiration}
\newpage