\kanfont
\begin{center}\aksharfont\textbf
ಅಥ ಪಂಚಮೋऽಧ್ಯಾಯಃ।\\
\end{center}
ಅರ್ಜುನ ಉವಾಚ ।\\
ಸಂನ್ಯಾಸಂ ಕರ್ಮಣಾಂ ಕೃಷ್ಣ ಪುನರ್ಯೋಗಂ ಚ ಶಂಸಸಿ ।\\
ಯಚ್ಛ್ರೇಯ ಏತಯೋರೇಕಂ ತನ್ಮೇ ಬ್ರೂಹಿ ಸುನಿಶ್ಚಿತಮ್ ॥ 1 ॥
\begin{quoting}
ಅರ್ಜುನನ್ನು ಹೇಳಿದನು,\\
ಕೃಷ್ಣ, ಕರ್ಮಗಳ ತ್ಯಾಗವನ್ನು ಹೇಳುತ್ತೀ,  ಫಲದಾಸೆ ಇಲ್ಲದೆ ಮಾಡೆಂದು ಹೇಳುತ್ತಿ. ಇವೆರಡರಲ್ಲಿ ಯಾವುದು ಮೇಲೋ ಆ ಒಂದನ್ನು ನನಗೆ ಗೊತ್ತು ಮಾಡಿ ಹೇಳು.\\
\end{quoting}
ಶ್ರೀಭಗವಾನುವಾಚ ।\\
ಸಂನ್ಯಾಸಃ ಕರ್ಮಯೋಗಶ್ಚ ನಿಃಶ್ರೇಯಸಕರಾವುಭೌ ।\\
ತಯೋಸ್ತು ಕರ್ಮಸಂನ್ಯಾಸಾತ್ಕರ್ಮಯೋಗೋ ವಿಶಿಷ್ಯತೇ ॥ 2 ॥
\begin{quoting}
ಭಗವಂತನು ಹೇಳಿದನು,\\
ಸನ್ಯಾಸ ಹಾಗೂ ಕರ್ಮಯೋಗ ಎರಡು ಶ್ರೇಯಸ್ಸನ್ನು ಉಂಟುಮಾಡುವಂತಹವೇ. ಅವುಗಳಲ್ಲಿ ಕರ್ಮ ಸನ್ಯಾಸಕ್ಕಿಂತ ಕರ್ಮ ಯೋಗವು ಶ್ರೇಯಸ್ಕರ.\\
\begin{inspiration}{\kanfont ಸ್ಪೂರ್ತಿ}
\small\it
Be true to yourself and you shall change for the better. Impartial observation is an essential life-skill.\\
It is not enough to have a mere wish to change oneself. One must Introspect each day on one’s thoughts, words, and actions
\end{inspiration}
\end{quoting}
ಙ್ಞೇಯಃ ಸ ನಿತ್ಯಸಂನ್ಯಾಸೀ ಯೋ ನ ದ್ವೇಷ್ಟಿ ನ ಕಾಂಕ್ಷತಿ ।\\
ನಿರ್ದ್ವಂದ್ವೋ ಹಿ ಮಹಾಬಾಹೋ ಸುಖಂ ಬಂಧಾತ್ಪ್ರಮುಚ್ಯತೇ ॥ 3 ॥
\begin{quoting}
ಮಹಾಬಾಹೋ! ಒಂದು ಬೇಕೆನ್ನದೆ,ಒಂದು ಬೇಡವೆನ್ನದೆ ಇರುವವನೇ ನಿತ್ಯ ಸನ್ಯಾಸಿ. ದ್ವಂದ್ವ ರಹಿತನಾದ ಅವನು ಬಂಧನದಿಂದ ಸುಖವಾಗಿ ತಪ್ಪಿಸಿಕೊಳ್ಳುತ್ತಾನೆ.\\
\end{quoting}
ಸಾಂಖ್ಯಯೋಗೌ ಪೃಥಗ್ಬಾಲಾಃ ಪ್ರವದಂತಿ ನ ಪಂಡಿತಾಃ ।\\
ಏಕಮಪ್ಯಾಸ್ಥಿತಃ ಸಮ್ಯಗುಭಯೋರ್ವಿಂದತೇ ಫಲಮ್ ॥ 4 ॥
\begin{quoting}
ಜ್ಞಾನಮಾರ್ಗ ಕರ್ಮ ಮಾರ್ಗಗಳು ಬೇರೆ ಬೇರೆ ಎಂಬುದು ತಿಳಿಯದವರ ಮಾತೇ ಹೊರತು ತಿಳಿದವರ ಮಾತಲ್ಲ. ಯಾವ ಒಂದನ್ನು ಚೆನ್ನಾಗಿ ನಡೆಸಿದರು ಅವನು ಎರಡರ ಫಲವನ್ನು ಪಡೆಯುತ್ತಾನೆ.\\
\end{quoting}
ಯತ್ಸಾಂಖ್ಯೈಃ ಪ್ರಾಪ್ಯತೇ ಸ್ಥಾನಂ ತದ್ಯೋಗೈರಪಿ ಗಮ್ಯತೇ ।\\
ಏಕಂ ಸಾಂಖ್ಯಂ ಚ ಯೋಗಂ ಚ ಯಃ ಪಶ್ಯತಿ ಸ ಪಶ್ಯತಿ ॥ 5 ॥
\begin{quoting}
ಯಾವ ಯೋಗಿಗಳಿಗೆ ಲಭಿಸುವ ಫಲವು ಕರ್ಮಯೋಗಿಗಳಿಗೂ ಲಭಿಸುತ್ತದೆ. ಆ ಎರಡನ್ನು ಒಂದೇ ಎಂದು ತಿಳಿದವನು ಯಥಾರ್ಥ ಜ್ಞಾನಿ.\\
\end{quoting}
ಸಂನ್ಯಾಸಸ್ತು ಮಹಾಬಾಹೋ ದುಃಖಮಾಪ್ತುಮಯೋಗತಃ ।\\
ಯೋಗಯುಕ್ತೋ ಮುನಿರ್ಬ್ರಹ್ಮ ನಚಿರೇಣಾಧಿಗಚ್ಛತಿ ॥ 6 ॥
\begin{quoting}
ಅರ್ಜುನ ಕರ್ಮ ಯೋಗದ ನೆರವಿಲ್ಲದೆ ರಾಗ ದ್ವೇಷಗಳನ್ನು ಮೀರಿ ನಿಲ್ಲುವುದು ಕಷ್ಟ. ಫಲದಾಸೆ ಇಲ್ಲದೆ ಕರ್ಮ ಮಾಡುವ ದ್ವಂದ್ವತೀತನು ತಡವಿಲ್ಲದೆ ಆತ್ಮ ಜ್ಞಾನವನ್ನು ಹೊಂದುತ್ತಾನೆ.\\
\end{quoting}
ಯೋಗಯುಕ್ತೋ ವಿಶುದ್ಧಾತ್ಮಾ ವಿಜಿತಾತ್ಮಾ ಜಿತೇಂದ್ರಿಯಃ ।\\
ಸರ್ವಭೂತಾತ್ಮಭೂತಾತ್ಮಾ ಕುರ್ವನ್ನಪಿ ನ ಲಿಪ್ಯತೇ ॥ 7 ॥
\begin{quoting}
 ಕರ್ಮ ಯೋಗದಲ್ಲಿ ನಿರತನಾಗಿ ಶುದ್ಧವಾದ ಮನಸ್ಸಿನಿಂದ ದೇಹವನ್ನು ಹಿಡಿತದಲ್ಲಿಟ್ಟುಕೊಂಡು ಇಂದ್ರಿಯಗಳನ್ನು ಬಿಗಿಹಿಡಿದು ಎಲ್ಲೆಡೆಯೂ ಅಂತರ್ಯಾಮಿ ಯಾಗಿರುವ ಭಗವಂತನನ್ನೇ ಸದಾ ನೆನೆಸುವವನು ಕರ್ಮ ಮಾಡಿದರೂ ಅದರ ಕಟ್ಟಿಗೊಳಗಾಗುವುದಿಲ್ಲ.\\
\end{quoting}
ನೈವ ಕಿಂಚಿತ್ಕರೋಮೀತಿ ಯುಕ್ತೋ ಮನ್ಯೇತ ತತ್ತ್ವವಿತ್ ।\\
ಪಶ್ಯಞ್ಶೃಣ್ವನ್ಸ್ಪೃಶಂಜಿಘ್ರನ್ನಶ್ನನ್ಗಚ್ಛನ್ಸ್ವಪಞ್ಶ್ವಸನ್ ॥ 8 ॥\\
ಪ್ರಲಪನ್ವಿಸೃಜನ್ಗೃಹ್ಣನ್ನುನ್ಮಿಷನ್ನಿಮಿಷನ್ನಪಿ ।\\
ಇಂದ್ರಿಯಾಣೀಂದ್ರಿಯಾರ್ಥೇಷು ವರ್ತಂತ ಇತಿ ಧಾರಯನ್ ॥ 9 ॥
\begin{quoting}
ಆತ್ಮ ಸ್ವರೂಪವನ್ನು ತಿಳಿದು ಮನಸ್ಸನ್ನು ಹಿಡಿತದಲ್ಲಿ ಇಟ್ಟುಕೊಂಡಿರುವವನು ನೋಡಿದರೂ ಕೇಳಿದರೂ ಮುಟ್ಟಿದರರೂ ಮೂಸಿದರೂ ತಿಂದರೂ ಹೋದರೂ ನಿದ್ರೆ ಮಾಡಿದರರೂ ಉಸಿರುಬಿಟ್ಟರೂ ಮಾತನಾಡಿದರೂ ಮಲಮೂತ್ರಗಳನ್ನು ಬಿಟ್ಟರೂ ತೆಗೆದುಕೊಂಡರೂ ಕಣ್ಣು ಬಿಟ್ಟರೂ ಕಣ್ಣು ಮುಚ್ಚಿದರೂ ಇಂದ್ರಿಯಗಳು ವಿಷಯಗಳೊಡನೆ ವ್ಯವಹರಿಸುತ್ತಿರುವವೆಂದು ಮನಸ್ಸಿನಲ್ಲಿ ಎಣಿಸುತ್ತಾ ತಾನು ಏನನ್ನೂ ಮಾಡುವವನಲ್ಲವೆಂದು ತಿಳಿಯುವನು.\\
\end{quoting}
ಬ್ರಹ್ಮಣ್ಯಾಧಾಯ ಕರ್ಮಾಣಿ ಸಂಗಂ ತ್ಯಕ್ತ್ವಾ ಕರೋತಿ ಯಃ ।\\
ಲಿಪ್ಯತೇ ನ ಸ ಪಾಪೇನ ಪದ್ಮಪತ್ರಮಿವಾಂಭಸಾ ॥ 10 ॥
\begin{quoting}
ಈಶ್ವರಾರ್ಪಣ ಬುದ್ಧಿಯಿಂದ ಫಲಾಸಕ್ತಿಯನ್ನು ಬಿಟ್ಟು ಕರ್ಮ ಮಾಡುವವನನ್ನು ಕಮಲದ ಎಲೆಯನ್ನು ನೀರು  ಮುಟ್ಟದಂತೆ, ಪಾಪವು ಅಂಟುವುದಿಲ್ಲ.\\
\end{quoting}
ಕಾಯೇನ ಮನಸಾ ಬುದ್ಧ್ಯಾ ಕೇವಲೈರಿಂದ್ರಿಯೈರಪಿ ।\\
ಯೋಗಿನಃ ಕರ್ಮ ಕುರ್ವಂತಿ ಸಂಗಂ ತ್ಯಕ್ತ್ವಾತ್ಮಶುದ್ಧಯೇ ॥ 11 ॥
\begin{quoting}
ಕರ್ಮಯೋಗಿಗಳು ಆಸಕ್ತಿಯನ್ನು ಬಿಟ್ಟು ಶರೀರದಿಂದ ಮನಸ್ಸಿನಿಂದ ಬುದ್ಧಿಯಿಂದ ಆಯಾ ಇಂದ್ರಿಯಗಳಿಂದ ಆತ್ಮ ಶುದ್ಧಿಗಾಗಿ ಕರ್ಮಗಳನ್ನು ಮಾಡುತ್ತಾರೆ.\\
\end{quoting}
ಯುಕ್ತಃ ಕರ್ಮಫಲಂ ತ್ಯಕ್ತ್ವಾ ಶಾಂತಿಮಾಪ್ನೋತಿ ನೈಷ್ಠಿಕೀಮ್ ।\\
ಅಯುಕ್ತಃ ಕಾಮಕಾರೇಣ ಫಲೇ ಸಕ್ತೋ ನಿಬಧ್ಯತೇ ॥ 12 ॥
\begin{quoting}
ಕರ್ಮದ ಫಲವನ್ನು ತೊರೆದು ಈಶ್ವರನ ಪ್ರೀತಿಗಾಗಿ ಮಾಡುವವನು ಶಾಶ್ವತವಾದ ಶಾಂತಿಯನ್ನು ಪಡೆಯುತ್ತಾನೆ. ಬಯಕೆಗಳಿಗೆ ಬಲಿಯಾಗಿ ಫಲದಾಸೆಯಿಂದ ಕರ್ಮ ಮಾಡುವವನು ಬಾಳಿನ ಬಂಧನಕ್ಕೂಳಗಾಗುವನು.\\
\end{quoting}
ಸರ್ವಕರ್ಮಾಣಿ ಮನಸಾ ಸಂನ್ಯಸ್ಯಾಸ್ತೇ ಸುಖಂ ವಶೀ ।\\
ನವದ್ವಾರೇ ಪುರೇ ದೇಹೀ ನೈವ ಕುರ್ವನ್ನ ಕಾರಯನ್ ॥ 13 ॥
\begin{quoting}
ಜ್ಞಾನಿಯು ದೇಹೇಂದ್ರಿಯಗಳನ್ನು ಹಿಡಿತದಲ್ಲಿಟ್ಟುಕೊಂಡು ತಾನು ಮಾಡುವವನಲ್ಲವೆಂಬ ಅನುಸಂಧಾನದಿಂದ 9 ಬಾಗಿಲು ಉಳ್ಳ ದೇಹವೆಂಬ ಪಟ್ಟಣದಲ್ಲಿ ಹಾಯಾಗಿರುವನು. ಅವನು ಮಾಡಿದರೂ ಮಾಡಿದಂತಲ್ಲ. ಮಾಡಿಸಿದರೂ ಮಾಡಿಸಿದ್ದಂತಲ್ಲ.\\
\end{quoting}
ನ ಕರ್ತೃತ್ವಂ ನ ಕರ್ಮಾಣಿ ಲೋಕಸ್ಯ ಸೃಜತಿ ಪ್ರಭುಃ ।\\
ನ ಕರ್ಮಫಲಸಂಯೋಗಂ ಸ್ವಭಾವಸ್ತು ಪ್ರವರ್ತತೇ ॥ 14 ॥
\begin{quoting}
ಜಗತ್ತಿನಲ್ಲಿ ಮಾಡುವಿಕೆ, ಮಾಡಿದ ಕ್ರಿಯೆ ಮತ್ತು ಕ್ರಿಯೆಗೆ ತಕ್ಕ ಪ್ರತಿಫಲ ಇವು ಯಾವುದನ್ನು ಆತ್ಮನು ನಿರ್ಮಿಸಲಾರ. ಎಲ್ಲವೂ ಸ್ವಭಾವಕ್ಕೆ ತಕ್ಕಂತೆ ಭಗವಂತನ ಇಚ್ಛೆಯಂತೆ ನಡೆದಿದೆ.\\
\end{quoting}
ನಾದತ್ತೇ ಕಸ್ಯಚಿತ್ಪಾಪಂ ನ ಚೈವ ಸುಕೃತಂ ವಿಭುಃ ।\\
ಅಙ್ಞಾನೇನಾವೃತಂ ಙ್ಞಾನಂ ತೇನ ಮುಹ್ಯಂತಿ ಜಂತವಃ ॥ 15 ॥
\begin{quoting}
ಎಲ್ಲೆಲ್ಲಿಯೂ ತುಂಬಿರುವ ಭಗವಂತ ಯಾರ ಪಾಪ ಪುಣ್ಯಗಳಿಗೂ ತಾನು ಬಾಗಿಯಲ್ಲ. ಅರಿವನ್ನು ಅಜ್ಞಾನದ ತೆರೆ ಮುಚ್ಚಿದೆ. ಅದರಿಂದ ಜನ ಭ್ರಮಿಸುತ್ತಾರೆ, ಅಷ್ಟೇ.\\
\end{quoting}
ಙ್ಞಾನೇನ ತು ತದಙ್ಞಾನಂ ಯೇಷಾಂ ನಾಶಿತಮಾತ್ಮನಃ ।\\
ತೇಷಾಮಾದಿತ್ಯವಜ್ಙ್ಞಾನಂ ಪ್ರಕಾಶಯತಿ ತತ್ಪರಮ್ ॥ 16 ॥
\begin{quoting}
ಆತ್ಮದ ಅರಿವಿನ ಬೆಳಕಿನಿಂದ ಅಜ್ಞಾನದ ಕತ್ತಲನ್ನು ಕಳೆದುಕೊಂಡವರಿಗೆ ಅವರ ಆ ಅರಿವೇ ಸೂರ್ಯನಂತೆ ಆ ಪರತತ್ವವನ್ನು ಬೆಳಗಿಸುತ್ತದೆ.\\
\end{quoting}
ತದ್ಬುದ್ಧಯಸ್ತದಾತ್ಮಾನಸ್ತನ್ನಿಷ್ಠಾಸ್ತತ್ಪರಾಯಣಾಃ ।\\
ಗಚ್ಛಂತ್ಯಪುನರಾವೃತ್ತಿಂ ಙ್ಞಾನನಿರ್ಧೂತಕಲ್ಮಷಾಃ ॥ 17 ॥
\begin{quoting}
ಭಗವಂತನಲ್ಲಿ ಬುದ್ಧಿ ನೆಲೆಗೊಳ್ಳಬೇಕು,ಬಾಳು ಅವನಿಗೆ ಅರ್ಪಿತವಾಗಬೇಕು.ಅವನೆ ಜೀವನದ ಗುರಿ ನೆಲೆಯಾಗಬೇಕು. ಇಂಥ ಅರವಿನಿಂದ ಪಾಪದ ಕೊಳೆಯನ್ನು ತೊಳೆದುಕೊಂಡವರು ಮತ್ತೆ ಮರಳದ ಶಾಶ್ವತ ಪದವನ್ನು ಪಡೆಯುವರು.\\
\end{quoting}
ವಿದ್ಯಾವಿನಯಸಂಪನ್ನೇ ಬ್ರಾಹ್ಮಣೇ ಗವಿ ಹಸ್ತಿನಿ ।\\
ಶುನಿ ಚೈವ ಶ್ವಪಾಕೇ ಚ ಪಂಡಿತಾಃ ಸಮದರ್ಶಿನಃ ॥ 18 ॥
\begin{quoting}
ವಿದ್ಯೆ ವಿನಯಗಳ ನೆಲೆಯಾದ ಬ್ರಹ್ಮ ಜ್ಞಾನಿ,ಹಸು, ಆನೆ,ನಾಯಿ, ನಾಯಮಾಂಸ ತಿಂದು ಬದುಕುವ ಅನಾಗರಿಕ ಈ ಎಲ್ಲರಲ್ಲಿಯೂ ಒಬ್ಬನೇ ಭಗವಂತ ನೆಲೆಸಿದ್ದಾನೆ.ಅವನಿಗೆ ಕೀಳು ಮೇಲೆಂಬುದಿಲ್ಲ ಎಂದು ಜ್ಞಾನಿಗಳು ತಿಳಿಯುತ್ತಾರೆ.\\
\end{quoting}
ಇಹೈವ ತೈರ್ಜಿತಃ ಸರ್ಗೋ ಯೇಷಾಂ ಸಾಮ್ಯೇ ಸ್ಥಿತಂ ಮನಃ ।\\
ನಿರ್ದೋಷಂ ಹಿ ಸಮಂ ಬ್ರಹ್ಮ ತಸ್ಮಾದ್ಬ್ರಹ್ಮಣಿ ತೇ ಸ್ಥಿತಾಃ ॥ 19 ॥
\begin{quoting}
ನಿರ್ದೋಷವಾದ ಬ್ರಹ್ಮವು ಸಮವಾಗಿದೆ. ಆ ಬ್ರಹ್ಮದಲ್ಲಿಯೇ ಇರುತ್ತಾರೆ. ಸಾಮ್ಯ ಬುದ್ಧಿಯುಳ್ಳ ದೊಡ್ಡವರು ಬದುಕಿರುವಾಗಲೇ ಮುಕ್ತರಾಗುತ್ತಾರೆ.\\
\end{quoting}
ನ ಪ್ರಹೃಷ್ಯೇತ್ಪ್ರಿಯಂ ಪ್ರಾಪ್ಯ ನೋದ್ವಿಜೇತ್ಪ್ರಾಪ್ಯ ಚಾಪ್ರಿಯಮ್ ।\\
ಸ್ಥಿರಬುದ್ಧಿರಸಂಮೂಢೋ ಬ್ರಹ್ಮವಿದ್ಬ್ರಹ್ಮಣಿ ಸ್ಥಿತಃ ॥ 20 ॥
\begin{quoting}
ಸ್ಥಿರಬುದ್ದಿಯುಳ್ಳವನೂ ಮೋಹರಹಿತನು ಆದ ಬ್ರಹ್ಮ ಜ್ಞಾನಿಯು ಬ್ರಹ್ಮದಲ್ಲಿಯೇ ಸ್ಥಿರವಾಗಿರುವುದರಿಂದ ಅವನಿಗೆ ಪ್ರಿಯ ವಸ್ತುವಿನಲ್ಲಿ ಇಚ್ಚೆ ಇಲ್ಲ, ಅಪ್ರಿಯವಾದ ವಸ್ತುವಿನಲ್ಲಿ ದ್ವೇಷವು ಇರುವುದಿಲ್ಲ.\\
\end{quoting}
ಬಾಹ್ಯಸ್ಪರ್ಶೇಷ್ವಸಕ್ತಾತ್ಮಾ ವಿಂದತ್ಯಾತ್ಮನಿ ಯತ್ಸುಖಮ್ ।\\
ಸ ಬ್ರಹ್ಮಯೋಗಯುಕ್ತಾತ್ಮಾ ಸುಖಮಕ್ಷಯಮಶ್ನುತೇ ॥ 21 ॥
\begin{quoting}
ಹೊರಗಣ ವಿಷಯಗಳಿಗೆ ಮನಸೋಲದವನು ತನ್ನೊಳಗೆಯೆ ಯಾವ ಸುಖವನ್ನು ಪಡೆಯುವನೂ ಅದೇ ಸುಖವನ್ನು ಬ್ರಹ್ಮತತ್ವದಲ್ಲಿ ಮನಸ್ಸನ್ನು ನೆಲೆಗೊಳಿಸಿದವನು ಅನಂತವಾಗಿ ಪಡೆಯುತ್ತಾನೆ.\\
\end{quoting}
ಯೇ ಹಿ ಸಂಸ್ಪರ್ಶಜಾ ಭೋಗಾ ದುಃಖಯೋನಯ ಏವ ತೇ ।\\
ಆದ್ಯಂತವಂತಃ ಕೌಂತೇಯ ನ ತೇಷು ರಮತೇ ಬುಧಃ ॥ 22 ॥
\begin{quoting}
ವಿಷಯಗಳ ಸಂಬಂಧದಿಂದಾಗುವ ಸುಖಗಳೆಲ್ಲ ದುಃಖಕ್ಕೆ ಮೂಲವಾದುವುಗಳೇ. ಅರ್ಜುನ,ಅವು ಮೊದಲೂ ಕೊನೆಯೂ ಉಳ್ಳವುಗಳು. ವಿವೇಕಿಯು ಅವುಗಳಲ್ಲಿ ಪ್ರೀತಿಯನ್ನಿಡುವುದಿಲ್ಲ.\\
\end{quoting}
ಶಕ್ನೋತೀಹೈವ ಯಃ ಸೋಢುಂ ಪ್ರಾಕ್ಶರೀರವಿಮೋಕ್ಷಣಾತ್ ।\\
ಕಾಮಕ್ರೋಧೋದ್ಭವಂ ವೇಗಂ ಸ ಯುಕ್ತಃ ಸ ಸುಖೀ ನರಃ ॥ 23 ॥
\begin{quoting}
ಸಾಯುವ ಮುಂಚೆಯೇ ಬಯಕೆ ಸಿಟ್ಟು ಮೊದಲಾದವುಗಳ ಸೆಳೆತವನ್ನು ಈ ಲೋಕದಲ್ಲಿಯೇ ತಡೆಯಬಲ್ಲವನೇ ನಿಜವಾದ ಯೋಗಿ, ಅವನೇ ನಿಜವಾದ ಸುಖಿ.\\
\end{quoting}
ಯೋऽಂತಃಸುಖೋऽಂತರಾರಾಮಸ್ತಥಾಂತರ್ಜ್ಯೋತಿರೇವ ಯಃ ।
\begin{quoting}
ಸ ಯೋಗೀ  ಆತ್ಮಸುಖದಲ್ಲಿ ಲೀನನಾಗಿ ಭಗವಂತನ ದರ್ಶನದಿಂದ ಆನಂದಗೊಳ್ಳುತ್ತ. ಒಳಗೆ ಆ ಬೆಳಕನ್ನೇ ತುಂಬಿಕೊಂಡಿರುವನೋ ಬ್ರಹ್ಮನಲ್ಲಿಯೇ ನೆಲೆಗೊಂಡ ಆ ಯೋಗಿಯು ಆನಂದರೂಪಿಯಾದ ಬ್ರಹ್ಮನನ್ನೇ ಪಡೆಯುತ್ತಾನೆ.\\
\end{quoting}
ಲಭಂತೇ ಬ್ರಹ್ಮನಿರ್ವಾಣಮೃಷಯಃ ಕ್ಷೀಣಕಲ್ಮಷಾಃ ।\\
ಛಿನ್ನದ್ವೈಧಾ ಯತಾತ್ಮಾನಃ ಸರ್ವಭೂತಹಿತೇ ರತಾಃ ॥ 25 ॥
\begin{quoting}
ಪಾಪರಹಿತರು, ಸಂಶಯ ಶೂನ್ಯರು, ಜಿತೇಂದ್ರಿಯರು, ಸರ್ವಭೂತಗಳ ಹಿತದಲ್ಲಿ ನಿರತರಾದ ಋಷಿಗಳು ಪರಮಮುಕ್ತಿಯನ್ನು ಪ್ರಾಪ್ತಿ ಮಾಡಿಕೊಳ್ಳುತ್ತಾರೆ.\\
\end{quoting}
ಕಾಮಕ್ರೋಧವಿಯುಕ್ತಾನಾಂ ಯತೀನಾಂ ಯತಚೇತಸಾಮ್ ।\\
ಅಭಿತೋ ಬ್ರಹ್ಮನಿರ್ವಾಣಂ ವರ್ತತೇ ವಿದಿತಾತ್ಮನಾಮ್ ॥ 26 ॥
\begin{quoting}
ಬಯಕೆ ಸಿಟ್ಟುಗಳನ್ನು ತೊರೆದು ಮನಸ್ಸನ್ನು ಬಿಗಿಹಿಡಿದು ಆತ್ಮವನ್ನರಿತ ಸನ್ಯಾಸಿಗಳಿಗೆ ಎಲ್ಲೆಡೆಯೂ ಆನಂದ ರೂಪವಾದ ಬ್ರಹ್ಮವೇ ತುಂಬಿದೆ.\\
\end{quoting}
ಸ್ಪರ್ಶಾನ್ಕೃತ್ವಾ ಬಹಿರ್ಬಾಹ್ಯಾಂಶ್ಚಕ್ಷುಶ್ಚೈವಾಂತರೇ ಭ್ರುವೋಃ ।\\
ಪ್ರಾಣಾಪಾನೌ ಸಮೌ ಕೃತ್ವಾ ನಾಸಾಭ್ಯಂತರಚಾರಿಣೌ ॥ 27 ॥\\
ಯತೇಂದ್ರಿಯಮನೋಬುದ್ಧಿರ್ಮುನಿರ್ಮೋಕ್ಷಪರಾಯಣಃ ।\\
ವಿಗತೇಚ್ಛಾಭಯಕ್ರೋಧೋ ಯಃ ಸದಾ ಮುಕ್ತ ಏವ ಸಃ ॥ 28 ॥
\begin{quoting}
ಹೊರಗಣ ವಿಷಯಗಳನ್ನು ಬಹಿಷ್ಕರಿಸಿ ಕಣ್ಣನ್ನು ಹುಬ್ಬುಗಳ ನಡುವೆ ನೆಲೆಗೊಳಿಸಿ ಮೂಗಿನೊಳಗೆ ಓಡಾಡುವ ಉಸಿರನ್ನು ಕುಂಭಕದಲ್ಲಿ ಬಿಗಿಹಿಡಿದು, ಇಂದ್ರಿಯ, ಮನಸ್ಸು, ಬುದ್ಧಿ ಇವುಗಳನ್ನು ಹಿಡಿತದಲ್ಲಿಟ್ಟುಕೊಂಡು, ಬಯಕೆ, ಭಯ, ಕೋಪಗಳನ್ನು ತೊರೆದು ಭವದ ಬಿಡುಗಡೆಯನ್ನೇ ಬಯಸುವ ಮುನಿ ಯಾವಾಗಲೂ ಮುಕ್ತನೇ.\\
\end{quoting}
ಭೋಕ್ತಾರಂ ಯಙ್ಞತಪಸಾಂ ಸರ್ವಲೋಕಮಹೇಶ್ವರಮ್ ।\\
ಸುಹೃದಂ ಸರ್ವಭೂತಾನಾಂ ಙ್ಞಾತ್ವಾ ಮಾಂ ಶಾಂತಿಮೃಚ್ಛತಿ ॥ 29 ॥
\begin{quoting}
ನನ್ನನ್ನು ಯಜ್ಞದ ಮತ್ತು ತಪಸ್ಸಿನ ಫಲವನ್ನು ಉಣ್ಣುವವನೆಂದೂ ಎಲ್ಲಾ ಲೋಕಗಳಿಗೂ ಒಡೆಯನೆಂದು ಎಲ್ಲ ಪ್ರಾಣಿಗಳಿಗೂ ಗೆಳೆಯನೆಂದು ತಿಳಿದವನಿಗೆ ಮುಕ್ತಿ ಕೈಯಗಂಟು.\\
\end{quoting}
\begin{center}
{\tiny\color{brown}
ಓಂ ತತ್ಸದಿತಿ ಶ್ರೀಮದ್ಭಗವದ್ಗೀತಾಸೂಪನಿಷತ್ಸು\\
ಬ್ರಹ್ಮವಿದ್ಯಾಯಾಂ ಯೋಗಶಾಸ್ತ್ರೇ ಶ್ರೀಕೃಷ್ಣಾರ್ಜುನಸಂವಾದೇ\\
ಕರ್ಮಸಂನ್ಯಾಸಯೋಗೋ ನಾಮ ಪಂಚಮೋऽಧ್ಯಾಯಃ ॥ 5 ॥\\}
\end{center}