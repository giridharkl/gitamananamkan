\footnotesize \mananamtext{ಜೀವನವು ಪ್ರತಿಯೊಬ್ಬರ ಮುಂದೆಯೂ ತನ್ನದೇ ಆದ ವಿಶಿಷ್ಟ ಸವಾಲುಗಳು ಮತ್ತು ಪರೀಕ್ಷೆಗಳನ್ನು ಒಡ್ಡುತ್ತದೆ. ಕುರುಕ್ಷೇತ್ರದ ಯುದ್ಧಭೂಮಿಯಲ್ಲಿ, ಅರ್ಜುನನು ಅತ್ಯಂತ ಮಹತ್ವದ ನೈತಿಕ ಮತ್ತು ಅಸ್ತಿತ್ವಾತ್ಮಕ ಸಂಕಟವನ್ನು ಎದುರಿಸಿದಾಗ, ಭಗವಾನ್ ಕೃಷ್ಣನು ತನ್ನ ವ್ಯಾಕುಲಗೊಂಡ ಶಿಷ್ಯನಿಗೆ, ಲೌಕಿಕ ಜೀವನದಲ್ಲೇ ಆತ್ಮತತ್ವ ಕಂಡುಹಿಡಿಯುವ ಪವಿತ್ರ ಉಪದೇಶವನ್ನು ನೀಡಿದನು — ಅದನ್ನು ಸಂಕ್ಷಿಪ್ತವಾಗಿಯೂ ಆಚರಣೆಯೋಗ್ಯವಾಗಿಯೂ ಬೋಧಿಸಿದನು. ಉಪನಿಷತ್ತುಗಳ ನಿಜ ಸತ್ವಗಳನ್ನು ಒಳಗೊಂಡ ಈ ಶ್ರೇಷ್ಠವಾದ ಬೋಧನೆಗಳು - ಋಷಿ ವೇದವ್ಯಾಸರು,  ಪವಿತ್ರವಾದ ‘ಒಂದು ದಿವ್ಯ ಗಾನದ‘ ರೂಪದಲ್ಲಿ ಶ್ರೀಮದ್ಭಗವದ್ಗೀತೆ ಯನ್ನು  ಪಾರಂಪಾರಿಕವಾಗಿ ನಮಗೆ ನೀಡಿರುವ ಒಂದು ಅಸೀಮವಾದ ಕೊಡುಗೆ.\\

ನಿಮ್ಮ ಪರಿಸ್ಥಿತಿಗಳು ಅರ್ಜುನನು ಎದುರಿಸಿದ್ದಕ್ಕಿಂತ ಭಿನ್ನವಾಗಿರಬಹುದು; ನಿಮ್ಮ ಸಂಘರ್ಷಗಳೂ ಸವಾಲುಗಳೂ ವಿಭಿನ್ನವಾಗಿರಬಹುದು, ಮತ್ತು ನಿಮ್ಮ ಪಥವು ಅನಿಶ್ಚಿತವಾಗಿರಬಹುದು. ಆದಾಗ್ಯೂ, ಗೀತೆಯ ಸಾರ್ವತ್ರಿಕ ಉಪದೇಶಗಳು ಸತ್ಯಾನ್ವೇಷಣೆ ಮಾಡುವ ಪ್ರತಿಯೊಬ್ಬ ಸಾಧಕನಿಗೂ ಆತ್ಮೋನ್ನತಿ ಮತ್ತು ಆಧ್ಯಾತ್ಮಿಕ ಬೆಳವಣಿಗೆಯ ಆದರ್ಶ ಮಾದರಿಯನ್ನು ನೀಡುತ್ತವೆ.\\

ಶ್ರೀಮದ್ಭಗವದ್ಗೀತೆಯ  ಉಪದೇಶಗಳು ಕೇವಲ ಆಧ್ಯಾತ್ಮಿಕ ಸಾಧಕರಿಗಷ್ಟೇ ಸಮರ್ಪಿತವಾದವುಗಳಲ್ಲ — ಅದು ಪ್ರತಿಯೊಬ್ಬರ ನಿತ್ಯಜೀವನದಲ್ಲಿ ಪ್ರಾಯೋಗಿಕ ಮಾರ್ಗದರ್ಶಕ  ಪುಸ್ತಕವಾಗಿಯೂ ಕಾರ್ಯನಿರ್ವಹಿಸುತ್ತದೆ; ಯಾರು  ಕೆಲಸ ಮತ್ತು ಕೌಟುಂಬಿಕ ಹೊಣೆಗಾರಿಕೆಗಳಲ್ಲಿ ಸಮತೋಲನಕ್ಕಾಗಿ ಅರಸುತ್ತಿರುವರೋ, ಮನಶಾಂತಿಯನ್ನು ಕಾಪಾಡಿಕೊಳ್ಳಲು ಮತ್ತು ಒತ್ತಡದಿಂದ ಮುಕ್ತಿ ಪಡೆಯಲು ಬಯಸುತ್ತಿರುವರೋ, ಅಂತಹವರು ಈ ಬೋಧನೆಗಳು ಅಮೂಲ್ಯವೆಂದು ಕಂಡುಕೊಳ್ಳುತ್ತಾರೆ.\\

ಈ ಕಾರ್ಯಪತ್ರ ಶೈಲಿಯ ದಿನಚರಿಯು ಗೀತೆಯ ಉಪದೇಶಗಳನ್ನು ಬೋಧಿಸುವುದು ಅಥವಾ ಅದನ್ನು ವ್ಯಾಖ್ಯಾನಿಸುವುದು ಎಂಬ ಉದ್ದೇಶದಿಂದ ರಚಿಸಲ್ಪಟ್ಟದ್ದಲ್ಲ — ಅನೇಕ ಪೂಜ್ಯ ಗುರುಗಳು ಮತ್ತು ಪಂಡಿತರು ಅದನ್ನು ಈಗಾಗಲೇ ಗಾಢವಾದ ತಾತ್ವಿಕ ಅರ್ಥದಲ್ಲಿ ವಿವರಿಸಿದ್ದಾರೆ. ಬದಲಿಗೆ, ಗೀತಾ ಮನನಂ ಎಂಬುದು ಗೀತೆಯ ಉಪದೇಶಗಳನ್ನು ಮನನಮಾಡಿ ಅವುಗಳನ್ನು ನಿಮ್ಮ ದೈನಂದಿನ ಜೀವನದಲ್ಲಿ ಅಳವಡಿಸಿಕೊಳ್ಳಲು ಸಹಾಯಮಾಡುವಂತೆ ರೂಪಿಸಲಾಗಿದೆ. ಗೀತೆಯ ಉಪನಿಷತ್ತೀಯ ಆತ್ಮದಾಳದಲ್ಲಿ ಬೇರೂರಿರುವ ಈ ದಿನಚರಿಯು, ಈಗಾಗಲೇ ಕೇಳಲಾದ (ಶ್ರವಣಂ) ಅಥವಾ ಅಧ್ಯಯನ ಮಾಡಿದ (ಅಧ್ಯಯನಂ) ವಿಷಯಗಳ ಮನನ — ಅಂದರೆ ‘ಚಿಂತನೆಮೂಲಕ ಗ್ರಹಿಕೆ’ — ಎಂಬ ಪರಂಪರಾತ್ಮಕ ಸಾಧನೆಯನ್ನು ಗೌರವಿಸುತ್ತದೆ.\\

ಈ ದಿನಚರಿಯು ನಿಮಗೆ ಗೀತೆಯ ಜ್ಞಾನವನ್ನು ಅನ್ವೇಷಿಸಲು, ನಿಮ್ಮ ಅರಿವನ್ನು ಗಾಢಗೊಳಿಸಲು ಮತ್ತು ಅದರ ಸನಾತನವಾದ ಅಂತರ್ದೃಷ್ಟಿಯನ್ನು, ನಿಮ್ಮ ವೈಯಕ್ತಿಕ ಹಾಗೂ ಆಧ್ಯಾತ್ಮಿಕ ಪಯಣದಲ್ಲಿ ಅನ್ವಯಿಸಲು ಆಹ್ವಾನಿಸುತ್ತದೆ. ಆಯ್ದ ಶ್ಲೋಕಗಳೊಡನೆ ನೀಡಿರುವ ಮನನಾತ್ಮಕ ಪ್ರಶ್ನೆಗಳು, ಉಪದೇಶಗಳೊಳಗಿನ ವೈಯಕ್ತಿಕ ಅರ್ಥವನ್ನು ಅನಾವರಣಗೊಳಿಸಲು ಹಾಗೂ ಜೀವನದ ಪರಿಸ್ಥಿತಿಗಳ ಮಧ್ಯೆ ಹೆಚ್ಚಿನ ಸ್ಪಷ್ಟತೆಯನ್ನು ಬೆಳೆಸಲು ವಿನ್ಯಾಸಗೊಳಿಸಲ್ಪಟ್ಟಿವೆ.\\

ಯಾವ ರೀತಿಯಲ್ಲಿ ಭಗವಾನ್ ಕೃಷ್ಣನು, ಅರ್ಜುನನನ್ನು ತನ್ನ ಧರ್ಮಯುದ್ಧದಲ್ಲಿ ತೊಡಗಿಸಿಕೊಳ್ಳಲು ಪ್ರೇರೇಪಿಸಿದನೋ, ಅದೇ ರೀತಿಯಾಗಿ ಈ ಕೃತಿಯ ಮೂಲಕ ಭಗವಂತನು ನಿಮ್ಮ ದೈನಂದಿನ ಕರ್ತವ್ಯಗಳನ್ನು ಧೈರ್ಯ ಮತ್ತು ವಿವೇಕದೊಂದಿಗೆ ಮುನ್ನಡೆಸಲು ನಿಮ್ಮನ್ನೂ ಪ್ರೇರೇಪಿಸಲಿ!  ನಿಮ್ಮ ಪ್ರಾಪಂಚಿಕ ಹೊಣೆಗಾರಿಕೆಗಳನ್ನು ಪೂರೈಸುತ್ತಿರುವ ಮಧ್ಯದಲ್ಲಿಯೂ, ನಿಶ್ಚಲವಾಗಿ  ಮನಸ್ಸನ್ನು ದೈವತ್ವದಲ್ಲಿಯೇ ನೆಲೆಗೊಳಿಸಿ,  ನಿಮ್ಮ ಅಂತರಂಗದ ಶಾಂತಿ ಮತ್ತು ವೈಯಕ್ತಿಕ ಪ್ರಗತಿಯನ್ನು ಪಡೆಯಲು ಸಹಾಯ ಮಾಡುವುದೇ ಶ್ರೀಮದ್ಭಗವದ್ಗೀತೆ – ಒಂದು ದಿವ್ಯಗಾನದ ನಿರಂತರ ಉದ್ದೇಶ!\\

ಈ ಪುಸ್ತಕದಲ್ಲಿ ಬಳಸಿದ ಕಲಾಕೃತಿಗಳಿಗಾಗಿ ನಾನು ಶ್ರೀಯುತರಾದ, ರಘುಪತಿ ಶೃಂಗೇರಿ ಮತ್ತು ಶೇಷಗಿರಿ ಕೆ. ಎಂ. ಅವರಿಗೆ ಹೃತ್ಪೂರ್ವಕ ಕೃತಜ್ಞತೆಯನ್ನು ವ್ಯಕ್ತಪಡಿಸುತ್ತೇನೆ. ನನ್ನ ಗೀತಾ ತರಗತಿಯ ಹಲವಾರು ವಿದ್ಯಾರ್ಥಿಗಳು ಈ ಕೃತಿಯ ತಿದ್ದುಪಡಿ, ಸಂಪಾದನೆ, ಅನುವಾದ, ವಿನ್ಯಾಸ ಮತ್ತು ಮುದ್ರಣ ಪ್ರಕ್ರಿಯೆಯ ಮೇಲ್ವಿಚಾರಣೆಯಲ್ಲಿ ತಮ್ಮನ್ನು ಸಂಪೂರ್ಣವಾಗಿ ಅರ್ಪಿಸಿಕೊಂಡಿದ್ದರು. ಎಲ್ಲ ಓದುಗರ ಹಿತಕ್ಕಾಗಿ ಅವರು ಸಲ್ಲಿಸಿದ ಈ ನಿಸ್ವಾರ್ಥ ಸೇವೆಗೆ ಭಗವಂತನ ಆಶೀರ್ವಾದ ದೊರೆಯಲೆಂದು ನಾನು ಪ್ರಾರ್ಥಿಸುತ್ತೇನೆ.\\\\}
{
{\normalsize\kanBold{ಸ್ವಾಮಿ ನಿರ್ಗುಣಾನಂದ ಗಿರಿ}}\\
{{ಋಷೀಕೇಶ – ಉತ್ತರಾಖಂಡ}}
}
