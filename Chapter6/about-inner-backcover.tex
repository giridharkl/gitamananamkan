ಸ್ವಾಮಿ ನಿರ್ಗುಣಾನಂದ ಗಿರಿ ಅವರು ಸಾಂಪ್ರದಾಯಿಕ ಹಿಂದೂ(ಸನಾತನ) ಸಂಪ್ರದಾಯದ ಸನ್ಯಾಸಿ(ಜ್ಞಾನಿಗಳು ಮತ್ತು ಸಾಧಕರು) ಆಗಿದ್ದಾರೆ. 
ಯೋಗ ಮತ್ತು ವೇದಾಂತದ ಸಂಪ್ರದಾಯದಲ್ಲಿ  ಬೇರೂರಿರುವ ಆಶ್ರಮಗಳಲ್ಲಿ ಅಧ್ಯಯನ, ತರಬೇತಿ ಮತ್ತು ಸೇವೆಗಾಗಿ ಅವರು ತಮ್ಮ ಜೀವನವನ್ನು ಮುಡಿಪಾಗಿಟ್ಟಿದ್ದಾರೆ. 
ನಿಪುಣ (ಧರ್ಮ ಶಾಸ್ತ್ರ ಮತ್ತು ವೇದಾಂತ ಪಾರಂಗತ)  ಆಚಾರ್ಯರ ಮಾರ್ಗದರ್ಶನದಲ್ಲಿ ಅವರು ಆಧ್ಯಾತ್ಮಿಕ ಅಭ್ಯಾಸಗಳಲ್ಲಿ ಮತ್ತು ವೇದ ಶಾಸ್ತ್ರಗಳ ಅಧ್ಯಯನದಲ್ಲಿ ತಮ್ಮನ್ನು ತಾವು ತೊಡಗಿಸಿಕೊಂಡರು.

ಪ್ರಸ್ತುತ, ಉತ್ತರಾಖಂಡದ ಋಷಿಕೇಶದಲ್ಲಿ ನೆಲೆಸಿರುವ ಸ್ವಾಮಿ ನಿರ್ಗುಣಾನಂದರು, ಹಿಮಾಲಯದ ಕೈಲಾಶ (ಕೈಲಾಸ) ಆಶ್ರಮ ಬ್ರಹ್ಮವಿದ್ಯಾ ಪೀಠದಿಂದ ಸಂನ್ಯಾಸ ದೀಕ್ಷೆಯನ್ನು ಪಡೆದರು, ಇದು ಭಾರತದ ಅನೇಕ ಪ್ರಸಿದ್ಧ ಆಚಾರ್ಯರು ಷಡ್-ದರ್ಶನಗಳನ್ನು (ಭಾರತೀಯ ತತ್ತ್ವಶಾಸ್ತ್ರದ ಆರು ಶಾಲೆಗಳು) ಕರಗತ ಮಾಡಿಕೊಂಡಿರುವ ಗೌರವಾನ್ವಿತ ಸಂಸ್ಥೆಯಾಗಿದೆ(ಆಶ್ರಮವಾಗಿದೆ).

ಬಹಳ (ಹಲವಾರು)ವರ್ಷಗಳ ಆಳವಾದ ಅಧ್ಯಯನ, ಚಿಂತನೆ ಮತ್ತು ಧ್ಯಾನದ ನಂತರ, ಸ್ವಾಮಿ ನಿರ್ಗುಣಾನಂದರು ಈಗ ಧರ್ಮಗ್ರಂಥದ ಜ್ಞಾನವನ್ನು ನೀಡುತ್ತಿದ್ದಾರೆ. 
ಅವರ ಭಗವದ್ಗೀತೆ ತರಗತಿಗಳು, ಅದರ ಸನಾತನ ಜ್ಞಾನವನ್ನು, ಜನಸಾಮಾನ್ಯರೂ ಕೂಡ ಪ್ರಾಯೋಗಿಕವಾಗಿ ತಮ್ಮ ಜೀವನಕ್ಕೆ ಅನ್ವಯಿಸಿ,  ಅಳವಡಿಸಿಕೊಳ್ಳುವುದರ  ಮೇಲೆ (ಅಳವಡಿಡಿಕೊಳ್ಳುವ ವಿಧಾನಗಳ ಬಗ್ಗೆ) ಕೇಂದ್ರೀಕರಿಸುತ್ತವೆ ಹಾಗೂ ಆಧ್ಯಾತ್ಮಿಕ ಬೆಳವಣಿಗೆಯ ಹಾದಿಯಲ್ಲಿ ಅನ್ವೇಷಕರನ್ನು ಪ್ರೇರೇಪಿಸುತ್ತವೆ.
