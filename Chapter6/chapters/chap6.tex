\slcol{ಶ್ರೀಭಗವಾನುವಾಚ ।\\
\Index{ಅನಾಶ್ರಿತಃ ಕರ್ಮಫಲಂ} ಕಾರ್ಯಂ ಕರ್ಮ ಕರೋತಿ ಯಃ ।\\
ಸ ಸಂನ್ಯಾಸೀ ಚ ಯೋಗೀ ಚ ನ ನಿರಗ್ನಿರ್ನ ಚಾಕ್ರಿಯಃ ॥ ೧ ॥}
\cquote{ಶ್ರೀ ಭಗವಂತನು ಹೇಳಿದನು,\\
ಕರ್ಮದ ಫಲಕ್ಕಾಗಿ ಹಂಬಲಿಸದೆ ಕರ್ತವ್ಯವೆಂದು ಕರ್ಮವನ್ನು ಮಾಡುವವನೇ ನಿಜವಾದ ಸನ್ಯಾಸಿ ಮತ್ತು ಕರ್ಮ ಯೋಗಿ ಹೊರೆತು ಸುಮ್ಮನೆ ಅಗ್ನಿಹೋತ್ರ ಮೊದಲಾದ ಕರ್ಮಗಳನ್ನು ಬಿಟ್ಟವನೂ ಯಾವ ಸತ್ಕರ್ಮಗಳನ್ನೂ ಮಾಡದವನೂ ಅಲ್ಲ.\\}
\slcol{\Index{ಯಂ ಸಂನ್ಯಾಸಮಿತಿ} ಪ್ರಾಹುರ್ಯೋಗಂ ತಂ ವಿದ್ಧಿ ಪಾಂಡವ ।\\
ನ ಹ್ಯಸಂನ್ಯಸ್ತಸಂಕಲ್ಪೋ ಯೋಗೀ ಭವತಿ ಕಶ್ಚನ ॥ ೨ ॥}
\cquote{ಅರ್ಜುನ, ಯಾವ ಸ್ಥಿತಿಯನ್ನು ಕರ್ಮ ಸನ್ಯಾಸವೆನ್ನುವರೋ ಅದನ್ನೇ ಕರ್ಮ ಯೋಗವೆಂದೂ ತಿಳಿ.ಏಕೆಂದರೆ ಫಲದ ಬಯಕೆಯ ಸನ್ಯಾಸವಿಲ್ಲದೆ ಕರ್ಮ ಮಾಡುವವನು ಕರ್ಮ ಯೋಗಿಯೂ ಆಗಲಾರ.\\}
\slcol{\Index{ಆರುರುಕ್ಷೋರ್ಮುನೇರ್ಯೋಗಂ} ಕರ್ಮ ಕಾರಣಮುಚ್ಯತೇ ।\\
ಯೋಗಾರೂಢಸ್ಯ ತಸ್ಯೈವ ಶಮಃ ಕಾರಣಮುಚ್ಯತೇ ॥ ೩ ॥}
\cquote{ಸಾಧನೆಯ ದಾರಿಯಲ್ಲಿ ಮೇಲೇರ ಬಯಸುವ ಸಾಧಕನಿಗೆ ಲೋಕಸೇವಾರೂಪವಾದ ಕರ್ಮವೂ ಉನ್ನತಿಯ ಸಾಧನ. ಸಿದ್ಧಿ ಪಡೆದು ಭಗವಂತನಲ್ಲಿ ನೆಲೆಗೊಂಡ ಜ್ಞಾನಿಗೆ ಲೌಕಿಕ ಕರ್ಮದ ನಿಯತಿಯಿಲ್ಲ.\\}
\slcol{\Index{ಯದಾ ಹಿ ನೇಂದ್ರಿಯಾರ್ಥೇಷು} ನ ಕರ್ಮಸ್ವನುಷಜ್ಜತೇ ।\\
ಸರ್ವಸಂಕಲ್ಪಸಂನ್ಯಾಸೀ ಯೋಗಾರೂಢಸ್ತದೋಚ್ಯತೇ ॥ ೪ ॥}
\cquote{ವಿಷಯಗಳಲ್ಲಿ ಕರ್ಮಗಳಲ್ಲಿ ಮಮತೆ ತೊರೆದು ಎಲ್ಲ ಕಾಮನೆಗಳನ್ನೂ ಮೀರಿ ನಿಂತವನು ಸಾಧನೆಯ ಮೆಟ್ಟಲೇರಿ ಸಿದ್ದಿ ಪಡೆದಂತವನು.\\}
\slcol{\Index{ಉದ್ಧರೇದಾತ್ಮನಾತ್ಮಾನಂ}
ನಾತ್ಮಾನಮವಸಾದಯೇತ್ ।\\
ಆತ್ಮೈವ ಹ್ಯಾತ್ಮನೋ ಬಂಧುರಾತ್ಮೈವ ರಿಪುರಾತ್ಮನಃ ॥ ೫ ॥}
\cquote{ತನ್ನನ್ನು ತನ್ನ ಮನೋಬಲದಿಂದಲೇ ಮೇಲಕ್ಕೆತ್ತಬೇಕು. ಪತನದ ದಾರಿಗೆ ತನ್ನನ್ನು ತಳ್ಳಬಾರದು. ಏಕೆಂದರೆ ನಮ್ಮ ಮನಸೇ ನಮಗೆ ನೆಂಟ ನಮ್ಮ ಮನಸೇ  ನಮಗೆ ಶತ್ರು.}

\newpage
\begin{mananam}{\mananamfont \large{ಮನನ ಶ್ಲೋಕ - ೧ ೨}}
\mananamtext ನಾನು ಏನನ್ನಾದರೂ ತ್ಯಜಿಸಲು ಅಥವಾ ಪಾತ್ರಕ್ಕೆ ರಾಜೀನಾಮೆ ನೀಡಲು ಪರಿಗಣಿಸಿದಾಗ ನನ್ನ ಪ್ರೇರಣೆ ಏನು? ಅಸ್ವಸ್ವತೆ   ಇಲ್ಲದ ಸುಲಭವಾದ ಜೀವನಕ್ಕಾಗಿ ಇದು ಬಯಕೆಯೇ? ಪರ್ಯಾಯವಾಗಿ ನಾನು ಉತ್ತಮವಾದ ಪ್ರತಿಫಲಗಳನ್ನು ಪಡೆಯುವ ನಿರೀಕ್ಷೆಯಿಂದ ಪ್ರೇರೇಪಿಸಲ್ಪಟ್ಟಿದ್ದೇನೆಯೇ? ನಾನು ಪವಿತ್ರವಾದ ಕರ್ತವ್ಯ ಭಾವನೆಯಿಂದ ಪ್ರಯತ್ನ ಮಾಡಲು ಅಥವಾ ಬೇಶರತ್ತಾಗಿ   ಸಹಾಯವನ್ನು ಮಾಡಲು ನನ್ನಲ್ಲೇ ನಾನು ಪ್ರೇರಣೆಯನ್ನು ಕಂಡುಕೊಳ್ಳಬಹುದೇ? ನಾನು ನಿಜವಾಗಿಯೂ ಅರ್ಹನಾದಾಗ ಉತ್ತಮ ಪ್ರತಿಫಲಗಳು ಸಿಗುತ್ತದೆ ಎಂಬ ಉನ್ನತ ನಂಬಿಕೆಯನ್ನು ಬೆಳೆಸಿಕೊಳ್ಳಲು ಸಾಧ್ಯವೇ?
\end{mananam}
\WritingHand\enspace\textbf{ಆತ್ಮ ವಿಮರ್ಶೆ}\\
\begin{inspiration}{\mananamfont \large ಸ್ಪೂರ್ತಿ}
ನಮ್ಮ ಪ್ರಾಚೀನ ಪರಿಕಲ್ಪನೆಯಲ್ಲಿ ಸನ್ಯಾಸಿ ಎಂದರೆ ಎಲ್ಲ ವೈದಿಕ ಆಚರಣೆಗಳನ್ನು ತ್ಯಜಿಸಿದವನು. ಅವರು ಈ ಆಚರಣೆಗಳನ್ನು ಏಕೆ ತ್ಯಜಿಸುತ್ತಾರೆ? ಏಕೆಂದರೆ ಸನ್ಯಾಸಿಯಾದ ಮೇಲೆ ಅವರು ಯಾವುದೇ ವಸ್ತು ಅಥವಾ ಪ್ರಪಂಚದಿಂದ ಹೆಚ್ಚಿನ ಪ್ರಯೋಜನಗಳನ್ನು ಬಯಸುವುದಿಲ್ಲ. ಜೀವನದ ಅತ್ಯುನ್ನತ ಕರೆಯಾದ  ಸನ್ಯಾಸದ ನಿಜವಾದ ಸಾರವೆಂದರೆ ನಮ್ಮ ಪ್ರಯತ್ನಗಳಿಂದ ಎಲ್ಲಾ ವೈಯಕ್ತಿಕ ನಿರೀಕ್ಷಗಳನ್ನು ತ್ಯಜಿಸುವುದು. ಆದರೆ ಕ್ರಿಯೆಗಳನ್ನಲ್ಲ  ಎಂದು ಭಗವಂತ ಸ್ಪಷ್ಟಪಡಿಸುತ್ತಾನೆ. ಕ್ರಿಯೆಗಳು ಸ್ವತಃ ಆಧ್ಯಾತ್ಮಿಕ ಪ್ರಗತಿಗೆ ಅಡ್ಡಿಯಾಗುವುದಿಲ್ಲ. ಆದರೆ ಅದರ ಫಲಿತಾಂಶಗಳ ಮೇಲಿನ ಗಮನ ಮತ್ತು ನಿರೀಕ್ಷೆಗಳು ಅಡೆತಡೆ ಉಂಟುಮಾಡುತ್ತದೆ.
\end{inspiration}
\newpage


\newpage
\begin{mananam}{\mananamfont \large{ಮನನ ಶ್ಲೋಕ - ೩ ೪}}
\mananamtext ಧ್ಯಾನಕ್ಕೆ ಕುಳಿತುಕೊಳ್ಳಲು ನನ್ನ ಮನಸ್ಸು ಸಿದ್ಧವಾಗಿದೆಯೇ? ಸಹಕರಿಸುತ್ತದೆಯೇ? ಕಳೆದು ಹೋದ ಅಥವಾ ಅತಿಯಾದ ಪ್ರಕ್ಷುಬ್ಧತೆಯ   ಭಾವನೆ ಇಲ್ಲದೆ ನಿತ್ಯವೂ ನಾನು ಎಷ್ಟು ಸಮಯವನ್ನು ಧ್ಯಾನಕ್ಕಾಗಿ ಮೀಸಲಿಡಬಹುದು? ಧ್ಯಾನ ಮಾಡಲು ಸಾಧ್ಯವಾಗದಿದ್ದರೆ ನಾನು ಆ ಸಮಯವನ್ನು ನಿಸ್ವಾರ್ಥವಾದ ಕೆಲಸಗಳನ್ನು ಮಾಡಲು ಬಳಸಬಹುದೇ? ದಿನನಿತ್ಯದ ಚಟುವಟಿಕೆಗಳೊಂದಿಗೆ ಧ್ಯಾನವನ್ನು ಹೇಗೆ ನಾನು ನಿಭಾಯಿಸಬಹುದು? ನನ್ನ ಧ್ಯಾನದ ಸಮಯವನ್ನು ನಾನು ರಾಜಿ ಮಾಡಿಕೊಳ್ಳದೆ, ನನ್ನ ಕಾರ್ಯವನ್ನು ಹೇಗೆ ನಿಭಾಯಿಸಬಹುದು?
\end{mananam}
\WritingHand\enspace\textbf{ಆತ್ಮ ವಿಮರ್ಶೆ}\\
\begin{inspiration}{\mananamfont \large ಸ್ಪೂರ್ತಿ}
ಒಬ್ಬ ನಿಜವಾದ ಆಧ್ಯಾತ್ಮಿಕ ಗುರು ಅಥವಾ ಅಧ್ಯಾತ್ಮಿಕ ಬೋಧನೆಯು ಪ್ರತಿಯೊಬ್ಬ ವಿದ್ಯಾರ್ಥಿಗೂ ಅವರವರ ಆಧ್ಯಾತ್ಮಿಕ ಪ್ರಯಾಣದ ಆಧಾರದ ಮೇಲೆ ಸೂಚನೆಗಳನ್ನು ಒದಗಿಸುತ್ತದೆ. ನಮ್ಮ ಮನಸ್ಸು ಶಾಂತವಾದಾಗ ಚಂಚಲತೆ ಮತ್ತು ಲೌಕಿಕ ಆಲೋಚನೆಗಳಿಂದ ಮುಕ್ತವಾದಾಗ ಧ್ಯಾನದ ನಿಶ್ಚಲತೆಯನ್ನು ಆನಂದಿಸಲು ಎಲ್ಲಾ ದೈಹಿಕ ಮತ್ತು ಮಾನಸಿಕ ಪ್ರಯತ್ನಗಳನ್ನು ನಿಲ್ಲಿಸಬೇಕು. ದೀರ್ಘಾವಧಿಯ ಧ್ಯಾನಕ್ಕಾಗಿ ಕುಳಿತುಕೊಳ್ಳಲು ಸಿದ್ಧವಾಗಿರದಿದ್ದಲ್ಲಿ ನಿಸ್ವಾರ್ಥವಾದ ಕ್ರಿಯೆಗಳಲ್ಲಿ ನಮ್ಮನ್ನು ನಾವು ತೊಡಗಿಸಿಕೊಳ್ಳಬೇಕು.
\end{inspiration}
\newpage


\slcol{\Index{ಬಂಧುರಾತ್ಮಾತ್ಮನಸ್ತಸ್ಯ} ಯೇನಾತ್ಮೈವಾತ್ಮನಾ ಜಿತಃ ।\\
ಅನಾತ್ಮನಸ್ತು ಶತ್ರುತ್ವೇ ವರ್ತೇತಾತ್ಮೈವ ಶತ್ರುವತ್ ॥ ೬ ॥}
\cquote{ಮನೋಬಲದಿಂದ ತನ್ನನ್ನು ತಾನೇ ಗೆದ್ದವನಿಗೆ ಅವನ ಮನಸ್ಸು ನೆಂಟ. ಯಾರು ಮನಸ್ಸನ್ನು ಗೆದೆಯಲಾರ ಅವನನ್ನು ಮನಸ್ಸೇ ಶತ್ರುವಾಗಿ ಕಾಡುತ್ತದೆ.\\}
\slcol{\Index{ಜಿತಾತ್ಮನಃ ಪ್ರಶಾಂತಸ್ಯ} ಪರಮಾತ್ಮಾ ಸಮಾಹಿತಃ ।\\
ಶೀತೋಷ್ಣಸುಖದುಃಖೇಷು ತಥಾ ಮಾನಾಪಮಾನಯೋಃ ॥ ೭ ॥}
\cquote{ಶೀತೋಷ್ಣ, ಸುಖದುಃಖ,ಮಾನ ಅಪಮಾನಗಳಲ್ಲಿ ಚಿತ್ತವನ್ನು ಜಯಿಸಿದ ಪ್ರಶಾಂತನಿಗೆ ಪರಮಾತ್ಮನ ಅನುಭವ ಉಂಟಾಗುವುದು.\\}
\slcol{\Index{ಜ್ಞಾನವಿಜ್ಞಾನತೃಪ್ತಾತ್ಮಾ} ಕೂಟಸ್ಥೋ ವಿಜಿತೇಂದ್ರಿಯಃ ।\\
ಯುಕ್ತ ಇತ್ಯುಚ್ಯತೇ ಯೋಗೀ ಸಮಲೋಷ್ಟಾಶ್ಮಕಾಂಚನಃ ॥ ೮ ॥}
\cquote{ಜ್ಞಾನ-ವಿಜ್ಞಾನದಿಂದ ತೃಪ್ತ ಅಂತ:ಕರಣವುಳ್ಳವನ ಸ್ಥಿತಿಯು ವಿಕಾರರಹಿತವಾಗಿರುತ್ತದೆ. ಇಂದ್ರಿಯಗಳನ್ನು ಪೂರ್ಣವಾಗಿ ಗೆದ್ದುಕೊಂಡು ಕಲ್ಲು, ಮಣ್ಣು ಮತ್ತು ಬಂಗಾರ ಸಮಾನವಾಗಿರುವ ಯೋಗಿಯು ಯುಕ್ತನು,ಅರ್ಥತ್ ಭಗವತ್ ಪ್ರಾಪ್ತನಾಗಿದ್ದಾನೆ ಎಂದು ಹೇಳಲಾಗುತ್ತದೆ.\\}
\slcol{\Index{ಸುಹೃನ್ಮಿತ್ರಾರ್ಯುದಾಸೀನಮಧ್ಯ}ಸ್ಥದ್ವೇಷ್ಯಬಂಧುಷು ।\\
ಸಾಧುಷ್ವಪಿ ಚ ಪಾಪೇಷು ಸಮಬುದ್ಧಿರ್ವಿಶಿಷ್ಯತೇ ॥ ೯ ॥}
\cquote{ಅಕಾರಣ ಬಂಧುಗಳು, ಆಪತ್ತಿಗೆ ಒದಗದ ಗೆಳೆಯರು, ಹಗೆಗಳು, ಯಾವ ಪಕ್ಷವನ್ನೂ ಬಯಸದವರು,ಎರಡು ಪಕ್ಷಗಳ ಹಿತವನ್ನು ಬಯಸುವವರು, ಅಪ್ರಿಯವಾದದ್ದನ್ನೇ ಮಾಡುವವರು, ನೆಂಟರು, ಸಜ್ಜನರು, ದುರ್ಜನರು-ಇವರೆಲ್ಲರನ್ನು ನಿರ್ವಿಕಾರ ಬುದ್ಧಿಯಿಂದ ಕಾಣುವವನು ಹೆಚ್ಚಿನವನು.\\}
\slcol{\Index{ಯೋಗೀ ಯುಂಜೀತ} ಸತತಮಾತ್ಮಾನಂ ರಹಸಿ ಸ್ಥಿತಃ ।\\
ಏಕಾಕೀ ಯತಚಿತ್ತಾತ್ಮಾ ನಿರಾಶೀರಪರಿಗ್ರಹಃ ॥ ೧೦ ॥}
\cquote{ಯೋಗಿಯಾದವನು ಏಕಾಂತದಲ್ಲಿ ಒಂಟಿಯಾಗಿದ್ದು ದೇಹವನ್ನು ಮನಸ್ಸನ್ನು ನಿಯಂತ್ರಿಸಿಕೊಂಡು ಏನನ್ನು ಬಯಸದೆ ಪರರ ಸ್ವತ್ತಿಗೆ ಕೈಯೊಡ್ಡದೆ ಯಾವಾಗಲೂ ಆತ್ಮವನ್ನು ಭಗವಂತನಲ್ಲಿಯೇ ನೆಲೆಗೊಳಿಸಬೇಕು.\\}
\slcol{\Index{ಶುಚೌ ದೇಶೇ ಪ್ರತಿಷ್ಠಾಪ್ಯ} ಸ್ಥಿರಮಾಸನಮಾತ್ಮನಃ ।\\
ನಾತ್ಯುಚ್ಛ್ರಿತಂ ನಾತಿನೀಚಂ ಚೈಲಾಜಿನಕುಶೋತ್ತರಮ್ ॥ ೧೧ ॥\\
ತತ್ರೈಕಾಗ್ರಂ ಮನಃ ಕೃತ್ವಾ ಯತಚಿತ್ತೇಂದ್ರಿಯಕ್ರಿಯಾಃ ।\\
ಉಪವಿಶ್ಯಾಸನೇ ಯುಂಜ್ಯಾದ್ಯೋಗಮಾತ್ಮವಿಶುದ್ಧಯೇ ॥ ೧೨ ॥}
\cquote{ಶುದ್ಧವಾದ ಕಡೆ ಹೆಚ್ಚು ಎತ್ತರವು ಹೆಚ್ಚು ತೆಗ್ಗು ಆಗದಂತೆ ದರ್ಬೆ, ಹುಲಿಯ ಅಥವಾ ಜಿಂಕೆಯ ಚರ್ಮ, ಬಟ್ಟೆ ಇವುಗಳನ್ನು ಕ್ರಮವಾಗಿ ಒಂದರ ಮೇಲೊಂದಾಗಿ ಹಾಸಿ ತನಗೆ ಭದ್ರವಾದ ಆಸನವನ್ನು ಸಿದ್ಧಪಡಿಸಿಕೊಂಡು ಆ ಪೀಠದ ಮೇಲೆ ಕುಳಿತು ಮನಸ್ಸಿನ ಮತ್ತು ಇಂದ್ರಿಯಗಳ ಚಲನವನ್ನು ತಡೆದು ಮನಸ್ಸನ್ನು ಒಮ್ಮುಕವಾಗಿ  ಮಾಡಿ ಮನಸ್ಸಿನ ಶುದ್ದಿಗಾಗಿ ಮನೊನಿಗ್ರಹವನ್ನು ಅಭ್ಯಾಸ ಮಾಡಬೇಕು.}

\newpage
\begin{mananam}{\mananamfont \large{ಮನನ ಶ್ಲೋಕ - ೫ ೬}}
\mananamtext ನನ್ನ ದಿನನಿತ್ಯದ ಜೀವನದಲ್ಲಿ ನನ್ನ ನಿಜವಾದ ಸ್ನೇಹ ಮತ್ತು ಮಾರ್ಗದರ್ಶಿಯಾಗಿರುವ ನನ್ನ ತ್ಮಸಾಕ್ಷಿಯಜೊತೆಗೆ ಹೊಂದಿಕೊಂಡಿದೆಯೇ? ನನ್ನ ಸಕರಾತ್ಮಕ ಉದ್ದೇಶಗಳು ಮತ್ತು ನಿರ್ಣಯಗಳಿಗೆ ನಾನು ಬದ್ಧನಾಗಿದ್ದೇನೆಯೇ? ಅಥವಾ ಮಾರ್ಗದರ್ಶನ ನೀಡುವ ನನ್ನ ಆಂತರಿಕ ಧ್ವನಿಯನ್ನು ನಾನು ವಿರೋಧಿಸುತ್ತೇನೆಯೇ?\\
ಒಳ್ಳೆಯ ಉದ್ದೇಶವುಳ್ಳ ಒಳ್ಳೆಯ ಜನರ ವಿರುದ್ಧ ನಾನು ಹದಿಹರೆಯದವರಂತೆ ವಿರೋಧಿಸುತ್ತೇನೆಯೇ? ಈ ತರಹ ಗುರುಗಳನ್ನು ಮತ್ತು ಧರ್ಮ ಗ್ರಂಥಗಳನ್ನು ವಿರೋಧಿಸುವುದು ನನ್ನ ಸ್ವಂತ ಅಧ್ಯಾತ್ಮಿಕ ಪ್ರಗತಿಗೆ ಅಡ್ಡಿಯಾಗುತ್ತದೆ ಎಂದು ನನಗೆ ಅರಿವಿದೆಯೇ? ನನ್ನ ಜೀವನದಲ್ಲಿ ಇರುವ ಅಹಂಕಾರದ ಶಕ್ತಿಯ ಬಗ್ಗೆ ನನಗೆ ಅರಿವಿದೆಯೇ?
\end{mananam}
\WritingHand\enspace\textbf{ಆತ್ಮ ವಿಮರ್ಶೆ}\\
\begin{inspiration}{\mananamfont \large ಸ್ಪೂರ್ತಿ}
ಶುದ್ಧ ಪ್ರಜ್ಞೆ ಮತ್ತು ಆನಂದವನ್ನು ಒಳಗೊಂಡಿರುವ ನಮ್ಮ ನಿಜವಾದ ಆತ್ಮಕ್ಕೆ ನಾವು ಹೊಂದಿಕೊಂಡಾಗ ನಮ್ಮ ಮನಸ್ಸು ಧನಾತ್ಮಕ ಉನ್ನತ ಸ್ಥಿತಿಯಲ್ಲಿರುತ್ತದೆ. ಅಂತಹ ಮನಸ್ಸು ನಮ್ಮ ಬುದ್ಧಿಯೊಂದಿಗೆ ಸಹಕರಿಸಿ ಒಟ್ಟಾರೆ ಯೋಗ ಕ್ಷೇಮವನ್ನು ಉತ್ತೇಜಿಸುತ್ತದೆ. ಅದಾಗಿಯೂ ಮನಸ್ಸು ಇಂದ್ರಿಯಗಳ ಪ್ರಭಾವಿತವಾದಾಗ ಅದು ಕ್ಷಣಿಕ ಲೌಕಿಕ ಸಂತೋಷ ಮತ್ತು ತೃಪ್ತಿಯನ್ನು ಅನುಸರಿಸುತ್ತದೆ. ಹಾಗೆ ಮಾಡಿದಾಗ ಅದು ಪ್ರಜ್ಞೆಗೆ ಹೊಂದಿಕೊಳ್ಳದೆ ಬುದ್ಧಿವಂತಿಕೆಯ ವಿರುದ್ಧ ಬಂಡಾಯವೆದ್ದು, ‘ಅಜ್ಞಾನವೇ ಆನಂದ’ ಎಂಬ ಗಾದೆಯಂತಾಗುತ್ತದೆ. ದುರದೃಷ್ಟವಶಾತ್ ಅಂತಹ ಜೀವನ ಮಾರ್ಗವು ದುಃಖದಿಂದ ತುಂಬಿದೆ. 
\end{inspiration}
\newpage

\newpage
\begin{mananam}{\mananamfont \large{ಮನನ ಶ್ಲೋಕ - ೮ ೯}}
\mananamtext ನಾನು ಎಲ್ಲದರಲ್ಲೂ ದೈವಿಕ ಸಾರವನ್ನು ಗ್ರಹಿಸುತ್ತೇನೆಯೇ? ಮಾನಸಿಕ ಒಲವು ಮತ್ತು ಹಿಂದಿನ ಅನುಭವಗಳ ಆಧಾರದ ಮೇಲೆ ವಸ್ತು ಮತ್ತು ಜನರನ್ನು ನೋಡುತ್ತೇನೆಯೇ? ನನ್ನ ನೆನಪುಗಳು ಇತರರೊಂದಿಗಿನ ನನ್ನ ಸಂವಹನವನ್ನು ಎಷ್ಟರಮಟ್ಟಿಗೆ   ಪ್ರಭಾವಿಸುತ್ತದೆ? ಜೀವನದ ಅನುಭವಗಳ ಕಡೆಗೆ ಹೊಸ ದೃಷ್ಟಿ ಕೋನವನ್ನು ಅಳವಡಿಸಿಕೊಳ್ಳುವುದರಿಂದ ಯಾವ ಪ್ರಯೋಜನಗಳಿವೆ? ಆ ಹೊಸ ದೃಷ್ಟಿಕೋನವನ್ನು ಕಾರ್ಯರೂಪಕ್ಕೆ ತರಲು ನನಗಿರುವ ಅಡೆತಡೆಗಳು ಯಾವುವು?
\end{mananam}
\WritingHand\enspace\textbf{ಆತ್ಮ ವಿಮರ್ಶೆ}\\
\begin{inspiration}{\mananamfont \large ಸ್ಪೂರ್ತಿ}
ಆತ್ಮಸಾಕ್ಷಾತ್ಕಾರವಾದ ಒಬ್ಬ ಯೋಗಿ ಎಲ್ಲರನ್ನೂ ಎಲ್ಲದರಲ್ಲಿಯೂ ಇರುವ ಆತ್ಮಪ್ರಜ್ಞೆಯನ್ನು ಗ್ರಹಿಸುತ್ತಾನೆ. ಅವರಿಗೆ ಪ್ರತಿಯೊಬ್ಬರೂ ದೈವಿಕ ಪ್ರಜ್ಞೆಯ ಭಾಗವಾಗಿ ಕಾಣಿಸುತ್ತಾರೆ. ಪರಿಣಾಮವಾಗಿ, ಅವರ ಎಲ್ಲ ಕೆಲಸಗಳು ಪೂರ್ವಗ್ರಹ ಮತ್ತು ಪಕ್ಷಪಾತಗಳಿಂದ ಮುಕ್ತವಾಗಿರುತ್ತದೆ. ಇದಕ್ಕೆ ವ್ಯತಿರಿಕ್ತವಾಗಿ ಪ್ರತಿಯೊಬ್ಬರಿಗೂ ಈ ಜಗತ್ತು ಒಂದು ಹೆಸರು, ವರ್ಗ, ಆಕಾರಗಳಿಂದ ಗುರುತಿಸಲ್ಪಟ್ಟಿದೆ.
\end{inspiration}
\newpage


\newpage
\begin{mananam}{\mananamfont \large{ಮನನ ಶ್ಲೋಕ - ೧೦}}
\mananamtext ನನ್ನ ದೈನಂದಿನ ಜೀವನದಲ್ಲಿ ನನ್ನೊಂದಿಗೆ ನಾನು ಇರಲು ಸಮಯವನ್ನು ಮೀಸಲಿಡುತ್ತೇನೆಯೇ? ಆಗಾಗ್ಗೆ  ಏಕಾಂತದಲ್ಲಿರುವುದು ನನಗೆ ಆರಾಮದಾಯಕವಾಗಿ ಅನ್ನಿಸುತ್ತದೆಯೇ? ಅಥವಾ ಅಸಮಾಧಾನ ಉಂಟುಮಾಡುತ್ತದೆಯೇ? ಏಕಾಂತವಾಗಿರುವಾಗ ನಾನು ಆತಂಕ, ಭಯ, ಬೇಸರ ಮತ್ತು ಕಳೆದುಹೋಗುವ ಭಾವನೆಯನ್ನು ಅನುಭವಿಸುತ್ತೇನೆಯೇ? ಏಕಾಂತದಲ್ಲಿರುವಾಗ ಸಂಗೀತ, ಟಿವಿ, ಅಥವಾ ಇತರ ಉಪಕರಣಗಳನ್ನು ಬಳಸುತ್ತೇನೆಯೇ? ಇದೆಲ್ಲವನ್ನು ಬೇರೆಯವರ ಜೊತೆ ಸಂವಹನೆಯ ಬದಲಿಗೆ ಬಳಸುತ್ತೇನಾ? ಒಂದು ಗಂಟೆಯ ಏಕಾಂತತೆ ಮತ್ತು ಚಿಂತನೆಯು ನನಗೆ ಯಾವ ಆಧ್ಯಾತ್ಮಿಕ ಮತ್ತು ಮಾನಸಿಕ ಪ್ರಯೋಜನಗಳನ್ನು ನೀಡುತ್ತದೆ?
\end{mananam}
\WritingHand\enspace\textbf{ಆತ್ಮ ವಿಮರ್ಶೆ}\\
\begin{inspiration}{\mananamfont \large ಸ್ಪೂರ್ತಿ}
\small \mananamtext ಪ್ರಾಚೀನ ಭಾರತದಲ್ಲಿದ್ದ ಜೀವನ ಶೈಲಿಯು ಈಗಿನ ಆಧುನಿಕ ಪ್ರಪಂಚದ ಜೀವನ ಶೈಲಿಗಿಂತ ಗಮನಾರ್ಹವಾಗಿ ಭಿನ್ನವಾಗಿದೆ. ತಂತ್ರಜ್ಞಾನ ಮತ್ತು ಸಾಮಾಜಿಕ  ಮಾಧ್ಯಮಗಳು ಮನುಷ್ಯರ ಏಕಾಂತದ ಅವಕಾಶವನ್ನು ಕಸಿದುಕೊಂಡಿದೆ. ಏಕಾಂತವು ಆತ್ಮವಲೋಚನೆ, ಆಂತರಿಕ ಶಾಂತಿ ಮತ್ತು ಮಾನಸಿಕ ಪ್ರಸನ್ನತೆಗೆ ಅವಕಾಶ ನೀಡುತ್ತದೆ. ಸೃಜನಶೀಲತೆ, ಚಿಂತನೆಯಲ್ಲಿ ಸ್ಪಷ್ಟತೆ ಮತ್ತು ನಮ್ಮ ಬಗ್ಗೆಯೇ ನಮಗೆ ಆಳವಾದ ತಿಳುವಳಿಕೆಯನ್ನು ಬೆಳೆಸುತ್ತದೆ. ಸಕರಾತ್ಮಕವಾಗಿ  ನಮ್ಮೊಂದಿಗೆ ನಾವು ಇರಲು ಕಲಿಯುವುದು ಒಂದು ಅಮೂಲ್ಯವಾದ ಕೌಶಲ್ಯವಾಗಿದೆ. ಈ ಕಲೆಯಲ್ಲಿ ಪ್ರವೀಣತೆಯನ್ನು ಪಡೆಯುವುದು ಉತ್ತಮವಾಗಿದೆ. ಈ ಕೌಶಲ್ಯವನ್ನು ನಿರಂತರವಾಗಿ ಅಭ್ಯಾಸ ಮಾಡುವುದು ವಿಜ್ಞಾನವೇ ಆಗಿದೆ. ನಮ್ಮ ಪ್ರಾಚೀನ ಋಷಿಗಳು ಈ ಕಲೆ ಮತ್ತು ವಿಜ್ಞಾನವನ್ನು ಧ್ಯಾನ ಎಂದು ಕರೆದರು.
\end{inspiration}
\newpage

\newpage
\begin{mananam}{\mananamfont \large{ಮನನ ಶ್ಲೋಕ - ೧೨}}
\mananamtext ಧ್ಯಾನ ಮತ್ತು ಅವಲೋಕನ ನಿರಂತರ ಅಭ್ಯಾಸ. ನನ್ನ ಜೀವನದಲ್ಲಿ ನಿತ್ಯವೂ ಬಿಡದೆ ಧ್ಯಾನದ ಅಭ್ಯಾಸವನ್ನು ಮಾಡುತ್ತೇನೆಯೇ? ಇಂದ್ರಿಯ ನಿಗ್ರಹ: ಧ್ಯಾನದ ಸಮಯದಲ್ಲಿ ಕಣ್ಣುಗಳನ್ನು ಮುಚ್ಚಿಕೊಂಡು ದೇಹದಲ್ಲಿ ಸ್ಥಿರತೆಯನ್ನು ಕಾಪಾಡಿಕೊಳ್ಳಬಹುದೇ? ಗಮನ ಮತ್ತು ಏಕಾಗ್ರತೆ: ನಾನು ಗಮನ ಮತ್ತು ಏಕಾಗ್ರತೆಯ ಸ್ಥಿತಿಯನ್ನು ಸಾಧಿಸಲು ಸಾಧ್ಯವೇ?  ಮನಸ್ಸಿನ ಸ್ಥಿತಿ: ನನ್ನ ಮನಸ್ಸು ಅಭಾಗ್ಯ, ಮಂದವಾಗಿ ಮತ್ತು ಚಂಚಲವಾಗಿದೆಯೇ? ಅರ್ಥ ಮಾಡಿಕೊಳ್ಳುವುದು ಮತ್ತು ಮಾರ್ಗದರ್ಶನ: ನಾನು ಧ್ಯಾನ ಪ್ರಕ್ರಿಯೆಯನ್ನು ಚೆನ್ನಾಗಿ ಅರ್ಥ ಮಾಡಿಕೊಂಡಿದ್ದೇನೆಯೇ? ಗುರುಗಳಿಂದ ಮಾರ್ಗದರ್ಶನ ಪಡೆಯಲು ನನ್ನ ಮನಸ್ಸು ತೆರೆದಿದೆಯೇ?
\end{mananam}
\WritingHand\enspace\textbf{ಆತ್ಮ ವಿಮರ್ಶೆ}\\
\begin{inspiration}{\mananamfont \large ಸ್ಪೂರ್ತಿ}
\mananamtext ಧ್ಯಾನ ಮನಸ್ಸಿನ ಆಂತರಿಕ ಶುದ್ಧೀಕರಣವೇ ಆಗಿದೆ.ದೇಹವನ್ನು ಸ್ನಾನದಿಂದ ಶುದ್ದಿ ಮಾಡಿದ ಹಾಗೆ ಮನಸ್ಸಿನ ಶುದ್ಧೀಕರಣಕ್ಕೆ ಧ್ಯಾನವೇ ಪ್ರಮುಖ ಅಭ್ಯಾಸವಾಗಿದೆ. ಶಾಂತಿ ಮತ್ತು ಆಂತರಿಕ ಸಂತೋಷದ  ಪ್ರಯೋಜನವನ್ನು ಪಡೆಯಲು ನಾವು ಔಪಚಾರಿಕ ಧ್ಯಾನದ ಪ್ರಕ್ರಿಯೆಯನ್ನು ಕರಗತ ಮಾಡಿಕೊಳ್ಳುವುದು ಅತ್ಯಗತ್ಯ.\\
ಭಗವದ್ಗೀತೆಯ ಪ್ರಕಾರ ಆದರ್ಶಯೋಗಿ ಎಂದರೆ ಪ್ರಕೃತಿಯನ್ನು ತಮ್ಮ ಇಚ್ಛೆಗೆ ತಕ್ಕಂತೆ ಬದಲಾಯಿಸುವ ಅಥವಾ ಪವಾಡಗಳನ್ನು ಮಾಡುವ ವ್ಯಕ್ತಿ ಅಲ್ಲ,ಬದಲಾಗಿ ಒಬ್ಬ ಯೋಗಿಯು ತನ್ನ ಇಂದ್ರಿಯ ಮತ್ತು ಮನಸ್ಸನ್ನು ನಿಯಂತ್ರಿಸುವ ಮೂಲಕ ಜಗತ್ತಿಗೆ ತನ್ನದೇ ಆದ ಸರಿಯಾದ ಪ್ರತಿಕ್ರಿಯೆಯನ್ನು ನೀಡಲು ಕಲಿಯುತ್ತಾರೆ.
\end{inspiration}
\newpage

\slcol{\Index{ಸಮಂ ಕಾಯಶಿರೋಗ್ರೀವಂ} ಧಾರಯನ್ನಚಲಂ ಸ್ಥಿರಃ ।\\
ಸಂಪ್ರೇಕ್ಷ್ಯ ನಾಸಿಕಾಗ್ರಂ ಸ್ವಂ ದಿಶಶ್ಚಾನವಲೋಕಯನ್ ॥ ೧೩ ॥\\
ಪ್ರಶಾಂತಾತ್ಮಾ ವಿಗತಭೀರ್ಬ್ರಹ್ಮಚಾರಿವ್ರತೇ ಸ್ಥಿತಃ ।\\
ಮನಃ ಸಂಯಮ್ಯ ಮಚ್ಚಿತ್ತೋ ಯುಕ್ತ ಆಸೀತ ಮತ್ಪರಃ ॥ ೧೪ ॥}
\cquote{ಮೈ, ತಲೆ, ಕೊರಳು - ಇವುಗಳು ಅಲ್ಲಾಡದಂತೆ ಸ್ಥಿರವಾಗಿ ನೆಟ್ಟಗೆ ಕುಳಿತು ಅತ್ತಿತ್ತ ದಿಕ್ಕುಗಳ ಕಡೆ ನೋಡದೆ ದೃಷ್ಟಿಯನ್ನು ಮೂಗಿನ ತುದಿಯಲ್ಲಿ ಕೇಂದ್ರೀಕರಿಸಿ ತಿಳಿ ಮನಸ್ಸಿನಿಂದ ನಿರ್ಭಯನಾಗಿ ಗುರು ಸೇವೆ ಮೊದಲಾದ ವ್ರತವನ್ನು ನಡೆಸುತ್ತಾ ಮನಸ್ಸನ್ನು ಬಿಗಿಹಿಡಿದು ನನ್ನಲ್ಲಿಟ್ಟು ನನ್ನನ್ನೇ ಧ್ಯಾನ ಮಾಡುತ್ತಾ ಇರಬೇಕು.\\}
\slcol{\Index{ಯುಂಜನ್ನೇವಂ ಸದಾತ್ಮಾನಂ} ಯೋಗೀ ನಿಯತಮಾನಸಃ ।\\
ಶಾಂತಿಂ ನಿರ್ವಾಣಪರಮಾಂ ಮತ್ಸಂಸ್ಥಾಮಧಿಗಚ್ಛತಿ ॥ ೧೫ ॥}
\cquote{ಹೀಗೆ ಯಾವಾಗಲೂ ಮನಸ್ಸನ್ನು ಬಿಗಿ ಹಿಡಿದು ತನ್ನನ್ನು ಧ್ಯಾನದಲ್ಲಿ ತೊಡಗಿಸಿದ ಯೋಗಿಯು ಈ ದೇಹ ತೊರೆದ ಮೇಲೆ ನನ್ನಲ್ಲಿ ನೆಲೆಯಾಗಿ ಸುಖವನ್ನು ಪಡೆಯುತ್ತಾನೆ.\\}
\slcol{\Index{ನಾತ್ಯಶ್ನತಸ್ತು ಯೋಗೋಽಸ್ತಿ} ನ ಚೈಕಾಂತಮನಶ್ನತಃ ।\\
ನ ಚಾತಿಸ್ವಪ್ನಶೀಲಸ್ಯ ಜಾಗ್ರತೋ ನೈವ ಚಾರ್ಜುನ ॥ ೧೬ ॥}
\cquote{ಅರ್ಜುನ, ಅಳತೆಗೆಟ್ಟು ತಿನ್ನುವವನಿಗೆ ಯೋಗದ ಮಾರ್ಗ ಮೈಗೂಡುವುದಿಲ್ಲ, ಎಷ್ಟು ಮಾತ್ರವೂ ಉಣ್ಣದೇ ಇರುವವನಿಗೂ ಇಲ್ಲ, ತುಂಬಾ ನಿದ್ದೆ ಮಾಡುವವನಿಗೂ ಇಲ್ಲ, ನಿದ್ದೆಗೇಡಿಗೂ ಇಲ್ಲ.\\}
\slcol{\Index{ಯುಕ್ತಾಹಾರವಿಹಾರಸ್ಯ} ಯುಕ್ತಚೇಷ್ಟಸ್ಯ ಕರ್ಮಸು ।\\
ಯುಕ್ತಸ್ವಪ್ನಾವಬೋಧಸ್ಯ ಯೋಗೋ ಭವತಿ ದುಃಖಹಾ ॥ ೧೭ ॥}
\cquote{ಮಿತವಾದ ಊಟ, ಮಿತವಾದ ಓಡಾಟ, ಮಿತವಾದ ಕೆಲಸ, ಮಿತವಾದ ನಿದ್ದೆ, ಮಿತವಾದ ಎಚ್ಚರ ಇರುವವರಿಗೆ ಧ್ಯಾನವು ಅಡಚಣೆಗಳನ್ನೆಲ್ಲ ದಾಟಬಲ್ಲದ್ದಾಗುತ್ತದೆ.\\}
\slcol{\Index{ಯದಾ ವಿನಿಯತಂ ಚಿತ್ತಮಾ}ತ್ಮನ್ಯೇವಾವತಿಷ್ಠತೇ ।\\
ನಿಃಸ್ಪೃಹಃ ಸರ್ವಕಾಮೇಭ್ಯೋ ಯುಕ್ತ ಇತ್ಯುಚ್ಯತೇ ತದಾ ॥ ೧೮ ॥}
\cquote{ಯಾವಾಗ ಒಮ್ಮುಖವಾದ ಮನಸ್ಸು ಭಗವಂತನಲ್ಲಿಯೇ ಇರುವುದೋ ಆಗ ಅವನು ಯಾವ ಬಯಕೆಯೂ ಇಲ್ಲದವನಾಗಿ ಧ್ಯಾನನಿಷ್ಟನೆನಿಸುವನು.\\}
\slcol{\Index{ಯಥಾ ದೀಪೋ ನಿವಾತಸ್ಥೋ} ನೇಂಗತೇ ಸೋಪಮಾ ಸ್ಮೃತಾ ।\\
ಯೋಗಿನೋ ಯತಚಿತ್ತಸ್ಯ ಯುಂಜತೋ ಯೋಗಮಾತ್ಮನಃ ॥ ೧೯ ॥}
\cquote{ಗಾಳಿ ಇಲ್ಲದ ಕಡೆಯ ದೀಪವು ಅಲ್ಲಾಡದೆ ಇರುವ ಸ್ಥಿತಿಯು ಧ್ಯಾನವನ್ನು ಅಭ್ಯಾಸ ಮಾಡುವ ಮನೋನಿಗ್ರಹವುಳ್ಳ ಸಾಧಕನಿಗೆ ಹೋಲಿಕೆ ಎಂದು ದೊಡ್ಡವರು ಹೇಳಿರುವರು.\\}


\newpage
\begin{mananam}{\mananamfont \large{ಮನನ ಶ್ಲೋಕ - ೧೭}}
\mananamtext ಸಮತೋಲಿತ ಮಾರ್ಗ ಆಹಾರ ಸಮತೋಲನ:
ನನಗೆ ಸಮತೋಲಿತ ಆಹಾರ ಕ್ರಮ ಯಾವುದು? ನಾನು ಕೆಲವೊಮ್ಮೆ ಅತಿಯಾಗಿ ಆಹಾರವನ್ನು ಸೇವಿಸುತ್ತೇನೆ ಮತ್ತು ನಂತರ ಅತಿಯಾದ ಉಪವಾಸದಿಂದ ಸರಿದೂಗಿಸಿಕೊಳ್ಳುತ್ತೇನೆಯೇ?\\
ಚಟುವಟಿಕೆಯ ಮಟ್ಟಗಳು: ದೈಹಿಕ ಚಟುವಟಿಕೆಯ ವಿಷಯದಲ್ಲಿ ನಾನು  ಸಮತೋಲದಲ್ಲಿದ್ದೇನೆಯೇ? ಸೋಮಾರಿ ಯಾಗಿರುವುದು ಎಲ್ಲಾ ಕೆಲಸಗಳನ್ನು ಮುಂದೂಡುವುದು,  ಪ್ರಕ್ಷುಬ್ಧತೆ ಮತ್ತು ಅತಿಯಾದ ಚಟುವಟಿಕೆಯ ನಡುವೆ ಓಡಾಡುತ್ತಿದ್ದೇನೆಯೇ? ನಿದ್ರೆಯ ಮಾದರಿಗಳು,
ನನ್ನ ನಿದ್ರೆಯ ಮಾದರಿಗಳು ಎಷ್ಟು ಸ್ಥಿರವಾಗಿದೆ? ನಾನು ಕೆಲವೊಮ್ಮೆ ಅತಿಯಾಗಿ ನಿದ್ರಿಸುತ್ತೇನೆಯೇ? ಮತ್ತು ಕೆಲವೊಮ್ಮೆ ಸಾಕಷ್ಟು ವಿಶ್ರಾಂತಿಯನ್ನು ತೆಗೆದುಕೊಳ್ಳುವುದಿಲ್ಲವೇ?
\end{mananam}
\WritingHand\enspace\textbf{ಆತ್ಮ ವಿಮರ್ಶೆ}\\
\begin{inspiration}{\mananamfont \large ಸ್ಪೂರ್ತಿ}
\mananamtext ಅಧ್ಯಾತ್ಮಿಕ ಹಾದಿಯಲ್ಲಿರುವವರು ದೀರ್ಘಾವಧಿಯ ಯಶಸ್ಸಿಗಾಗಿ ಜೀವನವನ್ನು  ಸಮತೋಲನವಾಗಿ ನಡೆಸಬೇಕು. ಅತಿ ಕಡಿಮೆ ಆಹಾರ ಸೇವನೆ, ಅತಿ ಕಡಿಮೆ  ನಿದ್ರೆಯಿಂದ ದೇಹವನ್ನು ದುರ್ಬಲಗೊಳಿಸದೇ ಇರುವುದು ಅತ್ಯಂತ ಮುಖ್ಯವಾಗಿದೆ. ಅಧ್ಯಾತ್ಮಿಕ ಸಾಧನೆಗೆ ಒಂದು ಆರೋಗ್ಯಕರ ದೇಹವು ಅತ್ಯುತ್ತಮ ಉಪಕರಣವಾಗಿದೆ. ಭಗವದ್ಗೀತೆಯು ನಮಗೆ ಅತಿಯಾಗಿರುವುದನ್ನು ಬಿಟ್ಟು ಮಧ್ಯಮ ಮಾರ್ಗವನ್ನು ಅಳವಡಿಸಿಕೊಳ್ಳಲು ಹೇಳುತ್ತದೆ.
\end{inspiration}
\newpage

\slcol{\Index{ಯತ್ರೋಪರಮತೇ ಚಿತ್ತಂ} ನಿರುದ್ಧಂ ಯೋಗಸೇವಯಾ ।\\
ಯತ್ರ ಚೈವಾತ್ಮನಾತ್ಮಾನಂ ಪಶ್ಯನ್ನಾತ್ಮನಿ ತುಷ್ಯತಿ ॥ ೨೦ ॥}
\cquote{ಯೋಗಾಭ್ಯಾಸದಿಂದ ವಶಪಡಿಸಿಕೊಂಡಿರುವ ಚಿತ್ತವು ಯಾವ ಅವಸ್ಥೆಯಲ್ಲಿ ಉಪರಥಿ ಉಂಟಾಗುತ್ತದೆ ಮತ್ತು ಯಾವ ಅವಸ್ಥೆಯಲ್ಲಿ ಪರಮಾತ್ಮನ ಧ್ಯಾನದಿಂದ ಶುದ್ಧವಾದ, ಸೂಕ್ಷ್ಮ ಬುದ್ಧಿಯ ಮೂಲಕ ಪರಮಾತ್ಮನನ್ನು ಸಾಕ್ಷಾತ್ಕರಿಸಿಕೊಳ್ಳುತ್ತಾ ಸಚ್ಚಿದಾನಂದಗನನಾದ ಪರಮಾತ್ಮನಲ್ಲಿಯೇ ಸಂತುಷ್ಟನಾಗಿರುತ್ತಾನೋ -\\}
\slcol{\Index{ಸುಖಮಾತ್ಯಂತಿಕಂ ಯತ್ತದ್ಬುದ್ಧಿ}ಗ್ರಾಹ್ಯಮತೀಂದ್ರಿಯಮ್ ।\\
ವೇತ್ತಿ ಯತ್ರ ನ ಚೈವಾಯಂ ಸ್ಥಿತಶ್ಚಲತಿ ತತ್ತ್ವತಃ ॥ ೨೧ ॥}
\cquote{ಇಂದ್ರಿಯಗಳಿಗೆ ನಿಲುಕದೆ ಬುದ್ಧಿಗೆ ಮಾತ್ರ ನಿಲುಕುವಂಥ ಪರಮ ಸುಖವನ್ನು ಅನುಭವಿಸುತ್ತಾನೋ, ಯಾವ ಸ್ಥಿತಿಯಲ್ಲಿ ನೆಲೆಗೊಂಡ ಯೋಗಿಯು ಅದರಿಂದ ವಿಚಲಿತನಾಗಲಾರನೋ,\\}
\slcol{\Index{ಯಂ ಲಬ್ಧ್ವಾ ಚಾಪರಂ} ಲಾಭಂ ಮನ್ಯತೇ ನಾಧಿಕಂ ತತಃ ।\\
ಯಸ್ಮಿನ್ಸ್ಥಿತೋ ನ ದುಃಖೇನ ಗುರುಣಾಪಿ ವಿಚಾಲ್ಯತೇ ॥ ೨೨ ॥}
\cquote{ಯಾವ ಸ್ಥಿತಿಯನ್ನು ಪಡೆದ ಮೇಲೆ ಬೇರೆ ಲಾಭವನ್ನು ಅದಕ್ಕಿಂತ ಹೆಚ್ಚಿನದೆಂದು ತಿಳಿಯುದಿಲ್ಲವೋ, ಯಾವ ಸ್ಥಿತಿಯಲ್ಲಿರುವವನು ಎಂಥ ದೊಡ್ಡ ವಿಪತ್ತಿನಿಂದಲೂ ಕಂಗಡುವುದಿಲ್ಲವೋ,\\}
\slcol{\Index{ತಂ ವಿದ್ಯಾದ್ದುಃಖಸಂಯೋಗ}ವಿಯೋಗಂ ಯೋಗಸಂಙ್ಞಿತಮ್ ।\\
ಸ ನಿಶ್ಚಯೇನ ಯೋಕ್ತವ್ಯೋ ಯೋಗೋಽನಿರ್ವಿಣ್ಣಚೇತಸಾ ॥ ೨೩ ॥}
\cquote{ದುಃಖದ ಸೋಂಕನ್ನು ದೂರಮಾಡುವ ಆ ಸ್ಥಿತಿಗೆ ಯೋಗವೆಂಬ ಹೆಸರೆಂದು ತಿಳಿ. ಆ ಯೋಗವನ್ನು ನಿರಾಶನಾಗದೆ ಬರವಸೆ ತಾಳಿ ಅಭ್ಯಾಸ ಮಾಡತಕ್ಕದ್ದು.\\}
\slcol{\Index{ಸಂಕಲ್ಪಪ್ರಭವಾನ್ಕಾಮಾಂ}ಸ್ತ್ಯಕ್ತ್ವಾ ಸರ್ವಾನಶೇಷತಃ ।\\
ಮನಸೈವೇಂದ್ರಿಯಗ್ರಾಮಂ ವಿನಿಯಮ್ಯ ಸಮಂತತಃ ॥ ೨೪ ॥\\
\Index{ಶನೈಃ ಶನೈರುಪರಮೇದ್ಬುದ್ಧ್ಯಾ} ಧೃತಿಗೃಹೀತಯಾ ।\\
ಆತ್ಮಸಂಸ್ಥಂ ಮನಃ ಕೃತ್ವಾ ನ ಕಿಂಚಿದಪಿ ಚಿಂತಯೇತ್ ॥ ೨೫ ॥}
\cquote{ಮನಸ್ಸಿನ ಗುಣಾಕಾರದಿಂದ ಹುಟ್ಟುವ ಬಯಕೆಗಳನ್ನೆಲ್ಲ ಸಂಪೂರ್ಣವಾಗಿ ಬಿಟ್ಟು ಮನಸ್ಸಿನಿಂದಲೇ ಇಂದ್ರಿಯಗಳನ್ನು ಎಲ್ಲ ಕಡೆಯಿಂದಲೂ ನಿಯಂತ್ರಿಸಿ ಧೈರ್ಯ ಪಡೆದ ಬುದ್ಧಿಬಲದಿಂದ ಮೆಲ್ಲಮೆಲ್ಲನೆ ವಿಷಯಗಳಿಂದ ನಿವೃತ್ತನಾಗಬೇಕು. ಮನಸ್ಸನ್ನು ಆತ್ಮನಲ್ಲಿ ನೆಲಗೊಳಿಸಿ ಬೇರೆ ಏನನ್ನು ಚಿಂತಿಸಬಾರದು.\\}
\slcol{\Index{ಯತೋ ಯತೋ ನಿಶ್ಚರತಿ} ಮನಶ್ಚಂಚಲಮಸ್ಥಿರಮ್ ।\\
ತತಸ್ತತೋ ನಿಯಮ್ಯೈತದಾತ್ಮನ್ಯೇವ ವಶಂ ನಯೇತ್ ॥ ೨೬ ॥}
\cquote{ಒಂದು ಕಡೆ ನಿಲ್ಲದೆ ಹರಿದಾಡುವ ಮನಸ್ಸು ಯಾವ ಯಾವ ವಿಷಯಗಳ ಬೆನ್ನು ಹಿಡಿದು ಹೊರಕ್ಕೆ ಹೊರಡುತ್ತದೋ ಆಯಾ ವಿಷಗಳಿಂದ ಅದನ್ನು ಹಿಮ್ಮೆಟ್ಟಿಸಿ ಪರಮಾತ್ಮನಲ್ಲಿಯೇ ನೆಲೆಗೊಳಿಸಬೇಕು.\\}

\newpage
\begin{mananam}{\mananamfont \large{ಮನನ ಶ್ಲೋಕ - ೨೨ ೨೩}}
\mananamtext ಯೋಗ ಮತ್ತು ನನಗೆ ಯೋಗ ಕ್ಷೇಮ ಉಂಟು ಮಾಡುವ ಕೆಲವು ಅಭ್ಯಾಸಗಳನ್ನು ನಾನು ಯಾವ ಕಾರಣಕ್ಕಾಗಿ ಮಾಡುತ್ತಿದ್ದೇನೆ. ಇದನ್ನೆಲ್ಲಾ ನಾನು ಜೀವನೋಪಾಯ, ಸಾಮಾಜಿಕ ಸ್ಥಾನಮಾನ, ಮನ್ನಣೆ ಮತ್ತು ಹೆಮ್ಮೆ ಇವುಗಳಿಗೋಸ್ಕರವೇ? ಅಥವಾ ದೈಹಿಕ ಮಾನಸಿಕ ಮತ್ತು ಭಾವನಾತ್ಮಕವಾದ ನೋವು ದುಃಖಗಳಿಂದ ಪರಾಗುವುದಕೋಸ್ಕರವೇ? ನಾನು ಇತರರನ್ನು ಮೆಚ್ಚಿಸಲು ಅಥವಾ ಯೋಗ ಕ್ಷೇಮಕ್ಕಾಗಿ ನಾನು ಇದನ್ನು ಮಾಡುತ್ತಿದ್ದೇನೆಯೇ? ಈ ಮಾರ್ಗವು ನನಗೆ ಉನ್ನತ ಮಟ್ಟದ ಯೋಗಕ್ಷೇಮಕ್ಕೆ ಕಾರಣವಾಗುತ್ತದೆ ಎಂದು ನನಗೆ ಮನವರಿಕೆಯಾಗಿದೆಯೇ?
\end{mananam}
\WritingHand\enspace\textbf{ಆತ್ಮ ವಿಮರ್ಶೆ}\\
\begin{inspiration}{\mananamfont \large ಸ್ಪೂರ್ತಿ}
\mananamtext ಎಲ್ಲ ಮಾನಸಿಕ ಕಷ್ಟಗಳ ಅಂತ್ಯ ಮತ್ತು ಅಂತರ್ಗತವಾದ ಆನಂದವನ್ನು ಅನುಭವಿಸುವುದು ನಮ್ಮ ಆಧ್ಯಾತ್ಮಿಕ ಅಭ್ಯಾಸ ಮತ್ತು ಅಧ್ಯಯನದ ಗುರಿಯಾಗಿದೆ. ಈ ಉದ್ದೇಶದ ಸ್ಪಷ್ಟ ನಿರ್ಣಯವಿಲ್ಲದೆ ಪ್ರಾಪಂಚಿಕ ಲಾಭಕ್ಕಾಗಿ ಈ ಬೋಧನೆಗಳನ್ನು ಅಭ್ಯಾಸ ಮಾಡುವುದು ಮತ್ತು ಅಧ್ಯಯನ ಮಾಡುವುದು ಅಂತಿಮವಾಗಿ ಅತೃಪ್ತಿ ಮತ್ತು ನಿರಾಶೆಗೆ ಕಾರಣವಾಗುತ್ತದೆ.ಯಾವಗ ನಮಗೆ ಸರಿಯಾದ ತಿಳುವಳಿಕೆ ಮತ್ತು ಅಭ್ಯಾಸದಲ್ಲಿ ದೃಢತೆ ಇದ್ದಲ್ಲಿ ಎಲ್ಲಾ ದುಃಖಗಳು ಕೊನೆಗೊಳ್ಳುತ್ತದೆ ಇದು ಯೋಗ ನೀಡುವ ಭರವಸೆ.
\end{inspiration}
\newpage

\newpage
\begin{mananam}{\mananamfont \large{ಮನನ ಶ್ಲೋಕ - ೨೫ ೨೬}}
\mananamtext ಧ್ಯಾನದ  ಅಭ್ಯಾಸ ಮಾಡುವಾಗ ಯಾವುದರ ಮೇಲೆ ನನ್ನ ಮನಸ್ಸನ್ನು ಕೇಂದ್ರೀಕರೀಸಬೇಕು ಎಂದು ಸ್ಪಷ್ಟವಾಗಿ ನಿರ್ಧರಿಸಿದ್ದೇನೆಯೇ? ಗೊಂದಲಗಳು ಉಂಟಾದಾಗಲೂ ನನಗೆ ನನ್ನ ಮನಸ್ಸಿನ ಬಗ್ಗೆ ಅರಿವು ಸತತವಾಗಿ ಇದೆಯೇ? ನಾನು ಮನಸ್ಸಿಗೆ ಪುನಃ ಧ್ಯಾನದ ವಸ್ತುವಿನ ಮೇಲೆ ಮನಸ್ಸನ್ನು ಕೇಂದ್ರೀಕರಿಸಲು ಹೇಳುತ್ತೇನೆಯೇ?\\
ಧ್ಯಾನ ಮಾಡುವಾಗ ಮನಸ್ಸು ಚಂಚಲವಾದಾಗ ನಾನು ತಾಳ್ಮೆಯಿಂದ ಇರುತ್ತೇನೆಯೇ? ಅಥವಾ ನಾನು ಬೇಗನೆ ಬಿಟ್ಟು ಬಿಡುತ್ತೀನಾ? ಧ್ಯಾನದಿಂದ ಉಂಟಾಗುವ ತಕ್ಷಣದ ಪ್ರಯೋಜನಗಳ ಬಗ್ಗೆ ನನ್ನ ಗಮನವಿದೆಯೇ? ಧ್ಯಾನದ  ನಂತರ ಮನಸ್ಸು ಶಾಂತವಾಗುತ್ತದೆಯೇ? ಅಥವಾ ಅದು ಸ್ವಲ್ಪ ಸಮಯದ ನಂತರ ಮನಸ್ಸು ಶಾಂತವಾಗುತ್ತದೆಯೇ?
\end{mananam}
\WritingHand\enspace\textbf{ಆತ್ಮ ವಿಮರ್ಶೆ}\\
\begin{inspiration}{\mananamfont \large ಸ್ಪೂರ್ತಿ}
\mananamtext ಎಲ್ಲಾ ಧ್ಯಾನದ  ತಂತ್ರಗಳು ಸರಳವಾದ ಸೂತ್ರಕ್ಕೆ ಮನ್ನಣೆ ಕೊಡುತ್ತದೆ. ಮನಸ್ಸು ವಿಚಲಿತವಾದಾಗ ನಮ್ಮ ಗಮನವನ್ನು ಪುನಃ ಕೇಂದ್ರೀಕರಿಸಲು ನೆನಪಿಸಬೇಕು. ಈ ಅಭ್ಯಾಸವು ನಮ್ಮ ಪ್ರಾಚೀನರ ಬುದ್ಧಿವಂತಿಕೆಯನ್ನು ಮತ್ತು ಆಧುನಿಕ ವಿಜ್ಞಾನ ಗಮನಿಸುವುದು, ನಿಯಂತ್ರಿಸುವುದು, ಈ ಎರಡನ್ನು  ಹೊಂದಿರುತ್ತದೆ. ಈ ಅಭ್ಯಾಸವನ್ನು ಮಾಡಲು ತಾಳ್ಮೆ ಮತ್ತು ದೃಢವಾಗಿರುವುದು ಯಶಸ್ಸಿಗೆ  ಕಾರಣವಾಗುತ್ತದೆ.
\end{inspiration}
\newpage

\slcol{\Index{ಪ್ರಶಾಂತಮನಸಂ ಹ್ಯೇನಂ} ಯೋಗಿನಂ ಸುಖಮುತ್ತಮಮ್ ।\\
ಉಪೈತಿ ಶಾಂತರಜಸಂ ಬ್ರಹ್ಮಭೂತಮಕಲ್ಮಷಮ್ ॥ ೨೭ ॥}
\cquote{ಶಾಂತವಾದ ಮನಸ್ಸುಳ್ಳ, ರಜೋಗುಣದ ದೋಷಗಳನ್ನು ದೂರಿಕರಿಸಿಕೊಂಡ, ಪಾಪಗಳನ್ನು ಕಳೆದುಕೊಂಡು ಬ್ರಹ್ಮನಲ್ಲಿ ನೆಲೆಗೊಂಡ ಇಂತ ಯೋಗಿ ಉತ್ತಮವಾದ ಸುಖವನ್ನು ಪಡೆಯುತ್ತಾನೆ.\\}
\slcol{\Index{ಯುಂಜನ್ನೇವಂ ಸದಾತ್ಮಾನಂ} ಯೋಗೀ ವಿಗತಕಲ್ಮಷಃ ।\\
ಸುಖೇನ ಬ್ರಹ್ಮಸಂಸ್ಪರ್ಶಮತ್ಯಂತಂ ಸುಖಮಶ್ನುತೇ ॥ ೨೮ ॥}
\cquote{ಪಾಪಗಳನ್ನೆಲ್ಲ ಕಳೆದುಕೊಂಡ ಯೋಗಿಯೂ ಹೀಗೆಯೇ ಯಾವಾಗಲೂ ಮನಸ್ಸನ್ನು ಭಗವಂತನಲ್ಲಿ ನೆಲೆಗೊಳಿಸಿದಾಗ ಅನಾಯಾಸವಾಗಿ ಬ್ರಹ್ಮನೊಡನೆ ಕೂಡುವುದೆಂಬ ಶಾಶ್ವತವಾದ ಸುಖವನ್ನು ಅನುಭವಿಸುವನು.\\}
\slcol{\Index{ಸರ್ವಭೂತಸ್ಥಮಾತ್ಮಾನಂ} ಸರ್ವಭೂತಾನಿ ಚಾತ್ಮನಿ ।\\
ಈಕ್ಷತೇ ಯೋಗಯುಕ್ತಾತ್ಮಾ ಸರ್ವತ್ರ ಸಮದರ್ಶನಃ ॥ ೨೯ ॥}
\cquote{ಈ ಬಗೆಯ ಯೋಗದಿಂದ ಮಾಗಿದ ಮನಸುಳ್ಳವನು ಎಲ್ಲ ವಿಷಯಗಳನ್ನೂ ಒಂದೇ ರೀತಿಯಲ್ಲಿ ಕಾಣುತ್ತಾ,ಭಗವಂತನು ಎಲ್ಲ ಪ್ರಾಣಿಗಳಲ್ಲಿ ತುಂಬಿರುವಂತೆಯೂ, ಎಲ್ಲ ಪ್ರಾಣಿಗಳು ಭಗವಂತನನ್ನು ಆಶ್ರಯಿಸಿಕೊಂಡಿರುವಂತೆಯೂ ಕಾಣುತ್ತಾನೆ.\\}
\slcol{\Index{ಯೋ ಮಾಂ ಪಶ್ಯತಿ ಸರ್ವತ್ರ} ಸರ್ವಂ ಚ ಮಯಿ ಪಶ್ಯತಿ ।\\
ತಸ್ಯಾಹಂ ನ ಪ್ರಣಶ್ಯಾಮಿ ಸ ಚ ಮೇ ನ ಪ್ರಣಶ್ಯತಿ ॥ ೩೦ ॥}
\cquote{ಯಾವನು ಹೀಗೆ ನನ್ನನ್ನು ಎಲ್ಲಡೆಯೂ ಕಾಣುತ್ತಾನೋ ಮತ್ತು ಎಲ್ಲವನ್ನೂ ನನ್ನಲ್ಲಿ ಕಾಣುತ್ತಾನೋ ಅವನನ್ನು ನಾನು ಕೈ ಬಿಡಲಾರೆ.ಅವನು ನನ್ನನ್ನು ಮರೆಯಲಾರ.\\}
\slcol{\Index{ಸರ್ವಭೂತಸ್ಥಿತಂ ಯೋ ಮಾಂ} ಭಜತ್ಯೇಕತ್ವಮಾಸ್ಥಿತಃ ।\\
ಸರ್ವಥಾ ವರ್ತಮಾನೋಽಪಿ ಸ ಯೋಗೀ ಮಯಿ ವರ್ತತೇ ॥ ೩೧ ॥}
\cquote{ಎಲ್ಲೆಡೆಯೂ ಇರುವ ಭಗವಂತನೊಬ್ಬನೇ ಎಂದು ತಿಳಿದು ನನ್ನನ್ನು ಭಜಿಸುವ ಜ್ಞಾನಿಯು ಹೇಗಿದ್ದರೂ ನನ್ನಲ್ಲಿರುತ್ತಾನೆ.\\}
\slcol{ಆತ್ಮೌಪಮ್ಯೇನ ಸರ್ವತ್ರ ಸಮಂ ಪಶ್ಯತಿ ಯೋಽರ್ಜುನ ।\\
ಸುಖಂ ವಾ ಯದಿ ವಾ ದುಃಖಂ ಸ ಯೋಗೀ ಪರಮೋ ಮತಃ ॥ ೩೨ ॥}
\cquote{ಅರ್ಜುನ, ಯಾವನು ಎಲ್ಲ ಪ್ರಾಣಿಗಳಲ್ಲಿಯೂ ಸುಖವಾಗಲೀ ದುಃಖವಾಗಲೀ ತನ್ನಂತೆಯೆ ಎಂದು ತಿಳಿಯುತ್ತಾನೋ ಆ ಜ್ಞಾನಿಯು ಎಲ್ಲರಿಗಿಂತ ಮೇಲೆನಿಸುವನು.\\}
\slcol{ಅರ್ಜುನ ಉವಾಚ ।\\
\Index{ಯೋಽಯಂ ಯೋಗಸ್ತ್ವಯಾ} ಪ್ರೋಕ್ತಃ ಸಾಮ್ಯೇನ ಮಧುಸೂದನ ।\\
ಏತಸ್ಯಾಹಂ ನ ಪಶ್ಯಾಮಿ ಚಂಚಲತ್ವಾತ್ಸ್ಥಿತಿಂ ಸ್ಥಿರಾಮ್ ॥ ೩೩ ॥}
\cquote{ಅರ್ಜುನನು ಹೇಳಿದನು,\\
ಕೃಷ್ಣ, ಸಮದೃಷ್ಟಿಯಿಂದ ನೋಡಬೇಕೆಂದು ನೀನು ಈ ಮನಸ್ಸನ್ನು ಬಿಗಿ ಹಿಡಿಯುವ ಸಾಧನವನ್ನು ಹೇಳಿದೆ ಅಷ್ಟೇ. ಮನಸ್ಸು ಚಂಚಲವಾದ್ದರಿಂದ ಅದು ಹೀಗೆ ಒಂದು ಕಡೆ ನಿಲ್ಲುವಂತೆ ನನಗೆ ಕಾಣುವುದಿಲ್ಲ.}

\newpage
\begin{mananam}{\mananamfont \large{ಮನನ ಶ್ಲೋಕ - ೩೦}}
\mananamtext ನನ್ನ ದಿನ ನಿತ್ಯ ಜೀವನದಲ್ಲಿ ದೇವರ ಬಗ್ಗೆ ನನ್ನ ತಿಳುವಳಿಕೆ ಏನು? ದೇವರ ಇರುವಿಕೆಯನ್ನು  ನಾನು ಎಲ್ಲ ಪರಿಸ್ಥಿತಿಗಳಲ್ಲೂ ಅನುಭವಿಸಬಲ್ಲೆನೇ? ದೇವರನ್ನು ನೆನಪಿಸುವ ವಿಗ್ರಹ, ಪಟ ಯಾವುದೂ ಇಲ್ಲದಿದ್ದಾಗ, ದೇವರೊಂದಿಗೆ ಸಂಪರ್ಕ ಕಡಿತದ ಭಾವನೆಯನ್ನು ಅನುಭವಿಸುತ್ತೇನೆಯೇ? ಇಲ್ಲದಿದ್ದಲ್ಲಿ ನಾನು ದೇವರೊಂದಿಗೆ ನಿರಂತರ ಸಂಪರ್ಕ ಮತ್ತು ಆಶ್ರಯ ಪಡೆಯಲು ಯಾವ ರೀತಿ ದೇವರ ಇರುವಿಕೆಯನ್ನು  ಅಳವಡಿಸಿಕೊಳ್ಳಬಹುದು? 
\end{mananam}
\WritingHand\enspace\textbf{ಆತ್ಮ ವಿಮರ್ಶೆ}\\
\begin{inspiration}{\mananamfont \large ಸ್ಪೂರ್ತಿ}
\mananamtext ದೇವರ ಸಾನಿಧ್ಯವನ್ನು ಸತತವಾಗಿ ಅನುಭವಿಸುವುದರಿಂದ ಮನಸ್ಸು ಉತ್ಕೃಷ್ಟವಾಗುತ್ತದೆ. ಇದು ಭೌತಿಕ ಜಗತ್ತನ್ನು ದೈವಿಕವಾಗಿ ಪರಿವರ್ತಿಸುತ್ತದೆ ಮತ್ತು ಕಳೆದು ಹೋದ ಭಾವನೆ ಮತ್ತು  ಅಸಹಾಯಕತೆ ಭಾವನೆಗಳನ್ನು ಹೊರ ಹಾಕುತ್ತದೆ. ಸಮರ್ಪಣಾ ಭಾವವಿರುವ ಆತ್ಮಕ್ಕೆ ದೇವರ ಸಾನಿಧ್ಯವನ್ನು ಎಲ್ಲದರಲ್ಲೂ ಅನುಭವಿಸುವುದೇ ಸುಖದಾಯಕವಾಗಿರುತ್ತದೆ.
\end{inspiration}
\newpage


\newpage
\begin{mananam}{\mananamfont \large{ಮನನ ಶ್ಲೋಕ - ೩೧ ೩೨}}
\mananamtext ನನ್ನ ಸುತ್ತಲಿರುವ ಪ್ರತಿಯೊಬ್ಬರಲ್ಲೂ ನಾನು ನನ್ನನ್ನೇ ಕಾಣಬಲ್ಲೆನೇ? ವ್ಯವಹಾರಿಕ ಉದ್ದೇಶಗಳ ಹೊರತಾಗಿ ಜನರನ್ನು ಯಾವ ಪ್ರಾಪಂಚಿಕ ಪಾತ್ರಕ್ಕೆ ಅಥವಾ ಅವರ ಸಾಧನೆಗಳ ಆಧಾರದ ಮೇಲೆ ಗೌರವಿಸುತ್ತೇನಾ? ಅಥವಾ ಅವರಲ್ಲಿ ಅಂತರ್ಯಾಮಿ ಆಗಿರುವ ದೇವರ ಸತ್ವದ ಆಧಾರದ ಮೇಲೆ ಗೌರವಿಸುತ್ತೇನಾ? ನನ್ನ ಬಾಹ್ಯಪಾತ್ರಗಳು, ಕೌಶಲ್ಯಗಳು ಮತ್ತು ವ್ಯಕ್ತಿತ್ವವನ್ನು ಮೀರಿ ನಾನು ನನ್ನ ಆತ್ಮ ತತ್ವಕ್ಕೆ ಹೇಗೆ ಸಂಬಂಧಿಸಬಹುದು? ನನ್ನಲ್ಲೂ ಮತ್ತು ಎಲ್ಲರಲ್ಲೂ ಅಂತರ್ಯಾಮಿ ಯಾಗಿರುವ ದೇವರ ತತ್ವದ ಅರಿವಿನಿಂದ ದೈನಂದಿನ ಜೀವನದಲ್ಲಿ ನಾನು ಇತರರೊಂದಿಗೆ ಹೇಗೆ ಸಂವಹನೆ ನಡೆಸುತ್ತೇನೆ ಎಂಬುದನ್ನು ಪರಿವರ್ತಿಸಬಹುದೇ?
\end{mananam}
\WritingHand\enspace\textbf{ಆತ್ಮ ವಿಮರ್ಶೆ}\\
\begin{inspiration}{\mananamfont \large ಸ್ಪೂರ್ತಿ}
\mananamtext ಜ್ಞಾನದ ಹಾದಿಯಲ್ಲಿರುವವರಿಗೆ ನಮ್ಮಲ್ಲಿರುವ ದೈವಿಕ ಪ್ರಜ್ಞೆಯೇ ಎಲ್ಲಾ ಜೀವಿಗಳಲ್ಲೂ ಇರುವುದು ಎಂದು ಗುರುತಿಸುವುದೇ ಅತ್ಯಮೂಲ್ಯ  ಅಭ್ಯಾಸವಾಗಿದೆ. ಈ ತಿಳುವಳಿಕೆಯನ್ನು ಅರ್ಥೈಸಿಕೊಂಡರೆ ಯಾವುದೇ ಭಯವಿರುವುದಿಲ್ಲ, ಯಾರೂ ಶತ್ರುಗಳಿರುವುದಿಲ್ಲ. ಏಕೆಂದರೆ ಇಡೀ ಪ್ರಪಂಚವೇ ನಮ್ಮ ಈ ಅಸ್ತಿತ್ವದ ವಿಸ್ತರಣೆಯಾಗಿದೆ.
\end{inspiration}
\newpage


\slcol{\Index{ಚಂಚಲಂ ಹಿ ಮನಃ} ಕೃಷ್ಣ ಪ್ರಮಾಥಿ ಬಲವದ್ದೃಢಮ್ ।\\
ತಸ್ಯಾಹಂ ನಿಗ್ರಹಂ ಮನ್ಯೇ ವಾಯೋರಿವ ಸುದುಷ್ಕರಮ್ ॥ ೩೪ ॥}
\cquote{ಕೃಷ್ಣ, ಮನಸ್ಸು ಹರಿದಾಡುವಂಥಾದ್ದು ನಮ್ಮನ್ನು ಗಾಸಿಗೊಳಿಸಿ ತನ್ನ ಪಟ್ಟನ್ನು ಬಿಡದೆ ಸಾಧಿಸಬಲ್ಲ ಬಲಶಾಲಿ. ಅದನ್ನು ಬಿಗಿಯುವುದು ಎಂದರೆ ಗಾಳಿಯನ್ನು ಕಟ್ಟಿ ಹಾಕಿದಂತೆ ಎಂದು ನನಗನ್ನಿಸುತ್ತದೆ.\\}
\slcol{ಶ್ರೀ ಭಗವಾನುವಾಚ ।\\
\Index{ಅಸಂಶಯಂ ಮಹಾಬಾಹೋ} ಮನೋ ದುರ್ನಿಗ್ರಹಂ ಚಲಮ್ ।\\
ಅಭ್ಯಾಸೇನ ತು ಕೌಂತೇಯ ವೈರಾಗ್ಯೇಣ ಚ ಗೃಹ್ಯತೇ ॥ ೩೫ ॥}
\cquote{ಶ್ರೀ ಭಗವಂತನು ಹೇಳಿದನು, ಅರ್ಜುನ ಮನಸ್ಸು ತಡೆಯುವುದಕ್ಕೆ ಆಗದು ಎಂಬುದು ಮತ್ತು ಹರಿದಾಟದ ಸ್ವಭಾವವೆಂಬುದು ನಿಶ್ಚಯ. ಆದರೆ ಅರ್ಜುನ, ಪ್ರಯತ್ನದಿಂದಲೂ, ವೈರಾಗ್ಯದಿಂದಲೂ ಅದನ್ನು ಬಿಗಿ ಹಿಡಿಯುವುದಕ್ಕಾಗುತ್ತದೆ.\\}
\slcol{\Index{ಅಸಂಯತಾತ್ಮನಾ ಯೋಗೋ} ದುಷ್ಪ್ರಾಪ ಇತಿ ಮೇ ಮತಿಃ ।\\
ವಶ್ಯಾತ್ಮನಾ ತು ಯತತಾ ಶಕ್ಯೋಽವಾಪ್ತುಮುಪಾಯತಃ ॥ ೩೬ ॥}
\cquote{ಮನಸ್ಸನ್ನು ಬಿಗಿಹಿಡಿಯಲಾರದವನಿಗೆ ಧ್ಯಾನವು ದಕ್ಕುವ  ಮಾತಲ್ಲವೆಂದು ನನ್ನ ಅಭಿಪ್ರಾಯ. ಪ್ರಯತ್ನದಿಂದ ಮನಸ್ಸನ್ನು ಹಿಡಿತಕ್ಕೆ ತಂದುಕೊಂಡವನಿಗೆ ಉಪಾಯದಿಂದ ಅದನ್ನು ಪಡೆಯುವುದಕ್ಕಾಗುತ್ತದೆ.\\}
\slcol{ಅರ್ಜುನ ಉವಾಚ ।\\
\Index{ಅಯತಿಃ ಶ್ರದ್ಧಯೋಪೇತೋ} ಯೋಗಾಚ್ಚಲಿತಮಾನಸಃ ।\\
ಅಪ್ರಾಪ್ಯ ಯೋಗಸಂಸಿದ್ಧಿಂ ಕಾಂ ಗತಿಂ ಕೃಷ್ಣ ಗಚ್ಛತಿ ॥ ೩೭ ॥}
\cquote{ಅರ್ಜುನನ್ನು ಹೇಳಿದನು,\\
ಕೃಷ್ಣ, ಶಾಸ್ತ್ರದಲ್ಲಿಯೂ ಗುರುವಿನಲ್ಲಿಯೂ ಶ್ರದ್ಧೆಯುಳ್ಳವನು ಪ್ರಯತ್ನ ಮಾಡಲಾರದೆ ಧ್ಯಾನದಿಂದ ಕದಲಿದ ಮನಸುಳ್ಳವನಾದರೆ ಯೋಗದ ಫಲವಾದ ಜ್ಞಾನವನ್ನು ಪಡೆಯದೆ ಅವನು ಮತ್ತೆ ಯಾವ ಗತಿಯನ್ನು ಪಡೆಯುತ್ತಾನೆ?\\}
\slcol{\Index{ಕಚ್ಚಿನ್ನೋಭಯವಿಭ್ರಷ್ಟ}ಶ್ಛಿನ್ನಾಭ್ರಮಿವ ನಶ್ಯತಿ ।\\
ಅಪ್ರತಿಷ್ಠೋ ಮಹಾಬಾಹೋ ವಿಮೂಢೋ ಬ್ರಹ್ಮಣಃ ಪಥಿ ॥ ೩೮ ॥}
\cquote{ಗುರಿತಪ್ಪಿ ಮೋಕ್ಷ ಮಾರ್ಗದಿಂದ ಜಾರಿದ ಯೋಗ ಬ್ರಷ್ಟನು ಎರಡಕ್ಕೂ ತಪ್ಪಿದವನಾಗಿ ಭಿನ್ನ-ಭಿನ್ನವಾದ ಮೋಡದಂತೆ ಹಾಳಾಗಿ ಹೋಗುವುದಿಲ್ಲವೇ?\\}
\slcol{\Index{ಏತನ್ಮೇ ಸಂಶಯಂ ಕೃಷ್ಣ} ಛೇತ್ತುಮರ್ಹಸ್ಯಶೇಷತಃ ।\\
ತ್ವದನ್ಯಃ ಸಂಶಯಸ್ಯಾಸ್ಯ ಛೇತ್ತಾ ನ ಹ್ಯುಪಪದ್ಯತೇ ॥ ೩೯ ॥}
\cquote{ಕೃಷ್ಣ, ನನ್ನ ಈ ಸಂಶಯವನ್ನು ಪೂರ್ಣವಾಗಿ ನೀನು ಕಳೆಯಬೇಕು, ಏಕೆಂದರೆ ನಿನ್ನ ಹೊರೆತು ಯಾರು ಈ ಸಂದೇಹವನ್ನು ಪರಿಹರಿಸಲಾರರು.\\}

\newpage
\begin{mananam}{\mananamfont \large{ಮನನ ಶ್ಲೋಕ - ೩೪ ೩೫}}
\mananamtext ನನ್ನ ಮನಸ್ಸಿನ ಚಂಚಲ  ಸ್ವಭಾವದ ಬಗ್ಗೆ ನನಗೆ ಅರಿವಿದೆಯೇ? ನಾನು ನನ್ನ ಮನಸ್ಸಿಗೆ ಆಗಾಗ ವಿರಾಮ ಕೊಡುತ್ತಿದ್ದೇನಾ? ಅದರ ಪ್ರಚೋದನೆಗಳು ಮತ್ತು ಪ್ರಕ್ಷೇಪಣಗಳ ಬಗ್ಗೆ ಅರಿವಿದೆಯೇ? ನಾನು ಪ್ರಜ್ಞಾಪೂರ್ವಕವಾಗಿ  ಹಗಲಿನಲ್ಲಿ ಮನಸ್ಸಿಗೆ ವಿರಾಮಗೊಳಿಸುವ ಅಭ್ಯಾಸವನ್ನು ಮಾಡುತ್ತಿದ್ದೇನೆಯೇ? ಮನಸ್ಸನ್ನು ಶಾಂತಿಗೊಳಿಸುವಲ್ಲಿ ನನ್ನ ನಿಯಮಿತ ಅಭ್ಯಾಸ ಎಷ್ಟು ಪರಿಣಾಮಕಾರಿಯಾಗಿದೆ?\\
ನಿರಾಸಕ್ತಿ  ಬಗ್ಗೆ ನನ್ನ ತಿಳುವಳಿಕೆ ಏನು ಗೊಂದಲವನ್ನು ತಪ್ಪಿಸಲು ಮತ್ತು ನಿರ್ದಿಷ್ಟ ಚಟುವಟಿಕೆಗಳಲ್ಲಿ ಗಮನವನ್ನು ಹೆಚ್ಚಿಸಲು ನಾನು ಅದನ್ನು ಹೇಗೆ ಅನ್ವಯಿಸಬಹುದು?
\end{mananam}
\WritingHand\enspace\textbf{ಆತ್ಮ ವಿಮರ್ಶೆ}\\
\begin{inspiration}{\mananamfont \large ಸ್ಪೂರ್ತಿ}
\mananamtext ಮನಸ್ಸಿನ ಅಂತರ್ಗತ ಸ್ವಭಾವವು ಪ್ರಕ್ಷುಬ್ಧವಾಗಿರುತ್ತದೆ ಮತ್ತು ನಾವು ಇದನ್ನು ಪ್ರಜ್ಞಾಪೂರ್ವಕವಾಗಿ  ಒಪ್ಪಿಕೊಂಡಾಗ ವಿಶೇಷವಾಗಿ ಅದರ ಪ್ರಕ್ಷುಬ್ದ ಕ್ಷಣಗಳಲ್ಲಿ ನಾವು ಅದರಿಂದ ದೂರವಿರುತ್ತೇವೆ. ಮನಸ್ಸನ್ನು ನಿಯಂತ್ರಿಸಲು ಇಚ್ಛೆ ಪಡುವವರಿಗೆ ನಿರಾಸಕ್ತಿ ಮತ್ತು ನಿರ್ಲಿಪ್ತತೆಯು ಅಮೂಲ್ಯವಾದ ಸಾಧನವಾಗಿ ಕಾರ್ಯನಿರ್ವಹಿಸುತ್ತದೆ. ವಿದ್ಯಾರ್ಥಿಗಳು ಕೂಡ ಅಧ್ಯಯನದ ಮೇಲೆ ಗಮನ ಕೇಂದ್ರೀಕರಿಸಲು ತಾತ್ಕಾಲಿಕವಾಗಿ ಇತರ ಮನೋರಂಜನೆಗಳು ಮತ್ತು ಗೊಂದಲಗಳನ್ನು ಬದಿಗಿಡಬೇಕು. ಶ್ರದ್ಧೆಯಿಂದ ಅಭ್ಯಾಸ ಮಾಡಿದರೆ ಒಬ್ಬರು ಯೋಗಿ ಆಗುತ್ತಾರೆ. ಯೋಗಿ ಎಂದರೆ ಅವರು ಅವರ ಮನಸ್ಸಿನ ಮತ್ತು ಪ್ರಪಂಚದ ಯಜಮಾನ.    
\end{inspiration}
\newpage



\newpage
\begin{mananam}{\mananamfont \large{ಮನನ ಶ್ಲೋಕ - ೩೭ ೩೮}}
\mananamtext ನನ್ನ ಮನಸ್ಸನ್ನು ನಿಯಂತ್ರಿಸಲು ಇರುವ ಪ್ರೇರಣೆ ಯಾವುದು?
ಪ್ರಕ್ಷುಬ್ದ ಮನಸ್ಸು, ಅಸಂತೋಷ ಇವೇ ಜೀವನದ ಎಲ್ಲಾ ಸಮಸ್ಯೆಗಳಿಗೂ ಕಾರಣ ಎಂಬ ಅರಿವು ನನಗಿದೆಯೇ?\\
ನನಗಿರುವ ದೌರ್ಬಲ್ಯಗಳು, ಅಹಿತಕರ ಲಕ್ಷಣಗಳನ್ನು ಜಯಿಸಲು ಸರಿಯಾಗಿ ಪ್ರಯತ್ನಿಸದೇ ಅಕಾಲಿಕವಾಗಿ ಬಿಟ್ಟು ಬಿಡುತ್ತೀನಾ?\\
ಅಪೇಕ್ಷಿತ ಆಂತರಿಕ ಸ್ಥಿತಿಯನ್ನು ಸಾಧಿಸುವುದು ಬಹು ದುಸ್ತರ ಎಂದು ಭಯಪಡುತ್ತೇನೆಯೇ? ಅದನ್ನು ಸಾಧಿಸಲು ಒಂದಕ್ಕಿಂತ ಹೆಚ್ಚು ಜೀವತಾವಧಿಯ ಅವಶ್ಯಕತೆ ಇದೆಯೇ?
\end{mananam}
\WritingHand\enspace\textbf{ಆತ್ಮ ವಿಮರ್ಶೆ}\\
\begin{inspiration}{\mananamfont \large ಸ್ಪೂರ್ತಿ}
\mananamtext ಜೀವನದಲ್ಲಿ ಎಲ್ಲಾ ತರಹದ ಪ್ರತಿಕೂಲ ಸನ್ನಿವೇಶಗಳನ್ನು ಮತ್ತು ಅಡೆತಡೆಗಳನ್ನು ಮೀರಿ ಜೀವನ ನಡೆಸಿದ ನಮ್ಮ ಋಷಿಮುನಿಗಳು ಜೀವನಕ್ಕೆ ಸ್ಪೂರ್ತಿಯಾಗಿದ್ದಾರೆ. ಅದಾಗಿಯೂ ಅಧ್ಯಾತ್ಮಿಕ ಮಾರ್ಗವನ್ನು ಪ್ರಾರಂಭಿಸುವವರಿಗೆ ಇದು ಹೆಚ್ಚೆನಿಸಬಹುದು. ಸ್ವಯಂ ನಂಬಿಕೆ ಮತ್ತು ಶ್ರದ್ಧೆಯ ಕೊರತೆಯ ಕಾರಣ ತಮ್ಮ ಅತ್ಯುನ್ನತ ಗುರಿಯನ್ನು ತ್ಯಜಿಸುತ್ತಾರೆ. ಆದರೂ ಆಂತರಿಕ ಸ್ವಾತಂತ್ರ್ಯವನ್ನು ಕಿಂಚಿತ್ತಾದರೂ ಅನುಭವಿಸಿದವರಿಗೆ ಕೇವಲ ಲೌಕಿಕ ಸಂತೋಷದಿಂದ ತೃಪ್ತರಾಗಲು ಸಾಧ್ಯವಿಲ್ಲ.
\end{inspiration}
\newpage


\slcol{ಶ್ರೀಭಗವಾನುವಾಚ ।\\
\Index{ಪಾರ್ಥ ನೈವೇಹ ನಾಮುತ್ರ} ವಿನಾಶಸ್ತಸ್ಯ ವಿದ್ಯತೇ ।\\
ನ ಹಿ ಕಲ್ಯಾಣಕೃತ್ಕಶ್ಚಿದ್ದುರ್ಗತಿಂ ತಾತ ಗಚ್ಛತಿ ॥ ೪೦ ॥}
\cquote{ಭಗವಂತನು ಹೀಗೆಂದನು,\\
ಹೇ ಪಾರ್ಥ, ಯೋಗಭ್ರಷ್ಟನಿಗೆ  ಇಹಪರಗಳಲ್ಲಿ ಎಲ್ಲಿಯೂ ಕೆಡುಕಿಲ್ಲ. ಅಯ್ಯ! ಒಳ್ಳೆಯದನ್ನು  ಮಾಡಿದವನು ಯಾವಾಗಲೂ ಕೆಡುವುದಿಲ್ಲ.\\}
\slcol{\Index{ಪ್ರಾಪ್ಯ ಪುಣ್ಯಕೃತಾಂ ಲೋಕಾ}ನುಷಿತ್ವಾ ಶಾಶ್ವತೀಃ ಸಮಾಃ ।\\
ಶುಚೀನಾಂ ಶ್ರೀಮತಾಂ ಗೇಹೇ ಯೋಗಭ್ರಷ್ಟೋಽಭಿಜಾಯತೇ ॥ ೪೧ ॥}
\cquote{ಅಂತ ಯೋಗ ಭ್ರಷ್ಟನು ಪುಣ್ಯವಂತರು ಪಡೆಯುವ ಲೋಕಗಳನ್ನು ಪಡೆದು ಅನೇಕ ವರ್ಷಗಳು ಅಲ್ಲಿದ್ದು ಧರ್ಮ ಶ್ರದ್ಧೆಯುಳ್ಳ ಭಾಗ್ಯವಂತರ ಮನೆಯಲ್ಲಿ ಹುಟ್ಟುತ್ತಾನೆ. ಅಥವಾ\\}
\slcol{\Index{ಯೋಗಿನಾಮೇವ ಕುಲೇ} ಭವತಿ ಧೀಮತಾಮ್ ।\\
ಏತದ್ಧಿ ದುರ್ಲಭತರಂ ಲೋಕೇ ಜನ್ಮ ಯದೀದೃಶಮ್ ॥ ೪೨ ॥}
\cquote{ಅಥವಾ ಅವನು ಜ್ಞಾನಿಗಳಾದ ಯೋಗಭ್ಯಾಸಿಗಳ ಕುಲದಲ್ಲಿಯೇ ಹುಟ್ಟುತ್ತಾನೆ. ಜಗತ್ತಿನಲ್ಲಿ ಈ ಬಗೆಯ ಹುಟ್ಟು ದೊರೆಯುವುದು ಬಹು ದುರ್ಲಭ.\\}
\slcol{\Index{ತತ್ರ ತಂ ಬುದ್ಧಿಸಂಯೋಗಂ} ಲಭತೇ ಪೌರ್ವದೇಹಿಕಮ್ ।\\
ಯತತೇ ಚ ತತೋ ಭೂಯಃ ಸಂಸಿದ್ಧೌ ಕುರುನಂದನ ॥ ೪೩ ॥}
\cquote{ಅರ್ಜುನ, ಅವನು ಅಲ್ಲಿ ಆ ಹಿಂದಿನ ಜನ್ಮದ ಬುದ್ಧಿಯನ್ನು ಪಡೆಯುತ್ತಾನೆ ಮತ್ತು ಜ್ಞಾನವನ್ನು ಮೈಗೂಡಿಸಿಕೊಳ್ಳುವುದಕ್ಕೆ ಇನ್ನೂ ಹೆಚ್ಚು ಪ್ರಯತ್ನವನ್ನು ಮಾಡುತ್ತಾನೆ.\\}
\slcol{\Index{ಪೂರ್ವಾಭ್ಯಾಸೇನ ತೇನೈವ} ಹ್ರಿಯತೇ ಹ್ಯವಶೋಽಪಿ ಸಃ ।\\
ಜಿಜ್ಞಾಸುರಪಿ ಯೋಗಸ್ಯ ಶಬ್ದಬ್ರಹ್ಮಾತಿವರ್ತತೇ ॥ ೪೪ ॥}
\cquote{ಏಕೆಂದರೆ ಅವನು ಹೆಚ್ಚು ಪ್ರಯತ್ನವಿಲ್ಲದೆ ಹಿಂದಿನ ಜನ್ಮದ ಅಭ್ಯಾಸದ ಕಡೆಗೆ ಕೊಚ್ಚಿಕೊಂಡು ಹೋಗುತ್ತಾನೆ. ಯೋಗದ ಸ್ವರೂಪವನ್ನು ತಿಳಿಯಲಪೇಕ್ಷಿಸುವವನು ಕೂಡ ಎಲ್ಲ ಕರ್ಮಗಳ ಫಲವನ್ನು ಮೀರಿದವನಾಗುತ್ತಾನೆ.\\}
\slcol{\Index{ಪ್ರಯತ್ನಾದ್ಯತಮಾನಸ್ತು} ಯೋಗೀ ಸಂಶುದ್ಧಕಿಲ್ಬಿಷಃ ।\\
ಅನೇಕಜನ್ಮಸಂಸಿದ್ಧಸ್ತತೋ ಯಾತಿ ಪರಾಂ ಗತಿಮ್ ॥ ೪೫ ॥}
\cquote{ಪ್ರಯತ್ನಪೂರ್ವಕವಾಗಿ ಯೋಗ ಸಾಧನೆಯಲ್ಲಿ ನಿರತನಾದ ಯೋಗಿಯಂತೂ ಪಾಪದ ಕೊಳೆಯನ್ನೆಲ್ಲ ಕಳೆದುಕೊಂಡು ಅನೇಕ ಜನ್ಮಗಳ ಸಾಧನೆಯ ಸಿದ್ದಿಯ ಫಲವಾಗಿ ಕೊನೆಗೆ ಪರಮ ಪದವನ್ನು ಪಡೆಯುತ್ತಾನೆ.\\}
\slcol{\Index{ತಪಸ್ವಿಭ್ಯೋಽಧಿಕೋ ಯೋಗೀ} ಜ್ಞಾನಿಭ್ಯೋಽಪಿ ಮತೋಽಧಿಕಃ ।\\
ಕರ್ಮಿಭ್ಯಶ್ಚಾಧಿಕೋ ಯೋಗೀ ತಸ್ಮಾದ್ಯೋಗೀ ಭವಾರ್ಜುನ ॥ ೪೬ ॥}
\cquote{ಈ ಯೋಗಿಯು ತಪಸ್ವಿಗಳಿಗಿಂತಲೂ, ಪರೋಕ್ಷ ಜ್ಞಾನಿಗಳಿಗಿಂತಲೂ, ಕರ್ಮಿಗಳಿಗಿಂತಲೂ ಶ್ರೇಷ್ಠನು.ಆದುದರಿಂದ ಹೇ ಅರ್ಜುನ, ನೀನು ಯೋಗಿಯಾಗು.}

\newpage
\begin{mananam}{\mananamfont \large{ಮನನ ಶ್ಲೋಕ - ೪೦}}
\mananamtext ನನ್ನ ಸಾಧನೆಗಳು ಮತ್ತು ವೈಫಲ್ಯಗಳನ್ನು ಅತಿಯಾಗಿ ವಿಮರ್ಶೆ ಮಾಡಿಕೊಳ್ಳುತ್ತೇನಾ? ಉದಾಹರಣೆಗೆ ನಾನು ವಿಷಯಗಳನ್ನು ಸಂಪೂರ್ಣ ಯಶಸ್ಸು ಅಥವಾ ಸಂಪೂರ್ಣ ವೈಫಲ್ಯ ಎಂದು ನೋಡುತ್ತೇನೆಯೇ? ನನ್ನನ್ನು ಬೇರೆಯವರೊಂದಿಗೆ ಹೋಲಿಸಿಕೊಂಡು ಮೌಲ್ಯ ಮಾಪನೆ ಮಾಡಿಕೊಳ್ಳುತ್ತೇನೆಯೇ? ಅಥವಾ ನನಗಾಗಿ ನಾನು ಒಂದು ನಿರ್ದಿಷ್ಟ ಗುರಿಯನ್ನು ಹೊಂದಿದ್ದೇನೆಯೇ?\\
ಹಲವು ವರ್ಷಗಳು ಮಾಡಿದ ಪ್ರಯತ್ನದಿಂದ ವರ್ತಮಾನದಲ್ಲಿ  ಬಂದಿರುವ ಸಾಧನೆಗಳು ಮತ್ತು ಕೌಶಲ್ಯಗಳನ್ನು ನಾನು ಗುರುತಿಸಬಹುದೇ? ಹಿಂದೆ ಮಾಡಿದ ಹಲವು ಕೆಲಸಗಳು ಹೇಗೆ ಪ್ರಸ್ತುತ ಸಾಮರ್ಥ್ಯಗಳಿಗೆ ಕೊಡುಗೆ ನೀಡುತ್ತದೆಯೋ ಹಾಗೆಯೇ ನನ್ನ ಈ ವರ್ತಮಾನದ ಪ್ರಯತ್ನಗಳು ಭವಿಷ್ಯದ ಬೆಳವಣಿಗೆಗೆ ಮತ್ತು ಮಾನಸಿಕ ಕೌಶಲ್ಯಗಳಿಗೆ ಸಮರ್ಥವಾಗಿ ಕಾರಣವಾಗಬಹುದು. ಈ ಎಲ್ಲ ಕಾರಣಗಳಿಂದ, ಧ್ಯಾನ, ಅಧ್ಯಾತ್ಮಿಕ ಪ್ರಯತ್ನಗಳು ಮತ್ತು ಎಲ್ಲ ಪ್ರಯತ್ನಗಳು ಕೂಡ ಎಂದಿಗೂ ವ್ಯರ್ಥವಾಗುವುದಿಲ್ಲ  ಎಂದು ನಿರ್ಣಯಿಸಬಹುದು.
\end{mananam}
\WritingHand\enspace\textbf{ಆತ್ಮ ವಿಮರ್ಶೆ}\\
\begin{inspiration}{\mananamfont \large ಸ್ಪೂರ್ತಿ}
\small \mananamtext ಪ್ರಪಂಚದ ದೃಷ್ಟಿಯಲ್ಲಿರುವ ಗೆಲುವು ಮತ್ತು ವೈಫಲ್ಯ ಗಳ ಆಧಾರದ ಮೇಲೆ ಜೀವನವನ್ನು ಅಳೆಯಲಾಗುವುದಿಲ್ಲ. ಅದೇ ರೀತಿ ನಮಗೆ ಗುರಿಗಳನ್ನು ಈ ಜೀವನದಲ್ಲಿ ಸಾಧಿಸಲು ಸಾಧ್ಯವಾಗಿಲ್ಲ ಎಂಬ ಮಾತ್ರಕ್ಕೆ ನಮ್ಮನ್ನು ನಾವು ಸುಧಾರಿಸಿಕೊಳ್ಳುವ ಪ್ರಯತ್ನವನ್ನು ಬಿಡಬಾರದು. ನಮ್ಮನ್ನು ನಾವು ಪರಿವರ್ತನೆಗೊಳಿಸಿಕೊಳ್ಳುವ ಇನ್ನೊಬ್ಬರಿಗೆ ಸೇವೆ ಮಾಡುವ ಪ್ರತಿಯೊಂದು ಪ್ರಯತ್ನವು ಅಮೂಲ್ಯ ಮತ್ತು ಅದು ಸೂಕ್ತ ಸಮಯದಲ್ಲಿ ಫಲ ನೀಡುತ್ತದೆ.ಭಾರತೀಯ ಪುರಾಣ ಶಾಸ್ತ್ರಗಳ ಪ್ರಕಾರ ಒಂದು ಜನ್ಮದಿಂದ ಇನ್ನೊಂದು ಜನ್ಮಕ್ಕೆ ನಮ್ಮ ಪ್ರಯತ್ನವನ್ನು ಎಲ್ಲಿ ಬಿಟ್ಟಿರುತ್ತೇವೆಯೋ  ಅಲ್ಲಿಂದ ಮುಂದುವರಿಯುತ್ತದೆ. ಆದ್ದರಿಂದ ನಮ್ಮ ಕೊನೆಯ ಉಸಿರಿರುವರೆಗೂ ನಮ್ಮ ಅಂತರಂಗದ ಪರಿಪೂರ್ಣತೆಗೆ ಪ್ರಯತ್ನಿಸುತ್ತಿರಬೇಕು.
\end{inspiration}
\newpage

\newpage
\begin{mananam}{\mananamfont \large{ಮನನ ಶ್ಲೋಕ - ೪೫ ೪೬}}
\mananamtext ಆಂತರಿಕ ಪರಿಪೂರ್ಣತೆ ಎಂದರೆ  ಅರ್ಥವೇನು? ಕಲೆ ಕ್ರೀಡೆ ಕೆಲಸಗಳಲ್ಲಿ ಕಂಡು ಬರುವ ಬಾಹ್ಯ ಪರಿಪೂರ್ಣತೆಗಿಂತ ಇದು ಹೇಗೆ  ಭಿನ್ನವಾಗುವುದು? ಮನಸ್ಸು ಮತ್ತು ಹೃದಯವನ್ನು ಶುದ್ಧೀಕರಿಸುವುದು ಆಧ್ಯಾತ್ಮಿಕ ಸಾಧನೆಯಲ್ಲಿ ಏಕೆ ಮುಖ್ಯವಾಗಿದೆ. ಇದು ನಮ್ಮ ಬೆಳವಣಿಗೆಗೆ ಹೇಗೆ ಪೂರಕವಾಗಿದೆ?
ಅಧ್ಯಾತ್ಮಿಕ ಪ್ರಯಾಣದಲ್ಲಿ ಅನೇಕ ಜನ್ಮಗಳಲ್ಲಿ ನಾವು ನಮ್ಮನ್ನು ಹೇಗೆ ಸುಧಾರಿಸಿಕೊಳ್ಳಬಹುದು? ವಿಕಾಸವಾಗಲು ಯಾವ ಅಭ್ಯಾಸಗಳು ಪ್ರಯೋಜನಕಾರಿಯಾಗುವುದು?
ನನ್ನ ಪ್ರಯತ್ನಗಳನ್ನು ಸ್ವೀಕರಿಸಿ ಮತ್ತು ಹೇಗೆ ಪ್ರೇರೇಪಿಸಿಕೊಳ್ಳಬಹುದು?ಯಾವ ಮನಸ್ಥಿತಿಯು ನಮ್ಮ ಪ್ರಗತಿಗೆ ಸಹಾಯಕವಾಗಬಲ್ಲದು.
\end{mananam}
\WritingHand\enspace\textbf{ಆತ್ಮ ವಿಮರ್ಶೆ}\\
\begin{inspiration}{\mananamfont \large ಸ್ಪೂರ್ತಿ}
\mananamtext ಅಧ್ಯಾತ್ಮಿಕ ಬೋಧನೆಗಳ ಪ್ರಕಾರ ಸ್ವಯಂ ನಿಯಂತ್ರಣ ಮತ್ತು ಆಂತರಿಕ ಶುದ್ಧೀಕರಣದ ಯೋಗವು ಅತ್ಯುನ್ನತ ಸ್ಥಾನವನ್ನು ಹೊಂದಿದೆ. ಸ್ವಯಂ ನಿಯಂತ್ರಣವನ್ನು ಕರಗತ ಮಾಡಿಕೊಂಡವನಿಗೆ ಅಪಾರ ಸಾಮರ್ಥ್ಯವಿರುತ್ತದೆ ಮತ್ತು ಸುಲಭವಾಗಿ ಏನನ್ನಾದರೂ ಸಾಧಿಸಬಹುದು. ಶುದ್ಧ ಹೃದಯ ಮತ್ತು ಕೇಂದ್ರೀಕೃತ ಮನಸ್ಸು ಅಧ್ಯಾತ್ಮಿಕ ಹಾದಿಯಲ್ಲಿರುವ ಅನ್ವೇಷಕನಿಗೆ ಅಮೂಲ್ಯವಾದ ಆಸ್ತಿಗಳಾಗಿವೆ. ಅಂತಹ ಅಧ್ಯಾತ್ಮಿಕ ಗುಣಗಳಿರುವ ಒಬ್ಬ ವ್ಯಕ್ತಿಯು ಭೌತಿಕವಾಗಿ ಶ್ರೀಮಂತನಾಗಿರುವ ವ್ಯಕ್ತಿಗಿಂತ ಶ್ರೀಮಂತನನ್ನಾಗಿ ಮಾಡುತ್ತದೆ.
\end{inspiration}
\newpage

\slcol{\Index{ಯೋಗಿನಾಮಪಿ ಸರ್ವೇಷಾಂ} ಮದ್ಗತೇನಾಂತರಾತ್ಮನಾ ।\\
ಶ್ರದ್ಧಾವಾನ್ಭಜತೇ ಯೋ ಮಾಂ ಸ ಮೇ ಯುಕ್ತತಮೋ ಮತಃ ॥ ೪೭ ॥}
\cquote{ಎಲ್ಲ ಬಗೆಯ ಯೋಗಿಗಳಲ್ಲಿ ನನ್ನಲ್ಲೆ ಮನಸ್ಸನ್ನಿಟ್ಟು ನನ್ನನ್ನೇ ಶ್ರದ್ಧೆಯಿಂದ ಯಾವನು ಧ್ಯಾನ ಮಾಡುತ್ತಾನೋ ಅವನು ಬಹಳ ಹೆಚ್ಚಿನವನೆಂದು ನನ್ನ ಅಭಿಪ್ರಾಯ.}
\begin{center}
ಓಂ ತತ್ಸದಿತಿ ಶ್ರೀಮದ್ಭಗವದ್ಗೀತಾಸೂಪನಿಷತ್ಸು ಬ್ರಹ್ಮವಿದ್ಯಾಯಾಂ\\ ಯೋಗಶಾಸ್ತ್ರೇ ಶ್ರೀಕೃಷ್ಣಾರ್ಜುನಸಂವಾದೇ\\
ಆತ್ಮಸಂಯಮಯೋಗೋ ನಾಮ ಷಷ್ಠೋಽಧ್ಯಾಯಃ ॥ ೬ ॥
\end{center}
